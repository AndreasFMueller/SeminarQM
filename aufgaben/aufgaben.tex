Die vorgeschlagenen Seminararbeitsthemen sind prim"ar im Moodle zum
Seminar beschrieben, und die dortige Beschreibung wird angepasst
werden, sollten sich zeigen, dass es sich lohnen k"onnte, den Focus
etwas anders zu legen, als bei der urspr"unglichen Formulierung des
Themas absch"atzbar war.

\subsection{Tunneldiode}
Eine Tunneldiode nutzt den quantenmechanischen Tunneleffekt aus,
um einen negativen Widerstand zu erm"oglichen (abnehmender Strom mit
h"oherer Spannung). Ausserdem sind Tunneldioden in der Lage, bei
wesentlich geringeren Spannungen bereits zu schalten, als dies bei
konventionellen Dioden m"oglich ist.

Es wird erwartet, dass Sie die Funktion einer Tunneldiode an Hand
eines vereinfachten Modells erkl"aren, und die zu erwartenden Str"ome
absch"atzen. Ausserdem stehen drei Tunneldioden des Typs 1N3721 zur
Verf"ugung, mit denen Sie die Eigenschaften dieses exotischen Bauteils
ausprobieren und demonstrieren k"onnen.

\subsection{Rubidium-Frequenznormal}
Ein Rubidium-Frequenznormal verwendet eine Eigenschaft von
Rubidium-Atomen, um ein hochpr"azises Frequenznormal ($10^{-11}$)
bereitzustellen. Solche Frequenznormale werden zum Beispiel in 3G
Basisstationen oder in Satelliten verwendet.

Es wird erwartet, dass Sie anhand eines vereinfachten Modells
erkl"aren, wie ein solches Frequenznormal funktioniert. Welche
"ausseren Umst"ande k"onnten die Frequenz beeinflussen? Es steht
ausserdem ein Exemplar eines LPRO-101 f"ur Experimente und Demonstrationen
zur Verf"ugung.

\subsection{Rastertunnelmikroskop}
Mit dem Rastertunnelmikroskop ist es m"oglich, einzelne Atome sichtbar
zu machen. Eine Abtastnadel wird so "uber die Oberfl"ache eines Objekts
gesteuert, dass der durch die Spitze zum Objekt fliessende Tunnelstrom
konstant ist. Ohne das Objekt zu ber"uhren folgt die Spitze in
konstantem Abstand, und zeichnet damit ein Abbild der Objektoberfl"ache.
Wie nahe an die Oberfl"ache muss die Nadel gebracht werden, und wie
gross ist der beobachtete Strom?

\subsection{Geladenes Teilchen in einem elektrischen Feld}
Ein geladenes Teilchen wir zwischen zwei unendlich hohen
Potentialbbarrieren in einer Dimension eingesperrt. Die zugeh"origen
Zust"ande und Wellenfunktionen wurden im Vorlesungsteil berechnet.
Jetzt legt man zus"atzlich ein schwaches elektrisches Feld an. Dadurch
ver"andern sich die Wellenfunktionen und ihre Energie, dies ist
bekannt als der Stark-Effekt. In Form des quantum confined Stark
effect (QCSE) spielt er zum Beispiel bei Laserdioden eine Rolle,
er hat einen Einfluss auf die Wellenl"ange der Diode und die Effizienz.

Es wird erwartet, dass Sie die Wellengleichung f"ur dieses Problem
aufstellen und L"osungen suchen. Die Energieniveaus k"onnen dann
entweder numerisch mit Hilfe der Nullstellen der Airy-Funktionen
oder durch St"orungsrechnung ermittelt werden.

\subsection{Fourier-Theorie f"ur die Kugeloberfl"ache}
Die Fourier-Theorie ist sehr erfolgreich darin, periodische Funktionen
durch die Amplituden der einzelnen Frequenzen zu beschreiben. Mit
Kugelfunktionen kann man etwas vergleichbares f"ur Kugeloberfl"achen
machen. Man kann also vom Spektrum eines "kugeligen" Signals sprechen,
und insbesondere solche Signale vergleichen. Diese auch Multipolentwicklung
genannte Technik kann zum Beispiel auch zur Analyse der
Abstrahlcharakteristik komplizierter Antennen herangezogen werden.
Mit ihr ist es zum Beispiel auch m"oglich, die im Mikrowellenhintergrund
beobachteten Unregelm"assigkeiten mit den Vorhersagen der Theorie
zu vergleichen, indem man "uberpr"uft, ob des "richtige Kugelspektrum"
entsteht.

Es wird erwartet, dass Sie eventuell an Hand von anschaulichen
Beispielen von Funktionen auf der Kugeloberfl"ache zeigen, wie man
solche Signale analysieren kann.

\subsection{3D Visualisierung der Zustandsfunktionen eines Atoms}
Im Vorlesungsteil werden die Wellenfunktionen eines Wasserstoff-Atoms
berechnet. Ziel dieser Aufgabe ist, eine Visualisierung dieser
Funktionen als 3D-Modell zu realisieren. Dazu m"ussen einerseits die
Funktionen in geeigneter Form ausgedr"uckt werden und in dreidimensionale
Objekte umgewandelt werden, andererseits ist daf"ur zu sorgen, dass
diese auf einem 3D-Drucker auch wirklich ausdruckbar sind.

\subsection{Flash-Speicher}
Wie funktioniert die Speicherzelle eines Flash-Speicher-Chips? Mit
Hilfe der Quantenmechanik kann man sowohl den Prozess der Programmierung
(Hot electron injection) wie auch das L"oschens (Tunneleffekt)
verstehen. Warum halten Flash-Speicher nur eine begrenzte Zahl von
Schreibzyklen aus? Wie entstehen ``read disturb'' Fehler?
Warum fliegen die ``heissen Elektronen'' nicht einfach "uber den
vom floating gate gebildeten Potentialtopf hinweg? 
Um diese Ph"anomen zu verstehen, muss man sich mit Plasmonen befassen,
dies sind Pseudo-Teilchen, die aus den Wellen im Elektronengas des
Gate entstehen. Versuchen Sie, mit einem einfachen Modell Plasmonen
zu modellieren und zu zeigen, wie diese den Elektronen Energie entziehen
k"onnen.

\subsection{Franck-Hertz-Experiment}
Das Franck-Hertz-Experiment testet die Vorstellung, dass Atome nur
diskrete Energieniveaus haben k"onnen.

\subsection{Anharmonischer Oszillator}
in Fadenpendel (der Zeitgeber einer Pendeluhr) ist nicht ein
harmonischer Oszillator, insbesondere h"angt die Periode von der
Amplitude ab. Verwenden Sie die St"orungstheorie, um aus der bekannten
L"osung der Bewegungsgleichungen f"ur den harmonischen Oszillator die
L"osung f"ur das Fadenpendel abzuleiten. Tun Sie dasselbe auch f"ur
den quantisierten harmonischen Oszillator mit Hamilton-Operator
$\frac12 (P^2+Q^2)+\varepsilon Q^3$.

\subsection{Maser}
Laser sind aus der modernen Elektronik nicht wegzudenken: sie tasten
optische Medien ab, "ubertragen Daten durch Glasfasern, und werden
gebraucht, um in einer Pr"asentation auf Dinge zu zeigen. Und auch
Maser werden t"aglich gebraucht: die GPS-Satelliten tragen eine
Atomuhr, die auf einem Wasserstoff-Maser aufgebaut ist. In der
einfachsten Form braucht ein Laser oder Maser ein Quantensystem mit
zwei Zust"anden, der grundlegende Formalismus aus Kapitel 2 ist
bereits in der Lage, die Wirkungsweise zu erkl"aren.

\subsection{Laser}
Bereits 1916 hat Einstein die Stimulierte Emission von Licht in
einer theoretischen Arbeit beschrieben, doch erst 1928 gelang ihr
experimenteller Nachweis. Bis 1960 musste man warten, damit daraus
Light Amplification by Stimulated Emission of Radiation, also ein
funktionsf"ahiger LASER werden konnte. Viele Laser-Systeme kann man
mit Hilfe eines Systems von gekoppelten gew"ohnlichen
Differentialgleichungen modellieren, den Ratengleichungen. Leiten
Sie die Ratengleichungen f"ur Zwei- und Dreizustandssysteme her,
erkl"aren Sie die Einsteinkoeffizienten, und zeigen Sie (Beweis oder
numerische Simulation), dass ein Zweizustandssystem nicht "lasern"
kann, ein Dreizustandssystem aber schon.

\subsection{MRI}
MRI Magnetic Resonance Imaging beruht auf einem quantenmechanischen
Effekt, in den der Spin der Wasserstoffeatome im K"orper involviert
sind. Je nach chemischer Umgebung spalten die Energieniveaus des
Kernspins von Wasserstoffatomen auf, und es ist m"oglich, die Kernspins
durch Einstrahlung umzuklappen. Ausserdem kann man die R"uckkehr in
den Grundzustand als Signal aus dem K"orper wieder messen. Allerdings
will man ja nicht alle Kernspins gleichzeitig messen, sondern nur
eine Auswahl. Dazu f"ugt man ein Magnetfeld hinzu, welches die
Energieniveaus weiter ver"andert. Damit wird es m"oglich, die

Beschreiben Sie mit Hilfe eines vereinfachten Modells, was bei einem
MRI im K"orper geschieht, und wie Magnetfeld und Kernspins
zusammenarbeiten, um die sogenannte Radon-Transformation eines
K"orperschnittes zu erhalten. Die R"uckrechnung eines Bildes aus den
empfangenen Daten wird nicht erwartet.

\subsection{Quanten-Kryptographie}
Auf der Basis von verschr"ankten Zust"anden kann man ein Verfahren
aufbauen, mit welchem abh"orsicher ein Schl"ussel ausgehandelt werden
kann. Erkl"aren Sie dieses Verfahren und warum es absolut abh"orsicher
ist.

\subsection{Quanten-Teleportation}
Zwar kann man einen Zustand in einem Quantencomputer nicht kopieren
(No-Cloning Theorem), aber man kann einen Zustand zerst"oren und an
einem anderen Ort wieder entstehen lassen. Die Information "uber den
Zustand wird dabei von einem klassischen Übertragungskanal
transportiert.

\subsection{Der Algorithmus von Simon}
Wirklich schwierige Probleme geh"oren der Klasse NP an, das Problem
von Simon ist ein solches. Die Probleme der Klasse NP zeichnet aus,
dass kein Algorithmus bekannt ist, der das Problem schneller als
in exponentieller Zeit l"osen kann. Der Algorithmus von Simon zeigt,
dass es NP-Probleme gibt, die ein Quantencomputer in polynomieller
Zeit l"osen kann.

\subsection{Was ist ein Quantenpunkt?}
Quantenpunkte sind Nanostrukturen, die quantenmechanische Eigenschaften
zeigen. Sie k"onnen in LEDs verwendet werden und sind  vielversprechende
Kandidaten f"ur Qubits.

\section{Abh"angigkeiten der Themen}
\def\x{$\times$}
\begin{table}
\centering
\begin{tabular}{|r|cccccccccccccc|cc|}
\hline
Thema& 1& 2& 3& 4& 5& 6& 7& 8& 9&10&11&12&13&14&FuVar&AutoSpr\\
\hline
  5.1&\x&\x&\x&  &\x&\x&  &  &  &  &  &  &  &  &     &       \\
  5.2&\x&\x&\x&  &\x&\x&  &  &  &\x&  &  &  &  &     &       \\
  5.3&\x&\x&\x&  &\x&\x&  &  &  &\x&  &  &  &  &     &       \\
  5.4&\x&\x&\x&  &\x&\x&  &  &  &\x&  &  &  &  &\x   &       \\
  5.5&\x&\x&\x&  &\x&\x&  &  &\x&  &  &  &  &  &\x   &       \\
  5.6&\x&\x&\x&  &\x&\x&  &  &\x&  &  &  &  &  &\x   &       \\
  5.7&\x&\x&\x&  &\x&\x&  &\x&  &\x&  &  &  &  &     &       \\
  5.8&\x&\x&\x&  &\x&\x&  &  &  &\x&  &  &  &  &     &       \\
  5.9&\x&\x&\x&  &\x&\x&  &\x&  &\x&  &  &  &  &     &       \\
 5.10&\x&\x&\x&  &\x&\x&  &  &  &\x&\x&  &  &  &     &       \\
 5.11&\x&\x&\x&  &\x&\x&  &  &  &  &  &  &  &  &     &       \\
 5.12&\x&\x&\x&  &\x&\x&  &  &  &\x&\x&  &\x&  &     &       \\
 5.13&\x&\x&\x&\x&  &  &  &  &  &  &  &  &  &  &     &\x     \\
 5.14&\x&\x&\x&\x&  &  &  &  &  &  &  &  &  &  &     &\x     \\
 5.15&\x&\x&\x&\x&  &  &  &  &  &  &  &  &  &  &     &\x     \\
 5.16&\x&\x&\x&  &\x&\x&\x&\x&  &\x&  &  &  &  &     &       \\
\hline
\end{tabular}
\caption{Voraussetzungen f"ur die Bearbeitung der verschiedenen
Seminararbeitsthemen von den Skript-Kapiteln und den Vorlesungen
FuVar und AutoSpr.
\label{tabelle}}
\end{table}
Nat"urlich ist v"ollig offen, wie weit Sie Ihr Seminararbeitsthema
ausgestalten wollen, und damit ist kaum definierbar, was Sie genau
brauchen werden.
Die Tabelle~\ref{tabelle} kann daher nur einen ungef"ahren Anhaltspunkt
geben, welche welche Kapitel des Skripts Sie verstanden haben m"ussen,
um das Seminar-Thema bearbeiten zu k"onnen.
Die letzten zwei Spalten der Tabelle zeigen, welche Aufgaben
m"oglicherweise nur vollst"andig bearbeitet werden k"onnen, wenn
sie bereits "uber gewisse Kenntnisse aus den angegebenen Vorlesungen
verf"ugen, die sie zur Zeit des Seminars m"oglicherweise erst besuchen.
Die Aufgaben 5.13--5.15 betreffen die Quanteninformatik, und sind
daher pr"aferenziell f"ur I-Teilnehmer gedacht.
