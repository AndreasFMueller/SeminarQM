%
% seminararbeit.tex -- Anleitung für die Seminararbeit
%
% (c) 2015 Prof Dr Andreas Mueller, Hochschule Rapperswil
%
\documentclass[a4paper]{article}
\usepackage{etex}
\usepackage{geometry}
\geometry{papersize={200mm,280mm},total={160mm,240mm},top=21mm}
\usepackage[ngerman]{babel}
\usepackage{times}
\usepackage{amsmath}
\usepackage{amssymb}
\usepackage{amsfonts}
\usepackage{amsthm}
\usepackage{txfonts}
\usepackage{verbatim}
\usepackage{array}
\begin{document}
\title{Anleitung zur Seminararbeit}
\author{Prof.~Dr.~Andreas M"uller}
\date{}
\maketitle
Das mathematische Seminar unterscheidet sich in gen"ugend vielen Aspekten
von jeder anderen Lehrveranstaltung an der HSR.
Dieses Dokument erkl"art die Ziele des Seminars und was es mit dem
Seminar-Buch auf sich hat.
Es gibt einige Hinweise zum Vortrag und zur Formatierung der
Arbeit in \LaTeX.
\section{Ziele}
Ein wesentlicher Teil des mathematischen Seminars ist Ihre eigene 
Seminararbeit.
Sie haben zu diesem Zeitpunkt schon eine grosse Zahl von mathematischen
Modulen absolviert, und viele "Ubungsaufgaben gel"ost.
In Ihrer Seminararbeit l"osen Sie ebenfalls ein mathematisches
Problem, allerdings sind die Voraussetzungen etwas anders:
\begin{enumerate}
\item
Es ist m"oglicherweise nicht klar, was eigentlich das Problem ist. 
Das kommt zwar gelegentlich auch bei "Ubungsaufgaben vor, dort ist
das aber eher in Hinweis darauf, dass Sie den Anschluss an die Vorlesung
verpasst haben.
Es ist Ihre Aufgabe, das Problem zu analysieren, zu verstehen, zu
mathematisieren und eine passende L"osung zu finden.
\item
Viele der Seminarthemen haben einen Anwendungshintergrund, und die
Wahl des mathematischen Modells h"angt davon ab, welchen Aspekt des
Anwendungsproblems man mit dem Modell verstehen will.
Welche Aspekte interessant und wichtig sind, m"ussen Sie m"oglicherweise
auch erst herausfinden.
\item
In anderen F"achern ist m"oglicherweise Ihr Ziel, die Mathematik
m"oglichst zu vermeiden.
Im Mathematischen Seminar wird von Ihnen erwartet, dass Sie die Mathematik
aktiv suchen.
\item
Das doppelt unterstrichene Resultat, das bei "Ubungsaufgaben 
den Erfolg signalisiert, bedeutet im Seminar nichts. 
Ihr mathematisches Modell soll helfen, die Anwendung zu verstehen.
Sie produzieren nicht Zahlen, sondern Einsichten.
\item
Ihr Resultat ist weniger wichtig als Ihr Weg dorthin. 
Die Resultate findet man auch in B"uchern, oder in der Wikipedia.
Ihre Aufgabe ist daher nicht nur, das Problem mathematisch in den
Griff zu bekommen, sondern auch, den anderen Seminarteilnehmern zu
erkl"aren, wie Sie das geschafft haben.
\item
Ihre Seminararbeit ist nicht nur f"ur Sie und f"ur den Dozenten,
der Ihnen daf"ur eine Note machen muss, sondern vor allem auch f"ur
Ihre Kollegen. 
Sie sind erfolgreich, wenn Sie die wesentlichen Ideen Ihrer Arbeit
Ihren Kollegen verst"andlich gemacht haben, und Ihr Text Ihren
Kollegen helfen kann, den Einstieg in Ihr Thema zu finden.
\end{enumerate}
Das Seminar soll Ihnen also zum ersten Mal die M"oglichkeit geben,
ein schwieriges mathematisches Thema zu "uberblicken und diese
"Ubersicht weiterzugeben.
M"oglicherweise ist dies auch Ihre erste Gelegenheit, ein
technisch/mathematisches Paper zu verfassen.

\section{Das Seminar-Buch}
Wie in fr"uheren Jahren wird auch in diesem Seminar wieder ein
Seminar-Buch entstehen.
Das Seminar-Buch ist der kollektive Leistungsausweis der Seminarteilnehmer.
Es beinhaltet das Skript sowie alle Seminararbeiten.

\subsection{Ziele des Buches}
Das Buch ist nicht einfach nur ein Erinnerung, es verfolgt ein paar
konkrete Ziele:
\begin{enumerate}
\item
Das Seminar ist eine Einf"uhrung in ein fortgeschrittenes mathematisches
Thema.
Es ist denkbar, dass Sie in ihrer Weiterausbildung das Thema vertiefen
(zum Beispiel, wenn Sie an der ETH weiterstudieren wollen).
Im besten Fall k"onnen Sie dann das Seminar-Buch wieder hervornehmen,
Ihre Kenntnisse auffrischen und leichter den Einstieg
in den neuen Stoff finden.
\item
Es ist zu hoffen, dass Sie in Ihrer Praxis irgendwann auf ein Problem
stossen, welches sich mit den im Seminar behandelten Techniken
l"osen l"asst. 
Im besten Fall k"onnen Sie dann das Seminar-Buch hervornehmen, und
schnell wieder den Einstieg in eine Technik finden, die Sie schon lange
nicht mehr gebraucht haben.
\item
Das Seminar vermittelt nicht einfach nur ein paar Techniken, deren
Beherrschung mit einer Modulbeschreibung und einer Note dokumentiert
werden kann.
Das Seminar hat jedes Jahr ein anderes Thema, und keine Modulbeschreibung,
die auf den Inhalt eingehen w"urde.
Das Seminar-Buch ist die einzige einigermassen vollst"andige Beschreibung
dessen, was im Seminar eigentlich passiert.
\item
Im Seminar werden oft Themen behandelt, die in keinem anderen Modul
Platz haben, aber durchaus f"ur ein breiteres Publikum von Interesse
sein k"onnten.
Das Seminar-Buch behandelt diese Theman auf eine Art, die mit den Grundlagen
aus anderen Vorlesungen verstanden werden k"onnen.
Oft ist es schwierig, ein Buch zu finden, welches vergleichbare
Voraussetzungen macht.
Das Seminar-Buch hat also auch die Aufgaben, anderen Studenten
eine Einstiegshilfe zu bieten, weshalb auch immer ein Exemplar
in der HSR-Bibliothek landet.
\item
Mit dem Seminar-Buch k"onnen Sie anderen einen ungewohnten Einblick 
in die Arbeit geben, die Sie in Ihrem Studium geleistet haben.
\end{enumerate}
Neben diesen ernsthaften Zielen d"urfen Sie sich selbstverst"andlich an
dem Resultat auch freuen und auf Ihren Beitrag dazu stolz sein.

\subsection{Besonderheiten 2015}
\begin{enumerate}
\item
So viele Teilnehmer wie dieses Jahr hatte das Seminar noch nie.
Mehr Teilnehmer bedeutet auch mehr Seminarbeiten, mehr und
breiter gestreute Themen.
\item
Dieses Jahr findet erstmals auch ein Seminar im MSE statt.
Daher ist der Skriptteil etwas ausf"uhrlicher, er enth"alt auch Teile,
die im MathSem Bachelor nicht oder weniger ausf"uhrlich behandelt werden.
\item
Die im MathSem-MSE erstellten Arbeiten sind ebenfalls im Seminar-Buch
enthalten.
\end{enumerate}

\subsection{Die Seminararbeit}
Die Ziele des Seminars und des Seminar-Buches haben Konsequenz daf"ur,
was Ihre Seminararbeit erreichen soll.
Hier ein paar konkrete Hinweise, worauf Sie achten sollten:
\begin{enumerate}
\item
Verschaffen Sie sich Klarheit dar"uber, was eigentlich f"ur Kernaussagen
mitteilen wollen.
\item
Pr"azision! 
Vielleicht haben Sie mal gelernt, sch"one Aufs"atze zu schreiben,
und blumige W"orter zu verwenden. Vergessen Sie alles.
In der Mathematik z"ahlt nur eines: die gew"ahlten W"orter/Formulierungen
m"ussen absolut pr"azis sein.
\item
Wenn Sie ein Argument aufschreiben, stellen Sie sicher, dass Sie alle
Schritte aufschreiben, die Sie selbst brauchen w"urden, um das 
Argument zu verstehen.
\item
Es gibt keine Vorgaben zur L"ange der Arbeit. 
Ihre Arbeit wird so lang wie n"otig, um Ihre Kernaussagen zu
begr"unden, aber nicht l"anger.
\item
Ihre Arbeit soll nicht m"oglichst kurz sein, sondern m"oglichst
verst"andlich.
\item
Ihre Arbeit soll nicht beliebig ausf"uhrlich sein, denn wenn man sich
durch tausend Details qu"alen muss, wird Ihre Arbeit schwer lesbar,
und entsprechend schwer verst"andlich.
\item 
Strukturieren Sie ein kompliziertes Argument, formulieren Sie sinnvolle,
eigenst"andige Zwischenresultate.
Verwenden Sie Abschnitte (\verb+\section+), \verb+\subsection+ und
\verb+\subsubsection+.
Formulieren Sie Definitionen in einem \verb+definition+-Environment,
und fassen Sie Erkenntnisse zum Beispiel in \verb+satz+- oder
\verb+hilfssatz+-Environments zusammen.
\item
Wiederholen Sie nicht unn"otigerweise Argumente, die man als bekannt
voraussetzen kann, weil sie zum Beispiel Gegenstand einer anderen
Lehrveranstaltung an der HSR sind.
Finden Sie eine Referenz (Buch aus der Bibliothek, Skript einer Vorlesung,
Wikipedia oder andere Internet-Quelle) und beziehen Sie sich in
Ihrem Text darauf.
\item
Ihre Arbeit wird nicht nach Gewicht bewertet, sondern nach Gehalt.
Enzyklop"adische Vollst"andigkeit kann man in der Bibliothek oder
auf dem Internet finden, sie tr"agt nur wenig zum Verst"andnis des
Problems bei.
\end{enumerate}

\section{Der Vortrag}
In Ihrem Vortrag machen Sie die Resultate ihren Kollegen zug"anglich.
Sie w"ahlen aus, welche Resultate aus Ihrer Arbeit im Vortrag 
Erw"ahnung finden sollen.

\begin{enumerate}
\item
Ihr Ziel darf nicht sein, mit Ihren Kenntnissen und Resultaten
zu imponieren. 
Sie bekommen eine gute Note, wenn sie m"oglichst n"utzliche 
Erkenntnisse zu vermitteln in der Lage sind.
\item
Ein Vortrag ist nicht ein vorgelesenes Paper.
Ein Vortrag ist eher ein Schauspiel mit mathematischem Inhalt.
\item
W"ahlen Sie f"ur jeden Schritt das optimale Pr"asentationsmittel.
Eine Rechnung oder Herleitung eignet sich nicht f"ur Folienpr"asentation,
eine Zusammenstellung von Fakten eignet sich nicht f"ur die Wandtafel.
\item
Verzichten Sie auf Effekte um der Effekte willen.
Powerpoint-Animationen bringen nur dann etwas, sie eine inhaltliche
Aussage sinnvoll unterst"utzen.
Andernfalls lenken sie nur ab.
\item
Bleiben Sie beim Thema.
Vermeiden Sie jede Ablenkung von Ihren Kernaussagen.
Eine elaborate und ausf"uhrlich kommentierte Choreographie von
Pr"asentatoren lenkt nur ab.
Wechseln Sie einfach ohne Kommentar ab.
Ist ist akzeptabel, wenn nur einer pr"asentiert.
\item
Seien Sie gefasst auf Fragen. 
Sie k"onnen sich im Team gegenseitig testen: der Zuh"orer versucht
jeden einzelnen Schritt zu hinterfragen, der Vortragende muss
antworten k"onnen.
\item
Einzelne Teams werden gebeten werden, ihren Vortrag auch im
Master-Seminar ein zweites Mal zu halten.
Damit das funktioniert, sollte ihr Vortrag m"oglichst eigenst"andig sein,
und sich nicht auf Vortr"age von Kollegen beziehen, sondern nur auf
Informationen, die auch im Skript zu finden sind.
\end{enumerate}

\section{Formatierung mit \LaTeX}
Ich bin der Herausgeber des Seminar-Buches, ich gebe den Stil des
Buches vor, und Ihre Arbeit hat sich daran zu halten. 
\begin{enumerate}
\item
Sie haben keine Kontrolle "uber die Kapitelnummer, insbesondere
kann es sein, dass Ihr Kapitel im letzten Moment vor Drucklegung
verschoben wird.
Ihre Verweise auf Formeln, Bilder oder Tabellen m"ussen also mit Hilfe
von Labeln (\verb+\label+) und Referenzen (\verb+\ref+) implementiert
sein.
\item
Sie haben keine Kontrolle "uber die Formatierung.
Formulierungen wie ``auf der n"achsten Seite'', im letzten Kapitel
 oder ``weiter unten'' 
sind nicht zweckm"assig, denn das Seitenformat kann kurzfristig
"andern, so dass sich ihr ganzer Text verschieben kann.
\item
Sie haben keine Kontrolle dar"uber, wo Bilder platziert werden.
Bilder werden mit dem \texttt{figure}-Environment erzeugt, ohne
Optionen, mit denen Sie die Platzierung steuern k"onnen.
Einzige Ausnahme: wenn Bilder eine "ahnliche Funktion wie Formeln haben
(wie die Analysator-Diagramme im Kapitel~2 des Skriptteiles).
\item
Wenn Sie Informationen aus Quellen "ubernehmen, dann m"ussen Sie
diese in einem Literaturverzeichnis ausweisen.
Literatur oder Websites, die Sie zu Ihrer Inspiration verwendet haben,
die aber keinen direkten Niederschlag in Ihrem Text gefunden hat,
geh"ort nicht ins Literaturverzeichnis.
\item
Formeln k"onnen im Fliesstext oder in sogenannten ``Displays'' 
dargestellt werden.
F"ur den Fliesstext eignen sich nur Formeln, die nicht h"oher als
eine Schriftzeile sind, und nur so kurz, dass sie nicht umgebrochen
werden m"ussen.
Freigestellte Formeln sollen so formuliert sein, dass sie m"oglichst
genau und leicht lesbar den mathematischen Gehalt wiedergeben.
So sind zum Beispiel Wurzelzeichen leichter lesbar als Exponenten $\frac12$.
\item
Ihre Graphiken sollen aussagekr"aftig und einfach lesbar sein. 
Vermeiden Sie die phantasievollen Verzierungen wie Farben,
und Schattierungen, mit denen Excel fehlende Substanz zu verdecken
versucht.
Screen-Shots oder Pixel-Graphiken sind selten f"ur ein gedrucktes
Dokument geeignet.
Am besten geeignet sind PDF-Graphiken, wie man sie mit Octave oder
R, oder mit etwas mehr Aufwand mit Metapost herstellen kann.
\item
Stellen Sie sicher, dass alle Begriffe und Konzepte, die man in Ihrer
Arbeit k"onnte suchen wollen, mit \verb+\index+-Befehlen ausgezeichnet
werden und so im Index des Seminar-Buches gesammelt werden.
\end{enumerate}

\section{Die Aufgaben}
\input aufgaben.tex

\end{document}
