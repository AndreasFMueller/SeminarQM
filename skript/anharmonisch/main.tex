\chapter{Anharmonischer Oszillator\label{chapter:anharmonisch}}
\lhead{Anharmonischer Oszillator}
\begin{refsection}
\chapterauthor{Joel Brunner und Christian Cavegn}

\newpage
\section{Einleitung}
\rhead{Einleitung}
In diesem Kapitel erweitern wir den harmonischen Oszillator aus Kapitel 8 mit Hilfe
der St"orungstheorie aus Kapitel 10. Der harmonische Oszillator ist eine gute
Approximation f"ur viele Quantenmechanische Systeme. Will man die Approximation
verbessern, kann man die Kräfte im Oszillator nicht mehr linear modellieren. Das
Potential ist also nicht mehr nur $Q^2$, sondern es kommen noch Terme h"oherer Ordnung
hinzu. Diese Terme können als St"orungen betrachtet werden.

\section{Anharmonizit"at}
\rhead{Anharmonizit"at}

Auf der Abbildung TODO sieht man das Potential des harmonischen Oszillators. Das
System ist linear und wurde in Kapitel 8 vollständig gelöst.\\
Auf der Abbildung TODO sieht man das Potential des gest"orten Oszillators.
In diesem Fall sind zus"atzlich noch Terme wie $aQ^3$ oder $bQ^4$ im Potential
enhalten.
Der harmonische Term ist aber immer noch der dominierende.

\section{Harmonische Gr"ossen}
\rhead{Harmonische Gr"ossen}

\subsection{Wellenfunktionen}
Der Grundzustand (Formel xxx) har die Form einer Gauss-Kurve.
\begin{equation}
\Psi_0(x)
=
\biggl(\frac{m\omega}{\pi\hbar}\biggr)^\frac14
e^{-\frac12\frac{m\omega}{\hbar}x^2}
\end{equation}
Von der Formel (8.10) TODO wird eine eingache Iterationsfunktion Abgeleitet.
\begin{equation}
\Psi_k(x)
=
\frac1{\sqrt{n!}}\biggl(\frac1{\sqrt{2}}
\biggl(\sqrt{\frac{m\omega}{\hbar}x}-
\sqrt{\frac{\hbar}{m\omega}}\frac{\partial}{\partial x}\biggr)\biggr)^k
\biggl(\frac{m\omega}{\pi\hbar}\biggr)^\frac14
e^{-\frac12\frac{m\omega}{\hbar}x^2}
\end{equation}
In die Gleichung substituiert man
\[
\alpha=\sqrt{\frac{m\omega}\hbar}
\]
und formt sie ein wenig um.
\begin{equation}
\Psi_k(x)
=
\frac1{\sqrt{n!}}\frac1{\sqrt{2^k}}
\biggl(\frac{m\omega}{\pi\hbar}\biggr)^\frac14
\biggl(\alpha x-\frac1{\alpha}\frac{\partial}{\partial x}\biggr)^k
e^{-\frac12\alpha^2x^2}
\end{equation}
Durch ausmultiplizieren erhaltet man die entsprechende Wellenfunktion
\begin{equation}
\biggl(\alpha x-\frac1{\alpha}\frac{\partial}{\partial x}\biggr)^1
e^{-\frac12\alpha^2x^2}
=
\biggl(\alpha x-\frac1{\alpha}\frac{\partial}{\partial x}\biggr)
e^{-\frac12\alpha^2x^2}
=
(2\alpha x)e^{-\frac12\alpha^2x^2}
\end{equation}

\begin{equation}
\biggl(\alpha x-\frac1{\alpha}\frac{\partial}{\partial x}\biggr)^2
e^{-\frac12\alpha^2x^2}
=
\biggl(\alpha x-\frac1{\alpha}\frac{\partial}{\partial x}\biggr)
(2\alpha x)e^{-\frac12\alpha^2x^2}
=
\biggl(2\alpha^2 x^2-2\frac{\partial}{\partial x}x\biggr)
e^{-\frac12\alpha^2x^2}
\end{equation}
\begin{equation}
(2\alpha^2x^2-(2-2\alpha^2x^2))e^{-\frac12\alpha^2x^2}
=
(4\alpha^2x^2-2)e^{-\frac12\alpha^2x^2}
\end{equation}
Man erh"alt die charakteristischen Hermitpolynome $H_k$,welche wir mit folgender
Gleichung erhalten werden k"onnen.
\begin{equation}
H_k(x)
=
e^{\frac{x^2}2}\biggl(x-\frac{\partial}{\partial x}\biggr)^k
e^{-\frac{x^2}2}.
\end{equation}
Dadurch wird die Gleichung gek"urzt zu
\begin{equation}
Phi_k(x)
=
\biggl(\frac{m\omega}{\pi\hbar}\biggr)^\frac14
\frac1{\sqrt{2^k k!}}H_k\biggl(\sqrt{\frac{m\omega}\hbar}x\biggr)
e^{-\frac12\frac{m\omega}{\hbar}x^2}
\end{equation}
Das Hamilton Polynom lässt sich mit folgender Differeintialgleichung berechnen.
\begin{equation}
H_k(x)
=
(-1)^k e^{x^2}\frac{\mathrm d^n}{\mathrm d x^n}
e^{-x^2},
\end{equation}
Der Vorteil dieser Notation ist die sehr einfache Implementierung in den g"angigen
Berechunugstools\\
\\
Beispiele Retourwerte\\
TODO

\subsection{Energielevel}
\begin{align*}
E_n
=
\hbar\omega\biggl(n+\frac12\biggr)
\end{align*}
Funktion\\
\\
Beispiele Retourwerte\\
\\
TODO

\subsection{Störungstheorie}
Schr"odingergleichung in der St"orungstheorie
\begin{equation}
(H_0+\varepsilon H_1)|\Psi_k(\varepsilon)\rangle
=
E_k(\varepsilon)|\Psi_k(\varepsilon)\rangle
\end{equation}
Koeffizienten
\begin{align*}
E_k(\varepsilon)
=
E_k^{(0)}+\varepsilon E_k^{(1)}+\varepsilon^2 E_k^{(2)}+\dotsb
\end{align*}
\begin{align*}
|\Psi_k(\varepsilon)\rangle
=
|\Psi_k^{(0)}\rangle+\varepsilon|\Psi_k^{(1)}\rangle+
\varepsilon^2|\Psi_k^{(2)}\rangle+\dotsb
\end{align*}
Durch Geeignetes ausmultiplizieren wie in Formel TODO beschrieben kommt man auf
folgende generische Gleichung:
\begin{equation}
\langle\Psi_l^{(0)}|\Psi_k^{(p)}\rangle
=
\frac{C_{lk}^{p}-\langle\Psi_l^{(0)}|H_1|\Psi_k^{(k)}\rangle}
{E_l^{(0)}-E_k^{(0)}}
\end{equation}
\begin{equation}
E_k^{(p)}
=
\langle\Psi_l^{(0)}|H_1|\Psi_l^{(k)}\rangle-C_{lk}^{(p)}
\end{equation}
mit
\begin{equation}
C_lk^{(p)}
=
\displaystyle\sum_{j=2}^{p} E_k^{p-j-1}
\langle\Psi_l^{(0)}|\Psi_k^{(j-1)}\rangle
\end{equation}
Diese k"onnen einfach Implementiert werden\\
\\
Code TODO\\
\\
Das Skalarprodukt beschreibt die Abh"angigkeit zweier Vektoren. Um sich eine 
bessere Vorstellung zu verschaffen dient folgende Gleichung
\begin{align*}
\langle\Psi_l^{(0)}|\Psi_k^{(p)}\rangle^2
=
P\langle\Psi_l^{(0)}|\Psi_l^{(p)}\rangle
\end{align*}
Wie Wahrscheinlich ist es, dass die ungest"orte Wellenfunktion
TODO $\Psi_l^{(0)}>$ in der Zustandskorrektur $\Psi_k^{(p)}>$ vorkommt.
In Abbildung TODO werden die Skalare der Wellenfunktionen 0 bis 50
dargestellt.
\begin{equation}
\Psi_k^{(p)}
=
\imath\gamma|\Psi_l^{(0)}\rangle+
\displaystyle\sum_{l\neq k} \langle\Psi_l^{(0)}|\Psi_k^{(p)}\rangle
|\Psi_l^{(0)}\rangle
\end{equation}

\subsection{Auswertung}

\section{Infrarotspektroskopie}
\rhead{Infrarotspektroskopie}
TODO

\printbibliography[heading=subbibliography]
\end{refsection}

