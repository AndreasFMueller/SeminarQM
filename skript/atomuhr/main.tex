\chapter{Atomuhr\label{chapter:atomuhr}}
\lhead{Atomuhr}
\begin{refsection}
\chapterauthor{Stefan Steiner und Pascal Stump}

\section{Einleitung}
% Aus Aufgabenstellung A. Müller
%Ein Rubidium-Frequenznormal verwendet eine Eigenschaft von
%Rubidium-Atomen, um ein hochpräzises Frequenznormal ($10^{-11}$)
%bereitzustellen. Solche Frequenznormale werden zum Beispiel in 3G
%Basisstationen oder in Satelliten verwendet.
%
%Es wird erwartet, dass Sie anhand eines vereinfachten Modells
%erklären, wie ein solches Frequenznormal funktioniert. Welche
%äusseren Umstände könnten die Frequenz beeinflussen? Es steht
%ausserdem ein Exemplar eines LPRO-101 für Experimente und Demonstrationen
%zur Verfügung.

Ohne die quantenmechanischen Erkenntnisse, welche besonders zu Beginn
des letzten Jahrhunderts gemacht wurden, g"abe es keine Atomuhren.
Sie stellen eine pr"azise Zeitmessung sicher und ebnen den Weg f"ur
immer genauere Messungen in der praktischen Physik.

Zudem pr"agen diese Ger"ate unseren Alltag, ohne dass dies stark
bemerkt wird.  Ortung durch ein Globales Satelliten Navigationssystem
(GNSS), wie GPS oder Galileo, sind stark von pr"aziser Zeitmessung
abh"angig.  Jeder Satellit dieser Systeme beinhaltet mehrere
Atomuhren.  Dies gilt auch f"ur das Mobilfunknetz. Die einzelnen
Basisstationen m"ussen untereinander sehr genau synchronisiert sein,
auch da helfen diese genauen Taktgeber.

Neben diesen praktischen Anwendungen ist die genaue Messung einer
Sekunde relevant f"ur die fundamentalen Definitionen der
SI-Basiseinheiten. In der Abbildung \ref{fig:siBasis} sind die sieben
SI-Basis\-ein\-hei\-ten dargestellt.  Die gr"unen Einheiten (Sekunde,
Ampere, Meter und Candela) bauen bei ihrer Definition auf die Sekunde
auf.  Somit profitieren diese Einheiten direkt von einer genaueren
Messung einer Sekunde.  Wie die SI-Sekunde definiert ist, wird im
Abschnitt \ref{sec:gesch-der-atom} beschrieben.

\begin{figure}
  \centering
  \begin{tikzpicture}
    \tikzstyle{nodeStyle} = [circle, very thick,
                             draw=gray!40!black, fill=gray!30,
                             minimum size=1cm];
    \tikzstyle{nodeStyleS} = [circle, very thick,
                              draw=green!40!black, fill=green!20,
                              minimum size=1cm];
    \tikzstyle{conn}      = [->, >=stealth',
                             shorten >=0.1cm,
                             very thick];

        
    \node at (90:1.7cm) [draw, nodeStyle] (K) {\si{\kelvin}};
    \node at (39:1.7cm) [draw, nodeStyleS] (s) {\si{\second}};
    \node at (-13:1.7cm) [draw, nodeStyleS] (m) {\si{\meter}};
    \node at (-64:1.7cm) [draw, nodeStyle] (kg) {\si{\kilogram}};
    \node at (141:1.7cm) [draw, nodeStyleS] (A) {\si{\ampere}};
    \node at (193:1.7cm) [draw, nodeStyle] (mol) {\si{\mol}};
    \node at (244:1.7cm) [draw, nodeStyleS] (cd) {\si{\candela}};
  \end{tikzpicture}
  \caption{Die sieben SI-Basiseinheiten \cite{wiki:si} (gr"un: Einheiten, welche f"ur
    ihre eigene Definition die Sekunde ben"otigen)}
  \label{fig:siBasis}
\end{figure}

Dieses Kapitel soll einen kurzen "Uberblick geben, was der
quantenmechanische Hintergrund eines solchen Ger"ates ist. Ausserdem
soll die technische Realisierung am Beispiel einer Rubidium Atomuhr
erl"autert werden.


\section{Quantenmechnanische Betrachtung}

In diesem Teil soll gekl"art werden, wie mithilfe der Theorie der Quantenmechanik eine Atomuhr technisch realisiert werden kann. Dabei wird zuerst eine kurze Repetition zum Wasserstoff gegeben. Dann wird erkl"art, was die Feinstruktur ist. Dies f"uhrt dann zum Schluss zur Theorie der Hyperfeinstruktur.

\subsection{Repetition Wasserstoffatom}
Ernest Rutherford gewann nach seinen Experimenten anfangs des 20. Jahrhunderts die Erkenntnis, dass Elektronen Planeten förmig um den Kern kreisen müssen. 
Er versuchte dieses Modell mit der Newtonschen Mechanik zu erklären. 
Das hätte jedoch zur Folge, dass die Atome in den Kern stürzen müssten. 
Niels Bohr löste dieses Stabilitätsproblem mithilfe des Plankschen Energiequantum 
\begin{equation}
\varDelta E = h\nu.
\end{equation}
Daraus resultierte das Bohrsche Atommodell benannt nach dem dänischen Physiker Niels Bohr \cite{wiki:bohr}. 

In Kapitel \ref{chapter:wasserstoff} konnte mithilfe der zeitunabhängigen Schr"odingergleichung dieses Modell hergeleitet werden.
Es resultiert ein diskreter Abstand der Elektronen zum Kern.
Dies gilt jedoch nicht nur f"ur das Wasserstoffatom, sondern Allgemein f"ur alle Atome.
Nach dieser Theorie können sich Elektronen nur in einem wohldefinierten Abstand um den Atomkern befinden. 

Springt nun ein Elektron von einem tieferen in einen h"oheren Entartungsgrad, so wird ein Photon absorbiert.
Das passiert bei jeglicher Energiezufuhr zum Atom. 
Vice versa sendet das Atom ein Photon aus, wenn ein Elektron von einem h"oheren Energiegrad zu einen tieferen einen Quantensprung vollzieht.

Aus dem Energieunterschied und dem Plankschen Wirkungsquantum lässt sich die Frequenz des Photons berechnen. %, was den Wellencharakter des Photons unterstreicht.

Aus Tabelle \ref{skript:h2wellenlaengen} sind die Wellenl"angen solcher "Uberg"ange ersichtlich. Daraus ist es möglich, die Photonenfrequenzen zu berechnen.
%TODO \cite{Quelle Gertsen, Atomphysik und ...}
\begin{center}
	"Ubergang $H\alpha: 3 \rightarrow 2: \lambda_1 = \SI{656.3}{\nano\meter}$

$\nu_1 = \dfrac{c}{\lambda_1} = \SI{457.3}{\tera\hertz}$
\vspace{.5cm}

"Ubergang $8 \rightarrow 2: \lambda_2 = \SI{388.8}{\nano\meter}$

$\nu_2 = \dfrac{c}{\lambda_2} = \SI{771.2}{\tera\hertz}$
\end{center}	

Diese "Uberg"ange sind im Bereich von Terahertz bis Petahertz und eignen sich darum nicht um elektronisch verarbeitet werden. Elektronische Ger"ate arbeiten im Bereich von Gigahertz.

\begin{figure}[h!]
	\centering
	%TODO im militär ist uploaden teuer, kopiere es dann in den richtigen Ordner, wenn ich zu Hause bin.
	\includegraphics[width = .6\columnwidth]{../vortrag/pictures/wasserstoffSpektrum.jpg}
	\caption{Wasserstoffspektrum} %TODO \cite{link im owncloud ordner}}
	\label{atomuhr:wasserstoffspektrum}
\end{figure}

Auf der Abbildung \ref{atomuhr:wasserstoffspektrum} sieht man das Frequenzspektrum der sogenannte Balmer-Serie des Wasserstoffs, welche zufälligerweise im sichtbaren Bereich ist. Der rote Strich ganz rechts stellt die $H\alpha$-Linie dar. Die nächste links $H\beta$ und so weiter.


\subsection{Feinstruktur"ubergang}
Damit jedoch ein solcher "Ubergang elektrisch verarbeitet werden kann, muss es einen Effekt geben, bei welchem das Photon mit einer Frequenz im Gigahertz Bereich ausgesendet wird. 
Der Feinstruktur"ubergang ist ein erster Schritt in diese Richtung. 

Als das Wasserstoffspektrum genauer untersuchte wurde, bemerkte man, dass die Spektrallinien nicht atomar sind. Wenn man beispielsweise die $H\alpha$ Spektrallinie besser auflöst, so entdeckt man, dass diese eine Linie aus zwei besteht, welche sehr nahe aneinander liegen. 

\begin{figure}[h!]
	\centering
	%TODO im militär ist uploaden teuer, kopiere es dann in den richtigen Ordner, wenn ich zu Hause bin.
	\includegraphics[width = .6\columnwidth]{../vortrag/pictures/fine_structure_hydrogen.png}
	\caption{$H\alpha$-Linie stark vergr"ossert, erkennbare Doppellinie} %TODO \cite{link im owncloud ordner}}
\end{figure}

Bis jetzt wurde nur mit der zeitunabhängigen Schr"odingergleichung gearbeitet.
Als erste N"aherung ist dies sicherlich nicht schlecht.
In der Realit"at spielt jedoch Zeitabh"angigkeit ein wichtiger Faktor. 
Mithilfe der St"orungstheorie ist es m"oglich, dieses Verhalten anzun"ahern.

\begin{itemize}
	\item[]  $H = \textcolor{red}{H_0}+ \textcolor{blue}{W_{SB}} + 
		\textcolor{green}{W_M} + \textcolor{violet}{W_D} $
	\item[]  $H$: relativistischer Hamiltonoperator
	\item[]  \textcolor{red}{$H_0$}: nicht relativistischer Hamiltonoperator
	\item[]  \textcolor{blue}{$W_{SB}$}: Spin Bahn Kopplung
	\item[]  \textcolor{green}{$W_M$}: Korrektur kin. Energie
	\item[]  \textcolor{violet}{$W_D$}: Korrektur pot. Energie
	
\end{itemize}
		
Auf die Korrekturfaktoren $W_M$ und $W_D$ wird hier nicht n"aher eingegangen, da diese nur zu geringf"ugigen Korrekturen f"uhrt und nicht f"ur die Doppellinie verantwortlich ist.

Der nicht relativistische Hamiltonoperator beschreibt das Bohrsche  Atommodell, welches bereits bekannt ist.

Bei der Spin Bahn Kopplung geht es um die reine Betrachtung des Elektrons. 
Dieses dreht sich in einer Kreisf"ormigen Bahn. 
Da das Elektron eine Ladung besitzt, entsteht ein magnetisches Feld. 
Dieses Feld wechselwirkt nun mit dem magnetischen Spin, den das Elektron auch noch besitzt. 
Da das Elektron entweder Spin +1/2 oder -1/2 hat, entstehen geringf"ugig unterschiedliche Frequenzen bei einem Quantensprung in das jeweilige Energielevel.

\begin{figure}
	\centering
	\includegraphics[width=.2\columnwidth]{../vortrag/pictures/feinstrukturelektron.jpg}
	\caption{Spin Bahn Kopplung} %TODO \cite{link im owncloud ordner}}
	\label{atomuhr:spinbahn}
\end{figure}

\subsubsection{Notation Termsymbol}
Jedes Atom hat nun verschiedene Entartungsgrade, in denen sich Elektronen befinden. Zus"atzlich ist die Annahme, dass das Elektron um das Atom kreist falsch. Die Elektronen befinden sich in einem bestimmten Sektor, welchem ein gewisser Bahndrehimpuls zugewiesen ist. 
Abbildung \ref{}
\begin{figure}
	\centering
	\includegraphics[width = 0.8\columnwidth]{../vortrag/pictures/orbitale.JPG}
	\caption{elektrische Bahndrehimpulse} %TODO \cite{link im owncloud ordner}}
	\label{atomuhr:bahndrehimpuls}
\end{figure}
Ausserdem hat das Elektron zwei verschiedene Spinzust"ande in dem es sich befinden kann. Der Aufenthaltsort eines Elektrons ist ein wenig komplizierter als noch beim Bohrschen Atommodell.
Dazu wurde die Termsymbol Notation entwickelt, welche den Zustand des Elektrons genau beschreibt. 
Diese wird nachfolgend kurz erkl"art.

\begin{equation}
	\text{Termsymbol:} \quad ^NL _J
\end{equation}

wobei

\begin{equation}
	J = L + S
\end{equation}

$N:$ Entartungsgrad

$L:$ elektrische Bahndrehimpuls

$J:$ Gesamtdrehimpuls

$S:$ Spin elektron

Dem elektrischen Bahndrehimpuls sind dabei den Zeichen folgende Zahlen zugewiesen.

\begin{table}
	\centering	
	\begin{tabular}{llll}
		s & p & d & f \\
		1 & 2 & 3 & 4 \\
	\end{tabular}
	\caption{elektrische Bahndrehimpulse}
	\label{atomuhr:drehimpulsnotation}
\end{table}

Mit diesem Wissen l"asst sich nun ein Feinstruktur"ubergang ausrechnen. 
\begin{center}
		"Ubergang: $^2P_{3/2} \rightarrow ^2P_{1/2}: \lambda = \SI{2.76}{\centi\meter}$
		
		$\nu_1 = \dfrac{c}{\lambda_1} = \SI{10.9}{\giga\hertz}$
\end{center}

Das ausgesendete Photon befindet sich bei diesen Feinstruktur"uberg"angen in einem Bereich, in dem sie schon eher genutzt werden k"onnte. F"ur Atomuhren wird jedoch der  Hyperfeinstruktur"ubergang genutzt. 

\subsection{Hyperfeinstruktur"ubergang}
\label{sec:hyperf}
Der Feinstruktur"ubergang wurde entdeckt, als das Bohrsche Modell experimentell genauer untersucht wurde. "Ahnlich ergeht es dem Hyperfeinstruktur"ubergang. Als das Spektrum des Feinstruktur"ubergang besser aufgel"ost werden konnte, ist aufgefallen, das es noch einen Effekt gibt, welcher zu einer Aufspaltung eines Energielevels beim Feinstruktur"ubergang f"uhrt. Dieser wurde Hyperfeinstruktur"ubergang genannt und wird hier nun erkl"art.

Dieser Struktur"ubergang kann mithilfe der Wechselwirkung zwischen Elektron und Kern erkl"art werden. Der Kern besitzt einen gewissen Kernspin. Der magnetische Spin des Elektrons kann nun parallel oder antiparallel zum Kern sein. Was zu unterschiedlichen Energieniveaus f"uhrt. 

Beim Wasserstoff, welches im Grundzustand besetzt ist, führt das zu einer Wellenl"ange von $\lambda = \SI{21}{\centi\meter}$, was einer Frequenz von $\nu = \SI{1.42}{\giga\hertz}$ entspricht. Das ist eine Frequenz, welche sich sehr gut mit elektronischen Ger"aten verarbeiten l"asst. 

\subsection{Alkalimetalle}
Gleich wie Wasserstoff geh"ort auch Rubidium und C"asium zu den Alkalimetallen. speziell an dieser Gruppe von Elementen ist, dass sie nur ein freies Elektron in der letzten besetzten Schale haben. 

Da alle Atome Edelgaskonfiguration anstreben, wollen die Alkalimetalle dieses freie Elekron m"oglichst abgeben und sind dadurch relativ Reaktionsfreudig. 
Die abgeschlossenen Schalen sind elektrisch gesehen neutral. 
Weshalb die Alkalimetalle mutmasslich f"ur Atomuhren gebraucht wird. 
Das Frequenzspektrum ist einfacher zu verstehen als bei anderen Atomgruppen mit mehreren Valenzelektronen.

Rubidium hat einen Hyperfrequenz"ubergang von $\nu = \SI{6.834}{\giga\hertz}$. Die h"ohere frequenz l"asst sich damit erkl"aren, dass sich das Valenzelektron weiter weg vom Kern befindet als beim Wasserstoffatom. Die Kraft auf die Wechselwirkung ist somit kleiner. 

\section{Technische Betrachtung}
In der technischen Betrachtung in diesem Abschnitt wird die
Funktionsf"ahigkeit des Rubidium-Frequenznormals LPRO-101 aufgezeigt.
Das Datenblatt dieser Atomuhr kann unter \cite{datasheet:lpro}
gefunden werden.  Ein ausf"uhrliches Datenblatt ist unter
\cite{datasheet:prs10m} zu finden, in welchem ein anderes Modell
erkl"ahrt wird.  Die Funktionsbeschreibung ist jedoch besser.

Solche Rubidium Frequenznormale besitzen die Gr"osse einer Handfl"ache
und erreichen eine Genauigkeit von ungef"ahr \num{e-9}.  In der
Abbildung \ref{fig:techBlock} ist der schematische Aufbau einer
solchen Atomuhr aufgezeigt.  Die Hauptfunktion dieser Schaltung ist
ein Regelkreis, welcher sich auf den Hyperfeinstrucktur"ubergang des
Rubidium-87 ein regelt, welcher bei \SI{6.8346875}{\giga\hertz} liegt.

\begin{figure}
  \centering
  \begin{tikzpicture}
    [node distance=1.3cm]

    \tikzstyle{conn} = [->,shorten >=0.1cm,>=stealth',very thick];

    \tikzstyle{connLamp} = [->, decorate, 
                            decoration={snake,
                              amplitude=1.5mm,
                              segment length=1.7mm,
                              post length=2.7mm},
                            shorten >=0.1cm,
                            >=stealth', very thick,
                            color=red!50!black];

    \tikzstyle{connAbsorber} = [->, shorten >=0.1cm,
                                >=stealth', very thick];

    \tikzstyle{connHfgen} = [->, decorate,
                             decoration={snake,
                               amplitude=1.5mm,
                               segment length=9mm,
                               post length=2.5mm},
                             shorten >=0.1cm,
                             >=stealth',very thick];

    \tikzstyle{lamp} = [draw=red!50!black,
                        thick,circle,inner sep=1.5mm,
                        fill=red!30];

    \tikzstyle{absorber} = [draw, thin, circle,
                            shade, shading=radial,
                            inner color=blue!70];

    \tikzstyle{vcxo} = [draw, thick, rectangle, inner sep=5mm,
                        align=center, minimum width=2.7cm];
    \tikzstyle{hfgen} = [draw,thick, rectangle, inner sep=5mm];

    % drawing
    \node[lamp, minimum size=1.7cm] (lamp) {$^{87}$Ru};

    \node[absorber, minimum size=1.7cm] (absorber) [right=of lamp] {};

    \node (diode) [right=of absorber] {};

    \draw[semithick](diode) ++ (0.0,-1.0) to[empty photodiode]
                            ++ (0.0,2.0);

    \node[vcxo] (vcxo) [right=of diode] {\SI{20}{\mega\hertz}\\VCXO};

    \node[hfgen] (hfgen) [below=of absorber] {\(\times\)};

    \node[xshift=1cm] (note) [left=of hfgen] {\SI{6.8346875}{\giga\hertz}};

    % connections
    \draw[connLamp] (lamp.east) -- (absorber.west);
    \draw[connAbsorber, shorten >=0.4cm] (absorber.east) -- (diode.west);
    \draw[conn] (diode.east) ++ (0.3cm,0.0) -- (vcxo.west);
    \draw[conn] (vcxo.south) |- (hfgen.east);
    \draw[connHfgen] (hfgen.north) -- (absorber.south);
\end{tikzpicture}
  \caption{Schematischer Aufbau einer Rubidium Atomuhr}
  \label{fig:techBlock}
\end{figure}

\subsection{Funktionsweise}
Die Rubidium-87 Entladungslampe wird angeregt und emittiert Licht bei
ca. \SI{780}{\nano\meter}.  Dieses Licht wird in die Resonator-Kammer
geleitet, in welcher ein Absorptionsgas aus Rubidium-87 vorhanden ist.
Weiter ist in der Entladungslampe und der Resonatorkammer ein
Filtergas aus Rubidium-85 vorhanden.  In der Abbildung
\ref{fig:uebergaenge} ist die Filterung dargestellt.

\begin{figure}
  \centering
  \begin{tikzpicture}
  \tikzstyle{trans} = [->,>=stealth',very thick];


  \draw (0,0) node[above left] {5s} --++ (7.5cm,0)
  node[above right] {\SI{6.8}{\giga\hertz}};
  \draw (0,0.4cm) --++ (7.5cm,0);
  \draw (0,4cm) node[left] {5p} --++ (7.5cm,0);

  % lamp
  \draw[very thick] (0,0cm)   --++ (1.5cm,0);
  \draw[very thick] (0,0.4cm) --++ (1.5cm,0);

  \draw[very thick] (0,4cm)   --++ (1.5cm,0);

  \draw[trans, red!70!black] (0.5cm,4cm) -- (0.5cm,0.4cm);
  \draw[trans, blue!70!black] (1cm,4cm) -- (1cm,0cm);

  \node[align=center] at (0.75cm,-1cm) {$^{87}$Rb\\Emission};

  % filter
  \draw[very thick] (3,0.4cm) --++ (1.5cm,0);
  \draw[very thick] (3,0.8cm) --++ (1.5cm,0);

  \draw[very thick] (3,4cm)   --++ (1.5cm,0);

  \draw[trans, red!70!black] (3.5cm,0.4cm) -- (3.5cm,4cm);
  \draw[trans, red!70!black] (4cm,4cm) -- (4cm,0.4cm);

  \node[align=center] at (3.75cm,-1cm) {$^{85}$Rb\\Filter};

  % resonator
  \draw[very thick] (6,0cm)   --++ (1.5cm,0);
  \draw[very thick] (6,0.4cm) --++ (1.5cm,0);

  \draw[very thick] (6,4cm)   --++ (1.5cm,0);

  \draw[trans, red!70!black] (6.5cm,4cm) -- (6.5cm,0.4cm);
  \draw[trans, blue!70!black] (7cm,0cm) -- (7cm,4cm);

  \node[align=center] at (6.75cm,-1cm) {$^{87}$Rb\\Resonator};
\end{tikzpicture}
  \caption{Energie"uberh"ange in einer Rubidium Atomuhr}
  \label{fig:uebergaenge}
\end{figure}

Die Entladungslampe emittiert zwei sehr "ahnliche Frequenzen, da das
untere Energieniveau 5s der Rubidium-87 Atome nur durch ihren
unterschiedlichen Spin aufgeteilt wird.  Die zur Filterung
eingesetzten Rubidium-85 Atome besitzen ein leicht verschobenes
unteres Energieniveau, welches mit einem der beiden 5s Energieniveaus
von Rubidium-87 "ubereinstimmt.  Dadurch wird einer der beiden
Emittierten Frequenzen des Rubidium-87 herausgefiltert.  So erreicht
nur eine der beiden Frequenzen die Resonator-Kammer.

Im Resonator absorbiert das Rubidium-87 die eintreffenden
Photonen. Das Elektron steigt in das h"ohere Energieniveau 5p auf und
f"allt spontan mit einer Wahrscheinlichkeit von \SI{50}{\percent} in
eine der beiden unteren Energieniveaus von 5s zur"uck.  Mit der Zeit
sammeln sich mehr Elektronen im oberen der beiden Energieniveaus von
5s an, da keine Photonen mit dieser ben"otigten Frequenz zum Resonator
(Filterung durch Rubidium-85) gelangen.  Es werden weniger Photonen
absorbiert, was der Detektor messen kann.  (Eine weitere Anwendung,
welche mit den Energie"uberg"angen arbeitet ist der Laser, zu finden
im Kapitel \ref{chapter:laser}.)

Der Detektor steuert einen spannungsgesteuerten Oszillator (VCXO) an,
welcher eine Frequenz um \SI{20}{\mega\hertz} ausgibt.  Dieses Signal
wird mit einem Multiplizierer auf die Frequenz des
Hyperfeinstruktur"ubergang hinauf multipliziert.

Diese hinauf multiplizierte Frequenz liegt nun im Mikrowellenbereich
und wird zur"uck in den Resonator geleitet.  Wird nun die Frequenz von
\SI{6.8346875}{\giga\hertz} genau getroffen, wird am VCXO genau
\SI{20}{\mega\hertz} ausgegeben.  Weiter entspricht die
Mikrowellenstrahlung nun genau dem Hyperfeinstrucktur"ubergang, was
dazu f"uhrt, dass es zu einer Durchmischung des unteren Energieniveaus
im Resonator kommt.  Die festsitzenden Elektroden gelangen nun wieder
in den untersten Energiezustand von 5s.  Somit k"onnen wieder mehr
Photonen absorbiert werden, was zu einer Abdunklung des Detektors
f"uhrt.

Der ganze Regelkreis regelt somit auf die maximale Absorption im
Resonator und f"uhrt zu einer sehr genauen Ausgangsfrquenz von
\SI{20}{\mega\hertz}.

\subsection{Tuning}
Es stellt sich nun die Frage, wie ein solcher Regelkreis abgeglichen
werden kann.  Zum Beispiel m"ussen die GNSS Satelliten untereinander
und mit der Bodenstation synchronisiert werden.  Wie man in den
quantenmechanischen Betrachtungen gesehen hat, h"angt der
Hyperfeinstruktur"ubergang (\ref{sec:hyperf}) nur vom Magnetfeld des
Atomkerns und des Spin Magnetfelds ab.

Nun kann man diese Magnetfelder mit einem externen Magnetfeld st"oren,
was zu einer Frequenzverschiebung des Hyperfeinstrucktur"ubergangs
f"uhrt.  Das Magnetfeld wird wie in Abbildung \ref{fig:tuning}
dargestellt, am Resonator angelegt.  Mit dieser St"orung kann nun die
gew"unschte Frequenz mit einer genaueren Caesium Atomuhr oder dem GPS
Takt abgeglichen werden.

Wie stark sich nun diese Frequenz durch eine magnetische St"orung
beeinflussen l"asst, wird im n"achsten Abschnitt ausgerechnet.

\begin{figure}
  \centering
  \input{atomuhr/tuning.tex}
  \caption{Frequenztuning durch Magnetfeld-St"orung am Resonator
    (gr"un angelegtes Magnetfeld).}
  \label{fig:tuning}
\end{figure}

\subsection{Rechnung}

\section{Geschichte der Atomuhr}
\label{sec:gesch-der-atom}
Bereits 1879 hatte Lord Kelvin die Idee, Atome als Zeit und L"angen
Standard zu verwenden.  Erst 1949 wurde dann in der USA am National
Institute of Standards and Technology (NIST) die erste Atomuhr gebaut
\cite{ieee:nist}.  Diese Atomuhr verwendete Ammoniak bei einer
Frequenz von \SI{23.87}{\giga\hertz}.  Die Genauigkeit dieses Atomuhr
lag bei ca. 1 zu \num{e-7}.

Diese Genauigkeit ist noch nicht sehr eindr"ucklich, wenn man
vergleicht, dass ein normaler Quarzoszillator bei einer Genauigkeit
von ca. \num{e-5} und eine beheizter Quarzoszillator (OCXO) durch den
reduzierten Temperaturdrift im Bereich von \num{2e-8} liegt.

Mit der Ammoniak Atomuhr konnte aber das Prinzip der Atomuhr
aufgezeigt werden.  Welche 1952 mit der ersten Caesium Atomuhr NBS-1
weitergef"uhrt werden konnte.  Die Genauigkeit und Stabilit"at der
Caesium Atomuhren konnte soweit gesteigert werden, dass 1967 die
SI-Sekunde als Hyperfeinstruktur"ubergang von Caesium-133 definiert
wurde.  Das f"uhrt dazu, dass eine Sekunde mit \num{9192631770} Takten
aufgel"ost wird.  Somit ist die Genauigkeit auf
\SI{108.78}{\nano\second} beschr"ankt.

Andere Atomuhren, welche in der Zwischenzeit entwickelt wurden,
verwenden die Elemente Rubidium und Wasserstoff.  Beide verwenden
ebenfalls den Hyperfeinstruktur"ubergang.  Die Rubidium Atomuhren
k"onnen sehr einfach und g"unstig aufgebaut werden besitzen jedoch
nicht die Genauigkeit von Caesium Atomuhren.  Genauer k"onnen die
Wasserstoffatomuhren gebaut werden, ben"otigen jedoch eine komplexere
K"uhlung der Atome.

Will man eine genauere Aufl"osung erreichen, gen"ugt es nicht die
Hyperfeinstruktur"uberg"ange zu benutzen.  Man muss eine h"ohere
Taktung erreichen.  Diese h"oheren Taktungen k"onnen erreicht werden,
wenn Atome mit Spektrallinien im optische Bereich verwendet werden.
Lange konnte dieser Bereich nicht genutzt werden, da diese hohen
Takte mit elektronischen Regelkreisen nicht gemessen werden konnten.

Gl"ucklicherweise wurde 1998 der Frequenzkamm entdeckt
\cite{SdW:kamm}.  Mit dessen Hilft k"onnen Frequenzen im optischen
Spektrum mit einer Schwebung in den technisch messbaren
Frequenzbereich gebracht werden.  Diese Technologie erm"oglichte es
andere Atome zu verwenden, mit welchen man 2014 eine Genauigkeit von
\num{e-18} erreichte, was ca. einer Sekunde auf das alter des
Universums entspricht.  In der Abbildung \ref{fig:periode} ist eine
nicht abschiessende Auflistung der verwendeten Elemente in Atomuhren
dargestellt.

\begin{figure}
  \centering
  \input{atomuhr/elements.tex}
  \caption{Nicht abschliessende Aufstellung der verwendeten Elemente
    in Atomuhren (gelb: Hyperfeinstruktur"ubergang; Rot: andere
    "Uberg"ange)}
  \label{fig:periode}
\end{figure}

\section{Zusammenfassung}
Die heute eingesetzten Atomuhren benutzen den
Hyperfeinstruktur"ubergang, als hoch genaue Frequenzkonstante.  Dieser
Hyperfeinstruktur"ubergang kann mit Hilfe der St"orungstheorie
erkl"art werden, wie am Anfang dieses Kapitels beschrieben wurde.
Obwohl in den Medien immer wieder von noch genaueren Atomuhren die
Rede ist, sind die in der Realit"at eingesetzten um einige 10er
Potenzen weniger genau, besitzen aber eine Langzeitstabilit"at die den
technischen Anforderungen entsprechen.  Vor allem werden heutzutage
Caesium Atomuhren eingesetzt, da auch die SI-Sekunde als
Hyperfeinstruktur"ubergang von Caesium-133 definiert ist.  In Zukunft
wird die Sekundendefinition nach dem Stand der zuk"unftigen Technik
angepasst werden.

\printbibliography[heading=subbibliography]
\end{refsection}

