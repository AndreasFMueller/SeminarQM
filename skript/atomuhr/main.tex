\chapter{Atomuhr\label{chapter:atomuhr}}
\lhead{Atomuhr}
\begin{refsection}
\chapterauthor{Stefan Steiner und Pascal Stump}

\Section{Einleitung} % (Aufgabenstellung)
%Ein Rubidium-Frequenznormal verwendet eine Eigenschaft von
%Rubidium-Atomen, um ein hochpräzises Frequenznormal ($10^{-11}$)
%bereitzustellen. Solche Frequenznormale werden zum Beispiel in 3G
%Basisstationen oder in Satelliten verwendet.
%
%Es wird erwartet, dass Sie anhand eines vereinfachten Modells
%erklären, wie ein solches Frequenznormal funktioniert. Welche
%äusseren Umstände könnten die Frequenz beeinflussen? Es steht
%ausserdem ein Exemplar eines LPRO-101 für Experimente und Demonstrationen
%zur Verfügung.

In der heutigen Zeit sind Atomuhren zwar unscheinbare, aber trotzdem sehr wichtige Geräte, da sie es ermöglichen, Zeit unglaublich präzise zu messen. 
Gewisse Systeme, wie die Positionsmessung über einen Satelliten durch ein Global Navigation Satellite System (GNSS), oder um Mobilfunk und mobile Datenübertragung zu ermöglichen, hängen sehr stark von solch genauer Zeit ab.
Mit Atomuhren ist es möglich dies zu bewerkstelligen. %blabla schlecht... aber so in diese Richtung

\Subsection{Anwendungen}
	

\Section{Quantenmechnanische Betrachtung}
%Quantenmechanische Hintergrund


%\Subsection{Repetition Wasserstoffatom}
%Wasserstoffatom übergang -> im GHz bereich

\Subsection{Feinstrukturübergang}
%Beweis für Elektronenspin
%Bild von übergang
\Subsection{Hyperfeinstrukturübergang}
% Wechselwirkung Elektron <-> Spin
% Rubidium \nu = 6.834 GHz 

\Section{Technische Betrachtung}

\Subsection{Rechnung}

\Section{Ausblick} 

\Section{Zusammenfassung}

\printbibliography[heading=subbibliography]
\end{refsection}

