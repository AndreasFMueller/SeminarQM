\chapter{Bellsche Ungleichung\label{chapter:bell}}
\lhead{Bellsche Ungleichung}
\begin{refsection}
\chapterauthor{Hannes Badertscher}

Im Mai 1935 ver\"offentlichten Albert Einstein, Boris Podolsky und
Nathan Rosen ein Gedankenexperiment mit welchem sie erkl\"arten, wieso
die Quantenmechanik nicht vollst\"andig sein kann. Als Alternative
sollten zus\"atzliche, verborgene, Variablen eingef\"ugt werden, welche
das zugrunde liegende Verhalten logisch erkl\"aren k\"onnen.

Bis 1971 war die Theorie der verborgenen lokalen Variablen nach Einstein
die verbreitete Ansicht -- die Frage war nur wann solche Variablen
entdeckt werden. In vielen Bereichen der Quantenmechanik wurden tatsächlich
kleinere Teilchen, Quantenzahlen und Variablen entdeckt, welche das
Verständnis der heutigen Quantenphysik prägten.
%TODO

\section{Lokalit\"at und Realismus\label{section:bell:lokalitaet}}
Zuerst stellt sich die Frage der Anforderungen an die Theorie der
Quantenmechanik. Eine zentrale Eigenschaft ist hierbei die Lokalit\"at.
Lokalit\"at bedeutet hierbei, dass jegliche Vorg\"ange nur einen Einfluss
auf ihre direkte Umgebung haben k\"onnen.

Die klassische Newtonsche Mechanik ist eine \emph{nicht}-lokale 
Theorie. Eine Masse hat in der Gravitation sofort, ohne zeitliche 
Verz"ogerung einen Einfluss auf andere Massen, unabh\"angig von der
r\"aumlichen Distanz zwischen den beiden Massen. 
In der speziellen Relativit\"atstheorie 
\textsuperscript{[Citation needed]}
wurden die Begriffe von Raum und Zeit so definiert, dass sich alle
Materie und Energie sich h\"ochstens mit Lichtgeschwindigkeit fortbewegen
kann. 
Die allgemeine Relativit\"atstheorie ist eine alternative Formulierung
der Newtonschen Gravitationstheorie, welche die in der speziellen
Relativit\"atstheorie geforderte Lokalit\"at erf\"ullt.
Von der Quantenmechanik kann ebenfalls erwartet werden, dass sich
jegliche Ereignisse h\"ochstens mit Lichtgeschwindigkeit ausbreiten
k\"onnen.

Eine Theorie wird als \emph{realistisch} bezeichnet, wenn die Theorie
nicht direkt von der Messung abh\"angt. Eine Messung hat also einen
vordefinierten Wert, auch wenn dieser nicht bekannt ist.
Die bekannte Frage ''Is the moon there if nobody looks?{``} zeigt auf,
dass der Realismus rein intuitiv immer erf\"ullt sein m\"usste. Schliesslich
zweifelt niemand daran, dass der Mond auch existiert wenn gerade niemand
zum Mond hinaufschaut.

\section{Das Einstein-Podolsky-Rosen Paradoxon}
Das Einstein-Podolsky-Rosen Paradoxon (EPR Paradoxon) ist ein 
Gedankenexperiment, welches 1935 im Paper \cite{Einstein1935}
\emph{Can quantum-mechanical description of physical reality be considered complete}
von Albert Einstein, Boris Podolsky und Nathan Rosen publiziert wurde,
bezeichnet.
Wie der Titel bereits verr\"at, gehen Einstein \emph{et al.} dabei auf
die Frage ein, ob die Theorie Quantenmechanik vollständig ist, oder
ob damit die Realit\"at (noch) nicht korrekt beschrieben werden kann.
Einstein geht dabei von den in Abschnitt \ref{section:bell:lokalitaet}
beschriebenen Annahmen der Lokalit\"at und des Realismus aus.

\subsection{Das EPR Paper}
Das EPR Paper betrachtet zwei quantenmechanische Systeme, $I$ und $II$,
welche von $t=0$ bis zu einem Zeitpunkt $t=T$ miteinander 
interagieren k\"onnen. Darauf ist keine Interaktion mehr m\"oglich.
Visualisiert wird das Gedankenexperiment durch eine Quelle, welche
Elektron-Positron Paare emittiert. 
Dabei wird das Elektron ($I$) zu einem Detektoren $A$ und das 
Positron ($II$) zu einem Detektoren $B$ gesandt.
Diese sind physikalisch getrennt, sodass kein Informationsaustausch
zwischen den Detektoren m\"oglich ist.
Der Messaufbau ist in \figurename~\ref{fig:bell:EPR_Messaufbau} abgebildet.

\begin{figure}
    \centering
    \includegraphics{bell/images/experiment_setup.pdf}
    \caption{Messaufbau im Gedankenexperiment von EPR}
    \label{fig:bell:EPR_Messaufbau}
\end{figure}

Diese zwei Partikel werden so generiert, dass sie einen komplement\"aren
Spin-Zustand besitzen. 
Wenn also bei Detektor $A$ der Spin des Elektrons entlang der $z$-Achse
gemessen wird und das Messresultat $+z$ ist, so besitzt das Positron bei
Detektor $B$ den Spin $-z$.

Aus der Heisenbergschen Unsch\"arferelation
\textsuperscript{[Citation needed]}
folgt, dass der Zustand zweier Observablen, deren Kommuntator
nicht $0$ betr\"agt, d.h. die nicht vertauschen, nicht gleichzeitig
bekannt sein kann.

Das Paper endet mit den Worten
\begin{quote}
    While we have thus shown that the wave function does not provide
    a complete description of the physical reality, we left open the
    question of whether or not such a description exists.
    We believe, however, that such a theory is possible.
\end{quote}
%TODO: weiter

\subsection{Mathematische Herleitung}
Anhand der quantenmechanischen Definition des Spins
\textsuperscript{[Citation needed]}
kann das EPR Paradoxon mathematisch beschrieben werden.
Die Spin Operatoren in $x-$, $y-$ bzw. $z-$ Richtung werden dabei durch
die Pauli Matrizen 

\begin{align*}
    S_x &= \frac{\hbar}{2} \begin{bmatrix}
    0 & 1 \\ 1 & 0
    \end{bmatrix}
    &
    S_y &= \frac{\hbar}{2} \begin{bmatrix}
    0 & -i \\ i & 0
    \end{bmatrix}
    &
    S_z &= \frac{\hbar}{2} \begin{bmatrix}
    1 & 0 \\ 0 & -1
    \end{bmatrix}
\end{align*}

mit den jeweiligen Eigenvektoren
\begin{align*}
    |{+}x\rangle &= \frac{1}{\sqrt{2}}\begin{pmatrix} 1\\1 \end{pmatrix} &
    |{-}x\rangle &= \frac{1}{\sqrt{2}}\begin{pmatrix} 1\\-1 \end{pmatrix} &
    |{+}y\rangle &= \frac{1}{\sqrt{2}}\begin{pmatrix} 1\\i \end{pmatrix} &
    |{-}y\rangle &= \frac{1}{\sqrt{2}}\begin{pmatrix} i\\1 \end{pmatrix} &
    |{+}z\rangle &= \begin{pmatrix} 1\\0 \end{pmatrix} &
    |{-}z\rangle &= \begin{pmatrix} 0\\1 \end{pmatrix} &
\end{align*}
beschrieben.

Damit ist Spin der z-Achse eines Elektron-Positron Paares
\begin{equation}
    |\psi\rangle = \frac{1}{\sqrt{2}} \left( 
        |{+}z\rangle \otimes |{-}z\rangle - |{}z\rangle \otimes |{+}z\rangle
     \right).
\end{equation}

Mit der Messung des z-Spins bei Detektor $A$ wird eine Projektion auf
entweder $|{+}z\rangle$ oder $|{-}z\rangle$ gemacht, womit der Zustand
$|\psi\rangle$ auf
\[
    |{+}z\rangle \otimes |{-}z\rangle
\]
respektive
\[
    |{-}z\rangle \otimes |{+}z\rangle
\]
reduziert wird.
Das Ergebnis einer Messung $S_z$ bei Detektor $B$ ist damit sofort, 
ohne \"uberhaupt eine Messung durchf\"uhren zu m\"ussen, bekannt und wird
$-z$ resp. $+z$ ergeben.

%TODO: kommuntatoren einfügen

\subsection{Verborgene lokale Variablen}
Die verbreitete Erkl\"arung zur L\"osung des EPR Paradoxons war die Theorie
der verborgenen, lokalen Variablen. Damit sind generell Theorien gemeint,
welche die Zuf\"alligkeiten der Quantenmechanik durch das Einf\"uhren von
zus\"atzlichen Variablen eliminieren. Da zu diesem Zeitpunkt keine solchen
gemessen werden konnten, wurden diese als \emph{verborgene}
Variablen bezeichnet. Weiter wird verlangt, dass diese Variablen sowohl
die Forderungen an Lokalit\"at wie auch Realismus erf\"ullen.

F\"ur das Beispiel des Elektron-Positron Paares ist eine 
einzelne 3-dimensionale Variable $\lambda$ ausreichend um jegliche
Kombinationen zwischen Spin $+$ und $-$ in allen 3 Dimensionen 
$x$, $y$ und $z$ abdecken zu k\"onnen.
Eine Visualisierung einer solchen Variable ist in
\figurename~\ref{fig:Bell:hidden_var} gezeigt. Dabei h\"angt der Zusammenhang
zwischen Spin in $x$, $y$ bzw. $z$ Richtung vom Quadranten ab, in welchem
die Variable $\lambda$ liegt. Der Wert von $\lambda$ wird hier in der Quelle
f\"ur beide Partikel gleich gesetzt und ist somit ab $t=T$ fix bestimmt.

\begin{figure}
    \centering
    \includegraphics{bell/images/hidden_vars.pdf}
    \caption{Visualisierung der verborgenen Variable $\lambda$ des Spin-Zustands}
    \label{fig:Bell:hidden_var}
\end{figure}


\section{Die Bell'sche Ungleichung}


\section{Experimente}
\subsection{Freedman Clauser}
\subsection{Alain Aspect}

\section{Zusammenfassung}


\printbibliography[heading=subbibliography]
\end{refsection}

