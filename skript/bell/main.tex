\chapter{Bellsche Ungleichung\label{chapter:bell}}
\lhead{Bellsche Ungleichung}
\begin{refsection}
\chapterauthor{Hannes Badertscher}

Im Mai 1935 ver\"offentlichten Albert Einstein, Boris Podolsky und
Nathan Rosen ein Gedankenexperiment mit welchem sie erkl\"arten, wieso
die Quantenmechanik nicht vollst\"andig sein kann. Als Alternative
sollten zus\"atzliche, verborgene, Variablen eingef\"ugt werden, welche
das zugrunde liegende Verhalten logisch erkl\"aren k\"onnen.

Bis 1971 war die Theorie der verborgenen lokalen Variablen nach Einstein
die verbreitete Ansicht -- die Frage war nur wann solche Variablen
entdeckt werden. In vielen Bereichen der Quantenmechanik wurden tatsächlich
kleinere Teilchen, Quantenzahlen und Variablen entdeckt, welche das
Verständnis der heutigen Quantenphysik prägten.
%TODO

\section{Lokalit\"at und Realismus\label{section:bell:lokalitaet}}
Zuerst stellt sich die Frage der Anforderungen an die Theorie der
Quantenmechanik. Eine zentrale Eigenschaft ist hierbei die Lokalit\"at.
Lokalit\"at bedeutet hierbei, dass jegliche Vorg\"ange nur einen Einfluss
auf ihre direkte Umgebung haben k\"onnen.

Die klassische Newtonsche Mechanik ist eine \emph{nicht}-lokale 
Theorie. Eine Masse hat in der Gravitation sofort, ohne zeitliche 
Verz"ogerung einen Einfluss auf andere Massen, unabh\"angig von der
r\"aumlichen Distanz zwischen den beiden Massen. 
In der speziellen Relativit\"atstheorie 
\textsuperscript{[Citation needed]}
wurden die Begriffe von Raum und Zeit so definiert, dass sich alle
Materie und Energie sich h\"ochstens mit Lichtgeschwindigkeit fortbewegen
kann. 
Die allgemeine Relativit\"atstheorie ist eine alternative Formulierung
der Newtonschen Gravitationstheorie, welche die in der speziellen
Relativit\"atstheorie geforderte Lokalit\"at erf\"ullt.
Von der Quantenmechanik kann ebenfalls erwartet werden, dass sich
jegliche Ereignisse h\"ochstens mit Lichtgeschwindigkeit ausbreiten
k\"onnen.

Eine Theorie wird als \emph{realistisch} bezeichnet, wenn die Theorie
nicht direkt von der Messung abh\"angt. Eine Messung hat also einen
vordefinierten Wert, auch wenn dieser nicht bekannt ist.
Die bekannte Frage ''Is the moon there if nobody looks?{``} zeigt auf,
dass der Realismus rein intuitiv immer erf\"ullt sein m\"usste. Schliesslich
zweifelt niemand daran, dass der Mond auch existiert wenn gerade niemand
zum Mond hinaufschaut.
\printbibliography[heading=subbibliography]
\end{refsection}

