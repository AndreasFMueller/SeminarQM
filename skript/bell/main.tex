% !TeX spellcheck = de_CH
\chapter{Bellsche Ungleichung\label{chapter:bell}}
\lhead{Bellsche Ungleichung}
\begin{refsection}
\chapterauthor{Hannes Badertscher}

In diesem Kapitel betrachten wir ein Theorem von Albert Einstein genauer.
Wie Randall Munroe von xkcd im
Comic~\ref{fig:bell:xkcd_einstein} illustriert konnten die Theorien von
Einstein bisher selten widerlegt werden.
Als einziges Beispiel Einstein zu widerlegen wird eine m\"oglicherweise falsche
Entscheidung Einsteins, als er bei seiner Arbeit im Patentamt ein umstrittenes
Patent zur Annahme empfahl, aufgef\"uhrt.
Obwohl scheinbar unm\"oglich, befassen wir uns in diesem Kapitel mit der Arbeit
von Albert Einstein und versuchen diese kritisch zu beleuchten.

\begin{figure}[b]
    \centering
    \includegraphics[width=0.5\linewidth]{bell/images/xkcd_einstein.png}
    \caption{XKCD Comic, welches darauf anspielt, dass sich kaum Fehler 
    in den Arbeiten von Albert Einstein finden lassen \cite{Bell:XkcdEinstein}}
    \label{fig:bell:xkcd_einstein}
\end{figure}


Im Mai 1935 ver\"offentlichten Albert Einstein, Boris Podolsky und
Nathan Rosen ein Gedankenexperiment mit welchem sie erkl\"arten, wieso
die Quantenmechanik nicht vollst\"andig sein kann. Als Alternative
sollten zus\"atzliche, verborgene, Variablen eingef\"ugt werden, welche
das zugrunde liegende Verhalten logisch erkl\"aren k\"onnen.
Die Theorie der verborgenen lokalen Variablen nach Einstein war lange
die verbreitete Ansicht -- die Frage war nur wann solche Variablen
entdeckt werden. In vielen Bereichen der Quantenmechanik wurden tats\"achlich
kleinere Teilchen, Quantenzahlen und Variablen entdeckt, welche das
Verst\"andnis der heutigen Quantenphysik pr\"agten.

Wir betrachten das Gedankenexperiment von Einstein, indem wir zuerst die
grundlegenden Annahmen der Lokalit\"at und des Realismus betrachten. Danach
gehen wir auf das von Einstein, Rosen und Podolski ver\"offentlichte Paradoxon
ein und kommen auf die darauf basierende Bell'sche Ungleichung, welche
1961 von John Bell ver\"offentlicht wurde.

\section{Lokalit\"at und Realismus\label{section:bell:lokalitaet}}
Zuerst stellt sich die Frage der Anforderungen an die Theorie der
Quantenmechanik. Eine zentrale Eigenschaft ist hierbei die Lokalit\"at.
Lokalit\"at bedeutet hierbei, dass jegliche Vorg\"ange nur einen Einfluss
auf ihre direkte Umgebung haben k\"onnen.

Die klassische Newton'sche Mechanik ist eine \emph{nicht}-lokale 
Theorie. Eine Masse hat in der Gravitation sofort, ohne zeitliche 
Verz"ogerung einen Einfluss auf andere Massen, unabh\"angig von der
r\"aumlichen Distanz zwischen den beiden Massen. 
In der speziellen Relativit\"atstheorie 
\textsuperscript{[Citation needed]}
wurden die Begriffe von Raum und Zeit so definiert, dass sich alle
Materie und Energie h\"ochstens mit Lichtgeschwindigkeit fortbewegen
kann. 
Die allgemeine Relativit\"atstheorie ist eine alternative Formulierung
der Newton'schen Gravitationstheorie, welche die in der speziellen
Relativit\"atstheorie geforderte Lokalit\"at erf\"ullt.
Von der Quantenmechanik kann ebenfalls erwartet werden, dass sich
jegliche Ereignisse h\"ochstens mit Lichtgeschwindigkeit ausbreiten
k\"onnen.

Eine Theorie wird als \emph{realistisch} bezeichnet, wenn die Theorie
nicht direkt von der Messung abh\"angt. Eine Messung hat also einen
vordefinierten Wert, auch wenn dieser nicht bekannt ist.
Die bekannte Frage \enquote{Is the moon there if nobody looks?} zeigt auf,
dass der Realismus rein intuitiv immer erf\"ullt sein m\"usste. Schliesslich
zweifelt niemand daran, dass der Mond auch existiert wenn gerade niemand
zum Mond hinaufschaut.

\section{Das Einstein-Podolsky-Rosen Paradoxon\label{section:bell:epr}}
Das Einstein-Podolsky-Rosen Paradoxon (EPR Paradoxon) ist ein 
Gedankenexperiment, welches 1935 im Paper \cite{Bell:Einstein1935}
\emph{Can quantum-mechanical description of physical reality be considered complete}
von Albert Einstein, Boris Podolsky und Nathan Rosen publiziert wurde.
Wie der Titel bereits verr\"at, gehen Einstein \emph{et al.} dabei auf
zwei Fragen ein: 
(1) \enquote{Ist die Theorie korrekt?}
und 
(2) \enquote{Ist die Beschreibung durch die Theorie vollst\"andig?}
Nur wenn beide diese Fragen mit \enquote{ja} beantwortet werden k\"onnen, kann
die Theorie die Realit\"at zufriedenstellend beschreiben.
Die erste Frage kann einfach gepr\"uft werden, indem die Resultate von
Experimenten mit den Vorhersagen der Theorie verglichen werden. 
Die Quantenmechanik konnte dabei in zahlreichen Experimenten die Wirklichkeit
korrekt vorhersagen. 
Einstein, Rosen und Podolski befassten sich deshalb mehr mit der zweiten
Frage. 
Die Vollst\"andigkeit einer Theorie wird dabei wie folgt definiert:

\begin{definition}\label{def:Bell:Vollstaendigkeit}
    Jedes Element der physikalischen Realit\"at muss durch die Theorie
    genau beschrieben werden k\"onnen.
\end{definition}

W\"ahrend Einstein \emph{et al.} in ihrem Paper das Gedankenexperiment sehr
allgemein halten, betrachten wir hier ein anschaulicheres Beispiel, welches
1957 von David Bohm und Yakir Aharonov \cite{Bell:Bohm1957} pr\"asentiert wurde.

\subsection{Das Gedankenexperiment\label{subsection:bell:epr:experiment}}
Wir betrachten zwei quantenmechanische Systeme, $I$ und $II$,
welche von $t=0$ bis zu einem Zeitpunkt $t=T$ miteinander 
interagieren k\"onnen. Darauf ist keine Interaktion mehr m\"oglich.
Visualisiert wird das Gedankenexperiment durch eine Quelle, welche
Elektron-Positron Paare emittiert. 
Dabei wird das Elektron (System $I$) zu einem Detektoren $A$ und das 
Positron (System $II$) zu einem Detektoren $B$ gesandt.
Diese sind physikalisch getrennt, sodass kein Informationsaustausch
zwischen den Detektoren m\"oglich ist.
Der Messaufbau ist in \figurename~\ref{fig:bell:EPR_Messaufbau} abgebildet.

\begin{figure}
    \centering
    \includegraphics{bell/images/experiment_setup.pdf}
    \caption{Messaufbau im Gedankenexperiment von EPR}
    \label{fig:bell:EPR_Messaufbau}
\end{figure}

Diese zwei Partikel werden so generiert, dass sie einen komplement\"aren
Spin-Zustand besitzen. 
Wenn also bei Detektor $A$ der Spin des Elektrons entlang der $z$-Achse
gemessen wird und das Messresultat $+z$ ist, so ist sofort klar, dass
das das Positron bei Detektor $B$ den Spin $-z$ besitzen muss. 
Ebenso wird, falls bei Detektor $A$ der Spin entlang der $x$-Achse als $+x$
gemessen wird, das Positron bei Detektor $B$ den Spin $-x$ besitzen.

Wie im Kapitel der Heisenbergschen Unsch\"arferelation
\textsuperscript{[Citation needed]}
erkl\"art wird, sind die Messungen des Spins in verschiedenen Dimensionen 
komplement\"are Operatoren.
Damit ist es unm\"oglich, gleichzeitig beide Eigenschaften eines Teilchens
beliebig genau zu kennen.
Wird nun jedoch bei Detektor $A$ der Spin des Elektrons in $z$-Richtung 
als $+z$ gemessen, so ist bekannt dass das Positron den Spin $-z$ besitzt.
Da jedoch noch keine Messung am Positron ausgef\"uhrt wurde, kann bei
Detektor $B$ der Spin in $x$-Richtung gemessen werden.
Es ist also m\"oglich f\"ur das Positron gleichzeitig die exakten Werte 
des Spins in $x$-, wie in $z$-Richtung zu kennen. 
Mit der uns bekannten Beschreibung der Quantenmechanik mit Wellenfunktionen
kann dieser Effekt \emph{nicht} beschrieben werden, womit wir zu der
logischen Schlussfolgerung Einsteins kommen:

\begin{satz}
    Die quantenmechanische Beschreibung der physikalischen Realit\"at mittels
    Wellenfunktionen ist keine vollst\"andige Theorie.
\end{satz}

\subsection{Mathematische Herleitung\label{subsection:bell:epr:herleitung}}
Anhand des Spins kann das EPR Paradoxon einfach mathematisch formuliert werden.
Dabei wird die Notation aus Kapitel~\ref{chapter:spin} (\nameref{chapter:spin})
verwendet.

Wie in den Gleichungen~\ref{spin:paulimatrizen}--\ref{spin:vektoroperator}
hergeleitet wird der Spin-Operator durch die hermiteschen Pauli-Matrizen
beschrieben.

\begin{align*}
    S_x &= \frac{\hbar}{2} \begin{pmatrix}
    0 & 1 \\ 1 & 0
    \end{pmatrix}
    &
    S_y &= \frac{\hbar}{2} \begin{pmatrix}
    0 & -i \\ i & 0
    \end{pmatrix}
    &
    S_z &= \frac{\hbar}{2} \begin{pmatrix}
    1 & 0 \\ 0 & -1
    \end{pmatrix}
\end{align*}

Die zugeh\"origen Eigenvektoren wurden in einer \"Ubung hergeleitet und
betragen:
\begin{align*}
    |{+}x\rangle &= \frac{1}{\sqrt{2}}\begin{pmatrix} 1\\1 \end{pmatrix} &
    |{-}x\rangle &= \frac{1}{\sqrt{2}}\begin{pmatrix} 1\\-1 \end{pmatrix} &
    |{+}y\rangle &= \frac{1}{\sqrt{2}}\begin{pmatrix} 1\\i \end{pmatrix} &
    |{-}y\rangle &= \frac{1}{\sqrt{2}}\begin{pmatrix} i\\1 \end{pmatrix} &
    |{+}z\rangle &= \begin{pmatrix} 1\\0 \end{pmatrix} &
    |{-}z\rangle &= \begin{pmatrix} 0\\1 \end{pmatrix} &
\end{align*}

Der Spin-Zustand $|\psi\rangle$ entlang der $z$-Achse eines 
Elektron-Positron Paares  kann durch alle m\"oglichen Kombination der beiden
Spins von Elektron und Positron beschrieben werden. 
Diese umfassen $+z$ f\"ur das Elektron und $-z$
f\"ur das Positron, sowie $-z$ f\"ur das Elektron und $+z$ f\"ur das Positron:
\begin{equation}
    |\psi\rangle = \frac{1}{\sqrt{2}} \Big( 
        |{+}z\rangle \otimes |{-}z\rangle - |{-}z\rangle \otimes |{+}z\rangle
     \Big)
\end{equation}
Damit gilt
\begin{align}
    |\psi\rangle &= \frac{1}{\sqrt{2}} 
    \left( 
        \begin{pmatrix} 1\\0 \end{pmatrix} 
        \otimes 
        \begin{pmatrix} 0\\1 \end{pmatrix}
        -
        \begin{pmatrix} 0\\1 \end{pmatrix}
        \otimes
        \begin{pmatrix} 1\\0 \end{pmatrix}
     \right)
     =
     \frac{1}{\sqrt{2}}\left(
         \begin{pmatrix} 0 \\ 1 \\ 0 \\ 0 \end{pmatrix}
         -
         \begin{pmatrix} 0 \\ 0 \\ 1 \\ 0 \end{pmatrix}
     \right)
     \stackrel{!}{=}
     \frac{1}{\sqrt{2}}\left(
         \frac{1}{2}
         \begin{pmatrix} \phantom{-}1 \\ \phantom{-}1 \\ -1 \\ -1 \end{pmatrix}
         -
         \frac{1}{2}
         \begin{pmatrix} \phantom{-}1 \\ -1 \\ \phantom{-}1 \\ -1 \end{pmatrix}
     \right)  \notag \\
     &=
    -\frac{1}{\sqrt{2}} \left( 
        \frac{1}{\sqrt{2}}\begin{pmatrix} 1 \\ 1 \end{pmatrix} 
        \otimes 
        \frac{1}{\sqrt{2}}\begin{pmatrix} \phantom{-}1 \\ -1 \end{pmatrix}
        -
        \frac{1}{\sqrt{2}}\begin{pmatrix} \phantom{-}1 \\ -1 \end{pmatrix}
        \otimes
        \frac{1}{\sqrt{2}}\begin{pmatrix} 1 \\ 1 \end{pmatrix} 
     \right) \notag  \\
      &= 
      -\frac{1}{\sqrt{2}} \Big( 
              |{+}x\rangle \otimes |{-}x\rangle - |{-}x\rangle \otimes |{+}x\rangle
           \Big)
\end{align}

Wir haben also gezeigt, dass mit der Spin-Zustand $|\psi\rangle$ des
Elektron-Positron Paares \"aquivalent in Abh\"angigkeit des $z$-, sowie
des $x$-Spin Zustands geschrieben werden kann.
Wird nun bei Detektor $A$ der Spin des Elektrons in $z$-Richtung gemessen, 
so findet eine Projektion des Zustands $|\psi\rangle$ auf entweder
$|{+}z\rangle$ oder $|{-}z\rangle$ statt.
Der Zustand reduziert sich auf
\begin{align*}
    |\psi_{1}\rangle &= |{+}z\rangle \otimes |{-}z\rangle
    & \text{bzw.} && 
    |\psi_{2}\rangle &= |{-}z\rangle \otimes |{+}z\rangle
\end{align*}
Damit ist der Spin in $z$-Richtung f\"ur das Positron bei Detektor $B$ ohne
eine Messung vorbestimmt, n\"amlich als $-z$ im ersten Fall bzw. $+z$ im
zweiten Fall.
Wenn nun bei Detektor $B$ der Spin des Positrons in $x$-Richtung gemessen
wird, so wird der urspr\"ungliche Spin-Zustand $|\psi\rangle$
ebenfalls reduziert auf
\begin{align*}
    |\psi_{a}\rangle &= |{+}x\rangle \otimes |{-}x\rangle
    & \text{bzw.} && 
    |\psi_{b}\rangle &= |{-}x\rangle \otimes |{+}x\rangle
\end{align*}
Wir erhalten damit zwei Wellenfunktionen, z.B. $|\psi_{1}\rangle$ und 
$|\psi_{a}\rangle$, welche \emph{gleichzeitig} die gleiche 
physikalische Realit\"at beschreiben.
Um zu pr\"ufen ob dies im Fall des Spin-Zustands \"uberhaupt m\"oglich 
ist, wenden wir Hilfssatz~\ref{skript:kommutatorannihliertev} an.
Der Kommutator $[S_x,S_z]$ ist
\begin{equation}
    [S_x, S_z] =  -i \hbar S_y \neq 0
\end{equation}
Da die Operatoren $S_x$ und $S_z$ also nicht vertauschen, ist nach der Theorie
der Quantenmechanik die gleichzeitige Kenntnis der beiden Messungen nicht
m\"oglich.
Dass dies jedoch hier der Fall ist, haben wir sowohl intuitiv in
Abschnitt~\ref{subsection:bell:epr:experiment}, wie auch mathematisch in
diesem Abschnitt bewiesen. 
Damit m\"ussen wir zur Schlussfolgerung kommen, dass entweder 
(1) eine sofortige Kommunikation zwischen dem Elektron und dem Positron
stattfindet, was  jedoch durch die Lokalit\"at ausgeschlossen wird,
oder (2) dass die Beschreibung der Quantenmechanik mittels der Wellenfunktion
nicht  vollst\"andig ist und ein fundamentaler, nicht messbarer, Mechanismus
bestimmt welches Ergebnis eine Messung des Spins ergeben wird.


\subsection{Die Theorie der verborgenen lokalen Variablen}
Besinnen wir uns zur\"uck auf die beiden initialen Fragen: 
(1) \enquote{Ist die Theorie korrekt?} und 
(2) \enquote{Ist die Beschreibung durch die Theorie vollst\"andig?}
Dabei mussten wir feststellen, dass zwar die Frage (1) dank diverser
Experimente mit \enquote{ja} beantwortet werden kann, doch die Frage (2) 
durch das besprochene Gedankenexperiment nur mit \enquote{nein} beantwortet
werden kann. 
Damit stellt sich die Frage, um welches Element die Theorie der Quantenmechanik
erweitert werden muss, sodass die Beschreibung vollst\"andig wird.

Ein Ansatz, welcher dabei auf der Hand liegt, wurde in
Abschnitt~\ref{subsection:bell:epr:herleitung} bereits angeschnitten: 
Es muss ein verborgener Mechanismus existieren, welcher das Resultat unseres
Experiments, also den Spin von Elektron und Positron, vorausbestimmt.
Dazu wird eine verborgene Variable $\lambda$ eingef\"uhrt.
$\lambda$ wird als \emph{verborgene} Variable bezeichnet, da diese nicht
messbar ist, sondern nur die Auswirkungen der Variable, also der jeweilige
Spin, gemessen werden kann.
Um das durch Einstein, Rosen und Podolski eingef\"uhrte Paradoxon zu l\"osen
reicht eine 2-dimensionale Variable aus, welche den Zusammenhang zwischen
dem Spin in $x$- und $z$-Richtung beschreibt.
Eine Visualisierung der m\"oglichen Werte von $\lambda$ ist in 
Abbildung~\ref{fig:bell:hidden_var} dargestellt.

\begin{figure}
    \centering
    \includegraphics{bell/images/hidden_vars.pdf}
    \caption{Visualisierung der verborgenen Variable $\lambda$ des Spin-Zustands}
    \label{fig:bell:hidden_var}
\end{figure}

Liegt die Variable $\lambda$ des Positrons im ersten Quadranten des Diagramms,
so ist unabh\"angig von der Messung des Elektrons bei Detektor $A$ eindeutig
bestimmt, dass der Spin in $x$- und in $z$-Richtung positiv sein wird. 
$\lambda$ ist jedoch nicht direkt messbar, es muss also trotzdem der Spin
in $z$-Richtung bei Detektor $A$ und in $x$-Richtung bei Detektor $B$ gemessen
werden, um den Wert von $\lambda$ bestimmen zu k\"onnen.
Da der Wert von $\lambda$ in der Quelle, beim Generieren des Elektron-Positron
Paares gesetzt wird und danach nicht ver\"andert wird, wird sich $\lambda$
an die Forderung der Lokalit\"at halten.
\enquote{Spukhafte Fernwirkungen}, wie Einstein die Beobachtungen seines
Gedankenexperiments nannte, werden damit ausgeschlossen und die Theorie
der Quantenmechanik wird so erweitert, dass sie \emph{vollst\"andig} ist.
Die Herausforderung eine solche lokale, verborgene Variable $\lambda$ wirklich 
in Form eines existierenden Teilchens zu finden bleibt jedoch bestehen.

\section{Die Bell'sche Ungleichung\label{section:bell:bell}}
%http://arxiv.org/pdf/quant-ph/0509061.pdf
Nach der Ver\"offentlichung des EPR Paradoxons gingen verschiedene bekannte
Physiker auf das Paradoxon ein.
Der gr\"osste Verfechter der Quantenmechanik, Niels Bohr, antwortete bereits
1935 mit einer eher schwammigen Widerlegung Einsteins, welche weitere 
Anh\"anger der Quantenmechanik zu jener Zeit bereits zufriedenstellte
\textsuperscript{[Citation needed]}.
Im Nachhinein betrachtet musste 1949 sogar Bohr selbst zugeben, dass seine
Reaktionen auf Einstein \enquote{ineffizient} waren und 
\enquote{es schwierig machten darauf zu antworten}. 
Erwin Schr\"odinger hingegen zweifelte nicht am EPR Paradoxon selbst, sondern
daran, dass sich die Erkenntnisse der Quantenmechanik auf die makroskopische
Welt \"ubertragen lassen, und somit ob der Effekt auch messbar sei.
Dies \"ausserte er insbesondere im ber\"uhmten Paradoxon von 
Schr\"odingers Katze, welches in Abbildung~\ref{fig:bell:xkcd_schroedinger}
verdeutlicht wird \textsuperscript{[Citation needed]}.

\begin{figure}
    \centering
    \includegraphics[width=0.8\linewidth]{bell/images/xkcd_schroedinger.jpg}
    \caption{xkcd Comic zum Paradoxon Schr\"odingers Katze, welches daran
    zweifelt, dass sich Quantenmechanische Effekte auf die makroskopische
    Welt \"ubertragen lassen. \cite{Bell:XkcdSchroedinger}}
    \label{fig:bell:xkcd_schroedinger}
\end{figure}

David Bohm hingegen versuchte das Gedankenexperiment Einsteins zu verteidigen
\textsuperscript{[Citation needed]}.
Er entwickelte dazu eine dem EPR-Experiment sehr \"ahnliche Idee, indem er
nicht Position und Drehmoment, sondern den Spin zweier Partikel betrachtete.
Dieses Gedankenexperiment ist einfacher zu analysieren (weshalb wir im
Abschnitt~\ref{section:bell:epr} das Experiment von Bohm, und nicht das
originale von Einstein betrachtet haben), sondern ist experimentell einfacher
zu pr\"ufen.


\section{Experimente}
\subsection{Freedman Clauser}
    
    \begin{figure}[ht!]
        \centering
        \includegraphics{bell/images/what_if.pdf}
    \end{figure}
    
\subsection{Alain Aspect}

\section{Zusammenfassung}


\printbibliography[heading=subbibliography]
\end{refsection}

