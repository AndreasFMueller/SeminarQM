\chapter{Bose-Einstein Kondensation\label{chapter:bose}}
\lhead{Bose-Einstein Kondensation}
\begin{refsection}
\chapterauthor{Reto Christen und Daniel Gubser}

\section{Einleitung}
1924 erkannte der indische Physiker Satyendranath Bose\index{Bose, Satyendranath}, dass es bei ganz tiefen Temperaturen einen noch unbekannten quantenmechanischen Aggregatszustand gibt, welcher spezielle Eigenschaften besitzt. Er teilte diese Erkenntnis mit Albert Einstein, \index{Einstein, Albert} welcher die Wichtigkeit dieses Postulates erkannte. Weitere Berechnungen von Albert Einstein ergaben, dass dieses sogenannte Bose-Einstein-Kondensat (BEC)\index{Bose-Einstein-Condensat, BEC} zwar m"oglich sei, dazu m"ussten allerdings Temperaturen knapp "uber dem absolutem Nullpunkt erreicht werden, welches dazumal als unm"oglich galt. Erst durch die wesentlichen Verbesserungen der Abk"uhlungstechniken konnte im Jahre 1995 das erste Bose-Einstein-Kondensat erzeugt werden. Daf"ur wurde den Physikern Wolfgang Ketterle\index{Ketterle, Wolfgang}, Eric Cornell\index{Cornell, Eric} und Carl Wieman\index{Wieman, Carl} im Jahre 2001 den Nobel Preis verliehen. 

\subsection{Eigenschaften}
Ein Bose-Einstein-Kondensat besteht aus Bosonen\index{Bosonen}, welche nahe auf den absoluten Nullpunkt heruntergek"uhlt werden. Dabei m"ussen Temperaturen unter $1~uK$ erreicht werden. Aktuelle Experimente am MIT \index{MIT} erreichen momentan Temperaturen von $177~nK$, was "uber einer Million Mal k"alter als im Weltall ist.
 
Bosonen haben die Eigenschaft, dass ihrer Spin $\pm 1$ betr"agt und somit immer ganzzahlig ist. Ein Spin von $\pm 1$ kann beispielsweise mit zwei wechselwirkenden Teilchen mit je einem Spin von $+1/2$ oder $-1/2$ erreicht werden. Solche Verbindungen sind unter anderem Cooper-Paare, welche im Kapitel \ref{chapter:supraleitung} genauer erl"autert werden. 

Eine weitere M"oglichkeit besteht darin, Elemente zu verwenden, welche bereits aus Bosonen bestehen. Das wichtigste Element in dieser Kategorie ist das Helium-4. Es besteht aus je zwei Protonen, Neutronen und Elektronen und ist somit komplett aus Bosonen aufgebaut. Am MIT werden aktuell Sodium-23 und Rubidium-87 verwendet. Beide Isotope bestehen aus Bosonen und besitzen somit einen Integer als Spin.

Werden diese Bosonen nun stark abgek"uhlt, bilden sie eine Einheit, sprich sie ''verschmieren'' zu einer gemeinsamer Wellenfunktion. Somit verhalten sich alle Teilchen im Bose-Einstein-Kondensat identisch und sind ununterscheidbar.

\subsection{Effekte}

Durch diese Koh"arenz\index{Koh"arenz} und Ununterscheidbarkeit der einzelnen Teile resultieren diverse Effekte wie die Suprafluidit"at\index{Suprafluidit"at} oder Supraleitung\index{Supraleitung}. Das Bose-Einstein-Kondensat ist f"ur diese Effekte zwar nicht die einzige Erkl"arung, hat aber einen wichtigen Einfluss darauf. 

Ebenfalls ergaben sich bei der Herstellung von Bose-Einstein-Kondensat weitere interessante Effekte, welche f"ur diverse Anwendungen genutzt werden k"onnen. Dies ist zum Beispiel die Erkl"arung, wie Neutronensterne\index{Neutronensterne} aufgebaut sind oder der Big Bang\index{Big Bang} entstand. Durch die absolute Koh"arenz k"onnen auch neue Laser\index{Laser} entwickelt werden, sogenannte Atom-Laser\index{Atom-Laser}. Diese Atom-Laser k"onnen im Quantencomputer\index{Quantencomputer} Bereich neue Fortschritte mit sich bringen sowie auch bei Quanten-Teleportation. Ein weiterer wichtiger Effekt f"ur Quantencomputer ist die Eigenschaft von Bose-Einstein-Kondensat, welche Licht auf extrem langsame Geschwindigkeiten abbremsen kann ohne dabei an Momentum zu verlieren. 

\section{Grundlagen}

Um ein Bose-Einstein-Kondensat zu erzeugen, sind Temperaturen nah dem Nullpunkt n"otig. Ausserdem k"onnen nur bestimmte Teilchen ein solches Kondensat bilden.

\subsection{Bosonen}

Die wichtigsten Eigenschaften von Bosonen nochmals kurz zusammengefasst.

\begin{itemize}
    \item	Bosonen sind die Elementarteilchen, welche gew"ohnlich f"ur die
            "Ubertragung von Kr"aften zust"andig sind.	 
    \item	Alle Bosonen haben einen ganzzahligen Spin. Sie unterliegen
            somit nicht dem Pauli-Prinzip.
    \item	Dies bedeutet, dass viele Bosonen denselben
            quantenmechanische Zustand besiedeln k"onnen.
\end{itemize}

\subsection{Abk"uhlung}

Um die tiefen Temperaturen erreichen zu k"onnen, werden zuerst mit Lasern die Atome abgek"uhlt und anschliessend mittels evaporativer K"uhlung die Zieltemperatur erreicht. 

Im Bild \ref{fig:LaserKuehlung} ist schematisch der Aufbau der Laserk"uhlung\index{Laserk"uhlung} dargestellt.
Falls sich ein Atom bewegt, wird es mit dem Laser beschossen. Sobald Atom und Photon aufeinander treffen, wird Impuls ausgetauscht und dadurch das Atom abgebremst.
Da die Bewegung eines Atoms ein Mass f"ur seine Energie oder auch Temperatur ist, k"onnen mithilfe der Laser die Atome bis zu einigen $100~\mu K$  abgek"uhlt werden. \cite{bose:LaserKuehlung}

\begin{figure}
	\centering
	\includegraphics[width = 0.5\textwidth]{./bose/laserkuehlung.png} 
	\caption[Laserk"uhlung]{Laserk"uhlung \cite{bose:WikiLaserKuehl}}
	\label{fig:LaserKuehlung}
\end{figure}

Um die Atome nun noch weiter abk"uhlen zu k"onnen, wird die evaporative K"uhlung\index{Evaporative K"uhlung} angewendet.

Dieses Prinzip kann sehr gut mit einer heissen Tasse Kaffee veranschaulicht werden, wie dies in Abbildung \ref{fig:EvaporativeKuehlung} dargestellt ist.

Die schnellen Atome haben eine hohe Temperatur und k"onnen beim ''Anhauchen'' der Tasse diejenige verlassen. Die langsameren und k"alteren Atome bleiben in der Tasse zur"uck.

Umgesetzt wird die Tasse durch Magnetfelder, somit entsteht eine Falle in der sich die Atome befinden. Die st"arke des Magnetfeldes bestimmt dabei welche Energien eingeschlossen werden k"onnen, langsame Atome bleiben in der Falle. Durch sukzessive Abschw"achung des Magnetfeldes, im Falle der Tasse eine Verkleinerung des Randes, verlassen immer mehr Teile das Magnetfeld und die gesamt Temperatur wird dadurch geringer. Schlussendlich beinhaltet das Magnetfeld nur noch wenige Teilchen, welche sich sehr nahe am absoluten Nullpunkt befinden. In aktuellen Experimenten sind dies Temperaturen um $177~nK$. \cite{bose:WikiEvaporativeKuehlung}

\begin{figure}
	\centering
	\includegraphics[width = 0.5\textwidth]{./bose/evaporation.png}
	\caption{Evaporative K"uhlung mit einer Tasse dargestellt.}
	\label{fig:EvaporativeKuehlung}
\end{figure}

\section{Mathematischer Hintergrund}

Grunds"atzlich werden in diesem Teil die gleichen Berechnungen wie im Buch von Richard P. Feynman\index{Feynman, Richard P.} \cite{bose:feynman} verwendet. F"ur die bessere Verst"andlichkeit werden die Formel, im Gegensatz zum Buch, etwas genauer erl"autert.

Als Ausgangslage dient ein geschlossenes System aus Teilchen, welches mit einem Teilchenreservoir verbunden ist. 
Zwischen dem System und dem Reservoir k"onnen Teilchen ausgetauscht werden. Dies ist in Abbildung \ref{fig:reservoir} dargestellt.

\begin{figure}
	\centering
	\includegraphics[width = 0.5\textwidth]{./bose/reservoir.png}
	\caption{Reservoir}
	\label{fig:reservoir}
\end{figure}

Die Energie eines Teilchens berechnet sich nach der Formel:

\begin{equation}
   E_n = \frac{p^2}{2m}
\end{equation}

Die Wahrscheinlichkeit, mit der diese Energie im System zu finden ist, l"asst sich mit der folgender Formel berechnen.

\begin{equation}
    P(E_n) = \frac{E_n}{\sum\limits_{n = 1}^{N} e^{-E_n/kT}}    
\end{equation}

Der Nenner stellt dabei die Verteilung der Energien dar.

\begin{equation}
    Q = \sum\limits_{n = 1}^{N} e^{-E_n/kT}
\end{equation}

Die mittlere Energie des Systems betr"agt:

\begin{equation}
    \langle U \rangle = \sum_{n} E_n \cdot P(E_n) = - \frac{d}{d \beta} \ln (Q)
\end{equation}

Nachfolgend wird Schritt f"ur Schritt gezeigt, wie diese Formel hergeleitet wird.

\begin{equation}
Q = \sum\limits_{i = 0}^{N} e^{- \beta \epsilon_i n_i } 
=  \sum_{i = 0}^{N}  \textcolor{red}{ e^{- \beta \epsilon_i n_i }} \textcolor{blue} {e^{- \beta \epsilon_1 n_1 }} e^{- \beta \epsilon_2 n_2 } ... = \textcolor{red} {(1+e^{- \beta \epsilon_1 }+e^{- \beta \epsilon_2*2  }+e^{- \beta \epsilon_3*3  }+...)} \cdot \textcolor{blue} {(1+e^{- \beta \epsilon_1 }+e^{- \beta \epsilon_1*2  }+e^{- \beta \epsilon_1*3  }+...)} \cdot ...
\end{equation}

\begin{equation}
\sum_{i} e^{- \beta \epsilon_i n_i } = (1+e^{- \beta \epsilon_0 }+e^{- \beta \epsilon_0*2  }+e^{- \beta \epsilon_0*3  }+...) = \prod_{a} \sum_{n_i} e^{- \beta \epsilon_a n_i }
\end{equation}

Die Summe stellt die geometrische Reihe dar, dadurch kann die ganze Summe durch einen simplen Ausdruck ersetzt werden.

\begin{equation}
\sum_{n_i} (e^{- \beta \epsilon_i })^{n_i} = \sum_{i} q^i = \frac {1}{1-q}
\end{equation}

\begin{equation}
    \prod_{a} \sum_{n_i} e^{- \beta \epsilon_a n_i } = \prod_{a} \frac {1}{1-e^{- \beta \epsilon_a}}
\end{equation}

Das Produkt kann durch Anwendung der Logarithmen Regeln in eine Summe umgeformt werden.

\begin{equation}
\ln \left[ \left( \frac{1}{1-e^{\beta \epsilon_{a_0}}} \right) \cdot \left( \frac{1}{1-e^{ \beta \epsilon_{a_1}}} \right) \cdot ...\right] = \ln \left( \frac{1}{1-e^{\beta \epsilon_{a_0}}} \right) + \ln \left( \frac{1}{1-e^{\beta \epsilon_{a_1}}} \right) + ...
\end{equation}

Dies f"uhrt schliesslich zur Formel:

\begin{equation}
\prod_{a} \frac {1}{1-e^{- \beta \epsilon_a}} = \sum_{a} \ln \left( 1-e^{- \beta \epsilon_a} \right)
\end{equation}

Damit Teilchen zwischen dem System und dem Reservoir ausgetauscht werden k"onnen, muss Energie aufgewendet werden.
Dies wird mit einem zus"atzlichen Term abgegolten. Damit wird $Q$ abh"angig von dieser Energie $\mu$.


 \begin{equation}
 Q^{(\mu)} = \sum e^{- \beta \sum_{a} (\epsilon_a - \mu)n_a} = e^{-\beta g}
 \end{equation}
 
 mit
 
  \begin{equation}
  -\frac{1}{ \beta } \ln(Q^{\mu}) = g
  \end{equation}

Durch Ableiten der Gleichung:

\color{green}TODO!\color{black}

Wird die zu Beginn der Herleitung vorgestellte Formel erreicht.


Die Berechnung wird durch folgende Annahme vereinfacht.

\begin{equation}
\frac{1}{\beta} \sum_{a} \ln(1-e^{-\beta (\epsilon_a -\mu)}) \approx \frac{1}{\beta} \int \ln(1-e^{-\beta (\epsilon_a-\mu)})
\end{equation}

Damit Berechnet sich die Gesamtenergie des Systems nach.

\begin{equation}
U = s \int  \frac{ \left(e^{ {\frac{- \beta p^2}{2m}} } e^{\beta \mu} \right) \cdot {\frac{p^2}{2m}} }{1-e^{ {\frac{- \beta p^2}{2m}} } e^{\beta \mu}} \cdot \frac{d^3 p}{(2 \pi \hbar)^3}
\end{equation}

Die Vereinfachung ist bis zu einer kritischen Temperatur zul"assig. Ab dieser Temperatur d"urfen die kleinsten Energien des Systems nicht mehr vernachl"assigt werden.

\begin{equation}
\frac{1}{\beta} \sum_{a} \ln(1-e^{-\beta (\epsilon_a-\mu)}) \neq \frac{1}{\beta} \int \ln(1-e^{-\beta (\epsilon_a-\mu)})
\end{equation}

\section{Anwendungen}
\subsection{Erkl"arung quantenphysikalischen Effekten}
Ein Bose-Einstein-Kondensat kann zur Erkl"arung des Big Bang beigezogen werden. Das sehr kalte Kondensat verh"alt sich "ahnlich wie sehr heisse Atome unter grossem Druck. In beiden F"allen ''verschmieren'' die Wellen zu einer einzigen Wellenfunktion. Die Selbe Erklärung kann auch für Neutronensterne benutzt werden. \cite{bose:MITvideo}

Durch das Bose-Einstein-Kondensat k"onnen auch die Suprafluidit"at sowie die Supraleitung besser verstanden werden. Es ist zwar nicht der einzige Effekt der mitwirkt aber ein wichtiger. 

Supraleiter k"onnen in n"aher Zukunft eine wichtige Rolle in der Energieverteilung "ubernehmen. Kurze Leitungen, welche mit grosser Leistung belastet werden, stehen dabei im Vordergrund. Da ein Supraleiter weder einen Widerstand noch einen Skin-Effekt hat, k"onnen bei relativ kleinem Querschnitt grosse Str"ome "ubertragen werden. 
Aktuell ist eine $1~km$ lange Leitung in Essen in Betrieb genommen. Dies dient zwar als Versuchsaufbau, die Betreiberinnen erhoffen sich dabei aber wichtige Erkenntnisse, wie auch eine zuverl"assige Energieversorgung. \cite{bose:SupraVerteilnetze}

\subsection{Atom-Laser}
Ein herk"ommlicher Laser nutzt die Koh"arenz von Photonen aus und erzeugt somit eine starke Lichtquelle. Werden nun aber Atome anstatt Photonen verwendet, liesse sich somit eine viel st"arkere Lichtquelle herstellen. Atome sind grunds"atzlich nicht koh"arent zu einander. Wird aber ein Bose-Einstein-Kondensat verwendet, in welchem alle Teile koh"arent zueinander sind, liesse sich ein Atom-Laser tats"achlich herstellen.

Mit einem Atom-Laser erg"aben sich grosse Vortschritte in der quantenmechanischer Teleportation und somit auch in Quantencomputern.

Zur aktueller Zeit wird dazu noch Grundlagenforschung betrieben, eine tats"achlich funktionierende Entwicklung ist noch nicht in Sicht. 

\subsection{Abbremsung von Photonen}

Ein weiterer Effekt, welcher f"ur die Entwicklung von Quantencomputern n"utzlich sein kann, ist die Abbremsung von Licht. Bis anhin konnte Licht nur direkt mit einem Hindernis abgebremst werden, dabei ging aber immer Momentum, sprich im Falle eines Quantencomputer Information, verloren. In Versuchen mit einem Bose-Einstein-Kondensat zeigte sich, dass Licht effektiv abgebremst werden kann, ohne dass Informationen verloren gehen. Dabei entspricht die Eintrittsgeschwindigkeit sowie auch die Austrittsgeschwindigkeit der Lichtgeschwindigkeit im aktuellen Medium (ca.~$300000~km/s$ oder $3 \cdot 10^9~ m/s$). Im Bose-Einstein-Kondensat selber kann die Geschwindigkeit auf $17~m/s$ reduziert werden, was "uber hundert Millionen Mal langsamer ist. \cite{bose:SlowLight}

\section{Schlusswort}

\printbibliography[heading=subbibliography]
\end{refsection}



