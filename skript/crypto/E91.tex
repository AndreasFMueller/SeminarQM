\section{E91}
  \index{E91}
  \rhead{E91}
  \subsection{Das E91 Protokoll}
  Das E91 Protokoll wurde 1991 von Artur Ekert entworfen.
  Dabei werden verschr\"ankte Photonen verwendet, welche von Alice, Bob oder sogar Eve erzeugt werden k\"onnen.

  \subsection{Der Schl\"usselaustausch mit E91}
  Das Protokoll beruht auf der Nutzung der Eigenschaften von Verschr\"ankten Quanten \cite{qc:verschraenkung}.
  Die beiden ausschlaggebenden Eigenschaften sind dabei:

  \begin{enumerate}
      \item Bei ``Singulett'' verschr\"ankten Quanten ist der Zustand nach dem Messen des Ersten genau bestimmt.
        Es nimmt den Komplement\"arzustand zum Gemessenen an.
        Siehe Abbildung \ref{crypto:tangtab}
      \item Jegliche Manipulation an einem der verschr\"ankten Quanten \"andert die Zust\"ande der anderen.
        Dadurch kann eine Manipulation eines Eavesdropper festgestellt werden.
        Dies gilt auch, wenn mehr als zwei Quanten verschr\"ankt wurden.
  \end{enumerate}

  \begin{figure}
  \centering
    \begin{tabular}{ l || c | c | c | c | c | c | c | r }
      \hline
      Erzeugen der versch\"ankten Photonen \\
      \hline
      A's gew\"ahlte Basen & $+$ & $+$ & $\times $ & $+$ & $\times $ & $\times $ & $\times $ & $+$ \\
      \hline
      A's gemessene Polarisation & $\uparrow$ & $\rightarrow$ & $\searrow$ & $\uparrow$ & $\searrow$ &     $\nearrow$ & $\nearrow$ & $\rightarrow$ \\
      \hline
      A's gemessene Bits & 0 & 1 & 1 & 0 & 1 & 0 & 1 & 1 \\
      \hline
      B's gew\"ahlte Basen & $+$ & $+$ & $\times $ & $\times $ & $+$ & $\times $ & $+$ & $+$ \\
      \hline
      B's gemessene Polarisation & $\rightarrow$ & $\uparrow$ & $\nearrow$ & $\nearrow$ & $\rightarrow$     & $\searrow$ & $\rightarrow$ & $\uparrow$ \\
      \hline
      B's gemessene Bits & 1 & 0 & 0 & ? & ? & 1 & ? & 0 \\
      \hline
      Basen Diskussion \\
      \hline
      Alice's Key Bits & 0 & 1 & 1 & & & 0 & & 1 \\
      Alternative Key Bits & 1 & 0 & 0 & & & 1 & & 0 \\
      \hline
    \end{tabular}
    \caption{Die Emittlung des Schl\"ussels mit E91 \label{crypto:tangtab}}
  \end{figure}

  Dabei verwendet es ebenfalls einen konventionellen und einen Quantenkommunikationskanal.
  Dabei k\"onnen ebenfalls beide unsicher sein, dabei darf der Konventionelle auch unauthentisiert sein.

  % Nach der Basendiskussion.

  Die beiden Parteien w\"ahlen jeweils drei aus vier Winkeln als Basis (f"ur Photonen $45^{\circ}$, $22.5^{\circ}$, $0^{\circ}$, $-22.5^{\circ}$), wobei genau 2 davon "ubereinstimmen m"ussen.
  Dann wird dem Ereignis ``Photon detektiert'' den Bitwert 0 oder 1 zugewiesen.
  F"ur den Schl"usselaustausch werden mit einer Quelle verschr"ankte Photonen erzeugt und
  jeweils eins davon an jeden Kommunikationspartner gesendet.


 
  % Detektieren mit Avalanche-Photodioden
  % Basen: A und B wählen jeweils 3 aus vier Winkel, wovon 2 identisch sein müssen. z.B. A: 45°, 22.5°, 0° B: 22.5°, 0°, -22.5° (für Elektronene wären die Winkel doppelt so gross.
  % Eve kann detektiert werden, weil die Bellsche Ungleichung einen zu tiefen Wert liefert.

