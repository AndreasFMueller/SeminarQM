\newcommand{\qcste}{$-22.5^{\circ}$}
\newcommand{\qcstz}{$0^{\circ}$}
\newcommand{\qcstd}{$22.5^{\circ}$}
\newcommand{\qcstv}{$45^{\circ}$}

\section{E91}
  \index{E91}
  \rhead{E91}
  \subsection{Verschr"ankung}
  Durch Verschr"ankung lassen sich mehrere Photonen mit dem gleichen Zustand als ein System beschreiben. Diese Beschreibung ist notwendig nichtlokal.\\
  	Wir besitzen zwei Systeme $A$ und $B$ mit den Hilbert R"aumen $H_A$ und $H_B$. Pro System sind zwei Basisvektoren,  $ \{|0\rangle_A,|1\rangle_A\}$ und $\{|0\rangle_B,|1\rangle_B\}$ gegeben. Somit ist der folgende Singulett Zustand verschr"ankt: $\frac{1}{\sqrt{2}}(\lvert0\rangle_A|1\rangle_B - |1\rangle_A|0\rangle_B) $. Dadurch ist der Zustand von $A$ und $B$ "uberlagert.\\
	Man nehme jetzt an Alice beobachtet System A und Bob das System B. F"uhrt Alice nun eine Messung durch k"onnen folgende Ereignisse gleich wahrscheinlich auftreten. Der Zustand des Systems kollabiert zu $\lvert0\rangle_A|1\rangle_B$ und Alice misst $\psi_0$ oder das System kollabiert zu $\lvert1\rangle_A|0\rangle_B$ und Alice misst $\psi_1$. F"ur den ersten Fall erh"alt Bob bei jeder Messung, welche er durchf"uhrt $\psi_1$ beim zweiten Fall immer $\psi_0$. Dank diesem Verhalten wird ein Abh"oren/Messen der "ubertragenen Photonen sofort bemerkt.

  \subsection{Das E91 Protokoll}
  Das E91 Protokoll wurde 1991 von Artur Ekert entworfen.
  Dabei werden verschr"ankte Photonen verwendet, welche von Alice, Bob oder sogar Eve erzeugt werden k"onnen.

  \subsection{Der Schl"usselaustausch mit E91}
  Das Protokoll beruht auf der Nutzung der Eigenschaften von verschr"ankten Quanten \cite{qc:verschraenkung}.
  Die beiden ausschlaggebenden Eigenschaften sind dabei:

  \begin{enumerate}
      \item Bei ``Singulett'' verschr"ankten Quanten ist der Zustand nach dem Messen des Ersten genau bestimmt.
        Es nimmt den Komplement"arzustand zum Gemessenen an.
        Siehe Abbildung \ref{crypto:tangtab}
      \item Jegliche Manipulation an einem der verschr"ankten Quanten "andert die Zust"ande der anderen.
        Dadurch kann eine Manipulation eines Eavesdropper festgestellt werden.
        Dies gilt auch, wenn mehr als zwei Quanten verschr"ankt wurden.
  \end{enumerate}

  \begin{figure}
	\centering
    \begin{tabular}{|l|c|c|c|c|c|c|c|c|}
    	\hline Anti-/Symmetrische Wellenfkt & A & A & A & A & S & S & S & S \\
    	\hline Von A genutzte Basen & \qcste & \qcstz & \qcstd & \qcstv & \qcste & \qcstz & \qcstd & \qcstv \\
    	\hline Von A gemessene Bits & 1 & 0 & 1 & 0 & 0 & 1 & 0 & 0 \\
    	\hline Von B genutzte Basen & \qcste & \qcstv & \qcstz & \qcste & \qcstd & \qcstd & \qcstd & \qcstd \\
    	\hline Von B gemessene Bits & 0 & ? & ? & ? & ? & ? & 0 & ? \\
      \hline Basendiskussion / Nutzen & \checkmark & $\times$ & CHSH & CHSH & $\times$ & CHSH & \checkmark & CHSH \\
    	\hline Basen Zwischenwinkel & \qcstz & \qcstv & \qcstd & $67.5^{\circ}$ & \qcstv & \qcstd & \qcstz & \qcstd \\
    	\hline Gemeinsame Schl"usselbits & 1 & $\times$ & $\times$ & $\times$ & $\times$ & $\times$ & 0 & $\times$ \\
    	\hline
    \end{tabular}
    \caption{Die Emittlung des Schl"ussels mit E91 \label{crypto:tangtab}}
  \end{figure}

  Dabei verwendet es ebenfalls einen konventionellen und einen Quantenkommunikationskanal.
  Dabei k"onnen ebenfalls beide unsicher sein, dabei darf der konventionelle auch unauthentisiert sein.

  % Nach der Basendiskussion.

  Die beiden Parteien w"ahlen jeweils drei aus vier Winkeln als Basis (f"ur Photonen \qcste, \qcstz, \qcstd, \qcstv), wobei genau 2 davon "ubereinstimmen m"ussen.
  Dann wird dem Ereignis ``Photon detektiert'' den Bitwert 0 oder 1 zugewiesen.
  F"ur den Schl"usselaustausch werden mit einer Quelle verschr"ankte Photonen erzeugt und
  jeweils eins davon an jeden Kommunikationspartner gesendet.

  Es werden zwei F"alle unterschieden:
  \begin{itemize}
      \item Die Photonen haben eine symmetrische Wellenfunktion, dann kann A sichergehen, dass B das gleiche Messresultat bekommen hat, wie sie, wenn B dieselbe Basis verwendet hat.
      \item  Die Photonen haben eine antisymmetrische Wellenfunktion, dann weiss A, dass B das invertierte Messresultat bekommen hat, wenn B dieselbe Basis verwendet hat.
  \end{itemize}

  Die F"alle, in denen die Basis um \qcstv verschieden sind, sind nutzlos, da das Resultat bei B nicht definiert ist.
  Die F"alle mit einem Zwischenwinkel von \qcstd oder $67.5^{\circ}$ werden verwendet um mit der CHSH Ungleichung (Eine verallgemeinerte Form der Bellschen Ungleichung \ref{chapter:bell}) die Sicherheit des Kommunikationkanals festzustellen.
