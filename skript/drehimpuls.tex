\chapter{Drehimpuls\label{chapter:drehimpuls}}
\lhead{Drehimpuls}
\rhead{}
Der Drehimpuls ist eine wichtige Erhaltungsgr"osse in der klassischen
Mechanik. Wir sollten in der Lage sein, einen entsprechenden Operator
f"ur den quantenmechanischen Drehimpuls zu konstruieren.

\section{Drehimpulsoperatoren}
In der klassischen Mechanik ist der Drehimpulsvektor definiert
als $\vec L=\vec r\times \vec p$. Die einzelnen Komponenten k"onnen
ausgerechnet werden:
\begin{align*}
L_1&=X_2P_3-X_3P_2,\\
L_2&=X_3P_1-X_1P_3,\\
L_3&=X_1P_2-X_2P_1.
\end{align*}
Auf die Reihenfolge der $x$- und $p$-Komponenten kommt es nicht an,
weil nur Produkte von Orts- und Impuls-Operatoren f"ur verschiedene
Richtungen verwendet werden.

Ausser dem Vektor $\vec L$ ist auch seine L"ange erhalten, also
die Gr"osse
\[
\vec L^2=L_1^2+L_2^2+L_3^2.
\]
Wir erwarten, dass auch in der Quantenmechanik der Drehimpuls
erhalten ist, dass also sowohl die einzelnen Komponenten $L_i$ 
als auch der Betrag $\vec L^2$ mit dem Hamilton-Operator vertauschen.

Wir werden daher zun"achst die algebraischen Eigenschaften, insbesondere
die Vertauschungsrelationen bestimmen. Damit k"onnen wir dann "ahnlich
wie beim harmonischen Oszillator die Drehimpulszust"ande ermitteln.
Die Drehimpulsoperatoren in Ortsdarstellung werden wir als Teile
des Laplace-Operators wiedererkennen, den wir bei der Berechnung
des Wasserstoffatoms bereits analysiert haben. Dies wird uns erlauben,
die Quantenzahlen $l$ und $m$ der Wasserstoffzust"ande als
Drehimpuls-Quantenzahlen zu verstehen.

\section{Algebraische Eigenschaften}
Zun"achst erinnern wir an die grundlegenden Vertauschungsrelationen zwischen
Ort und Impuls:
\[
[P,X]=i\hbar \operatorname{id}
\]
oder die "aquivalente Formulierung
\[
PX=XP+i\hbar \operatorname{id}
\]
Daraus sollen jetzt schrittweise die Vertauschungsrelationen f"ur die
Drehimpulskomponenten und f"ur $\vec L^2$ abgeleitet werden.
Die zweite Form der Relation erlaubt, $P$-Faktoren immer nach rechts zu
bringen, und so f"ur jeden Ausdruck in $X_i$ und $P_i$ eine
``Standardform'' zu finden, so dass solche ausdr"ucken sogar algorithmisch
miteinander verglichen werden k"onnen.

Zun"achst berechnen wir die Vertauschungsrelationen der
Drehimpulskomponenten mit $X$ und $P$:
\begin{align*}
[X_1,L_1]&=0\\
[X_2,L_1]
&=
X_2X_2P_3-X_2X_3P_2-X_2P_3X_2+X_3P_2X_2
=X_3[P_2,X_2]
=i\hbar X_3
\\
[X_3,L_1]
&=
X_3X_2P_3-X_3X_3P_2-X_2P_3X_3+X_3P_2X_3
%= X_3X_2P_3 - X_2P_3X_3
=
-X_2[P_3,X_3]
=
-i\hbar X_2
\\
[P_1,L_1]&=0\\
[P_2,L_1]
&=
P_2X_2P_3 - P_2X_3P_2 - X_2P_3P_2 + X_3P_2P_2
=
P_3[P_2,X_2]
=
i\hbar P_3
\\
[P_3,L_1]
&=
P_3X_2P_3 - P_3X_3P_2 - X_2P_3P_3 + X_3P_2P_3
=
-P_2[P_3,X_3]=-i\hbar P_2
\end{align*}
Weitere Vertauschungsrelationen k"onnen durch zyklische Vertauschung
gewonnen werden:
\begin{align*}
[X_1,L_1] &= 0          & [X_1,L_2] &=-i\hbar X_3 & [X_1,L_3] &= i\hbar X_2\\
[X_2,L_1] &= i\hbar X_3 & [X_2,L_2] &= 0          & [X_2,L_3] &=-i\hbar X_1\\
[X_3,L_1] &=-i\hbar X_2 & [X_3,L_2] &= i\hbar X_1 & [X_3,L_3] &= 0         \\
[P_1,L_1] &= 0          & [P_1,L_2] &=-i\hbar P_3 & [P_1,L_3] &= i\hbar P_2\\
[P_2,L_1] &= i\hbar P_3 & [P_2,L_2] &= 0          & [P_2,L_3] &=-i\hbar P_1\\
[P_3,L_1] &=-i\hbar P_2 & [P_3,L_2] &= i\hbar P_1 & [P_3,L_3] &= 0
\end{align*}
Die Drehimpulskomponenten vertauschen mit den Orts- und Impulskomponenten
mit gleicher Richtung, aber nicht mit den anderen.
Man kann von einem Teilchen also gleichzeitig nur die Impulskomponente
und die dazu parallele Drehimpulskomponente wissen. 
Oder: vollst"andige Kenntnis des Drehimpulses schliesst vollst"andige
Kenntnis des Bewegungszustandes aus.

Es fragt sich allerdings auch, ob man "uberhaupt vollst"andige Kenntnis
des Drehimpulszustandes haben kann.
Dazu m"ussen die Kommutatoren der Drehimpulskomponenten untereinander
berechnet werden:
\begin{align}
[L_1,L_2]
&=
[X_2P_3-X_3P_2,L_2]
=
X_2[P_3,L_2]-P_2[X_3,L_2]
=
X_2i\hbar P_1-P_2i\hbar X_1
=
i\hbar L_3.
\label{drehimpulskommutator}
\\
[L_2,L_3]&=i\hbar L_1
\notag
\\
[L_3,L_1]&=i\hbar L_2
\notag
\\
\{L_1,L_2\}
&=
L_1L_2+L_2L_1=i\hbar L_3+2L_2L_1
\label{drehimpulsantikommutator}
\end{align}
Die zweite und dritte Relation haben wir durch zyklische Vertauschung
gewonnen.
Die Drehimpulskomponenten vertauschen untereinander nicht, es ist also
nicht m"oglich, zwei Drehimpulskomponenten eines Teilchens
gleichzeitig exakt zu wissen.
Es gibt keine Zust"ande, die Eigenzust"ande f"ur mehr als einen
der Operatoren f"ur die Drehimpulskomponenten ist.

Die Gleiche Frage k"onnen wir auch stellen f"ur den Gesamtdrehimpuls
$\vec L^2=L_1^2+L_2^2+L_3^2$. Dazu m"ussen wir Vertauschungsrelationen
verschiedener Operatoren mit den einzelnen Summanden in $\vec L^2$, also
mit $L_i^2$ berechnen.
Solche Berechnungen k"onnen vereinfacht werden durch eine Hilfsformel:

\begin{hilfssatz}
\label{commutatora2b}
Seien $A$ und $B$ Operatoren, dann gilt
\begin{align*}
[A^2,B]
&=
\{A,[A,B]\}
=
[A,\{A,B\}].
\end{align*}
\end{hilfssatz}

\begin{proof}[Beweis]
Wir schreiben den Kommutator aus:
\begin{align*}
[A^2,B]
=
AAB-BAA
&=
AAB\underbrace{\mathstrut -ABA+ABA}_{=0}\mathstrut -BAA
=
A[A,B]+[A,B]A
=
\{A,[A,B]\}
\\
&=
AAB\underbrace{\mathstrut +ABA-ABA}_{=0}\mathstrut -BAA
=
A\{A,B\}-\{A,B\}A
=
[A,\{A,B\}]
\end{align*}
\end{proof}

Wir wenden diesen Hilfssatz auf die Berechnung der Vertauschungsrelationen
der Komponenten des Drehimpulses mit $\vec L^2$ an.
Dazu werden wir die Antikommutatoren des Drehimpulses brauchen, die
wir in (\ref{drehimpulsantikommutator}) schon berechnet haben.
Wir finden:
\begin{align*}
[L_1^2,L_1]&=0
\\
[L_2^2,L_1]
&=
\{L_2,[L_2,L_1]\}
=
-\{L_2,i\hbar L_3\}
=-i\hbar\{L_2,L_3\}
\\
[L_3^2,L_1]
&=
\{L_3,[L_3,L_1]\}
=
\{L_3,i\hbar L_2\}
=
i\hbar\{L_2,L_3\}
\\
[\vec L^2, L_1]
&=
[L_1^2,L_1]
+
[L_2^2,L_1]
+
[L_3^2,L_1]
=
0
-i\hbar\{L_2,L_3\}
+
i\hbar\{L_2,L_3\}
=
0
\\
[\vec L^2,L_2]&=0
\qquad
\qquad
\text{(durch zyklische Vertauschung)}
\\
[\vec L^2,L_3]&=0
\end{align*}
Die Komponenten des Drehimpulses vertauschen untereinander zwar nicht,
aber sie vertauschen mit dem Gesamtdrehimpuls. Es ist also m"oglich,
den Gesamtdrehimpuls gleichzeitig mit einer der Komponenten zu messen,
aber es ist nicht m"oglich, mehr als eine Komponente des Drehimpulses
zu messen.

Damit stellt sich aber die Frage, ob der Gesamtdrehimpuls etwas ist,
was man unabh"angig vom Bewegungszustand messen kann.
Dazu m"ussen die Vertauschungsrelationen von $\vec L^2$ mit den
Impulskomponenten oder den Ortskomponenten bestimmt werden.
\begin{align*}
[L_1^2,X_1]
&=
\{L_1,[L_1,X_1]\}
=\{L_1,0\}=0
\\
[L_2^2,X_1]
&=
\{L_2,[L_2,X_1]\}
=
\{L_2,-i\hbar X_3\}
\\
[L_3^2,X_1]
&=
\{L_3,[L_3,X_1]\}
=
\{L_3, i\hbar X_2\}
\\
[\vec L^2,X_1]
&=
i\hbar(-L_2X_3-X_3L_2+L_3X_2+X_2L_3)
\\
&=
i\hbar(-L_2X_3-L_2X_3+i\hbar X_1 +L_3X_2+L_3X_2-i\hbar X_1)
=
2i\hbar(-L_2X_3 +L_3X_2)
\\
&=
2i\hbar(
-X_3P_1X_3+X_1P_3X_3+X_1P_2X_2-X_2P_1X_2
)
\\
&=
2i\hbar(
-(X_2^2+X_3^2)P_1+X_1X_3P_3-i\hbar X_1+ X_1X_2P_2-i\hbar X_1
)
\\
&=
2i\hbar(
-(X_2^2+X_3^2)P_1
+X_1(X_3P_3 + X_2P_2)
-2i\hbar X_1
)
\end{align*}
Eine "ahnliche Rechnung f"ur die Impulskomponenten zeigt, dass der
Drehimpulsbetrag nicht gleichzeitig mit Ort- oder Impulszustand
gemessen werden kann.

\section{Drehimpulszust"ande}
Beim harmonischen Oszillator haben wir eine Technik kennengelernt, nicht
nur die Eigenwerte, sondern auch die Eigenvektoren algebraisch aus dem
Grundzustand des Systems abzuleiten.
Ein "ahnliches Vorgehen ist auch hier m"oglich, denn auch hier haben wir
wieder eine Observable, n"amlich $\vec L^2$, welche nur positive
Werte annehmen kann.
Unter den Eigenzust"anden dieser Observablen muss es daher solche mit
minimalem Eigenwert geben.
Wir brauchen dann nur noch Operatoren, welche uns erlauben, von einem
solchen Grundzustand des Drehimpulsoperator zu den Zust"anden 
h"oheren Drehimpulses aufzusteigen.
Wenn dieser Plan realsiert werden kann, w"urde sich auch gleich
eine weitere wichtige Aussage der Quantenmechanik ergeben: der Drehimpuls
kommt nur in Vielfachen von $\hbar$ vor.

\subsection{Eigenzust"ande von $L_3$}
Aus den Operatoren $L_1$ und $L_2$ k"onnen wir "ahnlich wie beim
harmonischen Oszillator zwei neue Operatoren
\begin{align*}
L_+&=L_1+iL_2
&
L_-&=L_1-iL_2
\end{align*}
konstruieren.
Auch hier gilt die Einschr"ankung, dass dies keine selbstadjungierten
Operatoren sind, $L_\pm$ entspricht also nicht einer beobachtbaren
physikalischen Gr"osse. 
Diese Operatoren haben die folgenden Vertauschungsrelationen mit
den Drehimpulskomponenten:
\begin{equation}
\begin{aligned} 
\phantom{ }	% this is a workaround to prevent from the aligned environment
		% from interpreting the next commutator as an option to the
		% environment
[L_\pm, L_1]
&=
[L_1,L_1]\pm i[L_2,L_1]
=
\mp i\hbar L_3
\\
[L_\pm, L_2]
&=
[L_1,L_2]\pm i[L_2,L_2]
=
i\hbar L_3
\\
[L_\pm,L_3]
&=
[L_1,L_3]\pm i[L_2,L_3]
=
-i\hbar L_2
\mp
\hbar L_1
=
\mp \hbar L_\pm
\end{aligned}
\label{lpmlkommutator}
\end{equation}
Nehmen wir jetzt an, dass $|\psi\rangle$ ein Eigenzustand von $L_3$ mit
Eigenwert $l_3$ ist, dann gilt
\[
L_3L_+|\psi\rangle
=
L_+L_3|\psi\rangle+\hbar L_+|\psi\rangle
=
l_3L_+|\psi\rangle+\hbar L_+|\psi\rangle
=
(l_3+\hbar)L_+|\psi\rangle,
\]
der Zustand $L_+|\rangle$ ist also wieder ein Eigenzustand von $L_3$,
allerdings ist der Eigenwert um $\hbar$ erh"oht worden.
Analog kann man aus
\[
L_3L_-|\psi\rangle
=
L_-L_3|\psi\rangle-\hbar L_-|\psi\rangle
=
(l_3-\hbar)L_-|\psi\rangle
\]
ablesen,  dass $L_-|\psi\rangle$ ein Eigenzustand von $L_3$ ist, dessen
Eigenwert gegen"uber dem von $|\psi\rangle$ um $\hbar$ verringert
worden ist.
Die Operatoren $L_+$ und $L_-$ haben also genau die Eigenschaften,
die wir bei den Auf- und Absteigeoperatoren beim harmonischen Oszillator
kennengelernt hatten.

Aus (\ref{lpmlkommutator}) und aus
Hilfssatz~\ref{commutatora2b} k"onnen wir jetzt die
Vertauschungsrelationen mit den Quadraten der Drehimpulskomponenten
und mit $\vec L^2$ berechnen:
\begin{align*}
\\
[L_\pm,L_1^2]
&=
\{ [L_\pm, L_1], L_1 \}
=
\{ \mp i\hbar L_3, L_1 \}
=
\pm \hbar \{ L_1, L_3 \}
\\
[L_\pm,L_2^2]
&=
\{ [L_\pm, L_2], L_2 \}
=
\{ i\hbar L_3, L_2 \}
=
i\hbar \{ L_2, L_3 \}
\\
[L_\pm,L_1^2 + L_2^2]
&=
\pm \hbar \{ L_1, L_3 \}
+
i\hbar \{ L_2, L_3 \}
=
\pm \hbar \{ L_1 \mp iL_2, L_3 \}
=
\pm \hbar\{L_\pm, L_3\}
\\
[L_\pm,L_3^2]
&=
\{ [L_\pm, L_3], L_3 \}
=
\mp \{ i\hbar L_\pm, L_3 \}
=
\mp i\hbar \{ L_\pm, L_3 \}
\\
\Rightarrow\qquad [L_\pm,\vec L^2]
&=0
\end{align*}
Die Operatoren $L_\pm$ vertauschen also mit $\vec L^2$.
Ist $|\psi\rangle$ ein Eigenzustand von $\vec L^2$ mit Eigenwert $\lambda$,
dann gilt auch 
\begin{align*}
\vec L^2|\psi\rangle&=\lambda|\psi\rangle
\\
L_\pm\vec L^2|\psi\rangle&=
\vec L^2L_\pm|\psi\rangle=
\lambda L_\pm|\psi\rangle
\end{align*}
also ist  auch $L_\pm|\psi\rangle$ ein Eigenzustand von $\vec L^2$ mit
dem gleichen Eigenwert.

%Gemeinsame Eigenzust"ande $|\lambda,m\rangle$ von $\vec L^2$ und $L_3$
%mit Eigenwerten $\lambda$ und $m$ k"onnen also mit den Operatoren 
%$L_\pm$ in neue Eigenzust"ande mit gleichem $\lambda$, aber verschiedenem
%$m$ umgewandelt werden.
%\begin{align*}
%L_+|\lambda,m\rangle&=|\lambda,m+1\rangle
%L_-|\lambda,m\rangle&=|\lambda,m-1\rangle
%\end{align*}
%
\subsection{Der $N$-Operator}
Beim harmonischen Oszillator konnten wir aus dem Produkt $a^+a$
den Hamilton-Operator rekonstruieren und daraus ableiten, dass es einen
Zustand minimaler Energie gibt.
Wir versuchen dasselbe f"ur den Drehimpulsoperator.

Dazu berechnen wir zun"achst die Produkte der Operatoren
$L_+$ und $L_-$
\begin{align*}
L_+L_-
&=
L_1^2+L_2^2 +iL_2L_1-iL_1L_2=L_1^2+L_2^2 -i[L_1,L_2]=L_1^2+L_2^2+\hbar L_3
\\
L_-L_+
&=
L_1^2  + L_2^2 +i[L_1,L_2]=L_1^2+L_2^2-\hbar L_3
\\
{\textstyle \frac12}(L_+L_-+L_-L_+)&=L_1^2+L_2^2.
\end{align*}

\subsection{Eigenzust"ande von $\vec L^2$}
Wir fragen uns, ob es solche Auf- und Absteigeoperatoren auch
f"ur den Operator $\vec L^2$ gibt, oder wenigstens einen Operator
wie $N$ .


Die Operatoren $L_+$ und $L_-$ haben folgende Produkte
\begin{align*}
L_+L_-
&=
L_1^2+L_2^2 +iL_2L_1-iL_1L_2=L_1^2+L_2^2 -i[L_1,L_2]=L_1^2+L_2^2+\hbar L_3
\\
L_-L_+
&=
L_1^2  + L_2^2 +i[L_1,L_2]=L_1^2+L_2^2-\hbar L_3
\end{align*}
Sei jetzt $|l,m\rangle$ ein Zustand, in dem sowohl $L_3$ und $\vec L^2$
genau bekannt sind.

\section{Drehimpuls in Ortsdarstellung}







