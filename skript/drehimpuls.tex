\chapter{Drehimpuls\label{chapter:drehimpuls}}
\lhead{Drehimpuls}
\rhead{}
Der Drehimpuls ist eine wichtige Erhaltungsgr"osse in der klassischen
Mechanik. Wir sollten in der Lage sein, einen entsprechenden Operator
f"ur den quantenmechanischen Drehimpuls zu konstruieren.

\section{Drehimpulsoperatoren}
In der klassischen Mechanik ist der Drehimpulsvektor definiert
als $\vec L=\vec r\times \vec p$. Die einzelnen Komponenten k"onnen
ausgerechnet werden:
\begin{align*}
L_1&=X_2P_3-X_3P_2,\\
L_2&=X_3P_1-X_1P_3,\\
L_3&=X_1P_2-X_2P_1.
\end{align*}
Auf die Reihenfolge der $x$- und $p$-Komponenten kommt es nicht an,
weil nur Produkte von Orts- und Impuls-Operatoren f"ur verschiedene
Richtungen verwendet werden.

Ausser dem Vektor $\vec L$ ist auch seine L"ange erhalten, also
die Gr"osse
\[
\vec L^2=L_1^2+L_2^2+L_3^2.
\]
Wir erwarten, dass auch in der Quantenmechanik der Drehimpuls
erhalten ist, dass also sowohl die einzelnen Komponenten $L_i$ 
als auch der Betrag $\vec L^2$ mit dem Hamilton-Operator vertauschen.

Wir werden daher zun"achst die algebraischen Eigenschaften, insbesondere
die Vertauschungsrelationen bestimmen. Damit k"onnen wir dann "ahnlich
wie beim harmonischen Oszillator die Drehimpulszust"ande ermitteln.
Die Drehimpulsoperatoren in Ortsdarstellung werden wir als Teile
des Laplace-Operators wiedererkennen, den wir bei der Berechnung
des Wasserstoffatoms bereits analysiert haben. Dies wird uns erlauben,
die Quantenzahlen $l$ und $m$ der Wasserstoffzust"ande als
Drehimpuls-Quantenzahlen zu verstehen.

\section{Algebraische Eigenschaften}
Zun"achst erinnern wir an die grundlegenden Vertauschungsrelationen zwischen
Ort und Impuls:
\[
[P,X]=i\hbar \operatorname{id}.
\]
Daraus sollen jetzt schrittweise die Vertauschungsrelationen f"ur die
Drehimpulskomponenten und f"ur $\vec L^2$ abgeleitet werden.

Zun"achst berechnen wir die Vertauschungsrelationen der
Drehimpulskomponenten mit $X$ und $P$:
\begin{align*}
[X_1,L_1]&=0\\
[X_2,L_1]
&=
X_2X_2P_3-X_2X_3P_2-X_2P_3X_2+X_3P_2X_2
=X_3[P_2,X_2]
=i\hbar X_3
\\
[X_3,L_1]
&=
X_3X_2P_3-X_3X_3P_2-X_2P_3X_3+X_3P_2X_3
%= X_3X_2P_3 - X_2P_3X_3
=
-X_2[P_3,X_3]
=
-i\hbar X_2
\\
[P_1,L_1]&=0\\
[P_2,L_1]
&=
P_2X_2P_3 - P_2X_3P_2 - X_2P_3P_2 + X_3P_2P_2
=
P_3[P_2,X_2]
=
i\hbar P_3
\\
[P_3,L_1]
&=
P_3X_2P_3 - P_3X_3P_2 - X_2P_3P_3 + X_3P_2P_3
=
-P_2[P_3,X_3]=-i\hbar P_2
\\
[L_1,L_2]
&=
[X_2P_3-X_3P_2,L_2]
=
X_2[P_3,L_2]-P_2[X_3,L_2]
=
X_2i\hbar P_1-P_2i\hbar X_1
=
i\hbar L_3.
\\
\{L_1,L_2\}
&=
L_1L_2+L_2L_1=i\hbar L_3+2L_2L_1
\end{align*}
Dazu kommen weitere Relationen durch zyklische Vertauschung.
F"ur die Vertauschungen der Komponenten des Drehimpulses mit $\vec L^2$
finden wir:
\begin{align*}
[L_1^2,L_1]&=0
\\
[L_2^2,L_1]
&=
L_2L_2L_1-L_1L_2L_2
=
L_2(-i\hbar L_3 +L_1L_2)
-
(i\hbar L_3+L_1L_2)L_2
=
-i\hbar\{L_2,L_3\}
\\
[L_3^2,L_1]
&=
L_3L_3L_1-L_1L_3L_3
=
L_3(i\hbar L_2 + L_1L_3)
-
(-i\hbar L_2 + L_3L_1)L_3
=
i\hbar\{L_2,L_3\}
\\
[\vec L^2, L_1]
&=
[L_1^2,L_1]
+
[L_2^2,L_1]
+
[L_3^2,L_1]
=
0
-i\hbar\{L_2,L_3\}
+
i\hbar\{L_2,L_3\}
=
0
\\
[\vec L^2,L_2]&=0
\\
[\vec L^2,L_3]&=0
\end{align*}
Die Komponenten des Drehimpulses vertauschen untereinander zwar nicht,
aber sie vertauschen mit dem Gesamtdrehimpuls. Es ist also m"oglich,
den Gesamtdrehimpuls gleichzeitig mit einer der Komponenten zu messen,
aber es ist nicht m"oglich, mehr als eine Komponente des Drehimpulses
zu messen.

\section{Drehimpulszust"ande}


\section{Drehimpuls in Ortsdarstellung}







