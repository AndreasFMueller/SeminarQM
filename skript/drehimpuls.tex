\chapter{Drehimpuls\label{chapter:drehimpuls}}
\lhead{Drehimpuls}
\rhead{}
Der Drehimpuls ist eine wichtige Erhaltungsgr"osse in der klassischen
Mechanik. Wir sollten in der Lage sein, einen entsprechenden Operator
f"ur den quantenmechanischen Drehimpuls zu konstruieren.

%
% XXX klassische Erklärung für Vertauschungsrelation: Präzession
%

\section{Drehimpulsoperatoren\label{section:drehimpulsoperatoren}}
\rhead{Drehimpulsoperatoren}
In der klassischen Mechanik ist der Drehimpulsvektor definiert
als $\vec L=\vec r\times \vec p$. Die einzelnen Komponenten k"onnen
ausgerechnet werden:
\begin{align*}
L_1&=X_2P_3-X_3P_2,\\
L_2&=X_3P_1-X_1P_3,\\
L_3&=X_1P_2-X_2P_1.
\end{align*}
\index{Drehimpuls!Komponenten}%
Auf die Reihenfolge der $x$- und $p$-Komponenten kommt es nicht an,
weil nur Produkte von Orts- und Impuls-Operatoren f"ur verschiedene
Richtungen verwendet werden, die zugeh"origen Operatoren vertauschen.

Ausser dem Vektor $\vec L$ ist auch seine L"ange erhalten, also
die Gr"osse
\[
\vec L^2=L_1^2+L_2^2+L_3^2.
\]
\index{Drehimpuls!Betrag}%
Wir erwarten, dass auch in der Quantenmechanik der Drehimpuls
erhalten ist, dass also sowohl die einzelnen Komponenten $L_i$ 
als auch der Betrag $\vec L^2$ mit dem Hamilton-Operator vertauschen.

Wir werden daher zun"achst die algebraischen Eigenschaften, insbesondere
die Vertauschungsrelationen bestimmen. Damit k"onnen wir dann "ahnlich
wie beim harmonischen Oszillator die Drehimpulszust"ande ermitteln.
Die Drehimpulsoperatoren in Ortsdarstellung werden wir als Teile
des Laplace-Operators wiedererkennen, den wir bei der Berechnung
des Wasserstoffatoms bereits analysiert haben. Dies wird uns erlauben,
die Quantenzahlen $l$ und $m$ der Wasserstoffzust"ande als
Drehimpuls-Quantenzahlen zu verstehen.

\section{Algebraische Eigenschaften\label{section:drehimpulsalgebra}}
\rhead{Drehimpulsalgebra}
Zun"achst erinnern wir an die grundlegenden Vertauschungsrelationen zwischen
Ort und Impuls:
\[
[P,X]=-i\hbar \operatorname{id}
\]
oder die "aquivalente Formulierung
\[
PX=XP-i\hbar \operatorname{id}.
\]
Daraus sollen jetzt schrittweise die Vertauschungsrelationen f"ur die
Drehimpulskomponenten und f"ur $\vec L^2$ abgeleitet werden.
Die zweite Form der Relation erlaubt, $P$-Faktoren immer nach rechts zu
bringen, und so f"ur jeden Ausdruck in $X_i$ und $P_i$ eine
``Standardform'' zu finden, so dass solche Ausdr"ucke sogar algorithmisch
miteinander verglichen werden k"onnen.

\subsection{Vertauschungsrelationen}
Zun"achst berechnen wir die Vertauschungsrelationen der
Drehimpulskomponenten mit $X$ und $P$:
\index{Drehimpuls!Vertauschungsrelationen}
\begin{align*}
[X_1,L_1]&=0\\
[X_2,L_1]
&=
X_2X_2P_3-X_2X_3P_2-X_2P_3X_2+X_3P_2X_2
=X_3[P_2,X_2]
=-i\hbar X_3
\\
[X_3,L_1]
&=
X_3X_2P_3-X_3X_3P_2-X_2P_3X_3+X_3P_2X_3
%= X_3X_2P_3 - X_2P_3X_3
=
-X_2[P_3,X_3]
=
i\hbar X_2
\\
[P_1,L_1]&=0\\
[P_2,L_1]
&=
P_2X_2P_3 - P_2X_3P_2 - X_2P_3P_2 + X_3P_2P_2
=
P_3[P_2,X_2]
=
-i\hbar P_3
\\
[P_3,L_1]
&=
P_3X_2P_3 - P_3X_3P_2 - X_2P_3P_3 + X_3P_2P_3
=
P_2[P_3,X_3]=i\hbar P_2
\end{align*}
Weitere Vertauschungsrelationen k"onnen durch zyklische Vertauschung
gewonnen werden:
\begin{align*}
[X_1,L_1] &= 0          & [X_1,L_2] &= i\hbar X_3 & [X_1,L_3] &=-i\hbar X_2\\
[X_2,L_1] &=-i\hbar X_3 & [X_2,L_2] &= 0          & [X_2,L_3] &= i\hbar X_1\\
[X_3,L_1] &= i\hbar X_2 & [X_3,L_2] &=-i\hbar X_1 & [X_3,L_3] &= 0         \\
[P_1,L_1] &= 0          & [P_1,L_2] &= i\hbar P_3 & [P_1,L_3] &=-i\hbar P_2\\
[P_2,L_1] &=-i\hbar P_3 & [P_2,L_2] &= 0          & [P_2,L_3] &= i\hbar P_1\\
[P_3,L_1] &= i\hbar P_2 & [P_3,L_2] &=-i\hbar P_1 & [P_3,L_3] &= 0
\end{align*}
Die Drehimpulskomponenten vertauschen mit den Orts- und Impulskomponenten
mit gleicher Richtung, aber nicht mit den anderen.
Man kann von einem Teilchen also gleichzeitig nur die Impulskomponente
und die dazu parallele Drehimpulskomponente wissen. 
Oder: vollst"andige Kenntnis des Drehimpulses schliesst vollst"andige
Kenntnis des Bewegungszustandes aus.

Es fragt sich allerdings auch, ob man "uberhaupt vollst"andige Kenntnis
des Drehimpulszustandes haben kann.
Dazu m"ussen die Kommutatoren der Drehimpulskomponenten untereinander
berechnet werden:
\begin{align}
[L_1,L_2]
&=
[X_2P_3-X_3P_2,L_2]
=
X_2[P_3,L_2]-P_2[X_3,L_2]
=
-X_2i\hbar P_1+P_2i\hbar X_1
=
i\hbar L_3.
\label{skript:drehimpulskommutator}
\\
[L_2,L_3]&=i\hbar L_1
\notag
\\
[L_3,L_1]&=i\hbar L_2
\notag
\\
\{L_1,L_2\}
&=
L_1L_2+L_2L_1=i\hbar L_3+2L_2L_1
\label{skript:drehimpulsantikommutator}
\end{align}
Die zweite und dritte Relation haben wir durch zyklische Vertauschung
gewonnen.
Die Drehimpulskomponenten vertauschen untereinander nicht, es ist also
nicht m"oglich, zwei Drehimpulskomponenten eines Teilchens
gleichzeitig exakt zu wissen.
Es gibt keine Zust"ande, die Eigenzust"ande f"ur mehr als einen
der Operatoren f"ur die Drehimpulskomponenten ist.

Die gleiche Frage k"onnen wir auch stellen f"ur den Gesamtdrehimpuls
$\vec L^2=L_1^2+L_2^2+L_3^2$. Dazu m"ussen wir Vertauschungsrelationen
verschiedener Operatoren mit den einzelnen Summanden in $\vec L^2$, also
mit $L_i^2$ berechnen.
Solche Berechnungen k"onnen vereinfacht werden durch eine Hilfsformel:

\begin{hilfssatz}
\label{skript:commutatora2b}
Seien $A$ und $B$ Operatoren, dann gilt
\begin{align*}
[A^2,B]
&=
\{A,[A,B]\}
=
[A,\{A,B\}].
\end{align*}
\end{hilfssatz}

\begin{proof}[Beweis]
Wir schreiben den Kommutator aus:
\begin{align*}
[A^2,B]
=
AAB-BAA
&=
AAB\underbrace{\mathstrut -ABA+ABA}_{=0}\mathstrut -BAA
=
A[A,B]+[A,B]A
=
\{A,[A,B]\}
\\
&=
AAB\underbrace{\mathstrut +ABA-ABA}_{=0}\mathstrut -BAA
=
A\{A,B\}-\{A,B\}A
=
[A,\{A,B\}]
\end{align*}
\end{proof}

Wir wenden diesen Hilfssatz auf die Berechnung der Vertauschungsrelationen
der Komponenten des Drehimpulses mit $\vec L^2$ an.
Dazu werden wir die Antikommutatoren des Drehimpulses brauchen, die
wir in (\ref{skript:drehimpulsantikommutator}) schon berechnet haben.
Wir finden:
\begin{align*}
[L_1^2,L_1]&=0
\\
[L_2^2,L_1]
&=
\{L_2,[L_2,L_1]\}
=
-\{L_2,i\hbar L_3\}
=-i\hbar\{L_2,L_3\}
\\
[L_3^2,L_1]
&=
\{L_3,[L_3,L_1]\}
=
\{L_3,i\hbar L_2\}
=
i\hbar\{L_2,L_3\}
\\
[\vec L^2, L_1]
&=
[L_1^2,L_1]
+
[L_2^2,L_1]
+
[L_3^2,L_1]
=
0
-i\hbar\{L_2,L_3\}
+
i\hbar\{L_2,L_3\}
=
0
\\
[\vec L^2,L_2]&=0
\qquad
\qquad
\text{(durch zyklische Vertauschung)}
\\
[\vec L^2,L_3]&=0
\end{align*}
Die Komponenten des Drehimpulses vertauschen untereinander zwar nicht,
aber sie vertauschen mit dem Gesamtdrehimpuls. Es ist also m"oglich,
den Gesamtdrehimpuls gleichzeitig mit einer der Komponenten zu messen,
aber es ist nicht m"oglich, mehr als eine Komponente des Drehimpulses
zu messen.

\begin{hilfssatz}
$[\vec L^2, L_3]=0$, es gibt gemeinsame Eigenzust"ande
der Vektoren $\vec L^2$ und $L_3$.
\end{hilfssatz}

Damit stellt sich die Frage, ob der Gesamtdrehimpuls etwas ist,
was man unabh"angig vom Bewegungszustand messen kann.
Dazu m"ussen die Vertauschungsrelationen von $\vec L^2$ mit den
Impulskomponenten oder den Ortskomponenten bestimmt werden.
\begin{align*}
[L_1^2,X_1]
&=
\{L_1,[L_1,X_1]\}
=\{L_1,0\}=0
\\
[L_2^2,X_1]
&=
\{L_2,[L_2,X_1]\}
=
\{L_2,-i\hbar X_3\}
\\
[L_3^2,X_1]
&=
\{L_3,[L_3,X_1]\}
=
\{L_3, i\hbar X_2\}
\\
\Rightarrow\qquad
[\vec L^2,X_1]
&=
i\hbar(-L_2X_3-X_3L_2+L_3X_2+X_2L_3)
\\
&=
i\hbar(-L_2X_3-L_2X_3+i\hbar X_1 +L_3X_2+L_3X_2-i\hbar X_1)
=
2i\hbar(-L_2X_3 +L_3X_2)
\\
&=
2i\hbar(
-X_3P_1X_3+X_1P_3X_3+X_1P_2X_2-X_2P_1X_2
)
\\
&=
2i\hbar(
-(X_2^2+X_3^2)P_1+X_1X_3P_3-i\hbar X_1+ X_1X_2P_2-i\hbar X_1
)
\\
&=
2i\hbar(
-(X_2^2+X_3^2)P_1
+X_1(X_3P_3 + X_2P_2)
-2i\hbar X_1
)
\end{align*}
Eine "ahnliche Rechnung f"ur die Impulskomponenten zeigt, dass der
Drehimpulsbetrag nicht gleichzeitig mit Ort- oder Impulszustand
gemessen werden kann.
Dies ist auch physikalisch einleuchtend.
Kennt man den Ort und den Drehimpuls exakt, erh"alt man auch exakte
Information "uber den Impuls.
Da man aber Ort und Impuls nicht gleichzeitig wissen kann, muss es auch
eine Vertauschungsrelation geben, die verhindert, dass wir zu viel
gleichzeitiges Wissen "uber Ort und Drehimpuls erhalten k"onnen.

\subsection{Auf- und Absteigeoperatoren}
\index{Drehimpuls!Eigenzust\"ande von $L_3$}%
\index{Drehimpuls!Auf- und Absteigeoperatoren}%
Beim harmonischen Oszillator haben wir f"ur den Operator $H=P^2+Q^2$ die
Auf- und Absteige-Operatoren $a=Q+iP$ und $a^+=Q-iP$ bilden\footnote{Wir haben
hier im Gegensatz zu der Behandlung in
Kapitel~\ref{chapter:harmonischeroszillator} einen Faktor $\frac12$ 
weggelassen, da er f"ur die Diskussion hier nicht relevant ist.}.
Damit liess sich $H$ schreiben als $\frac12(a^+a+aa^+)$. Aus den
Vertauschungsrelationen von $a$ und $a^+$ liess sich ableiten, dass
man alle Eigenzust"ande von $H$ durch Anwendung des Aufsteigeoperators 
$a^+$ aus dem Grundzustand bekommen konnte.

Ziel dieses Abschnittes ist, auf "ahnliche Weise einen "Uberblick 
"uber die Eigenzust"ande von $\vec L^2$ und $L_3$ zu gewinnen.
Dazu beachten wir, dass $\vec L^2$ "ahnlich wie $H$ eine Quadratsumme
ist, allerdings mit drei Termen.
Mindestens f"ur die ersten zwei Terme $L_1^2+L_2^2$ sollten wir
aber die gleiche Konstruktion durchf"uhren k"onnen, indem wir
$Q$ und $P$ durch $L_1$ und $L_2$ ersetzen.

Aus den Operatoren $L_1$ und $L_2$ k"onnen wir also "ahnlich wie beim
harmonischen Oszillator zwei neue Operatoren
\begin{equation*}
\begin{aligned}
L_+&=L_1+iL_2
&
&\text{und}&
L_-&=L_1-iL_2
\end{aligned}
\end{equation*}
konstruieren.
Auch hier gilt die Einschr"ankung, dass dies keine selbstadjungierten
Operatoren sind, $L_\pm$ entspricht also nicht einer beobachtbaren
physikalischen Gr"osse. 
Immerhin ist $L_+^*=L_-$.

\begin{hilfssatz}
$L_+$ und $L_-$ erf"ullen die Vertauschungsrelation
\begin{align*}
[L_+, L_-]&=2\hbar L_3.
\end{align*}
Ausserdem l"asst sich der Gesamtdrehimpuls durch $L_+$, $L_-$ und $L_3$ als
\[
\vec L^2 = \frac12(L_+L_- + L_-L_+)+L_3^2
\]
ausdr"ucken.
\end{hilfssatz}

\begin{proof}[Beweis]
Wir berechnen die Produkte der Operatoren $L_+$ und $L_-$:
\begin{align}
L_+L_-
&=
L_1^2+L_2^2 +iL_2L_1-iL_1L_2=L_1^2+L_2^2 -i[L_1,L_2]=L_1^2+L_2^2+\hbar L_3
=\vec L^2-L_3^2+\hbar L_3
\label{skript:l+l-}
\\
L_-L_+
&=
L_1^2  + L_2^2 +i[L_1,L_2]=L_1^2+L_2^2-\hbar L_3
=\vec L^2-L_3^2-\hbar L_3
\label{skript:l-l+}
\end{align}
Daraus kann man jetzt sowohl den Kommutator
\begin{align*}
[L_+,L_-]
&=
2\hbar L_3
\end{align*}
als auch den Gesamtdrehimpuls $\vec L^2$
\begin{align*}
{\textstyle \frac12}(L_+L_-+L_-L_+)&=L_1^2+L_2^2,
\\
\vec L^2
&=
{\textstyle\frac12}(L_+L_-+L_-L_+)+L_3^2.
\end{align*}
berechnen.
\end{proof}

Wir brauchen die Vertauschungsrelationen der Operatoren $L_+$ und $L_-$
mit den Drehimpulskomponenten.
Die Rechnung ergibt
\begin{equation}
\begin{aligned} 
\phantom{ }	% this is a workaround to prevent from the aligned environment
		% from interpreting the next commutator as an option to the
		% environment
[L_\pm, L_1]
&=
[L_1,L_1]\pm i[L_2,L_1]
=
\pm \hbar L_3
\\
[L_\pm, L_2]
&=
[L_1,L_2]\pm i[L_2,L_2]
=
i\hbar L_3
\\
[L_\pm,L_3]
&=
[L_1,L_3]\pm i[L_2,L_3]
=
-i\hbar L_2
\mp
\hbar L_1
=
\mp \hbar L_\pm
\end{aligned}
\label{skript:lpmlkommutator}
\end{equation}

\begin{hilfssatz}
Die Auf- und Absteigeoperatoren $L_\pm$ f"ur den Drehimpuls vertauschen 
nicht mit $L_3$, es gilt
\[
[L_3,L_\pm]=\pm\hbar L_{\pm},
\]
insbesondere gilt auch
\begin{align*}
L_3L_+&=(L_3+\hbar)L_+\qquad\text{und}\\
L_3L_-&=(L_3-\hbar)L_-.
\end{align*}
\label{skript:laufab}
\end{hilfssatz}

\begin{proof}[Beweis]
Nehmen wir an, dass $|\psi\rangle$ ein Eigenzustand von $L_3$ mit
Eigenwert $l_3$ ist, dann gilt
\[
L_3L_+\,|\psi\rangle
=
L_+L_3\,|\psi\rangle+\hbar L_+|\psi\rangle
=
l_3L_+\,|\psi\rangle+\hbar L_+|\psi\rangle
=
(l_3+\hbar)L_+\,|\psi\rangle,
\]
der Zustand $L_+\,|\rangle$ ist also wieder ein Eigenzustand von $L_3$,
allerdings ist der Eigenwert um $\hbar$ erh"oht worden.
Analog kann man aus
\[
L_3L_-\,|\psi\rangle
=
L_-L_3\,|\psi\rangle-\hbar L_-\,|\psi\rangle
=
(l_3-\hbar)L_-\,|\psi\rangle
\]
ablesen,  dass $L_-\,|\psi\rangle$ ein Eigenzustand von $L_3$ ist, dessen
Eigenwert gegen"uber dem von $|\psi\rangle$ um $\hbar$ verringert
worden ist.
Die Operatoren $L_+$ und $L_-$ haben also genau die Eigenschaften,
die wir bei den Auf- und Absteigeoperatoren beim harmonischen Oszillator
kennengelernt hatten.
\end{proof}

\begin{hilfssatz}
$[\vec L^2,L_\pm]=0.$
\end{hilfssatz}

\begin{proof}[Beweis]
Der Kommutator kann
aus (\ref{skript:lpmlkommutator}) und aus
Hilfssatz~\ref{skript:commutatora2b} berechnet werden:
\begin{align*}
\\
[L_\pm,L_1^2]
&=
\{ [L_\pm, L_1], L_1 \}
=
\{ \pm \hbar L_3, L_1 \}
=
\pm \hbar \{ L_1, L_3 \}
\\
[L_\pm,L_2^2]
&=
\{ [L_\pm, L_2], L_2 \}
=
\{ i\hbar L_3, L_2 \}
=
i\hbar \{ L_2, L_3 \}
\\
[L_\pm,L_1^2 + L_2^2]
&=
\pm \hbar \{ L_1, L_3 \}
+
i\hbar \{ L_2, L_3 \}
=
\pm \hbar \{ L_1 \mp iL_2, L_3 \}
=
\pm \hbar\{L_\pm, L_3\}
\\
[L_\pm,L_3^2]
&=
\{ [L_\pm, L_3], L_3 \}
=
\mp \{ i\hbar L_\pm, L_3 \}
=
\mp i\hbar \{ L_\pm, L_3 \}
\\
\Rightarrow\qquad [L_\pm,\vec L^2]
&=0
\end{align*}
\end{proof}

Die Operatoren $L_\pm$ vertauschen also mit $\vec L^2$.
Ist $|\psi\rangle$ ein Eigenzustand von $\vec L^2$ mit Eigenwert $\lambda$,
dann gilt auch 
\begin{align*}
\vec L^2\,|\psi\rangle&=\lambda\,|\psi\rangle
\\
L_\pm\vec L^2\,|\psi\rangle&=
\vec L^2L_\pm\,|\psi\rangle=
\lambda L_\pm\,|\psi\rangle
\end{align*}
also ist  auch $L_\pm\,|\psi\rangle$ ein Eigenzustand von $\vec L^2$ mit
dem gleichen Eigenwert.

\subsection{M"ogliche Eigenwerte}
Der Hilfssatz~\ref{skript:laufab} zeigt bereits, dass $L_+$ und $L_-$
innerhalb der Eigenzust"ande von $L_3$ in Einheiten von $\hbar$
auf- bzw.~absteigen. 
Wir vermuten daher, dass die Eigenwerte ganzzahlige Vielfache von $\hbar$
sind.
Es w"are aber auch m"oglich, dass es mehrere Mengen von Eigenvektoren
gibt, deren Eigenwerte sich zwar jeweils um ganzzahlige Vielfache von $\hbar$
unterscheiden, aber die Eigenwerte der verschiedenen Mengen haben nichts
miteinander zu tun.
Das Ziel dieses Abschnittes ist daher, die m"oglichen Eigenwerte zu
berechnen und zu zeigen, dass nur ganz- oder halbzahlige Vielfache von
$\hbar$ m"oglich sind.

\begin{hilfssatz}
F"ur ein beliebiges Polynom $p(x)$ gilt
\begin{align*}
p(L_3)L_+ &= L_+ p(L_3+\hbar)\\
p(L_3)L_- &= L_- p(L_3-\hbar)
\end{align*}
\end{hilfssatz}

\begin{proof}[Beweis]
F"ur $p(L_3)=L_3$ haben wir dies bereits in Hilfssatz~\ref{skript:laufab}
gezeigt.
F"ur beliebige Potenzen $L_3^k$ mit $k>1$
k"onnen wir es mittels vollst"andiger Induktion zeigen.
Es ist
\begin{align*}
L_3^kL_+ &= L_3^{k-1}L_+(L_3+\hbar) = L_+(L_3+\hbar)^{k-1}(L_3+\hbar)
= L_+(L_3+\hbar)^k,
\\
L_3^kL_- &= L_3^{k-1}L_-(L_3-\hbar) = L_-(L_3-\hbar)^{k-1}(L_3-\hbar)
= L_-(L_3-\hbar)^k.
\end{align*}
Ein beliebiges Polynom $p(L_3)$ ist eine Linearkombination von Potenzen
$L_3^k$.
\end{proof}

\begin{hilfssatz}
F"ur $k\ge 0$ gilt 
\begin{align*}
L_-^kL_+^k
&=
\prod_{j=1}^k (\vec L^2 - (L_3+(j-1)\hbar)(L_3+j\hbar))\qquad\text{und}
\\
L_+^kL_-^k
&=
\prod_{j=1}^k (\vec L^2 - (L_3-(j-1)\hbar)(L_3-j\hbar)).
\end{align*}
\end{hilfssatz}

\begin{proof}[Beweis]
Auch diese Eigenschaft kann mit vollst"andiger Induktion bewiesen
werden.
Die Induktionsverankerung, also der Fall $k=1$ wird bewiesen durch
die Formeln~(\ref{skript:l+l-}) und (\ref{skript:l-l+}).

F"ur den Induktionsschritt rechnen wir
\begin{align*}
L_+^kL_-^k
=
L_+ L_+^{k-1}L_-^{k-1}L_-
&=
L_+\biggl(
\prod_{j=1}^{k-1} (\vec L^2 - (L_3-(j-1)\hbar)(L_3-j\hbar))
\biggr)L_-
\\
&=
\biggl(
\prod_{j=1}^{k-1} (\vec L^2 - (L_3-j\hbar)(L_3-(j+1)\hbar))
\biggr)L_+L_-
\\
&=
\biggl(
\prod_{j=2}^{k} (\vec L^2 - (L_3-(j-1)\hbar)(L_3-j)\hbar))
\biggr)
(\vec L^2 -L_3^2+\hbar L_3)
\\
&=
\prod_{j=1}^{k} (\vec L^2 - (L_3-j\hbar)(L_3-(j+1)\hbar)).
\end{align*}
Die entsprechende Rechnung f"ur $L_-^kL_+^k$ ist
\begin{align*}
L_-^kL_+^k
=
L_-L_-^{k-1}L_+^{k-1}L_+
&=
L_-\biggl(
\prod_{j=1}^{k-1}
(\vec L^2-(L_3-(j-1)\hbar)(L_3-j\hbar))
\biggr)L_+
\\
&=
L_-L_+\biggl(
\prod_{j=1}^{k-1}
(\vec L^2-(L_3-j)\hbar)(L_3-(j+1)\hbar))
\biggr)
\\
&=
(\vec L^2-L_3^2-\hbar L_3)\biggl(
\prod_{j=2}^k
(\vec L^2-(L_3-(j-1)\hbar)(L_3-j\hbar))
\biggr)
\\
&=
\prod_{j=1}^k
(\vec L^2-(L_3-(j-1))\hbar)(L_3-j\hbar))
\end{align*}
Damit ist der Induktionsschritt auch im Fall $L_-^kL_+^k$ vollzogen, und
damit der Hilfssatz vollst"andig bewiesen.
\end{proof}

Sei jetzt $|\psi\rangle$ ein Eigenzustand von $\vec L^2$ und $L_3$.
Wir messen die Eigenwerte in Einheiten $\hbar^2$ bwz.~$\hbar$, wir
k"onnen also schreiben
\begin{align*}
\vec L^2\, |\psi\rangle &= l(l+1)\hbar^2\,|\psi\rangle \\
     L_3\, |\psi\rangle &= m\hbar\,       |\psi\rangle,
\end{align*}
wobei wir vorl"aufig noch keine Voraussetzungen an $m$ und $l$ machen.
Ziel ist zu zeigen, dass $l$ und $m$ ganz- oder halbzahlig sind.
Immerhin setzen wir voraus, dass $|\psi\rangle$ normiert ist, dass also
$\langle\psi|\psi\rangle=1$ ist.

Da jeder Eigenwert von $\vec L^2$ positiv sein muss, ist es keine
Einschr"ankung der Allgemeinheit, wenn wir verlangen, dass $l>0$ sein
soll.

Wir berechnen jetzt das Normquadrat von $L_-^k\,|\psi\rangle$. Wegen
$L_-^*=L_+$ gilt
\begin{align*}
0
\le
\langle\psi|\, (L_-^*)^kL_-^k\,|\psi\rangle
&=
\langle\psi|\, L_+^kL_-^k\,|\psi\rangle
=
\prod_{j=1}^k(l(l+1)-(m-(j-1))(m-j))\hbar^2
\\
0
\le
\langle\psi|\, (L_+^*)^kL_+^k\,|\psi\rangle
&=
\langle\psi|\, L_-^kL_+^k\,|\psi\rangle
=
\prod_{j=1}^k(l(l+1)-(m+(j-1))(m+j))\hbar^2
\end{align*}
Der Term $(m\pm (j-1))(m\pm j))$ w"achst mit $j$ quadratisch an, 
f"ur ausreichend grosses $j$ wird der Faktor im Produkt also negativ.
Die Ungleichung kann also nur dann nicht verletzt werden, wenn 
es ganze Zahlen $j_\pm$ gibt mit
\begin{align*}
l(l+1)-(m+(j_--1))(m+j_-))&=0
\\
l(l+1)-(m-(j_+-1))(m-j_+))&=0
\end{align*}
Fassen wir $j_\pm-1$ jeweils das $j_\pm$ mit $m$ zusammen, erhalten wir
\begin{align}
l(l+1)-((m+j_-)-1)(m+j_-))&=0
\label{skript:lmjminus}
\\
l(l+1)-((m-j_+)+1)(m-j_+))&=0
\label{skript:lmjplus}
\end{align}
Wir m"ussen $l$ und $m$ aus $j_\pm$ berechnen.
Durch Ausmultiplizieren erhalten wir
\begin{align*}
l(l+1)
-
m^2-2mj_--j_-^2+m+j_-
&=0
\\
l(l+1)
-
m^2+2mj_+-j_+^2-m+j_+
&=0
\end{align*}
Die Differenz dieser beiden Gleichungen ist
\begin{gather*}
2m(j_++j_-)-(j_+^2-j_-^2)-2m+(j_+-j_-)=0
\\
2m(j_++j_--1)-(j_++j_-)(j_+-j_-)+(j_+-j_-)=0
\\
2m(j_++j_--1)-(j_++j_--1)(j_+-j_-)=0
\\
(2m-(j_+-j_-))(j_++j_--1)=0
\\
m=\frac{j_+-j_-}2.
\end{gather*}
Damit ist bereits gezeigt, dass $m$ halbzahlig sein muss.

Setzen wir jetzt den Ausdruck f"ur $m$ in die urspr"unglichen Gleichungen
(\ref{skript:lmjminus}) und (\ref{skript:lmjplus}) ein, finden wir
\begin{align*}
l(l+1)-\biggl(\frac{j_++j_-}2-1\biggr)\frac{j_++j_-}2&=0,
\\
l(l+1)-\biggl(\frac{j_++j_-}2-1\biggr)\frac{j_++j_-}2&=0.
\end{align*}
Beide Gleichungen sind identisch, es sind quadratische Gleichungen f"ur $l$,
und man kann die beiden L"osungen erraten.
Man findet
\begin{align*}
l_1&=\frac{j_++j_-}2-1,
&
l_2&=-\frac{j_++j_-}2.
\end{align*}
Die zugeh"origen Werte von $l(l+1)$ sind
\begin{align*}
l_1(l_1+1)
&=
\biggl(\frac{j_++j_-}2-1\biggr)\frac{j_++j_-}2
\\
l_2(l_2+1)
&=
-\frac{j_++j_-}2
\biggl(-\frac{j_++j_-}2+1\biggr)
=
\biggl(\frac{j_++j_-}2-1\biggr)
\frac{j_++j_-}2
=l_1(l_1+1),
\end{align*}
sie f"uhren also beide zum gleichen Eigenwert von $\vec L^2$.
Da wir ausserdem gefordert hatten, dass $l$ positiv sein soll,
ist die negative L"osung nicht n"otig.

\begin{satz}
Die Eigenwerte der Operatoren $\vec L^2$ und $L_3$ sind von der Form
$l(l+1)\hbar^2$
bzw.~$m\hbar$.
Die Quantenzahl $l$ ist halbzahlig, also ein nat"urliches Vielfaches
von $\frac12$.
Die m"oglichen Werte von $m$ zu gleichem $l$ sind
\[
-l, -l+1,-l+2,\dots ,l-2,l-1,l.
\]
\end{satz}

\begin{proof}[Beweis]
Die behauptete Halbzahligkeit von $l$ folgt aus der Formel
\[
l=\frac{j_++j_-}2-1.
\]
Da $j_\pm\ge 1$, gilt $l\ge 0$.
Zu vorgegebenem $l$ sind die maximal m"oglichen Werte von $j_\pm$ so,
dass $(1+j_\pm)/2-1=l$, also $j_\pm=2(l+1)-1=2l+1$.
sind die m"oglichen Werte von $m$
\begin{center}
\begin{tabular}{>{$}c<{$}|>{$}c<{$}>{$}c<{$}>{$}c<{$}>{$}c<{$}>{$}c<{$}>{$}c<{$}>{$}c<{$}}
j_+ &  1  &  2   &   3  & \dots & 2l-1 &  2l & 2l+1\\
j_- & 2l+1&  2l  & 2l-1 & \dots &  3   &  2  &   1 \\
\hline
 m  & -l  & -l+1 & -l+2 & \dots &  l-2 & l-1 &   l \\
\end{tabular}
\end{center}
Damit ist auch die Behauptung "uber die m"oglichen Werte von $m$ 
bewiesen.
\end{proof}

\begin{figure}
\centering
\includegraphics{graphics/drehimpuls-3.pdf}
\caption{M"ogliche Werte von $L_3$ sind $\hbar m$ mit Quantenzahlen $m$, wenn
der Drehimpulsbetrag $\hbar^2 l(l+1)$ mit $l=4$ ist.
\label{skript:drehimpulsrange}}
\end{figure}%
\begin{figure}
\centering
\includegraphics{graphics/drehimpuls-4.pdf}
\caption{M"ogliche Kombinationen von $l$ und $m$. Rot eingezeichnet
die Wirkung der Auf- und Absteigeoperatoren $L_\pm$ f"ur die Komponente $L_3$.
\label{skript:drehimpulsspektrum}}
\end{figure}
\begin{figure}
\centering
\includegraphics{graphics/drehimpuls-5.pdf}
\caption{In einem Atom werden nur die hellblau hinterlegten Drehimpulszust"ande
mit ganzzahligem $l$ und $m$ realisiert.
\label{skript:realisiertedrehimpulszustaende}}
\end{figure}

Der Satz erlaubt, dass $l$ keine ganze Zahl ist.
Die m"oglichen Werte sind in den Abbildungen~\ref{skript:drehimpulsrange}
und \ref{skript:drehimpulsspektrum} dargestellt.
Andererseits hat die L"osung des Wasserstoffatoms gezeigt, dass die
Quantenzahlen $m$ und $l$ ganze Zahlen sind.
F"ur Drehimpulszust"ande eines Atomes sind also nur
die in Abbildung~\ref{skript:realisiertedrehimpulszustaende}
mit ganzzahligen Quantenzahlen m"oglich.

Der Spin (Kapitel~\ref{chapter:spin}) vieler Elementarteilchen
hat die gleichen Eigenschaften wie der Drehimpuls, aber mindestens
f"ur einzelne Teilchen wie Elektronen oder Protonen ist $l=\frac12$
und $m=\pm\frac12$.

%\section{Drehimpulszust"ande\label{section:drehimpulszustaende}}
%\rhead{Drehimpulszust"ande}
%Beim harmonischen Oszillator haben wir eine Technik kennengelernt, nicht
%nur die Eigenwerte, sondern auch die Eigenvektoren algebraisch aus dem
%Grundzustand des Systems abzuleiten.
%Ein "ahnliches Vorgehen ist auch hier m"oglich, denn auch hier haben wir
%wieder eine Observable, n"amlich $\vec L^2$, welche nur positive
%Werte annehmen kann.
%Unter den Eigenzust"anden dieser Observablen muss es daher solche mit
%minimalem Eigenwert geben.
%Wir brauchen dann nur noch Operatoren, welche uns erlauben, von einem
%solchen Grundzustand des Drehimpulsoperator zu den Zust"anden 
%h"oheren Drehimpulses aufzusteigen.
%Wenn dieser Plan realsiert werden kann, w"urde sich auch gleich
%eine weitere wichtige Aussage der Quantenmechanik ergeben: der Drehimpuls
%kommt nur in Vielfachen von $\hbar$ vor.
%
%\subsection{Eigenzust"ande von $L_3$}
%\index{Drehimpuls!Eigenzust\"ande von $L_3$}%
%\index{Drehimpuls!Auf- und Absteigeoperatoren}%
%Aus den Operatoren $L_1$ und $L_2$ k"onnen wir "ahnlich wie beim
%harmonischen Oszillator zwei neue Operatoren
%\begin{align*}
%L_+&=L_1+iL_2
%&
%&\text{und}&
%L_-&=L_1-iL_2
%\end{align*}
%konstruieren.
%Auch hier gilt die Einschr"ankung, dass dies keine selbstadjungierten
%Operatoren sind, $L_\pm$ entspricht also nicht einer beobachtbaren
%physikalischen Gr"osse. 
%Immerhin ist $L_+^*=L_-$.
%Diese Operatoren haben die folgenden Vertauschungsrelationen mit
%den Drehimpulskomponenten:
%\begin{equation}
%\begin{aligned} 
%\phantom{ }	% this is a workaround to prevent from the aligned environment
%		% from interpreting the next commutator as an option to the
%		% environment
%[L_\pm, L_1]
%&=
%[L_1,L_1]\pm i[L_2,L_1]
%=
%\pm \hbar L_3
%\\
%[L_\pm, L_2]
%&=
%[L_1,L_2]\pm i[L_2,L_2]
%=
%i\hbar L_3
%\\
%[L_\pm,L_3]
%&=
%[L_1,L_3]\pm i[L_2,L_3]
%=
%-i\hbar L_2
%\mp
%\hbar L_1
%=
%\mp \hbar L_\pm
%\end{aligned}
%\label{skript:lpmlkommutator}
%\end{equation}
%Nehmen wir jetzt an, dass $|\psi\rangle$ ein Eigenzustand von $L_3$ mit
%Eigenwert $l_3$ ist, dann gilt
%\[
%L_3L_+\,|\psi\rangle
%=
%L_+L_3\,|\psi\rangle+\hbar L_+|\psi\rangle
%=
%l_3L_+\,|\psi\rangle+\hbar L_+|\psi\rangle
%=
%(l_3+\hbar)L_+\,|\psi\rangle,
%\]
%der Zustand $L_+\,|\rangle$ ist also wieder ein Eigenzustand von $L_3$,
%allerdings ist der Eigenwert um $\hbar$ erh"oht worden.
%Analog kann man aus
%\[
%L_3L_-\,|\psi\rangle
%=
%L_-L_3\,|\psi\rangle-\hbar L_-\,|\psi\rangle
%=
%(l_3-\hbar)L_-\,|\psi\rangle
%\]
%ablesen,  dass $L_-\,|\psi\rangle$ ein Eigenzustand von $L_3$ ist, dessen
%Eigenwert gegen"uber dem von $|\psi\rangle$ um $\hbar$ verringert
%worden ist.
%Die Operatoren $L_+$ und $L_-$ haben also genau die Eigenschaften,
%die wir bei den Auf- und Absteigeoperatoren beim harmonischen Oszillator
%kennengelernt hatten.
%
%Aus (\ref{skript:lpmlkommutator}) und aus
%Hilfssatz~\ref{skript:commutatora2b} k"onnen wir jetzt die
%Vertauschungsrelationen mit den Quadraten der Drehimpulskomponenten
%und mit $\vec L^2$ berechnen:
%\begin{align*}
%\\
%[L_\pm,L_1^2]
%&=
%\{ [L_\pm, L_1], L_1 \}
%=
%\{ \pm \hbar L_3, L_1 \}
%=
%\pm \hbar \{ L_1, L_3 \}
%\\
%[L_\pm,L_2^2]
%&=
%\{ [L_\pm, L_2], L_2 \}
%=
%\{ i\hbar L_3, L_2 \}
%=
%i\hbar \{ L_2, L_3 \}
%\\
%[L_\pm,L_1^2 + L_2^2]
%&=
%\pm \hbar \{ L_1, L_3 \}
%+
%i\hbar \{ L_2, L_3 \}
%=
%\pm \hbar \{ L_1 \mp iL_2, L_3 \}
%=
%\pm \hbar\{L_\pm, L_3\}
%\\
%[L_\pm,L_3^2]
%&=
%\{ [L_\pm, L_3], L_3 \}
%=
%\mp \{ i\hbar L_\pm, L_3 \}
%=
%\mp i\hbar \{ L_\pm, L_3 \}
%\\
%\Rightarrow\qquad [L_\pm,\vec L^2]
%&=0
%\end{align*}
%Die Operatoren $L_\pm$ vertauschen also mit $\vec L^2$.
%Ist $|\psi\rangle$ ein Eigenzustand von $\vec L^2$ mit Eigenwert $\lambda$,
%dann gilt auch 
%\begin{align*}
%\vec L^2\,|\psi\rangle&=\lambda\,|\psi\rangle
%\\
%L_\pm\vec L^2\,|\psi\rangle&=
%\vec L^2L_\pm\,|\psi\rangle=
%\lambda L_\pm\,|\psi\rangle
%\end{align*}
%also ist  auch $L_\pm\,|\psi\rangle$ ein Eigenzustand von $\vec L^2$ mit
%dem gleichen Eigenwert.
%
%%Gemeinsame Eigenzust"ande $|\lambda,m\rangle$ von $\vec L^2$ und $L_3$
%%mit Eigenwerten $\lambda$ und $m$ k"onnen also mit den Operatoren 
%%$L_\pm$ in neue Eigenzust"ande mit gleichem $\lambda$, aber verschiedenem
%%$m$ umgewandelt werden.
%%\begin{align*}
%%L_+|\lambda,m\rangle&=|\lambda,m+1\rangle
%%L_-|\lambda,m\rangle&=|\lambda,m-1\rangle
%%\end{align*}
%%
%\subsection{Beziehungen zwischen den Eigenwerten}
%\index{Drehimpuls!Eigenwerte}
%Die Drehimpulskomponenten $L_3$ ist in der klassischen Mechanik kleiner
%als der Drehimpulsbetrag, und wir m"ochten nachpr"ufen, dass dies auch
%in der Quantenmechanik gilt.
%Dazu berechnen wir zun"achst die Produkte der Operatoren $L_+$ und $L_-$
%\begin{align*}
%L_+L_-
%&=
%L_1^2+L_2^2 +iL_2L_1-iL_1L_2=L_1^2+L_2^2 -i[L_1,L_2]=L_1^2+L_2^2+\hbar L_3
%\\
%L_-L_+
%&=
%L_1^2  + L_2^2 +i[L_1,L_2]=L_1^2+L_2^2-\hbar L_3
%\\
%[L_+,L_-]
%&=
%2\hbar L_3
%\\
%{\textstyle \frac12}(L_+L_-+L_-L_+)&=L_1^2+L_2^2,
%\\
%\vec L^2
%&=
%{\textstyle\frac12}(L_+L_-+L_-L_+)+L_3^2.
%\end{align*}
%Sei $|\lambda,l_3\rangle$ ein gemeinsamer Eigenzustand von $\vec L^2$
%und $L_3$, dann muss gelten
%\begin{align*}
%\langle \lambda,l_3|\,\vec L^2\,|\lambda,l_3\rangle
%&=
%\langle \lambda,l_3|
%\,
%{\textstyle\frac12}(L_+L_-+L_-L_+)
%\,
%|\lambda,l_3\rangle
%+
%\langle \lambda,l_3|
%\,
%L_3^2
%\,
%|\lambda,l_3\rangle
%\\
%\lambda
%&=
%\langle \lambda,l_3|
%\,
%{\textstyle\frac12}(L_+L_-+L_-L_+)
%\,
%|\lambda,l_3\rangle
%+
%l_3^2
%\end{align*}
%Der erste Term auf der rechten Seite ist aber auf jeden Fall positiv,
%wie man sich durch folgende Rechnung "uberzeugen kann:
%\[
%\langle \lambda,l_3|
%\,
%{\textstyle\frac12}(L_+L_-+L_-L_+)
%\,
%|\lambda,l_3\rangle
%=
%\langle \lambda,l_3|
%\,
%{\textstyle\frac12}(L_-^*L_-+L_+^*L_+)
%\,
%|\lambda,l_3\rangle
%=
%\frac12\|\,L_-\,|\lambda,l_3\rangle\|^2
%+
%\frac12\|\,L_+\,|\lambda,l_3\rangle\|^2
%\le 1.
%\]
%\begin{figure}
%\centering
%\includegraphics{graphics/drehimpuls-3.pdf}
%\caption{M"ogliche Werte von $L_3$ sind die Quantenzahlen $m$, wenn
%der Drehimpulsbetrag $h\hbar^2 \frac{l}2(\frac{l}2+1)$ mit $l=4$ ist.
%\label{skript:drehimpulsrange}}
%\end{figure}%
%Die Normquadrate auf der rechten Seite sind entweder $1$,
%wenn $L_\pm\,|\lambda,l_3\rangle$, oder 0, wenn $L_\pm\,|\lambda,l_3\rangle$
%der Nullvektor ist.
%\begin{figure}
%\centering
%\includegraphics{graphics/drehimpuls-1.pdf}
%\caption{M"ogliche Kombinationen von $\lambda$ und $l_3$. Rot eingezeichnet
%die Wirkung der Auf- und Absteigeoperatoren $L_\pm$ f"ur die Komponente $L_3$.
%\label{skript:drehimpulsspektrum}}
%\end{figure}
%Weiter k"onnen wir daraus ableiten, dass 
%\[
%\lambda\ge l_3^2\ge 0,
%\]
%der Eigenwert $l_3$ kann also nicht beliebig gross werden, es muss immer
%gelten
%\[
%-\sqrt{\lambda}\le l_3\le \sqrt{\lambda},
%\]
%wie auch die Abbildung~\ref{skript:drehimpulsrange} suggeriert.
%Je gr"osser $\lambda$ ist, desto mehr m"ogliche Eigenzust"ande von $L_3$
%gibt es.
%Dies wird auch die Abbildung~\ref{skript:drehimpulsspektrum} veranschaulicht,
%Punkte entsprechen zul"assigen Kombinationen von $\lambda$ und $l_3$.
%F"ur $\lambda=0$ kann es nur einen einzigen Eigenzustand von $L_3$ geben
%mit Eigenwert $0$.
%
%\subsection{Eigenzust"ande von $\vec L^2$}
%Der Drehimpulsbetrag $\vec L^2$ und die Drehimpulskomponenten $L_3$ 
%k"onnen also gleichzeitig bekannt sein, es gibt gemeinsame Eigenzust"ande,
%und wir sind bereits in der Lage, innerhalb der Eigenzust"ande mit einem
%bestimmten Wert f"ur $\vec L^2$ zwischen verschiedene
%Eigenwerten von $L_3$ hin- und hernavigieren.
%
%\begin{figure}
%\centering
%\includegraphics{graphics/drehimpuls-2.pdf}
%\caption{M"ogliche Drehimpulswerte in $a_1$ und $a_2$ Koordinaten und
%Zusammensetzung der Operatoren $L_+$ und $L_-$ aus den Operatoren
%$a_1$ und $a_2$.
%\label{skript:drehimpulsspektruma}}
%\end{figure}
%Beim harmonischen Oszillator konnten wir aus dem Produkt $a^+a$
%den Hamilton-Operator rekonstruieren und daraus ableiten, dass es einen
%Zustand minimaler Energie gibt.
%Wir versuchen dasselbe f"ur den Drehimpulsoperator.
%Das Bild~\ref{skript:drehimpulsspektrum} suggeriert aber, dass es nicht
%unbedingt einen zweiten Operator geben kann, der die
%Auf- und Absteigeschritte in $\lambda$-Richtung implementiert.
%Es scheint eher, dass wir in Abbildung~\ref{skript:drehimpulsspektruma}
%das falsche Koordinatensystem gew"ahlt haben.
%
%\subsubsection{Auf- und Absteigeoperatoren}
%So etwas kann nat"urlich nur funktionieren, wenn das Spektrum der
%m"oglichen Eigenzust"ande tats"achlich ungef"ahr so aussieht wie in
%Abbildung~\ref{skript:drehimpulsspektrum}, wir haben da einige Erkenntniss
%vorweggenommen.
%Wir nehmen also an, dass es zwei Paare von Auf- und Absteigeoperatoren
%$a_1^+$ und $a_1$ und $a_2^+$ und $a_2$ gibt, die den Vertauschungsrelationen
%\begin{align*}
%[a_1,a_1^+]&=1&[a_1,a_2^+]&=0&[a_1,a_2]&=0\\
%[a_2,a_2^+]&=1&[a_2,a_1^+]&=0&[a_1^+,a_2^+]&=0
%\end{align*}
%gen"ugen.
%Die beiden S"atze von Operatoren vertauschen also untereinander
%vollst"andig, nur die Operatorn $a_i$ und $a_i^+$ vertauschen nicht.
%Wie beim harmonischen Oszillator bilden wir die Operatoren
%\[
%N_i=a_i^+a_i
%\]
%mit den Vertauschungsrelationen
%\begin{align*}
%[N_i,a_i^+]
%&=
%a_i^+a_ia_i^+-a_i^+a_i^+a_i
%=
%a_i^+[a_i,a_i^+]
%=
%a_i^+,
%\\
%[N_i,a_i]
%&=
%a_i^+a_ia_i- a_ia_i^+a_i
%=
%[a_i^+,a_i]a_i
%=
%-a_i.
%\end{align*}
%Da auch die Operatoren $N_i$ vertauschen, gibt es eine gemeinsame
%Eigenvektorbasis. Ist $|n_1,n_2\rangle$ so ein Eigenvektor, dann muss
%gelten
%\begin{align*}
%N_i\,|n_1,n_2\rangle
%&=
%n_i\,|n_1,n_2\rangle.
%\\
%N_ia_i^+\,|n_1,n_2\rangle
%&=
%a_i^+N_i\,|n_1,n_2\rangle+a_i^+|n_1,n_2\rangle
%=
%(n_i+1)a_i^+\,|n_1,n_2\rangle
%\\
%N_ia_i\,|n_1,n_2\rangle
%&=
%a_iN_i\,|n_1,n_2\rangle-a_i\,|n_1,n_2\rangle
%=
%(n_i-1)a_i\,|n_1,n_2\rangle
%\end{align*}
%Insbesondere sind $a_i^+\,|n_1,n_2\rangle$ und $a_i\,|n_1,n_2\rangle$
%wieder Eigenvektoren von $N_i$, allerdings mit Eigenwert $n_i+1$.
%
%\subsubsection{Eigenzust"ande}
%Da die Operatoren $N_i$ positiv sind, kann man mit dem Absteigeoperator
%nicht beliebig weit hinunter absteigen, irgendwann muss der Nullvektor
%entstehen.
%Das heisst aber auch, dass $0$ ein Eigenwert sein muss, und dass alle
%Eigenwerte ganzzahlig sind.
%Der Zustand minimalen Eigenwertes hat als $n_1=n_2=0$, wir bezeichnen
%ihn als $|0,0\rangle$.
%Aus diesem Zustand k"onnen wir die "ubrigen Eigenzust"ande durch anwenden
%der Aufsteigeoperatoren konstruieren:
%\[
%(a_1^+)^{n_1} (a_2^+)^{n_2} \,|0,0\rangle
%\]
%Es ist allerdings nicht klar, dass diese Zustandsvektoren normiert sind.
%Dazu rechnen wir die Norm nach:
%\begin{align*}
%\|
%a_i^+\,|n_1,n_2\rangle
%\|^2
%=
%\langle n_1,n_2|\,a_i a_i^+\,|n_1,n_2\rangle
%&=
%\langle n_1,n_2|\,a_i^+a_i\,|n_1,n_2\rangle
%+
%\langle n_1,n_2|\,[a_i,a_i^+]\,|n_1,n_2\rangle
%\\
%&=
%\langle n_1,n_2|\,N_i\,|n_1,n_2\rangle
%+
%\langle n_1,n_2|n_1,n_2\rangle
%=(n_i+1)
%\end{align*}
%Damit 
%$(a_1^+)^{n_1} (a_2^+)^{n_2}\, |0,0\rangle$
%normiert ist, dann muss der Faktor $n_i+1$ komponsiert werden,
%der bei jeder Anwendung von $a_i^+$ hinzukommt.
%Der normierte Zustandsvektor ist dann
%\[
%|n_1,n_2\rangle
%=
%\frac1{\sqrt{n_1!n_2!}} (a_1^+) (a_2^+)  \, |0,0\rangle.
%\]
%
%\subsubsection{Rekonstruktion der Drehimpulsoperatoren}
%Wir m"ussen jetzt die Drehimpulsoperatoren aus den $a_i$ und $a_i^+$
%rekonstruieren.
%Dazu k"onnen wir die Abbildung~\ref{skript:drehimpulsspektruma} heranziehen.
%Der Operator $L_+$ f"ugt eine Einheit in $a_1$-Richtung hinzu und
%entfernt eine in Richtung $a_2$, also muss er proportional zu
%$a_1^+a_2$ sein.
%Analog muss $L_-$ proportional zu $a_1a_2^+$ sein. Wir setzen also:
%\begin{align*}
%L_+
%&=
%\hbar a_1^+a_2
%&
%L_1
%&=
%\frac12(L_++L_-)
%=
%\frac{\hbar}2(a_1^+a_2+a_1a_2^+)
%\\
%L_-
%&=
%\hbar a_2^+a_1
%&
%L_2
%&=
%\frac1{2i}(L_+-L_-)
%=
%\frac{\hbar}{2i}(a_1^+a_2-a_1a_2^+)
%\\
%&&
%L_3
%&=
%\frac{\hbar}2(a_1^+a_1+a_2^+a_2)
%=
%\frac{\hbar}2(N_1-N_2)
%\end{align*}
%Wir wissen bereits, wie der Drehimpulsbetrag in den Operatoren 
%$L_\pm$ und $L_3$ ausgedr"uckt werden kann:
%\begin{align*}
%\vec L^2
%&=
%\frac12(L_+L_-+L_-L_+)+L_3^2
%=
%\frac{\hbar^2}2(a_1^+a_2a_2^+a_1+a_2^+a_1a_1^+a_2)+\frac{\hbar^2}{4}(N_1-N_2)^2
%\\
%&=
%\frac{\hbar^2}4(
%2N_1(N_2+1)+2(N_1+1)N_2
%+N_1^2-2N_1N_2+N_2^2
%)
%\\
%&=
%\frac{\hbar^2}{4}(
%2N_1+2N_2
%+
%N_1^2+N_2^2
%+2N_1N_2
%+
%)
%\\
%&=
%\hbar^2\frac{N_1+N_2}2\biggl(\frac{N_1+N_2}2+1\biggr)
%=
%\hbar^2\frac{N}2\biggl(\frac{N}2+1\biggr).
%\end{align*}
%Darin setzen wir $N=N_1+N_2$. Da die Eigenwerte der Operatoren $N_i$
%nur nat"urliche Zahlen als Eigenwerte haben, k"onnen wir jetzt schliessen,
%dass die Eigenwerte von $\vec L^2$ von der Form $\hbar j(j+1)$ sind,
%wobei $j$ halbzahlige Werte annehmen muss.
%Und die Eigenwerte von $L_3$ sind von der Form $\hbar m$, wobei auch
%$m$ halbzahlig ist.
%Mit den Eigenwerten $n_i$ der Operatoren $N_i$ kann man die sogenannten
%Quantenzahlen $l$ und $m$ ausdr"ucken als
%\begin{align*}
%l
%&=
%\frac12(n_1+n_2)
%&
%m
%&=
%\frac12(n_1-n_2).
%\end{align*}
%Umgekehrt kann man die Eigenzust"ande jetzt mit den Quantenzahlen $l$ und $m$
%schreiben als
%\begin{align*}
%|l,m\rangle
%=
%\frac1{\sqrt{(l+m)!(l-m)!}}
%(a_1^+)^{\frac12(n_1+n_2)}
%(a_2^+)^{\frac12(n_1-n_2)}
%\,|0,0\rangle
%\end{align*}
%mit den Eigenwertgleichungen
%\begin{align*}
%\vec L^2\,|l,m\rangle&=\hbar l(l+1)\,|l,m\rangle,
%&
%L_3\,|l,m\rangle&=\hbar m\,|l,m\rangle.
%\end{align*}
%
%\subsubsection{Rechnungen}
%Bis jetzt ist das einfach nur ein algebraischer Versuch, die
%Drehimpulsoperatoren in den Griff zu bekommen und insbesondere
%das Bild~\ref{skript:drehimpulsspektrum} zu rekonstruieren.
%Erfolgreich sind wir erst, wenn diese neuen Operatoren die gleichen
%Vertauschungsrelationen der Drehimpulsoperatoren haben.
%Wir berechnen also die Vertauschungsrelationen, und beginnen dazu mit
%den Vertauschungsrelationen zwischen $L_\pm$ und $N_i$.
%\begin{align*}
%[L_+,N_1]
%&=
%\hbar[a_1^+a_2,a_1^+a_1]
%=
%\hbar( a_1^+a_2 a_1^+a_1 - a_1^+a_1 a_1^+a_2)
%\\
%&=
%\hbar a_1^+( a_1^+a_1 - a_1 a_1^+)a_2
%=
%\hbar a_1^+[ a_1^+,a_1]a_2
%=
%-\hbar a_1^+a_2
%=-L_+
%\\
%[L_+,N_2]
%&=
%\hbar[a_1^+a_2,a_2^+a_2]
%=
%\hbar( a_1^+a_2 a_2^+a_2 - a_2^+a_2 a_1^+a_2)
%\\
%&=
%\hbar a_1^+( a_2 a_2^+ - a_2^+a_2)a_2
%=
%\hbar a_1^+[a_2,a_2^+]a_2
%=
%\hbar a_1^+a_2
%=
%L_+
%\\
%[L_-,N_1]
%&=
%\hbar[a_2^+a_1,a_1^+a_1]
%=
%\hbar( a_2^+a_1 a_1^+a_1 - a_1^+a_1 a_2^+a_1)
%\\
%&=
%\hbar a_2^+(a_1 a_1^+ - a_1^+a_1)a_1
%=
%\hbar a_2^+[a_1,a_1^+]a_1
%=
%\hbar a_2^+a_1
%=
%L_-
%\\
%[L_-,N_2]
%&=
%\hbar[a_2^+a_1,a_2^+a_2]
%=
%\hbar( a_2^+a_1 a_2^+a_2 - a_2^+a_2 a_2^+a_1)
%\\
%&=
%\hbar a_2^+(a_2^+a_2 - a_2 a_2^+) a_1
%=
%\hbar a_2^+[a_2^+,a_2] a_1
%=
%-\hbar a_2^+a_1
%=-L_-
%\end{align*}
%Jetzt k"onnen wir bereits den Kommutator von $L_\pm$ mit $L_3$
%berechnen:
%\begin{align*}
%[L_+,L_3]
%&=
%\frac{\hbar}2[L_+,N_1]
%-
%\frac{\hbar}2[L_+,N_2]
%=
%\frac{\hbar}2((-L_+)-L_+)=-\hbar L_+
%\\
%[L_-,L_3]
%&=
%\frac{\hbar}2[L_-,N_1]
%-
%\frac{\hbar}2[L_-,N_2]
%=
%\frac{\hbar}2(L_--(-L_-))
%=\hbar L_-
%\end{align*}
%Daraus ergibt sich auch der Kommutator von $L_1$ und $L_2$ mit $L_3$:
%\begin{align*}
%[L_1,L_3]
%&=
%\frac12[L_++L_-,L_3]
%=
%\frac{\hbar}2(-L_+-L_-)
%=
%-i\hbar L_2
%\\
%[L_2,L_3]
%&=
%\frac1{2i}[L_+-L_-,L_3]
%=
%-i\frac{\hbar}{2}(-L_+-L_-)
%=
%i\hbar L_1
%\end{align*}
%Der Kommutator der Operatoren $L_\pm$ ist
%\begin{align*}
%[L_+,L_-]
%&=
%\hbar^2(a_1^+a_2a_2^+a_1-a_2^+a_1a_1^+a_2)
%=
%\hbar^2(
%a_1^+
%a_1
%a_2
%a_2^+
%-
%a_1
%a_1^+
%a_2^+
%a_2
%)
%\\
%&=
%\hbar^2
%(
%(N_1+1)N_2
%-
%N_1(N_2+1)
%)
%\\
%&=
%\hbar^2
%(
%N_2
%-
%N_1
%)
%=2\hbar L_3
%\end{align*}
%Jetzt k"onnen wir den Kommentator von $L_\pm$ mit den
%Drehimpulskomponenten berechnen:
%\begin{align*}
%% [L_+,L_1]
%[L_+,L_1]
%&=
%\frac12[L_+,L_++L_-]
%=
%\frac12[L_+,L_-]
%=
%\hbar L_3
%\\
%% [L_+,L_2]
%[L_+,L_2]
%&=
%\frac1{2i}[L_+,L_+-L_-]
%=
%-\frac1{2i}[L_+,L_-]
%=
%i\hbar L_3
%\\
%% [L_-,L_1]
%[L_-,L_1]
%&=
%\frac1{2}
%[L_-,L_++L_-]
%=
%\frac12[L_-,L_+]
%=
%-\hbar L_3
%\\
%% [L_-,L_2]
%[L_-,L_2]
%&=
%\frac1{2i}[L_-,L_+-L_-]
%=
%-\frac1{2i}[L_+,L_-]
%=
%i\hbar L_3
%\end{align*}
%Die Kommutatoren der Drehimpulskomponenten untereinander sind damit
%\begin{align*}
%[L_1,L_2]
%&=
%\frac12[L_++L_-,L_2]
%=
%\frac12(i\hbar L_3+i\hbar L_3)
%=
%i\hbar L_3.
%\end{align*}
%Die Operator $L_1$, $L_2$ und $L_3$, die aus $a_i$ und $a_i^+$
%aufgebaut wurden, erf"ullen genau die  Vertauschungsrelationen
%der Drehimpulskomponenten.
%Damit ist nachgewiesen, dass sich die Drehimpulskomponenten
%durch die Operatoren $a_i^+$ und $a_i$ auf die gezeigte Art
%darstellen lassen.

\section{Drehimpuls in Ortsdarstellung\label{section:drehimpulsortsdarstellung}}
\rhead{Ortsdarstellung}
Bisher haben wir in diesem Kapitel die Drehimpulsoperatoren ganz abstrakt
behandelt. 
Das Wasserstoffatom haben wir allerdings in der Ortsdarstellung 
gerechnet, und dort ebenfalls Quantenzahlen $l$ und $m$ gefunden.
Erst wenn wir die Drehimpulsoperatoren ebenfalls in die Ortsdarstellung
umrechnen, k"onnen wir sie als Teile des Hamilton-Operators des
Wasserstoffatoms identifizieren.
Das wird uns auch erlauben nachzurechnen, dass der Drehimpuls eine
Erhaltungsgr"osse ist.

\subsection{Drehimpulsoperatoren in kartesischen Koordinaten}
In der Ortsdarstellung m"ussen wir die Impulsoperatoren durch
Ableitungsoperatoren ersetzen:
\index{Drehimpuls!Operatoren in kartesischen Koordinaten}
\begin{align*}
L_1
&=
\frac{\hbar}{i}\biggl(
x_2\frac{\partial}{\partial x_3}
-
x_3\frac{\partial}{\partial x_2}
\biggr)
=
\frac{\hbar}{i}\biggl(
y\frac{\partial}{\partial z}
-
z\frac{\partial}{\partial y}
\biggr),
\\
L_2
&=
\frac{\hbar}{i}
\biggl(
x_3\frac{\partial}{\partial x_1}
-
x_1\frac{\partial}{\partial x_3}
\biggr)
=
\frac{\hbar}{i}
\biggl(
z\frac{\partial}{\partial x}
-
x\frac{\partial}{\partial z}
\biggr),
\\
L_3
&=
\frac{\hbar}{i}
\biggl(
x_1\frac{\partial}{\partial x_2}
-
x_2\frac{\partial}{\partial x_1}
\biggr)
=
\frac{\hbar}{i}
\biggl(
x\frac{\partial}{\partial y}
-
y\frac{\partial}{\partial x}
\biggr).
\end{align*}
Als Beispiel rechnen wir die Vertauschungsrelationen nach:
\begin{align*}
[L_1,L_2]
&=
-\hbar^2\biggl[
y\frac{\partial}{\partial z}
-
z\frac{\partial}{\partial y},
z\frac{\partial}{\partial x}
-
x\frac{\partial}{\partial z}
\biggr]
\\
&=
-\hbar^2\biggl(
\biggl(
y\frac{\partial}{\partial z}
-
z\frac{\partial}{\partial y}
\biggr)
\biggl(
z\frac{\partial}{\partial x}
-
x\frac{\partial}{\partial z}
\biggr)
-
\biggl(
z\frac{\partial}{\partial x}
-
x\frac{\partial}{\partial z}
\biggr)
\biggl(
y\frac{\partial}{\partial z}
-
z\frac{\partial}{\partial y}
\biggr)
\biggr)
\\
&=
-i\hbar\frac{\hbar}{i}\biggl(
y\frac{\partial}{\partial x}+yz\frac{\partial^2}{\partial x\,\partial z}
-xy\frac{\partial^2}{\partial z^2}
-z^2\frac{\partial^2}{\partial y\,\partial x}
+zx\frac{\partial^2}{\partial y\,\partial z}
\\
&\qquad\qquad
-
zy\frac{\partial^2}{\partial x\,\partial z}
+z^2\frac{\partial^2}{\partial x\,\partial y}
+xy\frac{\partial^2}{\partial z^2}
-x\frac{\partial}{\partial y}-xz\frac{\partial^2}{\partial z\,\partial y}
\biggr)
\\
&=
i\hbar\frac{\hbar}{i}\biggl(
x\frac{\partial}{\partial y}-y\frac{\partial}{\partial x}
\biggr)
=i\hbar L_3
\end{align*}
Die Drehimpulsoperatoren in Ortsdarstellung erf"ullen also wie erwartet
die Vertauschungsrelationen, die wir f"ur die abstrakten Drehimpulsoperatoren
kennengelernt haben.

\subsection{Kugelkoordinaten}
In Anhang~\ref{chapter:kugelkoordinaten} wurden die Umrechnungen f"ur
einen beliebigen Differentialoperator in ein anderes Koordinatensystem
bereits zusammengestellt, in diesem Abschnitt wollen wir dies verwenden
um die Drehimpulskomponenten und den Drehimpulsbetrag in Kugelkoordinaten
auszudr"ucken.
\subsubsection{Drehimpulskomponenten}
\index{Drehimpuls!Operatoren in Kugelkoordinaten}
Wir verwenden jetzt die Formeln f"ur die kartesischen Ableitungsoperatoren
in Kugelkoordinaten, um die Drehimpulsoperatoren auszudr"ucken.
Wir beginnen mit $L_3$
\begin{align}
\frac{i}{\hbar}
L_3
=
x\frac{\partial}{\partial y}
-
y\frac{\partial}{\partial x}
&=
r\sin\vartheta\cos\varphi
\biggl(
\sin\vartheta\sin\varphi
\frac{\partial}{\partial r}
+
\frac{\cos\vartheta\sin\varphi}{r}
\frac{\partial}{\partial\vartheta}
+
\frac{\cos\varphi}{r\sin\vartheta}
\frac{\partial}{\partial\varphi}
\biggr)
\notag
\\
&\qquad
-
r\sin\vartheta\sin\varphi
\biggl(
\sin\vartheta\cos\varphi
\frac{\partial}{\partial r}
+
\frac{\cos\vartheta\cos\varphi}{r}
\frac{\partial}{\partial\vartheta}
-
\frac{\sin\varphi}{r\sin\vartheta}
\frac{\partial}{\partial\varphi}
\biggr)
=
\frac{\partial}{\partial\varphi}
\label{skript:l3spherical}
\end{align}
Die anderen Komponenten sind weniger wichtig, wir rechnen sie trotzdem
aus:
\begin{align*}
\frac{i}{\hbar}L_1
=
y\frac{\partial}{\partial z}-z\frac{\partial}{\partial y}
&=
r\sin\vartheta\sin\varphi
\biggl(
\cos\vartheta
\frac{\partial}{\partial r}
-
\frac{\sin\vartheta}{r}
\frac{\partial}{\partial\vartheta}
\biggr)
\\
&\qquad
-
r\cos\vartheta
\biggl(
\sin\vartheta\sin\varphi
\frac{\partial}{\partial r}
+
\frac{\cos\vartheta\sin\varphi}{r}
\frac{\partial}{\partial\vartheta}
+
\frac{\cos\varphi}{r\sin\vartheta}
\frac{\partial}{\partial\varphi}
\biggr)
\\
&=
-\sin\varphi\frac{\partial}{\partial\vartheta}
-\cos\varphi\cot\vartheta\frac{\partial}{\partial\varphi}
\\
\frac{i}{\hbar}L_2
=
z\frac{\partial}{\partial x}-x\frac{\partial}{\partial z}
&=
r\cos\vartheta
\biggl(
\sin\vartheta\cos\varphi
\frac{\partial}{\partial r}
+
\frac{\cos\vartheta\cos\varphi}{r}
\frac{\partial}{\partial\vartheta}
-
\frac{\sin\varphi}{r\sin\vartheta}
\frac{\partial}{\partial\varphi}
\biggr)
\\
&\qquad
-
r\sin\vartheta\cos\varphi
\biggl(
\cos\vartheta
\frac{\partial}{\partial r}
-
\frac{\sin\vartheta}{r}
\frac{\partial}{\partial\vartheta}
\biggr)
\\
&=
\cos\varphi\frac{\partial}{\partial\vartheta}
-\sin\varphi\cot\vartheta\frac{\partial}{\partial\varphi}
\end{align*}
Als Beispiel rechnen wir nach, dass die Vertauschungsrelationen der
Drehimpulsoperatoren erf"ullt sind
\begin{align*}
[L_2,L_3]
&=-\hbar^2\biggl[
\cos\varphi\frac{\partial}{\partial\vartheta}
-\sin\varphi\cot\vartheta\frac{\partial}{\partial\varphi}
,\frac{\partial}{\partial\varphi}
\biggr]
\\
&=
-i\hbar\frac{\hbar}{i}\biggl(
\biggl(
\cos\varphi\frac{\partial}{\partial\vartheta}
-\sin\varphi\cot\vartheta\frac{\partial}{\partial\varphi}
\biggr)
\frac{\partial}{\partial\varphi}
-
\frac{\partial}{\partial\varphi}
\biggl(
\cos\varphi\frac{\partial}{\partial\vartheta}
-\sin\varphi\cot\vartheta\frac{\partial}{\partial\varphi}
\biggr)
\biggr)
\\
&=
-i\hbar\frac{\hbar}{i}\biggl(
\cos\varphi\frac{\partial^2}{\partial\vartheta\,\partial\varphi}
-\sin\varphi\cot\vartheta\frac{\partial^2}{\partial\varphi^2}
-
\biggl(
-\sin\varphi\frac{\partial}{\partial\vartheta}
-\cos\varphi\cot\vartheta\frac{\partial}{\partial\varphi}
\biggr)
\\
&\qquad\qquad
-
\cos\varphi\frac{\partial^2}{\partial\vartheta\,\partial\varphi}
+\sin\varphi\cot\vartheta\frac{\partial^2}{\partial\varphi^2}
\biggr)
\\
&=
i\hbar\frac{\hbar}{i}\biggl(
-
\sin\varphi\frac{\partial}{\partial\vartheta}
-
\cos\varphi\cot\vartheta\frac{\partial}{\partial\varphi}
\biggr)
=i\hbar L_1
\end{align*}

\subsubsection{Auf- und Absteigeoperatoren}
Die Operatoren $L_+$ und $L_-$ k"onnen nat"urlich auch in Kugelkoordinaten
dargestellt werden.
\begin{align*}
L_+
&=
L_1+iL_2
\\
&=
\frac{\hbar}{i}\biggl(
-\sin\varphi\frac{\partial}{\partial\vartheta}
-\cos\varphi\cot\vartheta\frac{\partial}{\partial\varphi}
\biggr)
+\hbar\biggl(
\cos\varphi\frac{\partial}{\partial\vartheta}
-\sin\varphi\cot\vartheta\frac{\partial}{\partial\varphi}
\biggr)
\\
&=
\hbar\biggl(
i\sin\varphi\frac{\partial}{\partial\vartheta}
+i\cos\varphi\cot\vartheta\frac{\partial}{\partial\varphi}
+
\cos\varphi\frac{\partial}{\partial\vartheta}
-\sin\varphi\cot\vartheta\frac{\partial}{\partial\varphi}
\biggr)
\\
&=\hbar e^{i\varphi}\biggl(
\frac{\partial}{\partial\vartheta}
+
i\cot\vartheta\frac{\partial}{\partial\varphi}
\biggr)
\\
L_-
&=
L_1-iL_2
\\
&=
\frac{\hbar}{i}\biggl(
-\sin\varphi\frac{\partial}{\partial\vartheta}
-\cos\varphi\cot\vartheta\frac{\partial}{\partial\varphi}
\biggr)
-\hbar\biggl(
\cos\varphi\frac{\partial}{\partial\vartheta}
-\sin\varphi\cot\vartheta\frac{\partial}{\partial\varphi}
\biggr)
\\
&=
\hbar\biggl(
i\sin\varphi\frac{\partial}{\partial\vartheta}
+
i\cos\varphi\cot\vartheta\frac{\partial}{\partial\varphi}
-
\cos\varphi\frac{\partial}{\partial\vartheta}
+\sin\varphi\cot\vartheta\frac{\partial}{\partial\varphi}
\biggr)
\\
&=
\hbar e^{-i\varphi}
\biggl(
-
\frac{\partial}{\partial\vartheta}
+i
\cot\vartheta\frac{\partial}{\partial\varphi}
\biggr)
\end{align*}
Diese Operatoren erlauben uns, aus einem Eigenzustand von $L_3$
einen neuen Eigenzustand zu berechnen, dessen $m$-Quantenzahl 
um $1$ h"oher oder tiefer sind.
Der Operator $L_3$ in der Ortsdarstellung ist
\[
L_3=\frac{\hbar}{i}\frac{\partial}{\partial\varphi},
\]
Bei der Analyse des Wasserstoffatoms hatten wir die Eigenfunktionen
als Produkt $\Theta(\vartheta) e^{im\varphi}$ angesetzt,  und es ist
klar, dass eine Funktion mit dieser $\varphi$-Abh"angigkeit ein
Eigenvektor von $L_3$ mit Eigenwert $\hbar m$ ist. Durch die Anwendung
von $L_+$ sollte daraus eine Funktion werden, deren $\varphi$-Abh"angigkeit
zu einer Eigenfunktion von $L_3$ mit Eigenwert $\hbar(m+1)$ bzw.~$\hbar(m-1)$
passt:
\begin{equation*}
\langle x|n,l,m\rangle
=
R(r)\Theta(\vartheta)e^{im\varphi}
\quad\Rightarrow\quad
\left\{
\begin{aligned}
L_+
\langle x|n,l,m\rangle
&=
f_+(r,\vartheta)
e^{i(m+1)\varphi}
\\
L_-
\langle x|n,l,m\rangle
&=
f_-(r,\vartheta)
e^{i(m-1)\varphi}
\end{aligned}
\right.
\end{equation*}
Die Auf- und Absteigeoperatoren in Kugelkoordinaten k"onnen zum
Beispiel dazu verwendet werden, aus einer L"osung der Schr"odingergleichung
f"ur das Wasserstoff-Atom f"ur $m=0$ die anderen f"ur alle zul"assigen
Werte von $m$ zu berechnen.

\subsubsection{Drehimpulsbetrag}
F"ur die Berechnung des Drehimpulsbetrages m"ussen wir uns zun"achst
auf $L_1^2+L_2^2$ konzentrieren, da die letzte Komponenten gem"ass
(\ref{skript:l3spherical}) sehr einfach ist.
\begin{align*}
L_1^2
&=-\hbar^2
\biggl(
-\sin\varphi\frac{\partial}{\partial\vartheta}
-\cos\varphi\cot\vartheta\frac{\partial}{\partial\varphi}
\biggr)
\biggl(
-\sin\varphi\frac{\partial}{\partial\vartheta}
-\cos\varphi\cot\vartheta\frac{\partial}{\partial\varphi}
\biggr)
\\
&=
-\hbar^2\biggl(
\sin^2\varphi\frac{\partial^2}{\partial \vartheta^2}
+
\sin\varphi\cos\varphi \frac{1}{\sin^2\vartheta}
\frac{\partial}{\partial\varphi}
-
\sin\varphi\cos\varphi\cot\vartheta\frac{\partial^2}{\partial\vartheta\,\partial\varphi}
\\
&\qquad\qquad
+\cos^2\varphi\cot\vartheta\frac{\partial}{\partial\vartheta}
+\cos\varphi\sin\varphi\cot\vartheta\frac{\partial^2}{\partial\vartheta\,\partial\varphi}
-\cos\varphi\sin\varphi\cot^2\vartheta\frac{\partial}{\partial\varphi}
+\cos^2\varphi\cot^2\vartheta\frac{\partial^2}{\partial\varphi^2}
\biggr)
\\
&=
-\hbar^2\biggl(
\sin^2\varphi\frac{\partial^2}{\partial\vartheta^2}
+
\cos^2\varphi\cot\vartheta\frac{\partial}{\partial\vartheta}
+
\cos^2\varphi\cot^2\vartheta\frac{\partial^2}{\partial\varphi^2}
\biggr)
\\
L_2^2
&=
-\hbar^2
\biggl(
\cos\varphi\frac{\partial}{\partial\vartheta}
-\sin\varphi\cot\vartheta\frac{\partial}{\partial\varphi}
\biggr)
\biggl(
\cos\varphi\frac{\partial}{\partial\vartheta}
-\sin\varphi\cot\vartheta\frac{\partial}{\partial\varphi}
\biggr)
\\
&=-\hbar^2
\biggl(
\cos^2\varphi\frac{\partial^2}{\partial\vartheta^2}
-\cos\varphi\sin\varphi\frac{1}{\sin^2\vartheta}\frac{\partial}{\partial\varphi}
+\cos\varphi\sin\varphi\cot\vartheta\frac{\partial^2}{\partial\vartheta\,\partial\varphi}
\\
&\qquad\qquad
+\sin^2\varphi\cot\vartheta\frac{\partial}{\partial\vartheta}
-\sin\varphi\cos\varphi\cot\vartheta\frac{\partial^2}{\partial\varphi\,\partial\vartheta}
+\sin\varphi\cos\varphi\cot^2\vartheta\frac{\partial}{\partial\varphi}
+\sin^2\varphi\cot^2\vartheta\frac{\partial^2}{\partial\varphi^2}
\biggr)
\\
&=
-\hbar^2\biggl(
\cos^2\varphi\frac{\partial^2}{\partial\vartheta^2}
+
\sin^2\varphi\cot\vartheta\frac{\partial}{\partial\vartheta}
+
\sin^2\varphi\cot^2\vartheta\frac{\partial^2}{\partial\varphi^2}
\biggr)
\end{align*}
In beiden F"allen haben wir die Identit"at
\[
\frac1{\sin^2\vartheta}
-
\cot^2\vartheta
=\frac{1-\cos^2\vartheta}{\sin^2\vartheta}=1
\]
verwendet.
Die Summe der beiden Quadrate ist damit
\begin{align*}
L_1^2+L_2^2
&=
-\hbar^2\biggl(
\frac{\partial^2}{\partial\vartheta^2}
+
\cot\vartheta\frac{\partial}{\partial\vartheta}
+
\cot^2\vartheta\frac{\partial^2}{\partial\varphi^2}
\biggr)
\end{align*}
Der Drehimpulsbetrag ist daher
\[
L_1^2+L_2^2+L_3^2
=
-\hbar^2\biggl(
\frac{\partial^2}{\partial\vartheta^2}
+
\cot\vartheta\frac{\partial}{\partial\vartheta}
+
\cot^2\vartheta\frac{\partial^2}{\partial\varphi^2}
+
\frac{\partial^2}{\partial\varphi^2}
\biggr)
=
-\hbar^2\biggl(
\frac{\partial^2}{\partial\vartheta^2}
+
\cot\vartheta\frac{\partial}{\partial\vartheta}
+
\frac1{\sin^2\vartheta}\frac{\partial^2}{\partial\varphi^2}
\biggr)
\]
Darin haben wir die Identit"at
\[
1+\cot^2\vartheta
=
1+\frac{\cos^2\vartheta}{\sin^2\vartheta}
=
\frac{\sin^2\vartheta+\cos^2\vartheta}{\sin^2\vartheta}
=
\frac{1}{\sin^2\vartheta}
\]
verwendet.

\subsubsection{Laplaceoperator}
\index{Drehimpuls!im Laplace-Operator}
Der Winkelanteil des Laplaceoperators lautete
\begin{align*}
\frac1{\sin\vartheta}\frac{\partial}{\partial\vartheta}\biggl(
\sin\vartheta\frac{\partial}{\partial\vartheta}
\biggr)
+\frac1{\sin^2\vartheta}\frac{\partial^2}{\partial\varphi^2}
&=
\frac{\cos\vartheta}{\sin\vartheta}\frac{\partial}{\partial\vartheta}
+
\frac{\partial^2}{\partial\vartheta^2}
+
\frac{1}{\sin^2\vartheta}\frac{\partial^2}{\partial\varphi^2}
\\
&=
\cot\vartheta\frac{\partial}{\partial\vartheta}
+
\frac{\partial^2}{\partial\vartheta^2}
+
\frac{1}{\sin^2\vartheta}\frac{\partial^2}{\partial\varphi^2}
\end{align*}
Dies stimmt mit $\vec L^2$ ausgedr"uckt in der Ortsdarstellung "uberein.
Der Hamiltonoperator zerf"allt damit in zwei Teile
\begin{align*}
H
&=-\frac{\hbar^2}{2m}
\frac1{r^2}\frac{\partial}{\partial r}\biggl(r^2\frac{\partial}{\partial r}\biggr)
-\frac{1}{2mr^2}\vec L^2
\end{align*}
Den ersten Term kann man interpretieren als die kinetische Energie
der Radialbewegung, w"ahrend der zweite Term die Rotationsenergie ist.

\section*{"Ubungsaufgabe}
\rhead{"Ubungsaufgabe}
\begin{uebungsaufgaben}
\item
\input uebungsaufgaben/12001.tex
\end{uebungsaufgaben}
