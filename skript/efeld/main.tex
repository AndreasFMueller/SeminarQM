\chapter{Geladenes Teilchen im elektrischen Feld\label{chapter:efeld}}
\lhead{Teilchen im elektrischen Feld}
\begin{refsection}
\chapterauthor{Michael Cerny und Stefan Schindler}


In diesem Kapitel erweitern wir das Beispiel des Potentialtopfs 
(Kapitel \ref{subsection:potentialtopf}, Seite \pageref{subsection:potentialtopf})
um eine Störung.

\section{ 1. St"orungstheorie }
Grundsätzlich wir mit der St"orungstheorie ein einfaches Modell-System um eine St"orung
erg"antzt, statt von Anfang an mit einem komplexen Modell zu rechnen.

Wir erg\"anzen das Modell des einfachen Systems $\hat{H_0}$
(Siehe Abbildungen~\ref{skript:potentialtopf-loesungen-klein}
und \ref{skript:potentialtopf-loesungen})
um $\hat{H_1}$, welches das angelegte Feld darstellt.
\[
\hat{H} = \hat{H_0} + \hat{H_1}
\]



\section{ 2. Nummerisches Beispiel }

\subsection{ 1. \& 2. N"aherung }

\subsection{ Energie "Anderung }

\section{ 3. Anwendung Stark Effekt (evt.) }

\subsection{ Alg. Quantum Confined Stark Effect }

\subsection{ Laser, mehr Energie }
Was passiert mit der Frequenz des Emitierten Lichts, wenn man mehr oder weniger Energie an den Laser gibt.

\printbibliography[heading=subbibliography]
\end{refsection}
