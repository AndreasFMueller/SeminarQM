\chapter{Geladenes Teilchen im elektrischen Feld\label{chapter:efeld}}
\lhead{Teilchen im elektrischen Feld}
\begin{refsection}
\chapterauthor{Michael Cerny und Stefan Schindler}


In diesem Kapitel erweitern wir das Beispiel des Potentialkastenss 
(Kapitel \ref{subsection:potentialkasten}, Seite \pageref{subsection:potentialkasten})
um eine St\"orung.

\section{ St"orungstheorie }
Grundsätzlich wir mit der St\"orungstheorie ein einfaches Modell-System um eine St\"orung
erg"antzt, statt von Anfang an mit einem komplexen Modell zu rechnen.

Wir erg\"anzen das Modell des einfachen Systems $\hat{H_0}$
(Siehe Abbildung \ref{skript:potentialkasten})
um $\hat{H_1}$, welches das angelegte Feld darstellt.
\[
\hat{H} = \varepsilon^0 \hat{H_0} + \varepsilon^1 \hat{H_1}
\]





\section{ Beispiele }

\subsection{ Potentialkasten mit eFeld }
Als $\hat{H_0}$ nehmen wir
\[
  \hat{H_0} = \frac{\hbar^2}{2m} \frac{\partial^2}{\partial x^2} + V_0(x)
\]
f\"ur das Potential $V_0(x)$ gilt
\[
  V_0(x)=\begin{cases}
    0       & \qquad |x|<l\\
    \infty  & \qquad\text{sonst.}
  \end{cases}
\]

F\"ur $\hat{H_1}$ gilt
\[
  \hat{H_1} = \frac{\hbar^2}{2m} \frac{\partial^2}{\partial x^2} + V_1(x)
\]

f\"ur das Potential $V_1(x)$ gilt
\[
  V_1(x) = a*x +b ; b = 0 \text{ weil es dann symetisch ist }
\]

Die Ausgangsgleichung f\"ur $\hat{H}$ lautet
\[
  \hat{H} = \varepsilon^0 ( \frac{\hbar^2}{2m} \frac{\partial^2}{\partial x^2} + V_0(x) )
            + \varepsilon^1 ( \frac{\hbar^2}{2m} \frac{\partial^2}{\partial x^2} + a*x )
\]


\subsection{ 1. N"aherung }

Wie kamen wir darauf?

\subsection{ 2. N\"aherung (evt. nach Vortrag) }

\subsection{ Energie "Anderung }

Was passiert?

\section{ 3. Anwendung Potentialtopf }

\[
  V_1 = \text{ innen \& aussen am Topf ... }
\]
]

\section{ 3. Anwendung Stark Effekt (evt.) }

\subsection{ Alg. Quantum Confined Stark Effect }

\subsection{ Laser, mehr Energie }
Was passiert mit der Frequenz des Emitierten Lichts, wenn man mehr oder weniger Energie an den Laser gibt.

\printbibliography[heading=subbibliography]
\end{refsection}
