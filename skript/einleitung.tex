\chapter{Einleitung\label{chapter:einleitung}}
\lhead{Einleitung}
\rhead{}

Gegen Ende des 19.~Jahrhunderts war eine oft publizierte Meinung 
"uber den Stand der Physik, dass fast alles erforscht sei, und dass
nur noch wenige Einzelheiten zu kl"aren w"aren.
Tats"achlich konnte die Physik zu dieser Zeit beachtliche Erfolge 
vorweisen.
Die klassische Mechanik lieferte einen auf alle Anwendungen in den
Ingenieurwissenschaften anwendbare Grundlage.
Mit ihr konnten die Bewegungen der Himmelsk"orper fast perfekt
vorhergesagt werden\footnote{
\index{Merkur, Periheldrehung}
Nur gerade die Bewegung des Merkur zeigte eine Unregelm"assigkeit, die
\index{Relativit\"atstheorie}
Einstein mit seiner allgemeinen Relativit"atstheorie erkl"aren konnte.}
Die Thermodynamik bildete eine solide Grundlage f"ur den Bau von
W"armekraftmaschinen.
\index{Hertz, Heinrich}
Und mit Maxwells Gleichungen konnten 
die elektromagnetischen Wellen beschrieben werden, die Heinrich
Hertz in den achtziger Jahren des 19.~Jahrhunderts nachweisen konnte,
und die die Grundlage f"ur die drahtlose Telegraphie waren.

Zu den noch nicht endg"ultig gekl"arten Details geh"orten
zum Beispiel die folgenden Fragen:
\begin{enumerate}
\item Wie entsteht die spektrale Verteilung der Strahlung eines
schwarzen K"orpers? 

\index{Kirchhoff, Gustav}
\index{schwarzer K\"orper}
\index{Schwarzk\"orperstrahlung}
Gustav Kirchhoff erdachte sich ein idealisierte Strahlungsquelle, 
die jegliche einfallende Strahlung absorbiert, also perfekt schwarz
ist. Man kann zeigen, dass das Strahlungsspektrum einer solchen Quelle
nur von der Temperatur abh"angen sollte. Entsprechend sollte die Physik
in der Lage sein, den Zusammenhang zwischen Temperatur, Wellenl"ange
und Intensit"at zu beschreiben. Tats"achlich gelang es, sowohl f"ur 
grosse Wellenl"angen als auch f"ur kleine Wellenl"angen gute Approximationen
aus den Prinzipien der klassischen Physik herzuleiten, doch gab es keine
Formel, die f"ur alle Wellenl"angen funktioniert.

\index{Planck, Max}
Dies "anderte sich mit der Entdeckung von Max Planck im Jahre 1900.
Indem er annahm, dass das Licht in Form von Paketen ausgestrahlt w"urden,
deren Energie proportional zur Frequenz sind, konnte er ein Spektralgesetzt
herleiten, welches f"ur alle Wellenl"angen gut mit experimentellen
Befunden "ubereinstimmt. Den Proportionalit"atsfaktor zwischen Energie $E$
und Frequenz $\nu$ nannte er $h$, $E=h\nu$, er ist eine Naturkonstante
mit dem Wert $h=6.62606957\cdot 10^{-34}\text{Js}$ und der Dimension
einer Wirkung, auch genannt das Plancksche Wirkungsquantum\footnote{Zahlenwerte
in der Quantenmechanik wichtiger Naturkonstanten sind im
Anhang~\ref{chapter:konstanten} zusammengestellt.}.
\index{h@$\hbar$}
\index{h@$h$}
\index{Wirkungsquantum, Plancksches}

\item
\index{Photoeffekt}
Der Photoeffekt. Man wusste, dass Licht in der
Lage ist, Ladungstr"ager aus einer geladenen Elektrode herauszuschlagen.
\index{Photostrom}
Die klassische Physik ging zu dieser Zeit davon aus, dass Licht
eine Wellenph"anomen ist und erwartete, dass mit zunehmender Intensit"at des
Lichtes der Photostrom zunehmen w"urde.
Mit abnehmender Intensit"at w"urde der Photostrom geringer, und w"urde
beim Unterschreiten einer gewissen Intensit"at versiegen.
Experimente zeigten jedoch, dass die
Intensit"at zwar einen Einfluss auf den Photostrom hat, dass aber
die Wellenl"ange des Lichtes eine noch wichtigere Rolle spielt.
Wenn die Wellenl"ange zu lang ist, fliesst "uberhaupt kein Photostrom.
Wenn die Wellenl"ange kurz genug ist, fliest der Photostrom auch noch
bei beliebig geringer Intensit"at.
Der Photostrom verh"alt sich also v"ollig anders, als man nach der
klassischen Physik erwarten k"onnten.

\index{Einstein, Albert}
Im Jahre 1905 ver"offentlichte Albert Einstein eine Arbeit "uber den
Photoeffekt, in der er das Paradoxon erkl"arte. Wie Planck nahm er an,
dass Licht in diesem Falle nicht wie eine Welle, sondern wie in Strom
von Teilchen funktioniert, f"ur die Energie gilt $E=h\nu$, sie ist also
umgekehrt proportional zur Wellenl"ange.
Ein Photostrom
enstand immer dann, wenn die Lichtteilchen gen"ugend Energie hatten, um
Elektronen aus dem Metall herauszuschlagen. Dies erkl"arte die Tatsache,
dass das Licht ausreichen kurze Wellenl"ange haben musste. Eine geringe
Intensit"at des Lichtes bedeutet, dass die Lichtteilchen seltener sind,
aber immer noch die gleiche Energie haben. Sie k"onnen auch bei beliebig 
kleiner Intensit"at immer noch einen Photostrom erzeugen.

\item
Wie entstehen die periodischen chemischen Eigenschaften der
Elemente? Oft wurde dies gar nicht als eine Fragestellung f"ur die
Physik angesehen, sondern als ein Problem der Chemie.

Allgemeiner geht es hier um die Frage, wie die Struktur der Materie
erkl"art werden kann.
\index{Mendeleev, Dmitri}
\index{Periodensystem}
Mendeleews Periodensystem hat der Chemie ein Ordnungsprinzip gegeben,
welches m"ogliche Verbindungen und Reaktionen vorhersagen kann.
Doch waren keine physikalische Prinzipien bekannt, welche diese Beobachtungen
erkl"aren konnten.

\index{Bohr, Niels}
\index{Bohrsches Atommodell}
Bohrs Atommodell von 1913 konnte die periodischen Eigenschaften der Elemente
auf ein physikalisches Prinzip zur"uckf"uhren, n"amlich die Forderung,
dass Elektronen in einem Atom sich so bewegen m"ussen, dass sein Drehimpuls
ein ganzzahliges Vielfaches von $\hbar = h/2\pi$ sein m"usse.
Zusammen mit den bekannten Bewegungsgesetzen f"ur ein Elektron in einem
Coulomb-Potential konnte er so die beobachteten Spektrallinien von
Wasserstoffatomen erkl"aren. 
\index{Coulomb-Potential}
Bohrs Atommodell hat zwei entscheidende M"angel.
Ein um einen Kern kreisendes Elektron m"usste best"andig strahlen
und daher in den Atomkern st"urzen. Atome k"onnten also gar nicht
stabil sein.
Das zweite Problem besteht darin, dass die ``Quantenregel'' "uber den 
Drehimpuls zwar funktionierte, aber v"ollig ad hoc erschien.
Man erwartete ein allgemeines Prinzip, welches diese und m"ogliche
weitere Quantenregeln erkl"aren konnte.

Trotz dieser M"angel war das Bohrsche Modell recht erfolgreich, es
konnten damit Modelle f"ur eine ganze Reihe von physikalischen Effekten
aufgestellt werden, die gut mit experimentellen Befunden "ubereinstimmten.
\end{enumerate}
Dies war also der Stand der Physik zu Begin des 20.~Jahrhunderts.
Die Idee der Quantennatur der Physik war bereits etabliert, allerdings
eher in Form eines heuristischen Prinzips.

Nach dem Muster der allgemeinen Mechanik ist eine allgemeine
Theorie der Quanten gefragt, welche die Bewegung mikroskopischer Teilchen
beschreiben kann. Sie muss die klassische Mechanik als Spezialfall
f"ur schwere Teilchen enthalten. Ausserdem muss sie mit der
Thermodynamik und der Elektrodynamik zusammenspielen, und sollte
sowohl das Plancksche Strahlungsgesetz und photoelektrischen
Effekt erkl"aren.
Und sie sollte ausserdem die Struktur der Materie und alle Beobachtungen
der Chemie erkl"aren.

Dieses Ziel wurde in den zwanziger Jahren des 20.~Jahrhunderts mit 
der Quantenmechanik erreicht.
Der Vorteil der sp"aten Geburt erlaubt uns, diesen beschwerlichen Weg
abzuk"urzen, und direkt auf die ``richtigen'' Konzepte hinzusteuern.
Ein Darstellung des beschwerlichen historischen Weges ist enthalten
in \cite{skript:viascience}.
Diese Folge von Videos zeigt im "Ubrigen einen "ahnlichen Zugang,
wie wir ihn in diesem Skript verfolgen.
Wir werden dabei in folgenden Schritten vorgehen,
die Abh"angigkeiten der einzelnen Kapitel untereinander sind in
Abbildung~\ref{skript:dependencies} dargestellt:
\begin{enumerate}
\item Kapitel~\ref{chapter:einfache-quantensysteme}:
Aufbau eines Formalismus f"ur Quantensysteme, vorl"aufig ohne einen
direkten Bezug zur klassischen Mechanik, die f"ur die Quantenmechanik
eine Approximation sein soll.
\item Kapitel~\ref{chapter:hilbertraeume}:
die Quantenmechanik verwendet
einen mathematischen Kalk"ul, der Wellenbeschreibungen (Fouriertheorie)
und Wahrscheinlichkeitsrechnung zusammenzubringen gestattet, die Hilbertr"aume.
\item Kapitel~\ref{chapter:quantencomputer}: 
Der Formalismus von Kapitel~\ref{chapter:einfache-quantensysteme}
reicht bereits aus um zu verstehen, was ein Quantencomputer ist.
\item Kapitel~\ref{chapter:mechanik}:
Formulierung der Mechanik auf 
eine Weise, in der der "Ubergang zur Quantenmechanik m"oglich wird.
Wir lernen die Formulierungen der Mechanik von Lagrange und von Hamilton
kennen.
\item Kapitel~\ref{chapter:quantisierung}:
Die Quantisierungsregeln
gestatten die "Ubersetzung eines klassisch mechanischen Problems in 
ein quantenmechanisches Problem. In diesem Kapitel wird die
Schr"odinger-Gleichung hergeleitet.
\item Kapitel~\ref{chapter:heisenberg}:
Die Quantenmechanik sagt voraus, dass gewisse Gr"ossen nicht gleichzeitig
bekannt sein k"onnen. 
Der mathematische Ausdruck f"ur diese Einschr"ankung ist die 
Heisenbergsche Unsch"arferelation.
\item Kapitel~\ref{chapter:harmonischeroszillator}:
Der harmonische
Oszillator als ein Beispiel eines einfachen Systems, welches
viele interessante Eigenschaften von Quantensystemen zeigt und 
ausserdem besonders interessante algebraische Eigenschaften hat.
\item Kapitel~\ref{chapter:wasserstoff}:
Die Quantenmechanik soll
die Struktur der Materie erkl"aren. Insbesondere sollte sie 
die Eigenschaften eines Wasserstoffatoms zu berechnen gestatten.
\item Kapitel~\ref{chapter:stoerungstheorie}:
Viele Quantensysteme sind zu kompliziert, um ihre Schr"odingergleichung
direkt l"osen zu k"onnen.
St"orungstheorie erm"oglicht, Energieniveaus und Wellenfunktionen
n"aherungsweise zu berechnen, wenn ein bereits gel"ostes System
nur leicht modifiziert wird.
\item Kapitel~\ref{chapter:magnetfeld}:
Ein Elektron wird in von einem Magnetfeld abgelenkt, aber die daf"ur
verantwortliche Lorentzkraft leistet keine Arbeit.
Der Hamilton-Formalismus von Kapitel~\ref{chapter:mechanik} muss
daher erweitert werden.
\item Kapitel~\ref{chapter:drehimpuls}:
Die Quantisierungsregeln diktieren, wie Drehimpuls quantenmechanisch
zu behandeln ist.
Es stellt sich heraus, dass ein "ahnlicher algebraischer Formalismus 
wie beim harmonischen Oszillator die m"oglichen Drehimpulswerte
zu bestimmen erlaubt.
\item Kapitel~\ref{chapter:spin}:
Obwohl Elektronen in der Quantenmechanik als punktf"ormig betrachtet
werden, und sich daher keine Eigenrotation haben k"onnen,
haben sie Zust"ande, die sich wie eine Eigendrehung verhalten:
der Elektron-Spin.
\item Kapitel~\ref{chapter:festkoerper}:
Aus der elementaren Quantenmechanik lassen sich bereits einige
einfache Eigenschaften von Festk"orpern ableiten.
\end{enumerate}

\begin{figure}
\centering
\includegraphics{graphics/dependencies-1.pdf}
\caption{Abh"angigkeiten der einzelnen Kapitel untereinander.
Grau hinterlegt die Anh"ange mit mathematischem Grundlagenmaterial
und numerischen Werten der Naturkonstanten.
\label{skript:dependencies}}
\end{figure}

Dieses Skript wurde vor allem im zweiten Kapitel von Heinrich Mitters
Lehrbuch \cite{skript:mitter} stark beeinflusst. 
Im allgemeinen bevorzugen wir eine algebraische Darstellung, die
m"oglichst ohne Differentialgleichungen auskommt.
Eine elegante Realisierung dieses Ansatzes ist \cite{skript:green}.
Nat"urlich lassen sich Differentialgleichungen nicht vermeiden,
wir verwenden daher meistens eindimensionale Modelle, die sich mit
gew"ohnlichen Differentialgleichungen behandeln lassen, und trotzdem
die interessierenden quantenmechanischen Ph"anomene illustrieren.
Bei der Berechnung des Wasserstoffatoms ist jedoch das L"osen einer partiellen
Differentialgleichungen unvermeidlich, wir tun dies mit ad hoc entwickelten
Methoden, und verweisen den Leser auf spezialisierte Literatur wie 
\cite{skript:evans} f"ur eine vollst"andige Theorie.
Da die physikalische Plausibilit"at gegen"uber der mathematischen Strenge
im Vordergrund steht, scheint dies das ad"aquate Vorgehen zu sein.
Das Kapitel "uber Quantencomputer verwendet die von
\cite{skript:kaye-et-al} und vor allem von
\cite{skript:arorabarak} etablierte Darstellungsweise.
Das Kapitel "uber Festk"orperphysik folgt zum Teil der Darstellung von
\cite{skript:madelung1}.
Weitere Literaturhinweise im Literaturverzeichnis auf
Seite~\pageref{skript:literatur}.

