\chapter{Feldquantisierung\label{chapter:feldquantisierung}}
\lhead{Feldquantisierung}
\begin{refsection}
\chapterauthor{Hannes Diethelm}

\printbibliography[heading=subbibliography]
\end{refsection}

\section{Maxwell-Gleichungen und elektromagnetische Wellen}

Hilfreich dazu ist auch die Beschreibung von Magnetfeldern in Kapitel \ref{chapter:magnetfeld}. In diese Kapitel wird der Gradient durch $\nabla$ ersetzt \cite{fq:nabla}. Dadurch k"onnen die Gleichungen einfacher geschrieben werden. 

Die in der Elektrotechnik wohl bekannten Maxwell-Gleichungen in SI Einheiten lauten:
\begin{equation}
\begin{split}
\nabla\cdot E &= \frac{\rho}{\varepsilon_0} \\
\nabla\times B &= \mu_0( J  + \varepsilon_0\frac{\partial E}{\partial t}) \\
\nabla\cdot B &=0 \\
\nabla\times E &= -\frac{\partial B }{\partial t}\\
\end{split}
\end{equation}

Dieses Einheitensystem is willk"urlich \cite{fq:em_units}. Im Heaviside-
Lorentz System, das von nun an verwendet wird, lauten die Gleichungen:
\begin{equation}
\begin{split}
\nabla\cdot E &= \rho \\
\nabla\times B &= \frac{1}{c}( J  + \frac{\partial E}{\partial t}) \\
\nabla\cdot B &=0 \\
\nabla\times E &= -\frac{1}{c} \frac{\partial B }{\partial t}\\
\end{split}
\end{equation}

Da $\nabla \cdot B = 0 $ gilt k"onnen diese Gleichungen durch folgende Substitution umformuliert werden:
\begin{equation}
B = \nabla\times A 
\end{equation}

Dadurch gilt $\nabla \cdot B = 0 $ automatisch:
\begin{equation}
\nabla \cdot B = 0 \rightarrow \nabla \cdot ( \nabla\times A ) = 0 \text{ gilt f"ur jedes A! }
\end{equation}

Durch Einsetzen erh"alt man die Gleichung f"ur E:
\begin{equation}
\nabla\times E + \frac{1}{c} \frac{\partial B }{\partial t} = 0
\rightarrow \nabla\times E + \frac{1}{c} \frac{\partial \nabla\times A }{\partial t} = 0 \rightarrow E = -\frac{1}{c} \dfrac{\partial A}{\partial t} - \nabla \phi
\end{equation}

$\nabla \phi$ kann als Integrationskonstante angesehen werden und $\phi$ entspricht dem skalaren Potential des Feldes.

Durch weiteres Einsetzen k"onnen die vier Maxwell-Gleichungen in zwei Gleichungen umgeschrieben werden:
\begin{equation}
\begin{split}
 \nabla^2 \phi + \frac{1}{c} \dfrac{\partial \nabla A}{\partial t} &= -\rho \\
 \nabla^2 A - \frac{1}{c^2} \frac{\partial^2 A }{\partial t^2} - \nabla \left( \nabla \cdot A + \frac{1}{c} \frac{\partial \phi }{\partial t} \right) &= - \frac{1}{c} J
\end{split}
\end{equation}

dabei gelten die Korrespondenzen:
\begin{equation}
\begin{split}
B &= \nabla\times A \\
E &= -\frac{1}{c} \dfrac{\partial A}{\partial t} - \nabla \phi
\end{split}
\end{equation}

Es kann gezeigt werden, dass $\phi$ durch eine Eichtransformation (Siehe \ref{section:eichtransformation}) geeignet gew"ahlt werden kann, damit:

\begin{equation}
\nabla \cdot A + \frac{1}{c} \frac{\partial \phi }{\partial t} = 0
\end{equation}

Dadurch werden die zwei gekoppelten Gleichungen entkoppelt und es gilt:
\begin{equation}
\begin{split}
\nabla^2 \phi - \frac{1}{c^2} \dfrac{\partial^2 \nabla \phi}{\partial t^2} &= -\rho \\
\nabla^2 A - \frac{1}{c^2} \frac{\partial^2 A }{\partial t^2} &= - \frac{1}{c} J
\end{split}
\end{equation}

F"ur weiter wollen ein Feld im Vakkum betrachten. Hierf"ur gilt $J = 0$, da keine Leiter vorhanden sind.
In einem Transversalfeld im Vakkum gilt zudem $\nabla \cdot A = 0$. (ToDo: ??) Dadurch vereinfachen sich die gekoppelten Differentialgleichung zu einer Differentialgleichung in A:
\begin{equation}
\nabla^2 A - \frac{1}{c^2} \frac{\partial^2 A }{\partial t^2} = 0
\end{equation}

\section{Von der Welle zu gekoppelten Oszillatoren}
L"osungen dieser Gleichung f"ur periodische Randbedingungen und $t=0$ in einer Box mit Seitenl"ange $L = V^{1/3}$ sind durch die Fourier Reihe gegeben:

\begin{equation}
A(x,0) = \frac{1}{\sqrt{V}} \sum_K \sum_{\alpha=1,2} (c_{k,\alpha}(0) \epsilon^{(\alpha)} e^{ikx} + c^*_{k,\alpha}(0) \epsilon^{(\alpha)} e^{-ikx})
\end{equation}

oder durch setzen von $u_{k,\alpha}(x) = \epsilon^{(\alpha)} e^{ikx}$:
\begin{equation}
A(x,0) = \frac{1}{\sqrt{V}} \sum_K \sum_{\alpha=1,2} (c_{k,\alpha}(0)u_{k,\alpha}(x) + c^*_{k,\alpha}(0) u^*_{k,\alpha}(x))
\end{equation}

Wenn diese Gleichung ausgeschrieben wird, sieht man, dass $A(x,t)$ durch diese Wahl f"ur alle $c_{k,\alpha}(t)$ reell bleibt:
\begin{equation}
(a + ib)(\cos kx + i \sin kx ) + (a - ib)(\cos kx - i \sin kx ) = 2 ( a \cos kx - b \sin kx )
\end{equation}
%=a \cos kx + ib \cos kx + ia \sin kx - b \sin kx + a \cos kx - ib \cos kx - ia \sin kx - b \sin kx

$k$ ist der Ausbreitungsvektor der Welle und zeigt in die Ausbreitungsrichtung. $\epsilon^{(\alpha)}$ ist die Polarisation. Dabei wird vorausgesetzt, dass $(\epsilon^{(1)}, \epsilon^{(2)} , k/|k|)$ ein orthogonales Rechtssystem aus Einheitsvektoren bilden.

Da $\epsilon^{(\alpha)}$ und $k$ orthogonal sind gilt dabei auch automatisch:

\begin{equation}
\nabla \cdot A = \frac{1}{\sqrt{V}} \sum_K \sum_{\alpha=1,2} (i c_{k,\alpha}(0) \underbrace{\epsilon^{(\alpha)} k}_{=0} e^{ikx} - i c^*_{k,\alpha}(0) \underbrace{\epsilon^{(\alpha)} k}_{=0} e^{-ikx}) = 0
\end{equation}

Weiterhin gilt durch wegen der Orthogonalit"at auch:
\begin{equation}
\begin{split}
\dfrac{1}{A} \int c_{k,\alpha} \cdot c^*_{k',\alpha'} d^3 x &= \delta_{kk'}\delta{aa'} \\
\dfrac{1}{A} \int c_{k,\alpha} \cdot c_{k',\alpha'} d^3 x &= 0 \\
\dfrac{1}{A} \int c^*_{k,\alpha} \cdot c^*_{k',\alpha'} d^3 x &= 0
\end{split}
\end{equation}

Um $A(x,t)$ zu erhalten, wird:
\begin{equation}
c_{k,\alpha}(t) = c_{k,\alpha}(0) e^{-i \omega t}
\end{equation}

Dabei ist:
\begin{equation}
\begin{split}
\omega=|k|c \\
\lambda = \frac{2 \pi}{|k|}
\end{split}
\end{equation}

Die komplette Wellengleichung wird somit:
\begin{equation}
A(x,t) = \frac{1}{\sqrt{V}} \sum_K \sum_{\alpha=1,2} (c_{k,\alpha}(0) \epsilon^{(\alpha)} e^{i (kx - \omega t)} + c^*_{k,\alpha}(0) \epsilon^{(\alpha)} e^{-i(kx - \omega t)})
\end{equation}

Die Hamilton-Funktion einer elektromagnetischen Welle ist gegeben durch:
\begin{equation}
\begin{split}
H &= \frac{1}{2} \int (|B|^2 + |E|^2) d^3 x \\
	&= \frac{1}{2} \int (| \nabla\times A |^2 + \left| \frac{1}{c} \dfrac{\partial A}{\partial t} \right|^2) d^3 x 
\end{split}
\end{equation}

Es kann gezeigt werden, dass die L"osung dieses Integrals gegeben ist durch:
\begin{equation}
H = \sum_K \sum_{\alpha=1,2} 2 \left(\frac{\omega}{c}\right)^2 c^*_{k,\alpha}(t) c_{k,\alpha}(t)
\end{equation}

Durch folgende Definition:
\begin{equation}
Q_{k,\alpha} = \frac{1}{c}(c_{k,\alpha}(t) + c^*_{k,\alpha}(t)) \quad P_{k,\alpha} = -\frac{i\omega}{c}(c_{k,\alpha}(t) - c^*_{k,\alpha}(t)) 
\end{equation}

wird die Hamilton-Funktion zu:
\begin{equation} \label{fq:hamilton}
\begin{split}
H &= \sum_K \sum_{\alpha=1,2} 2 \left(\frac{\omega}{c}\right)^2 \left[ \frac{c(\omega Q_{k,\alpha} - i P_{k,\alpha})}{2 \omega} \right] \left[ \frac{c(\omega Q_{k,\alpha} + i P_{k,\alpha})}{2 \omega} \right] \\
&= \sum_K \sum_{\alpha=1,2} \frac{1}{2} (P_{k,\alpha}^2 + \omega^2 Q_{k,\alpha}^2)
\end{split}
\end{equation}

Hier sieh man nun, dass es m"oglich ist, eine Welle durch unabh"angige Oszillatoren dar zu stellen.

$Q_{k,\alpha}$ und $P_{k,\alpha}$ k"onnen nun als Koordinaten und Impulse der einzelnen Oszillatoren aufgefasst werden:
\begin{equation}
\dfrac{\partial H}{\partial Q_{k,\alpha}} = -\dot{P}_{k,\alpha} \quad \dfrac{\partial H}{\partial P_{k,\alpha}} = \dot{Q}_{k,\alpha}
\end{equation}

\section{Quantisierung der Welle}

\subsection{Vertauschungsrelationen sowie Auf- und Absteige Operatoren}
Wie beim harmonischen Oszillator können $Q_{k,\alpha}$ und $P_{k,\alpha}$ nun als Opperatoren aufgefasst werden. Die Vertauschungsrelationen werden dabei zu:
\begin{equation}
\begin{split}
[Q_{k,\alpha}, P_{k',\alpha'}] &= i \hbar \delta_{kk'}\delta_{aa'} \\
[Q_{k,\alpha}, Q_{k',\alpha'}] &= 0 \\
[P_{k,\alpha}, P_{k',\alpha'}] &= 0
\end{split}
\end{equation}

Wir definieren die Operatoren:
\begin{equation}
\begin{split}
a_{k,\alpha} &= (1/\sqrt{2 \hbar \omega})(\omega Q_{k,\alpha} + iP_{k,\alpha})) \\
a^+_{k,\alpha} &= (1/\sqrt{2 \hbar \omega})(\omega Q_{k,\alpha} - iP_{k,\alpha}))\\
N_{k,\alpha} &= a^+_{k,\alpha} a_{k,\alpha} = \frac{1}{2 \hbar \omega} (P_{k,\alpha}^2 + \omega^2 Q_{k,\alpha}^2 + i\omega [Q_{k,\alpha},P_{k,\alpha}] ) \\
 &= \frac{1}{ \hbar \omega } H - \frac{1}{2}
\end{split}
\end{equation}

Ein Vergleich mit \ref{fq:hamilton} liefert:
\begin{equation}
 c_{k,\alpha} \rightarrow c \sqrt{\hbar/2 \omega} \, a_{k,\alpha} \quad c^*_{k,\alpha} \rightarrow c \sqrt{\hbar/2 \omega} \, a^+_{k,\alpha}
\end{equation}
Somit entsprechen diese Operatoren den Fourier-Koeffizienten.

Die Kommentatoren f"ur diese Operatoren sind:
\begin{equation}
\begin{split}
[a_{k,\alpha} , a^+_{k',\alpha'}] &= - \frac{i}{2 \hbar} [Q_{k,\alpha}, P_{k',\alpha'}] + \frac{i}{2 \hbar} [P_{k,\alpha}, Q_{k',\alpha'}] \\
	 &= \delta_{kk'}\delta_{aa'} \\
[a_{k,\alpha} , a_{k',\alpha'}] &= [a^+_{k,\alpha} , a^+_{k',\alpha'}] = 0 \\
[a_{k,\alpha} , N_{k',\alpha'}] &= [a_{k,\alpha} , a^+_{k',\alpha'}]a_{k',\alpha'} - a^+_{k',\alpha'}[a_{k',\alpha'} , a_{k,\alpha}]\\
	&= \delta_{kk'}\delta_{aa'} a_{k,\alpha} \\
[a^+_{k,\alpha} , N_{k',\alpha'}] &= -\delta_{kk'}\delta_{aa'} a^+_{k,\alpha}
\end{split}
\end{equation}

\subsection{Wirkung auf Zustandsvektoren}

In diesem Unterkapitel werden Suffixes $_n$ und $_k$ weggelassen, diese gelten aber weiterhin. 

Diese Opperatoren werden nun wie beim harmonischen Oszillator auf den Zustand angewendet. Wir suchen einen Eigevektor $|n\rangle$ und Eigenwert $e_n$ zum Opperator $N$ so dass:
\begin{equation}
N|n\rangle = e_n|n\rangle
\end{equation}

Auf Grund der Beziehungen der Operatoren gilt dadurch:
\begin{equation}
\begin{split}
Na^+|n\rangle &= (a^+N + a^+)|n\rangle = (e_n + 1)a^+|n\rangle \\
Na|n\rangle &= (aN - a)|n\rangle = (e_n - 1)a|n\rangle
\end{split}
\end{equation}

Wiederum m"ussen die neuen Eigenvektoren normalisiert werden. Daf"ur werden die Konstanten $c_+$ und $c_-$ eingeführt.
\begin{equation}
\begin{split}
a^+|n\rangle &= c_+|n+1\rangle \\
a|n\rangle &= c_-|n-1\rangle
\end{split}
\end{equation}

$c_\pm$ wird folgendermassen berechnet:
\begin{equation}
\begin{split}
	|c_+|^2 &= |c_+|^2 \langle n+1 | n+1 \rangle = \langle a^+n | a^+n \rangle \\
		&= \langle n | aa^+ |n \rangle = \langle n | N + [a,a^+] |n \rangle \\
		&= e_n+1 \\
	|c_-|^2 &= 	\langle an | an \rangle = \langle n | a^+a | n \rangle = e_n
\end{split}
\end{equation}

Da die Energie nicht negativ sein kann gilt:
\begin{equation}
e_n = \langle n | N |n \rangle = \langle n | a^+a |n \rangle \geq 0
\end{equation}

