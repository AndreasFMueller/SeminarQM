\chapter{Feldquantisierung\label{chapter:feldquantisierung}}
\lhead{Feldquantisierung}
\begin{refsection}
\chapterauthor{Hannes Diethelm}

\printbibliography[heading=subbibliography]
\end{refsection}

\section{Maxwell-Gleichungen und elektromagnetische Wellen}

Die in der Elektrotechnik wohl bekannten Maxwell-Gleichungen in SI Einheiten lauten:
\begin{equation}
\begin{split}
\nabla\cdot E &= \frac{\rho}{\varepsilon_0} \\
\nabla\times B &= \mu_0( J  + \varepsilon_0\frac{\partial E}{\partial t}) \\
\nabla\cdot B &=0 \\
\nabla\times E &= -\frac{\partial B }{\partial t}\\
\end{split}
\end{equation}

Dieses Einheitensystem is willk"urlich \cite{skript:em_units}. Im Heaviside-
Lorentz System, das von nun an verwendet wird, lauten die Gleichungen:

\begin{equation}
\begin{split}
\nabla\cdot E &= \rho \\
\nabla\times B &= \frac{1}{c}( J  + \frac{\partial E}{\partial t}) \\
\nabla\cdot B &=0 \\
\nabla\times E &= -\frac{1}{c} \frac{\partial B }{\partial t}\\
\end{split}
\end{equation}

Diese Gleichungen können durch folgende Substitution umformuliert werden:
\begin{equation}
B = \nabla\times A 
\end{equation}

Daraus folgt:
\begin{equation}
\nabla \cdot B \rightarrow \nabla \cdot ( \nabla\times A ) = 0 \quad \text{ da } \operatorname{div}(\operatorname{rot}(A)) = 0
\end{equation}

\begin{equation}
\nabla\times E + \frac{1}{c} \frac{\partial B }{\partial t} = 0
\rightarrow \nabla\times E + \frac{1}{c} \frac{\partial \nabla\times A }{\partial t} = 0 \rightarrow E = -\frac{1}{c} \dfrac{\partial A}{\partial t} - \nabla \phi
\end{equation}

$\nabla \phi$ kann als Integrationskonstante angesehen werden und $\phi$ entspricht dem skalaren Potential des Feldes.

Durch weiteres Einsetzen können die vier Maxwell-Gleichungen in zwei Gleichungen umgeschrieben werden:
\begin{equation}
\begin{split}
 \nabla^2 \phi + \frac{1}{c} \dfrac{\partial \nabla A}{\partial t} &= -\rho \\
 \nabla^2 A - \frac{1}{c^2} \frac{\partial^2 A }{\partial t^2} - \nabla \left( \nabla \cdot A + \frac{1}{c} \frac{\partial \phi }{\partial t} \right) &= - \frac{1}{c} J
\end{split}
\end{equation}

dabei gelten die Korrespondenzen:
\begin{equation}
\begin{split}
B &= \nabla\times A \\
E &= -\frac{1}{c} \dfrac{\partial A}{\partial t} - \nabla \phi
\end{split}
\end{equation}

Es kann gezeigt werden, dass $\phi$ durch eine Eichtransformation geeignet gew"ahlt werden kann, damit:

\begin{equation}
\nabla \cdot A + \frac{1}{c} \frac{\partial \phi }{\partial t} = 0
\end{equation}

In einem Transversalfeld im Vakkum gilt zudem $\nabla \cdot A = 0$  sowie $J = 0$.

Dadurch vereinfachen sich die gekoppelten Differentialgleichung zu einer Differentialgleichung in A:
\begin{equation}
\nabla^2 A - \frac{1}{c^2} \frac{\partial^2 A }{\partial t^2} = 0
\end{equation}