\chapter{Feldquantisierung\label{chapter:feldquantisierung}}
\lhead{Feldquantisierung}
\begin{refsection}
\chapterauthor{Hannes Diethelm}

\printbibliography[heading=subbibliography]
\end{refsection}

\section{Maxwell-Gleichungen und elektromagnetische Wellen}

Hilfreich dazu ist auch die Beschreibung von Magnetfeldern in Kapitel \ref{chapter:magnetfeld}. In diese Kapitel wird der Gradient durch $\nabla$ ersetzt \cite{skript:nabla}. Dadurch k"onnen die Gleichungen einfacher geschrieben werden. 

Die in der Elektrotechnik wohl bekannten Maxwell-Gleichungen in SI Einheiten lauten:
\begin{equation}
\begin{split}
\nabla\cdot E &= \frac{\rho}{\varepsilon_0} \\
\nabla\times B &= \mu_0( J  + \varepsilon_0\frac{\partial E}{\partial t}) \\
\nabla\cdot B &=0 \\
\nabla\times E &= -\frac{\partial B }{\partial t}\\
\end{split}
\end{equation}

Dieses Einheitensystem is willk"urlich \cite{skript:em_units}. Im Heaviside-
Lorentz System, das von nun an verwendet wird, lauten die Gleichungen:
\begin{equation}
\begin{split}
\nabla\cdot E &= \rho \\
\nabla\times B &= \frac{1}{c}( J  + \frac{\partial E}{\partial t}) \\
\nabla\cdot B &=0 \\
\nabla\times E &= -\frac{1}{c} \frac{\partial B }{\partial t}\\
\end{split}
\end{equation}

Da $\nabla \cdot B = 0 $ gilt k"onnen diese Gleichungen durch folgende Substitution umformuliert werden:
\begin{equation}
B = \nabla\times A 
\end{equation}

Dadurch gilt $\nabla \cdot B = 0 $ automatisch:
\begin{equation}
\nabla \cdot B = 0 \rightarrow \nabla \cdot ( \nabla\times A ) = 0 \text{ gilt f"ur jedes A! }
\end{equation}

Durch Einsetzen erhält man die Gleichung f"ur E:
\begin{equation}
\nabla\times E + \frac{1}{c} \frac{\partial B }{\partial t} = 0
\rightarrow \nabla\times E + \frac{1}{c} \frac{\partial \nabla\times A }{\partial t} = 0 \rightarrow E = -\frac{1}{c} \dfrac{\partial A}{\partial t} - \nabla \phi
\end{equation}

$\nabla \phi$ kann als Integrationskonstante angesehen werden und $\phi$ entspricht dem skalaren Potential des Feldes.

Durch weiteres Einsetzen k"onnen die vier Maxwell-Gleichungen in zwei Gleichungen umgeschrieben werden:
\begin{equation}
\begin{split}
 \nabla^2 \phi + \frac{1}{c} \dfrac{\partial \nabla A}{\partial t} &= -\rho \\
 \nabla^2 A - \frac{1}{c^2} \frac{\partial^2 A }{\partial t^2} - \nabla \left( \nabla \cdot A + \frac{1}{c} \frac{\partial \phi }{\partial t} \right) &= - \frac{1}{c} J
\end{split}
\end{equation}

dabei gelten die Korrespondenzen:
\begin{equation}
\begin{split}
B &= \nabla\times A \\
E &= -\frac{1}{c} \dfrac{\partial A}{\partial t} - \nabla \phi
\end{split}
\end{equation}

Es kann gezeigt werden, dass $\phi$ durch eine Eichtransformation (Siehe \ref{section:eichtransformation}) geeignet gew"ahlt werden kann, damit:

\begin{equation}
\nabla \cdot A + \frac{1}{c} \frac{\partial \phi }{\partial t} = 0
\end{equation}

Dadurch werden die zwei gekoppelten Gleichungen entkoppelt und es gilt:
\begin{equation}
\begin{split}
\nabla^2 \phi - \frac{1}{c^2} \dfrac{\partial^2 \nabla \phi}{\partial t^2} &= -\rho \\
\nabla^2 A - \frac{1}{c^2} \frac{\partial^2 A }{\partial t^2} &= - \frac{1}{c} J
\end{split}
\end{equation}

F"ur weiter wollen ein Feld im Vakkum betrachten. Hierf"ur gilt $J = 0$, da keine Leiter vorhanden sind.
In einem Transversalfeld im Vakkum gilt zudem $\nabla \cdot A = 0$. (ToDo: ??) Dadurch vereinfachen sich die gekoppelten Differentialgleichung zu einer Differentialgleichung in A:
\begin{equation}
\nabla^2 A - \frac{1}{c^2} \frac{\partial^2 A }{\partial t^2} = 0
\end{equation}

L"osungen dieser Gleichung f"ur periodische Randbedingungen und $t=0$ in einer Box mit Seitenl"ange $L = V^{1/3}$ sind durch die Fourier Reihe gegeben:

\begin{equation}
A(x,0) = \frac{1}{\sqrt{V}} \sum_K \sum_{\alpha=1,2} (c_{k,\alpha}(0) \epsilon^{(\alpha)} e^{ikx} + c^*_{k,\alpha}(0) \epsilon^{(\alpha)} e^{-ikx})
\end{equation}

oder durch setzen von $u_{k,\alpha}(x) = \epsilon^{(\alpha)} e^{ikx}$:
\begin{equation}
A(x,0) = \frac{1}{\sqrt{V}} \sum_K \sum_{\alpha=1,2} (c_{k,\alpha}(0)u_{k,\alpha}(x) + c^*_{k,\alpha}(0) u^*_{k,\alpha}(x))
\end{equation}

Wenn diese Gleichung ausgeschrieben wird, sieht man, dass $A(x,t)$ durch diese Wahl f"ur alle $c_{k,\alpha}(t)$ reell bleibt:
\begin{equation}
(a + ib)(\cos kx + i \sin kx ) + (a - ib)(\cos kx - i \sin kx ) = 2 ( a \cos kx - b \sin kx )
\end{equation}
%=a \cos kx + ib \cos kx + ia \sin kx - b \sin kx + a \cos kx - ib \cos kx - ia \sin kx - b \sin kx

$k$ ist der Ausbreitungsvektor der Welle und zeigt in die Ausbreitungsrichtung. $\epsilon^{(\alpha)}$ ist die Polarisation. Dabei wird vorausgesetzt, dass $(\epsilon^{(1)}, \epsilon^{(2)} , k/|k|)$ ein orthogonales Rechtssystem aus Einheitsvektoren bilden.

Da $\epsilon^{(\alpha)}$ und $k$ orthogonal sind gilt dabei auch automatisch:

\begin{equation}
\nabla \cdot A = \frac{1}{\sqrt{V}} \sum_K \sum_{\alpha=1,2} (i c_{k,\alpha}(0) \underbrace{\epsilon^{(\alpha)} k}_{=0} e^{ikx} - i c^*_{k,\alpha}(0) \underbrace{\epsilon^{(\alpha)} k}_{=0} e^{-ikx}) = 0
\end{equation}

