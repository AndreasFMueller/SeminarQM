\chapter{Festk"orper\label{chapter:festkoerper}}
\lhead{Festk"orper}
\rhead{}
Festk"orper sind aus vielen Atomen zusammengesetzte Quantensysteme.
Im Gegensatz zu einer Fl"ussigkeit sind jedoch die Atomkerne mehr
oder weniger unverr"uckbar in einem Gitter angeordnet.
Bei elektrischen Leitern ist ein Teil der Elektronen weitgehend frei
beweglich.
Nat"urlich ist die Gitterstruktur ebenfalls die Folge der Interaktionen
zwischen den Elektronen, doch f"ur unsere Untersuchungen k"onnen wir
die Entstehung der Gitterstruktur vernachl"assigen, es einfach als
gegeben ansehen, und nur noch die Eigenschaften der frei beweglichen
Elektronen studieren.
Diese Elektronen bewegen sich in dem periodischen Potential, das von
den Atomkernen des Gitters erzeugt wird.
Jeder Atomkern erzeugt einen Potentialtopf, in dem eines oder mehrere
Elektronen gefangen sind.
Bleiben nach der Besetzung dieser lokalisierten Zust"ande noch
Elektronen "ubrig, dann haben diese mehr Energie als die Schwellen zwischen
den Potentialt"opfen hoch sind, aber nicht gen"ugend Energie, um
aus dem Festk"orper auszutreten.
Diese wie auch die Elektronen, die mit nur wenig Anregungsenergie
in einen ausgedehnten Zustand versetzt werden k"onnen, interessieren
uns in diesem Kapitel.

\section{Fermikugel}
In erster N"aherung betrachen wir den Festk"orper als einen
sehr grossen Potenialkasten, aus dem die Elektronen nicht austreten
k"onnen.
Die gebundenen Elektronen haben so tiefe Energie, dass wir davon
ausgehen k"onnen, dass diese Zust"ande immer besetzt sind.
Zur Leitung tragen sie weiter nichts bei, wir brauchen also f"ur
doe Modellierung von Leitungsph"anomenen nur diejenigen Elektronen
zu ber"ucksichtigen, die beweglich genug sind, sich im ganzen
Festk"orper bewegen zu k"onnen.
Weiter nehmen wir an, dass sich die Elektronen gegenseitig 
nicht beeinflussen.
Jedes Elektron bewegt sich im Potential, welches von den Atomkernen
und allen anderen Elektronen erzeugt wird.
Jedes Elektron sieht also im wesentlichen den gleichen Potentialkasten.
Die Energieniveaus f"ur die Zust"ande in einem Potentialkasten haben
wir in (\ref{skript:3dzustaende}) bereits berechnet.
\begin{figure}
\centering
\includegraphics{graphics/fest-1.pdf}
\caption{Fermi-Kugel im Raum der Zust"ande eines Elektrons in einem
Potentialkasten
\label{skript:fermi-kugel}}
\end{figure}
Nat"urlich m"ussen sich die Elektronen ausserdem an das Pauli-Prinzip
halten: jeder Energiezustand kann mit h"ochstens zwei Elektronen
besetzt sein. 
Im Zustand minimaler Energie ist der Festk"orper also, wenn sich
alle Elektronen in Zust"anden befinden, die einer Gleichung
\[
\frac{h^2}{32ml^2}(
n_x^2
+
n_y^2
+
n_z^2
)
<
E_F
\]
gen"ugen.
Die tats"achlich besetzten Elektronenzust"ande bilden also eine Kugel
im Raum der m"oglichen Elektronenzust"ande, die Fermi-Kugel.
Je gr"osser die Dichte der frei beweglichen Elektronen ist, desto
h"oher ist auch die Fermi-Energie $E_F$.

\begin{figure}
\centering
\includegraphics{graphics/fest-2.pdf}
\caption{Gebundene Niveaus in einem Kristall
\label{skript:gebundene-niveaus}}
\end{figure}
\begin{figure}
\centering
\includegraphics{graphics/fest-3.pdf}
\caption{Durch Erh"ohung der Dichte werden Elektronen auf einstmals
gebundenen Niveaus frei beweglich.
\label{skript:gebundene-niveaus-komprimiert}}
\end{figure}
Wenn also gen"ugen Elektronen vorhanden sind, dass $E_F$ "uber dem
``Bodenniveaus'' des Potentialkastens liegt, dann liegt ein Leiter vor.
Diese Situation kann auch erzwungen werden.
Presst man die Atomkerne weiter zusammen, werden auch die
Coulomb-Potential-L"ocher kleiner, die Elektronen haben f"ur normale
$s$-Orbitale gar keinen Platz mehr, und stehen daher als frei bewegliche
Elektronen zur Verf"ugung.
Die Fermi-Energie steigt also an, der K"orper wird ein Leiter.
Diesen Zustand erreicht Wasserstoff unter extrem hohem Druck,
man vermutet dass man solchen metallischen Wasserstoff im Inneren
von Jupiter und anderen Gas-Riesen finden kann.

\section{Teilchen in einem schwachen periodischen Potential}
Das im letzten Abschnitt verwendete Modell eines Festk"orpers ging davon aus,
dass sich die Elektronen innerhalb des Festk"orpers im wesentlichen
ungehindert bewegen k"onnen.
Der Festk"orper wurde im wesentlichen ein im Vergleich zum Abstand 
der Atome untereinander sehr grosser Potentialtopf betrachtet.
Ein realer Festk"orper zeichnet sich durch mehr oder weniger regelm"assige
Anordnung der Atomkerne aus, die ein entsprechen strukturiertes Potential
erzeugen, in dem sich die Elektronen des Festk"orpers bewegen.
Die m"oglichen Energieniveaus werden sich durch die Wirkung des Potentials
ver"andern, und mit ihnen m"oglicherweise die elektrischen Eigenschaften.

Ziel dieses Abschnittes ist das Energiespektrum in einem periodischen
Potential zu verstehen, insbesondere das Enstehen von Energieb"andern.
Wir verwenden dazu ein eindimensionales Modell, in dem sich die
wesentlichen Einfl"usse bereits verstehen lassen.

\subsection{Gitter}
Ein Gitter ist die von drei Basisvektoren $\vec a_i\in\mathbb R^3$
erzeugte Menge
\[
\Gamma
=
\{
n_1\vec a_1+
n_2\vec a_2+
n_3\vec a_3
\,|\,
n_i\in \mathbb Z
\}\subset{\mathbb R}^3.
\]
Die Vektoren in $\Gamma$ bezeichnen die Pl"atze, an denen sich die
Kerne der Gitteratome befinden. 

Ein eindimensionales Gitter braucht nur einen einzigen Gittervektor,
wir k"onnen auf die Vektorschreibweise verzichten.
Wir schreiben f"ur das Gitter
\[
\Gamma=\{ na\,|\,n\in\mathbb Z\}.
\]
Das von Atomen an den Gitterpunkten erzeugte Potential $V(x)$
ist Translationsinvariant, es gilt
\[
V(x+v)=V(x)\qquad\forall v\in\Gamma.
\]
Wir schreiben $T_v$ f"ur den Verschiebe-Operator, definiert durch
\[
(T_vf)(x)=f(x+v).
\]
Mit dem Verschiebeoperator l"asst sich die Periodizit"at des Potentials
durch die Gleichung $T_vV=V$ ausdr"ucken.

\subsection{Hamilton-Operator}
Der Hamilton-Operator eines Elektrons in einem Gitter hat in der Ortsdarstellung
die Form
\[
H=-\frac{\hbar^2}{2m_e}\Delta + V(x),
\]
wobei $V(x)$ eine gitterperiodische Funktion ist, also $T_vV=V$ f"ur alle
$v\in\Gamma$.
Der Operator $H$ ist translationsinvariant, was man auch durch
die Vertauschungsrelation $[H,T_v]=0$ f"ur alle $v\in\Gamma$
ausdr"ucken kann.

In einem ersten Schritt betrachten wir den Fall $V=0$.
In diesem Fall wird der Hamilton-Operator zum Hamilton-Operator eines
freien Elektrons, f"ur den wir sofort ebene Wellen als Eigenzust"ande
angeben k"onnen:
\[
\psi(x)=e^{i\vec k\cdot \vec x}
\]
Diese Eigenfunktionen von $H$ sind aber nicht gleichzeitig auch
Eigenfunktionen des Translationsoperators. Wegen $[H,T_v]=0$
sollte es aber m"oglich sein, eine Basis von gleichzeitigen
Eigenvektoren von $H$ und $T_v$ zu w"ahlen.

\subsection{Das reziproke Gitter}
Ebene Wellen haben die Form $e^{i\vec k\cdot\vec x}$.

\subsection{Folgen der Translations-Invarianz}

\subsection{Kristall-Elektronen}

\subsection{Energieb"ander}

