\chapter{Festk"orper\label{chapter:festkoerper}}
\lhead{Festk"orper}
\rhead{}
Festk"orper sind aus vielen Atomen zusammengesetzte Quantensysteme.
Im Gegensatz zu einer Fl"ussigkeit sind jedoch die Atomkerne mehr
oder weniger unverr"uckbar in einem Gitter angeordnet.
Bei elektrischen Leitern ist ein Teil der Elektronen weitgehend frei
beweglich.
Nat"urlich ist die Gitterstruktur ebenfalls die Folge der Interaktionen
zwischen den Elektronen, doch f"ur unsere Untersuchungen k"onnen wir
die Entstehung der Gitterstruktur vernachl"assigen, es einfach als
gegeben ansehen, und nur noch die Eigenschaften der frei beweglichen
Elektronen studieren.
Diese Elektronen bewegen sich in dem periodischen Potential, das von
den Atomkernen des Gitters erzeugt wird.
Jeder Atomkern erzeugt einen Potentialtopf, in dem eines oder mehrere
Elektronen gefangen sind.
Bleiben nach der Besetzung dieser lokalisierten Zust"ande noch
Elektronen "ubrig, dann haben diese mehr Energie als die Schwellen zwischen
den Potentialt"opfen hoch sind, aber nicht gen"ugend Energie, um
aus dem Festk"orper auszutreten.
Diese wie auch die Elektronen, die mit nur wenig Anregungsenergie
in einen ausgedehnten Zustand versetzt werden k"onnen, interessieren
uns in diesem Kapitel.

\section{Fermikugel}
\rhead{Fermikugel}
\begin{figure}
\centering
\includegraphics{graphics/fest-12.pdf}
\caption{Zust"ande mit vorgegebener Wellenzahl im Ortsraum (links) und
die zugeh"origen Punkte im $k$-Raum.
(Nach einer Idee von Nicola Ochsenbein und Simon Kuster, siehe auch
Kapitel~\ref{chapter:supraleitung}).
\label{skript:kraum}}
\end{figure}
In erster N"aherung betrachen wir den Festk"orper als einen
sehr grossen Potentialkasten, aus dem die Elektronen nicht austreten
k"onnen.
Die gebundenen Elektronen haben so tiefe Energie, dass wir davon
ausgehen k"onnen, dass diese Zust"ande immer besetzt sind.
Zur Leitung tragen sie weiter nichts bei, wir brauchen also f"ur
doe Modellierung von Leitungsph"anomenen nur diejenigen Elektronen
zu ber"ucksichtigen, die beweglich genug sind, sich im ganzen
Festk"orper bewegen zu k"onnen.
Weiter nehmen wir an, dass sich die Elektronen gegenseitig 
nicht beeinflussen.
Jedes Elektron bewegt sich im Potential, welches von den Atomkernen
und allen anderen Elektronen erzeugt wird.
Jedes Elektron sieht also im wesentlichen den gleichen Potentialkasten.
Die Energieniveaus f"ur die Zust"ande in einem Potentialkasten haben
wir in (\ref{skript:3dzustaende}) bereits berechnet.
\begin{figure}
\centering
\includegraphics{graphics/fest-1.pdf}
\caption{Fermi-Kugel im Raum der Impuls-Zust"ande eines Elektrons in einem
zweidimensionalen Potentialkasten.
\label{skript:fermi-kugel}}
\end{figure}
Nat"urlich m"ussen sich die Elektronen ausserdem an das Pauli-Prinzip
halten: jeder Energiezustand kann mit h"ochstens zwei Elektronen
besetzt sein. 
Im Zustand minimaler Energie ist der Festk"orper also, wenn sich
alle Elektronen in Zust"anden befinden, die einer Gleichung
\[
\frac{h^2}{32ml^2}(
n_x^2
+
n_y^2
+
n_z^2
)
<
E_F
\]
gen"ugen.
Die tats"achlich besetzten Elektronenzust"ande bilden also eine Kugel
im Raum der m"oglichen Elektronenzust"ande, die Fermi-Kugel.
Je gr"osser die Dichte der frei beweglichen Elektronen ist, desto
h"oher ist auch die Fermi-Energie $E_F$.

% XXX Wellenzahl

\subsection{Periodische Randbedingungen}
Das Modell des Potentialkastens f"uhrt auf mathematisch sehr unangenehme
Randbedingungen, die sich in speziellen Randeffekten "aussern.
Solange unser Ziel ist, vor allem das Innere des Festk"orpers zu verstehen,
sind wir frei, andere Randbedingungen zu w"ahlen, solange Sie im Wesentlichen
das gleiche Energiespektrum ergeben. Erst wenn der Festk"orper im Vergleich
zu den Atomabst"anden klein wird, werden die Randeffekte "uberhand nehmen,
dies ist zum Beispiel bei Quantum-Dots der Fall. 

Ein besonders einfaches Modell ist, sich den Festk"orper unendlich 
ausgedehnt vorzustellen. Diese Modell kann aber nicht zutreffen, denn
es f"uhrt dazu, dass wir mit unendlich vielen Elektronen arbeiten m"ussen.
Die m"oglichen Wellenzahlen $k_x$, $k_y$ und $k_z$ w"urden stetig, die
diskrete Natur der Quantenmechanik geht in diesem Modell verloren.

Wir m"ussen dem unendlichen Festk"orpermodell also zus"atzlich Bedingungen
aufzwingen, die die diskrete Natur zur"uckbringen. Die Fourier-Transformation
auf $\mathbb R$ f"uhrt uns in einen stetigen Frequenzraum, wenn wir
aber nur die periodischen Funktionen auf $\mathbb R$ betrachten, sind
nur noch diskrete Frequenzen m"oglich.
Wir fordern daher, dass die Wellenfunktionen periodisch sind,
\begin{align*}
\psi(x-l, y  ,z  )&=\psi(x+l,y  ,z  ),
\\
\psi(x  , y-l,z  )&=\psi(x  ,y+l,z  ),
\\
\psi(x  , y  ,z-l)&=\psi(x  ,y  ,z+l).
\end{align*}
Elektronen mit dem Hamilton-Operator 
\[
H=\frac{p^2}{2m_e}=-\frac{\hbar^2}{2m_e}\Delta
\]
haben als Wellenfunktionen ebenen Wellen
\[
|\vec k\rangle
=
\psi_{\vec k}(\vec x)
=
e^{\frac{i}{\hbar}\vec k\cdot \vec x}
\]
mit Energie
\[
E_{\vec k}=\frac{\hbar^2\vec k^2}{2m_e}.
\]
Die periodischen Randbedingungen verlangen, dass $2l k_x$, $2lk_y$ und $2lk_z$
Vielfache von $2\pi$ sind, dass es also Zahlen $n_x$, $n_y$ und $n_z$ gibt mit
\[
k_x=\frac{2\pi n_x}{2l}=\frac{\pi n_x}{l},\qquad
k_y=\frac{2\pi n_y}{2l}=\frac{\pi n_y}{l}\qquad\text{und}\qquad
k_z=\frac{2\pi n_z}{2l}=\frac{\pi n_z}{l}.
\]
Die m"oglichen $\vec k$-Zust"ande bilden also wieder ein Gitter im
$\vec k$-Raum wie in Abbildung~\ref{skript:fermi-kugel}, und das Modell
der Fermi-Kugel ist auch f"ur diese Randbedingungen angemessen.

\subsection{Gebundene Elektronen\label{skript:gebundeneelektronen}}
\begin{figure}
\centering
\includegraphics{graphics/fest-2.pdf}
\caption{Gebundene Niveaus in einem Kristall
\label{skript:gebundene-niveaus}}
\end{figure}
\begin{figure}
\centering
\includegraphics{graphics/fest-3.pdf}
\caption{Durch Erh"ohung der Dichte werden Elektronen auf einstmals
gebundenen Niveaus frei beweglich.
\label{skript:gebundene-niveaus-komprimiert}}
\end{figure}
Das Modell der Fermi-Kugel ist nur anwendbar auf Elektronen, die sich
innerhalb des Festk"orpers im Wesentlichen frei bewegen k"onnen.
Im Allgemeinen wird nur ein Teil der Elektronen diese Bedingung erf"ullen,ischer Leiter ist.
die meisten Elektronen werden sich auf tiefen Energieniveaus in unmittelbarerischer Leiter ist.
N"ahe der Atomkerne aufhalten, wie in Abbildung~\ref{skript:gebundene-niveaus}
dargestellt.

Erst wenn gen"ugend Elektronen vorhanden sind, dass alle diese gebundenen
Niveaus zu besetzt sind, k"onnen weitere Elektronen sich Elektronen mit
noch h"oherer Energie freier bewegen.
Erst f"ur diese Elektronen ist es sinnvoll, von der Fermi-Kugel zu sprechen,
und erst wenn solche Elektronen vorhanden sind, k"onnen wir erwarten,
dass der Festk"orper ein elektrischer Leiter ist.

Wenn also gen"ugen Elektronen vorhanden sind, dass $E_F$ "uber dem
``Bodenniveaus'' des Potentialkastens liegt, dann liegt ein Leiter vor.
Diese Situation kann auch erzwungen werden.
Presst man die Atomkerne weiter zusammen, werden auch die
Coulomb-Potential-L"ocher kleiner, die Elektronen haben f"ur normale
$s$-Orbitale gar keinen Platz mehr, und stehen daher als frei bewegliche
Elektronen zur Verf"ugung.
Die Fermi-Energie steigt also an, der K"orper wird ein Leiter.
Diesen Zustand erreicht Wasserstoff unter extrem hohem Druck,
man vermutet dass man solchen metallischen Wasserstoff im Inneren
von Jupiter und anderen Gas-Riesen finden kann.

\section{Elektronen in einem periodischen Potential}
\rhead{Elektronen im periodischen Potential}
Das im letzten Abschnitt verwendete Modell eines Festk"orpers ging davon aus,
dass sich die Elektronen innerhalb des Festk"orpers im wesentlichen
ungehindert bewegen k"onnen.
Der Festk"orper wurde im wesentlichen ein im Vergleich zum Abstand 
der Atome untereinander sehr grosser Potentialtopf betrachtet.
Ein realer Festk"orper zeichnet sich durch mehr oder weniger regelm"assige
Anordnung der Atomkerne aus, die ein entsprechen strukturiertes Potential
erzeugen, in dem sich die Elektronen des Festk"orpers bewegen.
Die m"oglichen Energieniveaus werden sich durch die Wirkung des Potentials
ver"andern, und mit ihnen m"oglicherweise die elektrischen Eigenschaften.

Ziel dieses Abschnittes ist das Energiespektrum in einem periodischen
Potential zu verstehen, insbesondere das Enstehen von Energieb"andern.
Wir beschr"anken uns dabei auf ein eindimensionales Modell, in dem sich die
wesentlichen Einfl"usse bereits verstehen lassen.

\subsection{Gitter}
Ein Gitter ist die von drei Basisvektoren $\vec a_i\in\mathbb R^3$
erzeugte Menge
\[
\Gamma
=
\{
n_1\vec a_1+
n_2\vec a_2+
n_3\vec a_3
\,|\,
n_i\in \mathbb Z
\}\subset{\mathbb R}^3.
\]
Die Vektoren in $\Gamma$ bezeichnen die Pl"atze, an denen sich die
Kerne der Gitteratome befinden. 

Ein eindimensionales Gitter braucht nur einen einzigen Gittervektor,
wir k"onnen auf die Vektorschreibweise verzichten.
Wir schreiben f"ur das Gitter
\[
\Gamma=\{ na\,|\,n\in\mathbb Z\}.
\]
Das von Atomen an den Gitterpunkten erzeugte Potential $V(x)$
ist Translationsinvariant, es gilt
\[
V(x+v)=V(x)\qquad\forall v\in\Gamma.
\]
Wir schreiben $T_v$ f"ur den Verschiebe-Operator, definiert durch
\[
(T_vf)(x)=f(x+v).
\]
Mit dem Verschiebeoperator l"asst sich die Periodizit"at des Potentials
durch die Gleichung $T_vV=V$ ausdr"ucken.

Die Translationsoperatoren sind mit der Addition vertr"aglich:
\begin{align*}
T_{u+v}\psi(x)&=\psi(x+u+v)=T_u\psi(x+v)=T_uT_v\psi(x),\\
T_{nv}\psi(x)&=T_v^n\psi(x).
\end{align*}

\subsection{Hamilton-Operator}
Der Hamilton-Operator eines Elektrons in einem Gitter hat in der Ortsdarstellung
die Form
\[
H=-\frac{\hbar^2}{2m_e}\Delta + V(x),
\]
wobei $V(x)$ eine gitterperiodische Funktion ist, also $T_vV=V$ f"ur alle
$v\in\Gamma$.
Der Operator $H$ ist translationsinvariant, was man auch durch
die Vertauschungsrelation $[H,T_v]=0$ f"ur alle $v\in\Gamma$
ausdr"ucken kann.

In einem ersten Schritt betrachten wir den Fall $V=0$.
In diesem Fall wird der Hamilton-Operator zum Hamilton-Operator eines
freien Elektrons, f"ur den wir sofort ebene Wellen als Eigenzust"ande
angeben k"onnen:
\begin{equation}
\psi_{\vec k}(x)=e^{i\vec k\cdot \vec x}
\label{skript:ebenewelle}
\end{equation}
Diese Eigenfunktionen von $H$ sind aber nicht gleichzeitig auch
Eigenfunktionen des Translationsoperators. Wegen $[H,T_v]=0$
sollte es aber m"oglich sein, eine Basis von gleichzeitigen
Eigenvektoren von $H$ und $T_v$ zu w"ahlen.

\subsection{Das reziproke Gitter}
\begin{figure}
\centering
\includegraphics{graphics/fest-4.pdf}
\bigskip

\includegraphics{graphics/fest-5.pdf}
\caption{Hexagonales Gitter (oben) und dazu geh"origes reziprokes Gitter
(unten)
\label{skript:hexagonalesgitter}}
\end{figure}
Welche Vektoren $\vec k$ ergeben ebenen Wellen, die periodisch sind?
F"ur jeden Vektor $v\in\Gamma$ des Gitters muss gelten
\[
\psi_{\vec k}(x+v)=\psi_{\vec k}(x).
\]
Mit (\ref{skript:ebenewelle}) folgt daraus, dass
$\vec k\cdot v$ ein Vielfaches von $2\pi$ sein muss.
Ausgehend von einer Basis $\vec v_i$ des Gitters sind
\[
\vec k\cdot v_i=2\pi n_i
\]
linear unbh"angige Gleichungen f"ur den Vektor $\vec k$. Schreibt man
die Koeffizienten der Vekoren $v_i$ als Zeilen in eine Matrix $V$
und die Zahlen $n_i$ in als Spaltenvektore $\vec n$,
dann kann man das Gleichungssystem als
\[
V\vec k=2\pi\vec n
\]
schreiben, und man kann die L"osungen mit der inversen Matrix als
\[
\vec k = 2\pi V^{-1}\vec n
\]
finden.
Insbesondere bilden die Vektoren $\vec k$, die zu gitterperiodischen
ebenen Wellen f"uhren, selbst ein Gitter, welches von den Spaltenvektoren
von $2\pi V^{-1}$ aufgespannt wird.
Es heisst das {\em reziproke Gitter} $\Gamma'$ von $\Gamma$.
\index{reziprokes Gitter}%

\begin{beispiel}
Sei $\Gamma = \{ na\,|\, n\in\mathbb Z\}$ ein eindimensionales Gitter.
Es wird erzeugt von $v_1=a$.
Also wird das reziproke Gitter von $2\pi a^{-1}$ aufgespannt.
\end{beispiel}

\begin{beispiel}
Wir betrachten das zweidimensionale, rechteckige Gitter
aufgespannt von den Vektoren
\[
v_1=\begin{pmatrix}1\\0\end{pmatrix}
\qquad\text{und}\qquad
v_2=\begin{pmatrix}0\\a\end{pmatrix}.
\]
Das reziproke Gitter wird erzeugt von den Spalten von $2\pi V^{-1}$,
\[
V=\begin{pmatrix} 1&0\\ 0&a \end{pmatrix}
\qquad\Rightarrow\qquad
V^{-1}=\begin{pmatrix}1&0\\0&\frac1a\end{pmatrix}
\qquad\Rightarrow\qquad
\vec k_1=\begin{pmatrix}2\pi\\0\end{pmatrix},\quad
\vec k_2=\begin{pmatrix}0\\\frac{2\pi}a\end{pmatrix}
\]
Das reziproke Gitter ist wieder ein rechteckiges Gitter.
\end{beispiel}

\begin{beispiel}
Wir betrachten das hexonale Gitter aufgespannt von den Vektoren
\[
v_1=\begin{pmatrix}1\\0\end{pmatrix}
\qquad\text{and}\qquad
v_2=\begin{pmatrix}\frac12\\\frac{\sqrt{3}}2\end{pmatrix}.
\]
Die Inverse der Matrix $V$ ist
\[
V=\begin{pmatrix}
1&0\\
\frac12&\frac{\sqrt{3}}2
\end{pmatrix}
\qquad\Rightarrow\qquad
V^{-1}=\begin{pmatrix}
1&0\\
-\frac1{\sqrt{3}}&\frac{2}{\sqrt{3}}
\end{pmatrix}.
\]
Die Spaltenvektoren haben beide die L"ange $2/\sqrt{3}$, und der
Zwischenwinkel ist 
\[
\cos\alpha
=
\frac{
\begin{pmatrix}1\\-\frac1{\sqrt{3}}\end{pmatrix}
\cdot
\begin{pmatrix}0\\\frac{2}{\sqrt{3}}\end{pmatrix}
}{\displaystyle\frac{4}{3}}
=
\frac{3}{4}\cdot \biggl(-\frac{2}{3}\biggr)
=
-\frac12
\qquad \Rightarrow \qquad
\alpha=\frac{2\pi}3.
\]
Abbildung~\ref{skript:hexagonalesgitter} zeigt das hexagonale Gitter 
und das zugeh"orige reziproke Gitter.
\end{beispiel}
Die Beispiele illustrieren, wie l"angere Basisvektoren f"ur das Gitter
$\Gamma$ zu entsprechend k"urzeren Basisvektoren des reziproken
Gitters $\Gamma'$ f"uhren.


\subsection{Folgen der Translations-Invarianz}
Im Folgenden beschr"anken wir uns wieder auf ein eindimensionales Gitter
$\Gamma$ mit Gitterkonstante $a$.
Vektoren im Gitter $v\in\Gamma$ sind Vielfache $v=na$ der Gitterkonstanten,
und es gilt $T_{v}=T_{na}=T_a^n$.
Das reziproke Gitter $\Gamma'$ hat die Gitterkonstante $2\pi/a$.

Da der Hamilton-Operator $H$ mit allen Translationen $T_v$ vertauscht,
gibt es eine Basis aus gleichzeitigen Eigenfunktionen von $H$ und $T_v$.
Sei also $|\psi\rangle$ ein translationsinvarianter Eigenzustand mit
Wellenfunktion $\psi(x)$. Da $|\psi\rangle$ ein Eigenzustand aller $T_v$
ist, muss gelten
\[
T_v\psi(x)=\psi(x+v)=\lambda_v\psi(x)
\]
f"ur jeden Vektor $v=na$.
Weil $T_v=T_a^n$ ist, muss es eine Zahl $\lambda=\lambda_a$ geben mit
$\lambda_v=\lambda^n$.

Ausserdem gelten weiterhin die periodischen Randbedingungen,
was soviel bedeutet wie dass $T_{2l}$ der Einheitsoperator ist%
\footnote{Hier
steckt der Grund, warum wir uns f"ur die aktuelle Diskussion wieder
auf den eindimensionalen Fall beschr"anken. Die Diskussion l"asst sich
selbstverst"andlich auf ein dreidimensionales Gitter "ubertragen, aber
die Definition der periodischen Randbedingungen muss mit etwas mehr
Sorgfalt erfolgen.}.
Dies bedeutet, dass es eine ganze Zahl $n$ geben muss so,
dass $T_a^n=\operatorname{id}$. F"ur den Eigenvektor $|\psi\rangle$
von $T_a$ mit Eigenwert $\lambda$, bedeutet dies, dass $\lambda^n=1$
sein muss. Insbesondere muss $|\lambda|=1$ sein. 

Die Zahl $\lambda$ kann in der Form $\lambda=e^{iKa}$ geschrieben werden,
und wegen $\lambda^n=e^{inKa}$ muss $nKa$ ein Vielfaches von $2\pi$ sein.
$K$ ist nicht eindeutig bestimmt, denn addiert man $k\in\Gamma'$, dann ist
\[
e^{i(K+k)a}=e^{iKa}\underbrace{e^{ika}}_{=1}=e^{iKa}.
\]
$K$ ist also nur bis auf einen Vektor in $\Gamma'$ bestimmt.
Insbesondere kann man unter alle in Frage kommenden Vektoren $K$ immer
denjenigen ausw"ahlen, der am n"achsten beim Nullpunkt liegt.

Wir schreiben jetzt
\[
u(x)=e^{-iKx}\psi(x)
\qquad\Rightarrow\qquad
\psi(x)=e^{iKx}u(x).
\]
Die Wirkung des Operators $T_a$ kann man jetzt ausrechnen:
\begin{equation}
\left.
\begin{aligned}
T_a\psi(x)&=\psi(x+a)=e^{iK(x+a)}u(x+a)=\lambda e^{iKx}u(x+a)
\\
=\psi(x)&=e^{iKx}u(x)
\end{aligned}
\right\}
\qquad\Rightarrow\qquad
u(x+a)=u(x).
\end{equation}
Die Funktion $u(x)$ ist also gitterperiodisch, sie heisst auch
{\em Bloch-Funktion}.
\index{Bloch-Funktion}%

Die m"oglichen Werte von $K$ sind nicht beliebig, da $e^{iKx}$ die
periodischen Randbedingungen des Kristalls erf"ullen muss. Wenn $l=Na$
ist, dann muss $K$ ein Vielfaches von $\frac{2\pi}{Na}$ sein.
Je gr"osser der Kristall ist, desto n"aher liegen die m"oglichen
Wert von $K$ beeinander.
Wir nehmen im Folgenden an, dass $N$ sehr gross ist, so dass die
m"oglichen Werte von $K$ so nahe beeinander liegen, dass man sie in
einer graphischen Darstellung nicht mehr unterscheiden kann.

\subsection{Kristall-Elektronen}
Ein Energiezustand $|\psi\rangle$ von $H$ mit Energie $E$
kann immer beschrieben werden als ein Produkt $e^{iKx} u(x)$,
wobei $K$ ein beliebiger Wellenzahlvektor ist.
Die Wirkung des Hamilton-Operators k"onnen wir nat"urlich auch
berechnen:
\[
H\,|\psi\rangle
=
e^{iKx}\biggl(
\frac{\hbar^2K^2}{2m_e}-\frac{\hbar^2}{2m_e}\frac{\partial^2}{\partial x^2}
+V(x)
\biggr)u(x)
=
e^{iKx}E(K)u(x) + e^{iKx} Hu(x)
=
e^{iKx}Eu(x)
\]
mit
\[
E(K)=\frac{\hbar^2K^2}{2m_e}.
\]
Die Funktion $u$ ist also ein Eigenzustand des Operators $H-E(K)$. Umgekehrt
kann man aus jeder periodischen Eigenfunktion $u(x)$ von $H$ mit Energie
$E_u$ einen Eigenvektor $e^{iKx}u(x)$ von $H$ mit Energie $E_u+E(K)$, machen.

Die L"osungen $u(x)$ ist in der Umgebung jedes Gitterpunktes gleich.
Da die Gitteratome nicht unterscheidbar sind, kann ein Elektron
nicht ``feststellen'', ob es sich beim richtigen Atom befindet.
Die Bloch-Funktion $u(x)$ beschreibt also ein Elektron,
welches sich in der N"ahe jedes Kristallatoms gleich verh"alt.
Man nennt diese periodischen Eigenzust"ande von $H$ daher auch
{\em Kristall-Elektronen}
oder
{\em Bloch-Elektronen}.
\index{Kristall-Elektronen}%
\index{Bloch-Elektronen}%
Der zugeh"orige Eigenwert ist die Energie des Elektrons als ans Gitter
gebundenes Elektron.

Der Faktor $e^{iKx}$ beschreibt die Bewegung eines Kristall-Elektrons
durch das Gitter. Er gibt dem Elektron die zus"atzliche Bewegungsenergie
$E(K)$.

Man beachte, dass diese Zerlegung der Eigenzust"ande selbst dann gilt,
wenn das Potential verschwindet.
Dann sind die Funktionen $u$ besonders einfach zu beschreiben, es
sind die gitterperiodischen ebenen Wellen, also die Funktionen $e^{ik_0x}$ mit
$k_0\in\Gamma'$.
Tats"achlich l"asst sich jede ebene Welle $e^{ikx}$ schreiben als
\[
e^{ikx}=e^{i(K+k_0)x} = e^{iKx}e^{ik_0x}
\]
schreiben.

\begin{figure}
\centering
\includegraphics[width=\hsize]{graphics/fest-6.pdf}
\caption{Energieparabeln f"ur verschiedene Vektoren $K$. Die Energie
eines Zustands setzt sich zusammen aus der Energie $E(K)$ und der
Energie des zugeh"origen Kristall-Elektrons.
Die auf der $K$-Achse eingezeichneten Punkte stellen das reziproke
Gitter dar.
Grau hinterlegt die Wigner-Seitz-Zelle des reziproken Gitters.
\label{skript:dispersion}}
\end{figure}

Abbildung~\ref{skript:dispersion} zeigt die Energieparabeln in Abh"angikeit
von $K$. Da die Wahl von $K$ nur bis auf einen Summanden im reziproken
Gitter eindeutig ist, kann man einen Zustand gen"ugend hoher Gesamtenergie
auf verschiedene Arten in ein Kristall-Elektron und einen Faktor 
$e^{iKx}$ zerlegen.
In der Wigner-Seitz-Zelle des reziproken Gitters (hellgrau hinterlegt)
l"asst sich bereits
das ganze Energiespektrum der Elektronen im Festk"orper ablesen.

\subsection{Gebundene Elektronen 2.0\label{skript:gebundeneelektronen20}}
Wir haben uns im Abschnitt~\ref{skript:gebundeneelektronen20} bereits 
qualititativ Gedanken gemacht.
Mit unserer Kenntnis der Kristall-Elektronen sind wir jetzt in der Lage,
diese Diskussion wesentlich zu vertiefen.

Auch die gebundenen Elektronen lassen sich durch eine Bloch-Funktion $u(x)$ 
und eine ebene Welle $e^{iKx}$ beschreiben, $\psi(x)=e^{iKx}u(x)$.
Ein gebundenes Elektron befindet sich vor allem in der N"ahe ``seines''
Atomkerns, wo es das Potential der anderen Atomkerne kaum ``sp"urt''.
Seine Wellenfunktion sieht daher in der N"ahe des Atomkernes wie die die
Elektron-Wellenfunktion eines einzelnen Atoms aus.
Es muss also mindestens f"ur die gebundenen Zust"ande mit tiefer Energie
eine Korrespondenz zwischen Elektronwellenfunktionen eines einzelnen
Atoms und den Bloch-Funktionen des Kristalls geben. Die Bloch-Funktionen
lassen sich daher nach den Orbital-Bezeich\-nungen gruppieren, die wir
im Kapitel~\ref{chapter:wasserstoff} kennengelernt haben.

Zu jeder Bloch-Funktion $u(x)$ mit Energie $E_u$ gibt es die Wellenfunktionen
\[
\psi_{u,K}(x)=e^{iKx}u(x)
\qquad\text{mit Energie}\qquad
E=E_u+\frac{\hbar^2K^2}{2m_e},
\]
wobei $K$ nur kleine Werte durchl"auft.
Ausser dem ``Grundzustand'' mit Energie $E_u$ gibt es also weitere
Zust"ande mit gegen"uber $E_u$ um $\hbar^2K^2/2m_e$ erh"ohter Energie.
Zu jeder Bloch-Funktion geh"ort also ein ganzes Energieband, in dem man
Zust"ande mit gleicher Bloch-Funktion finden kann.
Diese B"ander kann man also wieder mit den Orbitalen des Einzelatoms
in Beziehung setzen.
Die minimale Energie des Bandes, die {\em Bandunterkante}, ist die Energie $E_u$
die maximale Energie, die {\em Bandoberkante}, wird durch den gr"osstm"oglichen
Vektor $K$ in der Wigner-Seitz-Zelle gegeben.
\index{Energieband}%
\index{Bandkante}%

\begin{figure}
\centering
\includegraphics{graphics/fest-11.pdf}
\caption{Zustandsdichte der Energieb"ander in einem eindimensionalen
Kristall (qualitativ)\label{skript:zustandsdichte1d}}
\end{figure}
\begin{figure}
\centering
\includegraphics{graphics/fest-10.pdf}
\caption{Zustandsdichte der Energieb"ander in einem dreidimensionalen
Kristall (qualitativ). Die Zustandsdichte w"achst oberhalb der
Bandkante proportional zu $(E-E_u)^{\frac12}$ an, der Abfall bei
der Bandoberkante h"angt von der Geometrie des reziproken Gitters des
Kristalles ab.
\label{skript:zustandsdichte3d}}
\end{figure}

In einem eindimensionalen Kristall sind die m"oglichen $K$-Werte
gleichverteilt, die Zustandsdichte in den B"ander ist daher eine
Rechteckfunktion (Abbildung~\ref{skript:zustandsdichte1d}).
\index{Zustandsdichte}

In einem mehrdimensionalen Kristall definiert die Geometrie der
Wigner-Seitz-Zelle des reziproken Gitters die Zustandsdichte
(Abbildung~\ref{skript:zustandsdichte3d}).
Wir stellen uns im reziproken Gitter eine Kugel zur Energie $E$ vor.
F"ur kleine Energie w"achst die Zahl der Zust"ande mit dem Volumn der
Kugel, also mit der dritten Potenz des Radius. Die Zahl der Zust"ande
w"achst proportional zu $E^{\frac32}$, die Zustandsdichte w"achst daher
proportional zu $E^{\frac12}$.
Sobald die Kugel den Rand der Wigner-Seitz-Zelle trifft kann die 
Zahl der Zust"ande aber nicht mehr gleich wachsen, die Zustandsdichte
wird daher weniger stark zunehmen oder sogar abnehmen.
Sie wird ganz verschwinden, wenn die Kugel die ganze Wigner-Seitz-Zelle
umfasst.

F"ur einen Leiter ist dieses Modell jedoch unzul"anglich.
Die Annahme, dass Elektronen an die Atomkerne gebunden sind, ist offensichtlich
falsch.
Mindestens ein Teil der Elektronen kann sich im Kristall mehr oder weniger
frei bewegen, f"ur diese Elektronen ist das Potential der Atomkerne
als schwach anzusehen.

\subsection{Bragg-Reflexionen}
Wir erwarten in einem Leiter ausgedehnte Elektronen-Wellenfunktionen
"ahnlich der ebenen Wellen, die von den Atomkernen nur leicht modifiziert
werden.
Nach dem Huygensschen Prinzip erwarten wir, dass die Wellen an den
Atomkernen gestreut werden, und es durch das Zusammenwirken einer grosser
Zahl von Atomkernen zu Reflexionen kommt.
Wir wollen den Zusammenhang zwischen solchen Reflexionen und der Geometrie
des reziproken Gitters kl"aren.
%Solche Reflexionen heissen {\em Bragg-Reflexionen}.

\subsubsection{Netzebenen}
Zu einem gegebenen Vektor $\vec K\in \Gamma'$ im reziproken Gitter ist
jede der Mengen
\[
\sigma(\vec K,N)=\{ \vec x\in \mathbb R^3\,|\, \vec K\cdot \vec x=2\pi N\}
\]
eine Ebene.
Diese Ebenen heissen {\em Netzebenen} senkrecht auf $\vec K$.
\index{Netzebene}%
Nach Definition des reziproken Gitters sind alle Gitterpunkte in
einer Netzebene enthalten. 
Es ist aber durchaus m"oglich, dass $\sigma(\vec K,N)$ nicht
f"ur jedes $N$ "uberhaupt Gitterpunkte erh"alt.

\begin{beispiel}
In einem eindimensionalen Gitter $\Gamma$ mit Gitterkonstante $a$ besteht das
reziproke Gitter $\Gamma'$ aus den Vielfachen von $2\pi/a$.
Mit $K=2\cdot 2\pi/a$ gilt f"ur die Punkte des Gitters
\[
K\cdot x=2\frac{2\pi}a\cdot na=2n\cdot 2\pi,
\]
insbesondere gibt es keine Gitterpunkte, f"ur die der Wert $2\pi$
erreicht wird.
Die Ebenen $\sigma(\vec K,1)$ enth"alt daher keine Gitterpunkte.
\end{beispiel}

%Atomkerne in einem Gitter sind in der Lage, Wellen zu streuen.
Da ein Kristall aus vielen Netz\-ebenen mit Gitteratomen besteht,
m"ussen die an verschiedenen Netzebenen reflektierten Wellen konstruktiv
interferieren, andernfalls kann sich die reflektierte Welle im
Kristall nicht ausbreiten.
Eine solche Reflexion nennt man {\em Bragg-Reflexion}.
\index{Bragg-Reflexion}%
Die Wellenzahlvektoren $\vec k$ und $\vec k'$  zweier ebener Wellen,
die durch Bragg-Reflexion auseinander hervorgehen, m"ussen zwei
Bedingunge erf"ullen:
\begin{enumerate}
\item Spiegelungsbedingung: $\vec k'$ muss die Spiegelung des Vektors
$\vec k$ an der Ebene senkrecht einen Vektor $\vec K\in\Gamma'$ sein.
\item Phasenbedingung: Der durch die Reflexion an unterschiedlichen
Netzebenen verursachte Wegunterschied muss ein Vielfaches der Wellenl"ange
sein.
\end{enumerate}

\subsubsection{Spiegelungsbedingung}
\begin{figure}
\centering
\includegraphics{graphics/fest-7.pdf}
\caption{Herleitung der Formeln f"ur die Spiegelung eines Vektors $\vec k$
an einer Ebene mit Normale $\vec K$.
\label{skript:spiegelungsbedingung}}
\end{figure}
Sei jetzt $\vec K\in\Gamma'$ ein reziproker Gittervektor, der eine
Schar von Netzebenen beschreibt.
Wir schreiben $\vec K^0=\vec K/|\vec K|$ f"ur den Einheitsvektor mit
Richtung $\vec K$.
In Abbildung~\ref{skript:spiegelungsbedingung} wird die Bedingung f"ur die
Spiegelung dargestellt. 
Der Vektor $\vec k_{\perp}$ ist die Orthogonalprojektion von $\vec k$
auf den Vektor $\vec K$, es gilt
\begin{align*}
\vec k_{\perp}&=(\vec k\cdot\vec K^0)\vec K^0,
\\
\vec k_{\|}&=\vec k -\vec k_{\perp} = \vec k-(\vec k\cdot \vec K^0)\vec K^0.
\end{align*}
Der gespiegelte Wellenzahlvektor ist dann
\begin{equation}
\vec k'=\vec k_{\|}-\vec k_{\perp}=\vec k-2(\vec k\cdot\vec K^0)\vec K^0.
\end{equation}
Die Differenz der beiden Wellenzahlvektoren ist
\begin{equation}
\vec k- \vec k'=2(\vec k\cdot\vec K^0)\vec K^0,
\label{skript:wellenzahldifferenz}
\end{equation}
also ein Vielfaches von $\vec K^0$.
Es ist klar, dass die Spiegelungsbedingung auch zur Folge hat, dass
$|\vec k|=|\vec k'|$.

\subsubsection{Phasenbedingung}
\begin{figure}
\centering
\includegraphics{graphics/fest-8.pdf}
\caption{Phasenbedingung f"ur ein Bragg-Reflexion
\label{skript:phasenbedingung}}
\end{figure}
Die Phasenbedingung verlangt, dass sich die Wegl"angen f"ur die zwei in
Abbildung~\ref{skript:phasenbedingung} dargestellten Wege um ein
ganzahliges Vielfaches der Wellenl"ange unterscheiden.
Der Wegl"angenunterschied ist $2s$, dabei ist $s$ die Projektion des
Vektors $\vec d$ auf den Vektor $\vec k$. Der Abstand der Netzebenen
ist $d=2\pi / |\vec K|$,
\[
s
=
\frac{\vec k\cdot \vec d}{|\vec k|}
=
\frac{\vec k\cdot \vec K^0}{|\vec k|}\frac{2\pi}{|\vec K]}.
\]
Die Phasenbedingung lautet jetzt, dass $2s$ ein Vielfaches der Wellenl"ange
$\lambda = 2\pi/|\vec k|$ sein muss, also
\begin{align}
2s
=
2\cdot \frac{\vec k\cdot \vec K^0}{|\vec k|}\frac{2\pi}{|\vec K|}
&=
\lambda N
=
\frac{2\pi}{|\vec k|} N
\notag
\\
2\cdot \frac{\vec k\cdot \vec K^0}{|\vec K|}
&=
N
\notag
\\
\vec k\cdot \vec K^0=\frac{N}2 |\vec K|
\label{skript:kK0skalarprodukt}
\end{align}
Setzen wir den Wert~(\ref{skript:kK0skalarprodukt}) in die Formel
(\ref{skript:wellenzahldifferenz})
f"ur die Wellenzahldifferenz ein, finden wir
\[
\vec k-\vec k'=2 \frac{N}2 |\vec K|\vec K^0=N\vec K.
\]
Damit haben wir im Wesentlichen folgendes Resultat hergeleitet:
\begin{satz}
\label{skript:braggsatz}
Die Welle mit Wellenzahl $\vec k'$ geht aus der Welle mit Wellenzahl
$\vec k$ durch
Bragg-Reflexion an der durch $\vec K\in\Gamma'$ definierten Netzebene
hervor, wenn $\vec k-\vec k'$ ein Vielfaches von $\vec K$ ist und
wenn $|\vec k|=|\vec k'|$.
\end{satz}

Als Spezialfall weisen wir noch darauf hin, dass f"ur ein eindimensionales
Gitter die Bedingung des Satzes nur erf"ullt werden kann, wenn
$k=-k'$ und $2k=K\in\Gamma'$ gilt.

\begin{proof}[Beweis]
Die Phasenbedingung ist unter den Voraussetzungen des Satzes erf"ullt,
es ist aber nicht klar, dass auch die Reflexionsbedingung erf"ullt ist.
Wenn $|\vec k|=|\vec k'|$, dann ist das Dreieck $\vec k$, $\vec k'$
und $\vec k- \vec k'$ ein gleichseitigs Dreieck, also sind die Winkel
zwischen $\vec k$ bzw.~$\vec k'$ und $\vec K$ gleich gross, und die
Reflexionsbedingung ist erf"ullt.
\end{proof}

\subsection{Schwaches Potential}
In den bisherigen "Uberlegungen haben wir das Potential $V(x)$ der Gitteratome
vernachl"assigt.
Wir wollen jetzt den Fall eines schwachen Potentials untersichen.
Auch dies ist noch kein realisitisches Modell f"ur alle Elektronen im 
Kristall. Elektronen mit geringer Energie werden sich bevorzugt in der
N"ahe der Atomkerne aufhalten, und entsprechend ein starkes Potential
sp"uren.
Dadurch schirmen sie die positive Kernladung wenigstens zum Teil ab, 
so dass die Elektronen mit hoher Energie nur noch ein stark reduziertes
Feld sp"uren. Es ist also plausibel, dass die N"aherung eines schwachen
Feldes f"ur Elektronen mit hoher Energie anwendbar ist.

Ein freies Elektron mit Wellenzahl $k$ hat Energie $\hbar^2k^2/2m_e$, und wir
k"onnen die Wellenfunktion zerlegen in ein Kristall-Elektron und einen
Faktor $e^{iKx}$.
Das Kristall-Elektron ist ebenfalls eine ebene Welle.
Unter der Wirkung eines schwachen Potentials ist die Zerlegung in
Kristall-Elektron $u(k,x)$ und eine ebene Welle $e^{iKx}$.
Der Beitrag der Energie vom Faktor $e^{iKx}$ ist immer noch $\hbar^2K^2/2m_e$,
aber die Gesammtenergie $E(k)$ wird wohl von $\hbar^2k^2/2m_e$ abweichen.
Ziel ist, die Funktion $E(k)$ zu berechnen.

Wir wollen die Wellenfunktion des Kristall-Elektrons berechnen 
f"ur eine gegebene Wellenzahl $k$.
Da die Funktion $u(x)$ periodisch ist, k"onnen wir sie in eine Fourier-Reihe
entwickeln, wir nennen die Koeffizienten $u_k$:
\[
u(x)=\sum_{k'\in\Gamma'} u_{k'} e^{ik'x}.
\]

Die Wellenfunktion $\psi(k,x)$ kann mit den Koeffizienten $u_{k'}$ als
\[
\psi(k,x)
=
e^{ikx}\sum_{k'\in\Gamma'}u_{k'}e^{ik'x}
=
\sum_{k'\in\Gamma'}u_{k'}e^{i(k+k')x}.
\]
ausgedr"uckt werden.
F"ur $k'=0$ erhalten wir eine ebene Welle.
Da wir von einem schwachen Potential ausgehen, darf die Wellenfunktion
nicht stark von der ebenen Welle abweichen. Wir k"onnen also davon ausgehen,
dass $u_0$ nahe bei $1$ ist und dass die Koeffizienten $u_{k'}$
mit $k'\ne 0$ sind also wesentlich kleiner als $u_0$.

Das periodische Potential $V(x)$ kann ebenfalls in eine Fourier-Reihe
entwickeltw werden, wir schreiben
\[
V(x) = \sum_{k'\in\Gamma'}V_ke^{ik'x}.
\]
Wir nehmen an, dass der Mittelwert von $V(x)$ verschwindet, dass also $V_0=0$.

Damit haben wir alles in Fourier-Koeffizienten entwickelt, und k"onnen die
Ausdr"ucke in die Schr"odingergleichung einsetzen:
\begin{align}
H\psi(k,x)
=
-\frac{\hbar^2}{2m_e}\frac{\partial^2}{\partial x^2} \psi(k,x)
+V(x)\psi(k,x)
&=E(k)\psi(k,x)
\notag
\\
\sum_{k'\in\Gamma'}\biggl(
\frac{\hbar^2}{2m_e}(k+k')^2
+
\sum_{k''\in\Gamma'}V_{k''}e^{ik''x}
\biggr)u_{k'}e^{i(k+k')x}
&=
E(k)\sum_{k'\in\Gamma'} u_{k'}e^{i(k+k')x}
\notag
\\
\sum_{k'\in\Gamma'}\biggl(
\frac{\hbar^2}{2m_e}(k+k')^2
-E(k)
+
\sum_{k''\in\Gamma'}V_{k''}e^{ik''x}
\biggr)u_{k'}e^{i(k+k')x}
&=
0
\label{skript:periodicschroedinger}
\end{align}
Die innere Summe hindert uns daran, direkt einen Koeffizientenvergleich
anzustellen.
Wir formen daher den Potentialterm wie folgt um
\begin{equation*}
\sum_{k'\in\Gamma'}
\sum_{k''\in\Gamma'}V_{k''}u_{k'}e^{i(k+k'+k'')x}
=
\sum_{\tilde k\in\Gamma'}
\sum_{k''\in\Gamma'}V_{\tilde k-k''}u_{k''}e^{i(k+\tilde k)x}.
\end{equation*}
Dabei haben wir $\tilde k=k'+k''$ gesetzt, $k'=\tilde k-k''$.
Schreiben wir jetzt wieder $k'=\tilde k$, dann k"onnen wir die
Gleichung~(\ref{skript:periodicschroedinger}) vereinfachen zu
\begin{align*}
\sum_{k'\in\Gamma'}\biggl(
\frac{\hbar^2}{2m_e}(k+k')^2
-E(k)
\biggr)u_{k'}e^{i(k+k')x}
+
\sum_{k'\in\Gamma'}\sum_{k''\in\Gamma'}V_{k'-k''}u_{k''}e^{i(k+k')x}
&=
0
\\
\sum_{k'\in\Gamma'}\biggl(
\biggl(
\frac{\hbar^2}{2m_e}(k+k')^2
-E(k)
\biggr)u_{k'}
+
\sum_{k''\in\Gamma'}V_{k'-k''}u_{k''}
\biggl)
e^{i(k+k')x}
&=
0.
\end{align*}
Koeffizientenvergleich besagt nun, dass der grosse Klammerausdruck
verschwinden muss f"ur jeden Vektor $k'\in\Gamma'$:
\begin{equation}
\biggl(
\frac{\hbar^2}{2m_e}(k+k')^2
-E(k)
\biggr)u_{k'}
+
\sum_{k''\in\Gamma'}V_{k'-k''}u_{k''}
=
0.
\label{skript:ukgleichung}
\end{equation}
Aus dieser Gleichung k"onnen wir zwar die $u_{k'}$ nicht exakt
bestimmen, aber es sollten sich mindestens in erster Ordnung
Absch"atzungen f"ur die Koeffizienten und vor allem f"ur $E(k)$
machen lassen.

\subsubsection{Bestimmung der $u_{k'}$ in erster N"aherung}
Die Energie $E(k)$ wird in erster N"aherung mit der Energie f"ur das
freie Elektron "ubereinstimmen.
Die Koeffizienten $u_{k''}$ sind bis auf $u_0$ in erster N"aherung
vernachl"assigbar, damit wird die Gleichung~(\ref{skript:ukgleichung})
zu
\begin{align*}
\frac{\hbar^2}{2m_e}((k+k')^2 -k^2)
u_{k'}
+
V_{k'}
&=
0,
\end{align*}
die wir nach $u_{k'}$ aufl"osen k"onnen
\begin{align*}
u_{k'}
&=
-
\frac{\displaystyle V_{k'}}{
\displaystyle
\frac{\hbar^2}{2m_e}((k+k')^2 -k^2)
}.
\end{align*}
Solange der Nenner klein ist, werden auch die Koeffizienten $u_{k'}$
klein sein.
Diese N"aherung wird aber nicht mehr funktionieren, wenn $(k+k')^2\simeq k^2$
wird.
Nach Satz~\ref{skript:braggsatz} ist dies genau die Bedingung daf"ur,
dass die Wellen mit Wellenzahl $k$ und $k+k'$ durch eine Braggreflexion
an der Netzebene mit Normale $\vec k'$ auseinander hervorgehen.

Diese Situation tritt ein, wenn $k\simeq -n\pi/a$ und $k'\simeq 2n\pi/a$.
In diesem Fall k"onnen wir auch nicht mehr annehmen,
dass $E(k)=\hbar^2k^2/2m_e$.

%\subsubsection{Bragg-Reflexionen}
%Die Bedingung $(k+k')^2\simeq k^2$ f"uhrt auf $k=\mp\pi n/a$ und
%$k'=\pm 2\pi n/a$.
%Die ebene Welle zur Wellenzahl $-\pi n/a$ ist die reflektierte 
%Welle mit der Wellenzahl $\pi n/a$.
%Dies sind die einzigen Wellen, deren Refklexionen konstruktiv interferieren
%k"onnen, sie heissen Bragg Reflexionen.
%Das Intervall $[-\frac{\pi}a,\frac{\pi}a]$ heisst Brillouin-Zone.
%
%Ich in einem dreidimensionalen Kristall treten solche Reflexionen auf.
%Eine ebene Welle mit Wellenzahlvektor $\vec k$ hat als Bragg-refkletiert
%Welle jene mit Wellenzahl $\vec k'$, wenn $|\vec k|=|\vec k'|$
%und die Differenz ein Vektor des reziproken Gitters ist
%$\vec k-\vec k'\in\Gamma'$.

\subsubsection{$E(k)$ in der N"ahe von Bragg-Reflexionen}
Bragg-Reflexionen verst"arken die Komponenten $u_{\pm K}$,
mit $k'=\frac{2\pi n}a$, man kann also diesen Koeffizienten nicht mehr
als klein voraussetzen.
In den Gleichungen~\ref{skript:ukgleichung} m"ussen wir jetzt weitere
Terme hinzunehmen.

Wir schreiben $K=2\pi n/a$.
Sei $k$ ein Wellenzahl in der N"ahe von $-\pi n/a$, eine Wellenzahl, welche
zu Bragg-Reflexionen f"uhrt.
Die Gleichungen~(\ref{skript:ukgleichung})
f"ur $k'=0$ und $k'=K$ sind:
\begin{equation}
\begin{aligned}
\biggl(
\frac{\hbar^2}{2m_e}k^2
-E(k)
\biggr)u_{0}
+
V_{-K}u_{K}
&=
0,
\\
\biggl(
\frac{\hbar^2}{2m_e}(k+K)^2
-E(k)
\biggr)u_{K}
+
V_{K}u_{0}
&=
0.
\end{aligned}
\label{skript:u0ukgleichungen}
\end{equation}
Da $V(x)$ reell ist, sind die Fourier-Koeffizienten zu entgegengesetzten
$k'$ konjugiert komplex, also $V_{-k'}=\bar V_{k'}$.
Ausserdem ist $k$ so gew"ahlt, dass $(k+K)^2=k^2$ ist.
Die Gleichungen~(\ref{skript:u0ukgleichungen}) bilden ein lineares
homogenes Gleichungssystem f"ur $u_0$ und $u_K$.
Wir schreiben es in vertrauterer Form:
\begin{equation}
\begin{linsys}{3}
\displaystyle \biggl(
\frac{\hbar^2}{2m_e}k^2
-E(k)
\biggr)u_{0}
&
+
&
\bar V_{K}u_{K}
&=&
0\phantom{.}
\\
V_{K}u_{0}
&+&
\displaystyle \biggl(
\frac{\hbar^2}{2m_e}k^2
-E(k)
\biggr)u_{K}
&=&
0.
\end{linsys}
\end{equation}
Es kann nur dann
eine nichttriviale L"osung haben, wenn die Determinate verschwindet:
\[
\left|
\begin{matrix}
\displaystyle\frac{\hbar^2}{2m_e}k^2
-E(k)
&
\bar V_{K}
\\
V_{K}
&
\displaystyle\frac{\hbar^2}{2m_e}k^2
-E(k)
\end{matrix}
\right|
=
\biggl(
\frac{\hbar^2}{2m_e}k^2-E(k)
\biggr)^2-|V_K|^2=0.
\]
Aufl"osen nach $E(k)$ ergibt
\begin{equation}
E(k)=\frac{\hbar^2}{2m_e}k^2\pm|V_K|.
\label{skript:Ekbaender}
\end{equation}
\begin{figure}
\centering
\includegraphics[width=\hsize]{graphics/fest-9.pdf}
\caption{Gem"ass (\ref{skript:Ekbaender}) Modifizierte Energieparabeln
mit verbotenen B"andern.
Grau hinterlegt die Bereiche, in denen die Energie m"oglicher 
Eigenzust"ande des Hamilton-Operators liegen k"onnen.
\label{skript:baenderstruktur}}
\end{figure}%
Das vormals stetige Energiespektrum zerf"allt jetzt in ein Reihe von
B"andern.

\section{Hartree-Fock-N"aherung}
\rhead{Hartree-Fock-N"aherung}
Bisher wurden immer nur einzelne Elektronen betrachtet, in einem 
Festk"orper wirkt jedoch eine grosse Zahl von Elektronen zusammen.
Ein realistischeres Modell muss alle Elektronen modellieren, und
insbesondere deren Wechselwirkung ber"ucksichtigen.

\subsection{Elektronen-Wechselwirkung}
Wir suchen Hamilton-Operator und Wellenfunktionen eines Systems mit einer
grossen Zahl $N$ von Elektronen, die sich in einem periodischen Gitter mit dem
Potential $V(x)$ bewegen k"onnen.
Wir bezeichnen die Koordinaten der einzelnen Elektronen mit $x_k$, jedes
$x_k$ ist eine 3-dimensionale Vektor.
Die gemeinsame Wellenfunktion aller Elektronen h"angt von allen
Elektronen-Koordinaten ab, wir schreiben sie
\[
\Psi(x_1,x_2,\dots,x_N).
\]
Der zugeh"orige Hamilton-Operator beschreibt einerseits die Bewegung
jedes einzelnen Elektrons im Gitter, jedes hat seinen eigenen
Hamilton-Operator
\[
H_k=\frac{p_k^2}{2m_e}+V(x_k).
\]
Die Energie aller Elektronen ist
\[
H_0= \sum_{k=1}^N H_k
\]
Andererseits muss er die Wechselwirkung aller Elektronen beschreiben.
Wir nehmen dabei an, dass sich die Elektronen so langsam bewegen, dass
es gen"ugt, die Coulomb-Wechselwirkung zu betrachten, also die den
Operator
\[
H_{\text{int}}
=
\frac12\sum_{k\ne k'}\frac{1}{4\pi\varepsilon_0}\frac{e^2}{|x_k-x_{k'}|}.
\]
Der Faktor $\frac12$ vor der Summe kommt daher, dass wir jedes Paar
$k,k'$ zweimal in der Summe gez"ahlt haben.
Der Grunzustand des Kristalls ist eine Wellenfunktion $\Psi(x_1,\dots,x_N)$
minimaler Energie.
Es scheint allerdings ziemlich aussichtslos, das Eigenwertproblem
\begin{equation}
H=H_0+H_{\text{int}}
\qquad \Rightarrow\qquad
H\,|\Psi\rangle = E\,|\Psi\rangle
\label{skript:multielektronewproblem}
\end{equation}
direkt zu l"osen. Wir werden uns daher mit einer N"aherungsl"osung 
begn"ugen m"ussen.

\subsection{Grundzustand als Minimalproblem}
Sei $A$ eine beliebige selbstadjungierte Matrix mit ausschliesslich
positiven Eigenwerten.
Zur Vereinfachung nehmen weiter an, dass der kleinste Eigenwert nicht 
entwartet ist.

\begin{satz}
Der Einheitsvektor $v$, der das Skalarprodukt $v^*Av$ minimiert,
ist ein Eigenvektor zum kleinsten Eigenwert von $A$.
\label{skript:ewminimalprinzip}
\end{satz}

\begin{proof}[Beweis]
Sei $v_0$ der Eigenvektor der L"ange $1$ zum minimalen Eigenwert $\lambda_0$,
dann gilt
\[
v_0^*Av_0=v_0^*\lambda_0 v_0=\lambda_0 v_0^*v_0=\lambda_0 \|v_0\|^2=\lambda_0,
\]
das Minimum von $v^*Av$ ist also h"ochstens so gross wie der kleinste
Eigenwert von $A$.

Sei jetzt $v$ ein Einheitsvektor, f"ur den $v^*Av$ den minimalen Wert 
$\lambda$ annimmt.
Wir m"ussen zeigen, dass $Av=\lambda v$ ist. 
Dazu berechnen wir
\begin{align*}
f(t)
&=
\frac{(v+ty)^*A(v+ty)}{(v+ty)^*(v+ty)}
\\
&=
\frac{v^*Av+t(y^*Av+v^*Ay)+t^2y^*Ay}{v^*v+t(y^*v+v^*y)+t^2v^*v}.
\end{align*}
Das Minimum von $f$ muss f"ur $t=0$ angenommen werden, also muss
die Ableitung $f'(t)$ f"ur $t=0$ verschwinden:
\begin{align*}
f'(0)
&=
\frac{(y^*Av+v^*Ay)v^*v + v^*Av(y^*v+v^*y)}{(v^*v)^2}
\\
&=
\frac{y^*Av+v^*Ay + \lambda(y^*v+v^*y)}{v^*v}
\\
&=
2\frac{v^*Ay-\lambda v^*y}{v^*v}
\\
&=
2\frac{(Av-\lambda v)^*y}{v^*v}=0.
\end{align*}
Da dies f"ur jeden Vektor $y$ gilt, folgt $Av-\lambda v$, also $Av=\lambda v$.
Damit ist gezeigt, dass $v$ ein Eigenvektor sein muss.
\end{proof}

Im allgemeinen sind quantenmechanische Zustandsr"aume unendlichdimensionale
Hilbertr"aume, wir k"onnen aber argumentieren dass die Zust"ande mit grosser
Energie ohnehin nicht physikalisch realsierbar sind, und dass wir daher
jedes Problem durch ein endlichedimensionales Problem approximieren k"onnen.
Allerdings n"utzt uns das nichts, denn wir m"ochten gerne in der
Ortsdarstellung arbeiten, also im Hilbertraum $L^2({\mathbb R}^3)$,
der zweifellos unendlich dimensional ist.

Wir werden daher im folgenden einfach annehmen, dass ein Minimalprinzip
wie Satz~\ref{skript:ewminimalprinzip} f"ur unsere Hamilton-Operatoren gilt.

\subsection{Hartree-Gleichung}
Wir versuchen jetzt den Grundzustand des Hamilton-Operators $H$ in
(\ref{skript:multielektronewproblem}) zu approximieren.
Dies allein vereinfacht das Eigenwertproblem noch nicht. 
Zus"atzlich m"ochten wir eine L"osung, die uns gestattet,
die Wahrscheinlichkeitsverteilung eines Elektrons als unabh"angig von 
der aktuellen Position der anderen Elektronen zu betrachten.
Dazu muss die Eigenfunktion $\Psi$ von der Form
\[
\Psi(x_1,x_2,\dots,x_N)=\varphi_1(x_1)\varphi_2(x_2)\dots\varphi_N(x_N)
\]
sein.

Wir suchen jetzt eine Wellenfunktion dieser Form, die ausserdem 
$\langle\Psi|\,H\,|\Psi\rangle$ minimiert:
\begin{align}
\langle\Psi|\,H\,|\Psi\rangle
&=
\sum_{k}\biggl\langle \varphi_k\biggl|\,
\frac{p_k^2}{2m_e}+V(x_k)
\,\biggr|\varphi_k\biggr\rangle
+
\frac12
\sum_{k\ne k'}\biggl\langle \varphi_k\varphi_{k'}\biggl|\,
\frac{e^2}{4\pi\varepsilon_0}\frac{1}{|x_k-x_{k'}|}
\,\biggr|\varphi_k\varphi_{k'} \biggl\rangle
\notag
\\
&=\sum_k\langle\varphi_k|\,H_k\,|\varphi_k\rangle
+
\frac12
\sum_{k\ne k'}\biggl\langle \varphi_k\varphi_{k'}\biggl|\,
\frac{e^2}{4\pi\varepsilon_0}\frac{1}{|x_k-x_{k'}|}
\,\biggr|\varphi_k\varphi_{k'} \biggl\rangle
\label{skript:hffunktional}
\end{align}
Man beachte, dass diese Notation nicht ganz eindeutig ist. Die Schreibweise
der Bra- und Ket-Vektoren macht keine Voraussetzungen dar"uber, wie die
zugeh"origen Variablen heissen sollen. Das Symbol $x_k$ in
(\ref{skript:hffunktional}) ist also zu lesen als ``diejenige Variable,
die als Argument der Funktion $\varphi_k$ verwendet wird''.

Ausserdem m"ussen die Funktionen $\varphi_k$ normiert sein, es muss
also $\langle\varphi_k|\varphi_k\rangle=1$ gelten f"ur alle $k$.
Wir k"onnen dies auch als $\langle\varphi_k|\varphi_k\rangle-1=0$
schreiben.

\subsubsection{Minimalproblem}
Wir haben also ein Minimalproblem f"ur die Funktion
\[
F(\varphi_1,\dots,\varphi_N)=\langle\Psi|\,H\,|\Psi\rangle
\]
mit Nebenbedingungen
\[
G_k(\varphi_1,\dots,\varphi_N)=\langle\varphi_k|\varphi_k\rangle-1
\]
zu l"osen.
Es handelt sich zwar wieder um ein unendlichdimensionales Problem,
doch wir werden es behandeln, also ob es ein endlichdimensionales Problem
w"are, mit der gleichen heuristischen Argumentation wie in fr"uheren F"allen.

\subsubsection{Variation}
Die Methode der Wahl f"ur die L"osung solcher Probleme ist die Methode
der Lagrange-Multiplikatoren. 
Variabel sind die Funktionen $\varphi_k$, wir m"ussen also zun"achst 
kl"aren, wie die Ableitungen nach $\varphi_k$ zu interpretieren sind.

Ist eine $\delta\varphi_j$ eine Funktion von $x_j$, und $f$ eine Funktion von
$\varphi_j$, dann schreiben wir $\delta f(\varphi_j)$ f"ur die
Ableitung
\[
\delta f(\varphi_k)
=
\frac{d}{dt}f(\varphi_k+t\delta\varphi_k)\bigg|_{t=0}.
\]
Diese Notation ist jedoch wenig geeignet f"ur den Fall einer Funktion
$f(\varphi_1,\dots,\varphi_N)$, die von mehreren der Funktion $\varphi$
abh"angt.
Die Notation zeigt weder, welche der Funktion abgeleitet wird,
noch in welche Richtung die Ableitung erfolgt.

Wir schreiben daher $\delta_j$ f"ur die Ableitung nach der Funktion $\varphi_j$,
\[
\delta_j f(\varphi_,\dots,\varphi_N)
=
\frac{d}{dt}f(\varphi_1,\dots,\varphi_j+\delta\varphi_j,\dots,\varphi_N)\bigg|_{t=0},
\]
so wird wenigstens die Funktion angezeigt, nach der abgeleitet wird.
Wir nennen $\delta_j f$ die Variation von $f$ nach der Funktion $\varphi_j$.

Im Folgenden werden wir ausschliesslich Funktionen $f$ treffen, die 
linear oder sesquilinear in $\varphi_j$ sind. Solche Funktionen
kann man als Skalarprodukte schreiben:
\begin{align*}
\delta_j\langle\psi|\varphi_j\rangle
&=
\frac{d}{dt}
\langle\psi|\varphi_j+t\delta\varphi_j\rangle\bigg|_{t=0}
=
\langle\psi|\delta\varphi_j\rangle
\\
\delta_j\langle\varphi_j|\psi\rangle
&=
\langle\delta\varphi_j|\psi\rangle.
\end{align*}
F"ur einen selbstadjungierten Operator $A$ k"onnen wir den Erwartungswert
$\langle\varphi_j|\,A\,|\varphi_j\rangle$ bilden.
Seine Variation ist
\[
\delta_j\langle\varphi_j|\,A\,|\varphi_j\rangle
=
\langle\delta\varphi_j|\,A\,|\varphi_j\rangle
+
\langle\varphi_j|\,A\,|\delta\varphi_j\rangle
=
2\operatorname{Re}\langle\delta\varphi_j|\,A\,|\varphi_j\rangle.
\]
Da wir $\varphi_j$ beliebig w"ahlen k"onnen, k"onnen wir $\varphi_j$
auch mit einem beliebigen Faktor multiplizieren, so dass das
Produkt $\langle\delta\varphi_j|\,A\,|\varphi_j\rangle$ reell ist.
Die Forderung, dass die Variation
$\delta_j\langle\varphi_j|\,A\,|\varphi_j\rangle$
f"ur jede Wahl von $\delta\varphi_j$
verschwinden soll ist also gleichbedeutend damit, dass
$\langle\delta\varphi_j|\,A\,|\varphi_j\rangle=0$
f"ur jede Wahl von $\delta\varphi_j$ verschwindet, oder dass
$A\,|\varphi_j\rangle=0$ sein muss.

Die Funktionen $G_k$ und $F$ sind genau von dieser Art.

\subsubsection{L"osung des Minimalproblems}
Damit k"onnen wir jetzt die Variationen $\delta_j$
von $\langle\Psi|\,H\,|\Psi\rangle$
und von allen Nebenbedingungen $\langle\varphi_k|\varphi_k\rangle-1=0$
berechnen.
Der zweite Term in (\ref{skript:hffunktional}) enth"alt die Funktion
zweimal, einmal f"ur $k=j$ und einmal f"ur $k'=j$, seine Variation
ist daher
\begin{equation*}
\frac12\sum_{j\ne k'}
\biggl\langle\delta\varphi_j\varphi_{k'}\biggl|\,
\frac{e^2}{4\pi\varepsilon_0}
\frac{1}{|x_j-x_{k'}|}
\,\biggr|\varphi_j\varphi_{k'}\biggr\rangle
+
\frac12\sum_{k\ne j}
\biggl\langle\varphi_k\delta\varphi_j\biggl|\,
\frac{e^2}{4\pi\varepsilon_0}
\frac{1}{|x_k-x_j|}
\,\biggr|\varphi_k\varphi_j\biggr\rangle.
\end{equation*}
Bis auf den Namen des Summationsindex stimmen die beiden
Summen "uberein, wir k"onnen sie also in die eine Summe
\begin{equation*}
\sum_{k\ne j}
\biggl\langle\delta\varphi_j\varphi_k\biggl|\,
\frac{e^2}{4\pi\varepsilon_0}
\frac{1}{|x_k-x_j|}
\,\biggr|\varphi_j\varphi_k\biggr\rangle
\end{equation*}
zusammenfassen.
\begin{align}
\delta_j G_j
&=
\delta_j(\langle\varphi_j|\varphi_j\rangle-1)
=
2\langle\delta\varphi_j|\varphi_j\rangle
\label{skript:variationGj}
\\
\delta_j F
&=
\delta_j\langle\Psi|\,H\,|\Psi\rangle
=
2 \langle\delta\varphi_j|\,H_j\,|\varphi_j\rangle
+
\frac{e^2}{4\pi\varepsilon_0}
\sum_{k\ne j}
2
\biggl\langle\delta\varphi_j\varphi_k\biggl|\,
\frac{1}{|x_j-x_k|}
\,\biggr|\varphi_j\varphi_k \biggr\rangle
\label{skript:variationF}
\end{align}
\index{Lagrange-Multiplikatoren}
Da $\delta_jG_k=0$ f"ur $j\ne k$ ist, werden in den Lagrange-Gleichungen
nur $\delta_jG_j$ auftreten.
Die Methode der Lagrange-Multiplikatoren besagt, dass im Minimum
die Variationen die Gleichungen
\begin{equation}
\delta_j F - \sum_{k} E_k\, \delta_j G_k
=
\delta_j F -  E_j\, \delta_j G_j
=
0
\label{skript:lagrangeequation}
\end{equation}
erf"ullen, die ebenfalls zu bestimmenden Konstanten $E_j$ heissen 
die Lagrange-Multiplikatoren.

Setzen wir jetzt (\ref{skript:variationF}) und (\ref{skript:variationGj})
in (\ref{skript:lagrangeequation}) ein, erhalten wir die Gleichungen
\[
\biggl\langle
\delta\varphi_j\biggl|\,
H_j
+
\frac{e^2}{4\pi\varepsilon_0}
\sum_{k\ne j}\biggl\langle\varphi_k\biggl|\,
\frac1{|x_k-x_j|}
\,\biggr|\varphi_k\biggr\rangle-E_j
\,\biggl|\varphi_j \biggr\rangle=0.
\]
Da dies f"ur jede beliebige Richtung $\delta\varphi_j$ gelten muss,
folgt
\begin{equation}
\biggl(
H_j
+
\frac{e^2}{4\pi\varepsilon_0}
\sum_{k\ne j}\biggl\langle\varphi_k\biggl|\,
\frac1{|x_k-x_j|}
\,\biggr|\varphi_k\biggr\rangle\biggr)
\,|\varphi_j \rangle
=
E_j\,|\varphi_j\rangle.
\label{skript:hglsystem}
\end{equation}
Das Minimalproblem ist also "aquivalent mit dem Gleichungssystem
(\ref{skript:hglsystem}) f"ur $j=1,\dots,N$.

\subsubsection{Hartree-Gleichung}
Der Zustand $|\varphi_j\rangle$ ist also eine Eigenzustand des
Hamilton-Operators
\[
H_j
+
\frac{e^2}{4\pi\varepsilon_0}
\sum_{k\ne j}\int \frac{|\varphi_k(x_k)|^2}{|x_k-x_j|}\,d^3x_k
\]
mit Energie $E_j$. Die Funktionen $\varphi_j(x_j)$ erf"ullen also
die Eigenwertgleichungen
\[
\biggl(
-\frac{\hbar^2}{2m_e}\Delta_j
+V(x_j)
+\frac{e^2}{4\pi\varepsilon_0}
\sum_{k\ne j}\int\frac{|\varphi_k(x_k)|^2}{|x_k-x_j|}\,d^3x_k\biggr)\varphi_j(x_j) = E_j\varphi_j(x_j).
\]
Sie heissen die Hartree-Gleichungen.
\index{Hartree-Gleichung}
Jedes Elektron kann also betrachtet
werden als ein Elektron, welches sich zus"atzlich zum Potential der Gitterionen
im mittleren Coulomb-Potential
\[
V_{H,j}(x_j)
=
\frac{e^2}{4\pi\varepsilon_0}
\sum_{k\ne j}\int\frac{|\varphi_k(x_k)|^2}{|x_k-x_j|}\,d^3x_k
\]
aller anderen Elektronen.

\subsection{Slater-Determinante}
Bisher haben wir den Spin nicht ber"ucksichtigt.
Jedes Elektron hat ausser seinen Ortskoordinaten $x_k$ eine Spin-Koordinate,
wir bezeichnen die Gesamtheit der Koordinaten des Elektrons $k$ mit $q_k$.
Die Funktionen $\varphi_k$, die wir f"ur die Hartree-Gleichung verwendet
haben, h"angen jetzt von $q_k$ ab, wir schreiben sie wieder $\varphi_k$ 
oder $\varphi_k(q_k)$.

In Abschnitt~\ref{skript:austauschoperator} haben wir gesehen, dass eine
Wellenfunktion f"ur Fermionen total antisymmetrisch sein muss.
Das Produkt, welches wir als Ansatz $\Psi$ verwendet haben, erf"ullt
diese Bedingung nicht.
Eine Determinante wie
\[
\Phi(q_1,\dots,q_N)=\frac{1}{\sqrt{N!}}
\left|\begin{matrix}
\varphi_1(q_1)&\dots &\varphi_N(q_1)\\
\vdots        &\ddots&\vdots        \\
\varphi_1(q_N)&\dots &\varphi_N(q_N)
\end{matrix}\right|
\]
erf"ullt die Bedingung. In ihr kommt jede m"ogliche Permutation von
Funktionen und Koordinaten mit den richtigen Vorzeichen vor, und der
Faktor vor der Determinante stellt sicher, dass $\Phi$ normiert ist.
$\Phi$ heisst die Slater-Determinante.
\index{Slater-Determinante}%
Sind zwei der Funktionen $\varphi_k$ gleich, dann verschwindet die 
{\em Slater-Determinante}, weil zwei Spalten identisch sind.
Die Slater-Determinante realisiert also das Pauli-Prinzip.

Wir d"urfen ausserdem annehmen, dass die $\varphi_k$ orthonormiert sind. 
Sind sie es nicht, wenden wir das Orthonormalisierungsverfahren von 
Gram-Schmidt auf sie an. Dadurch wird jede Funktion $\varphi_k$
durch eine Linearkombination der Funktion $\varphi_{k'}$ mit $k'\le k$
ersetzt, die Determinante, also $\Phi$ "andert dabei nicht.

Ist $U$ eine unit"are $N\times N$-Matrix mit Matrixelementen
$u_{kk'}$, dann k"onnen statt der Funktionen $\varphi_k$
auch die Funktionen 
\begin{equation}
\varphi'_k = \sum_{k'}u_{kk'}\varphi_{k'}
\label{skript:hfunitaer}
\end{equation}
f"ur die Bildung der Slater-Determinante verwendet werden.
Daduch werden die Zeilen von $\Phi$ einer unit"aren Abbildung
unterworfen, was die Determinante nicht "andert.

Bei der Berechnung von $\langle\Psi|\,H\,|\Psi\rangle$ konnten wir
davon ausgehen, dass die Funktion $\varphi_k$ von den Koordinaten
$x_k$ abh"angt. Eine solche Zuordnung ist jetzt nicht mehr m"oglich,
jede Funktion kommt mit jeder Koordinate als Argument vor.
Problematisch wird dies vor allem f"ur Ausdr"ucke der Form
\begin{equation}
\biggl\langle\varphi_k\varphi_{k'}\biggl|\,
\frac{1}{|x_k-x_{k'}|}
\,\biggr|\varphi_{k'}\varphi_k\biggr\rangle
\label{skript:hfnotation}
\end{equation}
die wir eigentlich als
\begin{equation}
\int \varphi_k^*(q_1)\varphi_{k'}^*(q_2)
\frac{1}{|x_1-x_2|}
\varphi_{k'}(q_1)\varphi_k(q_2)\,dq_1\,dq_2
\label{skript:integrationsvariablen}
\end{equation}
lesen m"ochten. In diesem Beispiel haben die beiden auftretenden $\varphi_k$ 
verschiedene Koordinaten als Argumente.
Wir verwenden daher die folgende Konvention.
Bei der Berechnung eines Skalarproduktes in der Ortsdarstellung 
werden den Funktionen in einem Bra- oder Ket-Vektor Integrationsvariablen
der Reihe nach zugeteilt, so wie wir das in (\ref{skript:integrationsvariablen})
gemacht haben. 
Wir ersetzen ausserdem den inversen Abstand 
\[
\frac{1}{|x_1-x_2|}
\qquad
\text{durch}
\qquad
\frac{1}{|\Delta x|}.
\]
Damit k"onnen wir (\ref{skript:hfnotation}) in
\[
\biggl\langle \varphi_k\varphi_{k'}\biggl|\,
\frac1{|\Delta x|}
\,\biggr|\varphi_{k'}\varphi_k\biggl\rangle
\]
umformen, diese Notation vermeidet das Problem der Zuordnung der Variablen.

\subsection{Hartree-Fock-Gleichung}
\index{Hartree-Fock-Gleichung}%
\subsubsection{Minimalproblem}
In der Herleitung der Hartree-Gleichung m"ussen wir jetzt
$\langle\Psi|\,H\,|\Psi\rangle$ 
durch
$\langle\Phi|\,H\,|\Phi\rangle$ 
ersetzen:
\begin{align*}
\langle\Phi|\,H\,|\Phi\rangle
&=
\sum_k\langle\varphi_k|\,H_k\,|\varphi_k\rangle
+
\frac12
\sum_{k\ne k'}\biggl\langle\Phi\biggl|\,
\frac{e^2}{4\pi\varepsilon_0}
\frac{1}{|x_k-x_{k'}|}
\,\biggr|\Phi\biggr\rangle
\end{align*}
In $\Phi$ kommt jedes m"ogliche Produkt von Funktionen $\varphi_k$,
insbesondere kommen Produkte
%\[
%\varphi_k^*(q_1)
%\varphi_{k'}^*(q_2)
%\varphi_k(q_1)
%\varphi_{k'}(q_2)
%=
%|\varphi_k(q_1)|^2\,|\varphi_{k'}(q_2)|^2
%\qquad\text{und}\qquad
%\varphi_k^*(q_1)
%\varphi_{k'}^*(q_2)
%\varphi_k(q_2)
%\varphi_{k'}(q_1)
%\]
\[
\biggl\langle\varphi_k\varphi_{k'}\biggr|
\frac{1}{|\Delta x|}
\biggl|\varphi_k\varphi_{k'}\biggr\rangle
\qquad\text{und}\qquad
\biggl\langle\varphi_k\varphi_{k'}\biggr|
\frac{1}{|\Delta x|}
\biggl|\varphi_{k'}\varphi_k\biggr\rangle
\]
vor.
Das zweite Produkt enth"alt die Funktionen in umgekehrter Reihenfolge,
in der Slater-Deter\-minante hat es also das entgegengesetze Vorzeichen.
Der Erwartungswert der Energie erh"alt damit einen zus"atzlichen Term:
\begin{align*}
\langle\Phi|\,H\,|\Phi\rangle
=
\sum_k\langle\varphi_k|\,H_k\,|\varphi_k\rangle
+
\frac12
\frac{e^2}{4\pi\varepsilon_0}
\sum_{k\ne k'}
\biggl\langle\varphi_k\varphi_{k'}\biggr|
\frac{1}{|\Delta x|}
\biggl|\varphi_k\varphi_{k'}\biggr\rangle
-
\frac12
\frac{e^2}{4\pi\varepsilon_0}
\sum_{k\ne k'}
\biggl\langle\varphi_k\varphi_{k'}\biggr|
\frac{1}{|\Delta x|}
\biggl|\varphi_{k'}\varphi_k\biggr\rangle
\end{align*}
Wie bei der Hartree-Gleichung m"ochten wir
$\langle\Phi|\,H\,|\Phi\rangle$ 
minimieren, unter geeigneten Nebenbedingungen.

Wir erwarten wieder, das die Funktionen $\varphi_k$ als Zustandsvektoren
einzelner Elektronen betrachtet werden k"onnen.
Wir d"urfen die Funktionen $\varphi_k$ als orthonormiert voraussetzen,
und k"onnen als physikalischen Grund daf"ur anf"uhren, dass
nach dem Pauli-Prinzip Eigenzust"ande
von verschiedenen Elektronen sich in mindestens einer Quantenzahl
unterscheiden m"ussen, und die Wellenfunktionen daher orthogonal
sein m"ussen.

Wir erhalten damit zus"atzliche Nebenbedingungen in der Form
\[
G_{kk'}=\langle\varphi_k|\varphi_{k'}\rangle-\delta_{kk'}.
\]
Die Variationen davon sind
\[
\delta_k G_{kk'}
=
\langle\delta\varphi_k |\varphi_{k'}\rangle
\]
\subsubsection{L"osung des Minimalproblems}
Die zus"atzlichen Nebenbedingungen liefern uns zus"atzliche
Lagrage-Multiplikatoren, wir bezeicnen den Lagrange-Multiplikator
zur Nebenbedingung $G_{kk'}$ mit $\lambda_{kk'}$.

Wie fr"uher argumentiert, behalten wir in den Variationen der
bilinearen Terme immer nur die Variation des ersten Faktors.
Die Variationen von $F=\langle\Phi|\,H\,|\Phi\rangle$
\begin{align*}
\delta_j F
&=
2\langle\delta\varphi_j|\,H_j\,|\varphi_j\rangle
+
\frac{e^2}{4\pi\varepsilon_0}
\sum_{k\ne j}
\biggl\langle\delta\varphi_j\varphi_k\biggr|
\frac{1}{|\Delta x|}
\biggl|\varphi_j \varphi_k\biggr\rangle
-
\frac{e^2}{4\pi\varepsilon_0}
\sum_{k\ne j}
\biggl\langle\delta\varphi_j\varphi_k\biggr|
\frac{1}{|\Delta x|}
\biggl|\varphi_k \varphi_j\biggr\rangle.
\end{align*}

Die Gleichungen der Methode der Lagrange-Multiplikatoren werden 
jetzt zu
\[
\delta_j F - \sum_{k,k'}\lambda_{kk'}\delta_j G_{kk'}
=
\delta_j F
-
\sum_{k}\lambda_{jk}\delta_j G_{jk}
=\delta_jF
-\sum_k\lambda_{jk}\langle\delta\varphi_j|\varphi_k\rangle
=0,
\]
oder ausgeschrieben
\begin{align}
\bigg\langle\delta\varphi_j\bigg|
\bigg(
H_1|\varphi_j\rangle
&+
\frac{e^2}{4\pi\varepsilon_0}
\sum_{k\ne j}
\biggl\langle\varphi_k\biggr|
\frac{1}{|x_j-\tilde x_1|}
\biggl|\varphi_k\biggr\rangle
\biggl|\varphi_j\biggr\rangle
\notag
-
\frac{e^2}{4\pi\varepsilon_0}
\sum_{k\ne j}
\biggl\langle\varphi_k\biggr|
\frac{1}{|\tilde x_1-x_j|}
\biggl|\varphi_j\biggr\rangle
\biggl|\varphi_k\biggr\rangle
\biggr)
\\
\qquad\qquad
&=
\sum_k\lambda_{jk}\langle\delta\varphi_j|\varphi_k\rangle
=
\langle\delta\varphi_j|
\sum_k\lambda_{jk}
|\varphi_k\rangle.
\label{skript:hf1}
\end{align}
Die L"osung des Minimalproblems erf"ullt also die Gleichungen
\begin{equation}
H_1|\varphi_j\rangle
+
\biggl(
\frac{e^2}{4\pi\varepsilon_0}
\sum_{k\ne j}
\biggl\langle\varphi_k\biggl|\,
\frac{1}{|\Delta x|}
\,\biggr|\varphi_k\biggr\rangle
\biggr)
|\varphi_j\rangle
-
\frac{e^2}{4\pi\varepsilon_0}
\sum_{k\ne j}
\biggl\langle\varphi_k\biggl|\,
\frac{1}{|\Delta x|}
\,\biggr|\varphi_j\biggr\rangle
|\varphi_k\rangle
=
\sum_k\lambda_{jk}
|\varphi_k\rangle.
\label{skript:hf2}
\end{equation}
Diese Vektor-Gleichung ist von der Form
\[
A\varphi = \Lambda \varphi.
\]
Man kann zeigen, dass $A$ selbstadjungiert ist.
Es folgt dann, dass auch die Matrix $\Lambda$ selbstadjungiert sein muss,
und dass man die Vektoren $\varphi_k$ mit Hilfe einer unit"aren Matrix durch
neue Vektoren $\varphi'_k$ ersetzt werden, so dass die
Matrix $\Lambda$ diagonal wird.
Eine solche Transformation "andert nach (\ref{skript:hfunitaer})
an der Slater-Determinante nichts.
Wir nennen die transformierten Vektoren $\varphi'_k$ wieder mit $\varphi_k$,
und erhalten aus (\ref{skript:hf2}) die Gleichungen
\begin{align}
H_1\,|\varphi_j\rangle
+
\frac{e^2}{4\pi\varepsilon_0}
\sum_{k\ne j}
\biggl\langle\varphi_k\biggl|
\frac{1}{|\Delta x|}
\biggr|\varphi_k\biggr\rangle
|\varphi_j\rangle
-
\frac{e^2}{4\pi\varepsilon_0}
\sum_{k\ne j}
\biggl\langle\varphi_k\biggl|
\frac{1}{|\Delta x|}
\biggr|\varphi_j\biggr\rangle
|\varphi_k\rangle
=
E_j
| \varphi_j\rangle
\label{skript:hf3}
\end{align}

\subsubsection{Spin}
Die Terme
\[
c_{kj}
=
\biggl\langle\varphi_k\biggl|
\frac{1}{|\Delta x|}
\biggr|\varphi_j\biggr\rangle
\]
in (\ref{skript:hf3})  k"onnen weiter ausgerechnet werden.
Die Funktionen $\varphi_k$ h"angen ja nicht nur von der Position, sondern
auch von den Spin-Koordinaten ab.
Da es in unserem Modell keine Kopplung zwischen den Spin- und
den Raumkoordinaten gibt (keine Spin-Bahn-Kopplung), lassen sich die
Funktionen als Produkt
\[
\varphi_k(q_1)=\varphi_{x,1}(x_1)\cdot\varphi_{s,1}(s_1)
\]
schreiben.

Das Skalarprodukt zweier solcher Funktionen zerf"allt in ein Integral
"uber die Raumkoordinaten und eine Summe "uber die Spin-Koordinaten:
\[
\langle\varphi_k|\varphi_{k'}\rangle
=
\langle\varphi_{x,k}|\varphi_{x,k'}\rangle
\langle\varphi_{s,k}|\varphi_{s,k'}\rangle
=
\underbrace{
\int\varphi_{x,k}^*(x_1)\varphi_{x,k'}(x_1)\,dx_1
}_{=\langle\varphi_{x,k}|\varphi_{x,k'}\rangle}
\cdot
\big(
\underbrace{
\varphi_{s,k}^*(\textstyle \frac12) \varphi_{s,k'}(\textstyle \frac12)
+
\varphi_{s,k}^*(\textstyle-\frac12) \varphi_{s,k'}(\textstyle-\frac12)
}_{=\langle\varphi_{s,k}|\varphi_{s,k'}\rangle}
\big)
\]
Da es nur zwei Spin-Zust"ande gibt, kann das Skalarprodukt der Spinfunktion
nur $1$ sein, bei parallelem Spin, oder $0$, bei antimparallelem Spin.
Das gleiche gilt dann f"ur  $c_{kj}$
\[
\biggl\langle\varphi_k\biggl|
\frac{1}{|\Delta x|}
\biggr|\varphi_j\biggr\rangle
=
\begin{cases}
\langle\varphi_{x,k}|\, 1/|\Delta x|\,|\varphi_{x,j}\rangle
%\displaystyle
%\int\varphi_{x,k}^*(x_1)\frac1{|\Delta x|}\varphi_{x,k'}(x_1)\,dx_1
&\qquad\text{Spin parallel}
\\
0&\qquad\text{Spin antiparallel}.
\end{cases}
\]

\subsubsection{Hartree-Fock-Gleichung}
Damit haben wir jetzt die Hartree-Fock-Gleichung hergeleitet:
\begin{align}
H_1\,|\varphi_j\rangle
+
V_{H,j}
|\varphi_j\rangle
-
\frac{e^2}{4\pi\varepsilon_0}
\sum_{{k\ne j}\atop{\text{Spin parallel}}}
\biggl\langle\varphi_k\biggl|\,
\frac{1}{|\Delta x|}
\,\biggr|\varphi_j\biggr\rangle
|\varphi_k\rangle
=
E_j
| \varphi_j\rangle
\label{skript:hf4}
\end{align}
Sie beschreibt Elektronen im Hartree-Potential und zus"atzlich unter
dem Einfluss einer Wechselwirkung mit allen anderen Elektronen
gleichen Spins.



