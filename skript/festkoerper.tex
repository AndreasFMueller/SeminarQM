\chapter{Festk"orper\label{chapter:festkoerper}}
\lhead{Festk"orper}
\rhead{}
Festk"orper sind aus vielen Atomen zusammengesetzte Quantensysteme.
Im Gegensatz zu einer Fl"ussigkeit sind jedoch die Atomkerne mehr
oder weniger unverr"uckbar in einem Gitter angeordnet.
Bei elektrischen Leitern ist ein Teil der Elektronen weitgehend frei
beweglich.
Nat"urlich ist die Gitterstruktur ebenfalls die Folge der Interaktionen
zwischen den Elektronen, doch f"ur unsere Untersuchungen k"onnen wir
die Entstehung der Gitterstruktur vernachl"assigen, es einfach als
gegeben ansehen, und nur noch die Eigenschaften der frei beweglichen
Elektronen studieren.
Diese Elektronen bewegen sich in dem periodischen Potential, das von
den Atomkernen des Gitters erzeugt wird.
Jeder Atomkern erzeugt einen Potentialtopf, in dem eines oder mehrere
Elektronen gefangen sind.
Bleiben nach der Besetzung dieser lokalisierten Zust"ande noch
Elektronen "ubrig, dann haben diese mehr Energie als die Schwellen zwischen
den Potentialt"opfen hoch sind, aber nicht gen"ugend Energie, um
aus dem Festk"orper auszutreten.
Diese wie auch die Elektronen, die mit nur wenig Anregungsenergie
in einen ausgedehnten Zustand versetzt werden k"onnen, interessieren
uns in diesem Kapitel.

\section{Fermikugel}
In erster N"aherung betrachen wir den Festk"orper als einen
sehr grossen Potenialkasten, aus dem die Elektronen nicht austreten
k"onnen.
Die gebundenen Elektronen haben so tiefe Energie, dass wir davon
ausgehen k"onnen, dass diese Zust"ande immer besetzt sind.
Zur Leitung tragen sie weiter nichts bei, wir brauchen also f"ur
doe Modellierung von Leitungsph"anomenen nur diejenigen Elektronen
zu ber"ucksichtigen, die beweglich genug sind, sich im ganzen
Festk"orper bewegen zu k"onnen.
Weiter nehmen wir an, dass sich die Elektronen gegenseitig 
nicht beeinflussen.
Jedes Elektron bewegt sich im Potential, welches von den Atomkernen
und allen anderen Elektronen erzeugt wird.
Jedes Elektron sieht also im wesentlichen den gleichen Potentialkasten.
Die Energieniveaus f"ur die Zust"ande in einem Potentialkasten haben
wir in (\ref{skript:3dzustaende}) bereits berechnet.
\begin{figure}
\centering
\includegraphics{graphics/fest-1.pdf}
\caption{Fermi-Kugel im Raum der Zust"ande eines Elektrons in einem
Potentialkasten
\label{skript:fermi-kugel}}
\end{figure}
Nat"urlich m"ussen sich die Elektronen ausserdem an das Pauli-Prinzip
halten: jeder Energiezustand kann mit h"ochstens zwei Elektronen
besetzt sein. 
Im Zustand minimaler Energie ist der Festk"orper also, wenn sich
alle Elektronen in Zust"anden befinden, die einer Gleichung
\[
\frac{h^2}{32ml^2}(
n_x^2
+
n_y^2
+
n_z^2
)
<
E_F
\]
gen"ugen.
Die tats"achlich besetzten Elektronenzust"ande bilden also eine Kugel
im Raum der m"oglichen Elektronenzust"ande, die Fermi-Kugel.
Je gr"osser die Dichte der frei beweglichen Elektronen ist, desto
h"oher ist auch die Fermi-Energie $E_F$.

\begin{figure}
\centering
\includegraphics{graphics/fest-2.pdf}
\caption{Gebundene Niveaus in einem Kristall
\label{skript:gebundene-niveaus}}
\end{figure}
\begin{figure}
\centering
\includegraphics{graphics/fest-3.pdf}
\caption{Durch Erh"ohung der Dichte werden Elektronen auf einstmals
gebundenen Niveaus frei beweglich.
\label{skript:gebundene-niveaus}}
\end{figure}
Wenn also gen"ugen Elektronen vorhanden sind, dass $E_F$ "uber dem
``Bodenniveaus'' des Potentialkastens liegt, dann liegt ein Leiter vor.
Diese Situation kann auch erzwungen werden.
Presst man die Atomkerne weiter zusammen, werden auch die
Coulomb-Potential-L"ocher kleiner, die Elektronen haben f"ur normale
$s$-Orbitale gar keinen Platz mehr, und stehen daher als frei bewegliche
Elektronen zur Verf"ugung.
Die Fermi-Energie steigt also an, der K"orper wird ein Leiter.
Diesen Zustand erreicht Wasserstoff unter extrem hohem Druck,
man vermutet dass man solchen metallischen Wasserstoff im Inneren
von Jupiter und anderen Gas-Riesen finden kann.

\section{Kristalle}
\subsection{Symmetrien}
Kristalle sind Festk"orper, deren Atomkerne in einem regelm"assigen
Gitter angeordnet sind.
Im einfachsten Fall gibt es eine Menge 
$ \Gamma  \subset \mathbb R^3 $
so, dass eine Verschiebung aller Atomkerne um einen Vektor aus $\Gamma$
den Festk"orper nicht "andert.
Wir k"onnen die Verschiebung um einen Vektor $\vec v\in\Gamma$ als
Operator $T_{\vec v}$ auf dem Hilbertraum der Zustandsvektoren
betrachten:
\[
(T_{\vec v}\psi)(x)=\psi(x-\vec v).
\]
Der Hamilton-Operator "andert sich nicht, wenn man die Atomkerne um
einen Vektor $\vec v\in\Gamma$ verschiebt.
Also muss der Hamilton-Operator mit allen Translationen $T_{\vec v}$ 
vertauschen.
Es gibt also eine Basis von Eigenvektoren von $H$, welche auch
Eigenvektoren aller Operatoren $T_{\vec v}$ sind, f"ur jeden Vektor
$\vec v\in\Gamma$.

Die Menge $\Gamma$ hat noch etwas mehr Struktur als wir bisher
verwendet haben.
Wenn zwei Vektoren $\vec v_1,\vec v_2\in\Gamma$ sind, dann muss
auch deren Summe $\vec v_1+\vec v_3\in\Gamma$ sein.
Man nennt $\Gamma$ eine Gruppe.

Ausserdem ist es nicht n"otig, sich auf die Translationen $\Gamma$
zu beschr"anken.
Der Festk"orper kann durchaus noch weitere Symmetrien haben, zum
Beispiel Drehungen oder Spiegelungen.
Auch diese Symmetrien k"onnen als Operatoren beschrieben
werden, die auf den Zustandsvektoren wirken.
Die Menge aller Symmetrietransformationen nennen wir $G$.
Die Menge $G$ bildet ebenfalls eine Gruppe: jeder Verkn"upfung von
Symmetrieoperationen ist wieder eine Symmetrieoperation, und die
Inverse einer Symmetrieoperation ist ebenfalls eine Symmetrieoperation.
Die m"oglichen Symmetriegruppen von Kristallen sind vollst"andig
klassifiziert worden.

Eine Transformation darf die Norm der Zustandsvektoren nicht "andern,
die Transformationen in $G$ sind also unit"are Operatoren,
$U^*U=UU^*=\operatorname{id}$ f"ur alle $U\in G$.
Die Eigenwerte von Transformationen in $G$ sind daher alle in $U(1)$.
Die Transformationen einer Eigenfunktion von $H$ und allen Operatoren in $G$
"andert eine Wellenfunktion als h"ochstens um einen Phasenfaktor.
Zu jedem Operator $U\in G$ gibt es also einen Eigenwert $\chi(U)\in U(1)$,
und die Zusammensetzung von Operatoren muss mit der Abbildung $\chi$
vertauschen, d.~h.
\[
\begin{aligned}
\chi(UV)&=\chi(U)\chi(V)&&U,V\in G\\
\chi(U^*)&=\bar\chi(U)&&U\in G
\end{aligned}
\]
Man nennt eine solche Abbildung einen Darstellung der Gruppe $G$.
Auch die Darstellungen der Gruppen $G$ sind vollst"andig klassifiziert
worden.

\subsection{Periodisches Potential}
Als Beispiel betrachten wir einen eindimensionalen Kristall, dessen
Atomkerne an den Stellen $an$, $n\in\mathbb Z$ platziert sind.
$a$ ist der Abstand zwischen den Atomkernen.
Statt Coulomb-Potentials verwenden wir ein schmalen Potentialtopf.
Dieser Kristall hat einerseits die Translationssymmetrie $T_{an}$ um
ganzzahlige Vielfache von $a$, und andererseits eine Spiegelung $S$.
Die Translationen und die Verschiebungen vertauschen nicht, es gilt
\[
T_{na}S=ST_{-na}.
\]
Da die Spiegelung $S^2=\operatorname{id}$ erf"ullt, kann sie nur
Eigenwerte $1$ oder $-1$ haben.
Wenden wir darauf $\chi$ an, finden wir
\[
\chi(a)^n\chi(S)=\chi(S)\bar\chi(a)^n
\qquad \Rightarrow \qquad
\chi(a)^2=1
\]

Wir suchen also Zustandsvektoren, welche Eigenwerte der
Symmetrieoperatoren sind.

\section{Valenz- und Leitungsband}



