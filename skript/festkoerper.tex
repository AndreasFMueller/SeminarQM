\chapter{Festk"orper\label{chapter:festkoerper}}
\lhead{Festk"orper}
\rhead{}
Festk"orper sind aus vielen Atomen zusammengesetzte Quantensysteme.
Im Gegensatz zu einer Fl"ussigkeit sind jedoch die Atomkerne mehr
oder weniger unverr"uckbar in einem Gitter angeordnet.
Bei elektrischen Leitern ist ein Teil der Elektronen weitgehend frei
beweglich.
Nat"urlich ist die Gitterstruktur ebenfalls die Folge der Interaktionen
zwischen den Elektronen, doch f"ur unsere Untersuchungen k"onnen wir
die Entstehung der Gitterstruktur vernachl"assigen, es einfach als
gegeben ansehen, und nur noch die Eigenschaften der frei beweglichen
Elektronen studieren.
Diese Elektronen bewegen sich in dem periodischen Potential, das von
den Atomkernen des Gitters erzeugt wird.
Jeder Atomkern erzeugt einen Potentialtopf, in dem eines oder mehrere
Elektronen gefangen sind.
Bleiben nach der Besetzung dieser lokalisierten Zust"ande noch
Elektronen "ubrig, dann haben diese mehr Energie als die Schwellen zwischen
den Potentialt"opfen hoch sind, aber nicht gen"ugend Energie, um
aus dem Festk"orper auszutreten.
Diese wie auch die Elektronen, die mit nur wenig Anregungsenergie
in einen ausgedehnten Zustand versetzt werden k"onnen, interessieren
uns in diesem Kapitel.

\section{Fermikugel}
In erster N"aherung betrachen wir den Festk"orper als einen
sehr grossen Potentialkasten, aus dem die Elektronen nicht austreten
k"onnen.
Die gebundenen Elektronen haben so tiefe Energie, dass wir davon
ausgehen k"onnen, dass diese Zust"ande immer besetzt sind.
Zur Leitung tragen sie weiter nichts bei, wir brauchen also f"ur
doe Modellierung von Leitungsph"anomenen nur diejenigen Elektronen
zu ber"ucksichtigen, die beweglich genug sind, sich im ganzen
Festk"orper bewegen zu k"onnen.
Weiter nehmen wir an, dass sich die Elektronen gegenseitig 
nicht beeinflussen.
Jedes Elektron bewegt sich im Potential, welches von den Atomkernen
und allen anderen Elektronen erzeugt wird.
Jedes Elektron sieht also im wesentlichen den gleichen Potentialkasten.
Die Energieniveaus f"ur die Zust"ande in einem Potentialkasten haben
wir in (\ref{skript:3dzustaende}) bereits berechnet.
\begin{figure}
\centering
\includegraphics{graphics/fest-1.pdf}
\caption{Fermi-Kugel im Raum der Impuls-Zust"ande eines Elektrons in einem
zweidimensionalen Potentialkasten.
\label{skript:fermi-kugel}}
\end{figure}
Nat"urlich m"ussen sich die Elektronen ausserdem an das Pauli-Prinzip
halten: jeder Energiezustand kann mit h"ochstens zwei Elektronen
besetzt sein. 
Im Zustand minimaler Energie ist der Festk"orper also, wenn sich
alle Elektronen in Zust"anden befinden, die einer Gleichung
\[
\frac{h^2}{32ml^2}(
n_x^2
+
n_y^2
+
n_z^2
)
<
E_F
\]
gen"ugen.
Die tats"achlich besetzten Elektronenzust"ande bilden also eine Kugel
im Raum der m"oglichen Elektronenzust"ande, die Fermi-Kugel.
Je gr"osser die Dichte der frei beweglichen Elektronen ist, desto
h"oher ist auch die Fermi-Energie $E_F$.

% XXX Wellenzahl

\subsection{Periodische Randbedingungen}
Das Modell des Potentialkastens f"uhrt auf mathematisch sehr unangenehme
Randbedingungen, die sich in speziellen Randeffekten "aussern.
Solange unser Ziel ist, vor allem das Innere des Festk"orpers zu verstehen,
sind wir frei, andere Randbedingungen zu w"ahlen, solange Sie im Wesentlichen
das gleiche Energiespektrum ergeben. Erst wenn der Festk"orper im Vergleich
zu den Atomabst"anden klein wird, werden die Randeffekte "uberhand nehmen,
dies ist zum Beispiel bei Quantum-Dots der Fall. 

Ein besonders einfaches Modell ist, sich den Festk"orper unendlich 
ausgedehnt vorzustellen. Diese Modell kann aber nicht zutreffen, denn
es f"uhrt dazu, dass wir mit unendlich vielen Elektronen arbeiten m"ussen.
Die m"oglichen Wellenzahlen $k_x$, $k_y$ und $k_z$ w"urden stetig, die
diskrete Natur der Quantenmechanik geht in diesem Modell verloren.

Wir m"ussen dem unendlichen Festk"orpermodell also zus"atzlich Bedingungen
aufzwingen, die die diskrete Natur zur"uckbringen. Die Fourier-Transformation
auf $\mathbb R$ f"uhrt uns in einen stetigen Frequenzraum, wenn wir
aber nur die periodischen Funktionen auf $\mathbb R$ betrachten, sind
nur noch diskrete Frequenzen m"oglich.
Wir fordern daher, dass die Wellenfunktionen periodisch sind,
\begin{align*}
\psi(x-l, y  ,z  )&=\psi(x+l,y  ,z  ),
\\
\psi(x  , y-l,z  )&=\psi(x  ,y+l,z  ),
\\
\psi(x  , y  ,z-l)&=\psi(x  ,y  ,z+l).
\end{align*}
Elektronen mit dem Hamilton-Operator 
\[
H=\frac{p^2}{2m_e}=-\frac{\hbar^2}{2m_e}\Delta
\]
haben als Wellenfunktionen ebenen Wellen
\[
|\vec k\rangle
=
\psi_{\vec k}(\vec x)
=
e^{\frac{i}{\hbar}\vec k\cdot \vec x}
\]
mit Energie
\[
E_{\vec k}=\frac{\hbar^2\vec k^2}{2m_e}.
\]
Die periodischen Randbedingungen verlangen, dass $2l k_x$, $2lk_y$ und $2lk_z$
Vielfache von $2\pi$ sind, dass es also Zahlen $n_x$, $n_y$ und $n_z$ gibt mit
\[
k_x=\frac{2\pi n_x}{2l}=\frac{\pi n_x}{l},\qquad
k_y=\frac{2\pi n_y}{2l}=\frac{\pi n_y}{l}\qquad\text{und}\qquad
k_z=\frac{2\pi n_z}{2l}=\frac{\pi n_z}{l}.
\]
Die m"oglichen $\vec k$-Zust"ande bilden also wieder ein Gitter im
$\vec k$-Raum wie in Abbildung~\ref{skript:fermi-kugel}, und das Modell
der Fermi-Kugel ist auch f"ur diese Randbedingungen angemessen.

\subsection{Gebundene Elektronen}
\begin{figure}
\centering
\includegraphics{graphics/fest-2.pdf}
\caption{Gebundene Niveaus in einem Kristall
\label{skript:gebundene-niveaus}}
\end{figure}
\begin{figure}
\centering
\includegraphics{graphics/fest-3.pdf}
\caption{Durch Erh"ohung der Dichte werden Elektronen auf einstmals
gebundenen Niveaus frei beweglich.
\label{skript:gebundene-niveaus-komprimiert}}
\end{figure}
Das Modell der Fermi-Kugel ist nur anwendbar auf Elektronen, die sich
innerhalb des Festk"orpers im Wesentlichen frei bewegen k"onnen.
Im Allgemeinen wird nur ein Teil der Elektronen diese Bedingung erf"ullen,ischer Leiter ist.
die meisten Elektronen werden sich auf tiefen Energieniveaus in unmittelbarerischer Leiter ist.
N"ahe der Atomkerne aufhalten, wie in Abbildung~\ref{skript:gebundene-niveaus}
dargestellt.

Erst wenn gen"ugend Elektronen vorhanden sind, dass alle diese gebundenen
Niveaus zu besetzt sind, k"onnen weitere Elektronen sich Elektronen mit
noch h"oherer Energie freier bewegen.
Erst f"ur diese Elektronen ist es sinnvoll, von der Fermi-Kugel zu sprechen,
und erst wenn solche Elektronen vorhanden sind, k"onnen wir erwarten,
dass der Festk"orper ein elektrischer Leiter ist.

Wenn also gen"ugen Elektronen vorhanden sind, dass $E_F$ "uber dem
``Bodenniveaus'' des Potentialkastens liegt, dann liegt ein Leiter vor.
Diese Situation kann auch erzwungen werden.
Presst man die Atomkerne weiter zusammen, werden auch die
Coulomb-Potential-L"ocher kleiner, die Elektronen haben f"ur normale
$s$-Orbitale gar keinen Platz mehr, und stehen daher als frei bewegliche
Elektronen zur Verf"ugung.
Die Fermi-Energie steigt also an, der K"orper wird ein Leiter.
Diesen Zustand erreicht Wasserstoff unter extrem hohem Druck,
man vermutet dass man solchen metallischen Wasserstoff im Inneren
von Jupiter und anderen Gas-Riesen finden kann.

\section{Elektronen in einem periodischen Potential}
Das im letzten Abschnitt verwendete Modell eines Festk"orpers ging davon aus,
dass sich die Elektronen innerhalb des Festk"orpers im wesentlichen
ungehindert bewegen k"onnen.
Der Festk"orper wurde im wesentlichen ein im Vergleich zum Abstand 
der Atome untereinander sehr grosser Potentialtopf betrachtet.
Ein realer Festk"orper zeichnet sich durch mehr oder weniger regelm"assige
Anordnung der Atomkerne aus, die ein entsprechen strukturiertes Potential
erzeugen, in dem sich die Elektronen des Festk"orpers bewegen.
Die m"oglichen Energieniveaus werden sich durch die Wirkung des Potentials
ver"andern, und mit ihnen m"oglicherweise die elektrischen Eigenschaften.

Ziel dieses Abschnittes ist das Energiespektrum in einem periodischen
Potential zu verstehen, insbesondere das Enstehen von Energieb"andern.
Wir beschr"anken uns dabei auf ein eindimensionales Modell, in dem sich die
wesentlichen Einfl"usse bereits verstehen lassen.

\subsection{Gitter}
Ein Gitter ist die von drei Basisvektoren $\vec a_i\in\mathbb R^3$
erzeugte Menge
\[
\Gamma
=
\{
n_1\vec a_1+
n_2\vec a_2+
n_3\vec a_3
\,|\,
n_i\in \mathbb Z
\}\subset{\mathbb R}^3.
\]
Die Vektoren in $\Gamma$ bezeichnen die Pl"atze, an denen sich die
Kerne der Gitteratome befinden. 

Ein eindimensionales Gitter braucht nur einen einzigen Gittervektor,
wir k"onnen auf die Vektorschreibweise verzichten.
Wir schreiben f"ur das Gitter
\[
\Gamma=\{ na\,|\,n\in\mathbb Z\}.
\]
Das von Atomen an den Gitterpunkten erzeugte Potential $V(x)$
ist Translationsinvariant, es gilt
\[
V(x+v)=V(x)\qquad\forall v\in\Gamma.
\]
Wir schreiben $T_v$ f"ur den Verschiebe-Operator, definiert durch
\[
(T_vf)(x)=f(x+v).
\]
Mit dem Verschiebeoperator l"asst sich die Periodizit"at des Potentials
durch die Gleichung $T_vV=V$ ausdr"ucken.

Die Translationsoperatoren sind mit der Addition vertr"aglich:
\begin{align*}
T_{u+v}\psi(x)&=\psi(x+u+v)=T_u\psi(x+v)=T_uT_v\psi(x),\\
T_{nv}\psi(x)&=T_v^n\psi(x).
\end{align*}

\subsection{Hamilton-Operator}
Der Hamilton-Operator eines Elektrons in einem Gitter hat in der Ortsdarstellung
die Form
\[
H=-\frac{\hbar^2}{2m_e}\Delta + V(x),
\]
wobei $V(x)$ eine gitterperiodische Funktion ist, also $T_vV=V$ f"ur alle
$v\in\Gamma$.
Der Operator $H$ ist translationsinvariant, was man auch durch
die Vertauschungsrelation $[H,T_v]=0$ f"ur alle $v\in\Gamma$
ausdr"ucken kann.

In einem ersten Schritt betrachten wir den Fall $V=0$.
In diesem Fall wird der Hamilton-Operator zum Hamilton-Operator eines
freien Elektrons, f"ur den wir sofort ebene Wellen als Eigenzust"ande
angeben k"onnen:
\begin{equation}
\psi_{\vec k}(x)=e^{i\vec k\cdot \vec x}
\label{skript:ebenewelle}
\end{equation}
Diese Eigenfunktionen von $H$ sind aber nicht gleichzeitig auch
Eigenfunktionen des Translationsoperators. Wegen $[H,T_v]=0$
sollte es aber m"oglich sein, eine Basis von gleichzeitigen
Eigenvektoren von $H$ und $T_v$ zu w"ahlen.

\subsection{Das reziproke Gitter}
Welche Vektoren $\vec k$ ergeben ebenen Wellen, die periodisch sind?
F"ur jeden Vektor $v\in\Gamma$ des Gitters muss gelten
\[
\psi_{\vec k}(x+v)=\psi_{\vec k}(x).
\]
Mit (\ref{skript:ebenewelle}) folgt daraus, dass
$\vec k\cdot v$ ein Vielfaches von $2\pi$ sein muss.
Ausgehend von einer Basis $\vec v_i$ des Gitters sind
\[
\vec k\cdot v_i=2\pi n_i
\]
linear unbh"angige Gleichungen f"ur den Vektor $\vec k$. Schreibt man
die Koeffizienten der Vekoren $v_i$ als Zeilen in eine Matrix $V$
und die Zahlen $n_i$ in als Spaltenvektore $\vec n$,
dann kann man das Gleichungssystem als
\[
V\vec k=2\pi\vec n
\]
schreiben, und man kann die L"osungen mit der inversen Matrix als
\[
\vec k = 2\pi V^{-1}\vec n
\]
finden.
Insbesondere bilden die Vektoren $\vec k$, die zu gitterperiodischen
ebenen Wellen f"uhren, selbst ein Gitter, welches von den Spaltenvektoren
von $2\pi V^{-1}$ aufgespannt wird.
Es heisst das reziproke Gitter $\Gamma'$ von $\Gamma$.

\begin{beispiel}
Sei $\Gamma = \{ na\,|\, n\in\mathbb Z\}$ ein eindimensionales Gitter.
Es wird erzeugt von $v_1=a$.
Also wird das reziproke Gitter von $2\pi a^{-1}$ aufgespannt.
\end{beispiel}

\begin{beispiel}
Wir betrachten das zweidimensionale, rechteckiges Gitter
aufgespannt von den Vektoren
\[
v_1=\begin{pmatrix}1\\0\end{pmatrix}
\qquad\text{und}\qquad
v_2=\begin{pmatrix}0\\a\end{pmatrix}.
\]
Das reziproke Gitter wird erzeugt von den Spalten von $2\pi V^{-1}$,
\[
V=\begin{pmatrix} 1&0\\ 0&a \end{pmatrix}
\qquad\Rightarrow\qquad
V^{-1}=\begin{pmatrix}1&0\\0&\frac1a\end{pmatrix}
\qquad\Rightarrow\qquad
\vec k_1=\begin{pmatrix}2\pi\\0\end{pmatrix},\quad
\vec k_2=\begin{pmatrix}0\\\frac{2\pi}a\end{pmatrix}
\]
Das reziproke Gitter ist wieder ein rechteckiges Gitter.
\end{beispiel}

\begin{figure}
\centering
\includegraphics{graphics/fest-4.pdf}
\bigskip

\includegraphics{graphics/fest-5.pdf}
\caption{Hexagonales Gitter (oben) und dazu geh"origes reziprokes Gitter
(unten)
\label{skript:hexagonalesgitter}}
\end{figure}

\begin{beispiel}
Wir betrachten das hexonale Gitter aufgespannt von den Vektoren
\[
v_1=\begin{pmatrix}1\\0\end{pmatrix}
\qquad\text{and}\qquad
v_2=\begin{pmatrix}\frac12\\\frac{\sqrt{3}}2\end{pmatrix}.
\]
Die Inverse der Matrix $V$ ist
\[
V=\begin{pmatrix}
1&0\\
\frac12&\frac{\sqrt{3}}2
\end{pmatrix}
\qquad\Rightarrow\qquad
V^{-1}=\begin{pmatrix}
1&0\\
-\frac1{\sqrt{3}}&\frac{2}{\sqrt{3}}
\end{pmatrix}.
\]
Die Spaltenvektoren haben beide die L"ange $2/\sqrt{3}$, und der
Zwischenwinkel ist 
\[
\cos\alpha
=
\frac{
\begin{pmatrix}1\\-\frac1{\sqrt{3}}\end{pmatrix}
\cdot
\begin{pmatrix}0\\\frac{2}{\sqrt{3}}\end{pmatrix}
}{\displaystyle\frac{4}{3}}
=
\frac{3}{4}\cdot \biggl(-\frac{2}{3}\biggr)
=
-\frac12
\qquad \Rightarrow \qquad
\alpha=\frac{2\pi}3.
\]
Abbildung~\ref{skript:hexagonalesgitter} zeigt das hexagonale Gitter 
und das zugeh"orige reziproke Gitter.
\end{beispiel}
Die Beispiele illustrieren, wie l"angere Basisvektoren f"ur das Gitter
$\Gamma$ zu entsprechend k"urzeren Basisvektoren des reziproken
Gitters $\Gamma'$ f"uhren.

\subsection{Folgen der Translations-Invarianz}
Im Folgenden beschr"anken wir uns wieder auf ein eindimensionales Gitter
$\Gamma$ mit Gitterkonstante $a$.
Vektoren im Gitter $v\in\Gamma$ sind Vielfache $v=na$ der Gitterkonstanten,
und es gilt $T_{v}=T_{na}=T_a^n$.
Das reziproke Gitter $\Gamma'$ hat die Gitterkonstante $2\pi/a$.

Da der Hamilton-Operator $H$ mit allen Translationen $T_v$ vertauscht,
gibt es eine Basis aus gleichzeitigen Eigenfunktionen von $H$ und $T_v$.
Sei also $|\psi\rangle$ ein translationsinvarianter Eigenzustand mit
Wellenfunktion $\psi(x)$. Da $|\psi\rangle$ ein Eigenzustand aller $T_v$
ist, muss gelten
\[
T_v\psi(x)=\psi(x+v)=\lambda_v\psi(x)
\]
f"ur jeden Vektor $v=na$.
Weil $T_v=T_a^n$ ist, muss es eine Zahl $\lambda=\lambda_a$ geben mit
$\lambda_v=\lambda^n$.

Ausserdem gelten weiterhin die periodischen Randbedingungen,
was soviel bedeutet wie dass $T_{2l}$ der Einheitsoperator ist%
\footnote{Hier
steckt der Grund, warum wir uns f"ur die aktuelle Diskussion wieder
auf den eindimensionalen Fall beschr"anken. Die Diskussion l"asst sich
selbstverst"andlich auf ein dreidimensionales Gitter "ubertragen, aber
die Definition der periodischen Randbedingungen muss mit etwas mehr
Sorgfalt erfolgen.}.
Dies bedeutet, dass es eine ganze Zahl $n$ geben muss so,
dass $T_a^n=\operatorname{id}$. F"ur den Eigenvektor $|\psi\rangle$
von $T_a$ mit Eigenwert $\lambda$, bedeutet dies, dass $\lambda^n=1$
sein muss. Insbesondere muss $|\lambda|=1$ sein. 

Die Zahl $\lambda$ kann in der Form $\lambda=e^{iKa}$ geschrieben werden,
und wegen $\lambda^n=e^{inKa}$ muss $nKa$ ein Vielfaches von $2\pi$ sein.
$K$ ist nicht eindeutig bestimmt, denn addiert man $k\in\Gamma'$, dann ist
\[
e^{i(K+k)a}=e^{iKa}\underbrace{e^{ika}}_{=1}=e^{iKa}.
\]
$K$ ist also nur bis auf einen Vektor in $\Gamma'$ bestimmt.
Insbesondere kann man unter alle in Frage kommenden Vektoren $K$ immer
denjenigen ausw"ahlen, der am n"achsten beim Nullpunkt liegt.

Wir schreiben jetzt
\[
u(x)=e^{-iKx}\psi(x)
\qquad\Rightarrow\qquad
\psi(x)=e^{iKx}u(x).
\]
Die Wirkung des Operators $T_a$ kann man jetzt ausrechnen:
\begin{equation}
\left.
\begin{aligned}
T_a\psi(x)&=\psi(x+a)=e^{iK(x+a)}u(x+a)=\lambda e^{iKx}u(x+a)
\\
=\psi(x)&=e^{iKx}u(x)
\end{aligned}
\right\}
\qquad\Rightarrow\qquad
u(x+a)=u(x).
\end{equation}
Die Funktion $u(x)$ ist also gitterperiodisch.

Die m"oglichen Werte von $K$ sind nicht beliebig, da $e^{iKx}$ die
periodischen Randbedingungen des Kristalls erf"ullen muss. Wenn $l=Na$
ist, dann muss $K$ ein Vielfaches von $\frac{2\pi}{Na}$ sein.
Je gr"osser der Kristall ist, desto n"aher liegen die m"oglichen
Wert von $K$ beeinander.
Wir nehmen im Folgenden an, dass $N$ sehr gross ist, so dass die
m"oglichen Werte von $K$ so nahe beeinander liegen, dass man sie in
einer graphischen Darstellung nicht mehr unterscheiden kann.

\subsection{Kristall-Elektronen}
Ein Energiezustand $|\psi\rangle$ von $H$ mit Energie $E$
kann immer beschrieben werden als ein Produkt $e^{iKx} u(x)$,
wobei $K$ ein beliebiger Wellenzahlvektor ist.
Die Wirkung des Hamilton-Operators k"onnen wir nat"urlich auch
berechnen:
\[
H|\psi\rangle
=
e^{iKx}\biggl(
\frac{\hbar^2K^2}{2m_e}-\frac{\hbar^2}{2m_e}\frac{\partial^2}{\partial x^2}
+V(x)
\biggr)u(x)
=
e^{iKx}E(K)u(x) + e^{iKx} Hu(x)
=
e^{iKx}Eu(x)
\]
mit
\[
E(K)=\frac{\hbar^2K^2}{2m_e}.
\]
Die Funktion $u$ Ist also ein Eigenzustand des Operators $H-E(K)$. Umgekehrt
kann man aus jeder periodischen Eigenfunktion $u(x)$ von $H$ mit Energie
$E_u$ einen Eigenvektor $e^{iKx}u(x)$ von $H$ mit Energie $E_u+E(K)$, machen.

Die L"osungen $u(x)$ ist in der Umgebung jedes Gitterpunktes gleich.
Da die Gitteratome nicht unterscheidbar sind, kann ein Elektron
nicht ``feststellen'', ob es sich beim richtigen Atom befindet.
$u(x)$ beschreibt also ein Elektron, welches sich in der N"ahe jedes
Kristallatoms gleich verh"alt.
Man nennt die periodischen Eigenzust"ande von $H$ daher auch
{\em Kristall-Elektronen}.
Der zugeh"orige Eigenwert ist die Energie des Elektrons als ans Gitter
gebundenes Elektron.

Der Faktor $e^{iKx}$ beschreibt die Bewegung eines Kristallelektrons
durch das Gitter. Er gbit dem Elektron die zus"atzliche Bewegungsenergie
$E(K)$.

Man beachte, dass diese Zerlegung der Eigenzust"ande selbst dann gilt,
wenn das Potential verschwindet.
Dann sind die Funktionen $u$ besonders einfach zu beschreiben, es
sind die gitterperiodischen ebenen Wellen, also die Funktionen $e^{ik_0x}$ mit
$k_0\in\Gamma'$.
Tats"achlich l"asst sich jede ebene Welle $e^{ikx}$ schreiben als
\[
e^{ikx}=e^{i(K+k_0)x} = e^{iKx}e^{ik_0x}
\]
schreiben.

\begin{figure}
\centering
\includegraphics[width=\hsize]{graphics/fest-6.pdf}
\caption{Energieparabeln f"ur verschiedene Vektoren $K$. Die Energie
eines Zustands setzt sich zusammen aus der Energie $E(K)$ und der
Energie des zugeh"origen Kristallelektrons.
Die auf der $K$-Achse eingezeichneten Punkte stellen das reziproke
Gitter dar.
Grau hinterlegt die Wigner-Seitz-Zelle des reziproken Gitters.
\label{skript:dispersion}}
\end{figure}

Abbildung~\ref{skript:dispersion} zeigt die Energieparabeln in Abh"angikeit
von $K$. Da die Wahl von $K$ nur bis auf einen Summanden im reziproken
Gitter eindeutig ist, kann man einen Zustand gen"ugend hoher Gesamtenergie
auf verschiedene Arten in ein Kristallelektron und einen Faktor 
$e^{iKx}$ zerlegen.
In der Wigner-Seitz-Zelle des reziproken Gitters (hellgrau hinterlegt)
l"asst sich bereits
das ganze Energiespektrum der Elektronen im Festk"orper ablesen.

\subsection{Schwaches Potential}
In den bisherigen "Uberlegungen haben wir das Potential $V(x)$ der Gitteratome
vernachl"assigt.
Wir wollen jetzt den Fall eines schwachen Potentials untersichen.
Auch dies ist noch kein realisitisches Modell f"ur alle Elektronen im 
Kristall. Elektronen mit geringer Energie werden sich bevorzugt in der
N"ahe der Atomkerne aufhalten, und entsprechend ein starkes Potential
sp"uren.
Dadurch schirmen sie die positive Kernladung wenigstens zum Teil ab, 
so dass die Elektronen mit hoher Energie nur noch ein stark reduziertes
Feld sp"uren. Es ist also plausibel, dass die N"aherung eines schwachen
Feldes f"ur Elektronen mit hoher Energie anwendbar ist.

Ein freies Elektron mit Wellenzahl $k$ hat Energie $\hbar^2k^2/2m_e$, und wir
k"onnen die Wellenfunktion zerlegen in ein Kristallelektron und einen
Faktor $e^{iKx}$.
Das Kristallelektron ist ebenfalls eine ebene Welle.
Unter der Wirkung eines schwachen Potentials ist die Zerlegung in
Kristallelektron $u(k,x)$ und eine ebene Welle $e^{iKx}$.
Der Beitrag der Energie vom Faktor $e^{iKx}$ ist immer noch $\hbar^2K^2/2m_e$,
aber die Gesammtenergie $E(k)$ wird wohl von $\hbar^2k^2/2m_e$ abweichen.
Ziel ist, die Funktion $E(k)$ zu berechnen.

Wir wollen die Wellenfunktion des Kristallelektrons berechnen 
f"ur eine gegebene Wellenzahl $k$.
Da die Funktion $u(x)$ periodisch ist, k"onnen wir sie in eine Fourier-Reihe
entwickeln, wir nennen die Koeffizienten $u_k$:
\[
u(x)=\sum_{k'\in\Gamma'} u_{k'} e^{ik'x}.
\]

Die Wellenfunktion $\psi(k,x)$ kann mit den Koeffizienten $u_{k'}$ als
\[
\psi(k,x)
=
e^{ikx}\sum_{k'\in\Gamma'}u_{k'}e^{ik'x}
=
\sum_{k'\in\Gamma'}u_{k'}e^{i(k+k')x}.
\]
ausgedr"uckt werden.
F"ur $k'=0$ erhalten wir eine ebene Welle.
Da wir von einem schwachen Potential ausgehen, darf die Wellenfunktion
nicht stark von der ebenen Welle abweichen. Wir k"onnen also davon ausgehen,
dass $u_0$ nahe bei $1$ ist und dass die Koeffizienten $u_{k'}$
mit $k'\ne 0$ sind also wesentlich kleiner als $u_0$.

Das periodische Potential $V(x)$ kann ebenfalls in eine Fourier-Reihe
entwickeltw werden, wir schreiben
\[
V(x) = \sum_{k'\in\Gamma'}V_ke^{ik'x}.
\]
Wir nehmen an, dass der Mittelwert von $V(x)$ verschwindet, dass also $V_0=0$.

Damit haben wir alles in Fourier-Koeffizienten entwickelt, und k"onnen die
Ausdr"ucke in die Schr"odingergleichung einsetzen:
\begin{align}
H\psi(k,x)
=
-\frac{\hbar^2}{2m_e}\frac{\partial^2}{\partial x^2} \psi(k,x)
+V(x)\psi(k,x)
&=E(k)\psi(k,x)
\notag
\\
\sum_{k'\in\Gamma'}\biggl(
\frac{\hbar^2}{2m_e}(k+k')^2
+
\sum_{k''\in\Gamma'}V_{k''}e^{ik''x}
\biggr)u_{k'}e^{i(k+k')x}
&=
E(k)\sum_{k'\in\Gamma'} u_{k'}e^{i(k+k')x}
\notag
\\
\sum_{k'\in\Gamma'}\biggl(
\frac{\hbar^2}{2m_e}(k+k')^2
-E(k)
+
\sum_{k''\in\Gamma'}V_{k''}e^{ik''x}
\biggr)u_{k'}e^{i(k+k')x}
&=
0
\label{skript:periodicschroedinger}
\end{align}
Die innere Summe hindert uns daran, direkt einen Koeffizientenvergleich
anzustellen.
Wir formen daher den Potentialterm wie folgt um
\begin{equation*}
\sum_{k'\in\Gamma'}
\sum_{k''\in\Gamma'}V_{k''}u_{k'}e^{i(k+k'+k'')x}
=
\sum_{\tilde k\in\Gamma'}
\sum_{k''\in\Gamma'}V_{\tilde k-k''}u_{k''}e^{i(k+\tilde k)x}.
\end{equation*}
Dabei haben wir $\tilde k=k'+k''$ gesetzt, $k'=\tilde k-k''$.
Schreiben wir jetzt wieder $k'=\tilde k$, dann k"onnen wir die
Gleichung~(\ref{skript:periodicschroedinger}) vereinfachen zu
\begin{align*}
\sum_{k'\in\Gamma'}\biggl(
\frac{\hbar^2}{2m_e}(k+k')^2
-E(k)
\biggr)u_{k'}e^{i(k+k')x}
+
\sum_{k'\in\Gamma'}\sum_{k''\in\Gamma'}V_{k'-k''}u_{k''}e^{i(k+k')x}
&=
0
\\
\sum_{k'\in\Gamma'}\biggl(
\biggl(
\frac{\hbar^2}{2m_e}(k+k')^2
-E(k)
\biggr)u_{k'}
+
\sum_{k''\in\Gamma'}V_{k'-k''}u_{k''}
\biggl)
e^{i(k+k')x}
&=
0.
\end{align*}
Koeffizientenvergleich besagt nun, dass der grosse Klammerausdruck
verschwinden muss f"ur jeden Vektor $k'\in\Gamma'$:
\begin{equation}
\biggl(
\frac{\hbar^2}{2m_e}(k+k')^2
-E(k)
\biggr)u_{k'}
+
\sum_{k''\in\Gamma'}V_{k'-k''}u_{k''}
=
0.
\label{skript:ukgleichung}
\end{equation}
Aus dieser Gleichung k"onnen wir zwar die $u_{k'}$ nicht exakt
bestimmen, aber es sollten sich mindestens in erster Ordnung
Absch"atzungen f"ur die Koeffizienten und vor allem f"ur $E(k)$
machen lassen.

\subsubsection{Bestimmung der $u_{k'}$ in erster N"aherung}
Die Energie $E(k)$ wird in erster N"aherung mit der Energie f"ur das
freie Elektron "ubereinstimmen.
Die Koeffizienten $u_{k''}$ sind bis auf $u_0$ in erster N"aherung
vernachl"assigbar, damit wird die Gleichung~(\ref{skript:ukgleichung})
zu
\begin{align*}
\frac{\hbar^2}{2m_e}((k+k')^2 -k^2)
u_{k'}
+
V_{k'}
&=
0,
\end{align*}
die wir nach $u_{k'}$ aufl"osen k"onnen
\begin{align*}
u_{k'}
&=
-
\frac{\displaystyle V_{k'}}{
\displaystyle
\frac{\hbar^2}{2m_e}((k+k')^2 -k^2)
}.
\end{align*}
Solange der Nenner klein ist, werden auch die Koeffizienten $u_{k'}$
klein sein.
Diese N"aherung wird aber nicht mehr funktionieren, wenn $(k+k')^2\simeq k^2$
wird.
Diese Situation tritt ein, wenn $k\simeq -n\pi/a$ und $k'\simeq 2n\pi/a$.
In diesem Fall k"onnen wir auch nicht mehr annehmen,
dass $E(k)=\hbar^2k^2/2m_e$.

\subsubsection{Bragg-Reflexionen}
Die Bedingung $(k+k')^2\simeq k^2$ f"uhrt auf $k=\mp\pi n/a$ und
$k'=\pm 2\pi n/a$.
Die ebene Welle zur Wellenzahl $-\pi n/a$ ist die reflektierte 
Welle mit der Wellenzahl $\pi n/a$.
Dies sind die einzigen Wellen, deren Refklexionen konstruktiv interferieren
k"onnen, sie heissen Bragg Reflexionen.
Das Intervall $[-\frac{\pi}a,\frac{\pi}a]$ heisst Brillouin-Zone.

\subsubsection{$E(k)$ in der N"ahe von Bragg-Reflexionen}
Bragg-Reflexionen verst"arken die Komponenten $u_{\pm K}$,
mit $k'=\frac{2\pi n}a$, man kann also diesen Koeffizienten nicht mehr
als klein voraussetzen.
In den Gleichungen~\ref{skript:ukgleichung} m"ussen wir jetzt weitere
Terme hinzunehmen.

Wir schreiben $K=2\pi n/a$.
Sei $k$ ein Wellenzahl in der N"ahe von $-\pi n/a$, eine Wellenzahl, welche
zu Bragg-Reflexionen f"uhrt.
Die Gleichungen~(\ref{skript:ukgleichung})
f"ur $k'=0$ und $k'=K$ sind:
\begin{equation}
\begin{aligned}
\biggl(
\frac{\hbar^2}{2m_e}k^2
-E(k)
\biggr)u_{0}
+
V_{-K}u_{K}
&=
0,
\\
\biggl(
\frac{\hbar^2}{2m_e}(k+K)^2
-E(k)
\biggr)u_{K}
+
V_{K}u_{0}
&=
0.
\end{aligned}
\label{skript:u0ukgleichungen}
\end{equation}
Da $V(x)$ reell ist, sind die Fourier-Koeffizienten zu entgegengesetzten
$k'$ konjugiert komplex, also $V_{-k'}=\bar V_{k'}$.
Ausserdem ist $k$ so gew"ahlt, dass $(k+K)^2=k^2$ ist.
Die Gleichungen~(\ref{skript:u0ukgleichungen}) bilden ein lineares
homogenes Gleichungssystem f"ur $u_0$ und $u_K$.
Wir schreiben es in vertrauterer Form:
\begin{equation}
\begin{linsys}{3}
\displaystyle \biggl(
\frac{\hbar^2}{2m_e}k^2
-E(k)
\biggr)u_{0}
&
+
&
\bar V_{K}u_{K}
&=&
0\phantom{.}
\\
V_{K}u_{0}
&+&
\displaystyle \biggl(
\frac{\hbar^2}{2m_e}k^2
-E(k)
\biggr)u_{K}
&=&
0.
\end{linsys}
\end{equation}
Es kann nur dann
eine nichttriviale L"osung haben, wenn die Determinate verschwindet:
\[
\left|
\begin{matrix}
\displaystyle\frac{\hbar^2}{2m_e}k^2
-E(k)
&
\bar V_{K}
\\
V_{K}
&
\displaystyle\frac{\hbar^2}{2m_e}k^2
-E(k)
\end{matrix}
\right|
=
\biggl(
\frac{\hbar^2}{2m_e}k^2-E(k)
\biggr)^2-|V_K|^2=0.
\]
Aufl"osen nach $E(k)$ ergibt
\[
E(k)=\frac{\hbar^2}{2m_e}k^2\pm|V_K|.
\]




