\chapter{Flash-Speicher\label{chapter:flash}}
\lhead{Flash-Speicher}
\begin{refsection}
\chapterauthor{Roger Billeter und Gabriel Looser}

\newpage
\section{Einleitung}
\rhead{Einleitung}
Der Flash-Speicher ist ein elektrisches Speicherelement, welches weit verbreitet ist. Grunds"atzlich wird zwischen zwei Vorg"angen unterschieden, dem Schreib- und dem L"oschvorgang. Der L"oschvorgang beruht auf dem Tunneleffekt, welcher im Kapitel Tunneldiode ausf"uhrlich erkl"art wird. F"ur den Speichervorgang m"ussen Elektronen mittels 'hot electron injection' in das Floating-Gate gebracht werden. 

\section{Flash-Typen}
\rhead{Flash-Typen}
Bei den Flash-Speichern wird zwischen zwei Typen unterschieden, den NAND und den NOR Typen. Diese unterscheiden sich in der Zusammenschaltung der einzelnen Transistoren. 

\subsection{NAND Typ}
\rhead{NAND Typ}
Bei den NAND Flash-Speicher werden die einzelnen Transistoren zu einer NAND Speicherzelle zusammen geschaltet. 
Die Speicherzellen werden in Gruppen untereinander in Reihe geschaltet. Die Speicherzellengruppe teilt sich eine Datenleitung, dadurch ist muss das Lesen und Schreiben immer in der ganzen Gruppe sequentiell erfolgen. Dadurch wird die Anzahl Datenleitungen reduziert und der Flash-Speicher ben"otigt weniger Platz. NAND Flash-Speicher haben ihre Verwendung vor allem in Bereichen, wo viel Speicher m"oglichst wenig Platz brauchen darf und die Zugriffszeit keine grosse Rolle spielt.
\subsection{NOR Typ}
\rhead{NOR Typ}
Bei den NOR Flash-Speicher werden die Transistoren zu einer NOR Speicherzelle zusammen geschaltet.
Die Speicherzellen werden "uber die Datenleitungen parallel geschaltet, dadurch kann auf jede Speicherzelle direkt zugegriffen werden. Die Parallelleitungen brauchen mehr Platz, es lassen sich jedoch k"urzere Zugriffszeiten realisieren.

\section{Floating Gate Transistor}
\rhead{Floating Gate}
Floating Gate Transistoren z"ahlen zu den Feldeffekttransistoren. Das Ziel ist es den Zustande des Floating Gates zu erh"ohen. Dies kann man erreichen, indem im Floating Gate eine gewisse Ladung gespeichert wird. Diese kann erh"oht werden, indem das Potential am Control Gate erh"oht wird. Nun fliessen die Elektronen in Richtung Source. 

\begin{figure}[h]
\centering
\includegraphics[width=0.7\textwidth]{graphics/Floatingate.pdf}
\caption{Floating Gate Transistor: Quelle http://upload.wikimedia.org/wikipedia/commons/a/ae/Floating_gate_transistor-en.svg
\label{skript:Flash}}
\end{figure}

Je nachdem was es f"ur ein Potential am Control Gate hat, können mehrere Zust"ande gespeichert werden. In einem Floating Gate Transistor kann man auch mehrere Bits speichern, "ublicherweise werden 2 Bits gespeichert.

Der Source Anschluss wird meistens an GND geh"angt und Drain wird beim Leseprozess auf 0V gehängt und beim Schreibeprozess "ublicherweise auf über 10V. 


\section{Zyklenzahlen}
\rhead{Zyklenzahlen}
Ein grosses Problem bei Flashspeichern ist die begrenzte schreib-und Leseanzahl. Das Problem ist, dass die Oxidschicht um das Floating Gate herum. Dies bewirkt, dass immer mehr der Ladung abfliessen kann. Bei den NOR-Flashs hat man etwa 10'000 bis 100'000 Zyklen. Hingegen bei den NAND-Flashs sind es doch gegen 2 Millionen Zyklen. Dieses Problem ist jedoch nicht so problematisch, da der Speicher mit einzelnen Blöcken aufgebaut ist. Wenn einer ausf"allt hat man immer noch einige andere Speicherzellen.

\section{Modell}
\rhead{Modell des Potentialtopf}
Als Modell f"ur das Floating-Gate wird ein Potentialtopf verwendet, in dem die Elektronen gespeichert werden. 
Als Vereinfachung des Schreibvorgangs kann man sich das Putten beim Golf vorstellen. Das Elektron ist der Golfball und der Potentialtopf das Loch. Das Elektron landet jedoch nicht wegen der Schwerkraft im Potentialtopf sondern wird durch die Plasmonen abgebremst. Wird das Elektron stark genug abgebremst, landet es im Potentialtopf.\\
Am Anfang befindet sich im Potentialtopf ein Grundpotential $E_{0}$ das Ziel ist es dieses Potential auf ein h"oheres Potential $E_{1}$ zu bringen. Da die L"osung der Urspr"unglichen Gleichung auf ein partielles Differential-Gleichungssystem f"uhrt und da das L"osen partieller Differentialgleichungen nicht Stoff des Bachelorstudiums ist, werden die Gleichungen mittels Skalarpordukt auf ein normales Differential-Gleichungssystem vereinfacht. Schlussendlich wird die Wahrscheinlichkeit eines Zustandes im Potentialtopf errechnet. Diese Wahrscheinlichkeit ist abh"angig von der Energie des Elektrons, vom Abstand des Elektrons zum Potentialtopf und der Geschwindigkeit respektive der Zeit mit der das Elektron auf den Potentialtopf wirkt (Siehe MatLab plots, Bild Potentialtopf).


\section{Anregung}
\rhead{Anregung}
Der Potentialtopf mit dem Grundpotential $E_{A}$ hat eine Grundschwingung $cos(x)$.
Als Vereinfachung dieser Grundschwingung dient die Vorstellung eines Wackelpuddings, welcher durch das vorbei fliegende Elektron, zus"atzlich in Schwingung versetzt wird. 
Das Grundpotential wird asymmetrisch angeregt. Wenn man das auf das Modell des Wackelpudding "ubertr"agt, erkennt man, dass wenn man einen Wackelpudding gleichm"assig zusammendrückt, wird dieser nicht zus"atzlich in Schwingung versetzt. Wird der Wackelpudding jedoch auf der einen Seite angestossen, wird der Wackelpudding zus"atzlich in Schwingung versetzt und erreicht ein h"oheres Potential.\\
F"ur die Berechnung wird der Potentialtopf "uber die h"alfe mit einem Rechteckimpuls angeregt. Dies vereinfacht das L"osen der Differentialgleichungen.
%(Bild Anregung)

\section{L"oschvorgang}
\rhead{Lo"schvorgang}
Der L"oschvorgang kann erreicht werden, indem das Potential am Gate heruntergesetzt wird. So wird erreicht, dass es zum Tunneleffekt kommt. 

% Verweis zu Stefan Hedinger??? 

\section{Schreibvorgang}

\subsection{Schr"odingergleichung}
\rhead{Schr"odingergleichung}
Aus der Anregung und dem Protentialtopf kann man nun die Schr"odingergleichung aufstellen.
\[
\ \imath\hbar\frac{d}{dt}|\psi(t)\rangle = H|\psi(t)\rangle = (-\frac{\hbar^2}{2m} \frac{\partial^2}{\partial x^2}+V(x)+V_{e}(x,t))|\psi(t)\rangle
\]

Wenn man den Zustand nun von 0 auf 1 erh"oht sieht dies folgendermassen aus:
\[
\ \imath\hbar\frac{d}{dt}\langle1|\psi(t)\rangle = \langle1|-\frac{\hbar^2}{2m} \frac{\partial^2}{\partial x^2}+V(x)+V_{e}(x,t)|\psi(t)\rangle
\]

Was nun genauer ausgerechnet wird ist $langle1|\psi(t)\rangle$, denn dies entspricht der Wahrscheinlichkeit, dass der $|1\rangle$ erreicht ist.

Wenn man die Gleichung nun noch vereinfacht, sieht man das diese Wahrscheinlichkeit mehrfach vorkommt.
\[
\ \imath\hbar\frac{d}{dt}\langle1|\psi(t)\rangle = E_{1}\langle1|\psi(t)\rangle+\langle1|\psi(t)\rangle V(t)
\]

Das Ganze kann man nun auch mit dem $|0\rangle$ machen. Diese Gleichung sieht sehr "ahnlich aus.
\[
\ \imath\hbar\frac{d}{dt}\langle0|\psi(t)\rangle = E_{0}\langle0|\psi(t)\rangle+\langle0|\psi(t)\rangle V(t)
\]

Einfachheitshalber schreibt man anstatt $\langle1|\psi(t)\rangle$ $a_1$ entsprechend für $\langle0|\psi(t)\rangle$ $a_0$. Somit ergibt sich dann folgendes Diefferentialgleichungssystem:
\[
\ \imath\hbar\frac{d}{dt}a_{0}(t)\langle0|0\rangle +\imath\hbar\frac{d}{dt}a_{1}(t)\langle0|1\rangle = a_{0}(t)E_{1}\langle0|0\rangle + a_{1}(t)E_{1}\langle0|1\rangle + a_{0}(t)\langle0|V(t)|0\rangle+ a_{1}(t)\langle1|V(t)|0\rangle
\]
\[
\ \imath\hbar\frac{d}{dt}a_{0}(t)\langle1|0\rangle +\imath\hbar\frac{d}{dt}a_{1}(t)\langle1|1\rangle = a_{0}(t)E_{1}\langle1|0\rangle + a_{1}(t)E_{1}\langle1|1\rangle + a_{0}(t)\langle0|V(t)|0\rangle+ a_{1}(t)\langle1|V(t)|0\rangle
\]

\subsection{Skalarprodukt}
\rhead{Skalarprodukt}
Mittels Skalarprodukt kann das Differentialgleichungssystem weiter vereinfacht werden.
Denn wie man weiss ergibt das Skalarprodukt von $\langle0|0\rangle$ und $\langle1|1\rangle = 1$. Genauso wie das Skalarprodukt von $\langle0|1\rangle$ und $\langle1|0\rangle = 0$.
Ausserdem kann  $\langle0|V(t)|0\rangle$ als $V_{0}v_{00}$ geschrieben werden, $\langle1|V(t)|0\rangle$ als $V_{0}v_{01}$,$\langle0|V(t)|0\rangle$ als $V_{0}v_{11}$ und $\langle1|V(t)|0\rangle$ als $V_{0}v_{10}$.
Daraus folgt:
\[
\ \imath\hbar\dot{a_{0}} = (E_{0} + V_{0} v_{00}) a_{0} + V_{0} v_{01} a_{1}
\]
\[
\ \imath\hbar\dot{a_{1}}_dot = V_{0} v_{10} a_{0} + (E_{1} V_{0} v_{11}) a_{1}
\]


\section{Variation der Konstanten}
\rhead{Variation der Konstanten}
Die vereinfachte Gleichung besitzt durch die Grundschwingung die Eigenschaft, dass sie relativ leicht oszilliert. Diese Oszillation kann jedoch heraus gefiltert werden, indem man die Konstanten variiert. \\ Das Prinzip der Variation der Konstanten wird an folgendem Beispiel illustriert.\\

\[
\ \dot{a} = C a(t) + D(t) \Rightarrow a(t) = k(t) e^{C t}
\] 
durch ableiten und das einsetzen von $ a(t)$ und  $ \dot{a} $ erhalten wir\\

\[
\ k'(t) e^{C t} + C k(t) e^{C t} = C k(t) e^{C t} + D(t)
\] 
Hier k"urzt sich $ C k(t) e^{C t} $ raus.
Daraus folgt:\\
\[
\ k'(t) = D(t) e^{-C t}
\] 
integriert diese Gleichung nach der Zeit folgt:\\
\[
\ k(t) = \int D(t) e^{-C t} dt 
\]\\
 
Angewandt auf unser System:\\
\[
\ \dot{a_{0}}(t) = C_{00}a_{0}(t) + C_{01}a_{1}(t)
\]
\[
\ \dot{a_{1}}(t) = C_{10}a_{0}(t) + C_{11}a_{1}(t)
\]

Wenn $ C_{01}$ und $ C_{10}$ $ =0$ dann w"are die L"osung:\\
\[
\ a_{0}(t) = konst_{0} e^{C_{00} t} 
\]
\[
\ a_{1}(t) = konst_{1} e^{C_{11} t}
\]
Wenn die Konstanten auch von der Zeit abh"angig sind, ist die L"osung: \\
\[
\ a_{0}(t) = konst_{0}(t) e^{C_{00} t} 
\]
\[
\ a_{1}(t) = konst_{1}(t) e^{C_{11} t} 
\]
Eingesetzt in das Differentialgleichungssystem folgt daraus\\
\[
\ k'_{0}(t) e^{C_{00} t} + k_{0}(t) C_{00} e^{C_{00} t} = k_{0}(t) C_{00} e^{C_{00} t} + k_{1}(t)C_{01}e^{C_{11} t}
\]
\[
\ k'_{1}(t) e^{C_{11} t} + k_{1}(t) C_{11} e^{C_{11} t} = k_{0}(t) C_{10} e^{C_{00} t} + k_{1}(t)C_{11}e^{C_{11} t}
\]
Gemeinsame Terme werden heraus gek"urtzt, daraus folgt:\\
\[
\ k'_{0}(t) e^{C_{00} t} = k_{1} C_{01} e^{C_{11} t}
\]
\[
\ k'_{1}(t) e^{C_{11} t} = k_{0} C_{10} e^{C_{00} t}
\]
Diese Gleichungen nach $ k'_{0}(t)$ und $ k'_{1}(t)$ aufgel"ost ergibt:\\
\[
\ k'_{0}(t) = k_{1}(t) C_{01} e^{(C_{11}-C_{00}) t}
\]
\[
\ k'_{1}(t) = k_{0}(t) C_{10} e^{(C_{00}-C_{11}) t}
\]
Unsere inhomogene Differential-Gleichung erster Ordnung verwandelt sich dadurch in eine homogene Differential-Gleichung zweiter Ordnung.\\
\[ 
\ k''_{0}(t) - C_{11}-C_{00} k'_{0}(t) - C_{10}C_{01}k(t) = 0
\]
\[
\ k''_{1}(t) - C_{00}-C_{11} k'_{1}(t) - C_{01}C_{10}k(t) = 0
\]
Das charakteristische Polynom lautet\\
\[
\ \lambda_{0}^{2} - C_{11}-C_{00}\lambda_{0} - C_{10}C_{01} = 0
\]
\[
\ \lambda_{1}^{2} - C_{00}-C_{11}\lambda_{1} - C_{01}C_{10} = 0
\]
Die $ \lambda $ der ersten Gleichung lauten:
\[
\ \lambda_{1,2} = \frac{(C_{11}-C_{00})\pm \sqrt{(C_{11}-C_{00})^2-4(C_{10}C_{01})}}{2}
\]
Die $ \lambda $ der zweite Gleichung lauten
\[
\ \lambda_{3,4} = \frac{(C_{00}-C_{11})\pm \sqrt{(C_{00}-C_{11})^2-4(C_{01}C_{10})}}{2}
\]

\printbibliography[heading=subbibliography]
\end{refsection}

