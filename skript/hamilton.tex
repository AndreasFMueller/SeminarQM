\chapter{Klassische Mechanik\label{chapter:mechanik}}
\lhead{Klassische Mechanik}
\rhead{}
\section{Motivation}
\rhead{Motivation}
W"ahrend Galileo Galileis Beschreibung von Bewegungen im Wesentlichen
eine kinematische war, haben Isaac
Newtons Gesetze erstmals einen Zusammenhang zwischen
der Bewegung und den Kr"aften hergestellt, also von Ursache und
Wirkung. Sein erstes Gesetz besagt zum Beispiel
\[
F=ma\qquad\text{oder}\qquad
F(x)=m\frac{d^2x}{dt^2}.
\]
Mit geeigneten Anfangsbedingungen l"asst sich daraus jede Bewegung 
vorhersagen.

Diese Formulierung ist jedoch nicht allgemein genug, insbesondere f"ur die
Zwecke der Quantenmechanik. Wenn Teilchen auch Wellen sind, wie soll man
sich dann die Kraftwirkung einer Welle auf eine andere Welle vorstellen?

Die Antwort ist die Hamiltonsche Formulierung der Mechanik, die von der
Energie als der alles bestimmenden Gr"osse ausgeht.
Mit Hilfe eines Standardformalismus lassen sich daraus die Newtonschen
Bewegungsgleichungen wieder gewinnen.
Viel wichtiger ist aber, dass sich daraus auch ein Methode gewinnen
l"asst, besonders geeignete Koordinaten zur Beschreibung eines
physikalischen Systems zu finden.
Diese Hamilton-Jacobi-Theorie genannte Methode ist die Basis eines
allgemein anwendbaren Quantisierungs-Verfahrens.

\section{Lagrange-Mechanik}
\rhead{Lagrange-Mechanik}
\index{Lagrange-Mechanik}
Das Fermatsche Prinzip besagt, dass Licht immer den Weg k"urzester Zeit
nimmt. Mit diesem Prinzip konnte Fermat sowohl das Reflexionsgesetz als
auch das Brechungsgesetz erkl"aren.
Im Lichte unserer beabsichtigen Vereinheitlichung zwischen Lichtwellen
und Teilchen sollte f"ur die Teilchenwelt ein entsprechendes Gesetz
gelten. Tats"achlich hat Pierre-Louis Maupertuis ein solches Prinzip
gefunden: ein mechanisches System entwickelt sich immer so, dass die
{\em Wirkung} minimal ist. Das Prinzip wurde von Euler, Lagrange 
und schliesslich Hamilton verallgemeinert.
\subsection{Wirkung}
Ein mechanisches System, zum Beispiel in sich bewegender Massepunkt,
wird durch eine Anzahl von Koordinaten beschrieben, die wir mit
$q_i,1\le i\le n$ bezeichnen. Die Koordinaten werden im allgemeinen 
von der Zeit abh"angen, man k"onnte also auch $q_i(t)$ schreiben.
Die Entwicklung des Systems wird durch Differentialgleichungen 
f"ur diese Funktionen $q_i(t)$ und Anfangsbedingungen beschrieben.
Zwischen den Zeitpunkten $t_0$ und $t_1$ entwickelt sich das
System
von einem Zustand $q(t_0)$ in einen Zustand $q(t_1)$.

Die Bewegung eines Teilchens wird im wesentlichen durch seine 
kinetische und potentielle Energie bestimmt. Die kinetische Energie
ist ein quadratischer Ausdruck in den Geschwindigkeiten:
\[
T=\frac12m_i\dot q_i^2.
\]
die potentielle Energie ist eine Funktion, die nur von der
Position des Teilchens abh"angt:
\[
V(q_1,\dots,q_n)=V(q).
\]
Die Lagrange-Funktion eines Systems ist die Differenz:
\[
L(\dot q, q) = T(\dot q) - V(q)
\]
Selbstverst"andlich ist es auch m"oglich, dass die kinetische oder
die potentielle Energie
zus"atzlich von der Zeit abh"angt, zum Beispiel "uber eine
zeitlich ver"anderliche Masse.
Sie kann auch noch weit komplizierter sein, um zum Beispiel Kr"afte
zu beschreiben, die von der Geschwindigkeit abh"angen wie der
Str"omungswiderstand oder die Bremskraft, die durch Wirbelstr"ome in
einem Magnetfeld verursacht wird.

Die Wirkung oder Aktion einer Zeitentwicklung zwischen den Zust"anden
$q(t_0)$ und $q(t_1)$ ist die Gr"osse
\begin{equation}
W=\int_{t_0}^{t_1} L(t, q(t), \dot q(t))\,dt.
\label{skript:wirkung}
\end{equation}
Die Wirkung hangt also von der konkreten Wahl des Pfades $q(t)$ ab,
um diese Abh"angigkeit auszudr"ucken, wird manchmal auch $W[q(t)]$
geschrieben.

Das Prinzip minimaler Wirkung besagt, ein mechanisches System unter
allen m"oglichen Pfaden, die es vom Zustand $q(t_0)$ in den Zustand
$q(t_1)$ "uberf"uhren, immer denjenigen w"ahlt, f"ur die die
Wirkung minimal wird.

\subsection{Euler-Gleichungen}
\index{Euler-Gleichungen}
\index{Wirkung}
Die Minimierung der Wirkung (\ref{skript:wirkung}) ist ein sogenanntes
Variationsproblem: gesucht ist ein Weg, welcher ein bestimmtes
Integral minimiert. Es ist verwandt mit der Aufgabe, den k"urzesten
Weg zu finden, der ein Gebiet mit einem bestimmten Fl"acheninhalt
umschliesst. Johann Bernoulli hat 1696 mit seinem Brachistochronen-Problem
die mathematische Welt mit einem ersten ber"uhmten Problem dieser Art
konfronriert, mit dem sich auch einige Mathematiker befasst haben.
Systematisch mit dieser Art von Problem hat sich aber erst Leonhard
Euler auseinandergesetzt, der auch einen allgemeinen L"osungsweg angegeben hat.

Gesucht wird also eine Kurve $q(t)$, welche $W[q(t)]$ minimal macht.
Jede andere Kurve, die sich nur wenig von $q(t)$ unterscheidet, sollte
daher eine gr"ossere Wirkung haben.
\begin{figure}
\centering
\includegraphics{graphics/lagrange-1.pdf}
\caption{Kurve und Nachbarkurven f"ur die Herleitung der Eulerschen
Differentialgleichung
\label{skript:nachbarkurven}}
\end{figure}
Eine solche Nachbarkurve k"onnen wir als
\[
q(t) + \varepsilon \eta(t)
\]
ansetzen (Abbildung~\ref{skript:nachbarkurven}),
wobei $\eta(t)$ eine beliebig w"ahlbare Funktion ist,
die f"ur $t_0$ und $t_1$ verschwindet, $\eta(t_0)=\eta(t_1)=0$.
Wenn $q(t)$ L"osung des Minimalproblems ist, muss
\begin{equation}
\varepsilon\mapsto W[q(t)+\varepsilon\eta(t)]
\label{skript:variation-ansatz}
\end{equation}
ein Minimum f"ur $\varepsilon=0$ haben, oder die Ableitung
dieser Funktion nach $\varepsilon$ muss $0$ sein.

Setzten wir den Ansatz (\ref{skript:variation-ansatz}) in die Wirkung ein
und leiten wir nach $\varepsilon$ ab, erhalten wir
\begin{align*}
\frac{d}{d\varepsilon}W[q(t)+\varepsilon\eta(t)]
&=
\frac{d}{d\varepsilon}\int_{t_0}^{t_1}L(t, q(t)+\varepsilon\eta(t),
\dot q(t)+\varepsilon\dot\eta(t))\,dt
\\
&=\int_{t_0}^{t_1}
\frac{\partial L}{\partial q}(t, q(t)+\varepsilon\eta(t), \dot q(t)+\varepsilon\dot\eta(t))\eta(t)
+
\frac{\partial L}{\partial \dot q}(t, q(t)+\varepsilon\eta(t), \dot q(t)+\varepsilon\dot\eta(t))\dot \eta(t)\,dt.
\end{align*}
Da $q$ eigentlich ein Vektor ist, ist $\partial L/\partial q$ zu lesen
als ein Vektor bestehend aus allen Ableitungen von $L$ nach den verschiedenen
Koordinaten $q_i$, und analog f"ur $\partial L/\partial \dot q$.
F"ur $\varepsilon=0$ soll dieser Ausdruck verschwinden:
\begin{equation}
0
=\int_{t_0}^{t_1}
\frac{\partial L}{\partial q}(t, q(t), \dot q(t))\eta(t)
+
\frac{\partial L}{\partial \dot q}(t, q(t), \dot q(t))\dot \eta(t)\,dt.
\label{skript:erstevariationsgleichung}
\end{equation}
Zur Abk"urzung schreiben wir im Folgenden
\begin{align*}
\frac{\partial L}{\partial q}(t, q(t), \dot q(t))
&=
\frac{\partial L}{\partial q}
&
&\text{und}
&
\frac{\partial L}{\partial \dot q}(t, q(t), \dot q(t))
&=
\frac{\partial L}{\partial \dot q}.
\end{align*}
Der Faktor $\dot \eta(t)$ im zweiten Term hindert uns daran zu verstehen,
wie dieses Integral verschwinden kann. 
Wir k"onnen ihn aber zum Verschwinden bringen, indem wir ihn mit
partieller Integration umformen:
\begin{align*}
0
&=\int_{t_0}^{t_1}
\frac{\partial L}{\partial q}(t, q(t), \dot q(t))\eta(t)
+
\frac{\partial L}{\partial \dot q}(t, q(t), \dot q(t))\dot \eta(t)\,dt
\\
&=
\int_{t_0}^{t_1}\frac{\partial L}{\partial q}\eta(t)\,dt
+\left[
\frac{\partial L}{\partial \dot q} \eta(t)
\right]_{t_0}^{t_1}
-\int_{t_0}^{t_1}\frac{d}{dt}\frac{\partial L}{\partial \dot q}\eta(t)\,dt
\\
&=\int_{t_0}^{t_1}\left(\frac{\partial L}{\partial q}
-\frac{d}{dt}\frac{\partial L}{\partial \dot q}\right) \eta(t)\,dt.
\end{align*}
Diese letzte Gleichung ist nur dann f"ur jede beliebige Funktion $\eta(t)$
erf"ullbar, wenn der Klammerausdruck verschwindet. So erhalten wir die
{\em Eulerschen Gleichungen} f"ur das Variationsproblem.

\begin{satz}
Die Wirkung
\[
W[q(t)] =\int_{t_0}^{t_1} L(t, q(t), \dot q(t))\,dt
\]
wird minimiert von einer Funktion $q(t)$, welche den 
{\em Eulerschen Differentialgleichungen}
\begin{equation}
\frac{d}{dt}\frac{\partial L}{\partial \dot q}-\frac{\partial L}{\partial q}=0
\qquad\Leftrightarrow\qquad
\frac{d}{dt}\frac{\partial L}{\partial \dot q_i}-\frac{\partial L}{\partial q_i}=0\quad\forall i
\label{skript:euler-dgl}
\end{equation}
gen"ugt.
Die Eulerschen Differentialgleichungen sind gew"ohnliche
Differentialgleichungen zweiter Ordnung f"ur die Funktionen $q_i(t)$.
\end{satz}

\subsection{Bewegungsgleichungen}
\index{Maupertuis}
\index{Wirkung!Prinzip der kleinsten}
Nach Maupertuis soll das Prinzip der kleinsten Wirkung eine Bewegung
entsprechen den Newtonschen Gesetzen ergeben.
Wir wenden daher die Eulergleichungen auf das Problem eines Teilchens der
Masse $m$ in einem Potential $V(q)$ an. Die Lagrange-Funktion ist
\[
L(t, q, \dot q)=\frac12m\dot q^2-V(q).
\]
Die Ableitungen nach $q$ und $\dot q$ sind:
\begin{align*}
\frac{\partial L}{\partial \dot q}&=m\dot q=p\\
\frac{\partial L}{\partial q}&=-\frac{\partial V}{\partial q}=-\operatorname{grad}V(q).
\end{align*}
Der Ausdruck $m\dot q$ ist aus der Newtonschen Mechanik bekannt als
der Impuls eines Teilchens.
Die Eulersche Gleichung (\ref{skript:euler-dgl}) wird daher zu
der Bedingung:
\[
\frac{d}{dt}\frac{\partial L}{\partial \dot q}=\frac{\partial L}{\partial q}
\qquad\Rightarrow\qquad
\frac{d}{dt}p=-\operatorname{grad}V(q).
\]
Der negative Gradient des Potentials ist die Kraft $F=-\operatorname{grad}V(q)$,
die auf das Teilchen an der Stellen $q$ wirkt.
Wir erhalten also die Newtonschen Bewegungsgleichungen.
Falls die Masse konstant ist, kann man die Ableitung nach der Zeit
noch vereinfachen, und erh"alt die wohlbekannte Form des Newtonschen
Gesetzes
\[
F= m\ddot q.
\]

Man beachte, dass diese Herleitung auch dann noch gilt, wenn die Masse von der
Zeit abh"angt (zum Beispiel bei einer Rakete).

\subsection{Fermat-Prinzip}
\index{Fermat-Prinzip}
Das Fermat-Prinzip besagt, dass das Licht immer den Weg mit k"urzester
Laufzeit zwischen zwei Punkten w"ahlt.
Wir beschreiben den Weg, den das Licht von $x_0$ zu $x_1$ nimmt,
als eine Kurve $x(s)$, wobei $s\in[s_0,s_1]$ der Parameter entlang
der Kurve ist.
Die Lichtgeschwindigkeit h"angt vom Brechungsindex des Mediums ab.
Sei $n(x)$ der Brechungsindex an der Stelle $x$, dann ist 
$c/n(x)$ die Lichtgeschwindigkeit an der Stelle $x$.
Zwei Punkte auf dem Strahl, deren Parameter sich um $\Delta s$
unterscheiden, haben die  Entfernung $|x'(s)|\Delta s$.
Das Licht braucht daher f"ur die ganze Kurve die Zeit
\[
T=\int_{s_0}^{s_1} |x'(s)|\, n(x(s))\frac1c\,ds.
\]
Der Faktor $1/c$ hat keinen Einfluss auf das Minimum, wir k"onnen ihn
weglassen.
Wir k"onnen jetzt Euler-Gleichungen f"ur die Lagrange-Funktion
$L(x', x)= |x'|\,n(x)$ 
ableiten:
\begin{align*}
\frac{\partial L}{\partial x_i}
&=
|x'|\frac{\partial n}{\partial x_i}
\\
\frac{\partial L}{\partial x'_i}
&=
n(x)\frac{\partial |x'|}{\partial x_i'}
=
n(x)\frac{\partial}{\partial x'_i}\sqrt{\sum_{k=1}^3x_k'^2}
=
n(x)\frac{x_i'}{|x'|}
\end{align*}
Die Wahl der Parametrisierung der Kurve darf keinen Einfluss auf den
gew"ahlten Weg haben, also k"onnen wir eine Parametrisierung w"ahlen,
f"ur die $|x'(s)|=n(x)$ ist.
Dadurch vereinfachen sich die Ausdr"ucke f"ur die partiellen
Ableitungen zu
\begin{align*}
\frac{\partial L}{\partial x_i}
&=
n(x)\frac{\partial n}{\partial x_i}
\\
\frac{\partial L}{\partial x_i'}
&=x_i'
\end{align*}
Damit wird die Bewegungsgleichung
\begin{align*}
\frac{d}{ds}\frac{\partial L}{x_i'}-\frac{\partial L}{\partial x_i}
&=
x_i''-n\frac{\partial n}{\partial x_i}=0
\\
x_i''
&=
n\frac{\partial n}{\partial x_i}=\frac12 \frac{\partial n(x)^2}{\partial x_i}.
\end{align*}
Die Kurven sehen aus wie die Bewegung eines Teilchens in einem
``Potential'' $n(x)^2$.

Das Fermat-Prinzip beschreibt die geometrische Optik, also die Ausbreitung
von Licht mit so kurzer Wellenl"ange, dass sie gegen"uber den
Abmessungen des Problems vernachl"assigbar ist.
Dies ist genau die gleiche Art von Grenz"ubergang, die wir f"ur
die Wellen der Quantenmechanik machen m"ochten.
Das Fermat-Prinzip als Grenzwert einer Wellenausbreitung f"ur
unendlich kurze Wellenl"ange f"uhrt auf eine Bewegungsgleichung,
die aussieht wie die Bewegung eines Teilchens in einem Potential.

\section{Hamiltonsche Mechanik}
\rhead{Hamiltonsche Mechanik}
\index{Hamiltonsche Mechanik}
Im Fermat-Prinzip haben wir einen Zusammenhang zwischen dem Formalismus
der Mechanik und der geometrischen Optik kennengelernt, der mindestens
andeutet, dass die Mechanik nur der Grenzfall f"ur sehr kurze Wellenl"ange
einer ``Wellenmechanik'' sein k"onnte.
Die Lagrange-Mechanik ist aber nicht allgemeine genug, um diesen
Schritt zu erm"oglichen.
Die noch etwas allgemeinere Formulierung der Mechanik durch Hamilton
schafft dies.

Die Lagrange-Formulierung der Mechanik f"uhrt auf die Eulergleichungen,
die Differentialgleichungen zweiter Ordnung f"ur die Funktionen $q_i(t)$ 
sind.
Zwar l"asst sich jede Differentialgleichung zweiter Ordnung einfach
dadurch als Differentialgleichung erster Ordnung schreiben, indem
zu den Variablen $q_i$ die zus"atzlichen Variablen $v_i=\dot q_i$ und die
Gleichungen $\dot q_i=v_i$ hinzugef"ugt werden, doch ist diese Technik
nicht ganz befriedigend, weil die $v_i$ physikalisch nicht so ``gute''
Variablen sind.
Die erweiterte Formulierung der Mechanik soll also durch
Differentialgleichungen erster Ordnung mit zus"atzlichen
Variablen $p_i$ erfolgen, die eine ``spannende'' physikalische
Bedeutung haben.

\subsection{Hamilton-Funktion}
\index{Hamilton-Funktion}
Die Hamilton-Funktion $H(p,q)$ ist die Energie eines physikalischen
Systems. Darin sind $q$ die allgemeinen Koordinaten, also
zum Beispiel die Raumkoordinaten $x$, $y$ und $z$, oder Winkel, mit
denen man die r"aumliche Ausrichtung eines Systems beschreibt.
Die Variablen $p$ sind die zugeh"origen Impulse, f"ur Ortskoordinaten
sind dies die gew"ohnlichen Impulskomponenten, f"ur die Drehwinkel ist
es der Drehimpuls.

Die Energie $H$ setzt sich zusammen aus der kinetischen Energie $T$ und der
potentiellen Energie $V$. F"ur ein kr"aftefreies Teilchen der Masse $m$
ist die Energie 
\[
H=\frac12mv^2=\frac1{2m}p^2.
\]
Befindet sich das Teilchen dagegen in einem Potential $V(q)$, dann 
ist die zugeh"orige Hamilton-Funktion
\begin{equation}
H=T+V=\frac1{2m}p^2+V(q).
\label{skript:hamilton-potential}
\end{equation}
Ein Elektron im elektrischen Feld eines Protons, welches im Nullpunkt
des $q$-Koordinatensystems sitzt, hat daher die Hamilton-Funktion
\[
H=\frac1{2m}p^2+\frac{e^2}{4\pi\varepsilon_0|q|}.
\]

Eine noch zu "uberwindende Schwierigkeit dieses Formalismus wird hier
bereits erkennbar: Kr"afte, die keine Arbeit leisten, k"onnen auch
keinen Beitrag zur Energie leisten.
Ein Beispiel ist
die Lorentzkraft eines Magnetfelds, welche die Bahn eines Elektrons 
kr"ummt, aber immer senkrecht auf der Bahn steht und daher keine Arbeit
leistet.

\subsection{Von der Lagrange-Funktion zur Hamilton-Funktion}
Die Frage, welche Impulse zu verwenden sind, wurde noch nicht beantwortet
worden.
Die Euler-Gleichung sagen, dass die Gr"osse, f"ur die wir die Zeitableitung
berechnen k"onnen, die Ableitung $\partial L/\partial\dot q$ ist.
Die Euler-Gleichung liefert also eine Bewegungsgleichung im Stile des
Newtonschen Gesetzes, wenn man setzt
\[
p=\frac{\partial L}{\partial \dot q}.
\]
Die kinetische Energie k"onnten wir auch als
\[
T=\sum_i \frac12m\dot q_i^2=\sum_i \frac12 p_i\dot q_i
\]
schreiben, also muss die gesamte Energie sein
\begin{equation}
H(p,q)=T+V = 2T - (T - V)
=
\sum_k p_k\dot q_k - L(\dot q, q).
\label{skript:von-lagrange-zu-hamilton}
\end{equation}
Dieser Ausdruck h"angt nicht mehr von $\dot q_i$ ab, denn wir
k"onnen ausrechnen
\[
\frac{\partial H}{\partial\dot q_i}
=
\frac{\partial}{\partial \dot q_i}\sum_k p_k\dot q_k
-
\frac{\partial L}{\partial \dot q_i}
=
p_i-p_i=0.
\]
Es ist also sicher m"oglich, die Hamilton-Funktion als Funktion von $p$
und $q$ zu schreiben.

\subsection{Bewegungsgleichungen}
Die Newtonschen Bewegungsgleichungen lassen sich aus der Hamilton-Funktion
gewinnen. Um dies zu verstehen, berechnen wir die partiellen Ableitungen
von $H$ nach Koordinaten $q$ und Impulsen $p$:
\begin{align*}
\frac{\partial H}{\partial p}&=\frac{1}{m}p=v=\frac{dq}{dt} \\
\frac{\partial H}{\partial q}&=\frac{\partial V}{\partial q}
\end{align*}
Die erste Gleichung ist nichts anders als die Aussage, dass die
Geschwindigkeit die Zeitableitung der Ortskoordinaten ist, sie
stellt den Zusammenhang zwischen $p$ und $q$ her.
Die Ableitung des Potentials auf der rechten Seite der zweiten
Gleichung hat die Bedeutung einer Kraft.
Die Newtonschen Bewegungsgleichungen sagen, dass die "Anderung des
Impulses durch die Kr"afte verursacht wird, also durch die Ableitungen
des Potentials:
\[
\frac{dp}{dt}=-\frac{\partial V}{\partial q}
\]
Damit sind die Bewegungsgleichungen jetzt:
\begin{align}
\frac{dq}{dt}&= \frac{\partial H}{\partial p},\label{skript:hamilton-v}\\
\frac{dp}{dt}&=-\frac{\partial H}{\partial q}.\label{skript:hamilton-newton}
\end{align}

\begin{beispiel}
\begin{figure}
\centering
\includegraphics{graphics/lagrange-2.pdf}
\caption{Bewegung im Phasenraum f"ur den harmonischen Oszillator.
Die Bewegung ({\color{green} gr"un}) verl"auft auf den Kurven konstanter
Energie $H=\operatorname{const}$, sie ist der Fluss des blau eingezeichneten
Hamiltonschen Vektorfeldes.
\label{skript:hamilton-harmonisch}}
\end{figure}
Ein harmonischer Oszillator
mit Masse $m$ und Federkonstante $K$ hat die potentielle Energie $\frac12Kx^2$,
die Hamilton-Funktion ist daher
\[
H=\frac1{2m}p^2+\frac12Kx^2.
\]
Die Bewegungsgleichungen nach dem Hamilton-Formalismus sind:
\begin{align*}
\frac{dx}{dt}&=\frac{\partial H}{\partial p}=\frac{p}{m}&&\Rightarrow&\dot x&=v\\
\frac{dp}{dt}&=\frac{\partial H}{\partial x}=-Kx&&\Rightarrow&ma&=-Kx
\end{align*}
Die erste Gleichung besagt, dass die Geschwindigkeit die Ableitung
der Ortskoordinate ist.
Die zweite Gleichung ist das erste Newtonsche Gesetz, denn die
r"ucktreibende Kraft einer um die Koordinate $x$ ausgelenkten Feder
mit Federkonstanten $K$ ist $-Kx$.
Insbesondere reproduziert der Hamilton-Formalismus die bekannten
Newtonschen Bewegungsgleichungen.

Die Bewegung ist auch dargestellt in Abbildung~\ref{skript:hamilton-harmonisch}.
Die Energie ist erhalten, der Punkt $(x,p)$ bewegt sich daher
auf einer Niveaulinie der Hamilton-Funktion.
Der Tangentialvektor der Bewegung ist das Hamiltonsche Vektorfeld.
\end{beispiel}

Ein besonderer Fall liegt vor, wenn die Hamilton-Funktion nicht von
der Zeit abh"angt. Dann ist
\[
\frac{dH}{dt}
=
\frac{\partial H}{\partial q}\frac{dq}{dt}
+
\frac{\partial H}{\partial p}\frac{dp}{dt}
+
\frac{\partial H}{\partial t}
=
-\frac{dp}{dt}\frac{dq}{dt}
+
\frac{dq}{dt}\frac{dp}{dt}
+
\frac{\partial H}{\partial t}
=
\frac{\partial H}{\partial t}=0,
\]
die Energie ist erhalten. Diese Situation tritt in abgeschlossenen
Systemen immer auf, insbesondere waren alle bisherigen Beispiel
von dieser Art.

\subsection{Poisson-Klammern}
\index{Poisson-Klammer}
Die Hamiltonsche Mechanik kann mit Hilfe der Poisson-Klammer besonders
elegant formuliert werden.

\begin{definition}
Seien $F(q,p)$ und $G(q,p)$ Funktionen in den Koordinaten und Impulsen. 
Dann ist die {\em Poisson-Klammer} die Funktion
\[
(F,G)
=
\sum_{k=1}^n
\biggl(
\frac{\partial F}{\partial q_k}\frac{\partial G}{\partial p_k}
-
\frac{\partial G}{\partial q_k}\frac{\partial F}{\partial p_k}
\biggr).
\]
\end{definition}

F"ur das Rechnen mit Poisson-Klammern sind die folgenden Rechenregeln
n"utzlich.
\begin{satz}
F"ur drei Funktionen $F$, $G$ und $H$ gilt
\begin{gather}
(F,G)=-(G,F),
\label{skript:poisson-antisymmetrie}
\\
(F,GH)
=
(F,G)H+G(F,H),
\label{skript:poisson-derivation}
\\
(F,(G,H))
+
(G,(H,F))
+
(H,(F,G))
=0.
\label{skript:poisson-jacobi}
\end{gather}
Die dritte Gleichung heisst auch die Jacobi-Identit"at.
\end{satz}
\index{Jacobi-Identit\"at!f\"ur Poission-Klammern}

\begin{proof}[Beweis]
Die erste Identit"at folgt sofort aus der Definition.
Die zweite Gleichung kann man durch Nachrechnen wie folgt erhalten:
\begin{align*}
(F,GH)
&=
\sum_{k=1}^3 \biggl(
\frac{\partial F}{\partial q_k}\frac{\partial GH}{\partial p_k}
-
\frac{\partial GH}{\partial q_k}\frac{\partial F}{\partial p_k}
\biggr)
\\
&=
\sum_{k=1}^3 \biggl(
\frac{\partial F}{\partial q_k}\frac{\partial G}{\partial p_k}H
+
G\frac{\partial F}{\partial q_k}\frac{\partial H}{\partial p_k}
-
\frac{\partial G}{\partial q_k}\frac{\partial F}{\partial p_k}H
-
G\frac{\partial H}{\partial q_k}\frac{\partial F}{\partial p_k}
\biggr)
=(F,G)H + G(F,H).
\end{align*}
Auch von der dritten Gleichung kann man sich durch Nachrechnen
"uberzeugen.
\end{proof}

Als Beispiel berechnen wir die Poisson-Klammern einer Funktion $G$
mit den Koordinaten und Impulsen
\begin{align*}
(q_j,G)
&=
\sum_{k=1}^n\biggl(
\frac{\partial q_j}{\partial q_k}\frac{\partial G}{\partial p_k}
-
\frac{\partial G}{\partial q_k}\frac{\partial q_j}{\partial p_k}
\biggr)
=
\sum_{k=1}^n\delta_{jk}\frac{\partial G}{\partial p_k}
=
\frac{\partial G}{\partial p_j}
\\
(p_j,G)
&=
\sum_{k=1}^n\biggl(
\frac{\partial p_j}{\partial q_k}\frac{\partial G}{\partial p_k}
-
\frac{\partial G}{\partial q_k}\frac{\partial p_j}{\partial p_k}
\biggr)
=
-\sum_{k=1}^n
\frac{\partial G}{\partial q_k}\delta_{jk}
=
-\frac{\partial G}{\partial q_j}.
\end{align*}
Damit k"onnen wir auch die Poisson-Klammer der Koordinaten und Impulse
ausrechnen:
\begin{align*}
(q_j,q_l)
&=
\frac{\partial q_l}{\partial p_j}
=
0
\\
(p_j,p_l)
&=
\frac{\partial p_l}{\partial q_j}
=
0
\\
(q_j,p_l)
&=
\frac{\partial p_l}{\partial p_j}
=\delta_{jl}.
\end{align*}
Die Poisson-Klammern der Koordinaten und Impulse mit der Hamilton-Funktion
erlauben, die Bewegungsgleichungen
\begin{align*}
(q_j,H)
&=
\frac{\partial H}{\partial p_j} = \frac{dq_j}{dt}
\\
(p_j,H)
&=
-\frac{\partial H}{\partial q_j} = \frac{dp_j}{dt}
\end{align*}
mit Poisson-Klammern auszudr"ucken.

\begin{satz}
\label{skript:zeitentwicklung-observable-klassisch}
Ist $A$ eine Funktion der Koordinaten und Impulse eines mechanischen Systems
mit Hamilton-Funktion $H$, dann ist die Zeitentwicklung von $A$ gegeben
durch die Poisson-Klammer:
\begin{equation}
\frac{d}{dt}A
=
(A,H).
\label{skript:zeitentwicklung-poisson}
\end{equation}
\end{satz}

\begin{proof}[Beweis]
Wir berechnen die Zeitableitung von $A$
\begin{align*}
\frac{dA}{dt}
&=
\frac{\partial A}{\partial q_k}\frac{dq_k}{dt}
+
\frac{\partial A}{\partial p_k}\frac{dp_k}{dt}
=
\frac{\partial A}{\partial q_k}\frac{\partial H}{\partial p_k}
-
\frac{\partial A}{\partial p_k}\frac{\partial H}{\partial q_k}
=
(A,H).
\end{align*}
\end{proof}

\subsection{Poisson-Klammern f"ur den Drehimpuls}
Der Drehimpuls ist der Vektor mit den Komponenten
\[
\vec L=\vec r\times \vec p
\qquad\Rightarrow\qquad
\left\{\quad
\begin{aligned}
L_1&=x_2p_3-x_3p_2,\\
L_2&=x_3p_1-x_1p_3,\quad\text{und}\\
L_3&=x_1p_2-x_2p_1.
\end{aligned}
\right.
\]
Die Poisson-Klammern von $L_1$  mit den Koordinaten und Impulsen sind
\begin{align*}
(x_1,L_1)
&=
0,
&
(p_1,L_1)
&=
0,
\\
(x_2,L_1)
&=
\frac{\partial L_1}{\partial p_2}
=
-x_3,
&
(p_2,L_1)
&=
-\frac{\partial L_1}{\partial x_2}
=
-p_3,
\\
(x_3,L_1)
&=
\frac{\partial L_1}{\partial p_3}
=
x_2,
&
(p_3,L_1)
&=
-\frac{\partial L_1}{\partial x_3}
=
p_2.
\end{align*}
Mit der Regel (\ref{skript:poisson-derivation}) k"onnen wir jetzt auch die
Poisson-Klammern der Drehimpulskomponenten untereinander berechnen:
\begin{align*}
(L_1,L_1)&=0
\\
(L_2,L_1)
&=
(x_3p_1-x_1p_3,L_1)
=
(x_3p_1,L_1)-(x_1p_3,L_1)
\\
&=
x_3\underbrace{(p_1,L_1)}_{=0}+\underbrace{(x_3,L_1)}_{=x_2}p_1
-x_1\underbrace{(p_3,L_1)}_{=p_2} -\underbrace{(x_1,L_1)}_{=0}p_3
=
x_2p_1 -x_1p_2=-L_3
\\
(L_3,L_1)
&=
(x_1p_2-x_2p_1,L_1)
=
(x_1p_2,L_1)-(x_2p_1,L_1)
\\
&=
x_1\underbrace{(p_2,L_1)}_{=-p_3} + \underbrace{(x_1,L_1)}_{=0}p_2
- x_2\underbrace{(p_1,L_1)}_{=0} - \underbrace{(x_2,L_1)}_{=-x_3}p_1
=
-x_1p_3+x_3p_1
=L_2
\\
(L_3,L_2)&=-L_1\qquad\text{(durch zyklische Vertauschung)}
\end{align*}
Damit k"onnen wir auch die Poisson-Klammern von $L_3$ mit dem Drehimpulsbetrag
$\vec L^2$ berechnen:
\begin{align*}
(L_1^2,L_3)
&=
L_1(L_1,L_3)+(L_1,L_3)L_1
=
-2L_2L_1
\\
(L_2^2,L_3)
&=
L_2(L_2,L_3)+(L_2,L_3)L_2
=
2L_2L_1
\\
(L_3^2,L_3)&=L_3(L_3,L_3)+(L_3,L_3)L_3=0
\\
\Rightarrow\qquad
(\vec L^2,L_3)
&=0
\end{align*}
Wir werden in Kapitel~\ref{chapter:drehimpuls} sehen,
dass die Drehimpulskomponenten in der
Quantenmechanik einer ganz "ahnlichen Algebra gen"ugen.
Die Algebra wird bereits ausreichen, die m"oglichen Zust"ande
des Drehimuplses zu berechnen.


\section*{"Ubungsaufgaben}
\rhead{"Ubungsaufgaben}
\begin{uebungsaufgaben}
\item
Betrachten Sie die Lagrange-Funktion eines eindimensionalen Teilchens
\[
\frac12m\dot q^2 + A(t,q)\dot q.
\]
\begin{teilaufgaben}
\item
Leiten Sie die zugeh"origen Bewegungsgleichungen ab.
\item
Stellen Sie die Hamilton-Funktion auf.
\item
Leiten Sie die Hamiltonschen Bewegungsgleichungen ab.
\end{teilaufgaben}


\begin{loesung}
\begin{teilaufgaben}
\item
Die Bewegungsgleichungen k"onnen aus den Euler-Gleichungen abgeleitet
werden:
\begin{align}
\frac{\partial L}{\partial q}&=\dot q \frac{\partial A}{\partial q}
\notag
\\
\frac{\partial L}{\partial \dot q}&=m\dot q+ A(t,q)
\label{05001:impuls}
\\
\frac{d}{dt}\frac{\partial L}{\partial\dot q}
&=
m\ddot q+\frac{\partial A}{\partial t}+\frac{\partial A}{\partial q}\dot q
\label{05001:derivative}
\\
\frac{d}{dt}\frac{\partial L}{\partial\dot q}-\frac{\partial L}{\partial q}
&=
m\ddot q+\frac{\partial A}{\partial t}
=0
\notag
\end{align}
Aus der letzten Gleichung lesen wir die Bewegungsgleichung in Newtonscher
Form ab:
\begin{align*}
m\ddot q&=-\frac{\partial A}{\partial t}.
\end{align*}
\item
Aus (\ref{05001:impuls}) lesen wir ab, dass wir als verallgemeinerte
Impuls-Koordinate f"ur die Hamilton-Gleichungen die Gr"osse
$P=m\dot q+A=p+A$ w"ahlen m"ussen.
Die Hamilton-Funktion muss jetzt die Energie in $q$ und dieser neuen
Koordinaten $P$ ausdr"ucken, man muss also jedes vorkommen von $p$
im Ausdruck f"ur die Energie durch $P-A$ ersetzen.
Die Hamilton-Funktion wird damit zu
\[
H(P,q)=\frac1{2m}(P-A(t,q))^2.
\]
\item
Die Hamiltonschen Bewegungsgleichungen sind
\begin{align*}
\frac{dq}{dt}
&=
\frac{\partial H}{\partial P}
=
\frac1m(P-A(t,q))
\\
\frac{dP}{dt}
&=
-\frac{\partial H}{\partial q}
=
+\frac1m(P-A(t,q))\frac{\partial A}{\partial q}
=
\dot q\frac{\partial A}{\partial q}
\end{align*}
In der zweiten Gleichung verschwindet das Vorzeichen wegen des zus"atzlichen
Vorzeichens von der inneren Ableitung.
Setzen wir in der zweiten Gleichung wieder $P=m\dot q+A$ ein, erhalten wir
f"ur die linke Seite
\begin{align*}
\frac{dP}{dt}
&=
\frac{d}{dt}(m\dot q+A(t,q))
=
m\ddot q+\frac{\partial A}{\partial t} + \frac{\partial A}{\partial q}\dot q
\end{align*}
F"ur die rechte Seite erhalten wir
\[
-\frac1m(P-A(t,q))\frac{\partial A}{\partial q}
=
-\dot q\frac{\partial A}{\partial q}.
\]
Zusammen erhalten wir die Bewegungsgleichungen
\begin{align*}
m\ddot q + \frac{\partial A}{\partial t}+\frac{\partial A}{\partial q}\dot q
&=
\frac{\partial A}{\partial q}\dot q
\\
m\ddot q
&=
-\frac{\partial A}{\partial t}.
\end{align*}
\end{teilaufgaben}
\end{loesung}

\begin{diskussion}
Wenn $A$ nicht von der Zeit abh"angt, ist die Energie erhalten.
Also leistet der Term $A$ keine Arbeit.
Die Wirkung von $A$ hat also "ahnliche Auswirkungen wie ein Magnetfeld
auf eine geladenes Teilchen,
denn die Lorentz steht immer senkrecht auf der Bewegungsrichtung und
leistet daher ebenfalls keine Arbeit.
Tats"achlich w"urden in einer dreidimensionalen Version dieses Problems
im Ausdruck (\ref{05001:derivative}) zus"atzliche Terme auftreten,
die in der Bewegungsgleichung die Lorentzkraft ergeben w"urden.
Dies wird im Abschnit~\ref{section:hamilton-funktion-im-magnetfeld}
voref"uhrt.
\end{diskussion}
 

\item
Rechnen Sie die Jacobi-Identit"at f"ur die Poisson-Klammern nach:
\begin{equation}
(F,(G,H))+ (G,(H,F))+ (H,(F,G))=0.
\label{skript:jacobi}
\end{equation}

\begin{loesung}
Wir setzen die Definition f"ur die erste Poisson-Klammer in
(\ref{skript:jacobi}) ein:
\begin{align*}
(F,(G,H))
&=
\sum_{k=1}^n\biggl(
\frac{\partial F}{\partial q_k} \frac{\partial (G,H)}{\partial p_k}
-
\frac{\partial (G,H)}{\partial q_k} \frac{\partial F}{\partial p_k}
\biggr)
\\
&=
\sum_{k,l=1}^n\biggl(
\frac{\partial F}{\partial q_k}
\frac{\partial}{\partial p_k}\biggl(
\frac{\partial G}{\partial q_l}\frac{\partial H}{\partial p_l}
-
\frac{\partial H}{\partial q_l}\frac{\partial G}{\partial p_l}
\biggr)
-
\frac{\partial}{\partial q_k}\biggl(
\frac{\partial G}{\partial q_l}\frac{\partial H}{\partial p_l}
-
\frac{\partial H}{\partial q_l}\frac{\partial G}{\partial p_l}
\biggr)
\frac{\partial F}{\partial p_k}
\biggr)
\\
&=
\sum_{k,l=1}^n\biggl(
\frac{\partial F}{\partial q_k}
\biggl(
\frac{\partial^2 G}{\partial q_l\partial p_k}\frac{\partial H}{\partial p_l}
+
\frac{\partial G}{\partial q_l}\frac{\partial^2 H}{\partial p_l\partial p_k}
-
\frac{\partial^2 H}{\partial q_l\partial p_k}\frac{\partial G}{\partial p_l}
-
\frac{\partial H}{\partial q_l}\frac{\partial^2 G}{\partial p_l\partial p_k}
\biggr)
\\
&\qquad\qquad
-
\biggl(
\frac{\partial^2 G}{\partial q_l\partial q_k}\frac{\partial H}{\partial p_l}
+
\frac{\partial G}{\partial q_l}\frac{\partial^2 H}{\partial p_l\partial q_k}
-
\frac{\partial^2 H}{\partial q_l\partial q_k}\frac{\partial G}{\partial p_l}
-
\frac{\partial H}{\partial q_l}\frac{\partial^2 G}{\partial p_l\partial q_k}
\biggr)
\frac{\partial F}{\partial p_k}
\biggr)
\end{align*}
Die volle Jacobi-Identit"at kann aus diesem Ausdruck durch zyklische
Vertauschung von $F$, $G$ und $H$ und Addition gewonnen werden.
Dabei entsteht eine grosse Zahl von Termen.
Um die "Ubersicht zu behalten, sammeln wir nur die zweiten Ableitungen
von $G$,
und unterscheiden ausserdem nach den verschiedenen Variablen:
\begin{align*}
\frac{\partial^2 G}{\partial q_k\partial q_l}:&
&&
\sum_{k,l=1}^n
\biggl(
-\frac{\partial F}{\partial p_k} \frac{\partial H}{\partial p_l}
+\frac{\partial F}{\partial p_l} \frac{\partial H}{\partial p_k}
\biggr)
\frac{\partial^2 G}{\partial q_k\partial q_l}
\\
\frac{\partial^2 G}{\partial p_k\partial q_l}:&
&&
\sum_{k,l=1}^n
\biggl(
\frac{\partial F}{\partial q_k}\frac{\partial H}{\partial p_l}
+
\frac{\partial F}{\partial p_l}\frac{\partial H}{\partial q_k}
-
\frac{\partial F}{\partial q_k}\frac{\partial H}{\partial p_l}
-
\frac{\partial F}{\partial p_l}\frac{\partial H}{\partial q_k}
\biggr)
\frac{\partial ^2G}{\partial p_k\partial q_l}
\\
\frac{\partial^2 G}{\partial p_k\partial p_l}:&
&&
\sum_{k,l=1}^n
\biggl(
-\frac{\partial F}{\partial q_k}\frac{\partial H}{\partial q_l}
+
\frac{\partial F}{\partial q_l}\frac{\partial H}{\partial q_k}
\biggr)
\frac{\partial^2 G}{\partial p_k\partial p_l}
\end{align*}
Jeder dieser Terme verschwindet, also verschwindet auch der gesamte
Ausdruck, was die Jacobi-Identit"at beweist.
\end{loesung}


\item
In dieser Aufgabe finden Sie die Bewegungsgleichungen eines Pendels.
Ein Pendel ist eine Masse $m$, die an einem Faden der L"ange $m$ 
aufgeh"angt ist, 

\begin{teilaufgaben}
\item Finden Sie die Lagrange-Funktion $L(\varphi)$.
\item Was muss als zu $\varphi$ geh"origer Impuls $p_\varphi$ verwendet werden?
\item Finden Sie die Hamilton-Funktion $H(\varphi, p_\varphi)$.
\item Stellen Sie die Bewegungsgleichungen auf.
\end{teilaufgaben}

\begin{loesung}
\begin{teilaufgaben}
\item Die Lagrange-Funktion ist die Differenz von kinetischer und
potentieller Energie:
\[
T
=
\frac12mv^2-mgh
=
\frac12m(l\dot\varphi)^2-mlg(1-\cos\varphi).
\]
\item
Als konjugierter Impuls muss die Ableitung von $L$ nach $\dot\varphi$
verwendet werden:
\[
p_\varphi
=
\frac{\partial L}{\partial\dot\varphi}
=
ml^2\dot\varphi.
\]
Man beachte, dass $ml^2$ das Tr"agheitsmoment der Masse $m$ um den
Aufh"angepunkt des Pendels ist, $p_\varphi$ ist also der Drehimpuls.
Wir dr"ucken $\dot\varphi$ noch durch den Drehimpuls aus:
\[
\dot\varphi
=
\frac{1}{ml^2}p_{\varphi}.
\]
\item
Die Hamilton-Funktion kann mit Hilfe der
Formel~(\ref{skript:von-lagrange-zu-hamilton})
bestimmt werden:
\begin{align*}
H(\varphi,p_\varphi)
&=
p_\varphi\dot\varphi-L(\varphi,\dot\varphi)
=
p_\varphi\dot\varphi - \frac12 m(l\dot\varphi)^2+mlg(1-\cos\varphi)
\\
&=
p_\varphi\frac{1}{ml^2}p_\varphi-\frac12m\frac{1}{m^2l^2}p_\varphi^2
+mlg(1-\cos\varphi)
\\
&=
\frac1{2ml^2}p_\varphi^2 + mlg(1-\cos\varphi).
\end{align*}
\item
Die Bewegungsgleichungen in Hamiltonscher Form sind
\begin{align*}
\frac{d\varphi}{dt}
&=
\frac{\partial H}{\partial p_\varphi}
=
\frac{1}{ml^2}p_\varphi,
\\
\frac{dp_\varphi}{dt}
&=
-\frac{\partial H}{\partial\varphi}
=
-mlg\sin\varphi.
\end{align*}
Die erste Gleichung ist nat"urlich identisch mit dem in b) gefundenen
Zusammenhang zwischen $p_\varphi$ und $\dot\varphi$.
Setzen wir diesen Zusammenhang in die zweite Gleichung ein, erhalten
wir
\begin{align*}
\frac{d}{dt}p_\varphi
&=
\frac{d}{dt}(ml^2\dot\varphi)
=
ml^2\ddot\varphi
=
-mlg\sin\varphi
\\
\Leftrightarrow
\qquad
\ddot\varphi
&=
-\frac{g}{l}\sin\varphi.
\end{align*}
Dies ist die bekannte Bewegungsgleichung f"ur ein Pendel. 
F"ur kleine Amplituden kann man die rechte Seite approximieren durch
$\sin\varphi\simeq\varphi$, und erh"alt eine Schwingungsdifferentialgleichung
mit Kreisfrequenz $\sqrt{g/l}$.
\end{teilaufgaben}
\end{loesung}


\end{uebungsaufgaben}

