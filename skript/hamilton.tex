\chapter{Hamiltonsche Mechanik}
\lhead{Hamilton Mechanik}
\rhead{}
\section{Motivation}
W"ahrend Galileo Galileis Beschreibung von Bewegungen im wesentlichen
eine kinematische war, haben Isaac
Newtons Gesetze erstmals einen Zusammenhang zwischen
der Bewegung und den Kr"aften hergestellt, also von Ursache und
Wirkung. Sein erstes Gesetz besagt zum Beispiel
\[
F=ma\qquad\text{oder}\qquad
F(x)=m\frac{d^2x}{dt^2}.
\]
Mit geeigneten Anfangsbedingungen l"asst sich daraus jede Bewegung 
vorhersagen.

Diese Formulierung ist jedoch nicht allgemein genug, insbesondere f"ur die
Zwecke der Quantenmechanik. Wenn Teilchen auch Wellen sind, wie soll man
sich dann die Kraftwirkung einer Welle auf eine andere Welle vorstellen?

Die Antwort ist die Hamiltonsche Formulierung der Mechanik, die von der
Energie als der alles bestimmenden Gr"osse ausgeht.
Mit Hilfe eines Standardformalismus lassen sich daraus die Newtonschen
Bewegungsgleichungen wieder gewinnen.
Viel wichtiger ist aber, dass sich daraus auch ein Methode gewinnen
l"asst, besonders geeignete Koordinaten zur Beschreibung eines
physikalischen Systems zu finden.
Diese Hamilton-Jacobi-Theorie genannte Methode ist die Basis eines
allgemein anwendbaren Quantisierungs-Verfahrens.

\section{Hamilton-Funktion}
Die Hamilton-Funktion $H(p,q)$ ist die Energie eines physikalischen
Systems. Darin sind $q$ die allgemeinen Koordinaten, also
zum Beispiel die Raumkoordinaten $x$, $y$ und $z$, oder Winkel, mit
denen man die r"aumliche Ausrichtung eines Systems beschreibt.
Die Variablen $p$ sind die zugeh"origen Impulse, f"ur Ortskoordinaten
sind dies die gew"ohnlichen Impulskomponenten, f"ur die Drehwinkel ist
es der Drehimpuls.

Die Energie $H$ setzt sich zusammen aus der kinetischen Energie $T$ und der
potentiellen Energie $V$. F"ur ein kr"aftefreies Teilchen der Masse $m$
ist die Energie 
\[
H=\frac12mv^2=\frac1{2m}p^2.
\]
Befindet sich das Teilchen dagegen in einem Potential $V(q)$, dann 
ist die zugeh"orige Hamilton-Funktion
\begin{equation}
H=T+V=\frac1{2m}p^2+V(q).
\label{hamilton-potential}
\end{equation}
Ein Elektron im elektrischen Feld eines Protons, welches im Nullpunkt
des $q$-Koordinatensystems sitzt, hat daher die Hamilton-Funktion
\[
H=\frac1{2m}p^2+\frac{e^2}{4\pi\varepsilon_0|q|}.
\]

Eine noch zu "uberwindende Schwierigkeit dieses Formalismus wird hier
bereits erkennbar: Kr"afte, die keine Arbeit leisten, k"onnen auch
keinen Beitrag zur Energie leisten.
Ein Beispiel ist
die Lorentzkraft eines Magnetfelds, welche die Bahn eines Elektrons 
kr"ummt, aber immer senkrecht auf der Bahn steht und daher keine Arbeit
leistet.

\section{Bewegungsgleichungen}
Die Newtonschen Bewegungsgleichungen lassen sich aus der Hamilton-Funktion
gewinnen. Um dies zu verstehen, berechnen wir die partiellen Ableitungen
von $H$ nach Koordinaten $q$ und Impulsen $p$:
\begin{align*}
\frac{\partial H}{\partial p}&=\frac{1}{m}p=v=\frac{dq}{dt} \\
\frac{\partial H}{\partial q}&=\frac{\partial V}{\partial q}
\end{align*}
Die erste Gleichung ist nichts anders als die Aussage, dass die
Geschwindigkeit die Zeitableitung der Ortskoordinaten ist, sie
stellt den Zusammenhang zwischen $p$ und $q$ her.
Die Ableitung des Potentials auf der rechten Seite der zweiten
Gleichung hat die Bedeutung einer Kraft.
Die Newtonschen Bewegungsgleichungen sagen, dass die "Anderung des
Impulses durch die Kr"afte verursacht wird, also durch die Ableitungen
des Potentials:
\[
\frac{dp}{dt}=-\frac{\partial V}{\partial q}
\]
Damit sind die Bewegungsgleichungen jetzt:
\begin{align}
\frac{dq}{dt}&= \frac{\partial H}{\partial p},\label{hamilton-v}\\
\frac{dp}{dt}&=-\frac{\partial H}{\partial q}.\label{hamilton-newton}
\end{align}

\begin{beispiel} Harmonischer Oszillator. Ein harmonischer Oszillator
mit Masse $m$ und Federkonstante $K$ hat die potentielle Energie $\frac12Kx^2$,
die Hamilton-Funktion ist daher
\[
H=\frac1{2m}p^2+\frac12Kx^2.
\]
Die Bewegungsgleichungen nach dem Hamilton-Formalismus sind:
\begin{align*}
\frac{dx}{dt}&=\frac{\partial H}{\partial p}=\frac{p}{m}&&\Rightarrow&\dot x&=v\\
\frac{dp}{dt}&=\frac{\partial H}{\partial x}=-Kx&&\Rightarrow&ma&=-Kx
\end{align*}
Die erste Gleichung besagt, dass die Geschwindigkeit die Ableitung
der Ortskoordinate ist.
Die zweite Gleichung ist das erste Newtonsche Gesetz, denn die
r"ucktreibende Kraft einer um die Koordinate $x$ ausgelenkten Feder
mit Federkonstanten $K$ ist $-Kx$.
Insbesondere reproduziert der Hamilton-Formalismus die bekannten
Newtonschen Bewegungsgleichungen.
\end{beispiel}

Ein besonderer Fall liegt vor, wenn die Hamilton-Funktion nicht von
der Zeit abh"angt. Dann ist
\[
\frac{\partial H}{\partial t}=0,
\]
die Energie ist erhalten. Diese Situation tritt in abgeschlossenen
Systemen immer auf, insbesondere waren alle bisherigen Beispiel
von dieser Art.

\section{Kanonische Transformation}
Kartesische Koordinaten sind nicht unbedingt die optimale Wahl.
Eine zweckm"assige Wahl der Koordinaten nimmt R"ucksicht auf die
Symmetrieeigenschaften des Systems.
Ein zweidimensional rotationssymmetrischer harmonischer Oszillator mit 
der Hamilton-Funktion
\[
H=\frac1{2m}(p_x^2+p_y^2)+K(x^2 + y^2)
\]
sollte sich auch in Polarkoordinaten $(r,\varphi)$ ausdr"ucken lassen. 
Doch welche Impulse sind dann zu verwenden, damit die Gleichungen
(\ref{hamilton-v}) und (\ref{hamilton-newton}) auch in den neuen
Koordinaten G"ultigkeit haben?

\section{Hamilton-Jacobi-Theorie}
Die Technik der kanonischen Transformationen erlaubt sogar, ein mechanisches
Problem mit Hilfe einer geeigneten Transformation vollst"andig zu l"osen.
Wie dies funktionieren k"onnte zeigt ein Vergleich mit einem freien
Teilchen, also der Hamilton-Funktion
\[
H=\frac1{2m}(p_x^2+p_y^2+p_z^2).
\]
Da die Koordinaten $x$, $y$ und $z$ gar nicht vorkommen, verschwinden
die Ableitungen von $H$ nach den Koordinaten und die Bewegungsgleichungen
sind
\begin{align*}
\frac{dp}{dt}&=0\\
\frac{dq}{dt}&=\frac{p}{m}
&&\Rightarrow&q=q_0+t\frac{p}{m}
\end{align*}
Die Ortskoordinaten h"angen linear von der Zeit ab.

\section{Wellen und Teilchen}
