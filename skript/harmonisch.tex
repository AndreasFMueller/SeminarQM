\chapter{Harmonischer Oszillator\label{chapter:harmonischeroszillator}}
\lhead{Harmonischer Oszillator}
\rhead{}
Das klassische Federpendel ist eines der einfachsten mechanischen 
Systeme. Es ist nicht nur von theoretischer Bedeutung, denn sehr
viele schwingende Systeme k"onnen mit dem gleichen Modell
oder als eine Zusammensetzung solcher einfacher Oszillatoren
beschrieben werden.

In der Quantenmechanik ist die Bedeutung der quantisierten Version
dieses einfachen Systems nicht geringer. Viele Systeme k"onnen
n"aherungsweise durch einen harmonischen Oszillator beschrieben
werden, zum Beispiel die Schwingungen von einfachen Molek"ulen.

Der quantenmechanische harmonische Oszillator hat aber eine Eigenschaft,
die der klassische Oszillator nicht hat: seine Energie ist nicht
kontinuierlich, nur eine diskrete Menge von Energieniveaus ist
m"oglich. Dies ist zwar nicht "uberraschend nach den bisher studierten
Beispielen, doch werden wir hier die mathematische Technik  der
Auf- und Absteige-Operatoren kennen lernen, mit der die Energieniveaus
und auch die Zustandsvektoren leicht berechnet werden k"onnen.

In der Anwendung auf die Schwingungen einfacher Molek"ule
bedeutet das diskrete Spektrum, dass solche Molek"ule Energie nur in
fest definierten Paketen aufnehmen k"onnen.
Jedes Molek"ul hat also eine charakteristische Menge von Wellenl"angen
von typischerweise infrarotem Licht, welche es absorbieren kann.
Tats"achlich stellt die Infrarotsprektroskopie eine wichtige
Technik in der Chemie dar, mit der Molek"ule identifiziert werden k"onnen.
Wenn ein Molek"ul nur bestimmte Energiepakete aufnehmen kann, dann
wird die W"armekapazit"at ebenfalls temperaturabh"angig sein, und
mit charakteristischen Spr"ungen ansteigen, welche der Anregung
von h"oherenergetischen Molek"ulschwingungen entsprechen.

\section{Der klassische Oszillator}
\rhead{Klasischer Oszillator}
Ein Federpendel ist eine Masse $m$, welche sich unter dem Einfluss
einer Feder mit Federkonstante $K$ bewegt. Ist $x$ die Auslenkung
der Masse aus der Ruhelage, dann "ubt die Feder eine Kraft $-Kx$ 
auf die Masse aus. Die potentielle Energie bei Auslenkung $x$ ist
\[
V(x)=\int_0^xK\xi\,d\xi=\frac12Kx^2.
\]
Die Hamilton-Funktion ist daher
\[
H(p,x)=\frac1{2m}p^2+\frac12Kx^2.
\]
Zur Kontrolle leiten wir mit Hilfe der Hamiltonschen Gleichungen
die Bewegungsgleichungen ab:
\begin{align*}
\frac{\partial H}{\partial x}&=Kx
&
&\Rightarrow&
\dot p&=-\frac{\partial H}{\partial x}=-Kx
\\
\frac{\partial H}{\partial p}&=\frac{p}{m}
&
&\Rightarrow&
\dot x&=\frac{\partial H}{\partial p}=\frac{p}{m}.
\end{align*}
Wir erhalten also die klassischen  Bewegungsgleichungen eines harmonischen
Oszillators. 

Die L"osungen dieser Bewegungsgleichungen sind Schwingungen mit 
Frequenz
\[
\omega = \sqrt{\frac{K}{m}},
\]
statt der Konstanten $K$ k"onnten wir auch $K=m\omega^2$ verwenden,
die Hamilton-Funktion ist dann
\begin{equation}
H=\frac1{2m}p^2+\frac12m\omega^2x^2.
\end{equation}

\section{Quantisierung}
\rhead{Quantisierung}
Die Quantisierungsregeln verlangen wieder, dass die Impuls-Variable
durch
\[
p\rightarrow\frac{\hbar}{i}\frac{\partial}{\partial x}
\]
ersetzt wird. Der Hamilton-Operator ist also
\begin{equation}
\hat H
=
-\frac{\hbar^2}{2m}
\frac{\partial^2}{\partial x^2}
+\frac12m\omega^2x^2.
\label{skript:harmoszhamilton}
\end{equation}

Wir suchen jetzt Eigenfunktionen des Hamilton-Operators (\ref{skript:harmoszhamilton}),
also Funktionen $\psi(x)$ und Konstanten $E$ so, dass
\[
\hat H\psi=E\psi.
\]
Da in diesem Problem nur eine einzige Variable auftritt, ist dies ein
Problem "uber gew"ohnliche Differentialgleichungen:
\begin{equation}
-\frac{\hbar^2}{2m} \psi''(x)+\frac12m\omega^2x^2\psi(x)=E\psi(x).
\label{skript:harmoszgleichung}
\end{equation}
In dieser Form ist das Problem etwas umst"andlich zu l"osen.
Wir verwenden statt $x$ die Variablen 
\[
q=\sqrt{\frac{m\omega}{\hbar}}x.
\]
Nennen wir die Funktion $u(q)=\psi(x)$, dann gilt
\begin{align*}
\frac{d}{dq}u(q)&=\frac{d\psi(x)}{dx}\frac{dx}{dq}
=\psi'(x)\sqrt{\frac{\hbar}{m\omega}}
&
\psi'(x)&=\sqrt{\frac{m\omega}{\hbar}}u'(q)
\\
\frac{d^2}{dq^2}u(q)&=\psi''(x)\frac{\hbar}{m\omega}
&
\psi''(x)&=\frac{m\omega}{\hbar}u''(q).
\end{align*}
Setzen wir dies in die Differentialgleichung (\ref{skript:harmoszgleichung})
ein, erhalten wir
\begin{equation}
-\frac{\hbar\omega}{2} u''(q)
+\frac12m\omega^2x^2\psi(x)=E\psi(x).
\end{equation}
Teilen wir dies durch $-\hbar\omega/2$, erhalten wir
\begin{align*}
u''(q) -\frac{m\omega}{\hbar}x^2u(q)=-\frac{2E}{\hbar\omega}u(q),
\end{align*}
oder mit den Abk"urzungen
\[
q=\sqrt{\frac{m\omega}{\hbar}}x
\quad\text{und}\quad
\varepsilon=\frac{E}{\hbar \omega}
\]
die endg"ultige Form der Differentialgleichung
\begin{equation}
u''(q)+(2\varepsilon-q^2) u(q)=0.
\label{skript:harmq}
\end{equation}

\section{Wellenfunktionen}
\rhead{L"osungen}
In diesem Abschnitt l"osen wir die Differentialgleichung (\ref{skript:harmq}).
\subsection{Grundzustand\label{skript:hogrundzustand}}
Da das Potential f"ur grosse Werte von $q$ beliebig gross wird,
muss die Wellenfunktion $u(q)$ f"ur grosse Werte von $q$ exponentiell
schnell abfallen.
Wir versuchen daher einen Ansatz in der Form $u_0(q)=e^{-\alpha q^2}$.
Die Ableitungen von
$u_0(x)$ sind
\begin{align*}
u_0'(q)&=-2\alpha qe^{-\alpha q^2}\\
u_0''(q)&=-2\alpha e^{-\alpha q^2}+4\alpha^2q^2e^{-\alpha q^2}.
\end{align*}
Einsetzen in (\ref{skript:harmoszgleichung})  liefert
\begin{align*}
(-2\alpha e^{-\alpha q^2}+4\alpha^2q^2e^{-\alpha q^2})
+
(2\varepsilon - q^2)e^{-\alpha q^2}&=0
\\
\Leftrightarrow\qquad
\biggl(
-2\alpha +4\alpha^2q^2
+
2\varepsilon - q^2
\biggr)e^{-\alpha q^2}
&=0
\end{align*}
Die letzte Gleichung kann nur erf"ullt werden, wenn der grosse Klammerausdruck
verschwindet, wenn also gilt
\begin{align*}
-2\alpha +4\alpha^2q^2
+
2\varepsilon - q^2
&=0
\\
\Leftrightarrow\qquad
(2\varepsilon-2\alpha)
+
(4\alpha^2-1)q^2
&=0
\end{align*}
Diese Gleichung muss f"ur alle $q$ gelten, die Klammerausdr"ucke
m"ussen also beide verschwinden:
\begin{align*}
(4\alpha^2-1)q^2&=0
&
&\Rightarrow&
\alpha&=\frac12
\\
\varepsilon-\alpha&=0
&
&\Rightarrow&
\varepsilon&=\frac12.
\end{align*}
Es gibt also tats"achlich eine L"osung, sie muss von der Form
\[
u_0(q)=e^{-\frac{q^2}2}
\]
sein. Der zugeh"orige Eigenwert ist $\varepsilon_0=\frac12$.

Substituieren wir wieder die
urspr"unglichen Koordinaten, erhalten wir
\begin{equation}
\psi_0(x)=u_0\biggl(\sqrt{\frac{m\omega}{\hbar}}x^2\biggr)
=
e^{-\frac{m\omega}{2\hbar}x^2}.
\label{skript:grundzustandwellenfunktion}
\end{equation}
als Wellenfunktion f"ur den Grundzustand.


% XXX Normierung dieser L"osung
\subsection{Angeregte Zust"ande}
Wir versuchen weitere L"osungen in der Form $u_n(q)=h_n(q)e^{-\frac{q^2}2}$
zu finden, wobei $h_n(q)$ ein Polynom $n$-ten Grades ist.
Wir brauchen wieder die Ableitungen:
\begin{align*}
u'(q)
&=
\bigl(h_n'(q)- qh_n(q)\bigr)e^{-\frac{q^2}2}
\\
u''(q)
&=
\bigl(
h_n''(q)- qh_n'(q)
- h_n(q)
- q h_n'(q)
+q^2h_n(q)
\bigr)e^{-\frac{q^2}2}
\\
&=
\bigl(
h_n''(x)-2 qh_n'(q)
+(q^2-1)h_n(q)
\bigr)e^{-\frac{q^2}2}
\end{align*}
Eingesetzt in die Differentialgleichung erhalten wir
\begin{align}
\bigl(
h_n''(x)-2 qh_n'(q)
+(q^2-1)h_n(q)
\bigr)e^{-\frac{q^2}2}
+(\varepsilon-q^2)h_n(q)e^{-\frac{q^2}2}&=0
\notag
\\
\Leftrightarrow\qquad
h_n''(x)-2 qh_n'(q)
+(\varepsilon - 1)h_n(q)&=0
\label{skript:hermiteequation}
\end{align}
\begin{figure}
\centering
\includegraphics[width=\hsize]{graphics/harmonisch-1.pdf}
\caption{Wellenfunktionen des harmonischen Oszillators, dargestellt auf
den zugeh"origen Energieniveaus $\hbar \omega (n+\frac12)$ f"ur
$n=0,\dots 7$.
\label{skript:harmonisch-wellenfunktionen}}
\end{figure}%
Die Gleichung (\ref{skript:hermiteequation}) heisst auch die 
Hermitesche Differentialgleichung.
Die Funktionen $h_n(q)$ heissen die Hermiteschen Polynome.
Man kann zeigen, dass $\varepsilon=n+\frac12$ sein muss.
Wir werden die Polynome hier nicht ausrechnen, da die Technik der
Auf- und Absteigeoperatoren eine einfachere Methode hierf"ur liefern
wird.
Aus dieser Technik werden sich die Energieniveaus ebenfalls ergeben.
In Abbildung~\ref{skript:harmonisch-wellenfunktionen} sind die Wellenfunktionen
des harmonischen Oszillators f"ur die Energieniveaus bis $n=7$ dargestellt.
Man beachte, dass wie beim Potentialtopf in
Abschnitt~\ref{subsection:potentialtopf} das Teilchen sich mit positiver
Wahrscheinlichkeit in einem Bereich aufhalten kann, der mit der Energie
des Teilchens klassisch verboten ist.
In diesem Bereich f"allt die Aufenthaltswahrscheinlichkeit jedoch
"ahnlich wie beim Potentialtopf exponentiell schnell ab.

\section{Algebraische Eigenschaften}
\rhead{Algebra}
Um die Berechnung der Energieniveaus des harmonischen Oszillators
vorzubereiten, wechseln wir zun"achst von den kanoninschen Koordinaten
$x$ und $p$ zu dimensionslosen Gr"ossen $Q$ und $P$, und berechnen
deren algebraischen Eigenschaften.
\subsection{Dimensionslose Operatoren}
Dazu gehen wir zur"uck zum Hamilton-Operator
\[
\hat H=-\frac{\hbar^2}{2m}\frac{\partial^2}{\partial x^2}
+\frac12m\omega^2 x^2
\]
und schreiben ihn unter Verwendung der Operatoren
\[
P=\frac{1}{\sqrt{\hbar m\omega}}p 
\qquad
\text{und}
\qquad
Q=\sqrt{\frac{m\omega}{\hbar}}x
\]
Da $\hbar\omega$ die Dimension einer Energie hat, hat $\hbar m\omega$ die
Dimension des Quadrates eines Impulses. Weil $m\omega^2x^2$ die Dimension
einer Energie hat, ist $m\omega x^2/\hbar$ dimensionslos, also auch
$Q$.
Man kann nachrechnen, dass 
\[
\frac12(P^2+Q^2)
=
\frac{1}{2\hbar m \omega}p^2
+
\frac{m\omega}{2\hbar}x^2
=
-\frac{\hbar}{2 m \omega}\frac{\partial^2}{\partial x^2}
+
\frac{m\omega}{2\hbar}x^2
=\frac{\hat H}{\hbar\omega}
=:\hat{\cal H}.
\]
Wir haben den Hamiltonoperator also umgeschrieben mit neuen Operatoren 
$P$ und $Q$.

\subsection{Vertauschungsrelationen}
Wir sollten uns davon "uberzeugen, dass die Operatoren $P$ und $Q$ 
mit den urspr"unglichen $p$ und $x$ vergleichbar sind, und berechnen
daher die Vertauschungsrelationen:
\begin{align*}
[P,Q]
&=
\frac{1}{\sqrt{\hbar m\omega}} \sqrt{\frac{m\omega}{\hbar}}[p, x]
=
\frac1{\hbar}[p,x].
\end{align*}
Wir erinnern an die Vertauschungsrelationen f"ur die Operatoren $p$ und $x$:
\begin{align*}
\left[\frac{\hbar}{i}\frac{\partial}{\partial x}, x\right]\psi(x)
&=
\frac{\hbar}{i}\frac{\partial}{\partial x}(x\psi(x))
-x\frac{\hbar}{i}\frac{\partial}{\partial x}\psi(x)
=
\frac{\hbar}{i}\psi(x)
x\frac{\hbar}{i}\frac{\partial \psi(x)}{\partial x}
-x\frac{\hbar}{i}\frac{\partial}{\partial x}\psi(x)
=\frac{\hbar}{i}\psi(x)
\\
[p,x]&=\frac{\hbar}{i}\operatorname{id}.
\end{align*}
Damit sind jetzt auch die Vertauschungsoperatoren f"ur $P$ und $Q$ 
bekannt:
\[
[P,Q]=\frac1{i}\operatorname{id}=-i\operatorname{id}.
\]

\section{Auf- und Absteigeoperatoren}
\rhead{Auf- und Absteigeoperatoren}
Die Methode der Auf- und Absteige-Operatoren erlaubt, die Eigenwerte
und Eigenvektoren direkt aus dem Grundzustand zu ermitteln.
Wir sehen sie hier in einem einfachen Spezialfall am Werk.
Sie l"asst sich aber verallgemeinern, man kann damit zum Beispiel
auch den Drehimpuls verstehen (Kapitel~\ref{chapter:drehimpuls}).
Ja die Methode hat sogar g"anzlich von der Quantenmechanik unabh"angige
Anwendungen bei der Konstruktion von L"osungen allgemeinerer Randwertprobleme
f"ur gew"ohnliche Differentialgleichung.

\subsection{Definition}
Wir definieren die Operatoren
\[
a=\frac1{\sqrt{2}}(Q+iP)
\qquad\text{und}\qquad
a^+=\frac1{\sqrt{2}}(Q-iP).
\]
Zun"achst ist klar, dass $a$ und $a^+$ nicht selbstadjungiert sind,
vielmehr ist
\[
a^*=\frac1{\sqrt{2}}(Q^*-iP^*)=\frac1{\sqrt{2}}(Q-iP)=a^+.
\]
Damit entsprechen die Operatoren $a$ und $a^+$ nicht einer physikalisch
messbaren Gr"osse. 

Unser Ziel ist die Berechnung der Eigenwerte und Eigenvektoren
des harmonischen Oszillators mit algebraischen Mitteln. Dazu brauchen
wir die algebraischen Eigenschaften der Operatoren $a$ und $a^+$,
insbesondere deren Produkte und Vertauschungsrelationen.
\begin{align*}
aa^+
&=
\frac1{\sqrt{2}}(Q+iP)\frac1{\sqrt{2}}(Q-iP)
=
\frac12(Q^2+P^2+i[P,Q])
=
\hat{\cal H}+\frac12
\\
a^+a
&=
\frac1{\sqrt{2}}(Q-iP)\frac1{\sqrt{2}}(Q+iP)
=
\frac12(Q^2+P^2+i[Q,P])
=
\hat{\cal H}-\frac12=:N
\end{align*}
Daraus folgt auch
\[
[a,a^+]=\operatorname{id},
\]
und die weiteren Vertauschungsrelationen
\begin{align*}
[N,a^+]
&=a^+aa^+-aa^+a^+=[a,a^+]a^+=a^+
&
[N,a]
&=a^+aa-aa^+a=[a^+,a]a=-a
\\
&=[\hat{\cal H},a^+]
&
&=[\hat{\cal H}, a].
\end{align*}

\subsection{Wirkung auf Zustandsvektoren}
Nehmen wir an, $|n\rangle$ sei der $n$-te Eigenvektor von $\hat{\cal H}$,
mit Energie $\varepsilon_n$, also
\[
\hat{\cal H}|n\rangle = \varepsilon_n|n\rangle.
\]
Wenden wir die Operatoren $a$ und $a^+$ auf $|n\rangle$ an, erhalten
wir einen neuen Eigenvektor:
\begin{align*}
\hat{\cal H}a^+\,|n\rangle
&=
([\hat{\cal H}, a^+] + a^+\hat{\cal H})\,|n\rangle
=
(1 + \varepsilon_n)a^+\,|n\rangle
\\
\hat{\cal H}a\,|n\rangle
&=
([\hat{\cal H}, a] + a\hat{\cal H})\,|n\rangle
=
(\varepsilon_n - 1)a\,|n\rangle
\end{align*}
Insbesondere ist $a^+\,|n\rangle$ ein Eigenvektor mit Energie
$E_n+1$, und $a|n\rangle$ ein Eigenvektor mit Energie $\varepsilon_n-1$.
Man kann also mit dem Operator $a^+$ zu Zust"anden h"oherer Energie
aufsteigen, und mit $a$ zu Zust"anden niedrigerer Energie absteigen.

Beim Auf- und Absteigen bleibt die Normierung aber nicht notwendigerweise
erhalten, wir k"onnen also nicht davon ausgehen, dass $a^+|n\rangle$
wieder ein normierter Eigenvektor ist.
Um die Norm zu korrigieren, berechen wir die Norm des neuen Vektors:
\[
(a^+\langle n|) a^+\,|n\rangle
=
\langle n|\,aa^+\,|n\rangle
=
\langle n|\,\hat{\cal H}+{\textstyle\frac12}\,|n\rangle
=
(\varepsilon_n+{\textstyle\frac12})\langle n|n\rangle
=
\varepsilon_n+\textstyle\frac12
\]
Wenn man also durch Aufsteigen wieder einen normierten Zustandsvektor
erhalten will, muss man den erhaltenen Vektor renormieren:
\begin{equation}
|n+1\rangle = \frac1{\sqrt{\varepsilon_n+\textstyle\frac12}}|n\rangle.
\label{skript:aufsteigrenormierung}
\end{equation}

\subsection{Eigenwerte und Zustandsvektoren}
Die Energie eines harmonischen Operators ist immer positiv,
also kann man mit $a$ nicht beliebig lange absteigen. Es muss einen
Zustand $|0\rangle$ kleinster Energie geben. M"ochte man weiter
absteigen, darf kein Zustandsvektor mehr entstehen, es muss also
$a|0\rangle=0$ sein. Durch Multiplikation mit $a^+$ folgt
\[
0=a^+a|0\rangle=(\hat{\cal H}-\textstyle\frac12)\,|0\rangle
\quad
\Rightarrow
\quad
\hat{\cal H}\,|0\rangle=\frac12\,|0\rangle
\quad
\Rightarrow
\quad
\varepsilon_0=\frac12.
\]
Der Grundzustand hat also immer den Eigenwert $\frac12$, oder die
Energie $\frac12\hbar\omega$. Diesen Wert haben wir auch schon in
Abschnitt~\ref{skript:hogrundzustand} erhalten.
Die angeregten Zust"ande haben Energie
\[
E_n
=
\hbar\omega\biggl(n+\frac12\biggr),
\]
wobei $n\ge 0$ eine nat"urliche Zahl sein muss.

Beim Aufsteigen ver"andert sich jeweils die Normierung.
Will man die Energieeigenzust"ande so konstruieren, muss man nach jedem
Aufsteigeschritt die Normierung mit Hilfe von (\ref{skript:aufsteigrenormierung})
korrigieren. Dabei kann man verwenden, dass $\varepsilon_n = n+\frac12$
ist. Nach $n$-maligem Aufsteigen aus dem Grundzustand hat man die
Norm um den Faktor
\begin{align}
N_n
&=
\frac12\cdot\biggl(1+\frac12\biggr)\cdot\biggl(2+\frac12\biggr)\cdot\ldots\cdot
\biggl(n+\frac12\biggr)
=
\frac12\cdot\frac32\cdot\frac52\cdot\ldots\cdot\frac{2n+1}2
\notag
\\
&=
\frac1{2^n}
\frac{1\cdot 2\cdot 3\cdot 4\cdot 5\cdot \ldots \cdot (2n+1)\cdot 2n}%
{2\cdot 4\cdot 6\cdot \ldots \cdot 2n}
=\frac1{2^{2n}}\frac{2n!}{n!}
\label{skript:aufsteigrenormierungn}
\end{align}
ver"andert.

Der Aufsteigeoperator $a^+$ erm"oglicht jetzt auch, aus dem Grundzustand
alle h"oheren Zust"ande zu konstruieren:
\begin{equation}
|n\rangle
=
\frac{1}{\sqrt{N_n}}(a^+)^n\,|0\rangle
=
2^n \sqrt{\frac{n!}{2n!}} (a^+)^n\,|0\rangle,
\end{equation}
wobei $N_n$ der Normierungsfaktor aus (\ref{skript:aufsteigrenormierungn}) ist.

Da wir den Grundzustand bereits aus Abschnitt~\ref{skript:hogrundzustand} kennen,
k"onnen wir die Wellenfunktionen f"ur die h"oheren Zust"ande berechnen,
indem wir den Operator $a^+$ wieder in den urspr"unglichen Koordinaten
ausdr"ucken:
\begin{equation}
a^+=\frac1{\sqrt{2}}(Q+iP)
=
\frac1{\sqrt{2}}\biggl(
\sqrt{\frac{m\omega}{\hbar}}x
+i
\frac{1}{\sqrt{\hbar m\omega}}\frac{\hbar}{i}\frac{\partial}{\partial x}
\biggr)
=
\frac1{\sqrt{2}}
\biggl(
\sqrt{\frac{m\omega}{\hbar}}x
+
\sqrt{\frac{\hbar}{m\omega}}\frac{\partial}{\partial x}
\biggr)
\label{skript:aufsteigeho}
\end{equation}
Die Wellenfunktionen des Zustands $|n\rangle$ erh"alt man also, indem
man den Operator $a^+$ in der Form \ref{skript:aufsteigeho} auf die Wellenfunktion
$\psi_0$ wie in (\ref{skript:grundzustandwellenfunktion}) anwendet.

Wir fassen die Resultate "uber den harmonischen Oszillator in einem Satz
zusammen:
\begin{satz}
Ein harmonischer Oszillator mit Hamilton-Operator
\[
\frac{1}{2m}p^2+\frac{m}{2}\omega^2x^2
=
-\frac{\hbar^2}{2m}\frac{\partial^2}{\partial x^2}
+\frac{m}2\omega^2x^2
\]
hat Eigenzust"ande mit Energie
\begin{equation}
E_n
=
\hbar\omega\biggl(n+\frac12\biggr),
\label{skript:hoenergieniveaus}
\end{equation}
die Eigenzust"ande k"onnen gefunden werden durch Anwendung des
Aufsteigeoperators $a^+$
aus (\ref{skript:aufsteigeho}) auf die Wellenfunktion des Grundzustandes
$\psi(x)=e^{-\frac{m\omega}{2\hbar}x^2}$
mit geeigneter Normierung.
\end{satz}

\section{Klassischer Grenzwert}
\begin{figure}
\centering
\includegraphics{graphics/harm-1.pdf}
\caption{Wellenfunktion $|\psi_n(x)|^2$ des quantenmechanischen 
harmonischen Oszillators (rot) und Aufenthaltswahrscheinlichkeit des
klassischen harmonischen Oszillators (blau) im klassisch erlaubten
Gebiet.
\label{skript:harmklass}}
\end{figure}
F"ur grosse Energie m"ussen die Wellenfunktionen den klassischen
harmonischen Oszillator approximieren.
Ein harmonischer Oszillator mit Bahnkurve $x(t)=\cos t$ hat in einem
Interval $\Delta x$ eine Verweildaur von ungef"ahr
\[
\frac{\Delta x}{|\dot x(t)|}=\frac{\Delta x}{|\sin t|}
=\frac{\Delta x}{\sqrt{1-\cos^2t}}.
\]
Als Funktion von $x$ ausgedr"uckt ist die Wahrscheinlichkeitsdichte,
den Oszillator im Punkt $x$ zu finden, proportional zu
\[
\frac{1}{\sqrt{1-x^2}}.
\]
In Abbildung~\ref{skript:harmklass} ist die die Aufenthaltswahrscheinlichkeit
im klassisch erlaubten Gebiet eines quantenmechanischen Oszillators f"ur
das Energieniveau $n=100$ dargestellt zusammen mit der durch die
Wellenfunktion gegebenen Wahrscheinlichkeitsdichte $|\psi_n(x)|^2$.

% XXX Anwendung auf Molek"ulschwingungen
%\section{Anwendung auf Molek"ulschwingungen}

\section*{"Ubungsaufgaben}
\rhead{"Ubungsaufgaben}
\begin{uebungsaufgaben}
\item
\input uebungsaufgaben/08001.tex
\item
\input uebungsaufgaben/08002.tex
\end{uebungsaufgaben}

