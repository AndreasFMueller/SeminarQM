\chapter{Heisenbergsche Unsch"arferelation\label{chapter:heisenberg}}
\lhead{Heisenbergsche Unsch"arferelation}
\rhead{}

In der Quantenmechanik lassen sich die Observablen nicht mehr einfach
vertauschen, wie das in der klassischen Mechanik m"oglich war.
F"ur Operatoren gilt im Allgemeinen kein Kommutativgesetz.
In diesem Abschnitt wollen wir die Konsequenzen dieser Tatsache
untersuchen.
Wir k"onnen aber bereits jetzt feststellen, dass die Gleichung
$AB=BA$ f"ur zwei quantenmechanische Observable ``fast'' stimmen
muss, in dem Sinne, dass der Unterschied der beiden Seiten sehr
klein sein muss. 

\section{Kommutator und Antikommutator}
Zu zwei Operatoren $A$ und $B$ kann man Kommutator und Antikommutator
bilden:
\begin{align*}
&\text{Kommutator:}&
[A,B]&=AB-BA
\\
&\text{Antikommutator:}&
\{A,B\}&=AB+BA
\end{align*}
Wenn $[A,B]=AB-BA=0$ folgt $AB=BA$,
der Kommutator gibt also an, ob die beiden Operatoren vertauschen (kommutieren).
Wenn $\{A,B\}=AB+BA=0$ folgt $AB=-BA$,
der Kommutator gibt also an, ob die beiden Operatoren antikommutieren.

Falls $A$ and $B$ Observable sind, die nicht vertauschen, dann k"onnen
gemeinsame Eigenvektoren nicht beliebig sein.
Nehmen wir n"amlich an, $|\psi\rangle$ sei ein gemeinsamer Eigenvektor
von $A$ mit Eigenwert $\alpha$ und $B$ mit Eigenwert $\beta$.
Dann wirkt der Kommutator darauf wie folgt
\[
[A,B]|\psi\rangle
=
(AB-BA)|\psi\rangle 
=
A\beta|\psi\rangle -B\alpha|\psi\rangle
=
\alpha\beta|\psi\rangle-\beta\alpha|\psi\rangle
=
0.
\]
Ein gemeinsamer Eigenvektor $|\psi\rangle$ wird also vom $[A,B]$
zu $0$ gemacht.

F"ur die Observablen $X$ f"ur Ort und $P$ f"ur den Impuls eines Teilchens,
k"onnen wir den Kommutator explizt ausrechnen:
\begin{align*}
[X,P]\psi(x)
&=
\left[
x,\frac{\hbar}{i}\frac{\partial}{\partial x}
\right]\psi(x)
=
\psi(x)\frac{\hbar}{i}\frac{\partial\psi(x)}{\partial x}
-
\frac{\hbar}{i}\frac{\partial}{\partial x}\biggl(x\psi(x)\biggr)
\\
&=
\psi(x)\frac{\hbar}{i}\frac{\partial\psi(x)}{\partial x}
-
\frac{\hbar}{i}\psi(x)
-
\frac{\hbar}{i}x\frac{\partial\psi(x)}{\partial x}
=
\frac{\hbar}{i}\psi(x).
\end{align*}
Der Kommutator ist also
\[
[X,P]=\frac{\hbar}{i}\operatorname{id},
\]
ein Vielfaches der identischen Abbildung.
Die identische Abbildung hat nat"urlich keine Vektoren, die von ihr
zu $0$ gemacht werden.
Orts- und Impuls-Operatoren k"onnen also keine gemeinsamen Eigenvektoren
haben.
Es gibt also keinen Zustand, in dem sowohl Ort als auch Impuls exakt
bestimmt sind.

\begin{satz}
Wenn zwei Observable einen Kommutator haben, der ein Vielfaches
der identischen Abbildung ist, dann haben die Observable keine
gemeinsamen Eigenvektoren.
Insbesondere k"onnen die beiden Observablen nicht gleichzeitig 
exakt bestimmt sein.
\end{satz}

In der klassischen Mechanik haben wir keine Schwierigkeit, Ort und
Impuls gleichzeitig festzustellen. Das liegt nat"urlich daran, dass
der Kommutator den sehr kleinen Faktor $\hbar$ enth"alt.
F"ur makroskopische Zwecke sind Ort und Impuls also gleichzeitig 
feststellbar.

Wenn es also nicht m"oglich ist, Ort und Impuls eines Teilchens
gleichzeitig exakt zu wissen, wie genau ist es dann m"oglich,
Ort und Impuls zu wissen?
Um diese Frage zu beantworten m"ussen wir zun"achst ein Mass f"ur
die Genauigkeit finden, mit der Ort oder Impuls in einem bestimmten
Zustand bekannt sein kann.
Wir werden es die Unsch"arfe nennen.
Wir erwarten dann eine Ungleichung, die uns sagt, wie grosse 
Unsch"arfe sein muss, eine Unsch"arferelation.


\section{Varianz, Kovarianz und Unsch"arfe}
Sie $A$ eine Observable und $|\psi\rangle$ ein Zustand. Dann ist
$\langle \psi|A|\psi\rangle$ der Erwartungswert der Observablen $A$ 
im Zustand $|\psi\rangle$.
Zur Abk"urzung schreiben wir $\langle A\rangle=\langle\psi|A|\psi\rangle$.
Mit $A$ ist auch $A-\langle A\rangle$ eine Observable, und ihr Erwartungswert
ist nat"urlich $0$.

In der Wahrscheinlichkeitsrechnung lernt man mit der Varianz eine
Gr"osse kennen, die angibt, wie nahe bei $\langle A\rangle$ die
Werte der Observable im Mittel anzutreffen sind.
Die Varianz ist die mittlere quadratische Abweichung, also der
Erwartungswert der Gr"osse $(A-\langle A\rangle)^2$, die auch wieder
eine Observable ist. Sie kann also mit unserem Formalismus berechnet
werden:
\begin{align*}
\operatorname{var}(A)
&=
\langle \psi|\, (A-\langle A\rangle)^2\,|\psi\rangle
=
\langle\psi|\, A^2-2\langle A\rangle A+\langle A\rangle^2\,|\psi\rangle
=
\langle\psi|A^2|\psi\rangle 
-2\langle A\rangle \langle\psi|A|\psi\rangle
+\langle A\rangle^2\langle\psi|\psi\rangle
\\
&=\langle A^2\rangle -\langle A\rangle^2.
\end{align*}
Dies entspricht der aus der Wahrscheinlichkeitsrechung bekannten Formel
$E(X^2)-E(X)^2$.
Wir nennen die Standardabweichung, also
\[
\sqrt{\operatorname{var}(A)}
=
\sqrt{\langle (A-\langle A\rangle)^2\rangle}
=
\sqrt{\langle A^2\rangle - \langle A\rangle^2}
\]
auch die {\em Unsch"arfe} der Messung von $A$, und schreiben daf"ur abgek"urzt
auch $\Delta A$.

Die Varianz ist ein Spezialfall der Kovarianz, die als 
\[
\operatorname{cov}(X,Y)
=
E((X-E(X))(Y-E(Y)))
\]
definiert war, $\operatorname{var}(X)=\operatorname{cov}(X,X)$.
Analog k"onnen wir jetzt auch eine Kovarianz f"ur Observable als
\[
\langle A,B\rangle
=
\langle\psi|
(A-\langle A\rangle)(B-\langle B\rangle)
|\psi\rangle
\]
definieren.
Man beachte, dass die Gr"osse $\langle A,B\rangle$ nicht symmetrisch zu
sein braucht, ausser wenn die Operatoren $A$ und $B$ vertauscht werden
k"onnen.

Man kann $\langle A,B\rangle$ auch als den Wert der Transformationsfunktion 
zweier Zust"ande ansehen:
\begin{equation}
\left.
\begin{aligned}
|a\rangle &= (A-\langle A\rangle)|\psi\rangle\\
|b\rangle &= (B-\langle B\rangle)|\psi\rangle
\end{aligned}
\right\}
\quad
\Rightarrow
\quad
\langle A,B\rangle = \langle a|b\rangle.
\end{equation}
Nat"urlich gilt die Cauchy-Schwarz-Ungleichung f"ur die beiden
Zust"ande $|a\rangle$ und $|b\rangle$, also
\begin{equation}
|\langle a|b\rangle|^2
\le
\langle a|a\rangle \langle b|b\rangle
=
\operatorname{var}(A)\operatorname{var}(B)
=\Delta A^2 \cdot \Delta B^2
\label{cauchy-schwarz-uncertainty}
\end{equation}
Falls also die linke Seite nicht $0$ ist, dann k"onnen wir die beiden
Gr"ossen $A$ und $B$ nicht beliebig genau kennen.
Die Ungenauigkeit, mit der $A$ und $B$ bekannt sein k"onnen, sind "uber
die Ungleichung (\ref{cauchy-schwarz-uncertainty}) miteinander verkn"upft.
Wenn die Unsch"arfe verkleinert wird, mit der $A$ bekannt ist, dann 
vergr"ossert sich die Unsch"arfe, mit der $B$ bekannt ist.

\section{Unsch"arferelation}
Die Ungleichung (\ref{cauchy-schwarz-uncertainty}) liefert eine
Unsch"arferelation, falls die linke Seite $>0$ ist. Es ist
also zu untersuchen, wie gross die linke Seite von 
(\ref{cauchy-schwarz-uncertainty}) tats"achlich ist.

Setzen wir $z=\langle a|b\rangle$, dann ist $|z|^2$ die linke Seite
von (\ref{cauchy-schwarz-uncertainty}). Den Betrag kann man mit Hilfe
von Real- und Imagin"arteil ausrechnen:
\[
|z|^2
=
(\operatorname{Re}z)^2+(\operatorname{Im}z)^2
=
\biggl(\frac{z+\bar z}2\biggr)^2 + \biggl(\frac{z-\bar z}{2i}\biggr)^2
\]
Jetzt setzen wir $z=\langle a|b\rangle$ ein:
\begin{equation}
|\langle a|b\rangle|^2
=
\biggl(\frac{\langle a|b\rangle + \langle a|b\rangle}2\biggr)^2
+
\biggl(\frac{\langle a|b\rangle - \langle a|b\rangle}{2i}\biggr)^2
\label{unschaerfe2}
\end{equation}
Wir m"ussen also die Terme $\langle a|b\rangle + \langle a|b\rangle$
und $\langle a|b\rangle - \langle a|b\rangle$ ausrechnen:
\begin{align*}
\frac{\langle a|b\rangle + \langle a|b\rangle}2
&=
\frac{
\langle\psi|AB+BA|\psi\rangle 
}2
-\langle A\rangle\langle B\rangle
=
\frac12 \langle\,\{A,B\}\,\rangle - \langle A\rangle\langle B\rangle,
\\
\frac{\langle a|b\rangle - \langle a|b\rangle}{2i}
&=
\frac{\langle\psi|AB-BA|\psi\rangle}{2i}
=
\frac1{2i}\langle [A,B]\rangle
\end{align*}
Eingesetzt in (\ref{unschaerfe2}) finden wir
\[
|\langle a|b\rangle|^2
=
\biggl(
\frac12\langle \{A,B\}\rangle - \langle A\rangle\langle B\rangle
\biggr)^2
+
\biggl(
\frac{[A,B]}{2i}
\biggr)^2.
\]
Eingesetzt in die urspr"ungliche Unsch"arferelation
(\ref{cauchy-schwarz-uncertainty}) erhalten wir jetzt die Unsch"arferelation
von Robertson und Schr"odinger:

\begin{satz}
Sind $A$ und $B$ selbstadjungierte Operatoren und $|\psi\rangle$ ein
Zustand, dann gilt die Unsch"arferelation
\begin{equation}
\Delta A^2\cdot\Delta B^2\ge 
\biggl(
\frac12\langle \{A,B\}\rangle - \langle A\rangle\langle B\rangle
\biggr)^2
+
\biggl(
\frac{[A,B]}{2i}
\biggr)^2.
\label{uncertainty}
\end{equation}
\end{satz}

F"ur Ort und Impuls haben wir den Kommutator schon ausgerechnet, es
muss also gelten
\begin{equation}
\Delta X^2\cdot \Delta P^2
\ge
\biggl(
\frac12\langle \{X,P\}\rangle - \langle X\rangle\langle P\rangle
\biggr)^2
+
\biggl(
\frac{[X,P]}{2i}
\biggr)^2
\ge
\biggl(
\frac{[X,P]}{2i}
\biggr)^2.
\ge \frac{\hbar^2}4
\end{equation}
oder
\begin{equation}
\Delta X\cdot\Delta P\ge \frac{\hbar}2.
\end{equation}
Dies ist die Heisenbergsche Unsch"arferelation.
