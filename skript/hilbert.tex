\chapter{Hilbertr"aume\label{chapter:hilbertraeume}}
\lhead{Hilbertr"aume}
\rhead{}
Wir wissen schon, dass Zust"ande Eigenvektoren des Hamilton-Operators
sein sollen, wir brauchen
also einen Vektorraum, der ausreichend gross ist, alle Zust"ande aufzunehmen.
Ein typisches Quantensystem hat eine sehr grosse Zahl von Zust"anden,
vielleicht sogar unendlich viele.
Wir brauchen daher einen unendlichdimensionalen komplexen Vektorraum als
B"uhne f"ur die Quantenmechanik.
Der Begriff des Hilbertraumes liefert die gesuchte Struktur.

Ausgehend von der gewohnten Vorstellung eines endlichdimensionalen
Vektorraumes kann in vier Schritten eine ausreichend reichhaltige
Struktur aufgebaut werden:
\begin{enumerate}
\item Erweiterung der Definition eines reellen Vektorraumes dahingehend,
dass auch komplexe Skalare zugelassen werden.
\item Erweiterung des Begriffs des Skalarprodukts auf den komplexen Fall.
Dazu geh"oren auch der Begriff der L"ange eines Vektors und Ungleichungen
wie die Dreiecksungleichung, die unsere Intuition f"ur das Verhalten von
Abst"anden auf den unendlichdimensionalen Fall erweitern.
\item Der Begriff des Grenzwertes einer Vektorfolge erlaubt, Zust"ande
zu approximieren.
\item Die Hilbertbasis liefert eine Technik, wie die Zust"ande eines
Quantensystems als Basisvektoren f"ur den Zustandsraum verwendet werden
k"onnen.
\end{enumerate}

\section{Komplexe Vektorr"aume}
Ein Vektorraum "uber $\mathbb C$ ist einem Menge $V$ von Vektoren, mit
einer Addition von Vektoren und einer Multiplikation von Vektoren mit
komplexen Zahlen, mit folgenden Eigenschaften
\begin{compactenum}
\item Es gibt ein Element $0\in V$ mit der Eigenschaft $0+v=v,\forall v\in V$
und $0v=0$.
\item Zu jedem $v\in V$ gibt es ein Elemente $-v=(-1)v$ mit der Eigenschaft
$v+(-v)=0$
\item $u+v=v+u$
\item $(u+v)+w=u+(v+w)$
\item $(\lambda\mu)v=\lambda (\mu v)$
\item $\lambda(u+v)=\lambda u+ \lambda v$, $\lambda\in\mathbb C$, $u,v\in V$
\item $(\lambda+\mu)u=\lambda u+\mu u$
\end{compactenum}
Wie im Falle reeller Vektorr"aume kann man den Vektorraum
\[
{\mathbb C}^n = \left\{\left .
\begin{pmatrix}x_1\\\vdots\\x_n\end{pmatrix}
\right|
x_i\in\mathbb C
\right\}
\]
mit komponentenweisen Operationen konstruieren.
Ebenso kann man lineare Gleichungssysteme mit komplexen Koeffizienten,
komplexe Matrizen und das Matrizenprodukt f"ur komplexe Matrizenprodukt
konstruieren.

Der Gauss-Algorithmus kann unver"andert auch f"ur komplexe Gleichungssysteme
verwendet werden, so dass alle Eigenschaften, die man mit Hilfe des
Gauss-Algorithmus hergeleitet hat, auch f"ur komplexe Vektorr"aume
gelten.
Insbesondere ist der Rang definiert, die L"osungsmenge
eines Gleichungssystems ist eine Menge von komplexen Linearkombinationen
von Vektoren, und die (komplexe) Determinate ist genau dann 0, wenn
die Matrix singul"ar ist.

\section{Komplexes Skalarprodukt}
Das Skalarprodukt in reellen Vektorr"aumen ist eine Funktion von
zwei Vektoren, die linear ist in jedem Faktor:
\begin{align*}
(a+b,c)&=(a,c)+(b,c)&(a,b+c)&=(a,b)+(a,c)\\
(\lambda a,b)&=\lambda(a,b)&(a,\lambda b)&=\lambda(a,b)
\end{align*}
W"urde man diese Definition f"ur komplexe Zahlen verwenden, dann
m"usste das Skalarprodukt eines komplexen Vektors mit sich selbst
die Eigenschaft 
\[
(ia,ia)=i^2(a,a)=-(a,a)
\]
haben. Insbesondere k"onnte man $(a,a)$ nicht mehr als das Quadrat der
L"ange des Vektors $a$ interpretieren, denn die f"ur reelle Vektoren
geltende Formel
\[
(\lambda a,\lambda a)=|\lambda |\, (a,a)
\]
w"urde nicht mehr gelten.

Man verlangt daher, dass das Skalarprodukt eine Funktion von zwei Vektoren
ist, die linear ist im ersten Faktor, aber konjugiert linear im zweiten:
\begin{definition}
Ein Funktion $(a,b)$ von zwei Vektoren heisst Sesquilinearform, wenn
sie linear im zweiten Faktor und konjugiert linear im ersten Faktor ist.
\begin{align*}
(a,\lambda b)&=\lambda (a,b)&(\lambda a,b)=\bar\lambda (a,b).
\end{align*}
\end{definition}
Dies reicht aber nicht f"ur die Definition eines Skalarproduktes.
Das reelle Skalarprodukt hat die Eigenschaft, dass man die Faktoren
vertauschen kann. F"ur eine Sesquilinearform reicht das auch nicht:
\[
i(u,v)=(u,iv)=(iv,u)=-i(v,u)=-i(u,v),
\]
was nur richtig sein kann, wenn $(u,v)=0$. Man verlangt daher mehr:

\begin{definition}
Ein Sesquilinearform heist hermitesch, wenn gilt
\[
(u,v)=\overline{(v,u)}.
\]
\end{definition}
F"ur eine hermitesche Sesquilinearform ist sichergestellt, dass
das Skalarprodukt eines Vektors mit sich selbst eine reelle Zahl
ist. Man kann das einsehen, indem man im Produkt $(u,u)$ die beiden
Faktoren vertauscht:
\[
(u,u)=\overline{(u,u)}\quad\Rightarrow\quad (u,u)\in\mathbb R.
\]
Es ist aber immer noch nicht sichergestellt, dass man $(u,u)$ als
L"ange des Vektors interpretieren kann. 

\begin{definition}
Ein hermitesche Sesquilinearform heist {\em positiv definit}, wenn
$(u,u)>0$ f"ur $u\ne 0$. Ein positive definite hermitesche Sesquilinearform
heisst ein Skalarprodukt. Ein komplexer Vektorraum mit einem Skalarprodukt
heisst ein Pr"ahilbertraum.
\end{definition}


F"ur die Vektorr"aume $\mathbb C^n$ kann man dann auch eine Formel
f"ur das Skalarprodukt angeben. Sind $a_i$ die Komponenten von $a$ und
$b_i$ die Komponenten von $b$, dann ist das Skalarprodukt
\[
(a,b)=\sum_{i=1}^n \bar a_ib_i.
\]

Im Falle der reellen Vektorr"aume konnte man das Skalarprodukt
auch mit Hilfe des Matrizenproduktes schreiben: $u\cdot v=u^tv$.
Dies ist f"ur das Skalarprodukt nicht mehr m"oglich, denn $u^tv$
ist linear in beiden Vektoren.
Damit das Produkt konjugiert linear ist im ersten Faktor, m"ussen
wir $u$ nicht nur transponieren, sondern die Komponenten komplex konjugieren.
Wir efinieren daher:
\[
u=\begin{pmatrix}u_1\\\vdots\\u_n\end{pmatrix}
\quad\Rightarrow\quad
u^*=\begin{pmatrix}\bar u_1&\dots&\bar u_n\end{pmatrix}
\]
Das Skalarprodukt ist dann $(u,v)=u^*v$.

\section{Norm und Grenzwert}
Ein Skalarprodukt in einem Pr"ahilbertraum kann dazu benutzt werden,
die L"ange von Vektoren zu definieren und damit auch das Konzept eines
Grenzwertes einer Folge von Vektoren.

\begin{definition}
Die Norm eines Vektors in einem Pr"ahilbertraum ist
\[
\| v\| = \sqrt{(v,v)}.
\]
\end{definition}

\begin{satz}[Cauchy-Schwarz-Ungleichung] F"ur zwei Vektoren $u$ und $v$
in einem Pr"ahilbertraum gilt die Ungleichung
\[
|(u,v)| \le \| u\|\cdot \| v\|
\]
\end{satz}

\begin{proof}[Beweis]
F"ur $t\in\mathbb C$ berechnen wir das Produkt $(u-tv, u-tv)$
\begin{align*}
0&\le (u-tv,u-tv)\\
 &=   (u,u) - t(u,v) - \bar t(v,u) +   t\bar t(v,v) 
\end{align*}
Jetzt setzen wir $t=\frac{(v,u)}{(v,v)}$:
\begin{align*}
0&\le (u,u) - \frac{(v,u)(u,v)}{(v,v)} - \frac{(u,v)(v,u)}{(v,v)} + \frac{(u,v)(v,u)}{(v,v)^2}(v,v)\\
 &=(u,u) - 2\frac{|(u,v)|^2}{(v,v)} +\frac{|(u,v)|^2}{(v,v)}\\
 &=(u,u) -\frac{|(u,v)|^2}{(v,v)}\\
|(u,v)|^2&\le (u,u)\cdot (v,v) = \|u\|^2\cdot \|v\|^2.
\end{align*}
Damit ist die Cauchy-Schwarz-Ungleichung bewiesen.
\end{proof}

Mit dem Konzept der Norm k"onnen jetzt beliebige Grenzwerte 
definiert werden.
Eine Folge $(v_n)_{n\in\mathbb N}$ von Vektoren in $\cal H$ ist eine
{\em Cauchy-Folge}, wenn es f"ur jedes $\varepsilon>0$ eine Zahl
$N$ gibt, so dass
\[
\|v_k-v_l\| <\varepsilon\quad\forall k,l>N.
\]
Dies bedeutet, dass die Vektoren ``gegen Ende der Folge'' beliebig
nahe zusammenliegen. 
Im besten Fall gibt es dann einen Grenzwert der Folge, der bereits in
$\cal H$ ist, also einen Vektor $v\in\cal H$ mit der Eigenschaft,
dass f"ur jedes $\varepsilon > 0$ eine Zahl $N$ existiert, so dass
\[
\|v_k-v\| < \varepsilon\quad\forall k>N.
\]

\begin{definition}
Ein Pr"ahilbertraum heisst vollst"andig, wenn jede Cauchy-Folge einen
Grenzwert hat.
Ein vollst"andiger Pr"ahilbertraum heisst auch ein Hilbertraum.
\end{definition}

Das folgende Beispiel zeigt, dass die Existenz der Grenzwerte von
Cauchy-Folgen nicht garantiert ist, und wie ein Raum vervollst"andigt
werden kann.

\begin{beispiel}
Die rationalen Zahlen $\mathbb Q$ bilden einen Pr"ahilbertraum. 
Darin konstruieren wir die Folge der Fibonacci-Zahlen mit der 
Rekursionsformel
\[
f_{n+1}=f_n+f_{n-1},\qquad f_1=1, \qquad f_0=0.
\]
Daraus bilden wir dann die rationalen Zahlen
\[
v_n=\frac{f_{n+1}}{f_n}\in\mathbb Q.
\]
In der Linearen Algebra wurde gezeigt, dass diese Folge gegen
die Zahl $(\sqrt{5}+1)/2$ konvergiert, die aber selbst nicht in
$\mathbb Q$ liegt. Indem man zur Menge $\mathbb Q$ alle Grenzwerte
von Cauchy-Folgen hinzuf"ugt, erh"alt man die Menge $\mathbb R$, in der
alle Cauchy-Folgen einen Grenzwert haben.
\end{beispiel}

Wie im Beispiel der rationalen und reellen Zahlen,
kann man auch einen Pr"ahilbertraum durch hinzuf"ugen aller Grenzwerte
von Cauchy-Folgen zu einem Hilbertraum machen.


\section{Hilbertbasis}
Die endlichdimensionalen Hilbertr"aume $\mathbb C^n$ haben die
Standardbasisvektoren als Basis, jeder Vektor in $\mathbb C^n$
kann als komplexe Linearkombination der Standardbasisvektoren
$e_i,i=1,\dots,n$ dargestellt werden.

Gr"ossere Hilbertr"aume $\cal H$ m"ussen nicht unbedingt endlich dimensional
sein. Dies bedeutet, dass es eine nicht endende Folge von Vektoren
$v_i\in\cal H$ gibt, die alle orthogonal und von L"ange $1$ sind:
\[
(v_i,v_j)=\delta_{ij}.
\]
Ein Hilbertraum $\cal H$ heisst separabel, wenn es eine solche Folge
gibt, so dass sich jeder Vektor $v\in\cal H$ beliebig genau als
Linearkombination von Vektoren $v_i$ approximiert werden kann.
Die Vektoren $v_i$ heissen Hilbertbasis von $\cal H$.

Um die Darstellung von $v\in\cal H$ als Linearkombination von Vektoren
der Hilbertbasis zu finden, setzen wir die unbekannten Koeffizienten
als 
\[
v=\sum_{i\in\mathbb N}c_iv_i
\]
an. Um die Koeffizienten $c_i$ zu bestimmen, berechnen wir die
Skalarprodukte
\[
(v_i,v)
=\sum_{j\in\mathbb N} c_j(v_i,v_j)=\sum_{j\in\mathbb N}c_j\delta_{ij}=c_i.
\]
Insbesondere folgt
\begin{align*}
v&=\sum_{i\in\mathbb N}(v_i,v) v_i,\\
\| v\|^2&=\sum_{i\in\mathbb N} |(v_i,v)|^2.
\end{align*}
Die zweite Gleichung heisst auch die Parseval-Gleichung.

Es gibt Hilbertr"aume, die nicht separabel sind, doch f"ur unsere Zwecke
der Quantenmechanik reichen die separablen Hilbertr"aume aus.

%
% Hilbertraeume l^2 und L^2
%
\section{Hilbertr"aume $l^2$ und $L^2$}
In diesem Abschnitt betrachten wir zwei Hilbertr"aume, die als 
Modelle f"ur quantenmechanische Zustandsr"aume dienen werden.

\subsection{$l^2$ als Erweiterung von $\mathbb C^n$}
Der Hilbertraum $l^2$ ist eine Erweiterung des endlichdimensionalen
Raumes $\mathbb C^n$.
Als Vektorraum besteht $l^2$ aus Folgen von komplexen Zahlen,
so dass die Summe der Betragsquadrate beschr"ankt ist:
\[
l^2=\biggl\{
(c_i)_{i\in\mathbb N}\,\biggl|\,c_i\in\mathbb C,
\sum_{i\in\mathbb N} |c_i|^2 <\infty
\biggr.\biggr\}
\]
Addition von Folgen und Multiplikation mit komplexen Zahlen erfolgt 
komponentenweise.
Das Skalarprodukt zweier Folgen ist
\[
(a, b)=\sum_{i\in\mathbb N} \bar a_i b_i.
\]
Diese Summe existiert, denn die Cauchy-Schwarz-Ungleichung zeigt,
dass
\begin{align*}
|(a,b)|^2
=
\biggl|
\sum_{i\in\mathbb N} \bar a_ib_i
\biggr|^2
=
\biggl(
\sum_{i\in\mathbb N} |a_i|^2
\biggr)
\biggl(
\sum_{i\in\mathbb N} |b_i|^2
\biggr)
<\infty.
\end{align*}
Damit wird der Raum der Folgen zu einem Hilbertraum.

Die Folgen $e_i=(\delta_{ij})_{j\in\mathbb N}$ bilden eine Hilbertbasis
des Hilbertraumes $l^2$. 

Nat"urlich kann man den Raum $l^2$ auch f"ur andere Indexmengen als
$\mathbb N$ bilden, solange sie abz"ahlbar sind. Zum Beispiel kann man
die Menge der beidseitig unendlichen, quadratisch summierbaren Folgen
\[
l^2(\mathbb Z)=
\biggl\{
(c_k)_{k\in\mathbb Z}\,\biggl|\,
\sum_{k\in\mathbb Z}|c_k|^2<\infty
\biggr.
\biggr\}
\]
bilden, der mit
\[
(c,d)=\sum_{k\in\mathbb Z} \bar c_kb_k
\]
als Skalarprodukt.

\subsection{Quadratintegrierbare Funktionen: $L^2$}
Zu einem Interval $I=[a,b]$ k"onnen wir den Vektorraum der stetigen Funktionen
bilden, deren Quadrat integrierbar ist:
\[
V=\left\{
f:I\to\mathbb C\,\left|\,
\text{$f$ stetig und $\displaystyle \int_a^b|f(x)|^2\,dx$ existiert}
\right. \right\}
\]
Die Vektorraumoperationen sind die punktweise Addition von Funktionen
und Multiplikation mit einer Zahl:
\begin{align*}
(f+g)(x)&=f(x)+g(x)\\
(\lambda f)(x)=\lambda f(x).
\end{align*}
Wir k"onnen diesen Vektorraum mit einem Skalarprodukt ausstatten:
\[
(f,g)=\int_a^b \bar f(x)g(x)\,dx.
\]
Dies ist tats"achlich ein Skalarprodukt, denn
\begin{enumerate}
\item $(f,g)$ ist sesquilinear:
\begin{align*}
(f,\lambda g)&=\int_a^b\bar f(x)\lambda g(x)\,dx
=\lambda\int_a^b\bar f(x)g(x)\,dx = \lambda(f,g)
\\
(\lambda f,g)&=\int_a^b\overline{(\lambda f(x))}g(x)\,dx
=
\bar\lambda\int_a^b\bar f(x)g(x)\,dx=\bar\lambda (f,g).
\end{align*}
\item $(f,g)$ ist hermitesch:
\begin{align*}
(f,g)&=\int_a^b\bar f(x)g(x)\,dx = \overline{
\int_a^b f(x)\bar g(x)\,dx
}
=
\overline{(g,f)}
\end{align*}
\item $f$ ist positiv definit, also $f\ne 0\Rightarrow (f,f) >0$:
\begin{align*}
(f,f)&=\int_a^b \bar f(x)f(x)\,dx=\int_a^b|f(x)|^2\,dx
\end{align*}
Wenn $f\ne 0$, dann ist $|f(x)|>\varepsilon$ auf einem kleinen Interval,
also ist auch das rechte Integral positiv.
\end{enumerate}
Allerdings ist die Menge $V$ noch kein Hilbertraum, weil Grenzwerte von
Folgen nicht unbedingt existieren m"ussen. Zum Beispiel kann man aus
der logistischen Funktion
\[
f(x)=\frac{2e^x}{e^x+1}-1=\frac{e^x-1}{e^x+1}
\]
die Folge
% XXX Bild der Funktionenfolge und Grenzfolge
\begin{figure}
\centering
\includegraphics[width=\hsize]{graphics/arctan-1.pdf}
\caption{In $L^2$ konvergente Folge von stetigen Funktionen, mit einer
unstetigen Grenzfunktion.
\label{logisticfunction}}
\end{figure}
\[
f_n(x)=f(nx)
\]
von Funktionen auf dem Interval $[-1,1]$ konstruieren. Die Funktionen
$f_n$ sind alle stetig.
Die Grenzfunktion (siehe auch Abbildung~\ref{logisticfunction})
\[
\lim_{n\to \infty}f_n(x)=\begin{cases}
1&\qquad x > 0\\
0&\qquad x = 0\\
-1&\qquad x < 0
\end{cases}
\]
ist aber sicher nicht stetig, sie hat eine Sprungstelle bei $x=0$.

F"ugt man der Menge $V$ alle Grenzfunktionen von Cauchy-Folgen hinzu,
entsteht der Raum $L^2(I)=L^2([a,b])$ der quadratintegrierbaren Funktionen.
Leider zeigt sich dabei, dass das Riemannsche Integral nicht leistungsf"ahig
genug ist.
Streng genommen muss der Integrationsbegriff daher erweitert
werden, bevor dies mathematisch korrekt durchf"uhrbar ist.
Wir begn"ugen uns in diesem Skript mit der Bemerkung, dass dies m"oglich ist, 
denn auf die praktische Berechnung von quantenmechanisch interessanten
Gr"ossen hat es keinen weiteren Einfluss.

Der Raum $L^2$ ist bereits bekannt aus der Fouriertheorie.
Quadratintegrierbare Funktionen auf dem Interval $[-\pi,\pi]$ haben
eine Fourierreihe, also eine Entwicklung in eine Reihe von
Exponentialfunktionen $e_k(x)=e^{ikx}$.
Die Fourierkoeffizienten der Fourierreihe
\[
f(x)=\sum_{k=-\infty}^\infty c_ke^{ikx}
\]
k"onnen berechnet werden mit
\[
c_k=c_k(f)=\frac{1}{\sqrt{2\pi}}\int_{-\pi}^{\pi} e^{-ikx} \,dx=(e_k,f),
\]
also dem Skalarprodukt. Die Fourierkoeffizienten bilden wieder eine
quadratintegrierbare Folge, also $(c_k)\in l^2(\mathbb Z)$. Man kann
nachweisen, dass die Fouriertransformation eine Isometrie von $L^2([-\pi,\pi])$
nach $l^2(\mathbb Z)$ ist.

Aus der allgemeinen Theorie der Hilbertr"aume
folgen praktisch n"utzliche, und direkt nicht ganz einfach
nachzuweisende Eigenschaften wie die Cauchy-Scharz-Ungleichung
\[
|(f,g)|^2
=
\left|
\int_{-\pi}^\pi \bar f(x)g(x)\,dx
\right|^2
\le 
\int_{-\pi}^\pi |f(x)|^2\,dx
\int_{-\pi}^\pi |g(x)|^2\,dx
=
(f,f)(g,g)=\|f\|^2\cdot\|g\|^2,
\]
oder die Parseval-Identit"at:
\[
(f,g)
=
\int_{-\pi}^{\pi}\bar f(x)g(x)\,dx
=
\sum_{k=-\infty}^\infty \bar c_k(f)c_k(g)
=
((c_k(f)),(c_k(g))).
\]

\section{Bra-Ket-Notation}
Richard Feynman hat eine Notation eingef"uhrt, welche die vielen verschiedenen
Notationen f"ur in der Quantenmechanik n"utzliche Hilbertr"aume
vereinheitlicht.
F"ur die physikalische Interpretation ist egal, welche Art von
Vektorraum f"ur die Beschreibung der Zust"ande eines Teilchens verwendet
wird.
Die Notation sollte sich also nicht "andern, wenn wir die Zust"ande eines
Elektrons in einem Atom mit Hilfe einer Wellenfunktion (also mit dem
Hilbertraum $L^2$) beschreiben. Alternativ k"onnten wir die diskreten
Zust"ande mit Hilfe eines Hilbertraumes $l^2$ verwenden.

Die Notation sollte alle Operationen mit Vektoren abzubilden erlauben,
und sie soll die f"ur die Quantenmechanik wichtigen Operationen besonders
bequem machen.

\subsection{Zustandsvektoren}
Die Zust"ande eines Teilchens k"onnen durch ganz verschiedene Parameter
beschrieben werden. Ein freies Teilchen kann zum Beispiel durch seinen
Impuls oder seine Position beschrieben werden. Ein Elektron in einem
Atom kann durch die Nummer der Schale beschrieben, in der es sich befindet. 
Statt von Vektoren (in $l^2$) oder Funktionen (in $L^2$) zu sprechen, k"onnen
wir also von irgend einer Art von Vektor in einem passenden Hilbertraum 
sprechen. Wir schreiben 
$|n=3,m=5\rangle$ f"ur einen Zustandsvektor, welcher zu den Quantenzahlen $n=3$ und $m=5$ geh"ort. Oder $|p\rangle$ f"ur den Zustandsvektor eines Teilchens
mit Impuls $p$. Oder $|x\rangle$ f"ur den Zustandsvektor eines Teilchens
mit Position $x$.

\subsection{Skalarprodukt}
Das Skalarprodukt wird in jedem Hilbertraum im Detail anders definiert,
einzig die algebraischen Eigenschaften sind dieselben. In der Bra-Ket-Notation
schreiben wir  f"ur das Skalarprodukt der Zustandsvektoren 
$|a\rangle$ und $|b\rangle$
\[
\langle a|b\rangle.
\]
In allen Beispielen von Hilbert-R"aumen hatten wir ein Art ``dualer''
Vektoren, im Falle eines endlichdimensionalen Vektorraumes waren dies
die hermitesch adjungierten Vektoren. Der Vektor $\langle a|$ ist eine
Verallgemeinerung dieser Idee, man k"onnte also schreiben
$\langle a|=|a\rangle^*$.

\subsection{Operatoren}
Operatoren sind lineare Abbildungen des Hilbertraumes. Der Operator $A$
erzeugt aus einem Zustandsvektor $|u\rangle$ einen neuen Vektor
$A|u\rangle$. Die Linearit"at von $A$ bedeutet
\[
A(\lambda |u\rangle + \mu |v\rangle)=\lambda A \,|u\rangle + \mu A\,|v\rangle.
\]

Ein Operator $A$ heisst selbstadjungiert, wenn $A^*=A$ gilt.
F"ur das Skalarprodukt bedeutet diese Eigenschaft, dass $(u, Av)=(Au,v)$ 
In der Bra-Ket-Notation schreiben wir 
\begin{equation}
(u,Av)=(Au,v)=\langle u|\,A\,|v\rangle.
\label{skalar-operator}
\end{equation}
F"ur selbstadjungierte Operatioren spielt es gar keine Rolle, auf welchen
der beiden Vektoren in einem Skalarprodukt er wirkt, und die Notation
(\ref{skalar-operator}) tr"agt dieser Eigenschaft Rechnung, indem sie
bez"uglich der beiden Vektoren symmetrisch ist.

Hat der Hilbertraum $\cal H$ eine Hilbertbasis $|i\rangle$, dann kann
man die Wirkung eines Operators auf einem beliebigen Zustandsvektor
bestimmen, wenn man seine Wirkung auf den Basisvektoren kennt.
Dazu muss man den Vektor $|u\rangle$ erst in der Hilbertbasis
ausgedr"ucken:
\[
|u\rangle = \sum_{i=0}^\infty u_i\, |i\rangle.
\]
Dann kann man $A\,|u\rangle$ mit Hilfe der Linearit"at berechnen:
\[
A\,|u\rangle = \sum_{i=0}^\infty u_iA\,|i\rangle.
\]
Wenn man das Resultat wieder in der Hilbertbasis ausdr"ucken will, 
muss man den Resultatvektor mit den Basisvektoren multiplizieren.
Der Koeffizient der Komponente $|j\rangle$ ist
\begin{equation}
\langle j|\,A\,|u\rangle
=
\sum_{i=0}^\infty u_i \langle j|\,A\,|i\rangle
=
\sum_{i=0}^\infty \langle j|\,A\,|i\rangle u_i
\label{matrixmultiplikation}
\end{equation}
Man nennt die Zahlen
\[
a_{ji}=\langle j|\,A\,|i\rangle
\]
die Matrixelemente des Operators $A$. Die Formel (\ref{matrixmultiplikation})
entspricht der bekannten Formel f"ur die Multiplikation einer Matrix
mit einem Vektor.

Wenn zwei Operatoren die gleichen Matrixelemente haben, dann stimmen
die Operatoren "uberein. Haben n"amlich zwei Operatoren $A$ und $A'$
die gleichen Matrixelemente, dann hat $A-A'$ die Matrixelemente $0$:
\[
\langle i|\,A-A'\,|j\rangle =0\qquad\forall i,j
\]

F"ur eine grosse Klasse von selbstadjungierten Operatoren $A$ gilt ein
Spektralsatz: es gibt eine Hilbertbasis des Hilbertraumes $\cal H$
von Eigenvektoren von $A$, d.~h.~Vektoren $|i\rangle>$, $i\in\mathbb N$
mit der Eigenschaft, dass 
\[
\langle i|j\rangle = \delta_{ij}=\begin{cases}
1&\qquad i=j\\
0&\qquad i\ne j
\end{cases}
\]
und 
\[
A\,|i\rangle=\alpha_i\,|i\rangle\qquad\forall i,
\]
d.~h.~$\alpha_i$ sind die Eigenwerte von $A$.
F"ur endlichdimensionale komplexe Vektorr"aume ist dies der bekannte
Satz, dass hermitesche Matrizen diagonalisiert werden k"onnen.

\subsection{Projektion}
Projektionen sind Operatoren $P$ mit der Eigenschaft $P^2=P$.
Ist $|u\rangle$ ein Zustandsvektor, dann kann man eine Projektion $P_u$
auf diesen Zustandsvektor konstruieren:
\begin{align*}
P_u\,|u\rangle&=|u\rangle,\\
P_u\,|v\rangle&=0\qquad \text{f"ur $|v\rangle$ mit $\langle v|u\rangle=0$}
\end{align*}
Wir schreiben 
\[
P_u=
|u\rangle\langle u|
.
\]
Diese Notation ist konsistent, denn 
\begin{align*}
|u\rangle\langle u|u\rangle&=|u\rangle\cdot 1\\
|u\rangle\langle u|v\rangle&=|u\rangle\cdot 0=0,\qquad
\text{f"ur $\langle u|v\rangle = 0$}
\end{align*}
Sind $|i\rangle$ mit $i=1,\dots, n$ orthonormale Zustandsvektoren, also
\[
\langle i|j\rangle =\delta_{ij},
\]
dann ist der Operator
\[
P=\sum_{i=1}^n |i\rangle \langle i|
\]
eine Projektion, denn
\[
P^2=
\sum_{i=1, j = 1}^n
|i\rangle \langle i|j\rangle \langle j|
=
\sum_{i=1, j = 1}^n
\delta_{ij} \,|i\rangle\langle j|=\sum_{i=1}^n\,|i\rangle\langle i|=P.
\]

Wenn es f"ur den Operator $A$ auf dem Hilbertraum $\cal H$ eine
Hilbertbasis aus Eigenvektoren $|i\rangle$ mit Eigenwerten $\alpha_i$ gibt,
dann kann der Operator selbst
als eine Linearkombination von Projektoren geschrieben werden. Wir setzen
\[
A'=\sum_{i=0}^\infty |i\rangle \,\alpha_i\, \langle i|.
\]
Wir vergleichen die Matrixelemente von $A'$ und $A$:
\begin{align*}
\langle k|\, A' \,|l\rangle
&=
\sum_{i=0}^\infty \langle k\,|i\rangle\, \alpha_i \,\langle i|l\rangle
=\sum_{i=0}^\infty \delta_{ki}\alpha_i\delta_{il}
=\alpha_l\delta_{kl}
\\
\langle k|\,A\,|l\rangle
&=
\langle k|\,\alpha_l\,|l\rangle=\alpha_l\delta_{kl}
\end{align*}
Die Matrixelemente stimmen "uberein, also ist $A=A'$.

