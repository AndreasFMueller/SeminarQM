\chapter{Hilbertr"aume}
\lhead{Hilbertr"aume}
\rhead{}
Wir wissen schon, dass Zust"ande Eigenvektoren sein sollen, wir brauchen
also einen Vektorraum, der ausreichend gross ist, alle Zust"ande aufzunehmen.
Ein typisches Quantensystem hat eine sehr grosse Zahl von Zust"anden,
vielleicht sogar unendlich viele.
Wir brauchen daher einen unendlichdimensionalen komplexen Vektorraum als
B"uhne f"ur die Quantenmechanik.
Der Begriff des Hilbertraumes liefert die gesuchte Struktur.

Ausgehend von der gewohnten Vorstellung eines endlichdimensionalen
Vektorraumes kann in vier Schritten eine ausreichend reichhaltige
Struktur aufgebaut werden:
\begin{enumerate}
\item Erweiterung der Definition eines reellen Vektorraumes dahingehend,
dass auch komplexe Skalare zugelassen werden.
\item Erweiterung des Begriffs des Skalarprodukts auf den komplexen Fall.
Dazu geh"oren auch der Begriff der L"ange eines Vektors und Ungleichungen
wie die Dreiecksungleichung, die unsere Intuition f"ur das Verhalten von
Abst"anden auf den unendlichdimensionalen Fall erweitern.
\item Der Begriff des Grenzwertes einer Vektorfolge erlaubt, Zust"ande
zu approximieren.
\item Die Hilbertbasis liefert eine Technik, wie die Zust"ande eines
Quantensystems als Basisvektoren f"ur den Zustandsraum verwendet werden
k"onnen.
\end{enumerate}

\section{Komplexe Vektorr"aume}
\section{Komplexes Skalarprodukt}
\section{Norm und Grenzwert}
\section{Hilbertbasis}
\section{Hilbertr"aume $l^2$ und $L^2$}
