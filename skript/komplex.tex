\chapter{Anhang: Komplexe Zahlen}
\lhead{Komplexe Zahlen}
\rhead{}
Der mathematische Formalismus der Quantenmechanik kann nicht ohne komplexe
Zahlen auskommen.
Leonhard Euler sah die Zahlen $\sqrt{-1}$ noch als imagin"ar an,
also als ohne Gegenst"uck in der realen Welt.
Elektroingenieure verwenden komplexe Zahlen mit grossem Erfolg in ihren
Anwendungen, sie spielen aber vor allem die Rolle eines praktischen
Werkzeugs. Die Regeln, mit denen am Schluss solcher Rechnungen sichergestellt
wird, dass die Resultate reell sind zeigt ausserdem, dass man alles auch
ohne komplexe Zahlen durchrechnen k"onnte, wenn auch wesentlich weniger
elegant.

In der Quantenmechanik geht es aber nicht mehr ohne komplexe Zahlen,
die physikalischen Gr"ossen selbst sind komplex. Es gibt zwar auch
hier wieder Regeln, die sicherstellen, dass Messresultate reell sind
(Operatoren m"ussen selbstadjungiert sein), aber sie erlauben nicht,
die ganze Quantenmechanik auf eine Art zu beschreiben, die ohne komplexe
Zahlen auskommt.

\section{Der K"orper $\mathbb C$ der komplexen Zahlen}
In den reellen Zahlen $\mathbb R$ k"onnen alle Grundoperationen ausgef"uhrt
werden, es ist jedoch nicht m"oglich, die Quadratwurzeln aus negativen
Zahlen zu ziehen. Eine analoge Situation trifft man schon viel fr"uher.
In den nat"urlichen Zahlen $\mathbb N$ kann man zwar addieren und
multiplizieren, aber nicht subtrahieren.
F"ugt man die negativen Zahlen hinzu erh"alt man eine Menge $\mathbb Z$,
in der die Subtraktion uneingeschr"ankt m"oglich ist. Division ist aber
immer noch nur f"ur spezielle Divisoren m"oglich. F"ugt man jedoch die
Br"uche zu $\mathbb Z$ hinzu, erh"alt man die Menge der rationalen Zahlen
$\mathbb Q$, in der Division uneingeschr"ankt m"oglich ist.
Doch auch $\mathbb Q$ ist nicht vollst"andig, die Zahl $\sqrt{2}$ ist
keine rationale Zahl. Nat"urlich kann man $\sqrt{2}$ durch eine
Folge von Br"uchen $r_n\in\mathbb Q$ approximieren, doch der Grenzwert
dieser Folge $\lim_{n\to\infty}r_n=\sqrt{2}$ ist nicht in $\mathbb Q$.
F"ugt man jedoch alle Grenzwerte von konvergenten Folgen zu $\mathbb Q$
hinzu, erh"alt man die Menge $\mathbb R$ der reellen Zahlen, in der
auch beliebige Wurzeln von positiven Zahlen gezogen werden k"onnen,
oder andere Grenzwerte wie $\pi$, $e$, die Werte von $\sin x$ und $\cos x$
und weitere.

\subsection{Grundoperationen f"ur die komplexen Zahlen}
Nach analogem Muster k"onnen wir auch $\mathbb R$ erweitern, so dass auch
die Wurzeln aus negativen Zahlen bestimmt werden k"onnen. Es reicht
sogar, nur die Wurzel von $-1$ hinzuzuf"ugen, denn jede andere Wurzel
einer negativen Zahl ist $\sqrt{-a}=\sqrt{-1}\cdot\sqrt{\mathstrut a}$.
Euler hat die Bezeichnung $i=\sqrt{-1}$ f"ur die imagin"are Einheit eingef"uhrt.
Es gilt nat"urlich $i^2=-1$.

\begin{definition}
Die Menge $\mathbb C=\{a+bi\,|\,a,b\in\mathbb R\}$ heisst die Menge der
komplexen Zahlen. Die komplexe Zahl $z=a+bi$ hat
Realteil $\operatorname{Re}z=a$ und Imagin"arteil $\operatorname{Im}z=b$.
Die Rechenoperationen sind so zu verstehen, die Rechenregeln
der Algebra erhalten bleiben\footnote{Man nennt dies das Permanenz-Prinzip}.
\end{definition}

Die Rechenoperationen folgen aus der Definition:
\begin{align*}
(a+bi)+(c+di)&=(a+c)+(b+d)i\\
(a+bi)(c+di)&=ac-i^2bd+(ad+bc)i=ac-bd+(ad+bc)i
\end{align*}
Die Division stellt noch ein Problem dar. Hier hilft das Konzept der
konjugiert komplexen Zahl.

\begin{definition}
Die Zahl $\bar z=a-bi$ heisst die zu $z=a+bi$ konjugiert komplexe Zahl.
\end{definition}

Zun"achst kann man mit der konjugiert komplexen Zahl den Betrag einer
komplexen Zahl definieren:
\[
z\bar z=(a+bi)(a-bi)=a^2+abi-abi-i^2b^2=a^2+b^2\qquad\Rightarrow\qquad
|z|^2=z\bar z.
\]
Andererseits kann man damit auch komplexe Br"uche berechnen, indem man
mit der konjugiert komplexen Zahl des Nenners erweitert:
\begin{align*}
\frac{a+bi}{c+di}&=
\frac{a+bi}{c+di}
\cdot
\frac{c-di}{c-di}=\frac{ac-bd+(ad+bd)i}{c^2+d^2}
\end{align*}
Die komplexen Zahlen k"onnen in einer Ebene visualisert werden: 
Realteil und Imagin"arteil werden entlang orthogonaler Achsen abgetragen.
Die Punkte $(x,y)$ der $x$-$y$-Ebene entsprechen also der komplexen Zahl
$x+yi$ der komplexen Zahlenebene.

\subsection{Polardarstellung}
Die Darstellung der komplexen Zahlen als Punkte einer Ebene suggeriert
auch eine alternative Schreibweise.
Ein Punkt $z$ der komplexen Ebene kann auch charakterisiert werden mit Hilfe von
Polarkoordinaten, also durch seine Entfernung $r=|z|$ vom Nullpunkt,
und durch Polarwinkel zwischen der reellen Achse und der Richtung
zur komplexen Zahl. Der Polarwinkel heisst auch Argument $\operatorname{arg}z$,
und es gilt
\[
\tan\operatorname{arg}z=\frac{\operatorname{Re}z}{\operatorname{Im}z}.
\]
Die Multiplikation von komplexen Zahlen bekommt in der Polardarstellung
eine besondere Interpretation:
\begin{align*}
z_1z_2
&=
(r_1\cos\varphi_1+ir_1\sin\varphi_1) (r_2\cos\varphi_2+ir_2\sin\varphi_2)
\\
&=
r_1r_2(\cos\varphi_1+i\sin\varphi_1) (\cos\varphi_2+i\sin\varphi_2)
\\
&=
r_1r_2\bigl(
\cos\varphi_1\cos\varphi_2-\sin\varphi_1\sin\varphi_2 +
(\cos\varphi_1\sin\varphi_2+\sin\varphi_1\cos\varphi_2)i\bigr)
\\
&=
r_1r_2(\cos(\varphi_1+\varphi_2)+i\sin(\varphi_1+\varphi_2))
\\
\Rightarrow \operatorname{arg}z_1z_2&=\arg z_1 + \arg z_2
\end{align*}
Die Multiplikation zweier komplexen Zahlen entspricht also der
Multiplikation der Betr"age, und der Addition der Argumente.

Wir versuchen jetzt, die Werte der Exponentialfunktion zu $e^z$ zu
bestimmen.
Die Exponentialgesetze sollten auch weiterhin gelten.
Sei also $z=a+bi$, dann ist
\[
e^z=e^{a+bi}=e^a\cdot e^{bi}.
\]
Die Exponentialfunktion reeller Zahlen ist bereits wohlbekannt, es muss
also nur noch untersucht werden, welche Bedeutung $e^{bi}$ hat.

Betrachten wir die Funktion $f(t)= e^{ti}$. Die Ableitungen von $f$ sind
\begin{align}
f'(t)&=ie^{ti}=if(t)\notag\\
f''(t)&=-f(t).\label{exp-dgl}
\end{align}
Die Funktion $f$ muss also eine L"osung der Differentialgleichung
(\ref{exp-dgl}) sein, welche die Anfangsbedingungen $f(0)=1$
$f'(0)=if(0)=i$ erf"ullen muss.
Doch die Differentialgleichung (\ref{exp-dgl}) hat die L"osungen
\[
f(t)=a\cos t+b\sin t.
\]
Setzt man die Anfangsbedingungen ein findet man
\begin{align*}
f(0)&=1&\Rightarrow&&a&=1\\
f'(0)&=1&\Rightarrow&&b&=i,
\end{align*}
so dass wir jetzt $e^{ti}$ ausrechnen k"onnen:
\begin{satz}[Euler]
\begin{align}
e^{it}=\cos t+i\sin t.
\label{euler-formula}
\end{align}
\end{satz}
Mit dieser Formel sind wir jetzt auch in der Lage, den Zusammenhang
zwischen einer komplexen Zahl und ihrem Betrag und Argument sehr pr"agnant
auszudr"ucken:
\[
z=|r|\cdot e^{i\operatorname{arg}z}.
\]

\subsection{Matrixdarstellung der komplexen Zahlen}
Die Algebra der komplexen Zahlen kann man auch als eine Algebra von Matrizen
schreiben. Dazu betrachten wir die Abbildung
\[
\varphi\colon
\mathbb C\to M_2(\mathbb R):
a+bi\mapsto\begin{pmatrix}a&b\\-b&a\end{pmatrix}
\]
Die imagin"are Einheit $i$ wird von $\varphi$ auf die Matrix
\[
\varphi(i)=J=\begin{pmatrix}0&1\\-1&0\end{pmatrix}
\]
abgebildet. Man kann nachrechnen, dass $J^2=-E$, und dass die Rechenregeln
f"ur die komplexen Zahlen durch die Abbildung $\varphi$ in die Rechenregeln
f"ur Matrizen transformiert werden.
Wir illustrieren dies f"ur die Multiplikation:
\begin{align*}
&(a+bi)(c+di)&&\mapsto&
&\begin{pmatrix}a&b\\-b&a\end{pmatrix}
\begin{pmatrix}c&d\\-d&c\end{pmatrix}
\\
&(ac-bd) + i(ad-bc)&&\mapsto&
&=\begin{pmatrix}
ac-bd&ad-bc\\
-ad+bc&ac-bd
\end{pmatrix}
\end{align*}

\section{Komplexe Matrizen}

\section{Eigenwertproblem f"ur komplexe Matrizen}

