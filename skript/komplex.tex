\chapter{Anhang: Komplexe Zahlen}
\lhead{Komplexe Zahlen}
\rhead{}
Der mathematische Formalismus der Quantenmechanik kann nicht ohne komplexe
Zahlen auskommen.
Leonhard Euler sah die Zahlen $\sqrt{-1}$ noch als imagin"ar an,
also als ohne Gegenst"uck in der realen Welt.
Elektroingenieure verwenden komplexe Zahlen mit grossem Erfolg in ihren
Anwendungen, sie spielen aber vor allem die Rolle eines praktischen
Werkzeugs. Die Regeln, mit denen am Schluss solcher Rechnungen sichergestellt
wird, dass die Resultate reell sind zeigt ausserdem, dass man alles auch
ohne komplexe Zahlen durchrechnen k"onnte, wenn auch wesentlich weniger
elegant.

In der Quantenmechanik geht es aber nicht mehr ohne komplexe Zahlen,
die physikalischen Gr"ossen selbst sind komplex. Es gibt zwar auch
hier wieder Regeln, die sicherstellen, dass Messresultate reell sind
(Operatoren m"ussen selbstadjungiert sein), aber sie erlauben nicht,
die ganze Quantenmechanik auf eine Art zu beschreiben, die ohne komplexe
Zahlen auskommt.

\section{Der K"orper der komplexen Zahlen $\mathbb C$}

\section{Komplexe Matrizen}

\section{Eigenwertproblem f"ur komplexe Matrizen}

