\chapter{Anhang: Kugelkoordinaten\label{chapter:kugelkoordinaten}}
\lhead{Kugelkoordinaten}
\rhead{}
In den Quantisierungsregeln wird festgelegt, wie Impulskomponenten
in kartesischen Koordinaten durch Differentialgoperatoren zu
ersetzen seien.
Allerdings tr"agt diese Wahl des Koordinatensystems der Kugelsymmetrie
des Problems nicht Rechnung. 
Wenn man aber die angemessenen Kugelkoordinaten verwenden will,
dann muss man in der Lage sein, jeden beliebigen Differentialoperator
durch Ableitungen nach Kugelkoordinaten auszudr"ucken.

\section{Differentialoperatoren in Kugelkoordinaten}
Wir erinnern an die Formeln f"ur Kugelkoordinaten:
\begin{align*}
x&=
r\sin\vartheta\cos\varphi
\\
y&=
r\sin\vartheta\sin\varphi
\\
z&=
r\cos\vartheta
\end{align*}
Ein Differentialoperator ist eine Linearkombination der Operatoren
\[
\frac{\partial}{\partial x},
\quad
\frac{\partial}{\partial y},
\quad\text{und}\quad
\frac{\partial}{\partial z},
\]
oder auch h"oherer Ableitungen.  Unser Ziel ist, einen solchen Operator
durch die Ableitungen nach den Kugelkoordinaten auszudr"ucken, also
als Linearkombination von
\[
\frac{\partial}{\partial r},
\quad
\frac{\partial}{\partial \vartheta},
\quad\text{und}\quad
\frac{\partial}{\partial \varphi}.
\]
Offenbar reicht es dazu, jeden einzelnen Ableitungsoperator nach kartesischen
Koordiaten in den Ableitungsoperatoren nach Kugelkoordinaten auszudr"ucken.
Denn es gilt
\begin{align*}
\frac{\partial}{\partial x}
&=
\frac{\partial r}{\partial x} \frac{\partial}{\partial r}
+
\frac{\partial \vartheta}{\partial x} \frac{\partial}{\partial \vartheta}
+
\frac{\partial \varphi}{\partial x} \frac{\partial}{\partial \varphi}
\\
\frac{\partial}{\partial y}
&=
\frac{\partial r}{\partial y} \frac{\partial}{\partial r}
+
\frac{\partial \vartheta}{\partial y} \frac{\partial}{\partial \vartheta}
+
\frac{\partial \varphi}{\partial y} \frac{\partial}{\partial \varphi}
\\
\frac{\partial}{\partial z}
&=
\frac{\partial r}{\partial z} \frac{\partial}{\partial r}
+
\frac{\partial \vartheta}{\partial z} \frac{\partial}{\partial \vartheta}
+
\frac{\partial \varphi}{\partial z} \frac{\partial}{\partial \varphi}
\end{align*}
Um also einen Differentialoperator in Kugelkoordinaten ausdr"ucken zu
k"onnen, m"ussen wir als erstes alle Kugelkoordinaten nach kartesischen
Koordinaten ableiten k"onnen.

\section{Ableitung von Kugelkoordinaten nach kartesischen Koordinaten}
Wir beginnen mit den Ableitungen von $r$. Dazu rechnen wir zun"achst die
Ableitungen von $r^2$ aus:
\[
\frac{\partial r^2}{\partial x}
=
2r\frac{\partial r}{\partial x}
\qquad
\Rightarrow
\qquad
\frac{\partial r}{\partial x}
=
\frac1{2r}\frac{\partial r^2}{\partial x}.
\]
Der Ausdruck  $r^2$ ist quadratisch in den Koordinaten, und ist damit
viel einfacher auszurechnen:
\[
\frac{\partial r^2}{\partial x}=\frac{\partial}{\partial x}(x^2+y^2+z^2)=2x
\]
Damit k"onnen wir jetzt die Ableitungen von $r$ nach den kartesischen
Koordinaten zusammenstellen
\begin{equation}
\begin{aligned}
\frac{\partial r}{\partial x}
&=
\frac1{2r}2x=\frac{x}{r}=\sin\vartheta\cos\varphi
\\
\frac{\partial r}{\partial y}
&=
\frac1{2r}2y=\frac{y}{r}=\sin\vartheta\sin\varphi
\\
\frac{\partial r}{\partial z}
&=
\frac1{2r}2z=\cos\vartheta
\end{aligned}
\label{ableitungenvonr}
\end{equation}
F"ur die Ableitungen von $\vartheta$ dr"ucken wir $z$ durch Kugelkoordinaten
aus, leiten nach den kartesischen Koordinaten ab und l"osen nach den
Ableitungen von $\varphi$ nach den kartesischen Koordinaten auf:
\begin{align*}
z&=r\cos\vartheta
&&\Rightarrow&
\frac{\partial z}{\partial z}
&=
\frac{\partial r}{\partial z}\cos\vartheta
-
r \sin\vartheta\frac{\partial \vartheta}{\partial z}
=1
&&\Rightarrow&
\frac{\partial\vartheta}{\partial z}
&=
-\frac1{r\sin\vartheta}
\biggl(1-\cos\vartheta\frac{\partial r}{\partial z}\biggr)
\\
&&&&
\frac{\partial z}{\partial x}
&=
\frac{\partial r}{\partial x}\cos\vartheta
	- r\sin\vartheta\frac{\partial\vartheta}{\partial x}
=0
&&\Rightarrow&
\frac{\partial\vartheta}{\partial x}
&=
\frac{\cos\vartheta}{r\sin\vartheta}\frac{\partial r}{\partial x}
\\
&&&&
\frac{\partial z}{\partial y}
&=
\frac{\partial r}{\partial y}\cos\vartheta
	- r\sin\vartheta\frac{\partial\vartheta}{\partial y}
=0
&&\Rightarrow&
\frac{\partial\vartheta}{\partial y}
&=
\frac{\cos\vartheta}{r\sin\vartheta}\frac{\partial r}{\partial y}
\end{align*}
Die Ableitungn von $r$ wurden in (\ref{ableitungenvonr}) bereits berechnet,
und k"onnen hier eingesetzt werden:
\begin{equation}
\begin{aligned}
\frac{\partial\vartheta}{\partial z}
&=
-\frac{1}{r\sin\vartheta}(1-\cos^2\vartheta)
=
-\frac{1}{r\sin\vartheta}\sin^2\vartheta
=
-\frac{\sin\vartheta}{r}
\\
\frac{\partial \vartheta}{\partial x}
&=
\frac{\cos\vartheta}{r\sin\vartheta}\sin\vartheta\cos\varphi
=
\frac{\cos\vartheta\cos\varphi}{r}
\\
\frac{\partial \vartheta}{\partial y}
&=
\frac{\cos\vartheta}{r\sin\vartheta}\sin\vartheta\sin\varphi
=
\frac{\cos\vartheta\sin\varphi}{r}
\end{aligned}
\label{ableitungenvonvartheta}
\end{equation}
Wir brauchen noch die Ableitungen von $\varphi$, zun"achst nach $x$:
\begin{align*}
1=
\frac{\partial x}{\partial x}
&=
\frac{\partial r}{\partial x}\sin\vartheta\cos\varphi
+
r\cos\vartheta \frac{\partial\vartheta}{\partial x}\cos\varphi
-
r\sin\vartheta\sin\varphi\frac{\partial\varphi}{\partial x}
\\
&=
\sin\vartheta \cos\varphi
\sin\vartheta\cos\varphi
+
r\cos\vartheta
\frac{\cos\vartheta \cos\varphi}{r}
\cos\varphi
-
r\sin\vartheta\sin\varphi
\frac{\partial\varphi}{\partial x}
\\
&=\cos^2\varphi-r\sin\vartheta\sin\varphi\frac{\partial\varphi}{\partial x}
\\
1-\cos^2\varphi
&=
\sin^2\varphi=
-r\sin\vartheta\sin\varphi\frac{\partial\varphi}{\partial x}
&\Rightarrow\quad
\frac{\partial\varphi}{\partial x}
&=-\frac{\sin\varphi}{r\sin\vartheta}
\\
0=
\frac{\partial y}{\partial x}
&=
\frac{\partial r}{\partial x} \sin\vartheta\sin\varphi
+
r\cos\vartheta \frac{\partial\vartheta}{\partial x}\sin\varphi
+
r\sin\vartheta\cos\varphi\frac{\partial\varphi}{\partial x}
\\
&=
\sin\vartheta\cos\varphi
\sin\vartheta\sin\varphi
+
r\cos\vartheta
\frac{\cos\vartheta\cos\varphi}{r}
\sin\varphi
+
r\sin\vartheta\cos\varphi\frac{\partial\varphi}{\partial x}
\\
&=
\cos\varphi\sin\varphi
+
r\sin\vartheta\cos\varphi
\frac{\partial\varphi}{\partial x}
&\Rightarrow\quad
\frac{\partial\varphi}{\partial x}
&=
-\frac{\sin\varphi}{r\sin\vartheta}
\end{align*}
Die Ableitung von $\varphi$ nach $y$:
\begin{align*}
0=\frac{\partial x}{\partial y}
&=
\frac{\partial r}{\partial y}
\sin\vartheta\cos\varphi
+
r\cos\vartheta
\frac{\partial\vartheta}{\partial y}
\cos\varphi
-
r\sin\vartheta\sin\varphi
\frac{\partial\varphi}{\partial y}
\\
&=
\sin\vartheta\sin\varphi
\sin\vartheta\cos\varphi
+
r\cos\vartheta
\frac{\cos\vartheta\sin\varphi}{r}
\cos\varphi
-
r\sin\vartheta\sin\varphi
\frac{\partial\varphi}{\partial y}
\\
&=
\sin^2\vartheta \sin\varphi \cos\varphi
+
\cos^2\vartheta \sin\varphi \cos\varphi
-
r\sin\vartheta\sin\varphi
\frac{\partial\varphi}{\partial y}
\\
&=
\sin\varphi \cos\varphi
-
r\sin\vartheta\sin\varphi
\frac{\partial\varphi}{\partial y}
&\Rightarrow\quad
\frac{\partial\varphi}{\partial y}
&=
\frac{\cos\varphi}{r\sin\vartheta}
\\
1=\frac{\partial y}{\partial y}
&=
\frac{\partial r}{\partial y}
\sin\vartheta \sin\varphi
+
r\cos\vartheta
\frac{\partial\vartheta}{\partial y}
\sin\varphi
+
r\sin\vartheta\cos\varphi
\frac{\partial\varphi}{\partial y}
\\
&=
\sin\vartheta \sin\varphi
\sin\vartheta \sin\varphi
+
r\cos\vartheta
\frac{\cos\vartheta\sin\varphi}{r}
\sin\varphi
+
r\sin\vartheta\cos\varphi
\frac{\partial\varphi}{\partial y}
\\
&=
\sin^2\vartheta \sin^2\varphi
+
\cos^2\vartheta
\sin^2\varphi
+
r\sin\vartheta\cos\varphi
\frac{\partial\varphi}{\partial y}
\\
1-
\sin^2\varphi
&=
r\sin\vartheta\cos\varphi
\frac{\partial\varphi}{\partial y}
&
\Rightarrow\quad
\frac{\partial\varphi}{\partial y}
&=
\frac{\cos\varphi}{r\sin\vartheta}
\end{align*}
Ableitungen von $\varphi$ nach $z$:
\begin{align*}
0=\frac{\partial x}{\partial z}
&=
\frac{\partial r}{\partial z}
\sin\vartheta\cos\varphi
+
r\cos\vartheta
\frac{\partial\vartheta}{\partial z}
\cos\varphi
-
r\sin\vartheta\sin\varphi
\frac{\partial\varphi}{\partial z}
\\
&=
\cos\vartheta
\sin\vartheta\cos\varphi
-
r\cos\vartheta
\frac{\sin\vartheta}{r}
\cos\varphi
-
r\sin\vartheta\sin\varphi
\frac{\partial\varphi}{\partial z}
\\
&=
-
r\sin\vartheta\sin\varphi
\frac{\partial\varphi}{\partial z}
&\Rightarrow\quad
\frac{\partial\varphi}{\partial z}
&=0
\\
0=\frac{\partial y}{\partial z}
&=
\frac{\partial r}{\partial z}
\sin\vartheta\cos\varphi
+
r\cos\vartheta
\frac{\partial\vartheta}{\partial z}
\cos\varphi
+
r\sin\vartheta\cos\varphi
\frac{\partial\varphi}{\partial z}
\\
&=
\cos\vartheta
\sin\vartheta\cos\varphi
-
r\cos\vartheta
\frac{\sin\vartheta}{r}
\cos\varphi
+
r\sin\vartheta\cos\varphi
\frac{\partial\varphi}{\partial z}
\\
&=
r\sin\vartheta\cos\varphi
\frac{\partial\varphi}{\partial z}
&\Rightarrow\quad
\frac{\partial\varphi}{\partial z}
&=0
\end{align*}
Wir haben alle Ableitungen der Kugelkoordinaten nach kartesischen
Koordinaten berechnet, und stellen die Resultate in der folgenden
Tabelle zusammen
\begin{center}
\begin{tabular}{|>{$}c<{$}|>{$}c<{$} >{$}c<{$} >{$}c<{$}|}
\hline
&r&\vartheta&\varphi\\
\hline
\displaystyle\frac{\partial^{\mathstrut}}{\partial x_{\mathstrut}}
	&\sin\vartheta\cos\varphi
		&\displaystyle\frac{\cos\vartheta\cos\varphi}{r}
			&\displaystyle-\frac{\sin\varphi}{r\sin\vartheta}
\\
\displaystyle\frac{\partial^{\mathstrut}}{\partial y_{\mathstrut}}
	&\sin\vartheta\sin\varphi
		&\displaystyle\frac{\cos\vartheta\sin\varphi}{r}
			&\displaystyle\frac{\cos\varphi}{r\sin\vartheta}
\\
\displaystyle\frac{\partial^{\mathstrut}}{\partial z_{\mathstrut}}
	&\cos\varphi
		&\displaystyle-\frac{\sin\vartheta}{r}
			&0
\\
\hline
\end{tabular}
\end{center}

Damit k"onnen wir jetzt die Ableitungen nach den kartesischen Koordinaten
vollst"andig durch die Ableitungen nach Kugelkoordinaten ausdr"ucken:
\begin{equation}
\begin{linsys}{4}
\displaystyle
\frac{\partial^{\mathstrut}}{\partial x_{\mathstrut}}
&=&
\displaystyle
\frac{\partial r}{\partial x}\frac{\partial}{\partial r}
+
\frac{\partial \vartheta}{\partial x}\frac{\partial}{\partial \vartheta}
+
\frac{\partial \varphi}{\partial x}\frac{\partial}{\partial \varphi}
&=&
\displaystyle
\sin\vartheta\cos\varphi
\frac{\partial}{\partial r}
&+&
\displaystyle
\frac{\cos\vartheta\cos\varphi}{r}
\frac{\partial}{\partial\vartheta}
&-&
\displaystyle
\frac{\sin\varphi}{r\sin\vartheta}
\frac{\partial}{\partial\varphi}
\\
\displaystyle
\frac{\partial^{\mathstrut}}{\partial y_{\mathstrut}}
&=&
\displaystyle
\frac{\partial r}{\partial y}\frac{\partial}{\partial r}
+
\frac{\partial \vartheta}{\partial y}\frac{\partial}{\partial \vartheta}
+
\frac{\partial \varphi}{\partial y}\frac{\partial}{\partial \varphi}
&=&
\displaystyle
\sin\vartheta\sin\varphi
\frac{\partial}{\partial r}
&+&
\displaystyle
\frac{\cos\vartheta\sin\varphi}{r}
\frac{\partial}{\partial\vartheta}
&+&
\displaystyle
\frac{\cos\varphi}{r\sin\vartheta}
\frac{\partial}{\partial\varphi}
\\
\displaystyle
\frac{\partial^{\mathstrut}}{\partial z_{\mathstrut}}
&=&
\displaystyle
\frac{\partial r}{\partial z}\frac{\partial}{\partial r}
+
\frac{\partial \vartheta}{\partial z}\frac{\partial}{\partial \vartheta}
+
\frac{\partial \varphi}{\partial z}\frac{\partial}{\partial \varphi}
&=&
\displaystyle
\cos\vartheta
\frac{\partial}{\partial r}
&-&
\displaystyle
\frac{\sin\vartheta}{r}
\frac{\partial}{\partial\vartheta}
& &
\end{linsys}
\label{diffopkugel}
\end{equation}
Damit haben wir alle kartesischen Ableitungsoperatoren durch
Ableitungsoperatoren in Kugelkoordinaten ausgedr"uckt. 
Ausgehend von den Formeln (\ref{diffopkugel}) kann man jetzt jeden
in kartesischen Koordinaten gegebenen Differentialgoperator
in Kugelkoordinaten ausdr"ucken, indem man jedes Vorkommen eines
Ableitungsoperators $\frac{\partial}{\partial x_i}$ durch den
entsprechenden Ausdruck aus (\ref{diffopkugel}) ersetzt.

Im Kapitel~\ref{chapter:drehimpuls} wird diese Technik im
Abschnitt~\ref{section:drehimpulsortsdarstellung} verwendet.
Dort wird gezeigt, wie man die Drehimpulsoperatoren in Kugelkoordinaten
darstellen.
Dies erlaubt, den einzelnen Komponenten des Laplaceoperators, der in
Kapitel~\ref{chapter:wasserstoff} bereits untersucht wurde, eine physikalische
Bedeutung zu geben.
Man kann diese Technik aber auch verwenden, um den Laplaceoperator selbst
in Kugelkoordinaten auszudr"ucken, was im n"achsten Abschnitt als
Beispiel f"ur dieses Programm durchgef"uhrt werden soll.

\section{Laplaceoperator}
Der Laplace-Operator in kartesischen Koordinaten ist
\[
\Delta
=
\frac{\partial^2}{\partial x^2}
+
\frac{\partial^2}{\partial y^2}
+
\frac{\partial^2}{\partial z^2}.
\]
In Kugelkoordinaten m"ussen wir alle Differentialoperatoren durch
ihre Versionen in Kugelkoordinaten gem"ass (\ref{diffopkugel}) 
ersetzen.
Um diese Rechnung etwas "ubersichtlicher zu gestalten, werden wir erst
die Quadrate der Ableitungen nach den Koordinaten ausrechnen, und diese
am Schluss zusammenbringen.
Wir beginnen mit den zweiten Ableitungen nach $x$:
\begin{align*}
\frac{\partial^2}{\partial x^2}
&=
\biggl(
\sin\vartheta\cos\varphi
\frac{\partial}{\partial r}
+
\frac{\cos\vartheta\cos\varphi}{r}
\frac{\partial}{\partial\vartheta}
-
\frac{\sin\varphi}{r\sin\vartheta}
\frac{\partial}{\partial\varphi}
\biggr)
\biggl(
\sin\vartheta\cos\varphi
\frac{\partial}{\partial r}
+
\frac{\cos\vartheta\cos\varphi}{r}
\frac{\partial}{\partial\vartheta}
-
\frac{\sin\varphi}{r\sin\vartheta}
\frac{\partial}{\partial\varphi}
\biggr)
\\
&=
\sin^2\vartheta\cos^2\varphi\frac{\partial^2}{\partial r^2}
\\
&\qquad
+
\sin\vartheta \cos\vartheta \cos^2\varphi
\biggl(
-\frac1{r^2}
\frac{\partial}{\partial\vartheta}
+\frac1{r}
\frac{\partial^2}{\partial r\,\partial\vartheta}
\biggr)
\\
&\qquad
+
\cos\varphi\sin\varphi\biggl(
\frac1{r^2}\frac{\partial}{\partial\varphi}
-\frac1{r}\frac{\partial^2}{\partial r\,\partial\varphi}
\biggr)
\\
&\qquad
+\frac1{r}
\cos\vartheta\cos^2\varphi
\biggl(
\cos\vartheta\frac{\partial}{\partial r}
+
\sin\vartheta \frac{\partial^2}{\partial r\,\partial\vartheta}
\biggr)
\\
&\qquad
+\frac1{r^2}
\cos\vartheta\cos^2\varphi\biggl(
-\sin\vartheta
\frac{\partial}{\partial\vartheta}
+
\cos\vartheta
\frac{\partial^2}{\partial\vartheta^2}
\biggr)
\\
&\qquad
-\frac1{r^2}
\cos\vartheta\cos\varphi\sin\varphi
\biggl(
-\frac{\cos\vartheta}{\sin^2\vartheta}
\frac{\partial}{\partial\varphi}
+\frac1{\sin\vartheta}
\frac{\partial^2}{\partial\vartheta\,\partial\varphi}
\biggr)
\\
&\qquad
-\frac1r\sin\varphi\biggl(
-\sin\varphi \frac{\partial}{\partial r}
+\cos\varphi \frac{\partial^2}{\partial r\,\partial\varphi}
\biggr)
\\
&\qquad
-\frac1{r^2}
\frac{ \cos\vartheta \sin\varphi }{\sin\vartheta}\biggl(
-\sin\varphi\frac{\partial}{\partial\vartheta}
+\cos\varphi\frac{\partial^2}{\partial\vartheta\,\partial\varphi}
\biggr)
\\
&\qquad
+\frac1{r^2}\frac{\sin\varphi}{\sin^2\vartheta}\biggl(
\cos\varphi\frac{\partial}{\partial\varphi}
+\sin\varphi\frac{\partial^2}{\partial\varphi^2}
\biggr)
\\
&=
\sin^2\vartheta\cos^2\varphi \frac{\partial^2}{\partial r^2}
+
\frac1r( \cos^2\vartheta\cos^2\varphi + \sin^2\varphi)
\frac{\partial}{\partial r}
\\
&\qquad
+
\frac{2}r\sin\vartheta\cos\vartheta\cos^2\varphi
\frac{\partial^2}{\partial r\,\partial\vartheta}
-
\frac2r\cos\varphi\sin\varphi
\frac{\partial^2}{\partial r\,\partial\varphi}
\\
&\qquad
+
\frac1r\cos^2\vartheta\cos^2\varphi \frac{\partial^2}{\partial \vartheta^2}
-
\frac1{r^2}\biggl(
2\cos\vartheta\sin\vartheta\cos^2\varphi
-\frac{\sin^2\varphi\cos\vartheta}{\sin\vartheta}
\biggr)
\frac{\partial}{\partial\vartheta}
\\
&\qquad
-
\frac2{r^2}\frac{ \cos\vartheta \sin\varphi \cos\varphi }{\sin\vartheta}
\frac{\partial^2}{\partial \vartheta\,\partial\varphi}
\\
&\qquad
+
\frac1{r^2}\frac{\sin^2\varphi}{\sin^2\vartheta}
\frac{\partial^2}{\partial \varphi^2}
+
\frac1{r^2}\cos\varphi\sin\varphi\biggl(
1+\frac{\cos^2\vartheta}{\sin^2\vartheta}
+
\frac1{\sin^2\vartheta}
\biggr)
\frac{\partial}{\partial\varphi}
\end{align*}
Des weiteren die zweiten Ableitungen nach $y$:
\begin{align*}
\frac{\partial^2}{\partial y^2}
&=
\biggl(
\sin\vartheta\sin\varphi
\frac{\partial}{\partial r}
+
\frac{\cos\vartheta\sin\varphi}{r}
\frac{\partial}{\partial\vartheta}
+
\frac{\cos\varphi}{r\sin\vartheta}
\frac{\partial}{\partial\varphi}
\biggr)
\biggl(
\sin\vartheta\sin\varphi
\frac{\partial}{\partial r}
+
\frac{\cos\vartheta\sin\varphi}{r}
\frac{\partial}{\partial\vartheta}
+
\frac{\cos\varphi}{r\sin\vartheta}
\frac{\partial}{\partial\varphi}
\biggr)
\\
&=
\sin^2\vartheta \sin^2\varphi \frac{\partial^2}{\partial r^2}
\\
&\qquad
+\sin\vartheta\cos\vartheta\sin^2\varphi\biggl(
-\frac1{r^2}\frac{\partial}{\partial\vartheta}+\frac1r\frac{\partial^2}{\partial r\,\partial\vartheta}
\biggr)
\\
&\qquad
+\sin\varphi\cos\varphi\biggl(
-\frac1{r^2}\frac{\partial}{\partial\varphi}+\frac1r\frac{\partial^2}{\partial r\,\partial\varphi}
\biggr)
\\
&\qquad
+\frac1r\cos\vartheta\sin^2\varphi\biggl(
\cos\vartheta\frac{\partial}{\partial r}
+\sin\vartheta\frac{\partial^2}{\partial r\,\partial\vartheta}
\biggr)
\\
&\qquad
+\frac1{r^2} \cos\vartheta \sin^2\varphi \biggl(
-\sin\vartheta\frac{\partial}{\partial\vartheta}
+\cos\vartheta\frac{\partial^2}{\partial\vartheta^2}
\biggr)
\\
&\qquad
+\frac1{r^2}\cos\vartheta\sin\varphi\cos\varphi\biggl(
-\frac{\cos\vartheta}{\sin^2\vartheta}\frac{\partial}{\partial\varphi}
+\frac1{\sin\vartheta}\frac{\partial^2}{\partial\vartheta\,\partial\varphi}
\biggr)
\\
&\qquad
+\frac1r\cos\varphi\biggl(
\cos\varphi\frac{\partial}{\partial r}
+\sin\varphi\frac{\partial^2}{\partial\varphi\,\partial r}
\biggr)
\\
&\qquad
+\frac1{r^2}\frac{\cos\vartheta}{\sin\vartheta}\cos\varphi\biggl(
\cos\varphi\frac{\partial}{\partial\vartheta}
+\sin\varphi\frac{\partial^2}{\partial\varphi\,\partial\vartheta}
\biggr)
\\
&\qquad
+\frac1{r^2}\frac{\cos\varphi}{\sin^2\vartheta}\biggl(
-\sin\varphi\frac{\partial}{\partial\varphi}
+\cos\varphi\frac{\partial^2}{\partial\varphi^2}
\biggr)
\\
&=
\sin^2\vartheta\sin^2\varphi \frac{\partial^2}{\partial r^2}
+
\frac1r(\cos^2\vartheta\sin^2\varphi + \cos^2\varphi)\frac{\partial}{\partial r}
\\
&\qquad
+
\frac2r\sin\vartheta\cos\vartheta\sin^2\varphi
\frac{\partial^2}{\partial r\,\partial\vartheta}
+
\frac2r
\sin\varphi\cos\varphi
\frac{\partial^2}{\partial r\,\partial\varphi}
\\
&\qquad
+
\frac1{r^2}\cos^2\vartheta\sin^2\varphi
\frac{\partial^2}{\partial\vartheta^2}
+
\frac1{r^2}
\biggl(
-2\cos\vartheta\sin\vartheta\sin^2\varphi
+\frac{\cos\vartheta}{\sin\vartheta}\cos^2\varphi
\biggr)
\frac{\partial}{\partial\vartheta}
\\
&\qquad
+
\frac2{r^2}\sin\varphi\cos\varphi\frac{\cos\vartheta}{\sin\vartheta}
\frac{\partial^2}{\partial\vartheta\,\partial\varphi}
\\
&\qquad
+
\frac1{r^2}\frac{\cos^2\varphi}{\sin^2\vartheta}
\frac{\partial^2}{\partial\varphi^2}
+
\frac1{r^2}
\sin\varphi\cos\varphi
\biggl(
-1
-
\frac{\cos^2\vartheta}{\sin^2\vartheta}
-\frac1{\sin^2\vartheta}
\biggr)
\frac{\partial}{\partial\varphi}
\end{align*}

Und abschliessen die zweiten Ableitungen nach $z$:
\begin{align*}
\frac{\partial^2}{\partial z^2}
&=
\biggl(
\cos\vartheta
\frac{\partial}{\partial r}
-
\frac{\sin\vartheta}{r}
\frac{\partial}{\partial\vartheta}
\biggr)
\biggl(
\cos\vartheta
\frac{\partial}{\partial r}
-
\frac{\sin\vartheta}{r}
\frac{\partial}{\partial\vartheta}
\biggr)
\\
&=
\cos^2\vartheta\frac{\partial^2}{\partial r^2}
-
\cos\vartheta \sin\vartheta\biggl(
-\frac1{r^2}\frac{\partial}{\partial\vartheta}
+\frac{\partial^2}{\partial r\,\partial\vartheta}
\biggr)
-
\frac1r\sin\vartheta\biggl(
-\sin\vartheta\frac{\partial}{\partial r}
+\cos\vartheta\frac{\partial^2}{\partial\vartheta\,\partial r}
\biggr)
\\
&\qquad
+
\frac1{r^2}\sin\vartheta\biggl(
\cos\vartheta\frac{\partial}{\partial\vartheta}
+\sin\vartheta\frac{\partial^2}{\partial\vartheta^2}
\biggr)
\\
&=\cos^2\vartheta\frac{\partial^2}{\partial r^2}
+
\frac1r\sin^2\vartheta\frac{\partial}{\partial r}
-
\frac2{r}\sin\vartheta\cos\vartheta
\frac{\partial^2}{\partial r\,\partial\vartheta}
+
\frac1{r^2}\sin^2\vartheta \frac{\partial^2}{\partial\vartheta^2}
+
\frac2{r^2}\sin\vartheta\cos\vartheta \frac{\partial}{\partial\vartheta}
\end{align*}

Da wir jetzt alle zweiten Ableitungen ausgerechnet haben, k"onnen wir auch
den Laplace-Operator zusammensetzen:
\begin{align}
\Delta
&=
\frac{\partial^2}{\partial x^2}+
\frac{\partial^2}{\partial y^2}+
\frac{\partial^2}{\partial z^2}
\notag
\\
&=
\biggl(
\sin^2\vartheta\cos^2\varphi
+
\sin^2\vartheta\sin^2\varphi
+
\cos^2\vartheta
\biggr)
\frac{\partial^2}{\partial r^2}
\notag
\\
&\qquad
+
\frac1r\biggl(
\cos^2\vartheta\cos^2\varphi+\sin^2\varphi
+
\cos^2\vartheta\sin^2\varphi +\cos^2\varphi
+
\sin^2\vartheta
\biggr)
\frac{\partial}{\partial r}
\notag
\\
&\qquad
+
\frac2r\biggl(
\sin\vartheta\cos\vartheta\cos^2\varphi
+
\sin\vartheta\cos\vartheta\sin^2\varphi
-\sin\vartheta\cos\vartheta
\biggr)
\frac{\partial^2}{\partial r\,\partial\vartheta}
\notag
\\
&\qquad
+
\frac{2}{r}
\biggl(
-\sin\varphi\cos\varphi
+ \sin\varphi\cos\varphi
\biggr)
\frac{\partial^2}{\partial r\,\partial\varphi}
\notag
\\
&\qquad
+
\frac1{r^2}
\biggl(
\cos^2\vartheta\cos^2\varphi
+\cos^2\vartheta\sin^2\varphi
+\sin^2\vartheta
\biggr)
\frac{\partial^2}{\partial\vartheta^2}
\notag
\\
&\qquad
+
\frac1{r^2}
\sin\vartheta\cos\vartheta
\biggl(
-2 \cos^2\varphi
	+\frac{\sin^2\varphi}{\sin^2\vartheta}
-2\sin^2\varphi
+\frac{\cos^2\varphi}{\sin^2\vartheta}
+
2
\biggr)
\frac{\partial}{\partial\vartheta}
\notag
\\
&\qquad
+
\frac1{r^2}\sin\varphi\cos\varphi\frac{\cos\vartheta}{\sin\vartheta}
\biggl(
-2+2
\biggr)
\frac{\partial^2}{\partial\vartheta\,\partial\varphi}
\notag
\\
&\qquad
+
\frac1{r^2}\frac{\sin^2\varphi+\cos^2\varphi}{\sin^2\vartheta}
\frac{\partial^2}{\partial\varphi^2}
\notag
\\
&\qquad
+
\frac1{r^2}\cos\varphi\sin\varphi
\biggl(
1+\frac{\cos^2\vartheta}{\sin^2\vartheta}+\frac1{\sin^2\vartheta}
-1-\frac{\cos^2\vartheta}{\sin^2\vartheta}-\frac1{\sin^2\vartheta}
\biggr)
\frac{\partial}{\partial\varphi}
\notag
\\
&=
\frac{\partial^2}{\partial r^2}
+\frac{2}{r}\frac{\partial}{\partial r}
+\frac1{r^2}
\frac{\partial^2}{\partial\vartheta^2}
+
\frac1{r^2}
\frac{\cos\vartheta}{\sin\vartheta}
\frac{\partial}{\partial\vartheta}
+
\frac1{r^2}\frac1{\sin^2\vartheta}\frac{\partial^2}{\partial\varphi^2}
\notag
\\
&=
\frac1{r^2}\frac{\partial}{\partial r}\biggl(
r^2\frac{\partial}{\partial r}
\biggr)
+\frac1{r^2\sin\vartheta}\frac{\partial}{\partial\vartheta}\biggl(
\sin\vartheta\frac{\partial}{\partial\vartheta}
\biggr)
+\frac1{r^2}\frac1{\sin^2\vartheta}\frac{\partial^2}{\partial\varphi^2}
\label{laplacekugel}
\end{align}
Dies ist die Form des Laplace-Operators in Kugelkoordinaten, die
wir in der Diskussion des Wasserstoffatoms verwenden.
