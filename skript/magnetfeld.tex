\chapter{Magnetfeld\label{chapter:magnetfeld}}
\lhead{Magnetfeld}
\rhead{}
\index{Lorentz-Kraft}
\index{Magnetfeld}
Elektronen interagieren auch mit Magnetfeldern. Die Lorentz-Kraft
steht aber immer senkrecht auf der Bahn eines Teilchens, insbesondere
leistet sie keine Arbeit, und kann daher auch nicht in die Energie
eingehen.
Es braucht daher eine grundlegend andere Beschreibung des Magnetfeldes
und der Hamilton-Funktion, um Magentfelder in den Formalismus
integrieren zu k"onnen.

\section{Vektorpotential\label{section:vektorpotential}}
\rhead{Vektorpotential}
\index{Vektorpotential}
F"ur die quantenmechanische Beschreibung eines Teilchens in einem
elektrischen Feld konnten wir das Potential verwenden, welches
direkt die Energie eines Teilchens aus der Position zu bestimmen
erlaubt. 
Das elektrische Feld $\vec E$ kann aus dem Potential mittels
\[
\vec E=\operatorname{grad}\varphi
=
\begin{pmatrix}
\frac{\partial\varphi}{\partial x}\\
\frac{\partial\varphi}{\partial y}\\
\frac{\partial\varphi}{\partial z}
\end{pmatrix}
\]
rekonstruiert werden.
Das Magnetfeld hat kein Potential, aus dem es als Gradient wiedergewonnen
werden k"onnte, und die Lorentz-Kraft, die das Magnetfeld
\index{Lorentz-Kraft}
auf ein Teilchen aus"ubt, sind nicht vom Ort, sondern von der Geschwindigkeit
abh"angig.

Zum Magnetfeld gibt es trotzdem eine "ahnliche Konstruktion.
Das Magnetfeld $\vec B$ hat keine Quellen, es gibt keine magnetischen
Monopole.
Mathematisch kann man dies dadurch ausdr"ucken, dass 
\[
\operatorname{div}\vec B
=
\frac{\partial B_x}{\partial x}
+
\frac{\partial B_y}{\partial y}
+
\frac{\partial B_z}{\partial z}
=0
\]
ist.
Man kann zeigen, dass es zu so einem Vektorfeld ein Vektorfeld $\vec A$
gibt, aus dem sich das Magnetfeld mit der Formel
\[
\vec B=\operatorname{rot}\vec A
=\begin{pmatrix}
\frac{\partial A_3}{\partial y}-\frac{\partial A_2}{\partial z}\\
\frac{\partial A_1}{\partial z}-\frac{\partial A_3}{\partial x}\\
\frac{\partial A_2}{\partial x}-\frac{\partial A_1}{\partial y}
\end{pmatrix}
\]
rekonstruieren l"asst.

\begin{beispiel}
Sei $\vec B_0$ ein fester Vektor, wir betrachten das Feld
\[
\vec A=\frac12\vec B_0\times \vec r=\begin{pmatrix}
B_2z-B_3y\\
B_3x-B_1z\\
B_1y-B_2x
\end{pmatrix}.
\]
Das zugeh"orige Magnetfeld ist
\begin{equation}
\vec B
=
\operatorname{rot}\vec A
=
\frac12
\begin{pmatrix}
\frac{\partial}{\partial y}(B_1y-B_2x)-\frac{\partial}{\partial z}(B_3x-B_1z)\\
\frac{\partial}{\partial z}(B_2z-B_3y)-\frac{\partial}{\partial x}(B_1y-B_2x)\\
\frac{\partial}{\partial x}(B_3x-B_1z)-\frac{\partial}{\partial y}(B_2z-B_3y)
\end{pmatrix}
=
\begin{pmatrix}
B_1\\B_2\\B_3
\end{pmatrix}.
\end{equation}
Dieses Vektorfeld $\vec A$ geh"ort zum homogenen Magnetfeld $\vec B_0$.
\end{beispiel}

Das Vektorpotential $\vec A$ kann sich mit der Zeit "andern, so wie sich
ja auch $\vec B$ "andern kann.
Da ein sich "anderndes $\vec B$-Feld nach dem Induktionsgesetz einen
Spannung induziert, muss etwas "ahnliches f"ur das Vektorpotential gelten.
Das Induktionsgesetz ist
\[
\operatorname{rot}\vec E +\frac{\partial\vec B}{\partial t}=0,
\]
Wendet man den Rotations-Operator auf
$\vec E = -\operatorname{grad}\varphi$, ergibt sich
\[
\operatorname{rot}\vec E=\operatorname{rot}(-\operatorname{grad}\varphi)=0,
\]
es fehlt also der Induktionsterm
\[
\frac{\partial\vec B}{\partial t}
=
\operatorname{rot}\frac{\partial\vec A}{\partial t}.
\]
Die richtige Definition ist also die folgende.

\begin{definition}
Sei ein elektrisches Feld $\vec E$ und ein magnetisches Feld $\vec B$
gegeben.
Das elektrische Potential $\varphi$ und das Vektorpotential $\vec A$ 
dieser Felder sind die jenigen Felder, mit denen sich $\vec E$ und $\vec B$
nach den Formeln
\begin{equation}
\begin{aligned}
\vec E
&=
-\operatorname{grad}\varphi
-\frac{\partial\vec A}{\partial t}
&&\text{und}&
\vec B
&=
\operatorname{rot}\vec A
\end{aligned}
\label{skript:potentiale}
\end{equation}
ausdr"ucken l"asst.
\end{definition}


\section{Hamilton-Funktion f"ur ein Teilchen im Magnetfeld\label{section:hamilton-funktion-im-magnetfeld}}
\rhead{Hamilton-Funktion im Magnetfeld}
Wie sieht die Hamilton-Funktion eines Teilchens aus, welches  sich in
einem Magnetfeld $\vec B$ bewegt?
Das Feld $\vec B$ kann in der Hamilton-Funktion nicht direkt
verwendet werden, denn da die Lorentz-Kraft immer senkrecht
auf dem Feld $\vec B$ steht, leistet $\vec B$ keine Arbeit, und kann
daher auch keinen Beitrag zu Energie leisten.

Mit dem Laplace-Formalismus kann man die korrekte Form der
Hamilton-Funktion finden.
Wir wollen diese Rechnung nicht durchf"uhren, sondern nur das Resultat
angeben.
Es zeigt sich, dass man eine andere Impulskoordinate $\vec P$
an Stelle der bekannten Impulse $\vec p$ verwenden muss. 
Die neue Impulskoordinate ist 
\[
\vec P = \vec p + e\vec A,
\]
sie beinhaltet das Vektorpotential.
Dadurch werden die Impulse und das Vektorpotential miteinander 
verkn"upft, wir m"ussen nur kontrollieren, ab die Verkn"upfung
tats"achlich so ist, dass die Bewegungsgleichungen uns auf die
Lorentz-Kraft f"uhren.

Die kinetische Energie muss n"aturlich mit den Impulsen $\vec p$ ausgedr"ucket
werden, die mittels $\vec p=\vec P-e\vec A$ durch $\vec P$
ausgedr"uckt werden k"onnen.
Die Hamilton-Funktion f"ur ein Teilchen in einem Potential $\varphi$
und Vektorpotential $\vec A$ wird damit zu
\begin{equation}
H(\vec P, \vec x)=\frac1{2m}(\vec P-e\vec A)^2+e\varphi.
\label{skript:hamiltonmitmagnetfeld}
\end{equation}

Wir verifizieren, dass die Bewegungsgleichungen, die sich aus dieser
Hamilton-Funktion ergeben, tats"achlich die Lorentzkraft auf ein
Teilchen im Magnetfeld richtig wiedergeben.
Zun"achst rechnen wir die Bewegungsgleichungen aus
\begin{align*}
\frac{dx_i}{dt}
&=
\frac{\partial H}{\partial P_i}
=
\frac1m(P_i-eA_i)
\\
\frac{dP_i}{dt}
&=
-\frac{\partial H}{\partial x_i}
=
\frac{e}{m}\sum_{j=1}^3( P_j-eA_j)\frac{\partial A_j}{\partial x_i}
-e\frac{\partial\varphi}{\partial x_i}
=
e\sum_{j=1}^3\frac{p_j}{m}\frac{\partial A_j}{\partial x_i}
-e\frac{\partial\varphi}{\partial x_i}
=
e\sum_{j=1}^3v_j\frac{\partial A_j}{\partial x_i}
-e\frac{\partial\varphi}{\partial x_i}
\end{align*}
Wir m"ussen die Bewegungsgleichungen in der Newtonschen Form 
finden, denn nur so k"onnen wir die Kr"afte identifizieren.
Wir leiten daher die erste Gleichung nochmals nach der Zeit ab,
und erhalten
\[
\frac{d^2 x_i}{dt^2}
=
\frac1m\frac{dP_i}{dt}-\frac{e}{m}\biggl(
\frac{\partial A_i}{\partial t}
+\sum_{j=1}^3\frac{\partial A_i}{\partial x_j}\frac{dx_j}{dt}
\biggr)
=
\frac1m\frac{dP_i}{dt}-\frac{e}{m}\biggl(
\frac{\partial A_i}{\partial t}
+
\sum_{j=1}^3\frac{\partial A_i}{\partial x_j}v_j
\biggr)
\]
Darin k"onnen wir die Ableitung von $P_i$ nach der Zeit aus der
zweiten Hamiltonschen Bewegungsgleichung einsetzen:
\begin{align*}
m\frac{d^2 x_i}{dt^2}
&=
e\sum_{j=1}^3v_j\frac{\partial A_j}{\partial x_i}
-e\frac{\partial\varphi}{\partial x_i}
-
e\biggl(
\frac{\partial A_i}{\partial t}
+\sum_{j=1}^3\frac{\partial A_i}{\partial x_j}v_j
\biggr)
\end{align*}
Dies wird "ubersichtlicher, wenn wir die rechte Seite f"ur $i=1$
ausrechnen:
\begin{align*}
m\frac{d^2 x_1}{dt^2}
&=
e\biggl(
v_1\frac{\partial A_1}{\partial x_1}
+
v_2\frac{\partial A_2}{\partial x_1}
+
v_3\frac{\partial A_3}{\partial x_1}
-
v_1\frac{\partial A_1}{\partial x_1}
-
v_2\frac{\partial A_1}{\partial x_2}
-
v_3\frac{\partial A_1}{\partial x_3}
\biggr)
-e\frac{\partial\varphi}{\partial x_1}
-e\frac{\partial A_1}{\partial t}
\\
&=
e\biggl(
v_2\frac{\partial A_2}{\partial x_1}
+
v_3\frac{\partial A_3}{\partial x_1}
-
v_2\frac{\partial A_1}{\partial x_2}
-
v_3\frac{\partial A_1}{\partial x_3}
\biggr)
-e\frac{\partial\varphi}{\partial x_1}
-e\frac{\partial A_i}{\partial t}
\\
&=
e\biggl(
v_2
\biggl(
\frac{\partial A_2}{\partial x_1}
-
\frac{\partial A_1}{\partial x_2}
\biggr)
-
v_3\biggr(
\frac{\partial A_1}{\partial x_3}
-
\frac{\partial A_3}{\partial x_1}
\biggr)
\biggr)
+e\biggl(
-\frac{\partial\varphi}{\partial x_1}
-\frac{\partial A_i}{\partial t}
\biggr)
\\
&=
e\biggl(
v_2B_3
-
v_3B_2
\biggr)
+eE_1
\end{align*}
In dieser Form stehen die Kr"afte auf der rechten Seite. 
In vektorieller Form
\[
m\frac{d^2\vec x}{dt^2}
=
e\vec v\times\vec B
+e\vec E
\]
erkennt man die Kr"afte: der erste Term ist die Lorentz-Kraft,
der zweite ist die Kraft des elektrischen Feldes auf die Ladung $e$.

\section{Hamilton-Operator f"ur ein Teilchen im Magnetfeld\label{section:hamilton-operator-im-magnetfeld}}
\rhead{Hamilton-Operator im Magnetfeld}
Die Bewegung eines quantenmechanischen Teilchens erhalten wir jetzt
aus dem Hamilton-Operator mit den "ublichen Quantisierungsregeln.
Die Impulse $P_i$ m"ussen durch Ableitungsoperatoren ersetzt werden.
Der Hamiltonoperator in Ortsdarstellung ist also
\[
H=\frac1{2m}\sum_{k=1}^3\biggl(
\frac{\hbar}{i}
\frac{\partial}{\partial x_k}
-eA_k
\biggr)^2
+e\varphi.
\]
Man kann ihn zum Beispiel verwenden, um den Einfluss eines Magnetfeldes
auf die Elektronen in einem Atom zu messen.
Zum Beispiel kann man ein homogenes Magnetfeld in $z$-Richtung
aus dem Vektorpotential
\[
\vec A=\begin{pmatrix}
-yB\\0\\0
\end{pmatrix}
\qquad
\Rightarrow
\qquad
\vec B=\operatorname{rot}\vec A
=\begin{pmatrix}
\frac{\partial A_3}{\partial y}-\frac{\partial A_2}{\partial z}\\
\frac{\partial A_1}{\partial z}-\frac{\partial A_3}{\partial x}\\
\frac{\partial A_2}{\partial x}-\frac{\partial A_1}{\partial y}
\end{pmatrix}
=
\begin{pmatrix}
0\\0\\B
\end{pmatrix}
\]
bekommen.
Ein Teilchen in einem homogenen Magnetfeld hat daher den Hamilton-Operator
\[
H
=
\frac1{2m}\biggl(\frac{\hbar}{i}\frac{\partial}{\partial x}+eBy\biggr)^2
-\frac{\hbar^2}{2m}\frac{\partial^2}{\partial y^2}
-\frac{\hbar^2}{2m}\frac{\partial^2}{\partial z^2}
\]
Man beachte, dass jetzt der Impulsoperator in $y$-Richtung nicht mehr
mit dem Impulsoperator in $x$-Richtung vertauscht, man kann also
nicht gleichzeitig die $x$- und $y$-Impuls wissen.
Allerdings vertauschen den die Impulsoperatoren in $x$- und $z$-Richtung
immer noch mit $H$, diese Impulskomponenten sind also Erhaltungsgr"ossen.

Man kann versuchen, eine L"osung der Schr"odingergleichung f"ur $H$ mit
einem Ansatz
\[
\psi(x,y,z)
=
e^{\frac{i}{\hbar}(xp_x+zp_z)}Y(y)
\]
zu finden.
Setzt man diesen Ansatz in die Schr"odingergleichung ein, bekommt man
\begin{align}
e^{\frac{i}{\hbar}(xp_x+zp_z)}
\biggl(
\frac1{2m}
(p_x+eyB)^2
Y(y)
-
\frac1{2m}
\hbar^2 Y''(y)
+
\frac1{2m}p_z^2
Y(y)
\biggr)
&=
E
e^{\frac{i}{\hbar}(xp_x+zp_z)}
Y(y)
\notag
\\
-\frac{\hbar^2}{2m} Y''(y)
+
\biggl(
-E
+
\frac{ p_z^2}{2m}
+
\frac1{2m}
(p_x+eyB)^2
\biggr)
Y(y)
&=
0
\label{skript:Ygl}
\end{align}
Indem wir schreiben
\[
y_0=-\frac{p_x}{eB}
\qquad
\text{und}
\qquad
\omega_B
=\frac{eB}{m}
\]
erhalten wir aus (\ref{skript:Ygl}) die Differentialgleichung
\begin{equation}
-\frac{\hbar}{2m}Y''
+\frac{m}2\omega_B^2(y-y_0)^2Y
=
\biggl(E-\frac{p_z^2}{2m}\biggr) Y
\label{skript:landauniveaus}
\end{equation}
Dies ist die Eigenwertgleichung eines harmonischen Oszillators mit
Kreisfrequenz $\omega_B$.
Dessen Energieniveaus haben wir im Kapitel~\ref{chapter:harmonischeroszillator}
bereits berechnet, sie sind Vielfachen von $\hbar\omega_B$, insbesondere
gilt:
\[
\biggl(E-\frac{p_z^2}{2m}\biggr) = \hbar\omega_B\biggl(n+\frac12\biggr)
\qquad\Rightarrow\qquad
E=
\hbar\omega_B\biggl(n+\frac12\biggr)
-
\frac{p_z^2}{2m}
\]
Bei gegebenem Impuls in $x$- und $z$-Richtung ist die Bewegung in
$y$-Richtung als quantisiert.
Die zul"assigen Energieniveaus unterscheiden sich 

Man kann sich dieses Ph"anomen auch klassisch vorstellen.
Die Lorentzkraft zwingt Elektronen in Spiralbahnen um die Feldlinien des
Magnetfeldes.
Betrachtet man die Teilchen als Wellen, muss eine Umlauf um die
Feldlinie zu konstruktiver Interferenz f"uhren, daher sind nur
ganz bestimmte Phasenfrequenzen erlaubt.

%
% Eichtransformationen
%
\section{Eichtransformationen\label{section:eichtransformation}}
\rhead{Eichtransformationen}
Das Vektorpotential ist die physikalisch entscheidende Gr"osse.
Im Bohm-Aharonov-Experiment
werden Elektronen durch ein Gebiet geleitet, in dem sich $\vec A$,
aber nicht $\vec B$ "andert, trotzdem stellt das Experiment eine
Beeinflussung fest.

\subsection{Eichtransformationen der Potentiale}
\index{Eichtransformation!der Potentiale}
Das Vektorpotential ist nicht eindeutig: wenn man zu $\vec A$
den Gradienten einer Funktion $\chi$ hinzuaddiert, "andert sich
$\vec B$ nicht, weil
\[
\operatorname{rot}(\vec A+\operatorname{grad}\chi)
=
\underbrace{\operatorname{rot}\vec A}_{\vec B}
+
\underbrace{\operatorname{rot}\operatorname{grad}\chi}_{=0}
=
\vec B.
\]
Wenn das Vektorpotential die physikalisch ``reale'' Gr"osse ist,
sie aber nicht eindeutig ist, dann d"urfen sich keine messbaren
Gr"ossen "andern, wenn man $\vec A$ auf zul"assige Art "andert.
Insbesondere d"urfen sich das direkt messbare elektrische Feld
$\vec E$ und das magnetische Feld $\vec B$ nicht "andern.
"Andert man $\vec A$, dann muss sich nach (\ref{skript:potentiale})
auch $\varphi$ "andern:

\begin{definition}
Sei $\vec A$ das Vektorpotential und $\varphi$ das elektrische
Potential. Sei weiter $\chi$ eine beliebige Funktion.
Die {\em Eichtransformation} von $\vec A$ und $\varphi$ durch $\chi$ ist
die Ersetzung
\begin{equation}
\begin{aligned}
\vec A&\mapsto \vec A + \operatorname{grad}\chi
&&\text{und}&
\varphi&\mapsto \varphi-\frac{\partial\chi}{\partial t}.
\end{aligned}
\label{skript:eichtransformation}
\end{equation}
\end{definition}

\begin{satz}
Eine Eichtransformation (\ref{skript:eichtransformation}) "andert nichts an den
Feldern $\vec E$ und $\vec B$.
\end{satz}

\begin{proof}[Beweis]
Wir berechnen aus den eichtransformierten Feldern das elektrische 
 $\vec E'$ und das magnetische Feld $\vec B'$:
\begin{align*}
\vec B'
&=
\operatorname{rot}(\vec A+\operatorname{grad}\chi)
=
\operatorname{rot}\vec A+\operatorname{rot}\operatorname{grad}\chi
=
\vec B
\\
\vec E'
&=
-\operatorname{grad}\biggl(\varphi -\frac{\partial\chi}{\partial t}\biggr)
-\frac{\partial}{\partial t}(\vec A+\operatorname{grad}\chi)
=
\underbrace{
\operatorname{grad}\varphi-\frac{\partial\vec A}{\partial t}
}_{=\vec E}
+
\underbrace{
\operatorname{grad}\frac{\partial\chi}{\partial t}
-\frac{\partial}{\partial t}\operatorname{grad}\chi
}_{=0}
=
\vec E
\end{align*}
Die aus den eichtransformierten Potentialen abgeleiteten Felder $\vec E'$
und $\vec B'$ stimmen also mit den urspr"unglichen Feldern $\vec E$
und $\vec B$ "uberein.
\end{proof}

\subsection{Eichtransformation der Schr"odingergleichung}
Wenn man eine Eichtransformation auf die Potentiale in der
Schr"odingergleichug anwendet, dann darf sich keine beobachtbare
quantenmechanische Gr"osse "andern. Insbesondere m"ussen alle
Wahrscheinlichkeiten gleich bleiben, die Wellenfunktion
$\psi$ darf sich also h"ochstens um einen Phasenfaktor "andern.

Wir gehen daher davon aus, dass $\psi$ eine Wellenfunktion ist,
die die Schr"odingergleichung des Hamilton-Operators mit Vektorpotential
$\vec A$ l"ost, dass also
\[
\frac{\hbar}{i}\frac{\partial}{\partial t}\psi
=
\frac1{2m}\sum_{k=1}^3\biggl(\frac{\hbar}{i}\frac{\partial}{\partial x_k}-eA_k\biggr)\psi
+e\varphi\psi
\]
Wenn wir die Ersetzung
$\vec A\mapsto \vec A+\operatorname{grad}\chi$
im Hamilton-Operator durchf"uhren, m"ussen wir die Wellenfunktion
mit einem Phasenfaktor $e^{if(x)}$ korrigieren, wenn sie weiterhin
eine L"osung der Schr"odingergleichung sein soll.

Nicht nur die Schr"odingergleichung, jede andere Observable darf sich
auch nicht "andern.
Der Impulsoperator $p_k$ darf sich nicht "andern, wenn wir in ihm die
Eichtransformation ausf"uhren.
Wir berechnen daher die Wirkung von $p_k$ mit eichtransformiertem
Vektorpotential auf die mit Phasenfaktor modifizierte Wellenfunktion:
\begin{align*}
\biggl(
\frac{\hbar}{i}\frac{\partial}{\partial x_k}+eA_k
% XXX is there a factor e missing?
+\frac{\partial\chi}{\partial x_k}
\biggr)e^{if(x)}\psi(x)
&=
e^{if(x)}
\biggl(
\frac{\hbar}{i}\frac{\partial}{\partial x_k}+eA_k
\biggr)\psi(x)
+
e^{if(x)}
\biggl(
\frac{\partial\chi}{\partial x_k}
+
\frac{\hbar}{i}i\frac{\partial f}{\partial x_k}
\biggr)
\psi(x)
\end{align*}
Die Addition von $\operatorname{grad}\chi$ zum Vektorpotential
kann nur dann ohne Folgen f"ur die Physik sein, wenn der
zus"atzlich auf der rechten Seite auftauchende Term verschwindet,
wenn also gilt
\[
\frac{\partial\chi}{\partial x_k}=-\hbar\frac{\partial f}{\partial x_k},
\]
oder $\operatorname{grad}\chi=-\frac1{\hbar}\operatorname{grad}f$.
Wenn man also das Vektorpotential um $\operatorname{grad}\chi$ "andert,
muss man die Wellenfunktion um $e^{-\frac{i}{\hbar}\chi}$ "andern. 
Setzt man dies jedoch auf der linken Seite der zeitabh"angigen
Schr"odingergleichung ein, erh"alt man
\[
\frac{\hbar}{i}\frac{\partial}{\partial t}
\bigl(
e^{-\frac{i}{\hbar}\chi}
\psi
\bigr)
=
-
e^{-\frac{i}{\hbar}\chi}
\frac{\partial\chi}{\partial t}
\psi
+
e^{-\frac{i}{\hbar}\chi}
\frac{\partial}{\partial t}
\psi.
\]
Wenn sich also die Schr"odingergleichung nicht "andern soll, dann
muss der Summand $\partial\chi/\partial t$ zum Potential im
Hamilton-Operator geschlagen werden.
Das ist aber ohnehin, was die Eichtransformation (\ref{skript:eichtransformation})
f"ur das elektrische Potential verlangt.

Wenn man also in der Schr"odingergleichung alle "Anderungen
\begin{equation}
\begin{aligned}
\vec A&\mapsto \vec A + \operatorname{grad}\chi,
&
\varphi&\mapsto \varphi-\frac{\partial\chi}{\partial t}
&
&\text{und}&
\psi
&\mapsto
e^{-\frac{i}{\hbar}\chi}\psi
\label{skript:eichtransformationpsi}
\end{aligned}
\end{equation}
gleichzeitig durchf"uhrt, dann "andert sich physikalisch nichts,
auch wenn sich sowohl das Vektorpotential als auch das
elektrische Potential "andert.

Die Transformation (\ref{skript:eichtransformationpsi}) heisst 
{\em Eichtransformation} der Felder und der Wellenfunktion.
\index{Eichtransformation!der Felder und der Wellenfunktion}
Die Schr"odingergleichung ist invariant bez"uglich Eichtransformationen.
Die Eichtransformation stellt eine innere Symmetrie der Theorie dar.
Unterwirft man das Problem einer Eichtransformation, zum Beispiel um
es einer L"osung zug"anglicher zu machen, spricht man auch von der {\em Wahl
einer Eichung}.





