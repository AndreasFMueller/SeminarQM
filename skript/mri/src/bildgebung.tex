%\subsection{Bildgebung}
%
%Es gibt verschiedene M"oglichkeiten Bilder zu generieren. Das betrifft das Messverfahren genauso wie die anschliessende Berechnung der Bilder. Eines der ersten Verfahren war \emph{Back Projection Imaging}.
%Bei diesem Verfahren wird mit einem eindimensionalen Gradientenfeld gearbeitet. Das Objekt, welches ausgemessen werden soll, wird dabei aus verschiedenen Winkeln mit dem RF-Impuls angeregt. Aus den verschiedenen Einzelmessungen kann danach ein Bild generiert werden.
%Die Bildgebung wird mit der Hilfe von Gradientenfelder bewerkstelligt. Dies sind zus"atzliche Magnetfelder, welche jeweils linear von der jeweiligen Ortskoordinaten abh"angen. Durch die "Uberlagerung des Gradientenfeldes mit dem homogenen Magnetfeld wird die Larmorfrequenz ortsabh"angig.
%\begin{equation}
%\overrightarrow{w_0} = -\gamma(\overrightarrow{B_0}+(\overrightarrow{G}\cdot \overrightarrow{r})\overrightarrow{e_x})
%\end{equation}
%\subsubsection{Ortskodierung}
%Durch HF-Impulse wird eine selektive Anregung des Spins innerhalb einer Schicht erreicht. Das Prinzip wird gleichermassen zur r"aumlichen Kodierung innerhalb der Beobachtungsschicht verwendet. Das ergibt den Auslesegradienten $G^x$, welcher die Abh"angigkeit der Resonanzfrequenz vor Ort enth"alt. Als Beispiel die Gleichung 
%entlang der $x$-Achse:
%\begin{equation}
%w_x = - \gamma G^x x
%\end{equation}

%Es gibt verschiedene M"oglichkeiten Bilder zu generieren. Das betrifft das Messverfahren genauso wie die anschliessende Berechnung der Bilder. Eines der ersten Verfahren war \emph{Back Projection Imaging}.
%Bei diesem Verfahren wird mit einem eindimensionalen Gradientenfeld gearbeitet. Das Objekt, welches ausgemessen werden soll, wird dabei aus verschiedenen Winkeln mit dem RF-Impuls angeregt. Aus den verschiedenen Einzelmessungen kann danach ein Bild generiert werden.
%
%Die Bildgebung wird mit der Hilfe von Gradientenfelder bewerkstelligt. Dies sind zus"atzliche Magnetfelder, welche jeweils linear von der jeweiligen Ortskoordinaten abh"angen. Durch die "Uberlagerung des Gradientenfeldes mit dem homogenen Magnetfeld wird die Lamaorfrequenz ortsabh"angig.
%\begin{equation}
%\overrightarrow{w_0} = -\gamma(\overrightarrow{B_0}+(\overrightarrow{G}\cdot \overrightarrow{r})\overrightarrow{e_x})
%\end{equation}
%\subsubsection{Ortskodierung}
%Durch HF-Impulse wird eine selektive Anregung des Spins innerhalb einer Schicht erreicht. Das Prinzip wird gleichermassen zur r"aumlichen Kodierung innerhalb der Beobachtungsschicht verwendet. Das ergibt den Auslesegradienten $G^x$, welcher die Abh"angigkeit der Resonanzfrequenz vor Ort enth"alt. Als Beispiel die Gleichung 
%entlang der $x$-Achse:
%\begin{equation}
%w_x = - \gamma G^x x
%\end{equation}
%>>>>>>> 929620516d75ba86c5f429616ecb9ce2ed2dde7d
