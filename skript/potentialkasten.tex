\subsection{Potentialkasten\label{subsection:potentialkasten}}
\index{Potentialkasten}
Wir wenden die Theorie an, um die Energieniveaus eines eindimensionalen
Teilchens zu berechnen, welches in einem Interval $[-l,l]$ eingesperrt ist.
Realisiert werden kann diese Situation n"aherungsweise dadurch, dass 
ein Teilchen durch eine sehr hohe (unendlich hohe) Potentialbarriere
am verlassen des Bereiches gehindert wird.
Die Wellenfunktion $\psi(x)$ eines solchen Teilchens muss ausserhalb des
Intervals verschwinden, denn die Wahrscheinlichkeit, das Teilchen dort zu
finden, soll ja $0$ sein.
Man erreicht dies, indem man in den Punkten $\pm l$ eine unendlich
hohe Potentialbarriere errichtet.

\subsubsection{Hamilton-Operator und Schr"odingergleichung}
Der Hamilton-Operator des Problems ist
\[
-\frac{\hbar^2}{2m}\frac{\partial^2}{\partial x^2}+V(x),
\]
das Potential hat die Form
\[
V(x)=\begin{cases}
0&\qquad |x|<l\\
\infty&\qquad\text{sonst.}
\end{cases}
\]
Die zugeh"orige zeitunabh"angige Schr"odingergleichung ist
\begin{equation}
-\frac{\hbar^2}{2m}\psi''(x)=E\psi(x)
\label{skript:schroedingerkasten}
\end{equation}
im Inneren des Intervals.
Wir suchen eine L"osung von (\ref{skript:schroedingerkasten}),
die ausserdem den Randbedingungen
\[
\psi(\pm l)=0
\]
gen"ugen soll, damit wird dem unendlich hohen Potential ausserhalb des
Kastens Rechnung getragen.
Das charakteristische Polynom der Gleichung (\ref{skript:schroedingerkasten}) ist
\[
-\frac{\hbar^2}{2m}\lambda^2-E=0,
\]
es hat die Nullstellen
\[
\lambda_\pm = \pm\sqrt{\frac{2m}{\hbar^2}(-E)}.
\]

\subsubsection{Symmetrie}
\index{Spiegelung}
Der Hamilton-Operator dieses Problems ist symmetrisch bez"uglich
der Spiegelung.
Der Spiegelungsoperator $S$, der aus der Wellenfunktion
$\psi(x)$ die Wellenfunktion $\psi(-x)$ macht, vertauscht mit
dem Hamilton-Operator, also kann man die Eigenzust"ande von $H$ so w"ahlen,
dass sie auch Eigenzust"ande von $S$ sind.
Da $S^2=\operatorname{id}$, sind $\pm1$ die einzig m"oglichen Eigenwerte
von $S$, also ist $\psi(x)=\psi(-x)$ oder $\psi(x)=-\psi(-x)$.
Wenn man eine L"osung der Schr"odingergleichung sucht, kann man also
zus"atzlich verlangen, dass die L"osungen gerade oder ungerade Funktionen
sind.

\subsubsection{Negative Energie}
Falls $E<$ ist, sind die Nullstellen des chrakteristischen
Polynoms reell, $\lambda_+>0$ und $\lambda_-=-\lambda_+ < 0$,
und die allgemeine 
L"osung der Gleichung (\ref{skript:schroedingerkasten}) hat die Form
\[
\psi(x)=A_+e^{\lambda_+x}+A_-e^{\lambda_-x}.
\]
Ausserdem m"ussen die Randbedingungen erf"ullt sein, die man erh"alt,
indem man $x=\pm l$ setze:
\[
\begin{linsys}{2}
e^{ \lambda_+l} A_+&+&e^{ \lambda_-l}A_-&=&0\\
e^{-\lambda_+l}A_+&+&e^{-\lambda_-l}A_-&=&0
\end{linsys}
\]
Dieses Gleichungssystem f"ur die Koeffizienten $A_+$ und $A_-$ hat
Determinante
\[
\left|\begin{matrix}
e^{ \lambda_+l}&e^{ \lambda_-l}\\
e^{-\lambda_+l}&e^{-\lambda_-l}
\end{matrix}\right|
=
e^{\lambda_+l-\lambda_-l}-e^{\lambda_-l-\lambda_+l}
=
e^{2l\lambda_+}-e^{-2l\lambda_+}
=
2\sinh 2l\lambda_+
\]
Dieses Ausdruck verschwindet nur, wenn $\lambda_+=0$, im Widerspruch
zu $\lambda_+>0$.
Es gibt also keine L"osung f"ur Teilchen mit negativer Energie.

\subsubsection{Positive Energie}
In diesem Falls sind die Nullstellen imagin"ar, $\lambda_\pm=\pm ik$
mit $k=\sqrt{2mE/\hbar^2}$,
und die allgemeine L"osung wird durch Linearkombinationen der Funktionen
$\cos kx$ und $\sin kx$ gegeben.
Wir m"ussen also herausfinden, f"ur welche Werte von $E$ und damit $k$
die Randbedingungen erf"ullt werden k"onnen.
Wir wissen bereits, dass wir zus"atzlich verlangen k"onnen, dass die
L"osungen gerade oder ungerade sind.
Eine gerade L"osung muss von der Form $A\cos kx$ sein, eine ungerade
L"osung von der Form $B\sin kx$.
In beiden F"allen ist die Randbedingung am linken Rand automatisch
erf"ullt, wenn die Bedingun am rechten Rand erf"ullt ist.
Wir setzen die Randbedingungen ein:
\begin{align*}
A\cos kl&=0
	&&&
		B\sin kl&=0\\
kl&=\frac{\pi}2+n\pi,\quad n\in\mathbb Z
	&&&
		kl&=n\pi,\quad n\in\mathbb Z\\
k&=\frac{\pi}{l}\biggl(n+\frac12\biggr),\quad n\in\mathbb Z
	&&&
		k&=\frac{\pi}{l}n,\quad n\in\mathbb Z.
\end{align*}
Es sind also alle $k$-Werte m"oglich, die ganzzahlige Vielfache von
$\pi/2l$ sind.
Die geraden Vielfache f"uhren zu einer $\sin$-L"osung, die ungeraden
zu einer $\cos$-L"osung.

Aus den m"oglichen $k$-Werten k"onnen wir jetzt die m"oglichen 
Energiewerte ableiten
\begin{align*}
E&=\frac{\hbar^2\pi^2}{2ml^2}\biggl(n+\frac12\biggr)^2
&&&
E&=\frac{\hbar^2\pi^2}{2ml^2}n^2
\end{align*}
In einer Formel zusammengefasst kann man schreiben
\[
E_n
=
\frac{\hbar^2\pi^2}{2ml^2}\biggl(\frac{n}{2}\biggr)^2
=
\frac{h^2n^2}{32ml^2}.
\]
wobei f"ur gerades $n$ eine ungerade L"osungsfunktion auftritt,
f"ur ungerades $n$ aber eine gerade L"osungsfunktion.
\begin{figure}
\centering
\includegraphics{graphics/potential-1.pdf}
\caption{Teilchen in einem Potentialkasten
\label{skript:potentialkasten}}
\end{figure}

\subsubsection{Normierung}
Wir m"ussen noch sicherstellen, dass die L"osungsfunktionen korrekt
normiert sind.
Dazu m"ussen wir die Integrale von $\cos^2$ und $\sin^2$ berechnen.
Die Graphen von $\cos^2$ und $\sin^2$ halbieren aber genau das
Rechteck zwischen $-l$ und $l$ mit H"ohe $1$, also ist der Wert
des Integrals in jedem Fall $l$, die L"osungsfunktionen mit
der richtigen Normierung sind also:
\begin{align}
\psi_n(x)
&=
\begin{cases}
\displaystyle
\frac{1}{\sqrt{l}}\cos\frac{n \pi x}{2l}&\qquad \text{$n$ ungerade}\\
\\
\displaystyle
\frac{1}{\sqrt{l}}\sin\frac{n \pi x}{2l}&\qquad \text{$n$ gerade}.
\end{cases}
\label{skript:potentialkasten-psi-normiert}
\end{align}
Der Grundzustand ist $n=1$, mit Grundzustandsenergie $E_1=h^2/32ml^2$.

\subsubsection{Klassische Quantisierungsbedingung}
Klassisch kann man sich das Teilchen als eine Welle vorstellen, die
zwischen den beiden W"anden des Potentialkastens hin und her l"auft.
Der Impuls des Teilchens ist $\hbar k$, die Wirkung f"ur einen Weg
hin und zur"uck ist daher:
\[
\oint p\,dq
=
2\cdot \hbar k \cdot 2l = 2\cdot \hbar \frac{\pi}{l}\frac{n}{2}\cdot 2l=hn,
\]
dies entspricht genau der Bohrschen Quantisierungsbedingung.

\subsubsection{Physikalische Interpretation}
Offenbar ist es nicht m"oglich, auf beliebig kleinem Raum einzusperren.
Je kleiner der zur Verf"ugung stehende Raum, desto h"oher ist die
Energie, die ein Teilchen mindestens haben wird.
Im Kapitel~\ref{chapter:heisenberg} werden wir mit der Unsch"arferelationen
einen weiteren Ausdruck dieses Ph"anomens kennenlernen.

In einem Atomkern sind die Protonen und Neutronen auf sehr kleinem
Raum eingesperrt, und wir k"onnen versuchen, die minimale Energie 
eines Neutrons abzusch"atzen.
Der Atomkern hat einen Durchmesser von $10\text{fm}=10\cdot 10^{-15}\text{m}$,
das Neutron hat eine Masse von $m_n=1.67\cdot 10^{-27}\text{kg}$,
die Grundzustandsenergie ist daher $E_1=3.28\cdot10^{-13}\text{J}$.
Mit der Formel $E=mc^2$ entspr"ache dieser Energie die Masse
$3.656\cdot 10^{-30}\text{kg}$, oder etwa $0.2\%$ der Masse eines
Neutrons, und mehr als der Masse eines Elektrons.

\subsubsection{H"ohere Dimensionen}
Man kann das Problem eines Teilchens in einem Potentialkasten auch
in drei Dimensionen l"osen.
Der Hamiltonoperator ist 
\[
H=-\frac{\hbar^2}{2m}\biggl(
\frac{\partial^2}{\partial x^2}
+
\frac{\partial^2}{\partial y^2}
+
\frac{\partial^2}{\partial z^2}
\biggr)
=-\frac{\hbar^2}{2m}\Delta
\]
Man verwendet dazu einen sogenannten Seprationsansatz, man schreibt
\[
\psi(x,y,z)=X(x)\cdot Y(y)\cdot Z(z),
\]
und erh"alt die Differentialgleichung
\[
\frac{\hbar^2}{2m}\biggl(
X''(x)Y(y)Z(z)
+
X(x)Y''(y)Z(z)
+
X(x)Y(y)Z''(z)
\biggr)
=
EX(x)Y(y)Z(z)
\]
Dividiert man durch $\psi$, erhalt man die Gleichung
\begin{equation}
-\frac{\hbar^2}{2m}\frac{X''(x)}{X(x)}
-
\frac{\hbar^2}{2m}\frac{Y''(y)}{Y(y)}
-
\frac{\hbar^2}{2m}\frac{Z''(z)}{Z(z)}
=
E.
\label{skript:potentialkasten-separiert}
\end{equation}
Man kann nach jedem Term aufl"osen, und erh"alt dann eine Gleichung,
deren linke Seite nur von einer Variablen, die rechte nur von den anderen
abh"angt.
Es folgt dann, dass beide Seiten konstant sein m"ussen, 
die Gleichung (\ref{skript:potentialkasten-separiert}) beinhaltet also eigentlich
drei Gleichungen
\begin{equation}
\begin{aligned}
-\frac{\hbar^2}{2m}\frac{X''(x)}{X(x)}&=E_x&&\Rightarrow&-\frac{\hbar^2}{2m}X''&=E_xX\\
-\frac{\hbar^2}{2m}\frac{Y''(y)}{Y(y)}&=E_y&&\Rightarrow&-\frac{\hbar^2}{2m}Y''&=E_yY\\
-\frac{\hbar^2}{2m}\frac{Z''(z)}{Z(z)}&=E_z&&\Rightarrow&-\frac{\hbar^2}{2m}Z''&=E_zZ
\end{aligned}
\label{skript:potentialkasten-einzelgleichungen}
\end{equation}
mit der zus"atzlichen Bedingung $E=E_x+E_y+E_z$.
Die Gleichungen (\ref{skript:potentialkasten-einzelgleichungen}) sind drei
unabh"angige Problem wie der eindimensionale Potentialkasten
(\ref{skript:schroedingerkasten}), deren Energieniveaus wir bereits berechnet
haben.
Daraus k"onnen wir unmittelbar die Energieniveaus im dredimensionalen
Fall ableiten:
\begin{equation}
E=\frac{h^2}{32ml^2}(n_x^2+n_y^2+n_z^2)
\label{skript:3dzustaende}
\end{equation}
mit $n_x,n_y,n_z>1$.

