\subsection{Potentialtopf\label{subsection:potentialtopf}}
\index{Potentialtopf}
Aus der L"osung des Problems eines Teilchens in einem Potentialkasten
k"onnte der Eindruck entstehen, dass die Quantisierung der Energieniveaus
mit der Randbedingung am Rande des Kastens zu tun hat.
Sobald diese Bedingung wegf"allt, k"onnten die Bedingungen nicht mehr
ausreichend streng sein, um diskrete Energieniveaus zu erzwingen.
Dass dem nicht so ist zeigt das hier zu behandelnde Beispiel eines
Teilchens in einem Potentialtopf, wie auch der sp"ater im
Kapitel~\ref{chapter:harmonisch} behandelte harmonische Oszillator.

\subsubsection{Problemstellung}
\begin{figure}
\centering
\includegraphics{graphics/potential-2.pdf}
\caption{Potentialtopf der Breite $2l$ und Tiefe $-V_0$.
\label{skript:potentialtopf}}
\end{figure}
Wir m"ochten die Schr"odingergleichung l"osen f"ur ein Teilchen, welches
dem in Abbildung~\ref{skript:potentialtopf} festgehalten wird.
Das Potential ist
\[
V(x)
=
\begin{cases}
-V_0&\qquad|x|<l\\
0&\qquad\text{sonst}
\end{cases}
\]
Die Schr"odingergleichung lautet
\[
-\frac{\hbar^2}{2m}\psi''(x)+V(x)\psi(x)=E\psi(x).
\]
Die L"osungsfunktion $\psi(x)$ muss also je nach Teilinterval
verschiedene Differentialgleichungen l"osen:
\begin{equation}
\begin{aligned}
-\frac{\hbar^2}{2m}\psi''(x)&=(V_0+E)\psi(x)&\text{f"ur $|x|\le l$}
\\
-\frac{\hbar^2}{2m}\psi''(x)&=E\psi(x)      &\text{f"ur $|x|>l$}
\end{aligned}
\label{skript:potentialtopf-gleichungen}
\end{equation}
Wir werden die Differentialgleichung in jedem dieser Teilintervalle
separat l"osen, und dann zu einer Funktion $\psi$ zusammensetzen.
Die Funktion $\psi(x)$ muss nat"urlich stetig sein, die einzelnen
Teile m"ussen also an den Stellen $x=\pm l$ ohne Sprung ineinander
"ubergehen.
Dasselbe muss f"ur die Ableitungen gelten, denn wenn $\psi'$  nicht
stetig w"are, w"urde die zweite Ableitung unendlich gross, was nicht
mit der Schr"odingergleichung vereinbar ist.

\subsubsection{L"osungen von (\ref{skript:potentialtopf-gleichungen})}
Die Gleichungen sind von der Form
\begin{equation}
\psi''(x)=C\psi(x),\qquad
C=-\frac{2m}{\hbar^2}(V_0+E)
\quad
\text{oder}
\quad
C=-\frac{2m}{\hbar^2}E,
\label{skript:potentialtopf-proto}
\end{equation}
die wir mit dem Standardverfahren f"ur lineare Differentialgleichungen
l"osen k"onnen.
Die charakteristische  Gleichung ist $\lambda^2=C$, mit Nullstellen
$\lambda=\sqrt{C}$. Je nach Vorzeichen von $C$ ist der Charakter der
L"osungen verschieden.
F"ur $C>0$ sind die $e^{x\sqrt{C}}$ und $e^{-x\sqrt{C}}$.

F"ur $C<0$ sind die L"osungen der charakterisischen Gleichung imagin"ar.
Man kann die L"osungen der Differentialgleichung mit Exponentialfunktionen
$e^{ix\sqrt{-C}}$ und $e^{-ix\sqrt{-C}}$ schreiben.
\index{Spiegelung}
Es wird aber viel einfacher, wenn man wieder die Symmetrieeigenschaften
des Hamilton-Operators unter Spiegelung
verwendet, und nach geraden und ungeraden L"osungsfunktionen sucht,
die sich mit $\cos x\sqrt{-C}$ bzw.~$\sin x\sqrt{-C}$ schreiben
lassen m"ussen.

\subsubsection{F"alle $E>0$ und $E<-V_0$}
F"ur $E>0$ sind die L"osungen "uberall Schwingungsl"osungen, deren
Amplitude auf der ganzen reellen Achse ausserhalb des Topfes
konstant ist. Eine solche Funktion ist nicht normierbar, insbesondere
kann es f"ur $E>0$ also keine L"osungen geben.

F"ur $E<-V_0$ ist das $C>0$ in (\ref{skript:potentialtopf-proto}), es kommen
also nur die Exponentialfunktionen f"ur eine L"osung in Frage.
Damit l"asst sich aber keine normierbare L"osung bauen: rechts von $l$
d"urfte man nur $e^{-x\sqrt{C}}$ verwenden, links von $-l$ nur
$e^{-x\sqrt{C}}$, und dazwischen m"usste man ein Funktion konstruieren,
die an den R"andern stetig differenzierbar in die beiden abfallenden
Exponentialfunktionen "ubergeht.
Dies scheitert am gleichen Gleichungssystem wie beim Potentialkasten
im Falle $E<0$.

\subsubsection{Der Fall $-V_0 < E < 0$}
In diesem Fall suchen wir jetzt eine gerade oder ungerade L"osung, die
im Interval $[-l,l]$ eine Schwingungsl"osung $a \cos kx$ oder $a \sin kx$
ist mit 
\[
k=\sqrt{\frac{2m}{\hbar^2}(V_0+E)},
\]
und f"ur $x>l$ eine exponentiell abfallende L"osung $be^{k'(x-l)}$
mit
\[
k'=\sqrt{-\frac{2m}{\hbar^2}E}.
\]
Wegen der Symmetrie reicht es jetzt, die Bedingungen nur noch beim
Punkt $l$ aufzustellen.
Die Stetigkeitsbedingungen lauten 
\begin{align*}
&&\psi(-x)&=\psi(x)	&	\psi(-x)&=-\psi(x)\\
&\text{Stetigkeit von $\psi$ bei $x=l$:}&
	a\cos kl&= b	&	a\sin kl&=b \\
&\text{Stetigkeit von $\psi'$ bei $x=l$:}&
	-ak\sin kl&=-bk'&     ak\cos kl&=-bk'
\end{align*}
Dividiert man die beiden Gleichungen, erhalten wir zwei
Gleichungen, die nur noch $k$ und $k'$ enthalten:
\begin{align}
\tan kl&=\frac{k'}{k},
&-\cot kl&=\frac{k'}{k},
\label{skript:potentialtopf-k-gleichungen}
\end{align}
Die Gr"ossen $k$ und $k'$ h"angen aber beide von $E$ ab, dies sind
als in Wirklichkeit Gleichungen f"ur $E$.

\begin{figure}
\centering
\includegraphics[width=\hsize]{graphics/potential-4.pdf}
\caption{L"osungen der Gleichungen (\ref{skript:potentialtopf-xigleichungen}).
Rote Kurve: $\sqrt{A^2-\xi^2}/\xi$, blaue Kurven: $\tan\xi$, gr"une
Kurven $-\cot\xi$.
\label{skript:loesungen-xigleichungen}}
\end{figure}%
\begin{figure}
\centering
\includegraphics{graphics/potential-5.pdf}
\caption{Wellenfunktionen f"ur Teilchen in einem Potentialtopf der
Tiefe $-V_0$ und der Breite $2l$. Nur die sechs Wellenfunktionen 
geringster Energie sind dargestellt, und die Energie "uber dem
Grund des Potentialtopfes ist 15fach "uberh"oht.
\label{skript:potentialtopf-loesungen-klein}}
\end{figure}
\begin{figure}
\centering
\includegraphics{graphics/potential-3.pdf}
\caption{Wellenfunktionen f"ur Teilchen in einem Potentialtopf der
Tiefe $-V_0$ under der Breite $2l$. Die Wellenfunktionen zu den sechs
untersten Energieniveaus sind nicht dargestellt, siehe dazu die
Abbildung~\ref{skript:potentialtopf-loesungen-klein}.
\label{skript:potentialtopf-loesungen}}
\end{figure}
Wir versuchen, alle Gr"ossen durch die dimensionslose Gr"osse $\xi=kl$
auszudr"ucken.
Zun"achst ist
\begin{align}
\xi=kl&=l\sqrt{\frac{2m}{\hbar^2}(V_0+E)}
\notag
\\
\xi^2
&=
\frac{2ml^2}{\hbar^2}(V_0+E)
=
\frac{2ml^2V_0}{\hbar^2} +\frac{2ml^2}{\hbar^2}E
=
\frac{2ml^2V_0}{\hbar^2}-l^2k'^2
\notag
\\
k'l
&=
\sqrt{\frac{2ml^2V_0}{\hbar^2} - \xi^2}
\notag
\\
E
&=
-\frac{\hbar^2}{2m}k'^2
=
-\frac{\hbar^2}{2m}
\biggl(
\frac{2mV_0}{\hbar^2} - \frac{\xi^2}{l^2}
\biggr)
=
-V_0+\frac{\hbar^2\xi^2}{2ml^2}
\label{skript:potentialtopf-energie}
\end{align}
Schreiben wir
\[
A=\sqrt{\frac{2ml^2V_0}{\hbar^2}}
\]
und setzen wir ein in die Gleichungen (\ref{skript:potentialtopf-k-gleichungen}),
erhalten wir die neuen Gleichungen f"ur $\xi$
\begin{align}
\tan \xi&=\frac{\sqrt{A^2-\xi^2}}{\xi}
&
-\cot \xi&=\frac{\sqrt{A^2-\xi^2}}{\xi}
\label{skript:potentialtopf-xigleichungen}
\end{align}
In Abbildung~\ref{skript:loesungen-xigleichungen} sind die rechten und linken
Seiten der Gleichungen (\ref{skript:potentialtopf-xigleichungen}) aufgezeichnet,
die Schnittpunkte geh"oren zu $\xi$-Werten, f"ur die es L"osungen
der Schr"odingergleichung gibt.
Die zugeh"origen Energien k"onnen mit (\ref{skript:potentialtopf-energie})
berechnet werden.

Die Gleichungen (\ref{skript:potentialtopf-xigleichungen}) k"onnen nicht 
in geschlossener Form gel"ost werden, die Energieniveaus k"onnen also
nur numerisch bestimmt werden.
Man kann immerhin ein paar qualitative Aussagen machen.
Aus Abbildung~\ref{skript:loesungen-xigleichungen} kann man ablesen, dass die
$\xi$-Werte f"ur die Niveaus mit niedrigester Energie nahe an $n\pi/2$
sind, die Energien sind also nahe bei
\[
-V_0+\frac{\hbar^2\xi^2}{2ml^2}
\simeq
-V_0+\frac{h^2n^2\pi^2}{32ml^2\pi^2}
=
-V_0+\frac{h^2n^2}{32ml^2},
\]
die Energie dieser Niveaus liegt also "ahnlich hoch "uber dem Grund
des Potentialtopfes, wie wir f"ur einen Potentialkasten gefunden haben.

Die L"osungsfunktionen sind in den
Abbildungen~\ref{skript:potentialtopf-loesungen-klein}
und \ref{skript:potentialtopf-loesungen}
dargestellt.
Man erkennt, dass man die Teilchen mit positiver Wahrscheinlichkeit in
der Wand antreffen wird, die Eindringtiefe ist zudem umso gr"osser,
je gr"osser die Energie eines Teilchens ist.
