\chapter{Quantencomputer\label{chapter:quantencomputer}}
\lhead{Quantencomputer}
\rhead{}
Die Technologie des zwanzigsten Jahrhunderts hat automatische
Rechenmaschinen hervorgebracht, beliebige Rechnung durchf"uhren,
sofern sie die Speichergr"osse der Maschine nicht sprengen.
Sie beruhen auf einer physischen Codierung der Zahlen, mit denen
gerechnet werden soll, und einer Maschine, welche Codierungen 
in neue Codierungen umwandelt.
Die Idee einer solchen Maschine geht auf Charles Babbage zur"uck,
der in den Jahren ab 1812 auch eine Realisierung als mechanische
Maschine angestrebt hat.

Moderne Computer sind alle konkrete Realsierungen des abstrakten
Konzeptes der Turing-Maschine, welches Alan Turing formuliert hat,
um zu analysieren, welche Arten von Berechnungen in welcher Zeit
"uberhaupt durchgef"uhrt werden k"onnen.
Es wurde erkannt, dass gewisse Probleme auf Turing-Maschinen
derart lange ben"otigen w"urden, dass man sie getrost als undurchf"uhrbar
betrachten kann.
Allgemein besteht die "Uberzeugung, dass das Problem der Faktorisierung
des Produktes von grossen Primzahlen in diese Kategorie geh"ort,
auch wenn dies bisher nicht bewiesen worden konnte.
Dies impliziert, dass eine solche Faktorisierung, auf der die Sicherheit
verschiedener kryptographischer Systeme basiert, ohne zus"atzliches
Wissen nicht durchf"uhrbar ist.

Eine wesentliche Eigenschaft solcher schwieriger Problem ist, dass
sie die parallel Evaluation sehr vieler M"oglichkeiten erfordern,
was in einem klassischen Computer nur sequenziell m"oglich ist.
Der Grund daf"ur ist, dass ein Bit immer nur in genau einem
Zustand sein kann, wenn man die verschiedenen M"oglichkeiten des
Bits evaluieren will, dann muss man das hintereinander tun.

In der Quantenmechanik gibt es aber Systeme, die in zwei Zust"anden
gleichzeitig sein k"onnen, wie wir im Abschnitt~\ref{section:cat}
illustrieren werden.
Wenn man also auch das Rechenwerk einer Turingmaschine durch eine
``Quantenschaltung'' ersetzen kann, dann kann ein relativ kleiner
Quantecomputer alle M"oglichkeiten aller Bits gleichzeitig evaluieren,
und damit das Problem in sehr kurzer Zeit l"osen.


\section{Klassische Computer}
Ein klassischer Computer kann bin"ar als Spannungen in einer
elektronischen Schaltung codierte Zahlen in ander Muster von
Spannungen umwandeln, welche das Resultat einer Rechenoperation
mit den Zahlen als Input darstellt.
\begin{figure}
\centering
\begin{tabular}{cccc}
&UND&ODER&XOR\\
\\
&
\includegraphics{graphics/gatter-1.pdf}&%
\includegraphics{graphics/gatter-2.pdf}&%
\includegraphics{graphics/gatter-5.pdf}
\\
\\
&
\includegraphics{graphics/gatter-3.pdf}&%
\includegraphics{graphics/gatter-4.pdf}&%
\includegraphics{graphics/gatter-6.pdf}
\end{tabular}
\caption{Logische Gatter von links nach rechts UND, ODER, XOR,
in der unteren Reihe invertiert.
\label{gates}}
\end{figure}
Die grundlegenden logischen Operationen werden dann durch sogenannte
Gatter implementiert, deren Schaltbilder in Abbildung~\ref{gates} dargestellt
sind.
\begin{figure}
\centering
\includegraphics{graphics/gatter-8.pdf}
\caption{Halbaddierer, der Ausgang $\Sigma$ ist die Summe der beiden
Eing"ange $A$ und $B$, der Ausgang $C_\text{out}$ wird aktiv, wenn
ein "Ubertrag auftritt.
\label{halfadder}}
\end{figure}
\begin{figure}
\centering
\includegraphics{graphics/gatter-7.pdf}
\caption{Volladdierer, berechnet die Summe der beiden Inputs $A$ und $B$
und den "Ubertrag $C_\text{in}$, und gibt die Summe $\Sigma$ und den
"Ubertrag $C_\text{out}$ aus.
\label{fulladder}}
\end{figure}
Aus den Grundoperationn lassen sich komplexere Schaltungen 
aufbauen, zum Beispiel das Halbaddierwerk in Abbildung~\ref{halfadder}
oder der Volladdierer in Abbildung~\ref{fulladder}.
Die Digitaltechnik lehrt, wie man durch Kombination solcher Schaltung
beliebig komplexe Berechnungen anstellen kann.

Die Verbindungen in all diesen Schaltungen k"onnen nur in jeweils einem
Zustand sein, ein oder aus, $1$ oder $0$.
Dies ist eine Einschr"ankung der Technologie.
Selbst die Verwendung verschiedener Spannungsniveaus auf den Verbindungen
w"urde daran nichts "andern: zu jeder Zeit kann jede Verbindung nur
ein einem der m"oglichen Zust"ande sein.

Eines der schwierig zu l"osenden Probleme ist SAT, die Frage, ob eine
logische Formel durch geeignete Wahrheitsbelegung der Inputs wahr
werden kann. Implementiert man die logische Formal als Schaltung,
wird die Frage gleichbedeutend damit, ob der Ausgang der Schaltung
durch geeignete Beschaltung der Inputs in den Zustand $1$  gehen kann.
Bedingt durch die Technologie k"onnen wir die Frage nur dadurch beantworten,
dass wir alle m"oglichen Inputs durchprobieren. Bei $n$ Inputs sind
dies $2^n$ Belegungen, entsprechend lange dauert es, eine L"osung
zu finden.

\section{Schr"odingers Katze\label{section:cat}}
Dass in der Quantenmechanik ein System nicht mir in einem reinen Zustand
zu sein braucht, hat Erwin Schr"odinger mit seinem ber"uhmten 
Gedankenexperiment mit der Katze illustriert.
In einer Kiste befindet sich eine Katze und ein Mechanismus,
der beim radioaktiven Zerfall des darin befindlichen radioaktiven
Atomkernes eine Phiole mit Gift zerbricht, so dass es ausweichen und
die Katze t"oten kann.
Die Katze kann offensichtlich in zwei m"oglichen Zust"anden sein,
lebendig $|L\rangle$ oder tot $|T\rangle$. 
Zu beginn des Experiments befindet es sich im Zustand $|L\rangle$.
Dieser Zustand der Katze wiederspiegelt nat"urlich nur den Zustand
des Atomkerns, der ebenfalls in zwei Zust"anden sein kann.

Mit fortschreitender Zeit steigt die Wahrscheinlichkeit, dass der
Atomkern zerfallen ist, und die Katze tot ist.
Genauer sind die Wahrscheinlichtkeiten, die Katze in den beiden
Zust"anden zu finden
\begin{align*}
|\langle L|\psi(t)\rangle|^2
&=
2^{-t/t_{\frac12}}
&&\text{und}
&
|\langle T|\psi(t)\rangle|^2
&=
1-2^{-t/t_{\frac12}}
\end{align*}
Die Katze ist also in einem Zustand
\[
|\psi(t)\rangle = 
\sqrt{2^{-t/t_{\frac12}}}e^{i\varphi_1}\,|L\rangle
+
\sqrt{1-2^{-t/t_{\frac12}}}e^{i\varphi_2}\,|T\rangle,
\]
die beiden Phasenfaktoren tragen der Tatsache Rechnung, dass komplexe
Linearkombinationen m"oglich sind.

Ist die Katze lebendig oder tot? Die Erfahrung mit makroskopischen
Katzen sagt uns, dass ein Katze entweder das eine oder andere ist.
Die Quantenmechanik sagt uns dagegen, dass wir es nicht wissen k"onnen.
Wir k"onnen nur wissen, dass die Katze in einem "Uberlagerungszustand
ist.
Das einzige, was wir daraus ableiten k"onnen, ist mit welcher
Wahrscheinlichkeit sie bereits tot ist.
Gewissheit dar"uber, in welchem Zustand sich die Katze befindet,
erhalten wer genau in dem Moment, wo wir das Experiment durchf"uhren
und den Zustand der Katze neu ermitteln.

Durch den Prozess der Beobachtung der Katze "andert sich deren Zustand.
Lebt sie noch, wissen wir, dass sie sich im Zustand $|L\rangle$ befindet.
Ist sie tot, befindet sie sich im Zustand $|T\rangle$.
Die Beobachtung hat also den Zustand ver"andert.

Wir erweitern jetzt das urspr"ungliche Gedankenexperiment von Schr"odinger.
Wir stellen uns vor, wir m"ochten gerne eine exotische Eigenschaft von
Katzen messen, insbesondere m"ochten wir wissen eine lebende Katze oder
eine tote Katze diese Eigenschaft hat. Das klassische Experiment
daf"ur w"are, je eine lebendige und eine tote Katze zu pr"aparieren,
und dann den Test f"ur die Eigenschaft auf beide anzuwenden.

F"ur das quantenmechanische Experiment brauchen wir offenbar einen
Operator, zwei m"ogliche Ausgangszust"ande hat: die Katze mit der
Eigenschaft $|Y\rangle$ und die Katze ohne die Eigenschaft $|N\rangle$.
Das zugeh"orige Experiment k"onnte man mit der Notation aus
Kapitel~\ref{chapter:einfache-quantensysteme} als
\begin{center}
\includegraphics{graphics/gatter-9.pdf}
\end{center}
darstellen. 

F"ur Schr"odingers Katze k"onnen wir die Frage, ob die Eigenschaft $Q$
in f"ur lebende oder tote Katzen vorhanden ist, jin einem einzigen
Durchgang beantworten.
Dazu stellen wir zuerst eine Katze in einem "Uberlagerungszustand
$|\psi\rangle$ von $|L\rangle$ und $|T\rangle$ her.
Auf diesen Zustand wenden wir den Operator $Q$ an, und testen dann,
ob es Katzen gibt, die sich im Zustand $|Y\rangle$ befindet, indem
wir $\langle Y|\,Q\,|\psi\rangle$ messen.
Falls wir herausfinden, dass $\langle Y|\,Q\,|\psi\rangle\ne 0$,
dann kann eine Katze die Eigenschaft haben, auch wenn wir noch nicht
wissen, ob es eine Eigenschaft von toten oder lebenden Katzen ist.

Schr"odingers Katze in ihrem "Uberlagerungszustand kann also dazu
verwendet werden, einen Quantencomputer zu bauen, der Probleme "uber
Katzeneigenschaften l"osen kann.
Im Gegensatz zu einem klassischen Computer k"onnen wir aber nicht
erwarten, dass wir in einer einzigen Messung das Resultat der
Berechnung bekommen k"onnen.
Das Quantensystem kann nur eine Wahrscheinlichkeitsaussage liefern,
und Messungen der Wahrscheinlichkeit verlangen immer eine grosse zahl
von Experimenten.

\section{Quantencomputer}
Nach dem einf"uhrenden Beispiel "uber den Schr"odingerschen
Katzen-Quanten-Computer k"onnen wir uns jetzt dar"uber Gedanken
machen, wie ein Quanten-Computer aussehen m"usste, der allgemeine
mathematische Probleme l"osen k"onnte.

Ein Quantencomputer muss zun"achst einen geeignet komplexen
"Uberlagerungszustand pr"aprieren, den Q-bits. Dann muss dieser Zustand
von einem oder mehreren Quanten-Gattern verarbeitet werden, sie
entsprechen dem Operator $Q$ im Schr"odinger-Katzen-Computer.
Schliesslich muss man das Experiment mehrmals durchf"uhren, bis man
Wahrscheinlichkeiten mit gen"ugend grosser Genauigkeit bestimmt hat,
dass man ein Resultat der Berechnung daraus ableiten kann.

\subsection{Q-bits}
Ein quantenmechanisches System mit zwei m"oglichen Zust"ande $|0\rangle$
und $|1\rangle$ muss sich nicht
notwendigerweise in genau einem der Zust"ande befinden.
Jede Linearkombination der beiden Zust"ande ist ebenfalls ein g"ultiger
Zustand, sofern sie als Vektor L"ange $1$ hat.
Der Zustand
\[
|\psi\rangle
=
\lambda \,|0\rangle + \mu\,|1\rangle
\]
hat L"ange $1$ wenn gilt
\[
\langle\psi|\psi\rangle
=
(
\bar\lambda
\langle 0|
+
\bar\mu
\langle 1|
)
(
\lambda \,|0\rangle + \mu\,|1\rangle
)
=
|\lambda|^2\langle 0|0\rangle + |\mu|^2\langle 1|1\rangle
=
|\lambda|^2+|\mu|^2
=
1
,
\]
da die gemischten Terme wegen $\langle 0|1\rangle=0$ wegfallen.
Es gen"ugt also, dass die Quadratsumme der Betr"age von $\lambda$ und $\mu$
den Wert $1$ ergibt.
Insbesondere gibt es selbst f"ur dieses einfache System bereits viel
mehr m"ogliche Zust"ande als bei einem klassischen Bit.
Ein solches quantenmechanisches System nennt man in Q-bit.

Ein klassisches System mit zwei Bits kann in 4 m"oglichen Zust"anden sein.
Das zugeh"orige quantenmechanische System kann in einer beliebigen
"Uberlagerung der vier m"oglichen reinen Zust"ande sein:
\[
|\psi\rangle
=
\alpha_{00}|00\rangle
+
\alpha_{01}|01\rangle
+
\alpha_{10}|10\rangle
+
\alpha_{11}|11\rangle
,\qquad
|\alpha_{00}|^2
+
|\alpha_{01}|^2
+
|\alpha_{10}|^2
+
|\alpha_{11}|^2
=1.
\]


\subsection{Gatter}

\subsection{Quanten-Turing-Maschinen}

\subsection{Probabilistische Algorithmen und Quantencomputer}
