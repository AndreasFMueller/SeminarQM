\chapter{Quantisierung\label{chapter:quantisierung}}
\lhead{Quantisierung}
\rhead{}

\section{Quantisierungsregeln}
\rhead{Quantisierungsregeln}
\index{Quantisierungsregeln}
In diesem Abschnitt untersuchen wir, wie ein klassisches System, dessen
Beschreibung wir bereits kennen, in ein quantenmechanisches System 
mit dem Formalismus aus Kapitel~\ref{chapter:einfache-quantensysteme}
umgeformt werden kann.
Dabei m"ussen alle klassisch beobachtbaren Gr"ossen, also alle Obervablen
wie Position, Impuls, Energie etc.~erhalten erhalten bleiben.

In der Bohrschen Quantenmechanik wurde gefordert, dass 
sich Teilchen auf Bahnen bewegen m"ussen, f"ur die eine
Quantisierungsbedingung gilt.
Typischerweise musste das Produkt $p\dot q$ entlang einer geschlossenen
Bahn aufintegriert ein ganzzahliges Vielfaches von $\hbar$ ergeben.
Dieses Quantisierungsprinzip geht davon aus, dass man immer noch sinnvoll
von einer Bahn sprechen kann, was wir aber schon ausgeschlossen haben.
Wir m"ussen daher ein Quantisierungsprinzip verwenden, welches den
Begriff einer Bahn gar nicht erst verwendet, sondern nur Dinge, die
auch tats"achlich beobachtbar sind, also Observable.

In der klassischen Mechanik sind Observable einfach Funktionen
der Koordinaten und Impulse,
in der Quantenmechanik werden sie durch selbstadjungierte
Operatoren dargestellt.
Um die klassischen Observablen von den Operatoren unterscheiden zu
k"onnen, werden wir in diesem Abschnitt die Operatoren mit einem
Circumflex kennzeichnen.
Der Hamilton-Operator wird also mit $\hat H$ bezeichnet, 
der Operator zur klassischen Variable $X$ wird mit $\hat X$ etc.

\subsection{Zeitentwicklung}
In Abschnitt~\ref{section:zeitentwicklung-von-observablen} haben wir die
Zeitentwicklung eines Quantensystems kennengelernt.
Wir nehmen an, dass der Hamilton-Operator $H$ nicht von der Zeit abh"angt.
Sei $\hat A$ eine Observable, und sei $|\psi(t)\rangle$ der Zustand des
Quantensystems zur Zeit $t$.
Dann gilt
\[
\frac{d}{dt}\langle \hat A\rangle
=
\biggl\langle -\frac{i}{\hbar}[\hat A,\hat H] \biggr\rangle.
\]

Als Spezialfall muss darin die Bewegung eines klassischen mechanischen Systems
enthalten sein, wie wir es in Kapitel~\ref{chapter:mechanik}
kennengelernt haben.
Ein solches System zeichnet sich dadurch aus, dass die Wellenfunktion
$\Psi(x,t)=\langle x|\psi(t)\rangle$
sehr gut lokalisiert ist.
Nach de~Broglie werden Quantenph"anomene nur auf einer Entfernung sichtbar,
die der de~Broglie-Wellenl"ange $\lambda=h/p$ des Teilchens entsprechen.
Weiter als einige de~Broglie-Wellenl"angen von $\langle x\rangle$
entfernt wird $\Psi$ verschwinden.

Eine Observable $A$ des klassischen Systems ist eine Funktion der
Koordinaten und Impulse, man kann ihre Zeitenticklung mit den
Bewegungsgleichungen~\ref{skript:zeitentwicklung-poisson}
berechnen:
\[
\frac{d}{dt}A=(A,H).
\]
Die Quantisierungsregel lautet daher:

\begin{satz}
\label{skript:quantisierung-poisson}
Ein klassisches System mit Hamilton-Funktion $H$ wird quantisiert, indem
jede Observable $A$ durch einen Operator $\hat A$ ersetzt wird, die
Hamilton-Funktion durch einen Hamilton-Operator $\hat H$, und jede
Poisson-Klammer durch einen Kommutator nach der Regel
\begin{equation}
(A,B)
\qquad
\mapsto
\qquad
-\frac{i}{\hbar}[\hat A,\hat B].
\label{skript:quantisierung-observable}
\end{equation}
\end{satz}
Aus den Regeln folgt auch, dass der Operator $\hat H$ tats"achlich
der Operator ist, der der Observablen $H$ entspricht, also der 
Energie des Systems.
\label{skript:hamilton-op-ist-energie}

\subsection{Ortsdarstellung}
Klassische Systeme werden in Koordinaten und Impulsen beschrieben, 
die nat"urlich auch Observable sind.
Ausserdem k"onnen wir den Zustandsvektor in der Ortsdarstellung
ausdr"ucken, als Wellenfunktion $\Psi(x,t)=\langle x|\psi(t)\rangle$
Es ist klar, dass der Operator, der zur Observablen $x$ geh"ort,
die Multiplikation mit $x$ ist:
\[
x
=
\sum_x |x\rangle\,x\,\langle x|
\qquad\Rightarrow\qquad
\langle x\rangle
=
\int x|\Psi(x,t)|^2\,dx.
\]
Doch wie muss man den Operator f"ur den Impuls ausdr"ucken?

Klassisch gilt f"ur jede Observable $G$
\begin{equation}
(p,G)=\frac{\partial G}{\partial x}.
\label{skript:pGobservable}
\end{equation}
Wenn $G$ nicht von $a$ abh"angt, wir die Poisson-Klammer also $0$ sein,
und auch der zugeh"orige Quantenmechanische Operator wird verschwinden.
Interessant sind also nur Observable, die nur von $x$ abh"angen,
aber wir wissen bereits, dass wir diese mit Funktionen von $x$
identifizieren k"onnen.
Die Quantisierungsvorschrift~\ref{skript:quantisierung-observable}
macht aus~(\ref{skript:pGobservable})
\[
-\frac{i}{\hbar}[\hat p,\hat G]
=
\widehat{\frac{\partial G}{\partial q}}
=
\frac{\partial G}{\partial q}.
\]
Will man davon einen Erwartungswert berechnen, erh"alt man
\begin{align*}
\biggl\langle
\widehat{\frac{\partial G}{\partial x}}
\biggr\rangle
&=\int \bar\Psi(x,t)\frac{\partial G}{\partial x}\Psi(x,t)\,dx.
\end{align*}
Diesen Ausdruck kann man mit partieller Integration noch etwas
vereinfachen:
\begin{align*}
\biggl\langle
\widehat{\frac{\partial G}{\partial x}}
\biggr\rangle
&=
\biggl[
\bar\Psi(x,t)\frac{\partial G}{\partial x}\Psi(x,t)
\biggr]_{-\infty}^\infty
-
\int_{-\infty}^{\infty}
\frac{\partial}{\partial x}
\bar\Psi(x,t) G
\Psi(x,t)
+
\bar\Psi(x,t) G
\frac{\partial}{\partial x}
\Psi(x,t)
\,dx.
\end{align*}
Dies stimmt genau dann mit dem Kommutator "uberein, wenn wir setzen
\[
\hat p = \frac{\hbar}{i}\frac{\partial}{\partial x},
\]
Wir bekommen damit die vereinfachten Quantisierungsregeln in der
Ortsdarstellung:
\begin{satz}
\label{skript:quantisierungsregeln-ortsdarstellung}
Ein mechanisches System wird in der Ortsdarstellung quantisiert,
indem als Impulsoperatoren die Differentialoperatoren
\[
\hat p=\frac{\hbar}{i}\frac{\partial}{\partial x}
\]
verwendet werden. Den Hamilton-Operator $\hat H$ erh"alt man, indem man
in der Hamilton-Funktion $H$ die Variablen $p$ durch die Operatoren 
$\hat p$ ersetzt.
\end{satz}

\section{Schr"odingergleichung}
\rhead{Schr"odingergleichung}
In diesem Abschnitt leiten wir aus den Quantisierungsregeln die
Schr"odingergleichung her, welche uns erm"oglicht, einfache
Quantensysteme mit Differentialgleichungen zu modellieren
und zum Beispiel die Energieniveaus und Wellenfunktionen direkt
zu berechnen.

\subsection{Schr"odingergleichung in Ortsdarstellung}
\index{schrodingergleichung@Schr\"odingergleichung!in Ortsdarstellung}
Wir wenden die vereinfachten Quantisierungsregeln auf ein
Teilchen in einem Potential an.
Dieses hat die Hamilton-Funktion
\[
H=\frac1{2m}p^2+V(x).
\]
Nach den Quantisierungsregeln in
Satz~\ref{skript:quantisierungsregeln-ortsdarstellung} wird daraus
der Hamilton-Operator:
\[
\hat H
=
\frac{1}{2m}\biggl(-\frac{\hbar}{i}\frac{\partial }{\partial x}\biggr)^2
+V(x)
=
-\frac{\hbar^2}{2m}\frac{\partial^2}{\partial x^2}+V(x).
\]
In drei Dimension treten entsprechend mehr Ableitungen auf:
\[
H=\frac1{2m}\sum_{k=1}^3p_i^2+V(x)
\qquad\Rightarrow\qquad
\hat H
=
-\frac{\hbar^2}{2m}\sum_{k=1}^3\frac{\partial^2}{\partial x_i^2}+V(x)
=
-\frac{\hbar^2}{2m}\Delta + V(x).
\]
Die zugeh"orige zeitabh"angige Schr"odingergleichung ist
\[
i\hbar\frac{\partial}{\partial x}\psi(x,t)
=
-\frac{\hbar^2}{2m}\Delta\Psi(x,t) + V(x)\Psi(x,t).
\]
Ein Teilchen mit Energie $E$ in einem Potential $V(x)$ wird entsprechend
durch eine Wellenfunktion $\Psi(x)$ beschrieben, die zeitunabh"angige
Schr"odingergleichung
\[
-\frac{\hbar^2}{2m}\Delta\Psi(x,t) + V(x)\Psi(x)
=
E\Psi(x)
\]
erf"ullt.
Ausserdem m"ussen die Wellenfunktionen nat"urlich normiert sein.

Damit lassen sich jetzt einfache Potential-Probleme l"osen.
Wir untersuchen im folgenden die Bewegung eines Teilchens in einfachen
Potentialen, um erste Erfahrungen mit der Schr"odingergleichung
zu sammeln und einen ersten Eindruck von den Eigenheiten der
des quantenmechanischen Verhaltens von Teilchen zu erhalten.

\input potentialkasten.tex
\input potentialtopf.tex

\section{Wahrscheinlichkeitsstrom}
% Kontinuitätsgleichung für die Wahrscheinlichkeitsdichte <psi(x)|psi(x)>
Die Wahrscheinlichkeit ein Teilchen in einer Umgebung des
Punktes $x$ zu finden, ist $\varrho(x)=\langle \psi(x)|\psi(x)\rangle$.
Die Zeitentwicklung f"uhrt dazu, dass $\varrho$ auch von der
Zeit abh"angt.
Dies gesamte Wahrscheinlichkeit muss nat"urlichh erhalten bleiben,
das Teilchen kann ja nicht einfach verschwinden.

Wenn sich mit der Zeitentwicklung das Teilchen woanders hin bewegt,
dann ist damit ein Stromdichtevektor verbunden, der angibt, 
wieviel Wahrscheinlichkeit durch eine Fl"ache fliesst.
Das Ziel dieses Abschnittes ist, eine solche Stromdichte zu
definieren.

\subsection{Kontinuit"atsgleichung}
\index{Kontinuit\"atsgleichung}
Wir stellen uns ein Medium mit Dichte $\varrho(x,t)$ vor.
Das Medium hat im Punkt $x$ die Str"omungsgeschwindigkeit $v(x)$.
In einem Zeitinterval $\Delta t$ nimmt die Masse des Mediums
in einem Interval der L"ange $\Delta x$ um den Betrag
\[
\Delta x\frac{\partial\varrho}{\partial t}\Delta t
\]
zu.
Dies Zunahme muss dadurch erfolgen, dass durch die Endpunkte
des $\Delta x$-Intervals mehr Material zu- als abfliesst.
Der Zufluss am linken Ende des Intervals ist
$
\varrho(x) v(x),
$
der Abfluss am rechten Ende ist $\varrho(x+\Delta x)v(x+\Delta x)$.
Die Bilanz ist
\[
\Delta x\frac{\partial\varrho}{\partial t}\Delta t
=
\Delta t(
\varrho(x) v(x)
-
\varrho(x+\Delta x) v(x+\Delta x)
)
\]
Wir teilen durch $\Delta x\,\Delta y$ und lassen $\Delta x$ gegen 0 gehen:
\begin{equation}
\frac{\partial\varrho}{\partial t}
+\frac{\partial}{\partial x}(\varrho(x)v(x))
=0.
\label{skript:kontinuitaetsgleichung1d}
\end{equation}
Die Gr"osse $j(x)=\varrho(x)v(x)$ beschreibt den Materialstrom.
Die Gleichung (\ref{skript:kontinuitaetsgleichung1d}) heisst die
Kontinuit"atsgleichung.
Sie dr"uckt aus, dass im Verlauf der Str"omung kein Material verlorgen
gehen kann.

In drei Dimensionen kann man ebenfalls ein Kontinuit"atsgleichung
f"ur die Dichte $\varrho(x)$ und den Strom $\vec\j(x)=\varrho(x) \vec v(x)$
definieren, die die dreidimensionale 
Kontinuit"atsgleichung
\[
\frac{\partial\varrho}{\partial x}
+
\frac{\partial j_1}{\partial x_1}
+
\frac{\partial j_2}{\partial x_2}
+
\frac{\partial j_3}{\partial x_3}
=
\frac{\partial\varrho}{\partial t}+\operatorname{div}\vec\j
=0
\]
erf"ullt.

\subsection{Wahrscheinlichkeitsstrom}
Wir suche jetzt einen Wahrscheinlichkeitsstrom, der zusammen mit
der Wahrscheinlichkeitsdichte $|\psi(x)|^2$ eine Kontinuit"atsgleichung
erf"ullt.
\begin{align*}
\frac{\partial\varrho(x,t)}{\partial t}
&=
\frac{\partial}{\partial t} \varrho(x)
=
\frac{\partial\varrho(x)}{\partial t}
=
\frac{\partial\overline{\psi(x,t)}}{\partial t}\psi(x,t)
+
\overline{\psi}(x,t)\frac{\partial\psi(x,t)}{\partial t}
\end{align*}
Im letzten Term k"onnen wir die Zeitableitungen durch die
Schr"odingergleichung ersetzen:
\begin{align*}
\frac{\partial\varrho(x,t)}{\partial t}
&=
\frac{\partial\overline{\psi(x,t)}}{\partial t}\psi(x,t)
+
\overline{\psi}(x,t)\frac{\partial\psi(x,t)}{\partial t}
\\
&=
\overline{ \frac{i}{\hbar}H \psi(x,t) }\psi(x,t)
+
\overline{\psi}(x,t)\frac{i}{\hbar} H \psi(x,t)
\\
&=
-
\frac{\hbar^2}{2m}\overline{\frac{\partial^2\psi(x,t)}{\partial x^2}}\psi(x,t)
+
\frac{\hbar^2}{2m}\frac{\partial^2\psi(x,t)}{\partial x^2}\overline{\psi(x,t)}
+
\varphi(x,t)|\psi(x,t)|^2
\end{align*}
Die ersten zwei Terme k"onnen wir auch als Ableitung
der Funktion
\[
j(x,t)=
\frac{\hbar^2}{2m}
\biggl(
-\frac{\partial\overline{\psi}(x,t)}{\partial x}\psi(x,t)
+\overline{\psi}(x,t)\frac{\partial\psi(x,t)}{\partial x}
\biggr)
\]
erhalten, denn
\begin{align*}
\frac{\partial j(x,t)}{\partial x}
&=
\frac{\hbar^2}{2m}
\frac{\partial}{\partial x}
\biggl(
-\frac{\partial\overline{\psi}(x,t)}{\partial x}\psi(x,t)
+\overline{\psi}(x,t)\frac{\partial\psi(x,t)}{\partial x}
\biggr)
\\
&=
\frac{\hbar^2}{2m}
\frac{\partial}{\partial x}
\biggl(
-
\frac{\partial^2\overline{\psi}(x,t)}{\partial x^2}
\psi(x,t)
-
\frac{\partial\overline{\psi}(x,t)}{\partial x}
\frac{\partial\psi(x,t)}{\partial t}
+
\frac{\partial\overline{\psi}(x,t)}{\partial x}
\frac{\partial\psi(x,t)}{\partial x}
+
\overline{\psi}(x,t)
\frac{\partial^2\psi(x,t)}{\partial x^2}
\biggr)
\\
&=
\frac{\hbar^2}{2m}
\frac{\partial}{\partial x}
\biggl(
-
\frac{\partial^2\overline{\psi}(x,t)}{\partial x^2}
\psi(x,t)
+
\overline{\psi}(x,t)
\frac{\partial^2\psi(x,t)}{\partial x^2}
\biggr)
\end{align*}
Somit erf"ullt die oben definierte Funktion $j(x,t)$ die
Kontinuit"atsgleichung
\[
\frac{\partial\varrho(x,t)}{\partial t}
=
\frac{\partial j(x,t)}{\partial x} +\varphi(x,t)\varrho(x,t).
\]





