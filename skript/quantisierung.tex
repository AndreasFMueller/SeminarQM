\chapter{Quantisierung\label{chapter:quantisierung}}
\lhead{Quantisierung}
\rhead{}

\section{Quantisierungsregeln}
% Operatoren in Ortsdarstellung
% Ebene Wellen und Fourier Transformation
\section{Schr"odingergleichung}
% Schrödingergleichung in Ortsdarstellung
\index{schrodingergleichung@Schr\"odingergleichung!in Ortsdarstellung}

\input potentialkasten.tex
\input potentialtopf.tex

\section{Wahrscheinlichkeitsstrom}
% Kontinuitätsgleichung für die Wahrscheinlichkeitsdichte <psi(x)|psi(x)>
Die Wahrscheinlichkeit ein Teilchen in einer Umgebung des
Punktes $x$ zu finden, ist $\varrho(x)=\langle \psi(x)|\psi(x)\rangle$.
Die Zeitentwicklung f"uhrt dazu, dass $\varrho$ auch von der
Zeit abh"angt.
Dies gesamte Wahrscheinlichkeit muss nat"urlichh erhalten bleiben,
das Teilchen kann ja nicht einfach verschwinden.

Wenn sich mit der Zeitentwicklung das Teilchen woanders hin bewegt,
dann ist damit ein Stromdichtevektor verbunden, der angibt, 
wieviel Wahrscheinlichkeit durch eine Fl"ache fliesst.
Das Ziel dieses Abschnittes ist, eine solche Stromdichte zu
definieren.

\subsection{Kontinuit"atsgleichung}
\index{Kontinuit\"atsgleichung}
Wir stellen uns ein Medium mit Dichte $\varrho(x,t)$ vor.
Das Medium hat im Punkt $x$ die Str"omungsgeschwindigkeit $v(x)$.
In einem Zeitinterval $\Delta t$ nimmt die Masse des Mediums
in einem Interval der L"ange $\Delta x$ um den Betrag
\[
\Delta x\frac{\partial\varrho}{\partial t}\Delta t
\]
zu.
Dies Zunahme muss dadurch erfolgen, dass durch die Endpunkte
des $\Delta x$-Intervals mehr Material zu- als abfliesst.
Der Zufluss am linken Ende des Intervals ist
$
\varrho(x) v(x),
$
der Abfluss am rechten Ende ist $\varrho(x+\Delta x)v(x+\Delta x)$.
Die Bilanz ist
\[
\Delta x\frac{\partial\varrho}{\partial t}\Delta t
=
\Delta t(
\varrho(x) v(x)
-
\varrho(x+\Delta x) v(x+\Delta x)
)
\]
Wir teilen durch $\Delta x\,\Delta y$ und lassen $\Delta x$ gegen 0 gehen:
\begin{equation}
\frac{\partial\varrho}{\partial t}
+\frac{\partial}{\partial x}(\varrho(x)v(x))
=0.
\label{skript:kontinuitaetsgleichung1d}
\end{equation}
Die Gr"osse $j(x)=\varrho(x)v(x)$ beschreibt den Materialstrom.
Die Gleichung (\ref{skript:kontinuitaetsgleichung1d}) heisst die
Kontinuit"atsgleichung.
Sie dr"uckt aus, dass im Verlauf der Str"omung kein Material verlorgen
gehen kann.

In drei Dimensionen kann man ebenfalls ein Kontinuit"atsgleichung
f"ur die Dichte $\varrho(x)$ und den Strom $\vec\j(x)=\varrho(x) \vec v(x)$
definieren, die die dreidimensionale 
Kontinuit"atsgleichung
\[
\frac{\partial\varrho}{\partial x}
+
\frac{\partial j_1}{\partial x_1}
+
\frac{\partial j_2}{\partial x_2}
+
\frac{\partial j_3}{\partial x_3}
=
\frac{\partial\varrho}{\partial t}+\operatorname{div}\vec\j
=0
\]
erf"ullt.

\subsection{Wahrscheinlichkeitsstrom}
Wir suche jetzt einen Wahrscheinlichkeitsstrom, der zusammen mit
der Wahrscheinlichkeitsdichte $|\psi(x)|^2$ eine Kontinuit"atsgleichung
erf"ullt.
\begin{align*}
\frac{\partial\varrho(x,t)}{\partial t}
&=
\frac{\partial}{\partial t} \varrho(x)
=
\frac{\partial\varrho(x)}{\partial t}
=
\frac{\partial\overline{\psi(x,t)}}{\partial t}\psi(x,t)
+
\overline{\psi}(x,t)\frac{\partial\psi(x,t)}{\partial t}
\end{align*}
Im letzten Term k"onnen wir die Zeitableitungen durch die
Schr"odingergleichung ersetzen:
\begin{align*}
\frac{\partial\varrho(x,t)}{\partial t}
&=
\frac{\partial\overline{\psi(x,t)}}{\partial t}\psi(x,t)
+
\overline{\psi}(x,t)\frac{\partial\psi(x,t)}{\partial t}
\\
&=
\overline{ \frac{i}{\hbar}H \psi(x,t) }\psi(x,t)
+
\overline{\psi}(x,t)\frac{i}{\hbar} H \psi(x,t)
\\
&=
-
\frac{\hbar^2}{2m}\overline{\frac{\partial^2\psi(x,t)}{\partial x^2}}\psi(x,t)
+
\frac{\hbar^2}{2m}\frac{\partial^2\psi(x,t)}{\partial x^2}\overline{\psi(x,t)}
+
\varphi(x,t)|\psi(x,t)|^2
\end{align*}
Die ersten zwei Terme k"onnen wir auch als Ableitung
der Funktion
\[
j(x,t)=
\frac{\hbar^2}{2m}
\biggl(
-\frac{\partial\overline{\psi}(x,t)}{\partial x}\psi(x,t)
+\overline{\psi}(x,t)\frac{\partial\psi(x,t)}{\partial x}
\biggr)
\]
erhalten, denn
\begin{align*}
\frac{\partial j(x,t)}{\partial x}
&=
\frac{\hbar^2}{2m}
\frac{\partial}{\partial x}
\biggl(
-\frac{\partial\overline{\psi}(x,t)}{\partial x}\psi(x,t)
+\overline{\psi}(x,t)\frac{\partial\psi(x,t)}{\partial x}
\biggr)
\\
&=
\frac{\hbar^2}{2m}
\frac{\partial}{\partial x}
\biggl(
-
\frac{\partial^2\overline{\psi}(x,t)}{\partial x^2}
\psi(x,t)
-
\frac{\partial\overline{\psi}(x,t)}{\partial x}
\frac{\partial\psi(x,t)}{\partial t}
+
\frac{\partial\overline{\psi}(x,t)}{\partial x}
\frac{\partial\psi(x,t)}{\partial x}
+
\overline{\psi}(x,t)
\frac{\partial^2\psi(x,t)}{\partial x^2}
\biggr)
\\
&=
\frac{\hbar^2}{2m}
\frac{\partial}{\partial x}
\biggl(
-
\frac{\partial^2\overline{\psi}(x,t)}{\partial x^2}
\psi(x,t)
+
\overline{\psi}(x,t)
\frac{\partial^2\psi(x,t)}{\partial x^2}
\biggr)
\end{align*}
Somit erf"ullt die oben definierte Funktion $j(x,t)$ die
Kontinuit"atsgleichung
\[
\frac{\partial\varrho(x,t)}{\partial t}
=
\frac{\partial j(x,t)}{\partial x} +\varphi(x,t)\varrho(x,t).
\]





