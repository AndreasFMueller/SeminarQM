\chapter{Quantisierung\label{chapter:quantisierung}}
\lhead{Quantisierung}
\rhead{}

\section{Quantisierungsregeln}
\rhead{Quantisierungsregeln}
\index{Quantisierungsregeln}
In diesem Abschnitt untersuchen wir, wie ein klassisches System, dessen
Beschreibung wir bereits kennen, in ein quantenmechanisches System 
mit dem Formalismus aus Kapitel~\ref{chapter:einfache-quantensysteme}
umgeformt werden kann.
Dabei m"ussen alle klassisch beobachtbaren Gr"ossen, also alle Obervablen
wie Position, Impuls, Energie etc.~erhalten erhalten bleiben.

In der Bohrschen Quantenmechanik wurde gefordert, dass 
sich Teilchen auf Bahnen bewegen m"ussen, f"ur die eine
Quantisierungsbedingung gilt.
\index{Quantisierungsbedingung von Bohr}
Typischerweise musste das Produkt $p\dot q$ entlang einer geschlossenen
Bahn aufintegriert ein ganzzahliges Vielfaches von $\hbar$ ergeben.
Dieses Quantisierungsprinzip geht davon aus, dass man immer noch sinnvoll
von einer Bahn sprechen kann, was wir aber schon ausgeschlossen haben.
Wir m"ussen daher ein Quantisierungsprinzip verwenden, welches den
Begriff einer Bahn gar nicht erst verwendet, sondern nur Dinge, die
auch tats"achlich beobachtbar sind, also Observable.

In der klassischen Mechanik sind Observable einfach Funktionen
der Koordinaten und Impulse,
in der Quantenmechanik werden sie durch selbstadjungierte
Operatoren dargestellt.
Um die klassischen Observablen von den Operatoren unterscheiden zu
k"onnen, werden wir in diesem Abschnitt die Operatoren mit einem
Circumflex kennzeichnen.
Der Hamilton-Operator wird also mit $\hat H$ bezeichnet, 
der Operator zur klassischen Variable $X$ wird mit $\hat X$ etc.

\subsection{Zeitentwicklung}
In Abschnitt~\ref{section:zeitentwicklung-von-observablen} haben wir die
Zeitentwicklung eines Quantensystems kennengelernt.
Wir nehmen an, dass der Hamilton-Operator $H$ nicht von der Zeit abh"angt.
Sei $\hat A$ eine Observable, und sei $|\psi(t)\rangle$ der Zustand des
Quantensystems zur Zeit $t$.
Dann gilt
\[
\frac{d}{dt}\langle \hat A\rangle
=
\biggl\langle -\frac{i}{\hbar}[\hat A,\hat H] \biggr\rangle.
\]
Als Spezialfall muss darin die Bewegung eines klassischen mechanischen Systems
enthalten sein, wie wir es in Kapitel~\ref{chapter:mechanik}
kennengelernt haben.
Ein solches System zeichnet sich dadurch aus, dass die {\em Wellenfunktion}
\label{Wellenfunktion}
$\Psi(x,t)=\langle x|\psi(t)\rangle$
sehr gut lokalisiert ist.
Nach de~Broglie werden Quantenph"anomene nur auf einer Entfernung sichtbar,
die der de~Broglie-Wellenl"ange $\lambda=h/p$ des Teilchens entsprechen.
Weiter als einige de~Broglie-Wellenl"angen von $\langle x\rangle$
entfernt wird $\Psi$ verschwinden.

Eine Observable $A$ des klassischen Systems ist eine Funktion der
Koordinaten und Impulse, man kann ihre Zeitentwicklung mit den
Bewegungsgleichungen~\ref{skript:zeitentwicklung-poisson}
berechnen:
\[
\frac{d}{dt}A=(A,H).
\]
Die Quantisierungsregel lautet daher:

\begin{satz}
\label{skript:quantisierung-poisson}
\index{Quantisierungsregel!mit Poisson-Klammern und Kommutatoren}
Ein klassisches System mit Hamilton-Funktion $H$ wird quantisiert, indem
jede Observable $A$ durch einen Operator $\hat A$ ersetzt wird, die
Hamilton-Funktion durch einen Hamilton-Operator $\hat H$, und jede
Poisson-Klammer durch einen Kommutator nach der Regel
\begin{equation}
(A,B)
\qquad
\mapsto
\qquad
-\frac{i}{\hbar}[\hat A,\hat B].
\label{skript:quantisierung-observable}
\end{equation}
\end{satz}
Aus den Regeln folgt auch, dass der Operator $\hat H$ tats"achlich
der Operator ist, der der klassischen Observablen $H$ entspricht, also der 
Energie des Systems.
\label{skript:hamilton-op-ist-energie}

\subsection{Ortsdarstellung}
Klassische Systeme werden in Koordinaten und Impulsen beschrieben, 
die nat"urlich auch Observable sind.
Ausserdem k"onnen wir den Zustandsvektor in der Ortsdarstellung
als Wellenfunktion $\Psi(x,t)=\langle x|\psi(t)\rangle$
ausdr"ucken.
Statt mit den abstrakten Vektoren $|\psi\rangle$ zu rechnen, m"ochten
wir mit den viel konkreteren Wellenfunktionen $\Psi(x,t)$ rechnen 
k"onnen. Wir m"ussen also alle Operationen mit $|\psi\rangle$ neu
als Operationen auf $\Psi(x,t)$ ausdr"ucken.

Nach den allgemeinen Grunds"atzen des fr"uher entwickelten Formalismus
muss $|\Psi(x,t)|^2$ die Wahrscheinlichkeitsdichte f"ur die Position
des Teilchens sein.
F"ur zwei verschiedene Zust"ande muss das Skalarprodukt daher als
Integral gebildet werden:
\[
\left.
\begin{aligned}
\Psi(x,t)&=\langle x|\psi\rangle\\
\Phi(x,t)&=\langle x|\varphi\rangle
\end{aligned}
\quad
\right\}
\qquad
\Rightarrow
\qquad
\langle\varphi|\psi\rangle
=
\int \langle\varphi|x\rangle\langle x|\psi\rangle\,dx
=
\int \overline{\langle x| \varphi\rangle}\langle x|\psi\rangle\,dx
=
\int \bar\Phi(x,t)\Psi(x,t)\,dx.
\]

Wir erinnern daran, wie Erwartungswerte zu berechnen sind.
F"ur eine klassische  Observable $G$, die nur vom Ort eines Teilchens abh"angt,
muss f"ur den Erwartungswert gelten
\[
\langle G\rangle = \int G(x)\cdot |\Psi(x,t)|^2\,dx
=
\int\bar\Psi(x,t)\,G(x)\,\Psi(x,t)\,dx.
\]
Der in der Ortsdarstellung zu $G$ geh"orige Operator ist also nichts
anderes als die Multiplikation mit $G(x)$.

Doch wie muss man den Operator f"ur den Impuls ausdr"ucken?
Um die Wirkung von $\hat p$ zu ermitteln, verwenden wir die
Quantisierungsregeln.
F"ur eine klassische Observable $G(x)$, die nur von $x$
abh"angt, gilt
\begin{equation}
(p,G)=-\frac{\partial G}{\partial x}.
\label{skript:pGobservable}
\end{equation}
Wir wenden jetzt die
Quantisierungsvorschrift~(\ref{skript:quantisierung-observable})
auf (\ref{skript:pGobservable}) an und erhalten
\[
\begin{CD}
\displaystyle(p,G)     @=   \displaystyle-\frac{\partial G}{\partial x}.\\
@V\text{Quantisierung}VV           @V\text{Quantisierung}VV   \\
\displaystyle-\frac{i}{\hbar}[\hat p,\hat G]
@=
\displaystyle-\widehat{\frac{\partial G}{\partial x}}.\\
@V\textstyle\langle\,.\,\rangle VV           @V\textstyle\langle\,.\,\rangle VV   \\
\displaystyle -\int \bar\Psi(x,t)\,\biggl(\frac{i}{\hbar}\hat p G(x)-G(x)\frac{i}{\hbar}\hat p\biggr)\,\Psi(x,t)\,dx
@=
\displaystyle -\int \bar \Psi(x,t)\,\frac{\partial G}{\partial x}\,\Psi(x,t)\,dx
\end{CD}
\]
Man beachte, dass in der letzten Gleichung auf der linken Seite der Operator
$\hat p$ auf das Produkt $G(x)\Psi(x,t)$ wirkt.
Damit daraus eine Ableitung von $G$ entstehen kann, muss der Operator
$\hat p$ selbst wie eine Ableitung operieren. Setzen wir
\begin{equation}
\frac{i}{\hbar}\hat p = \frac{\partial}{\partial x},
\label{skript:praeimpulsoperator}
\end{equation}
wird die letzte Gleichung zu
\begin{align}
-\int \bar\Psi(x,t)\,\biggl(\frac{\partial}{\partial x}G(x)-G(x)\frac{\partial}{\partial x}\biggr)\,\Psi(x,t)\,dx
&=
-\int \bar \Psi(x,t)\,\frac{\partial G}{\partial x}\,\Psi(x,t)\,dx
\notag
\\
-\int \bar\Psi(x,t)\,\biggl(
\frac{\partial G}{\partial x}
+
G(x)\frac{\partial}{\partial x}
-G(x)\frac{\partial}{\partial x}\biggr)\,\Psi(x,t)\,dx
&=
-\int \bar \Psi(x,t)\,\frac{\partial G}{\partial x}\,\Psi(x,t)\,dx.
\label{skript:produktregelpG}
\end{align}
Darin haben wir die Produktregel auf das Produkt $G(x)\Psi(x,t)$ angewendet.
Offenbar war (\ref{skript:praeimpulsoperator}) die richtige Wahl, und wir
haben eine vereinfachte Quantisierungsregel f"ur die Ortsdarstellung
gefunden.

\begin{satz}
\label{skript:quantisierungsregeln-ortsdarstellung}
\index{Quantisierungsregeln!in Ortsdarstellung}
Ein mechanisches System wird in der Ortsdarstellung quantisiert,
indem als Impulsoperatoren die Differentialoperatoren
\[
\hat p=\frac{\hbar}{i}\frac{\partial}{\partial x}
\]
verwendet werden. Den Hamilton-Operator $\hat H$ erh"alt man, indem man
in der Hamilton-Funktion $H$ die Variablen $p$ durch die Operatoren 
$\hat p$ ersetzt.
\end{satz}

{\small
Alternativ kann man die Quantisierungsregel f"ur den Impulsoperator
auch wie folgt verstehen.
Die Anwendung der Produktregel in (\ref{skript:produktregelpG}) kann
man auch als eine Anwendung der partiellen Integration ansehen.
Wir berechnen dazu direkt den Erwartungswert von $\partial G/\partial x$:
\begin{align*}
\biggl\langle
\frac{\partial G}{\partial x}
\biggr\rangle
&=
\int \bar\Psi(x,t)\frac{\partial G(x)}{\partial x}\Psi(x,t)\,dx
\\
&=
\underbrace{
\biggl[
\bar\Psi(x,t) G(x)\Psi(x,t)
\biggr]_{-\infty}^\infty}_{=0}
-
\int_{-\infty}^{\infty}
\biggl(\frac{\partial}{\partial x}
\bar\Psi(x,t)\biggr) G(x)
\Psi(x,t)
+
\bar\Psi(x,t) G(x)
\,
\frac{\partial}{\partial x}
\Psi(x,t)
\,dx
\\
&=
\int_{-\infty}^{\infty}
\biggl(-\frac{\partial}{\partial x}
\bar\Psi(x,t)\biggr) G(x)
\Psi(x,t)
-
\bar\Psi(x,t) G(x)
\,
\frac{\partial}{\partial x}
\Psi(x,t)
\,dx
\\
&=\biggl\langle
\biggl(-\frac{\partial}{\partial x}\biggr)^* G
- G\frac{\partial}{\partial x}
\biggr\rangle
\end{align*}
%Der Ableitungsoperator mit dem $\mathstrut^*$ in der letzten Zeile
%wirkt auf die Wellenfunktion links. Dieser Operator ist nicht
%hermitesch, vielmehr gilt
%\begin{align*}
%\biggl\langle \varphi\bigg|\,
%\biggl(\frac{\partial}{\partial x}\biggr)^*
%\,\bigg|\psi\biggr\rangle
%&=
%\int \frac{\partial\bar\Phi(x,t)}{\partial x}\Psi(x,t)\,dx
%%\\
%%&=
%\underbrace{
%\biggl[\bar\Phi(x,t)\Psi(x,t)\biggr]_{-\infty}^\infty
%}_{=0}
%-
%\int \bar\Phi(x,t)\frac{\partial\Psi(x,t)}{\partial x}\,dx
%=
%-\biggl\langle
%\varphi\bigg|\,
%\frac{\partial}{\partial x}
%\,\bigg|\psi
%\biggr\rangle
%\end{align*}
Setzt man jetzt wieder (\ref{skript:praeimpulsoperator}) ein, erh"alt
man daraus
\[
\biggl\langle
\frac{\partial G}{\partial x}
\biggr\rangle
=
\biggl\langle
\biggl(-\frac{i}{\hbar}\hat p
\biggr)^* G
- G\frac{i}{\hbar}\hat p
\biggr\rangle
=
\frac{i}{\hbar}
\biggl\langle
\hat p
 G
- G \hat p
\biggr\rangle
=
\frac{i}{\hbar}[\hat p,G],
\]
was wieder mit der Quantisierung der klassischen Beziehung
\ref{skript:pGobservable} "ubereinstimmt.
}

\section{Schr"odingergleichung}
\rhead{Schr"odingergleichung}
In diesem Abschnitt leiten wir aus den Quantisierungsregeln die
Schr"odingergleichung her, welche uns erm"oglicht, einfache
Quantensysteme mit Differentialgleichungen zu modellieren
und zum Beispiel die Energieniveaus und Wellenfunktionen direkt
zu berechnen.

\subsection{Schr"odingergleichung in Ortsdarstellung}
\index{schrodingergleichung@Schr\"odingergleichung!in Ortsdarstellung}
Wir wenden die vereinfachten Quantisierungsregeln auf ein
Teilchen in einem Potential an.
Dieses hat die Hamilton-Funktion
\[
H=\frac1{2m}p^2+V(x).
\]
Nach den Quantisierungsregeln in
Satz~\ref{skript:quantisierungsregeln-ortsdarstellung} wird daraus
der Hamilton-Operator:
\[
\hat H
=
\frac{1}{2m}\biggl(\frac{\hbar}{i}\frac{\partial }{\partial x}\biggr)^2
+V(x)
=
-\frac{\hbar^2}{2m}\frac{\partial^2}{\partial x^2}+V(x).
\]
In drei Dimension treten entsprechend mehr Ableitungen auf:
\[
H=\frac1{2m}\sum_{i=1}^3p_i^2+V(x)
\qquad\Rightarrow\qquad
\hat H
=
-\frac{\hbar^2}{2m}\sum_{i=1}^3\frac{\partial^2}{\partial x_i^2}+V(x)
=
-\frac{\hbar^2}{2m}\Delta + V(x).
\]
Die zugeh"orige zeitabh"angige Schr"odingergleichung ist
\[
i\hbar\frac{\partial}{\partial t}\Psi(x,t)
=
-\frac{\hbar^2}{2m}\Delta\Psi(x,t) + V(x)\Psi(x,t).
\]
Ein Teilchen mit Energie $E$ in einem Potential $V(x)$ wird entsprechend
durch eine Wellenfunktion $\Psi(x)$ beschrieben, die zeitunabh"angige
Schr"odingergleichung
\[
-\frac{\hbar^2}{2m}\Delta\Psi(x) + V(x)\Psi(x)
=
E\Psi(x)
\]
erf"ullt.
Ausserdem m"ussen die Wellenfunktionen nat"urlich normiert sein.

Damit lassen sich jetzt einfache Potential-Probleme l"osen.
Wir untersuchen im folgenden die Bewegung eines Teilchens in einfachen
Potentialen, um erste Erfahrungen mit der Schr"odingergleichung
zu sammeln und einen ersten Eindruck von den Eigenheiten der
des quantenmechanischen Verhaltens von Teilchen zu erhalten.

\subsection{Potentialkasten\label{subsection:potentialkasten}}
\index{Potentialkasten}
Wir wenden die Theorie an, um die Energieniveaus eines eindimensionalen
Teilchens zu berechnen, welches in einem Interval $[-l,l]$ eingesperrt ist.
Realisiert werden kann diese Situation n"aherungsweise dadurch, dass 
ein Teilchen durch eine sehr hohe (unendlich hohe) Potentialbarriere
am verlassen des Bereiches gehindert wird.
Die Wellenfunktion $\psi(x)$ eines solchen Teilchens muss ausserhalb des
Intervals verschwinden, denn die Wahrscheinlichkeit, das Teilchen dort zu
finden, soll ja $0$ sein.
Man erreicht dies, indem man in den Punkten $\pm l$ eine unendlich
hohe Potentialbarriere errichtet.

\subsubsection{Hamilton-Operator und Schr"odingergleichung}
Der Hamilton-Operator des Problems ist
\[
-\frac{\hbar^2}{2m}\frac{\partial^2}{\partial x^2}+V(x),
\]
das Potential hat die Form
\[
V(x)=\begin{cases}
0&\qquad |x|<l\\
\infty&\qquad\text{sonst.}
\end{cases}
\]
Die zugeh"orige zeitunabh"angige Schr"odingergleichung ist
\begin{equation}
-\frac{\hbar^2}{2m}\psi''(x)=E\psi(x)
\label{skript:schroedingerkasten}
\end{equation}
im Inneren des Intervals.
Wir suchen eine L"osung von (\ref{skript:schroedingerkasten}),
die ausserdem den Randbedingungen
\[
\psi(\pm l)=0
\]
gen"ugen soll, damit wird dem unendlich hohen Potential ausserhalb des
Kastens Rechnung getragen.
Das charakteristische Polynom der Gleichung (\ref{skript:schroedingerkasten}) ist
\[
-\frac{\hbar^2}{2m}\lambda^2-E=0,
\]
es hat die Nullstellen
\[
\lambda_\pm = \pm\sqrt{\frac{2m}{\hbar^2}(-E)}.
\]

\subsubsection{Symmetrie}
\index{Spiegelung}
Der Hamilton-Operator dieses Problems ist symmetrisch bez"uglich
der Spiegelung.
Der Spiegelungsoperator $S$, der aus der Wellenfunktion
$\psi(x)$ die Wellenfunktion $\psi(-x)$ macht, vertauscht mit
dem Hamilton-Operator, also kann man die Eigenzust"ande von $H$ so w"ahlen,
dass sie auch Eigenzust"ande von $S$ sind.
Da $S^2=\operatorname{id}$, sind $\pm1$ die einzig m"oglichen Eigenwerte
von $S$, also ist $\psi(x)=\psi(-x)$ oder $\psi(x)=-\psi(-x)$.
Wenn man eine L"osung der Schr"odingergleichung sucht, kann man also
zus"atzlich verlangen, dass die L"osungen gerade oder ungerade Funktionen
sind.

\subsubsection{Negative Energie}
Falls $E<$ ist, sind die Nullstellen des chrakteristischen
Polynoms reell, $\lambda_+>0$ und $\lambda_-=-\lambda_+ < 0$,
und die allgemeine 
L"osung der Gleichung (\ref{skript:schroedingerkasten}) hat die Form
\[
\psi(x)=A_+e^{\lambda_+x}+A_-e^{\lambda_-x}.
\]
Ausserdem m"ussen die Randbedingungen erf"ullt sein, die man erh"alt,
indem man $x=\pm l$ setze:
\[
\begin{linsys}{2}
e^{ \lambda_+l} A_+&+&e^{ \lambda_-l}A_-&=&0\\
e^{-\lambda_+l}A_+&+&e^{-\lambda_-l}A_-&=&0
\end{linsys}
\]
Dieses Gleichungssystem f"ur die Koeffizienten $A_+$ und $A_-$ hat
Determinante
\[
\left|\begin{matrix}
e^{ \lambda_+l}&e^{ \lambda_-l}\\
e^{-\lambda_+l}&e^{-\lambda_-l}
\end{matrix}\right|
=
e^{\lambda_+l-\lambda_-l}-e^{\lambda_-l-\lambda_+l}
=
e^{2l\lambda_+}-e^{-2l\lambda_+}
=
2\sinh 2l\lambda_+
\]
Dieses Ausdruck verschwindet nur, wenn $\lambda_+=0$, im Widerspruch
zu $\lambda_+>0$.
Es gibt also keine L"osung f"ur Teilchen mit negativer Energie.

\subsubsection{Positive Energie}
In diesem Falls sind die Nullstellen imagin"ar, $\lambda_\pm=\pm ik$
mit $k=\sqrt{2mE/\hbar^2}$,
und die allgemeine L"osung wird durch Linearkombinationen der Funktionen
$\cos kx$ und $\sin kx$ gegeben.
Wir m"ussen also herausfinden, f"ur welche Werte von $E$ und damit $k$
die Randbedingungen erf"ullt werden k"onnen.
Wir wissen bereits, dass wir zus"atzlich verlangen k"onnen, dass die
L"osungen gerade oder ungerade sind.
Eine gerade L"osung muss von der Form $A\cos kx$ sein, eine ungerade
L"osung von der Form $B\sin kx$.
In beiden F"allen ist die Randbedingung am linken Rand automatisch
erf"ullt, wenn die Bedingun am rechten Rand erf"ullt ist.
Wir setzen die Randbedingungen ein:
\begin{align*}
A\cos kl&=0
	&&&
		B\sin kl&=0\\
kl&=\frac{\pi}2+n\pi,\quad n\in\mathbb Z
	&&&
		kl&=n\pi,\quad n\in\mathbb Z\\
k&=\frac{\pi}{l}\biggl(n+\frac12\biggr),\quad n\in\mathbb Z
	&&&
		k&=\frac{\pi}{l}n,\quad n\in\mathbb Z.
\end{align*}
Es sind also alle $k$-Werte m"oglich, die ganzzahlige Vielfache von
$\pi/2l$ sind.
Die geraden Vielfache f"uhren zu einer $\sin$-L"osung, die ungeraden
zu einer $\cos$-L"osung.

Aus den m"oglichen $k$-Werten k"onnen wir jetzt die m"oglichen 
Energiewerte ableiten
\begin{align*}
E&=\frac{\hbar^2\pi^2}{2ml^2}\biggl(n+\frac12\biggr)^2
&&&
E&=\frac{\hbar^2\pi^2}{2ml^2}n^2
\end{align*}
In einer Formel zusammengefasst kann man schreiben
\[
E_n
=
\frac{\hbar^2\pi^2}{2ml^2}\biggl(\frac{n}{2}\biggr)^2
=
\frac{h^2n^2}{32ml^2}.
\label{skript:potentialkasten-e}
\]
wobei f"ur gerades $n$ eine ungerade L"osungsfunktion auftritt,
f"ur ungerades $n$ aber eine gerade L"osungsfunktion.
\begin{figure}
\centering
\includegraphics{graphics/potential-1.pdf}
\caption{Teilchen in einem Potentialkasten
\label{skript:potentialkasten}}
\end{figure}

\subsubsection{Normierung}
Wir m"ussen noch sicherstellen, dass die L"osungsfunktionen korrekt
normiert sind.
Dazu m"ussen wir die Integrale von $\cos^2$ und $\sin^2$ berechnen.
Die Graphen von $\cos^2$ und $\sin^2$ halbieren aber genau das
Rechteck zwischen $-l$ und $l$ mit H"ohe $1$, also ist der Wert
des Integrals in jedem Fall $l$, die L"osungsfunktionen mit
der richtigen Normierung sind also:
\begin{align}
\psi_n(x)
&=
\begin{cases}
\displaystyle
\frac{1}{\sqrt{l}}\cos\frac{n \pi x}{2l}&\qquad \text{$n$ ungerade}\\
\\
\displaystyle
\frac{1}{\sqrt{l}}\sin\frac{n \pi x}{2l}&\qquad \text{$n$ gerade}.
\end{cases}
\label{skript:potentialkasten-psi-normiert}
\end{align}
Der Grundzustand ist $n=1$, mit Grundzustandsenergie $E_1=h^2/32ml^2$.

\subsubsection{Klassische Quantisierungsbedingung}
Klassisch kann man sich das Teilchen als eine Welle vorstellen, die
zwischen den beiden W"anden des Potentialkastens hin und her l"auft.
Der Impuls des Teilchens ist $\hbar k$, die Wirkung f"ur einen Weg
hin und zur"uck ist daher:
\[
\oint p\,dq
=
2\cdot \hbar k \cdot 2l = 2\cdot \hbar \frac{\pi}{l}\frac{n}{2}\cdot 2l=hn,
\]
dies entspricht genau der Bohrschen Quantisierungsbedingung.

\subsubsection{Physikalische Interpretation}
Offenbar ist es nicht m"oglich, auf beliebig kleinem Raum einzusperren.
Je kleiner der zur Verf"ugung stehende Raum, desto h"oher ist die
Energie, die ein Teilchen mindestens haben wird.
Im Kapitel~\ref{chapter:heisenberg} werden wir mit der Unsch"arferelationen
einen weiteren Ausdruck dieses Ph"anomens kennenlernen.

In einem Atomkern sind die Protonen und Neutronen auf sehr kleinem
Raum eingesperrt, und wir k"onnen versuchen, die minimale Energie 
eines Neutrons abzusch"atzen.
Der Atomkern hat einen Durchmesser von $10\text{fm}=10\cdot 10^{-15}\text{m}$,
das Neutron hat eine Masse von $m_n=1.67\cdot 10^{-27}\text{kg}$,
die Grundzustandsenergie ist daher $E_1=3.28\cdot10^{-13}\text{J}$.
Mit der Formel $E=mc^2$ entspr"ache dieser Energie die Masse
$3.656\cdot 10^{-30}\text{kg}$, oder etwa $0.2\%$ der Masse eines
Neutrons, und mehr als der Masse eines Elektrons.

\subsubsection{H"ohere Dimensionen}
Man kann das Problem eines Teilchens in einem Potentialkasten auch
in drei Dimensionen l"osen.
Der Hamiltonoperator ist 
\[
H=-\frac{\hbar^2}{2m}\biggl(
\frac{\partial^2}{\partial x^2}
+
\frac{\partial^2}{\partial y^2}
+
\frac{\partial^2}{\partial z^2}
\biggr)
=-\frac{\hbar^2}{2m}\Delta
\]
Man verwendet dazu einen sogenannten Seprationsansatz, man schreibt
\[
\psi(x,y,z)=X(x)\cdot Y(y)\cdot Z(z),
\]
und erh"alt die Differentialgleichung
\[
\frac{\hbar^2}{2m}\biggl(
X''(x)Y(y)Z(z)
+
X(x)Y''(y)Z(z)
+
X(x)Y(y)Z''(z)
\biggr)
=
EX(x)Y(y)Z(z)
\]
Dividiert man durch $\psi$, erhalt man die Gleichung
\begin{equation}
-\frac{\hbar^2}{2m}\frac{X''(x)}{X(x)}
-
\frac{\hbar^2}{2m}\frac{Y''(y)}{Y(y)}
-
\frac{\hbar^2}{2m}\frac{Z''(z)}{Z(z)}
=
E.
\label{skript:potentialkasten-separiert}
\end{equation}
Man kann nach jedem Term aufl"osen, und erh"alt dann eine Gleichung,
deren linke Seite nur von einer Variablen, die rechte nur von den anderen
abh"angt.
Es folgt dann, dass beide Seiten konstant sein m"ussen, 
die Gleichung (\ref{skript:potentialkasten-separiert}) beinhaltet also eigentlich
drei Gleichungen
\begin{equation}
\begin{aligned}
-\frac{\hbar^2}{2m}\frac{X''(x)}{X(x)}&=E_x&&\Rightarrow&-\frac{\hbar^2}{2m}X''&=E_xX\\
-\frac{\hbar^2}{2m}\frac{Y''(y)}{Y(y)}&=E_y&&\Rightarrow&-\frac{\hbar^2}{2m}Y''&=E_yY\\
-\frac{\hbar^2}{2m}\frac{Z''(z)}{Z(z)}&=E_z&&\Rightarrow&-\frac{\hbar^2}{2m}Z''&=E_zZ
\end{aligned}
\label{skript:potentialkasten-einzelgleichungen}
\end{equation}
mit der zus"atzlichen Bedingung $E=E_x+E_y+E_z$.
Die Gleichungen (\ref{skript:potentialkasten-einzelgleichungen}) sind drei
unabh"angige Problem wie der eindimensionale Potentialkasten
(\ref{skript:schroedingerkasten}), deren Energieniveaus wir bereits berechnet
haben.
Daraus k"onnen wir unmittelbar die Energieniveaus im dredimensionalen
Fall ableiten:
\begin{equation}
E=\frac{h^2}{32ml^2}(n_x^2+n_y^2+n_z^2)
\label{skript:3dzustaende}
\end{equation}
mit $n_x,n_y,n_z>1$.


\subsection{Potentialtopf}
Aus der L"osung des Problems eines Teilchens in einem Potentialkasten
k"onnte der Eindruck entstehen, dass die Quantisierung der Energieniveaus
mit der Randbedingung am Rande des Kastens zu tun hat.
Sobald diese Bedingung wegf"allt, k"onnten die Bedingungen nicht mehr
ausreichend streng sein, um diskrete Energieniveaus zu erzwingen.
Dass dem nicht so ist zeigt das hier zu behandelnde Beispiel eines
Teilchens in einem Potentialtopf, wie auch der sp"ater im
Kapitel~\ref{chapter:harmonisch} behandelte harmonische Oszillator.

\subsubsection{Problemstellung}
\begin{figure}
\centering
\includegraphics{graphics/potential-2.pdf}
\caption{Potentialtopf der Breite $2l$ und Tiefe $-V_0$.
\label{potentialtopf}}
\end{figure}

Wir m"ochten die Schr"odingergleichung l"osen f"ur ein Teilchen, welches
dem in Abbildung~\ref{potentialtopf} festgehalten wird.
Das Potential ist
\[
V(x)
=
\begin{cases}
-V_0&\qquad|x|<l\\
0&\qquad\text{sonst}
\end{cases}
\]
Die Schr"odingergleichung lautet
\[
-\frac{\hbar^2}{2m}\psi''(x)+V(x)\psi(x)=E\psi(x).
\]
Die L"osungsfunktion $\psi(x)$ muss also je nach Teilinterval
verschiedene Differentialgleichungen l"osen:
\begin{equation}
\begin{aligned}
-\frac{\hbar^2}{2m}\psi''(x)&=(V_0+E)\psi(x)&\text{f"ur $|x|\le l$}
\\
-\frac{\hbar^2}{2m}\psi''(x)&=E\psi(x)      &\text{f"ur $|x|>l$}
\end{aligned}
\label{potentialtopf-gleichungen}
\end{equation}
Wir werden die Differentialgleichung in jedem dieser Teilintervalle
separat l"osen, und dann zu einer Funktion $\psi$ zusammensetzen.
Die Funktion $\psi(x)$ muss nat"urlich stetig sein, die einzelnen
Teile m"ussen also an den Stellen $x=\pm l$ ohne Sprung ineinander
"ubergehen.
Dasselbe muss f"ur die Ableitungen gelten, denn wenn $\psi'$  nicht
stetig w"are, w"urde die zweite Ableitung unendlich gross, was nicht
mit der Schr"odingergleichung vereinbar ist.

\subsubsection{L"osungen von (\ref{potentialtopf-gleichungen})}
Die Gleichungen sind von der Form
\begin{equation}
\psi''(x)=C\psi(x),\qquad
C=-\frac{2m}{\hbar^2}(V_0+E)
\;
\text{oder}
\;
C=-\frac{2m}{\hbar^2}E,
\label{potentialtopf-proto}
\end{equation}
die wir mit dem Standardverfahren f"ur lineare Differentialgleichungen
l"osen k"onnen.
Die charakteristische  Gleichung ist $\lambda^2=C$, mit Nullstellen
$\lambda=\sqrt{C}$. Je nach Vorzeichen von $C$ ist der Charakter der
L"osungen verschieden.
F"ur $C>0$ sind die $e^{x\sqrt{C}}$ und $e^{-x\sqrt{C}}$.

F"ur $C<0$ sind die L"osungen der charakterisischen Gleichung imagin"ar.
Man kann die L"osungen der Differentialgleichung mit Exponentialfunktionen
$e^{ix\sqrt{-C}}$ und $e^{-ix\sqrt{-C}}$ schreiben.
Es wird aber viel einfacher, wenn man wieder die Symmetrieeigenschaften
verwendet, und nach geraden und ungeraden L"osungsfunktionen sucht,
die sich mit $\cos x\sqrt{-C}$ bzw.~$\sin x\sqrt{-C}$ schreiben
lassen m"ussen.

\subsubsection{F"alle $E>0$ und $E<-V_0$}
F"ur $E>0$ sind die L"osungen "uberall Schwingungsl"osungen, deren
Amplitude auf der ganzen reellen Achse ausserhalb des Topfes
konstant ist. Eine solche Funktion ist nicht normierbar, insbesondere
kann es f"ur $E>0$ also keine L"osungen geben.

F"ur $E<-V_0$ ist das $C>0$ in (\ref{potentialtopf-proto}), es kommen
also nur die Exponentialfunktionen f"ur eine L"osung in Frage.
Damit l"asst sich aber keine normierbare L"osung bauen: rechts von $l$
d"urfte man nur $e^{-x\sqrt{C}}$ verwenden, links von $-l$ nur
$e^{-x\sqrt{C}}$, und dazwischen m"usste man ein Funktion konstruieren,
die an den R"andern stetig differenzierbar in die beiden abfallenden
Exponentialfunktionen "ubergeht.
Dies scheitert am gleichen Gleichungssystem wie beim Potentialkasten
im Falle $E<0$.

\subsubsection{Der Fall $-V_0 < E < 0$}
In diesem Fall suchen wir jetzt eine gerade oder ungerade L"osung, die
im Interval $[-l,l]$ eine Schwingungsl"osung $a \cos kx$ oder $a \sin kx$
ist mit 
\[
k=\sqrt{\frac{2m}{\hbar^2}(V_0+E)},
\]
und f"ur $x>l$ eine exponentiell abfallende L"osung $be^{k'(x-l)}$
mit
\[
k'=\sqrt{-\frac{2m}{\hbar^2}E}.
\]
Wegen der Symmetrie reicht es jetzt, die Bedingungen nur noch beim
Punkt $l$ aufzustellen.
Die Stetigkeitsbedingungen lauten 
\begin{align*}
&&\psi(-x)&=\psi(x)	&	\psi(-x)&=-\psi(x)\\
&\text{Stetigkeit von $\psi$ bei $x=l$:}&
	a\cos kl&= b	&	a\sin kl&=b \\
&\text{Stetigkeit von $\psi'$ bei $x=l$:}&
	-ak\sin kl&=-bk'&     ak\cos kl&=-bk'
\end{align*}
Dividiert man die beiden Gleichungen, erhalten wir zwei
Gleichungen, die nur noch $k$ und $k'$ enthalten:
\begin{align}
\tan kl&=\frac{k'}{k},
&-\cot kl&=\frac{k'}{k},
\label{potentialtopf-k-gleichungen}
\end{align}
Die Gr"ossen $k$ und $k'$ h"angen aber beide von $E$ ab, dies sind
als in Wirklichkeit Gleichungen f"ur $E$.

\begin{figure}
\centering
\includegraphics[width=\hsize]{graphics/potential-4.pdf}
\caption{L"osungen der Gleichungen (\ref{potentialtopf-xigleichungen}).
Rote Kurve: $\sqrt{A^2-\xi^2}/\xi$, blaue Kurven: $\tan\xi$, gr"une
Kurven $-\cot\xi$.
\label{loesungen-xigleichungen}}
\end{figure}%
\begin{figure}
\centering
\includegraphics{graphics/potential-5.pdf}
\caption{Wellenfunktionen f"ur Teilchen in einem Potentialtopf der
Tiefe $-V_0$ under der Breite $2l$. Nur die sechs Wellenfunktionen 
geringster Energie sind dargestellt, und die Energie "uber dem
Grund des Potentialtopfes ist 15fach "uberh"oht.
\label{potentialtopf-loesungen-klein}}
\end{figure}
\begin{figure}
\centering
\includegraphics{graphics/potential-3.pdf}
\caption{Wellenfunktionen f"ur Teilchen in einem Potentialtopf der
Tiefe $-V_0$ under der Breite $2l$. Die Wellenfunktionen zu den sechs
untersten Energieniveaus sind nicht dargestellt, siehe dazu die
Abbildung~\ref{potentialtopf-loesungen-klein}.
\label{potentialtopf-loesungen}}
\end{figure}
Wir versuchen, alle Gr"ossen durch die dimensionslose Gr"osse $\xi=kl$
auszudr"ucken.
Zun"achst ist
\begin{align}
\xi=kl&=l\sqrt{\frac{2m}{\hbar^2}(V_0+E)}
\notag
\\
\xi^2
&=
\frac{2ml^2}{\hbar^2}(V_0+E)
=
\frac{2ml^2V_0}{\hbar^2} +\frac{2ml^2}{\hbar^2}E
=
\frac{2ml^2V_0}{\hbar^2}-l^2k'^2
\notag
\\
k'l
&=
\sqrt{\frac{2ml^2V_0}{\hbar^2} - \xi^2}
\notag
\\
E
&=
-\frac{\hbar^2}{2m}k'^2
=
-\frac{\hbar^2}{2m}
\biggl(
\frac{2mV_0}{\hbar^2} - \frac{\xi^2}{l^2}
\biggr)
=
-V_0+\frac{\hbar^2\xi^2}{2ml^2}
\label{potentialtopf-energie}
\end{align}
Schreiben wir
\[
A=\sqrt{\frac{2ml^2V_0}{\hbar^2}}
\]
und setzen wir ein in die Gleichungen (\ref{potentialtopf-k-gleichungen}),
erhalten wir die neuen Gleichungen f"ur $\xi$
\begin{align}
\tan \xi&=\frac{\sqrt{A^2-\xi^2}}{\xi}
&
-\cot \xi&=\frac{\sqrt{A^2-\xi^2}}{\xi}
\label{potentialtopf-xigleichungen}
\end{align}
In Abbildung~\ref{loesungen-xigleichungen} sind die rechten und linken
Seiten der Gleichungen (\ref{potentialtopf-xigleichungen}) aufgezeichnet,
die Schnittpunkte geh"oren zu $\xi$-Werten, f"ur die es L"osungen
der Schr"odingergleichung gibt.
Die zugeh"origen Energien k"onnen mit (\ref{potentialtopf-energie})
berechnet werden.


Die Gleichungen (\ref{potentialtopf-xigleichungen}) k"onnen nicht 
in geschlossener Form gel"ost werden, die Energieniveaus k"onnen also
nur numerisch bestimmt werden.
Man kann immerhin ein paar qualitative Aussagen machen.
Aus Abbildung~\ref{loesungen-xigleichungen} kann man ablesen, dass die
$\xi$-Werte f"ur die Niveaus mit niedrigester Energie nahe an $n\pi/2$
sind, die Energien sind also nahe bei
\[
-V_0+\frac{\hbar^2\xi^2}{2ml^2}
\simeq
-V_0+\frac{h^2n^2\pi^2}{32ml^2\pi^2}
=
-V_0+\frac{h^2n^2}{32ml^2},
\]
die Energie dieser Niveaus liegt also "ahnlich hoch "uber dem Grund
des Potentialtopfes, wie wir f"ur einen Potentialkasten gefunden haben.

Die L"osungsfunktionen sind in den
Abbildungen~\ref{potentialtopf-loesungen-klein}
und \ref{potentialtopf-loesungen}
dargestellt.
Man erkennt, dass man die Teilchen mit positiver Wahrscheinlichkeit in
der Wand antreffen wird, die Eindringtiefe ist zudem umso gr"osser,
je gr"osser die Energie eines Teilchens ist.


\section{Wahrscheinlichkeitsstrom}
% Kontinuitätsgleichung für die Wahrscheinlichkeitsdichte <psi(x)|psi(x)>
Die Wahrscheinlichkeit ein Teilchen in einer Umgebung des
Punktes $x$ zu finden, ist $\varrho(x)=\langle \psi(x)|\psi(x)\rangle$.
Die Zeitentwicklung f"uhrt dazu, dass $\varrho$ auch von der
Zeit abh"angt.
Dies gesamte Wahrscheinlichkeit muss nat"urlichh erhalten bleiben,
das Teilchen kann ja nicht einfach verschwinden.

Wenn sich mit der Zeitentwicklung das Teilchen woanders hin bewegt,
dann ist damit ein Stromdichtevektor verbunden, der angibt, 
wieviel Wahrscheinlichkeit durch eine Fl"ache fliesst.
Das Ziel dieses Abschnittes ist, eine solche Stromdichte zu
definieren.

\subsection{Kontinuit"atsgleichung}
\index{Kontinuit\"atsgleichung}
Wir stellen uns ein Medium mit Dichte $\varrho(x,t)$ vor.
Das Medium hat im Punkt $x$ die Str"omungsgeschwindigkeit $v(x)$.
In einem Zeitinterval $\Delta t$ nimmt die Masse des Mediums
in einem Interval der L"ange $\Delta x$ um den Betrag
\[
\Delta x\frac{\partial\varrho}{\partial t}\Delta t
\]
zu.
Dies Zunahme muss dadurch erfolgen, dass durch die Endpunkte
des $\Delta x$-Intervals mehr Material zu- als abfliesst.
Der Zufluss am linken Ende des Intervals ist
$
\varrho(x) v(x),
$
der Abfluss am rechten Ende ist $\varrho(x+\Delta x)v(x+\Delta x)$.
Die Bilanz ist
\[
\Delta x\frac{\partial\varrho}{\partial t}\Delta t
=
\Delta t(
\varrho(x) v(x)
-
\varrho(x+\Delta x) v(x+\Delta x)
)
\]
Wir teilen durch $\Delta x\,\Delta t$ und lassen $\Delta x$ gegen 0 gehen:
\begin{equation}
\frac{\partial\varrho}{\partial t}
+\frac{\partial}{\partial x}(\varrho(x)v(x))
=0.
\label{skript:kontinuitaetsgleichung1d}
\end{equation}
Die Gr"osse $j(x)=\varrho(x)v(x)$ beschreibt den Materialstrom.
Die Gleichung (\ref{skript:kontinuitaetsgleichung1d}) heisst die
Kontinuit"atsgleichung.
Sie dr"uckt aus, dass im Verlauf der Str"omung kein Material verlorgen
gehen kann.

In drei Dimensionen kann man ebenfalls ein Kontinuit"atsgleichung
f"ur die Dichte $\varrho(x)$ und den Strom $\vec\j(x)=\varrho(x) \vec v(x)$
definieren, die die dreidimensionale 
Kontinuit"atsgleichung
\[
\frac{\partial\varrho}{\partial x}
+
\frac{\partial j_1}{\partial x_1}
+
\frac{\partial j_2}{\partial x_2}
+
\frac{\partial j_3}{\partial x_3}
=
\frac{\partial\varrho}{\partial t}+\operatorname{div}\vec\j
=0
\]
erf"ullt.

\subsection{Wahrscheinlichkeitsstrom}
Wir suche jetzt einen Wahrscheinlichkeitsstrom, der zusammen mit
der Wahrscheinlichkeitsdichte $|\psi(x)|^2$ eine Kontinuit"atsgleichung
erf"ullt.
\begin{align*}
\frac{\partial\varrho(x,t)}{\partial t}
&=
\frac{\partial}{\partial t} \varrho(x)
=
\frac{\partial\varrho(x)}{\partial t}
=
\frac{\partial\overline{\psi(x,t)}}{\partial t}\psi(x,t)
+
\overline{\psi}(x,t)\frac{\partial\psi(x,t)}{\partial t}
\end{align*}
Im letzten Term k"onnen wir die Zeitableitungen durch die
Schr"odingergleichung ersetzen:
\begin{align*}
\frac{\partial\varrho(x,t)}{\partial t}
&=
\frac{\partial\overline{\psi(x,t)}}{\partial t}\psi(x,t)
+
\overline{\psi}(x,t)\frac{\partial\psi(x,t)}{\partial t}
\\
&=
\overline{ \frac{i}{\hbar}H \psi(x,t) }\psi(x,t)
+
\overline{\psi}(x,t)\frac{i}{\hbar} H \psi(x,t)
\\
&=
-
\frac{\hbar^2}{2m}\overline{\frac{\partial^2\psi(x,t)}{\partial x^2}}\psi(x,t)
+
\frac{\hbar^2}{2m}\frac{\partial^2\psi(x,t)}{\partial x^2}\overline{\psi(x,t)}
+
V(x)|\psi(x,t)|^2
\end{align*}
Die ersten zwei Terme k"onnen wir auch als Ableitung
der Funktion
\[
j(x,t)=
\frac{\hbar^2}{2m}
\biggl(
-\frac{\partial\overline{\psi}(x,t)}{\partial x}\psi(x,t)
+\overline{\psi}(x,t)\frac{\partial\psi(x,t)}{\partial x}
\biggr)
\]
erhalten, denn
\begin{align*}
\frac{\partial j(x,t)}{\partial x}
&=
\frac{\hbar^2}{2m}
\frac{\partial}{\partial x}
\biggl(
-\frac{\partial\overline{\psi}(x,t)}{\partial x}\psi(x,t)
+\overline{\psi}(x,t)\frac{\partial\psi(x,t)}{\partial x}
\biggr)
\\
&=
\frac{\hbar^2}{2m}
\biggl(
-
\frac{\partial^2\overline{\psi}(x,t)}{\partial x^2}
\psi(x,t)
-
\frac{\partial\overline{\psi}(x,t)}{\partial x}
\frac{\partial\psi(x,t)}{\partial t}
+
\frac{\partial\overline{\psi}(x,t)}{\partial x}
\frac{\partial\psi(x,t)}{\partial x}
+
\overline{\psi}(x,t)
\frac{\partial^2\psi(x,t)}{\partial x^2}
\biggr)
\\
&=
\frac{\hbar^2}{2m}
\biggl(
-
\frac{\partial^2\overline{\psi}(x,t)}{\partial x^2}
\psi(x,t)
+
\overline{\psi}(x,t)
\frac{\partial^2\psi(x,t)}{\partial x^2}
\biggr)
\end{align*}
Somit erf"ullt die oben definierte Funktion $j(x,t)$ die
Kontinuit"atsgleichung
\[
\frac{\partial\varrho(x,t)}{\partial t}
=
\frac{\partial j(x,t)}{\partial x} +V(x)\varrho(x,t).
\]

\section*{"Ubungsaufgaben}
\rhead{"Ubungsaufgaben}
\begin{uebungsaufgaben}
\item
Quantisieren Sie das mechanische System mit der klassischen Hamilton-Funktion
\[
H(x,p)=\frac{1}{2m}p^2+ax^4.
\]
Berechnen Sie auch den Kommutator von $x$ mit dem Hamilton-Operator in
der Ortsdarstellung.

\begin{loesung}
Nach den Quantisierungsregeln in
Satz~\ref{skript:quantisierungsregeln-ortsdarstellung}
muss die Variable $p$ durch den Differentialoperator
\[
\hat p=\frac{\hbar}{i}\frac{\partial}{\partial x}
\]
ersetzt werden.
Man erh"alt
\[
\hat H
=
-\frac{\hbar^2}{2m}\frac{\partial^2}{\partial x^2}+ax^4
\]
als Hamilton-Operator.
Den Kommutator $[x,\hat H]$ in der Ortsdarstellung ermitteln wir durch 
seine Wirkung auf eine Wellenfunktion $\psi(x)$:
\begin{align*}
[x,\hat H]\psi(x)
&=
x\biggl(
-\frac{\hbar^2}{2m}\frac{\partial^2}{\partial x^2}+ax^4
\biggr)\psi(x)
-
\biggl(
-\frac{\hbar^2}{2m}\frac{\partial^2}{\partial x^2}+ax^4
\biggr)x\psi(x)
\\
&=
-\frac{\hbar^2}{2m}\biggl(
x\psi''(x)
-
x\psi''(x)
-
2\psi'(x)
\biggr)
=
\frac{\hbar^2}{m}\psi'(x)
=
-\frac{\hbar}{i}\frac1{m}\frac{\hbar}{i}\frac{\partial}{\partial x}\psi(x)
=
-\frac{\hbar}{i}\frac1{m}\hat p\psi(x).
\end{align*}
Zu Beginn der zweiten Zeile brauchen wir die Beziehung
\[
\frac{\partial^2}{\partial x^2}(x\psi(x))
=
\frac{\partial}{\partial x}\biggl(
\psi(x)+x\psi'(x)
\biggr)
=
\psi'(x)+\psi'(x)+x\psi''(x)
=
x\psi''(x)+2\psi'(x).
\]
Die Quantisierungsregeln aus Satz~\ref{skript:quantisierung-poisson}
besagen, dass Kommutatoren mit Poisson-Klammern in Beziehung stehen:
\begin{align*}
-\frac{i}{\hbar}
[x,\hat H]
&=
\frac1{m}\hat p.
\end{align*}
Der Operator auf der rechten Seite ist der Operator f"ur die
Geschwindigkeit,
und tats"achlich muss der Kommutator mit $\hat H$ ja die Zeitwentwicklung
der Observablen $x$ ausdr"ucken.
\end{loesung}


\item
Man berechne die Energieniveaus eines Teilchens der Masse $m$ in einem
Potential $V(x)=v|x|$.

\begin{loesung}
Die Quantisierungsregeln f"uhren auf die Schr"odingergleichung
\[
-\frac{\hbar^2}{2m}\psi''(x)+v|x|\psi(x)=E\psi(x).
\]
Wir k"onnen die L"osung $\psi(x)$ f"ur die ganze relle Achse gewinnen,
indem wir eine L"osung f"ur positive $x$ gerade oder ungerade
auf die negative reelle Achse fortsetzen. 
Ersteres funktioniert nur, wenn $\psi'(0)=0$ ist, zweites nur
wenn $\psi(0)=0$.

Wenn wir aber nur L"osungen f"ur $x>0$ suchen, dann k"onnen wir
auf das Betragszeichen verzichten.
Wir suchen also L"osungen der Differentialgleichung
\begin{equation}
-\frac{\hbar^2}{2m}\psi''(x)+vx \psi(x)=E\psi(x),
\label{06002:dgl}
\end{equation}
die f"ur $x\to\infty$ beschr"ankt bleiben und die oder deren Ableitung
bei $x=0$ verschwindet.
\begin{table}
\centering
\begin{tabular}{|>{$}c<{$}|>{$}c<{$}|}
\hline
n&x_n\\
\hline
\phantom{0}1&-\phantom{0}1.018793\\
\phantom{0}2&-\phantom{0}2.338107\\
\phantom{0}3&-\phantom{0}3.248198\\
\phantom{0}4&-\phantom{0}4.087949\\
\phantom{0}5&-\phantom{0}4.820099\\
\phantom{0}6&-\phantom{0}5.520560\\
\phantom{0}7&-\phantom{0}6.163307\\
\phantom{0}8&-\phantom{0}6.786708\\
\phantom{0}9&-\phantom{0}7.372177\\
10&-\phantom{0}7.944134\\
11&-\phantom{0}8.488487\\
12&-\phantom{0}9.022651\\
13&-\phantom{0}9.535449\\
14&-10.040174\\
15&-10.527660\\
16&-11.008524\\
17&-11.475057\\
18&-11.936016\\
\hline
\end{tabular}
\caption{Nullstellen von $\operatorname{Ai}(x)$ und den
Ableitungen $\operatorname{Ai}'(x)$.
F"ur $n$ ungerade ist $x_n$ eine Nullstelle von $\operatorname{Ai}'(x)$,
f"ur $n$ gerade eine von $\operatorname{Ai}(x)$.
\label{skript:airy-nullstellen}}
\end{table}

Die Differentialgleichung kann durch Verschiebung des Arguments $x$ 
und durch Skalierung in die Form
\[
y''-xy=0
\]
gebracht, werden, die Airysche Differentialgleichungen. 
\index{Airy!Differentialgleichung}
Ihre L"osungen sind Linearkombinationen der Airy-Funktionen
\index{Airy!Funktion}
$\operatorname{Ai}(x)$ und $\operatorname{Bi}(x)$ (siehe zum Beispiel
\cite{skript:airy})..
Da $\operatorname{Bi}(x)$ f"ur $x\to\infty$ unbeschr"ankt
w"achst, darf $\operatorname{Bi}(x)$ in der L"osung nicht
vorkommen.
Daraus folgt, dass die L"osung bis auf die Normierung von der
Form
\[
\psi(x)=\operatorname{Ai}(ax-b)
\]
sein muss.
Dies setzen wir jetzt in die Differentialgleichung~(\ref{06002:dgl}) ein,
und erhalten die Gleichung
\begin{align*}
-\frac{\hbar^2}{2m}a^2\operatorname{Ai}''(ax-b)
+
vx\operatorname{Ai}(ax-b)
&=
E\operatorname{Ai}(ax-b)
\\
-\frac{\hbar^2}{2m}a^2(ax-b)\operatorname{Ai}(ax-b)
+
vx\operatorname{Ai}(ax-b)
&=
E\operatorname{Ai}(ax-b)
\\
-\frac{\hbar^2}{2m}a^2(ax-b)
+
vx
&=
E
\end{align*}
Beide Seiten dieser Gleichung sind Polynome, durch Koeffizientenvergleich
finden wir daher die Bedingungen f"ur $a$ und $b$
\begin{align*}
-\frac{\hbar^2}{2m}a^3+v&=0
&
-\frac{\hbar^2}{2m}a^2b&=E
\\
\Rightarrow\qquad
a &= \biggl(\frac{\hbar^2}{2mv}\biggr)^{-\frac13}
&
\Rightarrow\qquad
b &= -\frac{E}{v}\biggl(\frac{\hbar^2}{2mv}\biggr)^{1-\frac23}
= -\frac{E}{v}\biggl(\frac{\hbar^2}{2mv}\biggr)^{-\frac13}
\end{align*}
Damit die L"osung f"ur $x>0$ sich zu einer L"osung f"ur alle $x$
erweitern l"asst, muss $\psi(0)=0$ oder $\psi'(0)=0$ sein,
es muss also $b$ eine Nullstelle von $\operatorname{Ai}(x)$ oder
von $\operatorname{Ai}'(x)$ sein.
Sind $x_n$ die Nullstellen von $\operatorname{Ai}(x)$ und
$\operatorname{Ai}'(x)$ mit $x_{n+1}<x_n$, dann sind die zugeh"origen
Energieniveaus:
\[
E_n=v\root 3\of{\frac{\hbar^2}{2mv}} x_n.
\]
Auch die Wellenfunktionen kann man angeben.
F"ur ungerade $n$ ist $x_n$ eine Nullstelle von $\operatorname{Ai}'(x)$,
die Wellenfunktion ist gerade:
\begin{equation}
\psi(x)
=
\operatorname{Ai}\biggl( \root3\of{\frac{2mv}{\hbar^2}}(|x|-x_n) \biggr).
\end{equation}
F"ur gerade $n$ ist $x_n$ eine Nullstelle von $\operatorname{Ai}(x)$
und die Wellenfunktion ist ungerade
\begin{equation}
\psi(x)
=
\begin{cases}
\displaystyle
\phantom{-}\operatorname{Ai}\biggl( \root3\of{\frac{2mv}{\hbar^2}} (x-x_n)\biggr)
	&\qquad x\ge 0\\
\displaystyle
-\operatorname{Ai}\biggl( \root3\of{\frac{2mv}{\hbar^2}}(x-x_n)\biggr)
	&\qquad x < 0
\end{cases}
\end{equation}
\begin{figure}
\centering
\includegraphics{graphics/linear-1.pdf}
\caption{Wellenfunktionen eines Teilchens in einem linearen Potential.
\label{skript:linear-wave-functions}}
\end{figure}%
In Abbildung~\ref{skript:linear-wave-functions} sind die Wellenfunktionen
f"ur die ersten 18 Energieniveaus dargestellt.
Die zugeh"origen Nullstellen sind in der Tabelle~\ref{skript:airy-nullstellen}
zusammengestellt.
Je h"oher die Energie eines Zustands ist, desto h"oher ist die Frequenz
der Wellenfunktion in der N"ahe von $0$.
Da die Differenz zwischen aufeinanderfolgenden Nullstellen immer kleiner
wird, werden auch die Differenzen zwischen den Energieniveaus immer kleiner.
\end{loesung}


\end{uebungsaufgaben}


