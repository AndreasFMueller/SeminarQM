\chapter{Algorithmus von Simon\label{chapter:simon}} 
\lhead{Algorithmus von Simon} 
\begin{refsection} 
\chapterauthor{Marc Juchli und Kirusanth Poopalasingam}

\section{Einleitung} 

In diesem Kapitel wird der Algorithmus von Simon vorgestellt.  Es wird zuerst
gezeigt, welches Problem sich damit l"osen l"asst und wie dieses Problem auf
einem klassischen Computer gel"ost werden kann. Danach wird der eigentlich
Algorithmus gezeigt. Schliesslich wird noch gezeigt, warum dieser Algorithums so
bedeutend ist.

\section{Problem} 

Gegeben ist eine unbekannte Funktion $f$, welche einen Bit String entgegennimmt
und einen neuen Bit String liefert.  Formal ausgedrückt bedeutet dies:
\[
    f\colon\{0,0\}^n\to\{0,1\}^n
\]

Die Defintion der Funktion ist auch bekannt:
\[
    f(x) = x \notin s
\]

Es wird einer 


Von der Funktion $f$ ist bekannt, dass sie bijektiv ist und dass es ein
$s\in\{0,1\}^n$ gibt, so dass: \[ f(x) = f(x) \leftrightarrow x \oplus y = s \]

Die Funktion hat damit zwei Urbilder falls $s \notin 0^n$, wobei $s$ die
Periode darstellt. Die Aufgabe ist nun $s$ zu finden.

\section{L"osung}

\subsection{Klassisch}

Auf einem klassischen Computer k"onnte diese Aufgabe mit durchprobieren gel"ost
werden. Dabei wird f"ur alle $x$ das Resultut $f(x)$ ausgewertet, solange ein
Resultate doppelt vorkommt. Mit dem vorhandenen $x$ und $x + s$ kann nun $s$
bestimmt werden. Dabei m"ussen $2^{n/2}$ Werte ausprobiert werden.

\subsection{Quantencomputer}

Auf einem Quantencomputer kann dieses Problem effizienter gel"ost werden.
Daf"ur wurde ein Algorithmus von Simon vorgestellt, im Folgenden wird dieser
genauer erklärt.

% TODO Kiru: Bild hier einfügen

F"ur den Algorithmus sind zwei Register notwendig. Als erstes werden diese mit
0 initialisiert.

\[ 
    |a\rangle=|0 ... 0 \rangle 
\]
\[
    |b\rangle=|0 ... 0 \rangle 
\]

Nun wird die Hadamard-Transformation auf dem ersten Register
angewendet, um diese in eine Superposition von allen möglichen Werten von $0
- 2^n$ zu bringen.

\[ 
    H|a\rangle=\frac{1}{\sqrt{2^n}} \sum_{x\in\{0,1\}^n}{|x\rangle}
\]
 
Nun wird die unbekannte Funktion $f$ angewendet und das Resultat $f(x)$ ins
zweite Register gespeichert. Die beiden Register befinden sich nun im Zustand:

\begin{align*}
  & |a\rangle = \frac{1}{\sqrt{2^n}} \sum_{x\in\{0,1\}^n} {|x\rangle} \\
  & |b\rangle = |f(x)\rangle \\
\end{align*}

Nun wird das zweite Register gemessen. Das Resultat von $f(x)$ ist irrelevant.
Was wichtiger ist, ist der Zustand des ersten Registers.  Dieser befindet sich
im Falle von $s\in0^n$ im Zustand:

\[
    |a\rangle = |f(x)\rangle
\]

Im Falle von $s \in \{0,1\}^n$ befindet sich dieser im Zustand:
\[
    |a\rangle = \frac{1}{\sqrt{2}} ( |x\rangle + |x \oplus s \rangle )
\]

Falls nun das erste Register gemessen wird, liefert diese entweder $x$ oder $x
\oplus s$. Um bei der Messung nur das $x \oplus s$ herauszumessen wird nun
wieder die Hadamard Operation angewendet und wie folgt vereinfacht:

\begin{align*}
    &H^{ \otimes n } \biggl( 
                     \frac{1}{\sqrt{2}} |x\rangle + 
                     \frac{1}{\sqrt{2}} |x \oplus s\rangle 
                     \biggr)
    = H^{ \otimes n } \frac{1}{\sqrt{2}} |x\rangle + 
       H^{ \otimes n } \frac{1}{\sqrt{2}} |x \oplus s\rangle 
    \\ 
    &= \frac1{\sqrt{2}} ( H^{ \otimes n } |x\rangle + H^{ \otimes n } |x \oplus s\rangle )
    \\
    &= \frac1{\sqrt{2}}
       \biggl( \frac1{\sqrt{2}^n} \sum_{z \in \{0,1\}^n} {( (-1)^{x \cdot z} |z\rangle )} + 
               \frac1{\sqrt{2}^n}  \sum_{z \in \{0,1\}^n} { ( (-1)^{(x \oplus s) \cdot z } |z\rangle)}
       \biggr)
    \\
    &= \frac1{\sqrt{2^{n + 1}}}
       \biggl( \sum_{z \in \{0,1\}^n}  { 
                   (-1)^{x \cdot z} |z\rangle + (-1)^{(x \oplus s) \cdot z } |z\rangle 
               } 
       \biggr)
    \\
    &= \frac1{\sqrt{2^{n + 1}}}
       \biggl( \sum_{z \in \{0,1\}^n}  { 
                  (-1)^{x \cdot z} |z\rangle + (-1)^{(x \cdot z) \oplus ( s \cdot z) } |z\rangle 
               } 
       \biggr)
    \\
    &= \frac1{\sqrt{2^{n + 1}}}
       \biggl( 
          \sum_{z \in \{0,1\}^n}  { (-1)^{x \cdot z} ( 1 + (-1)^{ s \cdot z}) |z\rangle } 
       \biggr)
    \\
\end{align*}

Es werden nun zwei F"alle unterschieden. Falls $s \cdot z = 1$ ist, dann gilt:
\begin{align*}
    &\frac1{\sqrt{2^{n + 1}}}
      \sum_{z \in \{0,1\}^n}  { (-1)^{x \cdot z} ( 1 + (-1)^{ s \cdot z}) |z\rangle } 
    \\
    &= 
    \frac1{\sqrt{2^{n + 1}}}
      \sum_{z \in \{0,1\}^n}  { (-1)^{x \cdot z} ( 1 + (-1)^1) |z\rangle } 
    \\
    &= 
    \frac1{\sqrt{2^{n + 1}}}
      \sum_{z \in \{0,1\}^n}  { (-1)^{x \cdot z} (0) |z\rangle } 
    \\
    &=
    0 |z\rangle
\end{align*}

Dies zeigt, dass die Wahrscheinlichkeit diesen Fall zu messen gleich 0 ist und
somit nie auftritt. Der zweite Fall ist $s \cdot z = 0$, dann gilt:

\begin{align*}
    &\frac1{\sqrt{2^{n + 1}}}
      \sum_{z \in \{0,1\}^n}  { (-1)^{x \cdot z} ( 1 + (-1)^{ s \cdot z}) |z\rangle } 
    \\
    &= 
    \frac1{\sqrt{2^{n + 1}}}
      \sum_{z \in \{0,1\}^n}  { (-1)^{x \cdot z} ( 1 + (-1)^0) |z\rangle } 
    \\
    &= 
    \frac1{\sqrt{2^{n + 1}}}
      \sum_{z \in \{0,1\}^n}  { (-1)^{x \cdot z} (2) |z\rangle } 
    \\
    &= 
    \frac{2}{\sqrt{2^{n + 1}}} 
      \sum_{z \in \{0,1\}^n}  { (-1)^{x \cdot z}|z\rangle } 
    &= 
    \frac1{\sqrt{2^{n - 1}}} 
      \sum_{z \in \{0,1\}^n}  { (-1)^{x \cdot z}|z\rangle } 
\end{align*}

Bei der Messung des ersten Registers wird nun ein n-String $z$ gemessen mit der
Bedingung, dass $s \cdot z = 0$ ist. Die Wahrscheinlichkeit daf"ur liegt bei:
\[
    |\frac1{\sqrt{2^{n - 1}}}|^2 = \frac1{2^{n-1}} 
\]

Bei genauer "Uberglegung macht dies auch Sinn, denn falls $s \notin \{0,1\}^n$
ist, gibt es genau $2^{n-1}$ M"oglich $z$-Werte. Die Wahrscheinlichkeit einen
solchen zu messen ist also gleich-verteilt.

Der Algorithmus von Simon liefert nun einen Wert $z$. Um den genauen Wert von
$s$ zu bestimmen, wird nun der Algorithmus $n-1$-Mal wiederholt, somit liegen
nun $n-1$ Gleichung vor und der Wert $s$ kann mittels Gauss berechnet werden.

\section{Schlussfolgerung}

Der Algorithmus von Simon zeigt auf, dass mittels eines Quantencomputers die
Anzahl der Aufrufe der Funktion $f$ auf $O(n)$ reduziert werden k"onnen. Die
Anzahl der Schritte, welche f"ur die effektive Berechnung von $s$ ben"otigt
werden, h"angt vom l"osen des linearen Gleichungssystems mit Gaußscher
Elimination ab, also $O(n^3)$.

%\section{Warum ist dies relevant?}

\printbibliography[heading=subbibliography] \end{refsection}


