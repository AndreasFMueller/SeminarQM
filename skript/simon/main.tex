\chapter{Algorithmus von Simon\label{chapter:simon}} 
\lhead{Algorithmus von Simon} 
\begin{refsection} 
\chapterauthor{Marc Juchli und Kirusanth Poopalasingam}

\section{Einleitung} 

In diesem Kapitel wird der Algorithmus von Simon vorgestellt.  Es wird zuerst
gezeigt, welches Problem sich damit l"osen l"asst, wie dieses Problem auf einem
klassischen Computer gel"ost werde kann. Danach wird der eigentlich Algorithmus
gzeigt. Schliess wird noch gezeigt, warum dieser Algorithums so bedeutend ist.

\section{Problem} 

Gegeben ist eine unbekannte Funktion $f$, welche einen String
von Bits $x\in\{0,1\}^n$ in ein String von Bits $y\in\{0,1\}^n$ \"uberführt:
\[
    %|x s\rangle \mapsto |x (s\oplus f(x))\rangle.
    f\colon\{0,0\}^n\to\{0,1\} 
\]
Von der Funktion ist bekannt, dass sie bijektiv ist und dass es ein
$s\in\{0,1\}^n$ gibt, so dass: \[ f(x) = f(x) \leftrightarrow x \oplus y = s \]

Die Funktion hat damit zwei Urbilder falls $s \notin 0^n$, wobei $s$ die
Periode darstellt. Die Aufgabe ist nun $s$ zu finden.

Auf einem klassischen Computer k"onnte diese Aufgabe mit durchprobieren gel"ost
werden. Dabei wird f"ur alle $x$ das Resultut $f(x)$ ausgewertet, solange ein
Resultate doppelt vorkommt. Mit dem vorhandenen $x$ und $x + s$ kann nun $s$
bestimmt werden. Dabei m"ussen $2^{n/2}$ Werte ausprobiert werden.

 Auf einem Quantencomputer kann dieses Problem effizienter gel"ost werden.
Daf"ur wurde ein Algorithmus von Simon vorgestellt, im Folgenden wird dieser
genauer erklärt.

% TODO Kiru: Bild hier einfügen

F"ur den Algorithmus sind zwei Register notwendig. Als erstes werden diese mit
0 initialisiert.

\[ 
    |a\rangle=|0 ... 0 \rangle 
\]
\[
    |b\rangle=|0 ... 0 \rangle 
\]

Nun wird die Hadamard-Transformation auf den ersten Register
angewendet, um diese in eine Superposition von allen möglichen Werten von $0
- 2^n$ zu bringen.  

\[ 
    H|a\rangle=\frac{1}{\sqrt{2^n}} \sum_{x\in\{0,1\}^n}{|x\rangle}
\]
 
Nun wird die unbekannte Funktion $f$ angewendet und das Resultat $f(x)$ ins
zweite Register gespeichert. Die beiden Register befinden sich nun im Zustand:

\[ 
    |a\rangle = \frac{1}{\sqrt{2^n}} \sum_{x\in\{0,1\}^n} {|x\rangle} 
\]
\[
    |b\rangle = |f(x)\rangle
\]

Nun wird das zweite Register gemessen. Das Resultat von $f(x)$ ist irrelevant.
Was wichtiger ist, ist der Zustand des ersten Reigsters.  Dieser befinden sich
im Falle von $s\in0^n$ im Zustand:

\[
    |a\rangle = |f(x)\rangle
\]

Im Falle von $s \in \{0,1\}^n$ befindet sich dieser im Zustand:
\[
    |a\rangle = \frac{1}{\sqrt{2}} ( |x\rangle + |x \oplus s \rangle )
\]

Falls nun das erste Register gemessen wird, liefert diese entweder $x$ oder $x
\oplus s$. Um bei der Messung nur das $x \oplus s$ herauszumessen wird nun
wieder die Hadamard Operation angewendet und wie folgt vereinfacht:

F"ur ein $n$-Qubit-Register $|x\rangle$ gilt analog
\begin{align*}
H^{\otimes n}\, |x\rangle
=
\frac1{(\sqrt{2})^n}
\sum_y (-1)^{x\cdot y}|y\rangle
\end{align*}

\begin{align*}
    &H^{ \otimes n } \biggl( 
                     \frac{1}{\sqrt{2}} |x\rangle + 
                     \frac{1}{\sqrt{2}} |x \oplus s\rangle 
                     \biggr)
    \\
    &= H^{ \otimes n } \frac{1}{\sqrt{2}} |x\rangle + 
       H^{ \otimes n } \frac{1}{\sqrt{2}} |x \oplus s\rangle 
    \\ 
    &= \frac1{\sqrt{2}^n} \sum_{z \in \{0,1\}^n} {( (-1)^{x \cdot z} |z\rangle )} + 
       \frac1{\sqrt{2}^n}  \sum_{z \in \{0,1\}^n} { ( (-1)^{(x \oplus s) \cdot z } |z\rangle)}
    \\
    &= \frac1{\sqrt{2}^n} \sum_{z \in \{0,1\}^n} {( (-1)^{x \cdot z} |z\rangle )} + 
       \frac1{\sqrt{2}^n}  \sum_{z \in \{0,1\}^n} { ( (-1)^{(x \oplus s) \cdot z } |z\rangle)}
\end{align*}


% \begin{align*}
% |x_1\rangle \otimes |x_2\rangle
% &\mapsto
% \frac1{\sqrt{2}} \biggl( |0\rangle + (-1)^{x_1}|1\rangle \biggr)
% \frac1{\sqrt{2}} \biggl( |0\rangle + (-1)^{x_2}|1\rangle \biggr)
% \\
% &=
% \frac12(
% |00\rangle
% +
% (-1)^{x_1}
% |10\rangle
% +
% (-1)^{x_2}
% |01\rangle
% +
% (-1)^{x_1+x_2}
% |11\rangle
% )
% \\
% &=
% \frac12
% \sum_{y} (-1)^{x\cdot y} |y\rangle.
% \end{align*}
% Die zugeh"orige unit"are Matrix in der Basis
% $\{|00\rangle,
% |10\rangle,
% |01\rangle,
% |11\rangle\}$
% ist
% 
\section{Klassische Lösugn}

\section{L\"osung mit einem Quantencomputer}

\section{Warum ist dies relevant?}

    % u=\frac1{2^{\frac{n}2}}\sum_{x\in\{0,1\}^n}|x\rangle
\printbibliography[heading=subbibliography] \end{refsection}


