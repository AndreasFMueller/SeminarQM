\chapter{Algorithmus von Simon\label{chapter:simon}} 
\lhead{Algorithmus von Simon} 
\begin{refsection} 
\chapterauthor{Marc Juchli und Kirusanth Poopalasingam}

\section{Einleitung} 

In diesem Kapitel wird der Algorithmus von Simon vorgestellt.  Es wird zuerst
gezeigt, welches Problem sich damit l"osen l"asst und wie dieses Problem auf
einem klassischen Computer gel"ost werden kann. Danach wird der eigentliche
Algorithmus gezeigt. Schliesslich wird noch gezeigt, warum dieser Algorithmus so
bedeutend ist.

\section{Problem} 

Gegeben ist eine unbekannte Funktion $f$, welche einen n-Bit langen String
entgegen nimmt und einen neuen Bit String liefert \cite{simon:cse599d-dave-bacon}. Formal ausgedr"uckt bedeutet
dies:

\[
    f\colon\{0,1\}^n\to\{0,1\}^n
\]


Die Defintion der Funktion ist auch bekannt:

\[
    f(x) = x \oplus s
\]

% Von Interesse sind Funktionen mit der zusätzlichen Eigenschaft von
% Periodizität.

% [Grafik 1:1]

% Nur dies mit dem $s$ reicht nicht aus.

% Des weiteren müssen folgende Einschränkungen bezüglich der Blackbox Funktion
% zu .. : 2 zu 1 Funktion.

% [Grafik 2:1]

Mit der Eingabe wird also eine Und-Verkn"ufung mit einer unbekannten Konstante
$s$ gemacht.  Gesucht ist nun $s$. Fall $s = 0^n$ ist, dann liefert diese
Funktion f"ur jeden Wert von $x$ jeweils unterschiedliche Werte von $f(x)$, sie
ist also bijektiv.

Falls $s$ sonst was anders ist, also $s \in \{0,1\}^n, s \neq 0^n$, liefert die
Funktion f"ur zwei $x$ denselben Wert $f(x)$. Formal kann dies so definiert
werden:

\[ 
    f(x) = f(y) \leftrightarrow x \oplus y = s 
\]

In diesem Fall ist die Funktion also nicht bijektiv, jedoch periodisch, wobei
$s$ die Period darstellt. Der Kern des Problem ist, dass eine m"oglicherweise
periodische Funktion gegeben ist und die Periode gesucht ist.

Beispiel f"ur einer Funktion mit $s = 0^n$ und $n = 3$:

\begin{figure}[H]
 \centering
 \begin{tikzpicture}[ele/.style={fill=black,circle,minimum width=1pt,inner sep=1pt},every fit/.style={ellipse,draw,inner sep=-2pt}]
  \node [] {$x$};
  \node [xshift=4.0cm] {$f(x)$};
  \node[ele,label=left:$000$] (a1) at (0,4.5) {};
  \node[ele,label=left:$001$] (a2) at (0,4) {};
  \node[ele,label=left:$010$] (a3) at (0,3.5) {};
  \node[ele,label=left:$011$] (a4) at (0,3) {};
  \node[ele,label=left:$100$] (a5) at (0,2.5) {};
  \node[ele,label=left:$101$] (a6) at (0,2) {};
  \node[ele,label=left:$110$] (a7) at (0,1.5) {};
  \node[ele,label=left:$111$] (a8) at (0,1) {};

  \node[ele,,label=right:$000$] (b1) at (4,4.5) {};
  \node[ele,,label=right:$001$] (b2) at (4,4) {};
  \node[ele,,label=right:$010$] (b3) at (4,3.5) {};
  \node[ele,,label=right:$011$] (b4) at (4,3) {};
  \node[ele,,label=right:$100$] (b5) at (4,2.5) {};
  \node[ele,,label=right:$101$] (b6) at (4,2) {};
  \node[ele,,label=right:$110$] (b7) at (4,1.5) {};
  \node[ele,,label=right:$111$] (b8) at (4,1) {};

  \node[draw,fit= (a1) (a2) (a3) (a4) (a5) (a6) (a7) (a8),minimum width=3cm] {} ;
  \node[draw,fit= (b1) (b2) (b3) (b4) (b5) (b6) (b7) (b8),minimum width=3cm] {} ;
  \draw[->,thick,shorten <=2pt,shorten >=2pt] (a1) -- (b1);
  \draw[->,thick,shorten <=2pt,shorten >=2] (a2) -- (b2);
  \draw[->,thick,shorten <=2pt,shorten >=2] (a3) -- (b3);
  \draw[->,thick,shorten <=2pt,shorten >=2] (a4) -- (b4);
  \draw[->,thick,shorten <=2pt,shorten >=2] (a5) -- (b5);
  \draw[->,thick,shorten <=2pt,shorten >=2] (a6) -- (b6);
  \draw[->,thick,shorten <=2pt,shorten >=2] (a7) -- (b7);
  \draw[->,thick,shorten <=2pt,shorten >=2] (a8) -- (b8);
 \end{tikzpicture}
\end{figure}

Eine periodische Funktion mit $s = 110$ ist folglich:

\begin{figure}[H]
 \centering
 \begin{tikzpicture}[ele/.style={fill=black,circle,minimum width=1pt,inner sep=1pt},every fit/.style={ellipse,draw,inner sep=-2pt}]
  \node [yshift=0cm] {$x$};
  \node [xshift=4.0cm, yshift=2cm] {$f(x)$};
  \node[ele,label=left:$000$] (aa1) at (0,4.5) {};
  \node[ele,label=left:$001$] (aa2) at (0,4) {};
  \node[ele,label=left:$010$] (aa3) at (0,3.5) {};
  \node[ele,label=left:$011$] (aa4) at (0,3) {};
  \node[ele,label=left:$100$] (aa5) at (0,2.5) {};
  \node[ele,label=left:$101$] (aa6) at (0,2) {};
  \node[ele,label=left:$110$] (aa7) at (0,1.5) {};
  \node[ele,label=left:$111$] (aa8) at (0,1) {};

  \node[ele,,label=right:$000$] (bb1) at (4,4.5) {};
  \node[ele,,label=right:$001$] (bb2) at (4,4) {};
  \node[ele,,label=right:$010$] (bb3) at (4,3.5) {};
  \node[ele,,label=right:$011$] (bb4) at (4,3) {};

  \node[draw,fit= (aa1) (aa2) (aa3) (aa4) (aa5) (aa6) (aa7) (aa8),minimum
  width=3cm] {} ;
  \node[draw,fit= (bb1) (bb2) (bb3) (bb4), minimum width=3cm] {} ;
  \draw[->,thick,shorten <=2pt,shorten >=2pt] (aa1) -- (bb1);
  \draw[->,thick,shorten <=2pt,shorten >=2] (aa2) -- (bb2);
  \draw[->,thick,shorten <=2pt,shorten >=2] (aa3) -- (bb3);
  \draw[->,thick,shorten <=2pt,shorten >=2] (aa4) -- (bb4);
  \draw[->,thick,shorten <=2pt,shorten >=2] (aa5) -- (bb1);
  \draw[->,thick,shorten <=2pt,shorten >=2] (aa6) -- (bb2);
  \draw[->,thick,shorten <=2pt,shorten >=2] (aa7) -- (bb3);
  \draw[->,thick,shorten <=2pt,shorten >=2] (aa8) -- (bb4);
 \end{tikzpicture}
\end{figure}

\section{L"osung}

\subsection{Klassisch}

Wie w"urde man dise Aufgabe auf einem klassichen Computer l"osen? Ein naiver
Ansatz w"are alle m"oglichen Werte auszuprobieren. Die folgende Tabelle zeigt
f"ur eine Beispielfunktion $f$ die Resultate.


\begin{center}
   \begin{tabular}{ l | c  }
    \hline
     x   & f(x) \\ \hline
     000 & 000  \\ \hline
     001 & 001  \\ \hline
     010 & 010  \\ \hline
     011 & 011  \\ \hline
     100 & 000  \\ \hline
     101 & 001  \\ \hline
     110 & 010  \\ \hline
     111 & 011  \\ \hline
     \hline
    \end{tabular}
\end{center}


Was ist nun die Periode? Die Frage l"asst sich einfach beantworten. Es m"ussen
nur zwei Werte $f(x)$ gefunden werden, die identisch sind. In diesem Beispiel
sind es $f(000) = 000$ und $f(100) = 000$. 

Nun kommt die obige Gleichung zum Zug:

\[
    f(000) = f(100) \leftrightarrow 000 \oplus 100 = s = 011
\] \

Wie effizient ist dieser Algorithmus? Es m"ussen alle M"oglichen Eingaben
ausprobiert werden. Bei der L"ange $3$ wurden $2^3$ Inputs generiert. Bei einem
$n$ Bit String werden also $2^n$ Inputs generiert. 

Bei genauer "Uberlegung war beim Beispiel nur $f(000)$ und $f(001)$ relevant.
Die Funktion muss nur solange aufgerufen werden, bis zwei Werte identisch sind.
Es gen"ugt daher $n^{1/2}$ Werte auszuprobieren.

% TODO Grafik -> Wachstum mit n^1/2
% TDOO Grafik kommentieren und "uberleitung zum Quantencomputer

\subsection{Quantencomputer}

Auf einem Quantencomputer kann dieses Problem effizienter gel"ost werden.
Daf"ur wurde ein Algorithmus von Daniel R. Simon vorgestellt, im Folgenden wird
dieser genauer erkl"art.

% TODO Bild vom Schaltkreis.

Die Grundidee des Algorithmus ist es alle M"oglichen Werte von $x$ auf einmal
auszuprobieren und zwei Werte f"ur $f(x)$ zu finden, die identisch sind.

F"ur den Algorithmus sind zwei Register notwendig. Als erstes werden diese mit
0 initialisiert.
\begin{align*}
  |a\rangle&=|0 \dots 0 \rangle 
  &
  |b\rangle&=|0 \dots 0 \rangle 
\end{align*}

Nun wird die Hadamard-Transformation auf dem ersten Register angewendet. Damit
befindet sich das Register in einer Superpostion, bei welcher alle m"oglichen
Werte von $x$ mit der gleichen Wahrscheinlichkeit auftreten. Sinnhaft
gesprochen heisst das, dass auf einmal alle m"oglichen Werte auf dem ersten
Register erzeugt werden.
\[ 
    H|a\rangle=\frac{1}{\sqrt{2^n}} \sum_{x\in\{0,1\}^n}{|x\rangle}
\]
Dieser Wert wird jetzt im ersten Register gespeichert.
\begin{align*}
  |a\rangle &= \frac{1}{\sqrt{2^n}} \sum_{x\in\{0,1\}^n} {|x\rangle} &
  |b\rangle&=|0 \dots 0 \rangle 
\end{align*}

Nun wird die unbekannte Funktion $f$ angewendet und das Resultat $f(x)$ ins
zweite Register gespeichert. Die beiden Register befinden sich nun im Zustand:

\begin{align*}
  |a\rangle &= \frac{1}{\sqrt{2^n}} \sum_{x\in\{0,1\}^n} {|x\rangle} &
  |b\rangle &= |f(a)\rangle
\end{align*}

Im ersten Register sind alle m"oglichen Werte $x$ repr"asentiert und im zweiten
Register deren $f(a)$. Nun wird das zweite Register gemessen, der gemessene Wert
sei $y$. 
Gleichzeitig wird auch das erste Register $|a\rangle$ beeinflusst.
M"ogliche Werte in diesem Register sind alle die von $f$ auf $y$ abgebildet
werden. Weil $f$ 2--1 und periodisch mit Periode $s$ ist, gibt es genau zwei
m"ogliche Werte $x$ und $x \oplus s$. Das erste Register befindet sich folglich
in einem "Uberlagerungszustand.
\begin{align*}
  |a\rangle &= \frac{1}{\sqrt{2}} \bigl( |x\rangle + |x \oplus s \rangle \bigr)
  &
  |b\rangle &= |y\rangle
\end{align*}
Dies sagt aus, dass bei der Messung des ersten Registers entweder $x$ oder $x
\oplus s$ herausgelesen wird. Das Problem ist nun, dass es sich nicht
unterscheiden l"asst, um welchen Wert es sich handelt.
Falls $x$ gemessen wird ist nicht klar, dass dieser nicht $x \oplus s$ ist. 
Es hilft auch nicht $s$ zu finden.

Um bei der Messung den Wert $x$ nicht zu bekommen wird ein Trick verwendet.
Daf"ur wird nochmals die Hadamard-Operation auf dem ersten Register angewendet.

\begin{hilfssatz}
  Der Wert der Hadamard-Operation auf dem Register $|a\rangle$ ist
  \begin{align}
    H^{ \otimes n } \biggl( 
                     \frac{1}{\sqrt{2}} |x\rangle + 
                     \frac{1}{\sqrt{2}} |x \oplus s\rangle 
                     \biggr)
    &= \frac1{\sqrt{2^{n + 1}}}
       \biggl( 
          \sum_{z \in \{0,1\}^n}  { (-1)^{x \cdot z} ( 1 + (-1)^{ s \cdot z}) |z\rangle } 
       \biggr).
  \label{simon:smalz}
\end{align}
\end{hilfssatz}

\begin{proof}[Beweis]
\begin{align*}
    H^{ \otimes n } \biggl( 
                     \frac{1}{\sqrt{2}} |x\rangle + 
                     \frac{1}{\sqrt{2}} |x \oplus s\rangle 
                     \biggr)
    &= H^{ \otimes n } \frac{1}{\sqrt{2}} |x\rangle + 
       H^{ \otimes n } \frac{1}{\sqrt{2}} |x \oplus s\rangle 
    \\ 
    &= \frac1{\sqrt{2}} ( H^{ \otimes n } |x\rangle + H^{ \otimes n } |x \oplus s\rangle )
    \\
    &= \frac1{\sqrt{2}}
       \biggl( \frac1{\sqrt{2}^n} \sum_{z \in \{0,1\}^n} {( (-1)^{x \cdot z} |z\rangle )} + 
               \frac1{\sqrt{2}^n}  \sum_{z \in \{0,1\}^n} { ( (-1)^{(x \oplus s) \cdot z } |z\rangle)}
       \biggr)
    \\
    &= \frac1{\sqrt{2^{n + 1}}}
       \biggl( \sum_{z \in \{0,1\}^n}  { 
                   (-1)^{x \cdot z} |z\rangle + (-1)^{(x \oplus s) \cdot z } |z\rangle 
               } 
       \biggr)
    \\
    &= \frac1{\sqrt{2^{n + 1}}}
       \biggl( \sum_{z \in \{0,1\}^n}  { 
                  (-1)^{x \cdot z} |z\rangle + (-1)^{(x \cdot z) \oplus ( s \cdot z) } |z\rangle 
               } 
       \biggr)
    \\
    &= \frac1{\sqrt{2^{n + 1}}}
       \biggl( 
          \sum_{z \in \{0,1\}^n}  { (-1)^{x \cdot z} ( 1 + (-1)^{ s \cdot z}) |z\rangle } 
       \biggr)
    \\
\end{align*}
\end{proof}

Das Resultat (\ref{simon:smalz}) zeigt auf, dass das Skalarprodukt $s \cdot z$ im Wesentlichen das
Resultat bestimmt. Die möglichen Werte sind $0$ oder $1$, da es sich um ein
bin"ares Skalarprodukt handelt.

\begin{hilfssatz}
  Die rechte Seite von (\ref{simon:smalz}) wird je nach Wert von $s \cdot z$
  \[
    \begin{cases}
      0 & \qquad s \cdot z = 1 \\
     \displaystyle \frac1{\sqrt{2^{n - 1}}} \sum_{z \in \{0,1\}^n}  { (-1)^{x \cdot
     z}|z\rangle } & \qquad s \cdot z = 0
    \end{cases}
  \]
\end{hilfssatz}

\begin{proof}
Falls $s \cdot z = 1$ ist, dann gilt:
\begin{align*}
    \frac1{\sqrt{2^{n + 1}}}
      \sum_{z \in \{0,1\}^n}  { (-1)^{x \cdot z} ( 1 + (-1)^{ s \cdot z}) |z\rangle } 
    &= 
    \frac1{\sqrt{2^{n + 1}}}
      \sum_{z \in \{0,1\}^n}  { (-1)^{x \cdot z} ( 1 + (-1)^1) |z\rangle } 
    \\
    &= 
    \frac1{\sqrt{2^{n + 1}}}
      \sum_{z \in \{0,1\}^n}  { (-1)^{x \cdot z} (0) |z\rangle } 
    \\
    &=
    0 |z\rangle
\end{align*}

Der zweite Fall ist $s \cdot z = 0$, dann gilt:

\begin{align*}
    \frac1{\sqrt{2^{n + 1}}}
      \sum_{z \in \{0,1\}^n}  { (-1)^{x \cdot z} ( 1 + (-1)^{ s \cdot z}) |z\rangle } 
    &= 
    \frac1{\sqrt{2^{n + 1}}}
      \sum_{z \in \{0,1\}^n}  { (-1)^{x \cdot z} ( 1 + (-1)^0) |z\rangle } 
    \\
    &= 
    \frac1{\sqrt{2^{n + 1}}}
      \sum_{z \in \{0,1\}^n}  { (-1)^{x \cdot z} (2) |z\rangle } 
    \\
    &= 
    \frac{2}{\sqrt{2^{n + 1}}} 
      \sum_{z \in \{0,1\}^n}  { (-1)^{x \cdot z}|z\rangle } 
    \\
    &= 
    \frac1{\sqrt{2^{n - 1}}} 
      \sum_{z \in \{0,1\}^n}  { (-1)^{x \cdot z}|z\rangle } 
\end{align*}
\end{proof}
Dies zeigt, dass die Wahrscheinlichkeit diesen Fall zu messen gleich 0 ist und
somit nie auftritt. 
Mit diesem Trick wird nun ein Wert $z$ gemessen mit der Eigenschaft:
\[
    s \cdot z = 0
\]

Bei der Messung des ersten Registers wird nun ein n-String langer Wert $z$
gemessen mit der Bedingung, dass $s \cdot z = 0$ ist. Die Wahrscheinlichkeit
daf"ur liegt bei:
\[
    \biggl|\frac1{\sqrt{2^{n - 1}}}\biggr|^2 = \frac1{2^{n-1}} 
\]

Bei genauer "Uberglegung macht dies auch Sinn, denn falls $s \notin \{0,1\}^n$
ist, gibt es genau $2^{n-1}$ m"oglichen $z$-Werte. Die Wahrscheinlichkeit einen
solchen zu messen ist also gleichverteilt.

Der Algorithmus von Simon liefert nun einen Wert $z$. Um den genauen Wert von
$s$ zu bestimmen, wird nun der Algorithmus $n-1$-Mal wiederholt, somit liegen
nun $n-1$ Gleichung vor und der Wert $s$ kann mittels Gauss berechnet werden.

\begin{align*}
    s \cdot z_{1} = 0
    \\
    s \cdot z_{2} = 0
    \\
    s \cdot z_{3} = 0
    \\
    ...
    \\
    s \cdot z_{n} = 0
\end{align*}

\section{Schlussfolgerung}

\begin{figure}[H]
\centering
\begin{tikzpicture}
  \draw[->] (0,0) -- (5,0) node[right] {$n$};
  \draw[->] (0,0) -- (0,5) node[above] {$\mathcal{O}{}$};
  \draw[scale=0.35,domain=-0:10,smooth,variable=\x,black] plot ({\x},{\x});
  \draw[scale=0.15,domain=-0:10,smooth,variable=\y,black]  plot ({\y},{2^(\y/2)});
  \node [xshift=4cm,yshift=4cm] {$\mathcal{O}(n^3)$};
  \node [xshift=2cm, yshift=5cm] {$\mathcal{O}(2^{n/2})$};
\end{tikzpicture}
\end{figure}

Der Algorithmus von Simon zeigt auf, dass mittels eines Quantencomputers die
Anzahl der Aufrufe der Funktion $f$ auf $O(n)$ reduziert werden k"onnen. Die
Anzahl der Schritte, welche f"ur die effektive Berechnung von $s$ ben"otigt
werden, h"angt vom l"osen des linearen Gleichungssystems mit Gaußscher
Elimination ab, also $O(n^3)$. 
\cite{simon:cs191}

Somit wurde ein Algorithmus, der auf dem klassischen Computer bei sehr grossen
Zahlen schnell w"achst ( in diesem Fall ist der Wachsumt mit $O(2^{n/2})$
angegeben ) auf eine L"osung mit einem viel kleineren Wachstum von $O(2^3)$
reduziert. In der Theoretischen Informatik ist dies ein grosser Schritt, denn
es zeigt auf, dass gewisse Problem auf dem Quantencomputer weit effizienter
gel"ost werden k"onnen als auf dem klassischen Computer.


\printbibliography[heading=subbibliography] \end{refsection}


