\chapter{Algorithmus von Simon\label{chapter:simon}} 
\lhead{Algorithmus von Simon} 
\begin{refsection} 
\chapterauthor{Marc Juchli und Kirusanth Poopalasingam}

\section{Einleitung} 

In diesem Kapitel wird ein der Algorithmus von Simon vorgestellt.  Es wird
zuerst gezeigt, welches Problem es damit lässt, wie dieses Problem auf einem
klassischen Computer gelöst werde kann.  Danach wird der eigentlich Algorithmus
gzeigt. Schliess wird noch gezeigt, warum dieser Algorithums so bedeutend ist.

\section{Problem} 

Gegeben ist eine unbekannte Funktion $f$, welche einen String
von Bits $x\in\{0,1\}^n$ in ein String von Bits $y\in\{0,1\}^n$ \"uberführt:

\[
    %|x s\rangle \mapsto |x (s\oplus f(x))\rangle.
    f\colon\{0,0\}^n\to\{0,1\} 
\]

Von der Funktion ist bekannt, dass sie bijektiv ist und dass es ein
$s\in\{0,1\}^n$ gibt, so dass: \[ f(x) = f(x) \leftrightarrow x \oplus y = s \]

Die Funktion hat damit zwei Urbilder falls $s \notin 0^n$, wobei $s$ die
Periode darstellt Die Aufgabe ist nun $s$ zu finden.

Auf einem klassischen Computer könnte diese Aufgabe mit durchprobieren gelöst
werden. Dabei wird für alle $x$ das Resultut $f(x)$ ausgewertet, solange ein
Resultate doppelt vorkommt. Mit dem vorhanden $x$ und $x + s$ kann nun $s$
bestimmt werden. Dabei müssen $2^(n/2)$ Werte ausprobiert werden.

Auf einem Quantencomputer kann dieses Problem effizienter gelöst werden. Dafür
wurde ein Algorithmus von Simon vorgestellt, im Folgenden werden dieser
genauer erklärt.


% TODO Kiru: Bild hier einfügen

Schritt 1: 

Als erste werden zwei Quantenregister mit 0 initialisiert.

\[ 
    |a\rangle = |0 ... 0 \rangle 
    |b\rangle = |0 ... 0 \rangle 
\]


Schritt 2:

Nun wird die Hadamard-Transformation au fden ersten Register
angewendet, um diese in eine Superposition von allen möglichen Werten von $0
- 2^n$ zu bringen.  

\[ 
    H|a> = 1/\sqrt{2^n}  \sum_{x\in\{0,1\}}} |x\rangle 
\]

Schritt 3: Nun wird die unbekannte Funktion $f$ angewednet und das Resultate
$f(x)$ ins zweite Register gespeicher. Die beien Register befinden sich nun im
Zustand:

\[ 
    |a> = 1/\sqrt{2^n}  \sum_{x\in\{0,1\}}} |x\rangle |b\rangle = |(f)x\rangle
\]

Schritt 4: 

Nun wird das zweite Register gemessen. Das Resultate von $f(x)$ ist
irrelevant. Was wichtiger ist, ist der Zustand des ersten Reigsters. 
Dieser befinden sich im Falle von $s\in0^n$ im Zustand

\[
    |a\rangle = |f(x)\rangle
\]

Im Falle von $s \in \{0,1\}^n$ befindet sich dieser im Zustand:
\[
    [a\rangle = 1/\sqrt{2} ( |x\rangle + |x \oplus s \rangle )
\]

% 


\section{Klassische Lösugn}

\section{L\"osung mit einem Quantencomputer}

\section{Warum ist dies relevant?}

    % u=\frac1{2^{\frac{n}2}}\sum_{x\in\{0,1\}^n}|x\rangle
\printbibliography[heading=subbibliography] \end{refsection}


