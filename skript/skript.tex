%
% skript.tex -- Skript ueber Quantenmechanik
%
% (c) 2014 Prof. Dr. Andreas Mueller, HSR
%
\documentclass{book}
\usepackage{etex}
\usepackage{geometry}
\geometry{papersize={170mm,240mm},total={140mm,200mm},top=21mm,bindingoffset=10mm}
\usepackage[ngerman]{babel}
\usepackage{times}
\usepackage{amsmath,amscd}
\usepackage{amssymb}
\usepackage{amsfonts}
\usepackage{amsthm}
\usepackage{graphicx}
\usepackage{fancyhdr}
\usepackage{textcomp}
\usepackage[all]{xy}
\usepackage{txfonts}
\usepackage{alltt}
\usepackage{verbatim}
\usepackage{paralist}
\usepackage{makeidx}
\usepackage{array}
\usepackage{hyperref}
\usepackage{tikz}
\usepackage{placeins}
\usepackage{subfigure}
\usepackage{csquotes}
\usepackage{float}
\usepackage{enumitem}
\usepackage{wasysym}
\usepackage{environ}
\usepackage{pifont}
\usepackage{appendix}
\usetikzlibrary{calc,intersections,through,backgrounds,graphs,positioning,shapes,arrows}
\usetikzlibrary{patterns,decorations.pathreplacing}
\usetikzlibrary{decorations.pathreplacing}
\usepackage{siunitx}
\usepackage{tabularx}
\usetikzlibrary{arrows}
\usepackage{listings}
\lstdefinestyle{Matlab}{
  numbers=left,
  belowcaptionskip=1\baselineskip,
  breaklines=true,
  frame=L,
  xleftmargin=\parindent,
  language=Matlab,
  showstringspaces=false,
  basicstyle=\footnotesize\ttfamily,
  keywordstyle=\bfseries\color{green!40!black},
  commentstyle=\itshape\color{purple!40!black},
  identifierstyle=\color{blue},
  stringstyle=\color{orange},
  numberstyle=\ttfamily\tiny
}
\usepackage{caption}
\usepackage{standalone}
\usepackage[backend=bibtex]{biblatex}
\addbibresource{references.bib}
% Quanteninformatik
\addbibresource{crypto/main.bib}
\addbibresource{teleport/main.bib}
\addbibresource{simon/main.bib}
% Halbleiterbauteile
\addbibresource{tunneldiode/main.bib}
\addbibresource{flash/main.bib}
% Anwendungen der Störungstheorie
\addbibresource{atomuhr/main.bib}
\addbibresource{efeld/main.bib}
\addbibresource{anharmonisch/main.bib}
% Sphaerische harmonische Analyse
\addbibresource{kugel/main.bib}
% Weitere Anwendungen der Quantenmechanik
\addbibresource{laser/main.bib}
\addbibresource{mri/main.bib}
% Supraleitung
\addbibresource{supraleitung/main.bib}
\addbibresource{bose/main.bib}

\addbibresource{heisenberg/main.bib}
\addbibresource{stark/main.bib}
\addbibresource{bell/main.bib}
\addbibresource{feldquantisierung/main.bib}
%\addbibresource{rtm/main.bib}
%\addbibresource{orbitale/main.bib}
%\addbibresource{franckhertz/main.bib}
%\addbibresource{maser/main.bib}
%\addbibresource{quantumdot/main.bib}
\AtEndDocument{\clearpage\ifodd\value{page}\else\null\clearpage\fi}
\makeindex
\begin{document}
\pagestyle{fancy}
\frontmatter
\newcommand\HRule{\noindent\rule{\linewidth}{1.5pt}}
\begin{titlepage}
\vspace*{\stretch{1}}
\HRule
\vspace*{5pt}
\begin{flushright}
{
\LARGE
Mathematisches Seminar\\
\vspace*{20pt}
\Huge
Quantenmechanik%
}
\vspace*{5pt}
\end{flushright}
\HRule
\begin{flushright}
\vspace{60pt}
\Large
Leitung: Andreas M"uller\\
\vspace{40pt}
\Large
Dorian~Amiet, Hannes~Badertscher, Roger~Billeter, Joel~Brunner,
Christian~Cavegn, Michael~Cerny, Reto~Christen, Hannes~Diethelm,
Benny~G"achter, Daniel~Gubser, Thomas~Gujer, Stefan~Hedinger,
Marc~Juchli, Simon~Kuster, Gabriel~Looser, Andreas~Linggi,
Daniel~Monti, Max~Obrist, Nicola~Ochsenbein, Kirusanth~Poopalasingam,
Nicol\'as~Rom\'an~L"uthold, Stefan~Schindler, Christoph~Schmitz-Dr"ager,
Arwed~Schudel, Tobias~Stauber, Stefan~Steiner,
Claudio~Stucki, Pascal~Stump, Martin~Stypinski
\end{flushright}
\vspace*{\stretch{2}}
\begin{center}
Hochschule f"ur Technik, Rapperswil, 2015
\end{center}
\end{titlepage}
\hypersetup{
    colorlinks=true,
    linktoc=all,
    linkcolor=blue
}
\newcounter{beispiel}
\newenvironment{beispiele}{
\bgroup\smallskip\parindent0pt\bf Beispiele\egroup

\begin{list}{\arabic{beispiel}.}
  {\usecounter{beispiel}
  \setlength{\labelsep}{5mm}
  \setlength{\rightmargin}{0pt}
}}{\end{list}}
\newcounter{uebungsaufgabe}
% environment fuer uebungsaufgaben
\newenvironment{uebungsaufgaben}{
\begin{list}{\arabic{uebungsaufgabe}.}
  {\usecounter{uebungsaufgabe}
  \setlength{\labelwidth}{2cm}
  \setlength{\leftmargin}{0pt}
  \setlength{\labelsep}{5mm}
  \setlength{\rightmargin}{0pt}
  \setlength{\itemindent}{0pt}
}}{\end{list}\vfill\pagebreak}
\newenvironment{teilaufgaben}{
\begin{enumerate}
\renewcommand{\labelenumi}{\alph{enumi})}
}{\end{enumerate}}
% Loesung
\def\swallow#1{
%nothing
}
\NewEnviron{loesung}[1][L"osung]{%
\begin{proof}[#1]%
\renewcommand{\qedsymbol}{$\bigcirc$}
\BODY
\end{proof}
}
\NewEnviron{bewertung}{%
\begin{proof}[Bewertung]%
\renewcommand{\qedsymbol}{}
\BODY
\end{proof}
}
\NewEnviron{diskussion}{
\begin{proof}[Diskussion]
\renewcommand{\qedsymbol}{}
\BODY
\end{proof}
}
\def\keineloesungen{%
\RenewEnviron{loesung}{\relax}
\RenewEnviron{bewertung}{\relax}
\RenewEnviron{diskussion}{\relax}
}
\newenvironment{beispiel}{%
\begin{proof}[Beispiel]%
\renewcommand{\qedsymbol}{$\bigcirc$}
}{\end{proof}}

\input linsys.tex
\allowdisplaybreaks

\lhead{Inhaltsverzeichnis}
\rhead{}
\tableofcontents
\newtheorem{satz}{Satz}[chapter]
\newtheorem{hilfssatz}{Hilfssatz}[chapter]
\newtheorem{definition}{Definition}[chapter]
\newtheorem{annahme}{Annahme}[chapter]
\mainmatter
\input vorwort.tex
\part{Grundlagen}
%\keineloesungen
\begin{refsection}
\input einleitung.tex
\input einfach.tex
\input hilbert.tex
\input quantencomputer.tex
\input hamilton.tex
\input quantisierung.tex
\input heisenberg.tex
\input harmonisch.tex
\input h.tex
\input stoerungstheorie.tex
\input magnetfeld.tex
\input drehimpuls.tex
\input spin.tex
\input festkoerper.tex
\begin{appendices}
\input komplex.tex
\input kugelkoordinaten.tex
\input konstanten.tex
\end{appendices}
\vfill
\pagebreak
\ifodd\value{page}\else\null\clearpage\fi
\lhead{}
\rhead{}
\printbibliography[heading=subbibliography]
\end{refsection}

\part{Anwendungen und Weiterf"uhrende Themen}
\lhead{Anwendungen}
\input uebersicht.tex
\def\chapterauthor#1{{\large #1}\bigskip\bigskip}
% Quanteninformatik
\input crypto/main.tex
\input teleport/main.tex
\input simon/main.tex
% Halbleiterbauelement
\input tunneldiode/main.tex
\input flash/main.tex
% Anwendung der Störungstheorie
\input atomuhr/main.tex
\input efeld/main.tex
\input anharmonisch/main.tex
% Sphärische harmonische Analyse
\input kugel/main.tex
% Weitere Anwendungen
\input laser/main.tex
\input mri/main.tex
% Supraleitung
\input supraleitung/main.tex
\input bose/main.tex
\input heisenberg/main.tex
%\input stark/main.tex
\input bell/main.tex
\input feldquantisierung/main.tex
%\input rtm/main.tex
%\input orbitale/main.tex
%\input franckhertz/main.tex
%\input maser/main.tex
%\input quantumdot/main.tex
\vfill
\pagebreak
\ifodd\value{page}\else\null\clearpage\fi
\lhead{Index}
\rhead{}
\input skript.ind

\end{document}
