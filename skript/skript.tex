%
% skript.tex -- Skript ueber Quantenmechanik
%
% (c) 2014 Prof. Dr. Andreas Mueller, HSR
%
\documentclass{book}
\usepackage{etex}
\usepackage{geometry}
\geometry{papersize={170mm,240mm},total={140mm,200mm},top=21mm,bindingoffset=10mm}
\usepackage[english,ngerman]{babel}
\usepackage{times}
\usepackage{amsmath,amscd}
\usepackage{amssymb}
\usepackage{amsfonts}
\usepackage{amsthm}
\usepackage{graphicx}
\usepackage{fancyhdr}
\usepackage{textcomp}
\usepackage[all]{xy}
\usepackage{txfonts}
\usepackage{alltt}
\usepackage{verbatim}
\usepackage{paralist}
\usepackage{makeidx}
\usepackage{array}
\usepackage{hyperref}
\usepackage{tikz}
\usepackage{pgfplots}
\usepackage{pgfplotstable}
\usepackage{placeins}
\usepackage{subfigure}
\usepackage[autostyle=false,english=american]{csquotes}
\usepackage{float}
\usepackage{enumitem}
\usepackage{wasysym}
\usepackage{environ}
\usepackage{pifont}
\usepackage{feynmp}
\usepackage{appendix}
\usetikzlibrary{calc,intersections,through,backgrounds,graphs,positioning,shapes,arrows}
\usetikzlibrary{patterns,decorations.pathreplacing}
\usetikzlibrary{decorations.pathreplacing}
\usepackage{siunitx}
\usepackage{tabularx}
\usetikzlibrary{arrows}
\usepackage{listings}
\lstdefinestyle{Matlab}{
  numbers=left,
  belowcaptionskip=1\baselineskip,
  breaklines=true,
  frame=L,
  xleftmargin=\parindent,
  language=Matlab,
  showstringspaces=false,
  basicstyle=\footnotesize\ttfamily,
  keywordstyle=\bfseries\color{green!40!black},
  commentstyle=\itshape\color{purple!40!black},
  identifierstyle=\color{blue},
  stringstyle=\color{orange},
  numberstyle=\ttfamily\tiny
}
\lstdefinelanguage{Maxima}{
  keywords={addrow,addcol,zeromatrix,ident,augcoefmatrix,ratsubst,diff,ev,tex,%
    with_stdout,nouns,express,depends,load,submatrix,div,grad,curl,matrix,%
    invert,lambda,facsum,expand,false,then,if,else,subst,%
    rootscontract,solve,part,assume,sqrt,integrate,abs,inf,exp,sin,cos,sinh,cosh},
  sensitive=true,
  comment=[n][\itshape]{/*}{*/}
}
\lstdefinestyle{Maxima}{
  numbers=left,
  belowcaptionskip=1\baselineskip,
  breaklines=true,
  frame=L,
  xleftmargin=\parindent,
  language=Maxima,
  showstringspaces=false,
  basicstyle=\footnotesize\ttfamily,
  keywordstyle=\bfseries\color{green!40!black},
  commentstyle=\itshape\color{purple!40!black},
  identifierstyle=\color{blue},
  stringstyle=\color{orange},
  numberstyle=\ttfamily\tiny
}
\usepackage{caption}
\usepackage[mode=buildnew]{standalone}
\usepackage[backend=bibtex]{biblatex}
\addbibresource{references.bib}
% Quanteninformatik
\addbibresource{crypto/main.bib}
\addbibresource{teleport/main.bib}
\addbibresource{simon/main.bib}
% Halbleiterbauteile
\addbibresource{tunneldiode/main.bib}
\addbibresource{flash/main.bib}
% Anwendungen der Störungstheorie
\addbibresource{atomuhr/main.bib}
\addbibresource{efeld/main.bib}
\addbibresource{anharmonisch/main.bib}
% Sphaerische harmonische Analyse
\addbibresource{kugel/main.bib}
% Weitere Anwendungen der Quantenmechanik
\addbibresource{laser/main.bib}
\addbibresource{mri/main.bib}
% Supraleitung
\addbibresource{supraleitung/main.bib}
\addbibresource{bose/main.bib}

\addbibresource{heisenberg/main.bib}
\addbibresource{stark/main.bib}
\addbibresource{bell/main.bib}
\addbibresource{feldquantisierung/main.bib}
%\addbibresource{rtm/main.bib}
%\addbibresource{orbitale/main.bib}
%\addbibresource{franckhertz/main.bib}
%\addbibresource{maser/main.bib}
%\addbibresource{quantumdot/main.bib}
\AtEndDocument{\clearpage\ifodd\value{page}\else\null\clearpage\fi}
\makeindex
\DeclareGraphicsRule{*}{mps}{*}{}
\begin{document}
\pagestyle{fancy}
\frontmatter
\newcommand\HRule{\noindent\rule{\linewidth}{1.5pt}}
\begin{titlepage}
\vspace*{\stretch{1}}
\HRule
\vspace*{5pt}
\begin{flushright}
{
\LARGE
Mathematisches Seminar\\
\vspace*{20pt}
\Huge
Quantenmechanik%
}
\vspace*{5pt}
\end{flushright}
\HRule
\begin{flushright}
\vspace{60pt}
\Large
Leitung: Andreas M"uller\\
\vspace{40pt}
\Large
Dorian~Amiet,
Hannes~Badertscher,
Roger~Billeter,
Joel~Brunner\\%,
Christian~Cavegn,
Michael~Cerny,
Reto~Christen,
Hannes~Diethelm\\%,
Benny~G"achter,
Daniel~Gubser,
Thomas~Gujer,
Stefan~Hedinger\\%,
Marc~Juchli,
Simon~Kuster,
Gabriel~Looser,
Andreas~Linggi\\%,
Daniel~Monti,
Max~Obrist,
Nicola~Ochsenbein\\%,
Kirusanth~Poopalasingam,
Nicol\'as~Rom\'an~L"uthold,
Stefan~Schindler\\%,
Christoph~Schmitz-Dr"ager,
Arwed~Schudel,
Tobias~Stauber\\%,
Stefan~Steiner,
Claudio~Stucki,
Pascal~Stump,
Martin~Stypinski
\end{flushright}
\vspace*{\stretch{2}}
\begin{center}
Hochschule f"ur Technik, Rapperswil, 2015
\end{center}
\end{titlepage}
\hypersetup{
    colorlinks=true,
    linktoc=all,
    linkcolor=blue
}
\newcounter{beispiel}
\newenvironment{beispiele}{
\bgroup\smallskip\parindent0pt\bf Beispiele\egroup

\begin{list}{\arabic{beispiel}.}
  {\usecounter{beispiel}
  \setlength{\labelsep}{5mm}
  \setlength{\rightmargin}{0pt}
}}{\end{list}}
\newcounter{uebungsaufgabe}
% environment fuer uebungsaufgaben
\newenvironment{uebungsaufgaben}{
\begin{list}{\arabic{uebungsaufgabe}.}
  {\usecounter{uebungsaufgabe}
  \setlength{\labelwidth}{2cm}
  \setlength{\leftmargin}{0pt}
  \setlength{\labelsep}{5mm}
  \setlength{\rightmargin}{0pt}
  \setlength{\itemindent}{0pt}
}}{\end{list}\vfill\pagebreak}
\newenvironment{teilaufgaben}{
\begin{enumerate}
\renewcommand{\labelenumi}{\alph{enumi})}
}{\end{enumerate}}
% Loesung
\def\swallow#1{
%nothing
}
\NewEnviron{loesung}[1][L"osung]{%
\begin{proof}[#1]%
\renewcommand{\qedsymbol}{$\bigcirc$}
\BODY
\end{proof}
}
\NewEnviron{bewertung}{%
\begin{proof}[Bewertung]%
\renewcommand{\qedsymbol}{}
\BODY
\end{proof}
}
\NewEnviron{diskussion}{
\begin{proof}[Diskussion]
\renewcommand{\qedsymbol}{}
\BODY
\end{proof}
}
\def\keineloesungen{%
\RenewEnviron{loesung}{\relax}
\RenewEnviron{bewertung}{\relax}
\RenewEnviron{diskussion}{\relax}
}
\newenvironment{beispiel}{%
\begin{proof}[Beispiel]%
\renewcommand{\qedsymbol}{$\bigcirc$}
}{\end{proof}}

\input{linsys.tex}
\allowdisplaybreaks

\lhead{Inhaltsverzeichnis}
\rhead{}
\tableofcontents
\newtheorem{satz}{Satz}[chapter]
\newtheorem{hilfssatz}{Hilfssatz}[chapter]
\newtheorem{definition}{Definition}[chapter]
\newtheorem{annahme}{Annahme}[chapter]
\mainmatter
\chapter*{Vorwort}
\lhead{Vorwort}
\rhead{}
Dieses Buch entstand im Rahmen des Mathematischen Seminars
im Fr"uhjahrssemester 2015 an der Hochschule f"ur Technik Rapperswil.
Die Teilnehmer, Studierende der Abteilungen f"ur Elektrotechnik und
Informatik der
HSR, erarbeiteten nach einer Einf"uhrung in das Themengebiet jeweils
einzelne Aspekte des Gebietes in Form einer Seminararbeit, "uber
deren Resultate sie auch in einem Vortrag informierten. 

Im Fr"uhjahr 2015 war das Thema des Seminars ``Quantenmechanik''.
Die Einf"uhrung bestand aus einigen Vorlesungsstunden, deren
Inhalt im ersten Teil dieses Skripts zusammengefasst ist.
Es ging darum, die mathematischen Grundlagen der Quantenmechanik zu
legen und auf die Berechnung einfacher quantenmechanischer 
Systeme anzuwenden. Das Ziel war, ein auch quantitatives Verst"andnis
daf"ur zu entwickeln, wie die Welt im kleinen funktioniert, und 
dieses Wissen auch zum Beispiel f"ur ein besseres Verst"andnis der
Mikroelektronik zu nutzen. Eine vor allem f"ur Informatiker interessante
Vertiefungsrichtung war zu verstehen, was mit einem Quantencomputer
gemeint ist.

Im zweiten Teil dieses Skripts kommen dann die Teilnehmer
selbst zu Wort. Ihre Arbeiten wurden jeweils als einzelne
Kapitel mit meist nur typographischen "Anderungen "ubernommen.
Diese weiterf"uhrenden Kapitel sind sehr verschiedenartig.
Eine "Ubersicht und Einf"uhrung befindet sich in der Einleitung
zum zweiten Teil auf Seite~\pageref{skript:uebersicht}.

In einigen Arbeiten wurde auch Code zur Demonstration der 
besprochenen Methoden und Resultate geschrieben, soweit
m"oglich und sinnvoll wurde dieser Code im Github-Repository
dieses Kurses \url{https://github.com/AndreasFMueller/SeminarQM.git}
abgelegt, in anderen F"allen verweisen die Artikel selbst auf
das zugeh"orige Code-Repository.

Im genannten Repository findet sich auch der Source-Code dieses
Skriptes, es wird hier unter einer Creative Commons Lizenz
zur Verf"ugung gestellt.


\part{Grundlagen}
%\keineloesungen
\begin{refsection}
\chapter{Einleitung}
\lhead{Einleitung}
\rhead{}

\chapter{Einfache Quantensysteme\label{chapter:einfache-quantensysteme}}
\lhead{Einfache Quantensysteme}
\rhead{}
In diesem Kapitel beschreiben wir einige einfache Quantensysteme und
entwickeln einen daf"ur geeigneten mathematischen Apparat. Dieser
Apparat ist bereits ausreichend, um zu erkl"aren, wie ein Quantencomputer
funktionieren k"onnte.

\section{Quantenmechanik als Zustandsmechanik}
\rhead{Quantenmechanik als Zustandsmechanik}
Eine Herleitung der Quantenmechanik im Stile des Beweises eines mathematischen
Satzes ist nat"urlich nicht m"oglich, und soll auch nicht versucht werden.
Hingegen sollen die wichtigsten Prinzipien aus einer heuristischen
Diskussion ``erraten'' werden. 

\subsection{Begriff des Zustandes}
In der Quantenmechanik gibt es kaum Experimente mit einzelnen Objekten.
Atome, Elektronen und Photonen sind einfach zu klein, oder gar nicht
isoliert herzustellen\footnote{Wir wissen, dass Protonen und Neutronen aus
Quarks zusammengesetzt sind, aber es ist nicht m"oglich, einzelne Quarks
zu isolieren (Confinement).}.
Vielmehr werden solche Experimente immer
an einer grossen Zahl von gleichartigen Objekten.
Sinnvolle Vorhersagen "uber den Ausgang eines solchen Experiments
sind allerdings nur m"oglich, wenn alle
Teilchen in allen f"ur das Experiment wesentlichen Aspekten
nicht unterscheidbar sind. Man kann dann sagen, dass sich die
Teilchen alle im gleichen {\em Zustand}\/ befinden, wenn sie in
das Experiment eintreten.
\index{Zustand}

Dieser abstrakte Zustandsbegriff ist eine Verallgemeinerung konkreter
Arten von Zust"anden, wovon wir ein paar Beispiele nennen wollen.

Atome setzen sich aus einem Kern aus Protonen und Neutronen und einer
H"ulle aus Elektronen zusammen.
Durch Absorbtion von Photonen k"onnen Eletronen in der H"ulle Energie
aufnehmen, ein solches angeregtes Atom befindet sich in einem anderen
Zustand als eines, dessen Elektronen so viel Energie wie m"oglich
abgegeben haben.
Letzteres nennt man den Grundzustand eines Atoms.
Ein Experiment k"onnte also damit beginnen,
dass man einen Strahl von Atomen eines bestimmten Elementes herstellt,
die sich alle im Grundzustand befinden, oder alle auf die gleiche Art
angeregt worden sind.

Ein Zustand ist also festgelegt durch eine Menge von Eigenschaften,
die sowohl stetige oder diskrete Zustandsvariablen sein k"onnen.
Es ist keine wesentliche Einschr"ankung anzunehmen, dass Zust"ande
Vektoren in einem geeigneten Vektorraum sind.
Jede mathematische Struktur kann n"amlich in einem Vektorraum eingebettet
werden. Dies ist ganz offensichtlich f"ur die Geometrie, wo die Verwendung
eines Koordinatensystems jedes geometrische Problem in ein Problem "uber
Vektoren in einem dreidimensionalen Vektorraum "ubersetzt. Die Details "uber den
Vektorraum, aus dem die Zust"ande kommen sollen, werden sp"ater festgelegt
werden. Wir schreiben daher einen Zustand als Vektor
\[
|\text{Beschreibung des Zustandes}\rangle
\]
in einem zun"achst nicht spezifizierten Vektorraum.

\index{CERN}
\index{LHC}
Am CERN werden im LHC Protonen auf eine sehr hohe Geschwindigkeit 
beschleunigt und dann zur Kollision gebracht. Es treffen also
Protonen mit einem bestimmten Impuls $p$ auf Protonen mit entgegengesetztem
Impuls $-p$. Es kollidieren also Protonen im Zustand $|p\rangle$ und
$|\text{$-p$}\rangle$. Das Experiment besteht dann darin zu messen, welche Arten von
Teilchen in welchen Zust"anden erzeugt worden sind.
Die Quantenmechanik hat
die Aufgabe, die m"oglichen Zielzust"ande vorherzusagen.

\index{Flasche!magnetische}
Eine magntische Flasche ist ein speziell konstruiertes magnetisches Feld,
in welchem geladene Elementarteilchen eingeschlossen werden k"onnen.
Den Teilchen in einer magnetischen Flasche kann einen Zustands zuschreiben,
der durch die Position bestimmt ist.
Wir k"onnten den Zustand eines Teilchens mit Position $x$ als $|x\rangle$
schreiben.

Die Quantenmechanik ist also eine Mechanik von Zust"anden: sie beschreibt,
unter welchen Umst"anden ein Teilchen von einem Zustand in einen anderen
Zustand "ubergef"uhrt werden k"onnen. 

\subsection{Observable}
\index{Observable}
Es ist charakteristisch f"ur die Quantenmechanik, dass sie mit diskreten
Mengen von Parameter zu tun hat. Zum Beispiel kann sich das Elektron eines
Wasserstoffatoms auf einer grossen Zahl von verschiedenen Energieniveaus
befinden. Gehen wir f"ur den Moment davon aus, dass es $n$ solche Zust"ande
gibt, die wir $|a_i\rangle, 1\le i\le n,$ schreiben wollen,
Ein $a$-Analysator zerlegt Teilchen im Zustand $|\psi\rangle$ in 
Teilchen in den Zust"anden $a_1,a_2,\dots,a_n$. Wir geben dies
schematisch durch das Diagramm
\index{Analysator}
\begin{center}
\includegraphics{graphics/analysator-1.pdf}
\end{center}
wieder.
\begin{figure}
\centering
\includegraphics{images/calcit.jpg}
\caption{Kalkspat (Calcit) kann also Analysator f"ur die Polarisation
von Photonen dienen. Er zerlegt einen Strahl von Photonen in zwei
Strahlen mit orthogonaler Polarisation.
\label{skript:calcit}}
\end{figure}
\index{Kalkspat}
\index{Calcit}
\index{Doppelbrechung}
Kalkspat (Calcit) (Abbildung~\ref{skript:calcit}) teilt einen Strahl
von Photonen auf in zwei Strahlen mit orthogonaler Polarisation,
funktioniert also als Analysator f"ur zwei Polarisationsrichtungen
von Photonen.

\index{Analysatorkreis}
Man kann sich auch vorstellen, die vom Analysator getrennten Strahlen
wieder zusammenzuf"uhren, schematisch:
\begin{center}
\includegraphics{graphics/analysator-2.pdf}
\end{center}
Wir nennen ein solches Objekt einen {\em Analysatorkreis}, auch wenn wie
im n"achsten Beispiel nicht alle Pfade wieder zum Ausgang zusammengef"ugt
werden.

Wir k"onnen vor dem Zusammenf"ugen der Strahlen einzelne davon
blockieren, z.~B.~$a_2$:
\begin{center}
\includegraphics{graphics/analysator-3.pdf}
\end{center}
\index{Projektor}
Wir nennen diese Schaltung auch einen {\em Projektor}, er reduziert den
urspr"unglichen Strahl auf einen Zustand, dem die Komponenten $a_2$
fehlt. Nat"urlich k"onnen Projektoren f"ur eine beliebige Teilmenge
$A=\{a_{i_1},\dots, a_{i_k}\}$ definiert werden: der Projektor $P_A$
blockiert genau die Komponenten aus der Menge $A$.
\index{idempotent}
Damit die oben gezeigten Schemata f"ur Analysatoren ihre G"ultigkeit
haben, sollten Projektoren {\em idempotent}, d.~h.~zwei identische 
Projektoren wirken genau gleich wie ein einzelner, also $P^2=P$ oder
schematisch:
\begin{center}
\includegraphics{graphics/analysator-4.pdf}
\end{center}

Dies ist nur m"oglich, wenn die Wahscheinlichkeit, ein Teilchen nach
einem Projektor $P_{\{a_i\}}$ im Zustand $a_i$ zu finden, verschwindet.
Jeder andere Zustand muss hingegen unver"andert durchkommen, also
\[
P(a_i|a_j)=\delta_{ij}=\begin{cases}
1&\qquad i\ne j\\
0&\qquad\text{sonst.}
\end{cases}
\]
Ein solcher Satz von Zust"anden nennen wir eine Basis.

Nat"urlich sollen die Zust"ande alle m"oglichen Situationen abdecken
k"onnen, es sollte also keinen Zustand $|\psi\rangle$ geben, so dass
\[
P(a_i|\psi)=0\quad\forall i.
\]

\index{Observable}
In der Quantenmechanik sind also genau jene Gr"ossen messbar, f"ur die
es einen Analysator gibt, man nennt solche Gr"ossen {\em Observable}.
Die Polarisation von Photonen, die Position oder der Impuls
eines Teilchens, das elektrische Dipolmoment eines Atoms sind alle
Observable.

\subsection{Photonenpolarisation}
\begin{figure}
\centering
\includegraphics{graphics/analysator-5.pdf}
\caption{Projektion des Zustandes $|a_2\rangle$ auf die beiden
um $\alpha$ verdrehten Basiszust"ande $|b_1\rangle$ und $|b_2\rangle$.
\label{skript:polarisation-rotation}}
\end{figure}
\index{Polarisation}%
Betrachten wir als Beispiel Photonen und ihre Polarisation. 
Statt einer Basis aus den Zust"anden, die horizontal ($a_1$) und vertikal
($a_2$) polarisierte Photonen beschreiben, k"onnen wir auch zwei 
beliebige andere Polarisationsrichtungen $b_1$ und $b_2$ verwenden,
oder sogar rechts ($c_R$) und links ($c_L$) zirkul"ar polarisierte Photonen.
Seien also $b_1$ und $b_2$ gegen"uber $a_1$ und $a_2$ um den Winkel
$\alpha$ verdreht (Abbildung~\ref{skript:polarisation-rotation}),
dann wird die Intensit"at eines Strahls aus $a_2$
bei der Analyse mit $b_1$ und $b_2$ folgende Werte ergeben:
\begin{equation}
\begin{aligned}
P(b_1|a_2)&=\sin^2\alpha
&\qquad
P(b_2|a_2)&=\cos^2\alpha
\\
P(b_1|a_1)&=\cos^2\alpha
&
P(b_2|a_1)&=\sin^2\alpha.
\end{aligned}
\label{skript:polarisation-projektion}
\end{equation}
Die Wahrscheinlichkeiten sind proportional zu den Intensit"aten,
die proportional zum Quadrat der Amplituden sind (daher die Quadrate
in (\ref{skript:polarisation-projektion})).
Man beachte, dass sich aus der Abbildung auch ableiten l"asst, dass
\begin{equation}
\begin{aligned}
P(a_1|b_1)&=\cos^2\alpha
&\qquad
P(a_1|b_2)&=\sin^2\alpha
\\
P(a_2|b_1)&=\sin^2\alpha
&
P(a_2|b_2)&=\cos^2\alpha.
\end{aligned}
\label{skript:polarisation-projektion-inverse}
\end{equation}
\index{Ubergangswahrscheinlichkeit@\"Ubergangswahrscheinlichkeit}
Man beachte, dass die "Ubergangswahrscheinlichkeiten
symmetrisch sind: $P(a_i|b_j)=P(b_j|a_i)$.
Ausserdem ergeben die Elemente $P(a_i|b_j), 1\le i\le n$ f"ur festes $j$,
zusammen den Wert $1$:
\[
\sum_{i=1}^nP(a_i|b_j)=1\quad\forall j.
\]
F"ur zirkul"ar polarisierte Photonen gilt immer 
\[
P(z_L|a_i)=P(z_R|a_i)=\frac12.
\]
Das Experiment in Abbildung~\ref{skript:linear-zirkulaer} verbindet
Projektoren f"ur lineare Polarisation mit Projektoren f"ur zirkul"are
Polarisation.
\begin{figure}
\centering
\includegraphics{graphics/analysator-6.pdf}
\caption{Experiment zur Messung des Zusammenhanges zwischen zirkul"arer
und linearer Polarisation. Der rot hervorgehobene Block wird sp"ater
zur Berechnung von $P(c_L|a_2)$ verwendet.
\label{skript:linear-zirkulaer}}
\end{figure}
An der Stelle~1 ist der Strahl in Richtung $a_2$ linear polarisiert.
An der Stelle~2 ist er links-zirkul"ar polarisiert. Die Wahrscheinlichkeit,
das Intensit"atsverh"altnis der Strahlen zwischen den Punkten $1$ und $2$
ist
\begin{equation}
\frac{I_2}{I_1}=P(c_L|a_2)=\frac12.
\label{skript:intensitaetsverhaeltnis}
\end{equation}

\subsection{Doppelspaltexperiment}
\index{Doppelspaltexperiment}
\begin{figure}
\centering
\includegraphics{graphics/analysator-7.pdf}
\caption{Doppelspaltexperiment Teilchen mit gleichem Impuls oder Photonen
mit gleicher Wellenl"ange treffen von rechts auf eine Blende mit zwei
Spalten. Auf dem Schirm links werden die Teilchen gez"ahlt, die dort
eintreffen. Es entsteht ein Interferenzmuster.
\label{skript:doppelspalt-bild}}
\end{figure}
Das ber"uhmte Doppelspaltexperiment (Abbildung~\ref{skript:doppelspalt-bild})
erzeugt zun"achst einen Strahl
von Teilchen, die alle den gleichen Impuls $p$ haben, wir haben also
Teilchen im Zustand $|p\rangle$. Dieser Strahl
wird dann auf die zwei Spalte gerichtet. Dadurch wird der Strahl
beeinflusst, es entsteht also ein neuer Zustand, den zu berechnen
wir uns f"ur sp"ater zur Aufgabe machen wollen.
Im Moment bezeichnen wir diesen Zustand nach dem Doppelspalt 
als $|\psi\rangle$.

Gemessen wird dann, mit welcher Wahrscheinlichkeit
ein Teilchen im Zustand mit Position $y$ auf dem Schirm
gemessen wird, wenn das Teilchen im Zustands $|\psi\rangle$
ist. Dies ist ein bedingte Wahrscheinlichkeit, die wir also
$P(y|\psi)$ geschrieben werden kann.

Wir k"onnen dieses Experiment auch mit dem bisher entwickelten
Formalismus analysieren.
Zun"achst k"onnen wir die Blende als einen Analysator auffassen, der
den Strahl aufspaltet in verschiedene Zust"ande von Teilchen, die
den einen oder anderen Spalt traversiert haben.
Die Zust"ande $B_1$, $B_2$ und $B_3$ entsprechen Teilchen, die die
Blende nicht passiert haben. Die Zust"ande $S_1$ und $S_2$ entsprechen
Teilchen, die durch die beiden Spalte $S_1$ und $S_2$ geflogen sind.

Die unendlich vielen m"oglichen Positionen $y$ auf dem Schirm
k"onnen wir ebenfalls auf diskrete Zust"ande abbilden.
Dazu unterteilen wir den Schirm links in
disjunkte Zonen, die das Eintreffen eines Teilchens registrieren k"onnen.
Ein Teilchen kann offenbar nur in jeweils einer Zone registriert werden.
Bezeichnen wir den Zustand eines Teilchens, welches in der Zone mit
der Nummer $i$ registriert wird, mit $y_i$, dann funkioniert der Schirm als
eine Kombination von Projektoren auf die Zust"ande $|y_i\rangle$.
\begin{figure}
\centering
\includegraphics{graphics/analysator-8.pdf}
\caption{Analyse des Doppelspalt-Experiments
\label{skript:doppelspalt-analyse}}
\end{figure}
Das kombinierte Experiment sieht dann aus wie in
Abbildung~\ref{skript:doppelspalt-analyse}

\section{Algebraischer Formalismus der Quantenmechanik}
\rhead{Algebraischer Formalismus}
Im letzten Abschnitt haben wir einige Prinzipien zusammengestellt,
die ein quantenmechanischer Kalk"ul wiederzugeben in der Lage sein muss.
Wir brauchen nur noch eine algebraische Struktur, mit der wir die 
verschiedenen Objekte abbilden k"onnen.

\subsection{Analysatoren}
\index{Analysator}
Eine ununterbrochene Kurve in einem Analysatorkreis durch den Zustand
$a_i$ stellen wir durch das ``Messsymbol'' 
\[
|a_i\rangle\langle a_i|
\]
dar. Das Symbol ist von rechts zu lesen: wenn ein Teilchen
im Zustand $a_i$ auf den Analysator trifft, dann gibt er ein solches
Teilchen weiter.

Ein Analysatorkreis, der am Input nichts "andert, entspricht dem Symbol
\[
\sum_{i=1}^n |a_i\rangle \langle a_i|=\operatorname{id},
\]
dieses Objekt wirkt wie die identische Abbildung.
\index{Abbildung!identische}

\subsection{Projektoren}
\index{Projektor}
Ein einzelner Projektor auf einen Basiszustand $a_i$ entspricht dem Symbol
$P= |a_i\rangle\langle a_i|$, also muss gelten
\[
P^2 = 
|a_i\rangle\langle a_i|
\cdot
|a_i\rangle\langle a_i|
=P
=
|a_i\rangle\langle a_i|.
\]
Zwei verschiedene Projektoren m"ussen sich hingegen aufheben:
\[
|a_i\rangle\langle a_i|
\cdot
|a_j\rangle\langle a_j|
=
0 \quad\text{f"ur $i\ne j$}.
\]
Man kann also schreiben
\[
|a_i\rangle\langle a_i|
\cdot
|a_j\rangle\langle a_j|
=\delta_{ij} |a_i\rangle\langle a_j|.
\]
Ein allgemeiner Projektor $P$, der nur die Zust"ande $a_{i_1},\dots ,a_{i_m}$
akzeptiert, wird als
\[
P = \sum_{k=1}^m |a_{i_k}\rangle \langle a_{i_k}|
\]
geschrieben.
Nach den eben abgeleiteten Rechenregeln gilt
\begin{align*}
P^2
&=
\sum_{k,l=1}^m |a_{i_k}\rangle\langle a_{i_k}|a_{i_l}\rangle\langle a_{i_l}|
=
\sum_{k,l=1}^m \delta_{i_ki_l}|a_{i_k}\rangle\langle a_{i_l}|
=
\sum_{k=1}^m |a_{i_k}\rangle\langle a_{i_k}| = P
\end{align*}
$P$ ist also tats"achlich ein Projektor.

Diese Beobachtungen sind konsistent mit der Annahme, dass
\[
\langle a_i|a_j\rangle = \delta_{ij}.
\]

\subsection{Transformationsfunktion}
\index{Transformationsfunktion}
Bisher wurden nur Messsymbole miteinander verkn"upft, deren 
Verkn"upfung bereits bekannt war, es stellt sich daher die
Frage gar nicht, was das naheliegende Symbol $\langle B|C\rangle$
in
\[
|A\rangle\langle B|\cdot |C\rangle\langle D|
=
|A\rangle \langle B|C\rangle \langle D|
\]
f"ur eine Bedeutung haben soll. Die bisherigen Beispiel konnte
man so interpretieren, dass $\langle B|C\rangle$ eine Zahl war
(in den Beispielen jeweils $0$ oder $1$).
Eine naheliegende Verallgemeinerung ist daher, immer anzunehmen,
dass $\langle B|C\rangle$ eine Zahl ist, und dass man mit den
Messsymbolen wie f"ur Matrizen "ublich rechnen kann.
Es kommt also auf die Reihenfolge der Faktoren an, aber skalare
Faktoren, wie eben auch $\langle B|C\rangle$, k"onnen beliebig
verschoben werden.
$\langle \,\cdot\, |\,\cdot\, \rangle$ heisst die Transformationsfunktion.

%Die Transformationsfunktion taucht auf als Koeffizient in Linearkombinationen
%von Messsymbolen.

Wenden wir den eben entwickelten Formalismus auf das Doppelspalt-Experiment
gem"ass Abbildung~\ref{skript:doppelspalt-analyse} an, erhalten wir das Symbol
\[
|y_i\rangle \langle y_i|\; (|S_1\rangle\langle S_1| + |S_2\rangle \langle S_2|)
=
\langle y_i|S_1\rangle
\cdot
|y_i\rangle \langle S_1|
+
\langle y_i|S_2\rangle
\cdot
|y_i\rangle \langle S_2|,
\]
\index{Uberlagerung@\"Uberlagerung}
Insbesondere entsteht die Intensit"at in der Zonen $y_i$ durch eine
"Uberlagerung von zwei Termen mit verschiedenen
Koeffizienten $\langle y_i|S_j\rangle, j=1,2$.
\index{Interferenz}
Das Resultat der Durchf"uhrung des Experiments zeigt ein Interferenz-Muster,
was alleine mit reellen Werten f"ur die Transformationsfunktion nicht
zu bewerkstelligen w"are.
Wir m"ussen daher davon ausgehen, dass $\langle B|C\rangle\in\mathbb C$ gilt.

Mit Hilfe der Transformationsfunktion kann mit Hilfe einer Basis von Zust"anden
$|a_i\rangle, i=1,\dots, n,$ und einem Zustand $|\psi\rangle$ einen Vektor
\[
\begin{pmatrix}
\langle a_1|\psi\rangle\\
\vdots\\
\langle a_n|\psi\rangle
\end{pmatrix}
\in\mathbb C^n
\]
zuordnen.
Eine Basis transportiert eine Problem "uber Zust"ande in ein
Problem "uber Vektoren in $\mathbb C^n$.

\subsection{Wahrscheinlichkeiten}
\index{Wahrscheinlichkeit}
Zu dem Experiment in Abbildung~\ref{skript:linear-zirkulaer} geh"ort das Symbol
\[
|a_2\rangle\langle a_2|\cdot
|c_L\rangle\langle c_L|\cdot
|a_2\rangle\langle a_2|
=
|a_2\rangle\langle a_2
|c_L\rangle\langle c_L
|a_2\rangle\langle a_2|
=
\langle a_2
|c_L\rangle\langle c_L
|a_2\rangle
|a_2\rangle
\langle a_2|
\]
Andererseits haben wir das Intensit"atsverh"altnis zwischen den Punkten
$1$ und $2$ bereits in (\ref{skript:intensitaetsverhaeltnis}) ausgerechnet.
Der rot eingerahmte Teil des Experimentes scheint dem Ausdruck
\[
\langle a_2
|c_L\rangle\langle c_L
|a_2\rangle
\]
zu entsprechen.
Dies suggeriert, dass wir ihn als
\[
\langle a_2
|c_L\rangle\langle c_L
|a_2\rangle
=
P(c_L|a_2) 
\]
interpretieren k"onnten. F"ur diesen Ausdruck ist die fr"uher formulierte
Symmetriebedingung 
\[
P(A|B)
=
\langle A |B\rangle
\langle B |A\rangle
=
\langle B |A\rangle
\langle A |B\rangle
=
P(B|A)
\]
ebenfalls erf"ullt.

Da $P(A|A) =1$ ist, folgt
\begin{align*}
P(A|A)&=\langle A|A\rangle \langle A|A\rangle = \langle A|A\rangle^2 = 1
\\
\langle A|A\rangle&=\pm 1
\end{align*}
Der negative Wert ist nicht weiter n"utzlich, so dass wir im folgenden
davon ausgehen, dass $\langle A|A\rangle = 1$ gilt.

Die Wahrscheinlichkeit $P(A|B)$ muss $\ge 0$ sein, also insbesondere
reell. Das ist nur dann garantiert, wenn $\langle A|B\rangle$ und
$\langle B|A\rangle$ konjugiert komplex sind:
\begin{equation}
\langle B|A\rangle
=\overline{\langle A|B\rangle}.
\label{skript:hermiteschesymmetrie}
\end{equation}
Wir nennen dies die {\em hermitesche} Symmetrie der Transformationsfunktion.

\subsection{Observable und Operatoren}
\index{Observable}
Bisher haben wir nur Analysatoren und Projektoren untersucht.
Es ist aber durchaus denkbar, dass noch wesentlich komplexere
Versuchsaufbauten existieren, deren Details wir gar nicht unbedingt
im Detail kennen:
\begin{center}
\includegraphics{graphics/analysator-9.pdf}
\end{center}
Ist eine Basis $|a_i\rangle$ von Zust"anden gegeben, dann k"onnen
wir die Wirkung von $A$ auf die Wirkung auf den Zustandsvektoren
reduzieren, indem wir $A$ mit Analysatorkreisen zusammensetzen:
\begin{center}
\includegraphics{graphics/analysator-10.pdf}
\end{center}
Diesem Diagramm entspricht der Ausdruck
\[
|a_j\rangle \langle a_j|\, A \,|a_i\rangle \langle a_i|,
\]
insbesondere ist die Wirkung von $A$ vollst"andig festgelegt durch die
Zahlen
$\langle a_j|\,A\,|a_i\rangle$, sie heissen die {\em Matrixelemente} von $A$.
\index{Matrixelement}
Die Wirkung von $A$ auf dem Zustand $|\psi\rangle$ ist also
\begin{equation}
A\,|\psi\rangle = \sum_{i,j} |a_j\rangle
	\langle a_j|\,A\,|a_i\rangle
	\langle a_i|\psi\rangle,
\label{skript:A-wirkung}
\end{equation}
mit Hilfe einer Basis kann man also die Wirkung eines Operators auf
die Multiplikation von Matrizen und Vektoren
\[
\begin{pmatrix}
\langle a_1|\,A\,|\psi\rangle\\
\vdots\\
\langle a_n|\,A\,|\psi\rangle
\end{pmatrix}
=
\begin{pmatrix}
\langle a_1|\,A\,|a_1\rangle&\dots &\langle a_1|\,A\,|a_n\rangle\\
\vdots                  &\ddots&\vdots                  \\
\langle a_n|\,A\,|a_1\rangle&\dots &\langle a_n|\,A\,|a_n\rangle
\end{pmatrix}
\begin{pmatrix}
\langle a_1|\psi\rangle\\
\vdots\\
\langle a_n|\psi\rangle
\end{pmatrix}
\]
reduzieren.

In (\ref{skript:A-wirkung}) wurde angenommen, dass $A$ auf den Vektor
$|\psi\rangle$ wirkt. Die Notation ist jedoch symmetrisch, k"onnte
man auch eine Wirkung von $A$ auf Vektoren $\langle\psi|$ entwickeln?
Diese Wirkung schreiben wir
\[
A\,\langle \psi|=\langle\psi|\,A^*.
\]
Mit Hilfe der hermiteschen Symmetrierelation (\ref{skript:hermiteschesymmetrie})
kann man aber auch die Wirkung auf einen $\langle\psi|$-Vektor ausrechnen:
\begin{equation}
(A\langle a_i|)\,|a_j\rangle
=
\overline{\langle a_j|\,(A\,|a_i\rangle)}
=
\overline{\langle a_j|\,A\,|a_i\rangle}.
\end{equation}
Die Matrix von $A^*$
ist also die transponierte und komplex konjugierte Matrix von $A$,
sie heisst auch die {\em adjungierte} Matrix.
\index{adjungierte Matrix}

Operatoren sind beliebige Linearkombinationen von Messsymbolen,
F"ur Projektoren waren es ausschliesslich Messsymbole der Form
$|a_i\rangle\langle a_i|$, ein komplexer Operator $A$ kann jedes
Messymbol enthalten, die Matrixelemente sind die Koeffizienten 
der Linearkombination.

Damit k"onnen wir jetzt auch eine algebraische Definition einer
Observablen geben. Observable sind Gr"ossen, f"ur die es einen
Analysator gibt. Insbesondere gibt es eine Basis von Zust"anden,
die zu verschiedenen Werten der Observablen geh"oren.
Seien $|i\rangle$ die Zust"ande, und $v_i$ die Werte der Observablen
f"ur den Zustand $|i\rangle$. Dann k"onnen wir einen neuen
Operator
\[
V = \sum_{i} a_i\, |i\rangle\langle i|
\]
bauen. In der Basis $|i\rangle$ sind die Matrixelement von $V$
\[
\langle j|\,V\,|i\rangle = v_i\delta_{ij},
\]
$V$ hat also Diagonalform. In einer anderen Basis muss dies nicht
mehr erf"ullt sein. Es ist aber eine entscheidende Eigenschaft 
von $V$, dass es eine Basis gibt, in der $V$ diagonal wird.

Ist $|\psi\rangle$ ein beliebiger Zustand, dann ist
\[
\langle\psi|\,V\,|\psi\rangle
=
\sum_{i,j}\langle \psi|j\rangle\langle j|\,V\,|i\rangle\langle i|\psi\rangle
=
\sum_i\langle \psi|i\rangle\langle i|\,V\,|i\rangle\langle i|\psi\rangle
=
\sum_i v_i P(i|\psi)=E(V|\psi),
\]
also ist $\langle \psi|V|\psi\rangle$ der Erwartungswert der Observablen
$V$ im Zustand $\psi$.

% XXX Definition von selbstadjungiert...
\index{selbstadjungiert}
Allgemein betrachten wir selbstadjungierte Operatoren also 
die Repr"asentanten von Observablen, denn aus der abstrakten
Theorie der komplexen Vektorr"aume ist bekannt, dass selbstadjungierte
Matrizen diagonalisierbar sind.

\subsection{Zusammenfassungen}
Wir fassen den bisher entwickelten Formalismus zusammen.

\begin{enumerate}
\item
In der Quantenmechanik werden Zust"ande durch Vektoren $|\psi\rangle$
eines noch nicht spezifizierten Vektorraumes dargestellt.
\item
Zu jedem Vektor $|\phi\rangle$ gibt es auch einen Vektor $\langle \phi|$,
sowie die Paarungen
$|\phi\rangle\langle\psi|$
und
$\langle\phi|\psi\rangle$.
$|\phi\rangle\langle\psi|$ heisst das Messsymbol, es akzeptiert Teilchen im
Zustand $|\psi\rangle$ und wandelt sie in solche im Zustand $|\phi\rangle$
um.
\item
Die Gr"osse $\langle \phi|\psi\rangle$ ist eine komplexe Zahl, sie
heisst die Transformationsfunktion. Es gilt
\begin{align*}
\langle \phi|\psi\rangle
&=
\overline{
\langle \psi|\phi\rangle
}
\\
\langle\psi|\psi\rangle&=1
\end{align*}
Die physikalische Bedeutung von $\langle\phi|\psi\rangle$ ist die
Wahrscheinlichkeit
\[
P(\phi|\psi)=|\langle \phi|\psi\rangle|^2=
\langle\phi|\psi\rangle
\langle\psi|\phi\rangle,
\]
in einem Strahl von Teilchen im Zustand $|\psi\rangle$ ein Teilchen zu
finden, welches sich im Zustand $|\phi\rangle$ befindet.
\item
Eine Linearkombination $a\,|1\rangle + b\,|2\rangle$ bedeutet nicht,
dass der Anteil $|a|^2$ der Teilchen im Zustand $|1\rangle$ sind
w"ahrend $|b|^2$ der Teilchen im Zustand $|2\rangle$ sind.
Vielmehr befinden sich alle Teilchen im gleichen Zustand,
erst bei der Messung wird bestimmt, in welchem Zustand ein Teilchen
vorgefunden wird.
\item
Kann ein Zielzustand auf "uber verschiedene Zwischenzust"ande
erreicht werden, dann ist die Transformationsfunktion die Summe
der Transformationsfunktionen f"ur die Zwischenzust"ande:
\[
\langle A|B\rangle
=
\sum_{i=1}^n\langle A|b_i\rangle\;\langle b_i|B\rangle.
\]
\item
Besteht ein Weg aus mehreren Schritten, dann ist ihre Transformationsfunktion
das Produkt der Transformationsfunktionen der einzelnen Schritte.
\item 
Observable sind selbstadjungierte Operatoren.
Bilden die Zust"ande $|i\rangle$ eine Basis, in der die Observable $A$ 
diagonal ist, dann ist das Matrix-Element $\langle i|\,A\,|i\rangle$ der
Wert der Observablen im Zustand $|i\rangle$. F"ur einen beliebigen
Zustand $|\psi\rangle$ ist $\langle\psi|\,V\,|\psi\rangle$ der Erwartungswert
der Observablen $V$ im Zustand $|\psi\rangle$.
\end{enumerate}

%
% Zeitentwicklung
%
\section{Zeitentwicklung}
\rhead{Zeitentwicklung}
\index{Zeitentwicklung}
Die besondere Erkenntnis Newtons war, dass die Gleichung $F=ma$ eine
Differentialgleichung ist, die zusammen mit einer Anfangsbedingung die
gesamte zuk"unftige Bewegung festlegt.
Wir m"ochten ein "ahnlich leistungsf"ahiges Prinzip auch in der
Quantenmechanik haben.

\subsection{Schr"odingergleichung}
Getreu dem bisher entwickelten Formalismus k"onnen wir bestenfalls
erwarten, eine Differentialgleichung f"ur die Entwicklung des Zustandes
zu finden, also zum Beispiel
\begin{equation}
\frac{d}{dt}|\psi(t)\rangle = F(t, |\psi(t)\rangle)
\label{skript:zeitentwicklung}
\end{equation}
f"ur eine m"oglicherweise komplizierte Funktion $F$.

Die Quantenmechanik ist also immer noch vollst"andig {\em deterministisch},
zu einem gegebenen Anfangszustand geh"oren eindeutig bestimmte
zuk"unftige Zust"ande.
Die Schwierigkeit kann aber darin bestehen, dass der Anfangszustand nicht
unbedingt bekannt ist.
Wir werden zum Beispiel sehen, dass wir die Position eines Teilchens
nicht wirklich kennen k"onnen, man kann ein Teilchen nicht in
einen Zustand bringen, wo die Position mit beliebiger Genauigkeit
bekannt ist.
Ausserdem k"onnen wir aus der Zeitentwicklung nicht beliebige Aussagen
ableiten, sondern nur Informationen die sich aus $|\psi(t)\rangle$ 
mit Hilfe von Observablen errechnen lassen,
Dies sind aber ausschliesslich Erwartungswerte der Observablen oder
Wahrscheinlichkeiten f"ur einzelne Werte.
Die Quantenmechanik kann nicht vorhersagen, welchen Wert f"ur eine
bestimmte Observable wir in einem Experiment messen werden,
nur der Erwartungswert ist vorhersagbar.
Quantenmechanische Vorhersagen sind also immer von statistischer Art.


Die Funktion $F$ ist nicht beliebig, es muss ja zum Beispiel zu allen Zeiten
gelten
$\langle\psi(t)|\psi(t)\rangle=1$, was die Wahl der Funktion $F$ einschr"ankt.
In unserem Formalismus kann die Zeitentwicklung als abstrakter Block
gezeichnet werden:
\begin{center}
\includegraphics{graphics/analysator-11.pdf}
\end{center}
Wir erwarten daher, dass die Zeitentwicklung wieder ein Operator ist,
den wir $U(t)$ nennen:
\[
|\psi(t)\rangle = U(t)\,|\psi(0)\rangle.
\]
Das bedeutet aber, dass die Differentialgleichung (\ref{skript:zeitentwicklung})
linear sein muss, es muss also einen m"oglicherweise zeitabh"angigen
Operator $K(t)$ geben, so dass 
\begin{equation}
\frac{d}{dt}\,|\psi(t)\rangle = K(t)\,|\psi(t)\rangle.
\label{skript:zeitentwicklung-linear}
\end{equation}
Die L"osung dieser Differentialgleichung liefert den Operator $U(t)$,
wobei $U(0)$ die identische Abbildung (Einheitsmatrix) sein muss.

Zun"achst m"ussen wir sicherstellen, dass $\langle\psi(t)|\psi(t)\rangle=1$
f"ur alle Zeiten. Dies ist offenbar eine Aussage "uber den Operator $U(t)$,
den wir in diesem Absatz einfach als $U$ schreiben.
Zu jedem beliebigen Anfangszustand $|\psi(0)\rangle$ muss gelten
\[
\langle \psi(t)|\psi(t)\rangle
=
\langle \psi(t)|\,U\,|\psi(0)\rangle
=
\overline{\langle\psi(0)|\,U^*\,|\psi(t)\rangle}
=
\overline{\langle\psi(0)|\,U^*U\,|\psi(0)\rangle}
=
\langle\psi(0)|\,U^*U\,|\psi(0)\rangle.
\]
Dies ist nur m"oglich, wenn
\begin{equation}
U^*U=\operatorname{id}
\label{skript:unitaritaetsbedingung}
\end{equation}
die identische Abbildung ist. Der Operator $U(t)$ muss also unit"ar sein.
\index{unitar@unit\"ar}

Die Unitarit"at von $t$ schr"ankt $K(t)$ stark ein. Leiten wir die
Unitarit"atsbedinung (\ref{skript:unitaritaetsbedingung}) nach der Zeit ab,
und werten die Ableitung an der Stelle $t=0$ aus,
erhalten wir die Gleichung
\[
0
=
\left.\frac{d}{dt}\bigl(U(t)^*U(t)\bigr)\right|_{t=0}
=
\left.
\biggl(
\frac{dU(t)^*}{dt}U(t)+U(t)^*\frac{dU(t)}{dt}
\biggr)
\right|_{t=0}
=
\biggl(\frac{dU(0)}{dt}\biggr)^*
+
\biggl(\frac{dU(0)}{dt}\biggr).
\]
Schreiben wir $K$ f"ur die Ableitung von $U(t)$ an der Stelle $t=0$,
dann muss offenbar gelten
\[
K^*+K=0\qquad\Rightarrow\qquad K^*=-K,
\]
man sagt auch, $K$ sei {\em antihermitesch}. Aus $K$ kann man einen
neuen Operator
\[
H=-\frac{\hbar}{i}K=i\hbar K
\]
konstruieren, f"ur den gilt
\[
H^*
=
\left(-\frac{\hbar}{i}K\right)^*
=
+\frac{\hbar}{i}K^*
=
-\frac{\hbar}{i}K=H.
\]
Insbesondere ist $H$ ein selbstadjungierter Operator, oder
eine Observable.
\index{Hamilton-Operator}
Der Faktor $\hbar$ wird konventionell verwendet, es wird sich
herausstellen, dass sich damit $H$ einfacher mit einer bekannten
physikalischen Gr"osse identifizieren l"asst.
Die Masseinheit von $H$ ist die Masseinheit von $\hbar$ geteilt durch die Zeit.
Da $\hbar$ die Masseinheit $\text{J}\cdot\text{s}$ hat, hat
$H$ die Masseinheit einer Energie.
Nat"urlich ist das keine Begr"undung f"ur irgend etwas, denn die
Masseinheit entstand ja durch die Wahl des Faktors $\hbar$, f"ur
welche wir noch gar keine Begr"undung haben.
Diese Begr"undung wird folgen, wenn wir in
Satz~\ref{skript:quantisierung-poisson}
die allgemeinen Quantisierungsregeln kennengelernt haben, aus
denen die Bedeutung auf Seite~\ref{skript:hamilton-op-ist-energie}
folgen wird.

Die Differentialgleichung f"ur die Zeitentwicklung des Quantensystems wird
mit $H$ ausgedr"uckt zu
\begin{equation}
i\hbar\frac{d}{dt}|\psi(t)\rangle = H(t)\,|\psi(t)\rangle,
\label{skript:schroedingergleichungt}
\end{equation}
die {\rm zeitabh"angige Schr"odingergleichung}.
\index{schrodingergleichung@Schr\"odingergleichung!zeitabh\"angige}

Falls $H$ nicht von der Zeit abh"angt, wird die zeitabh"angige zur
zeitunabh"angigen Schr"odingergleichung
\index{schrodingergleichung@Schr\"odingergleichung!zeitunabh\"angige}
\begin{equation}
i\hbar \frac{d}{dt}\,|\psi(t)\rangle = H\,|\psi(t)\rangle,
\label{skript:schroedingergleichung}
\end{equation}
die mit Hilfe der Exponentialfunktion sofort gel"ost werden kann:
\[
|\psi(t)\rangle = e^{-\frac{i}{\hbar}H t}\,|\psi(0)\rangle
\]
Da $H$ eine Observable ist, k"onnen wir eine Basis von Eigenvektoren
von $H$ konstruieren, ist $|k\rangle$ ein Eigenvektor von $H$ mit
Eigenwert $E_k$, dann ist die Zeitentwicklung f"ur $|\psi(0)\rangle = |k\rangle$
% XXX 1/hbar?
\[
|\psi(t)\rangle
=
e^{-\frac{i}{\hbar}E_kt}\,|k\rangle.
\]
Insbesondere ist in diesem Spezialfall eines zeitunabh"angigen $H$-Operators
die Zeitentwicklung vollst"andig bekannt, wenn man die Eigenwerte und
Eigenvektoren von $H$ bestimmt hat.

%
% Zeitentwicklung des Zweizustandssystems
%
\subsection{Zeitentwicklung eines Zweizustandssystems}
Der Hamilton-Operator eines Zweizustandssystems ist eine hermitesche
$2\times 2$-Matrix
\[
H=\begin{pmatrix}H_{11}&H_{12}\\H_{21}&H_{22}\end{pmatrix}.
\]
In den meisten praktischen F"allen k"onnen wir davon ausgehen, dass
der Hamilton-Operator reell ist, also gilt sogar $H_{12}=H_{21}$.

Die Schr"odinger-Gleichung beschreibt die Zeitentwicklung eines
Zustands $|\psi(t)\rangle$ nach
\[
i\hbar\frac{d}{dt}\,|\psi(t)\rangle = H\, |\psi(t)\rangle.
\]
Wir bestimmen eine Basis aus Eigenvektoren von $H$.
Die charakteristische Gleichung ist
\begin{align*}
0&=
\left|\begin{matrix}
H_{11}-E&H_{12}\\
H_{21}&H_{22}-E
\end{matrix}\right|
=
(H_{11}-E)(H_{22}-E)-H_{12}H_{21}
\\
&=
E^2 - (H_{11}+H_{22})E + H_{11}H_{22}-H_{12}H_{21},
\end{align*}
woraus wir die Energieeigenwerte
\begin{align*}
E_{1,2}
&=
\frac{H_{11}+H_{22}}2\pm\sqrt{\frac{(H_{11}+H_{22})^2}4-H_{11}H_{22}+H_{12}H_{21}}
\\
&=
\frac{H_{11}+H_{22}}2\pm\sqrt{\frac{(H_{11}-H_{22})^2}4+H_{12}^2}
\\
H_{11}-E_1
&=
\frac{H_{11}-H_{22}}2-\sqrt{\frac{(H_{11}-H_{22})^2}4+H_{12}^2}
\\
H_{22}-E_2
&=
-\frac{H_{11}-H_{22}}2+\sqrt{\frac{(H_{11}-H_{22})^2}4+H_{12}^2}
\end{align*}
finden.
Die zugeh"origen Eigenvektoren k"onnen wenigstens bis auf die Normierung
ebenfalls berechnet werden,
\begin{align*}
|E_1\rangle
&\sim
\begin{pmatrix}
  -H_{12} \\
H_{11}-E_1
\end{pmatrix}
\sim
\begin{pmatrix}
H_{22}-E_1 \\
   -H_{12}
\end{pmatrix},
&
|E_2\rangle
&\sim
\begin{pmatrix}
   -H_{12} \\
H_{11}-E_2
\end{pmatrix}
\sim
\begin{pmatrix}
H_{22}-E_2 \\
   -H_{12}
\end{pmatrix}.
\end{align*} 
Ein Zustand $|\psi\rangle$ kann als Linearkombination dieser zwei
Basiszust"ande geschrieben werden, woraus sich die Zeitentwicklung
\[
|\psi\rangle
=
a_1\,|E_1\rangle
+
a_2\,|E_2\rangle
\qquad\Rightarrow\qquad
|\psi(t)\rangle
=
a_1e^{\frac{i}{\hbar}E_1t}\,|E_1\rangle
+
a_2e^{\frac{i}{\hbar}E_2t}\,|E_2\rangle
\]
ergibt.
Die Koeffizienten $a_1$ und $a_2$ k"onnen mit Hilfe der Skalarprodukte
\begin{align*}
a_1&=\langle E_1|\psi(0)\rangle,
&
a_2&=\langle E_2|\psi(0)\rangle
\end{align*}
gefunden werden.

%
% Zeitentwicklung von Observablen
%
\subsection{Zeitentwicklung von Observablen\label{section:zeitentwicklung-von-observablen}}
\index{Zeitentwicklung!von Observablen}
Sei $|\psi(t)\rangle$ ein Zustand, der die zeitunabh"angige
Schr"odingergleichung mit Hamilton-Operator $H$ erf"ullt. Wir m"ochten
gerne wissen, wie sich die Observable $A$ mit der Zeit entwickelt,
wie gross also
\begin{equation}
\langle A\rangle
=
\langle \psi(t)|\,A\,|\psi(t)\rangle
\label{skript:observable-zeitabhaengigkeit}
\end{equation}
sein wird. Dazu leiten wir (\ref{skript:observable-zeitabhaengigkeit}) nach der
Zeit ab und setzen die Schr"odingergleichung ein:
\begin{align}
\frac{d}{dt}\langle A\rangle
&=
\biggl(\frac{d}{dt}\langle\psi(t)|\biggr)A\,|\psi(t)\rangle
+
\langle\psi(t)|\,A\biggl(\frac{d}{dt}\,|\psi(t)\rangle\biggr)
\notag
\\
&=
\biggl(\frac{d}{dt}|\psi(t)\rangle\biggr)^*A\,|\psi(t)\rangle
+
\langle\psi(t)|\,A\biggl(\frac{d}{dt}\,|\psi(t)\rangle\biggr)
\notag
\\
&=
\biggl(\frac{1}{i\hbar}H\,|\psi(t)\rangle\biggr)^*A\,|\psi(t)\rangle
+
\langle\psi(t)|\,A\frac{1}{i\hbar}H\,|\psi(t)\rangle
\notag
\\
&=
-\langle \psi(t)| {\frac{1}{i\hbar}}H^*A\,|\psi(t)\rangle
+
\langle\psi(t)|\,A\frac{1}{i\hbar}H\,|\psi(t)\rangle
\notag
\\
&=
\frac{i}{\hbar}
\langle\psi(t)|\, HA-AH \,|\psi(t)\rangle
\notag
\\
\frac{d}{dt}\langle A\rangle
&=\langle\psi(t)|\, \frac{i}{\hbar}[H,A] \,|\psi(t)\rangle
=\biggl\langle
-\frac{i}{\hbar}[H,A]
\biggr\rangle
\label{skript:observableZeitentwicklung}
\end{align}
Die Zeitabh"angigkeit einer Observablen $A$ wird also im Wesentlichen
durch den Kommutator $[H,A]$ der Observablen mit dem Hamiltonoperator $H$
gegeben. Erhaltungsgr"ossen sind also Observable, die mit dem
Hamiltonoperator vertauschen.

\section*{"Ubungsaufgaben}
\rhead{"Ubungsaufgaben}
\begin{uebungsaufgaben}
\item
Seien $A$ und $B$ Observable.
Zeigen Sie, dass die Matrixelemente der Observable $AB$ 
das Matrizenprodukt der Matrixelemente von $A$ und $B$ sind.

\begin{loesung}
Die Matrixelemente von $AB$ sind in der Basis $|k\rangle$
sind $\langle k|\,AB\,|l\rangle$.
Da die Zust"ande eine Basis bilden, k"onnen wir diese zwischen
$A$ und $B$ einschieben:
\begin{align*}
\langle k|\,AB\,|l\rangle
&=
\sum_{r}
\langle k|\,A\,|r\rangle\langle r|\,B\,|l\rangle
\end{align*}
Dies ist aber genau das $kl$-Matrixelement das Matrizenprodukts
der Matrizen mit Matrixelementen
$\langle k|\,A\,|r\rangle$
und
$\langle r|\,B\,|l\rangle$.
\end{loesung}


\item
Beschreiben Sie die folgende Variante des Doppelspaltexperiments mit Hilfe 
von Analysatorkreisen.
Wie beim Originalexperiment werden zwei Teilchen auf die zwei Spalten
geschossen, die mit gleicher Wahrscheinlichkeit durch die Spalten fliegen.
Hinter den Spalten betrachten wir einen Sensor in der Mitte des Schirmes,
der konstruktive Interferenz feststellt.
Nun wird der eine Spalt ersetzt durch eine Kavit"at, in der sich die Phase
der Wellenfunktion eines Teilchens schneller "andert, als wenn es durch
den anderen Spalt fliegt\footnote{Elektronen, die eine Zone tieferen
Potentials durchqueren, erfahren genau so eine Phasenverschiebung in
ihrere Wellenfunktion.}
Die totale Phasenverschiebung ist einstellbar, sie ist $e^{i\delta}$.
Was beobachtet man im Sensor, wenn man $\delta$ ver"andert.

\begin{loesung}
Aus der Quelle kommen Teilchen im Zustand
\[
\frac1{\sqrt{2}}(\,|1\rangle+\,|2\rangle)
=
\frac1{\sqrt{2}}
\begin{pmatrix}
1\\1
\end{pmatrix}
\]
die am Spalt modifiziert werden mit der Matrix
\[
\begin{pmatrix}
1&0\\
0&e^{i\delta}
\end{pmatrix},
\]
bevor sie im Sensor $\langle S|$ "uberlagert werden.
Zusammengesetzt:
\[
\langle S|\,
\begin{pmatrix}
1&0\\
0&e^{i\delta}
\end{pmatrix}
\frac1{\sqrt{2}}
\begin{pmatrix}
1\\1
\end{pmatrix}
=
\frac1{\sqrt{2}}(
\langle S|1\rangle + e^{i\delta}\langle S|2\rangle
)
\]
Die Zahlen
$\langle S|l\rangle$
beschreiben die Wahrscheinlichkeit, dass ein Teilchen, welches durch
Spalt $l$ fliegt, im Detektor festgestellt wird.
Auf Grund der Annahmen der Aufgabe k"onnen wir davon ausgehen, dass
sie gleich, wir nennen sie $a$. So erhalten wir
\[
\frac{a}{\sqrt{2}}(1+e^{i\delta})
\]
Der Betrag des Klammerausdruckes ist
\begin{align*}
|1+e^{i\delta}|^2
&=2(1-\cos(\pi-\delta))
=4\sin^2\frac{\pi-\delta}2
\\
|1+e^{i\delta}|
&=
2\sin\frac{\pi-\delta}2
\end{align*}
Insbesondere wird bei einer Phasenverschiebung von $\delta=\pi$ 
destruktive Interferenz eintreten, und man wird im Sensor nichts
mehr detektieren.
\end{loesung}


\item
Ist der Operator 
\[
A=
\begin{pmatrix}
a^2&ab\\
 ab&b^2
\end{pmatrix}
\qquad\text{mit $a^2+b^2=1$}
\]
ein Projektor?
Wenn ja: auf welchen Vektor wird projiziert?

\begin{loesung}
$A$ ist ein Vektor genau dann, wenn $A^2=A$. Wir rechnen daher nach
\begin{align*}
A^2
&=
\begin{pmatrix}
a^2&ab\\
 ab&b^2
\end{pmatrix}
\begin{pmatrix}
a^2&ab\\
 ab&b^2
\end{pmatrix}
=
\begin{pmatrix}
 a^4+a^2b^2&  a^3b+ab^3\\
a^3b+ab^3  &a^2b^2+b^4
\end{pmatrix}
=
\begin{pmatrix}
a^2(a^2+b^2)& ab(a^2+b^2)\\
 ab(a^2+b^2)&b^2(a^2+b^2)
\end{pmatrix}
=A.
\end{align*}
Projiziert wird auf die Bildvektoren von $A$, das sind die Spalten von
$A$, also
\[
\begin{pmatrix}a^2\\ab\end{pmatrix}
\sim
\begin{pmatrix}a\\b\end{pmatrix}
\sim
\begin{pmatrix}ab\\b^2\end{pmatrix}.
\]
\end{loesung}



\item
Berechnen Sie die Entwicklung der Eigenzust"ande des Zweizustandssystems
mit zeitabh"angigem Hamilton-Operator
\[
H(t)
=
\begin{pmatrix}
E_1+\varepsilon\cos\omega t & 0                          \\
0                           & E_2-\varepsilon\cos\omega t
\end{pmatrix}
\]
f"ur $\varepsilon\ll E_i$.

\begin{loesung}
Die Schr"odingergleichung f"ur den zeitabh"angigen Zustandsvektor
\[
|\psi(t)\rangle
=
\begin{pmatrix}
v_1(t) \\
v_2(t)
\end{pmatrix}
\]
ist das Differentialgleichungssystem
\begin{align*}
i\hbar\dot v_1&=(E_1+\varepsilon\cos\omega t)v_1,\\
i\hbar\dot v_2&=(E_2-\varepsilon\cos\omega t)v_2.
\end{align*}
Durch Separieren
\[
\int\frac{dv_1}{v_1}
=
\frac{1}{i\hbar}\int E_1 + \varepsilon\cos\omega t\,dt
=
\frac1{i\hbar}\biggl(E_1t+\frac{\varepsilon}{\omega}\sin\omega t\biggr)
\]
kann man die L"osungen
\begin{align*}
v_1(t)
&=
v_1(0)e^{\frac1{i\hbar}(E_1t+\frac{\varepsilon}{\omega}\sin\omega t)}
,\qquad\text{und}
\\
v_2(t)
&=
v_2(0)e^{\frac1{i\hbar}(E_2t-\frac{\varepsilon}{\omega}\sin\omega t)}
\end{align*}
erraten, was man auch durch Einsetzen in die Schr"odingergleichung
nachpr"ufen kann.
\end{loesung}


\end{uebungsaufgaben}


\chapter{Hilbertr"aume\label{chapter:hilbertraeume}}
\lhead{Hilbertr"aume}
\rhead{}
Wir wissen schon, dass Zust"ande Eigenvektoren des Hamilton-Operators
sein sollen, wir brauchen
also einen Vektorraum, der ausreichend gross ist, alle Zust"ande aufzunehmen.
Ein typisches Quantensystem hat eine sehr grosse Zahl von Zust"anden,
vielleicht sogar unendlich viele.
Wir brauchen daher einen unendlichdimensionalen komplexen Vektorraum als
B"uhne f"ur die Quantenmechanik.
Der Begriff des Hilbertraumes liefert die gesuchte Struktur.

Ausgehend von der gewohnten Vorstellung eines endlichdimensionalen
Vektorraumes kann in vier Schritten eine ausreichend reichhaltige
Struktur aufgebaut werden:
\begin{enumerate}
\item Erweiterung der Definition eines reellen Vektorraumes dahingehend,
dass auch komplexe Skalare zugelassen werden.
\item Erweiterung des Begriffs des Skalarprodukts auf den komplexen Fall.
Dazu geh"oren auch der Begriff der L"ange eines Vektors und Ungleichungen
wie die Dreiecksungleichung, die unsere Intuition f"ur das Verhalten von
Abst"anden auf den unendlichdimensionalen Fall erweitern.
\item Der Begriff des Grenzwertes einer Vektorfolge erlaubt, Zust"ande
zu approximieren.
\item Die Hilbertbasis liefert eine Technik, wie die Zust"ande eines
Quantensystems als Basisvektoren f"ur den Zustandsraum verwendet werden
k"onnen.
\end{enumerate}

\section{Komplexe Vektorr"aume}
Ein Vektorraum "uber $\mathbb C$ ist einem Menge $V$ von Vektoren, mit
einer Addition von Vektoren und einer Multiplikation von Vektoren mit
komplexen Zahlen, mit folgenden Eigenschaften
\begin{compactenum}
\item Es gibt ein Element $0\in V$ mit der Eigenschaft $0+v=v,\forall v\in V$
und $0v=0$.
\item Zu jedem $v\in V$ gibt es ein Elemente $-v=(-1)v$ mit der Eigenschaft
$v+(-v)=0$
\item $u+v=v+u$
\item $(u+v)+w=u+(v+w)$
\item $(\lambda\mu)v=\lambda (\mu v)$
\item $\lambda(u+v)=\lambda u+ \lambda v$, $\lambda\in\mathbb C$, $u,v\in V$
\item $(\lambda+\mu)u=\lambda u+\mu u$
\end{compactenum}
Wie im Falle reeller Vektorr"aume kann man den Vektorraum
\[
{\mathbb C}^n = \left\{\left .
\begin{pmatrix}x_1\\\vdots\\x_n\end{pmatrix}
\right|
x_i\in\mathbb C
\right\}
\]
mit komponentenweisen Operationen konstruieren.
Ebenso kann man lineare Gleichungssysteme mit komplexen Koeffizienten,
komplexe Matrizen und das Matrizenprodukt f"ur komplexe Matrizenprodukt
konstruieren.

Der Gauss-Algorithmus kann unver"andert auch f"ur komplexe Gleichungssysteme
verwendet werden, so dass alle Eigenschaften, die man mit Hilfe des
Gauss-Algorithmus hergeleitet hat, auch f"ur komplexe Vektorr"aume
gelten.
Insbesondere ist der Rang definiert, die L"osungsmenge
eines Gleichungssystems ist eine Menge von komplexen Linearkombinationen
von Vektoren, und die (komplexe) Determinate ist genau dann 0, wenn
die Matrix singul"ar ist.

\section{Komplexes Skalarprodukt}
Das Skalarprodukt in reellen Vektorr"aumen ist eine Funktion von
zwei Vektoren, die linear ist in jedem Faktor:
\begin{align*}
(a+b,c)&=(a,c)+(b,c)&(a,b+c)&=(a,b)+(a,c)\\
(\lambda a,b)&=\lambda(a,b)&(a,\lambda b)&=\lambda(a,b)
\end{align*}
W"urde man diese Definition f"ur komplexe Zahlen verwenden, dann
m"usste das Skalarprodukt eines komplexen Vektors mit sich selbst
die Eigenschaft 
\[
(ia,ia)=i^2(a,a)=-(a,a)
\]
haben. Insbesondere k"onnte man $(a,a)$ nicht mehr als das Quadrat der
L"ange des Vektors $a$ interpretieren, denn die f"ur reelle Vektoren
geltende Formel
\[
(\lambda a,\lambda a)=|\lambda |\, (a,a)
\]
w"urde nicht mehr gelten.

Man verlangt daher, dass das Skalarprodukt eine Funktion von zwei Vektoren
ist, die linear ist im ersten Faktor, aber konjugiert linear im zweiten:
\begin{definition}
Ein Funktion $(a,b)$ von zwei Vektoren heisst Sesquilinearform, wenn
sie linear im zweiten Faktor und konjugiert linear im ersten Faktor ist.
\begin{align*}
(a,\lambda b)&=\lambda (a,b)&(\lambda a,b)=\bar\lambda (a,b).
\end{align*}
\end{definition}
Dies reicht aber nicht f"ur die Definition eines Skalarproduktes.
Das reelle Skalarprodukt hat die Eigenschaft, dass man die Faktoren
vertauschen kann. F"ur eine Sesquilinearform reicht das auch nicht:
\[
i(u,v)=(u,iv)=(iv,u)=-i(v,u)=-i(u,v),
\]
was nur richtig sein kann, wenn $(u,v)=0$. Man verlangt daher mehr:

\begin{definition}
Ein Sesquilinearform heist hermitesch, wenn gilt
\[
(u,v)=\overline{(v,u)}.
\]
\end{definition}
F"ur eine hermitesche Sesquilinearform ist sichergestellt, dass
das Skalarprodukt eines Vektors mit sich selbst eine reelle Zahl
ist. Man kann das einsehen, indem man im Produkt $(u,u)$ die beiden
Faktoren vertauscht:
\[
(u,u)=\overline{(u,u)}\quad\Rightarrow\quad (u,u)\in\mathbb R.
\]
Es ist aber immer noch nicht sichergestellt, dass man $(u,u)$ als
L"ange des Vektors interpretieren kann. 

\begin{definition}
Ein hermitesche Sesquilinearform heist {\em positiv definit}, wenn
$(u,u)>0$ f"ur $u\ne 0$. Ein positive definite hermitesche Sesquilinearform
heisst ein Skalarprodukt. Ein komplexer Vektorraum mit einem Skalarprodukt
heisst ein Pr"ahilbertraum.
\end{definition}


F"ur die Vektorr"aume $\mathbb C^n$ kann man dann auch eine Formel
f"ur das Skalarprodukt angeben. Sind $a_i$ die Komponenten von $a$ und
$b_i$ die Komponenten von $b$, dann ist das Skalarprodukt
\[
(a,b)=\sum_{i=1}^n \bar a_ib_i.
\]

Im Falle der reellen Vektorr"aume konnte man das Skalarprodukt
auch mit Hilfe des Matrizenproduktes schreiben: $u\cdot v=u^tv$.
Dies ist f"ur das Skalarprodukt nicht mehr m"oglich, denn $u^tv$
ist linear in beiden Vektoren.
Damit das Produkt konjugiert linear ist im ersten Faktor, m"ussen
wir $u$ nicht nur transponieren, sondern die Komponenten komplex konjugieren.
Wir efinieren daher:
\[
u=\begin{pmatrix}u_1\\\vdots\\u_n\end{pmatrix}
\quad\Rightarrow\quad
u^*=\begin{pmatrix}\bar u_1&\dots&\bar u_n\end{pmatrix}
\]
Das Skalarprodukt ist dann $(u,v)=u^*v$.

\section{Norm und Grenzwert}
Ein Skalarprodukt in einem Pr"ahilbertraum kann dazu benutzt werden,
die L"ange von Vektoren zu definieren und damit auch das Konzept eines
Grenzwertes einer Folge von Vektoren.

\begin{definition}
Die Norm eines Vektors in einem Pr"ahilbertraum ist
\[
\| v\| = \sqrt{(v,v)}.
\]
\end{definition}

\begin{satz}[Cauchy-Schwarz-Ungleichung] F"ur zwei Vektoren $u$ und $v$
in einem Pr"ahilbertraum gilt die Ungleichung
\[
|(u,v)| \le \| u\|\cdot \| v\|
\]
\end{satz}

\begin{proof}[Beweis]
F"ur $t\in\mathbb C$ berechnen wir das Produkt $(u-tv, u-tv)$
\begin{align*}
0&\le (u-tv,u-tv)\\
 &=   (u,u) - t(u,v) - \bar t(v,u) +   t\bar t(v,v) 
\end{align*}
Jetzt setzen wir $t=\frac{(v,u)}{(v,v)}$:
\begin{align*}
0&\le (u,u) - \frac{(v,u)(u,v)}{(v,v)} - \frac{(u,v)(v,u)}{(v,v)} + \frac{(u,v)(v,u)}{(v,v)^2}(v,v)\\
 &=(u,u) - 2\frac{|(u,v)|^2}{(v,v)} +\frac{|(u,v)|^2}{(v,v)}\\
 &=(u,u) -\frac{|(u,v)|^2}{(v,v)}\\
|(u,v)|^2&\le (u,u)\cdot (v,v) = \|u\|^2\cdot \|v\|^2.
\end{align*}
\end{proof}

\section{Hilbertbasis}
Die endlichdimensionalen Hilbertr"aume $\mathbb C^n$ haben die
Standardbasisvektoren als Basis, jeder Vektor in $\mathbb C^n$
kann als komplexe Linearkombination der Standardbasisvektoren
$e_i,i=1,\dots,n$ dargestellt werden.

Gr"ossere Hilbertr"aume $\cal H$ m"ussen nicht unbedingt endlich dimensional
sein. Dies bedeutet, dass es eine nicht endende Folge von Vektoren
$v_i\in\cal H$ gibt, die alle orthogonal und von L"ange $1$ sind:
\[
(v_i,v_j)=\delta_{ij}.
\]
Ein Hilbertraum $\cal H$ heisst separabel, wenn es eine solche Folge
gibt, so dass sich jeder Vektor $v\in\cal H$ beliebig genau als
Linearkombination von Vektoren $v_i$ approximiert werden kann.
Die Vektoren $v_i$ heissen Hilbertbasis von $\cal H$.

Um die Darstellung von $v\in\cal H$ als Linearkombination von Vektoren
der Hilbertbasis zu finden, setzen wir die unbekannten Koeffizienten
als 
\[
v=\sum_{i\in\mathbb N}c_iv_i
\]
an. Um die Koeffizienten $c_i$ zu bestimmen, berechnen wir die
Skalarprodukte
\[
(v_i,v)
=\sum_{j\in\mathbb N} c_j(v_i,v_j)=\sum_{j\in\mathbb N}c_j\delta_{ij}=c_i.
\]
Insbesondere folgt
\begin{align*}
v&=\sum_{i\in\mathbb N}(v_i,v) v_i,\\
\| v\|^2&=\sum_{i\in\mathbb N} |(v_i,v)|^2.
\end{align*}
Die zweite Gleichung heist auch die Parseval-Gleichung.

Es gibt Hilbertr"aume, die nicht separabel sind, doch f"ur unsere Zwecke
der Quantenmechanik reichen die separablen Hilbertr"aume aus.

\section{Hilbertr"aume $l^2$ und $L^2$}
In diesem Abschnitt betrachten wir zwei Hilbertr"aume, die als 
Modelle f"ur quantenmechanische Zustandsr"aume dienen werden.

\subsection{$l^2$ als Erweiterung von $\mathbb C^n$}
Der Hilbertraum $l^2$ ist eine Erweiterung des endlichdimensionalen
Raumes $\mathbb C^n$. Als Vektorraum besteht $l^2$ aus Folgen von
komplexen Zahlen:
\[
l^2=\left\{
(c_i)_{i\in\mathbb N}\,\left|\,c_i\in\mathbb C,
\sum_{i\in\mathbb N} |c_i|^2 <\infty
\right.\right\}
\]
Addition von Folgen und Multiplikation mit komplexen Zahlen erfolgt 
komponentenweise. Das Skalarprodukt zweier Folgen ist
\[
(a, b)=\sum_{i\in\mathbb N} \bar a_i b_i.
\]
Die Folgen $e_i=(\delta_{ij})_{j\in\mathbb N}$ bilden eine Hilbertbasis
des Hilbertraumes $l^2$.

\section{Bra-Ket-Notation}
Richard Feynman hat eine Notation eingef"uhrt, welche die vielen verschiedenen
Notationen f"ur in der Quantenmechanik n"utzliche Hilbertr"aume
vereinheitlicht.
F"ur die physikalische Interpretation ist egal, welche Art von
Vektorraum f"ur die Beschreibung der Zust"ande eines Teilchens verwendet
wird.
Die Notation sollte sich also nicht "andern, wenn wir die Zust"ande eines
Elektrons in einem Atom mit Hilfe einer Wellenfunktion (also mit dem
Hilbertraum $L^2$) beschreiben. Alternativ k"onnten wir die diskreten
Zust"ande mit Hilfe eines Hilbertraumes $l^2$ verwenden.

Die Notation sollte alle Operationen mit Vektoren abzubilden erlauben,
und sie soll die f"ur die Quantenmechanik wichtigen Operationen besonders
bequem machen.

\subsection{Zustandsvektoren}
Die Zust"ande eines Teilchens k"onnen durch ganz verschiedene Parameter
beschrieben werden. Ein freies Teilchen kann zum Beispiel durch seinen
Impuls oder seine Position beschrieben werden. Ein Elektron in einem
Atom kann durch die Nummer der Schale beschrieben, in der es sich befindet. 
Statt von Vektoren (in $l^2$) oder Funktionen (in $L^2$) zu sprechen, k"onnen
wir also von irgend einer Art von Vektor in einem passenden Hilbertraum 
sprechen. Wir schreiben 
$|n=3,m=5\rangle$ f"ur einen Zustandsvektor, welcher zu den Quantenzahlen $n=3$ und $m=5$ geh"ort. Oder $|p\rangle$ f"ur den Zustandsvektor eines Teilchens
mit Impuls $p$. Oder $|x\rangle$ f"ur den Zustandsvektor eines Teilchens
mit Position $x$.

\subsection{Skalarprodukt}
Das Skalarprodukt wird in jedem Hilbertraum im Detail anders definiert,
einzig die algebraischen Eigenschaften sind dieselben. In der Bra-Ket-Notation
schreiben wir  f"ur das Skalarprodukt der Zustandsvektoren 
$|a\rangle$ und $|b\rangle$
\[
\langle a|b\rangle.
\]
In allen Beispielen von Hilbert-R"aumen hatten wir ein Art ``dualer''
Vektoren, im Falle eines endlichdimensionalen Vektorraumes waren dies
die hermitesch adjungierten Vektoren. Der Vektor $\langle a|$ ist eine
Verallgemeinerung dieser Idee, man k"onnte also schreiben
$\langle a|=|a\rangle^*$.

\subsection{Operatoren}
Operatoren sind lineare Abbildungen des Hilbertraumes. Der Operator $A$
erzeugt aus einem Zustandsvektor $|u\rangle$ einen neuen Vektor
$A|u\rangle$. Die Linearit"at von $A$ bedeutet
\[
A(\lambda |u\rangle + \mu |v\rangle)=\lambda A|u\rangle + \mu A|v\rangle.
\]

Ein Operator $A$ heisst selbstadjungiert, wenn $A^*=A$ gilt.
F"ur das Skalarprodukt bedeutet diese Eigenschaft, dass $(u, Av)=(Au,v)$ 
In der Bra-Ket-Notation schreiben wir 
\begin{equation}
(u,Av)=(Au,v)=\langle u|A|v\rangle.
\label{skalar-operator}
\end{equation}
F"ur selbstadjungierte Operatioren spielt es gar keine Rolle, auf welchen
der beiden Vektoren in einem Skalarprodukt er wirkt, und die Notation
(\ref{skalar-operator}) tr"agt dieser Eigenschaft Rechnung, indem sie
bez"uglich der beiden Vektoren symmetrisch ist.

Hat der Hilbertraum $\cal H$ eine Hilbertbasis $|i\rangle$, dann kann
man die Wirkung eines Operators auf einem beliebigen Zustandsvektor
bestimmen, wenn man seine Wirkung auf den Basisvektoren kennt.
Dazu muss man den Vektor $|u\rangle$ erst in der Hilbertbasis
ausgedr"ucken:
\[
|u\rangle = \sum_{i=0}^\infty u_i\, |i\rangle.
\]
Dann kann man $A|u\rangle$ mit Hilfe der Linearit"at berechnen:
\[
A|u\rangle = \sum_{i=0}^\infty u_iA|i\rangle.
\]
Wenn man das Resultat wieder in der Hilbertbasis ausdr"ucken will, 
muss man den Resultatvektor mit den Basisvektoren multiplizieren.
Der Koeffizient der Komponente $|j\rangle$ ist
\begin{equation}
\langle j|A|u\rangle
=
\sum_{i=0}^\infty u_i \langle j|A|i\rangle
=
\sum_{i=0}^\infty \langle j|A|i\rangle u_i
\label{matrixmultiplikation}
\end{equation}
Man nennt die Zahlen
\[
a_{ji}=\langle j|A|i\rangle
\]
die Matrixelemente des Operators $A$. Die Formel (\ref{matrixmultiplikation})
entspricht der bekannten Formel f"ur die Multiplikation einer Matrix
mit einem Vektor.

Wenn zwei Operatoren die gleichen Matrixelemente haben, dann stimmen
die Operatoren "uberein. Haben n"amlich zwei Operatoren $A$ und $A'$
die gleichen Matrixelemente, dann hat $A-A'$ die Matrixelemente $0$:
\[
\langle i|A-A'|j\rangle =0\qquad\forall i,j
\]

F"ur eine grosse Klasse von selbstadjungierten Operatoren $A$ gilt ein
Spektralsatz: es gibt eine Hilbertbasis des Hilbertraumes $\cal H$
von Eigenvektoren von $A$, d.~h.~Vektoren $|i\rangle>$, $i\in\mathbb N$
mit der Eigenschaft, dass 
\[
\langle i|j\rangle = \delta_{ij}=\begin{cases}
1&\qquad i=j\\
0&\qquad i\ne j
\end{cases}
\]
und 
\[
A|i\rangle=\alpha_i|i\rangle\qquad\forall i,
\]
d.~h.~$\alpha_i$ sind die Eigenwerte von $A$.
F"ur endlichdimensionale komplexe Vektorr"aume ist dies der bekannte
Satz, dass hermitesche Matrizen diagonalisiert werden k"onnen.

\subsection{Projektion}
Projektionen sind Operatoren $P$ mit der Eigenschaft $P^2=P$.
Ist $|u\rangle$ ein Zustandsvektor, dann kann man eine Projektion $P_u$
auf diesen Zustandsvektor konstruieren:
\begin{align*}
P_u|u\rangle&=|u\rangle,\\
P_u|v\rangle&=0\qquad \text{f"ur $|v\rangle$ mit $\langle v|u\rangle=0$}
\end{align*}
Wir schreiben 
\[
P_u=
|u\rangle\langle u|
.
\]
Diese Notation ist konsistent, denn 
\begin{align*}
|u\rangle\langle u|u\rangle&=|u\rangle\cdot 1\\
|u\rangle\langle u|v\rangle&=|u\rangle\cdot 0=0,\qquad
\text{f"ur $langle u|v\rangle = 0$}
\end{align*}
Sind $|i\rangle$ mit $i=1,\dots, n$ orthonormale Zustandsvektoren, also
\[
\langle i|j\rangle =\delta_{ij},
\]
dann ist der Operator
\[
P=\sum_{i=1}^n |i\rangle \langle i|
\]
eine Projektion, denn
\[
P^2=
\sum_{i=1, j = 1}^n
|i\rangle \langle i|j\rangle \langle j|
=
\sum_{i=1, j = 1}^n
\delta_{ij} |i\rangle\langle j|=\sum_{i=1}^n|i\rangle\langle i|=P.
\]

Wenn es f"ur den Operator $A$ auf dem Hilbertraum $\cal H$ eine
Hilbertbasis aus Eigenvektoren $|i\rangle$ mit Eigenwerten $\alpha_i$ gibt,
dann kann der Operator selbst
als eine Linearkombination von Projektoren geschrieben werden. Wir setzen
\[
A'=\sum_{i=0}^\infty |i\rangle \alpha_i \langle i|.
\]
Wir vergleichen die Matrixelemente von $A'$ und $A$:
\begin{align*}
\langle k| A' |l\rangle
&=
\sum_{i=0}^\infty \langle k|i\rangle \alpha_i \langle i|l\rangle
=\sum_{i=0}^\infty \delta_{ki}\alpha_i\delta_{il}
=\alpha_l\delta_{kl}
\\
\langle k|A|l\rangle
&=
\langle k|\alpha_l|l\rangle=\alpha_l\delta_{kl}
\end{align*}
Die Matrixelemente stimmen "uberein, also ist $A=A'$.


\chapter{Quantencomputer\label{chapter:quantencomputer}}
\lhead{Quantencomputer}
\rhead{}
Die Technologie des zwanzigsten Jahrhunderts hat automatische
Rechenmaschinen hervorgebracht, beliebige Rechnung durchf"uhren,
sofern sie die Speichergr"osse der Maschine nicht sprengen.
Sie beruhen auf einer physischen Codierung der Zahlen, mit denen
gerechnet werden soll, und einer Maschine, welche Codierungen 
in neue Codierungen umwandelt.
\index{Babbage, Charles}
Die Idee einer solchen Maschine geht auf Charles Babbage zur"uck,
der in den Jahren ab 1812 auch eine Realisierung als mechanische
Maschine angestrebt hat.

Moderne Computer sind alle konkrete Realsierungen des abstrakten
\index{Turing, Alan}
Konzeptes der Turing-Maschine, welches Alan Turing formuliert hat,
um zu analysieren, welche Arten von Berechnungen in welcher Zeit
"uberhaupt durchgef"uhrt werden k"onnen.
Es wurde erkannt, dass gewisse Probleme auf Turing-Maschinen
derart lange ben"otigen w"urden, dass man sie getrost als undurchf"uhrbar
betrachten kann.
\index{Faktorisierung}
Allgemein besteht die "Uberzeugung, dass das Problem der Faktorisierung
des Produktes von grossen Primzahlen in diese Kategorie geh"ort,
auch wenn dies bisher nicht bewiesen worden konnte.
Dies impliziert, dass eine solche Faktorisierung, auf der die Sicherheit
verschiedener kryptographischer Systeme basiert, ohne zus"atzliches
Wissen nicht durchf"uhrbar ist.

Eine wesentliche Eigenschaft solcher schwieriger Problem ist, dass
sie die parallel Evaluation sehr vieler M"oglichkeiten erfordern,
was in einem klassischen Computer nur sequenziell m"oglich ist.
Der Grund daf"ur ist, dass ein Bit immer nur in genau einem
Zustand sein kann, wenn man die verschiedenen M"oglichkeiten des
Bits evaluieren will, dann muss man das hintereinander tun.

In der Quantenmechanik gibt es aber Systeme, die in zwei Zust"anden
gleichzeitig sein k"onnen, wie wir im Abschnitt~\ref{section:cat}
illustrieren werden.
Wenn man also auch das Rechenwerk einer Turingmaschine durch eine
``Quantenschaltung'' ersetzen kann, dann kann ein relativ kleiner
Quantecomputer alle M"oglichkeiten aller Bits gleichzeitig evaluieren,
und damit das Problem in sehr kurzer Zeit l"osen.

\section{Klassische Computer}
\rhead{Klassische Computer}
\index{Computer!klassischer}
\index{Gatter!logisches}
Ein klassischer Computer kann bin"ar als Spannungen in einer
elektronischen Schaltung codierte Zahlen in ander Muster von
Spannungen umwandeln, welche das Resultat einer Rechenoperation
mit den Zahlen als Input darstellt.
\begin{figure}
\centering
\begin{tabular}{cccc}
&UND&ODER&XOR\\
\\
&
\includegraphics{graphics/gatter-1.pdf}&%
\includegraphics{graphics/gatter-2.pdf}&%
\includegraphics{graphics/gatter-5.pdf}
\\
\\
&
\includegraphics{graphics/gatter-3.pdf}&%
\includegraphics{graphics/gatter-4.pdf}&%
\includegraphics{graphics/gatter-6.pdf}
\end{tabular}
\caption{Logische Gatter von links nach rechts UND, ODER, XOR,
in der unteren Reihe invertiert.
\label{skript:gates}}
\end{figure}
\index{Halbaddierer}
\index{Volladdierer}
Die grundlegenden logischen Operationen werden dann durch sogenannte
Gatter implementiert, deren Schaltbilder in Abbildung~\ref{skript:gates} dargestellt
sind.
\begin{figure}
\centering
\includegraphics{graphics/gatter-8.pdf}
\caption{Halbaddierer, der Ausgang $\Sigma$ ist die Summe der beiden
Eing"ange $A$ und $B$, der Ausgang $C_\text{out}$ wird aktiv, wenn
ein "Ubertrag auftritt.
\label{skript:halfadder}}
\end{figure}
\begin{figure}
\centering
\includegraphics{graphics/gatter-7.pdf}
\caption{Volladdierer, berechnet die Summe der beiden Inputs $A$ und $B$
und den "Ubertrag $C_\text{in}$, und gibt die Summe $\Sigma$ und den
"Ubertrag $C_\text{out}$ aus.
\label{skript:fulladder}}
\end{figure}
Aus den Grundoperationn lassen sich komplexere Schaltungen 
aufbauen, zum Beispiel das Halbaddierwerk in Abbildung~\ref{skript:halfadder}
oder der Volladdierer in Abbildung~\ref{skript:fulladder}.
Die Digitaltechnik lehrt, wie man durch Kombination solcher Schaltung
beliebig komplexe Berechnungen anstellen kann.

Die Verbindungen in all diesen Schaltungen k"onnen nur in jeweils einem
Zustand sein, ein oder aus, $1$ oder $0$.
Dies ist eine Einschr"ankung der Technologie.
Selbst die Verwendung verschiedener Spannungsniveaus auf den Verbindungen
w"urde daran nichts "andern: zu jeder Zeit kann jede Verbindung nur
ein einem der m"oglichen Zust"ande sein.

\index{SAT}
Eines der schwierig zu l"osenden Probleme ist SAT, die Frage, ob eine
logische Formel durch geeignete Wahrheitsbelegung der Inputs wahr
werden kann. Implementiert man die logische Formal als Schaltung,
wird die Frage gleichbedeutend damit, ob der Ausgang der Schaltung
durch geeignete Beschaltung der Inputs in den Zustand $1$  gehen kann.
Bedingt durch die Technologie k"onnen wir die Frage nur dadurch beantworten,
dass wir alle m"oglichen Inputs durchprobieren. Bei $n$ Inputs sind
dies $2^n$ Belegungen, entsprechend lange dauert es, eine L"osung
zu finden.

\section{Schr"odingers Katze\label{section:cat}}
\rhead{Schr"odingers Katze}
\index{Katze!Schr\"odinger}
\begin{figure}
\centering
\includegraphics[width=0.5\hsize]{images/catliveanddead2.png}
\caption{Schr"odingersche Katze in einem "Uberlagerungszustand
\label{skript:deadandalive}}
\end{figure}
Dass in der Quantenmechanik ein System nicht mir in einem reinen Zustand
zu sein braucht, hat Erwin Schr"odinger mit seinem ber"uhmten 
Gedankenexperiment mit der Katze illustriert.
In einer Kiste befindet sich eine Katze und ein Mechanismus,
der beim radioaktiven Zerfall des darin befindlichen radioaktiven
Atomkernes eine Phiole mit Gift zerbricht, so dass es ausweichen und
die Katze t"oten kann.
Die Katze kann offensichtlich in zwei m"oglichen Zust"anden sein,
lebendig $|\smiley\rangle$ oder tot $|\frownie\rangle$. 
Zu beginn des Experiments befindet es sich im Zustand $|\smiley\rangle$.
Dieser Zustand der Katze wiederspiegelt nat"urlich nur den Zustand
des Atomkerns, der ebenfalls in zwei Zust"anden sein kann.

Mit fortschreitender Zeit steigt die Wahrscheinlichkeit, dass der
Atomkern zerfallen ist, und die Katze tot ist.
Genauer sind die Wahrscheinlichtkeiten, die Katze in den beiden
Zust"anden zu finden
\begin{align*}
|\langle \smiley|\psi(t)\rangle|^2
&=
2^{-t/t_{\frac12}}
&&\text{und}
&
|\langle \frownie|\psi(t)\rangle|^2
&=
1-2^{-t/t_{\frac12}}
\end{align*}
Die Katze ist also in einem Zustand
\[
|\psi(t)\rangle = 
\sqrt{2^{-t/t_{\frac12}}}e^{i\varphi_1(t)}\,|\smiley\rangle
+
\sqrt{1-2^{-t/t_{\frac12}}}e^{i\varphi_2(t)}\,|\frownie\rangle,
\]
die beiden Phasenfaktoren tragen der Tatsache Rechnung, dass komplexe
Linearkombinationen m"oglich sind (Abbildung~\ref{skript:deadandalive}).

Ist die Katze lebendig oder tot? Die Erfahrung mit makroskopischen
Katzen sagt uns, dass ein Katze entweder das eine oder andere ist.
Die Quantenmechanik sagt uns dagegen, dass wir es nicht wissen k"onnen.
Wir k"onnen nur wissen, dass die Katze in einem "Uberlagerungszustand
ist.
Das einzige, was wir daraus ableiten k"onnen, ist mit welcher
Wahrscheinlichkeit sie bereits tot ist.
Gewissheit dar"uber, in welchem Zustand sich die Katze befindet,
erhalten wer genau in dem Moment, wo wir das Experiment durchf"uhren
und den Zustand der Katze neu ermitteln.

Durch den Prozess der Beobachtung der Katze "andert sich deren Zustand.
Lebt sie noch, wissen wir, dass sie sich im Zustand $|\smiley\rangle$ befindet.
Ist sie tot, befindet sie sich im Zustand $|\frownie\rangle$.
Die Beobachtung hat also den Zustand ver"andert.

Wir erweitern jetzt das urspr"ungliche Gedankenexperiment von Schr"odinger.
Wir stellen uns vor, wir m"ochten gerne eine exotische Eigenschaft von
Katzen messen, insbesondere m"ochten wir wissen eine lebende Katze oder
eine tote Katze diese Eigenschaft hat. Das klassische Experiment
daf"ur w"are, je eine lebendige und eine tote Katze zu pr"aparieren,
und dann den Test f"ur die Eigenschaft auf beide anzuwenden.

F"ur das quantenmechanische Experiment brauchen wir offenbar einen
Operator $Q$ (f"ur ``question''), der zwei m"ogliche Ausgangszust"ande hat:
die Katze mit der
Eigenschaft $|Y\rangle$ und die Katze ohne die Eigenschaft $|N\rangle$.
Das zugeh"orige Experiment k"onnte man mit der Notation aus
Kapitel~\ref{chapter:einfache-quantensysteme} als
\begin{center}
\includegraphics{graphics/gatter-9.pdf}
\end{center}
darstellen. 

F"ur Schr"odingers Katze k"onnen wir die Frage, ob die Eigenschaft $Q$
in f"ur lebende oder tote Katzen vorhanden ist, jin einem einzigen
Durchgang beantworten.
Dazu stellen wir zuerst eine Katze in einem "Uberlagerungszustand
$|\psi\rangle$ von $|\smiley\rangle$ und $|\frownie\rangle$ her\footnote{
Man beachte, dass dies nicht bedeutet, dass die H"alfte der so pr"aparierten
Katzen leben und die anderen tot sind.
Vielmehr befindet sich die Katze in beiden Zust"anden, erst bei der
Messung wird festgelegt, was der in dieser Durchf"uhrung des Experimentes
gemessene Zustand ist.
}.
Auf diesen Zustand wenden wir den Operator $Q$ an, und testen dann,
ob es Katzen gibt, die sich im Zustand $|Y\rangle$ befindet, indem
wir $\langle Y|\,Q\,|\psi\rangle$ messen.
Falls wir herausfinden, dass $\langle Y|\,Q\,|\psi\rangle\ne 0$,
dann kann eine Katze die Eigenschaft haben, auch wenn wir noch nicht
wissen, ob es eine Eigenschaft von toten oder lebenden Katzen ist.

Schr"odingers Katze in ihrem "Uberlagerungszustand kann also dazu
verwendet werden, einen Quantencomputer zu bauen, der Probleme "uber
Katzeneigenschaften l"osen kann.
Im Gegensatz zu einem klassischen Computer k"onnen wir aber nicht
erwarten, dass wir in einer einzigen Messung das Resultat der
Berechnung bekommen k"onnen.
Das Quantensystem kann nur eine Wahrscheinlichkeitsaussage liefern,
und Messungen der Wahrscheinlichkeit verlangen immer eine grosse Zahl
von Experimenten.

Ein praktisch n"utzlicher Quantencomputer muss noch weit gr"ossere
Qubits realisieren.
Ein Quantencomputer mit $n$ Qubits ist in der Lage, einen Zustand
herzustellen, in dem $2^n$ Zust"ande "uberlagert sind. 

\section{Probabilistische Algorithmen}
\rhead{Probabilistische Algorithmen}
\index{Algorithmus!probablistischer}
Bis jetzt ist klar, dass wir von einem Quantencomputer keine definitiven
Antworten erhalten k"onnen, sondern nur Aussagen "uber Wahrscheinlichkeiten.
Dies ist aber nicht wirklich etwas Neues, denn schon lange werden
zum Beispiel in der Kryptographie Algorithmen verwendet, die ebenfalls
nur probabilistische Aussagen machen.
Wenn man eine RSA Schl"usselpaar f"ur ein X.509-Zertifikat erzeugen will,
muss testen, ob eine mehrere Tausend Bit lange Zahl $n$ eine Primzahl ist.
Es w"urde viel zu lange dauern, die Primzahleigenschaft durch Testdivision 
zu ermitteln.
Daher verwendet man einen probabilistischen Test, der nur mit einer
gewissen Wahrscheinlichkeit erkennt, wenn eine Zahl nicht Primzahl ist
\cite{skript:miller-rabin}.

\begin{satz}[Miller-Rabin Primzahlkriterium]
\label{skript:quantencomputer:miller-rabin-kriterium}
\index{Primzahlkriterium!Miller-Rabin}
Gegeben ist eine Zahl $n$. W"ahle eine zuf"allige Zahl $a<n$. Sei $d$
eine ungerade Zahl so, dass $2^sd=n-1$. Wenn 
\[
a^d\not\equiv 1\mod n
\qquad\text{und}\qquad
a^{2^rd}\not\equiv -1\mod n\quad\forall 0\le r\le s-1
\]
dann ist $n$ keine Primzahl.
\end{satz}

Der Satz l"asst offen, ob man mit jeder Zahl $a$ erkennen kann, ob
eine Zahl $n$ nicht Primzahl ist.
Tats"achlich gilt f"ur einen Viertel der in Frage kommenden Zahlen
$a$, dass man mit ihnen nicht erkennen kann, ob eine Zahl $n$ prim ist.
Wenn also eine Zahl den Test besteht, dann ist die Wahrscheinlichkeit,
dass sie trotzdem keine Primzahl ist, immer noch $\frac14$. Wiederholt
man das Experiment mit mehreren Zufallszahlen $a$, kann man die
Wahrscheinlichkeit, eine Zahl $n$ nicht als zusammengesetzte Zahl zu
erkennen weiter reduzieren:

\begin{satz}[Miller-Rabin Primzahltest]
\label{skript:quantencomputer:miller-rabin-test}
\index{Primzahltest!Miller-Rabin}
Wenn eine Zahl $n$ f"ur $k$ zuf"allige Zahlen $a<n$ vom
Primzahlkriterium~\ref{skript:quantencomputer:miller-rabin-kriterium}
nicht als zusammengesetzt ist, dann ist die Wahrscheinlichkeit,
dass sie zusammengesetzt ist, kleiner als $4^{-k}$.
\end{satz}

Der Satz~\ref{skript:quantencomputer:miller-rabin-test} liefert einen Algorithmus
mit polynomieller Laufzeit, der die Primzahleigenschaft testen kann,
aber nur mit einer gewissen beschr"ankten Wahrscheinlichkeit.
Die Probleme, die sich mit einem probabilistischen Algorithmus mit
beschr"ankter Wahrscheinlichkeit in polynomieller Zeit gel"ost
werden k"onnen, bilden die Klasse BPP.
\index{BPP, Komplexit\"atsklasse}
\index{Komplexit\"atsklasse!BPP}
\index{P, Komplexit\"atsklasse}
\index{Komplexit\"atsklasse!P}
Die Klasse P der in polynomieller Zeit l"osbaren Probleme ist darin
enthalten: $\text{P}\subset\text{BPP}$.

\section{Quantencomputer}
\rhead{Quantencomputer}
\index{Quantencomputer}
\begin{figure}
\centering
\includegraphics[width=\hsize]{images/dilbert.png}
\caption{Quanten-Computer-Projekt bei Dilbert in einem quantenmechanischen
"Uberlagerungszustand, mit besonderer Ber"ucksichtigung der Wirkung einer
Beobachtung.
\label{skript:dilbert}}
\end{figure}
Nach dem einf"uhrenden Beispiel "uber den Schr"odingerschen
Katzen-Quanten-Computer k"onnen wir uns jetzt dar"uber Gedanken
machen, wie ein Quanten-Computer aussehen m"usste, der allgemeine
mathematische Probleme l"osen k"onnte.
Ein solcher Computer existiert noch nicht, aber ein grosse Zahl von 
Forschungsgruppen (siehe auch Abbildung~\ref{skript:dilbert})
arbeiten daran, die nachstehenden Ideen auf die eine
oder andere Art umzusetzen.

Ein Quantencomputer muss zun"achst einen geeignet komplexen
"Uberlagerungszustand pr"aprieren, den Qubits. Dann muss dieser Zustand
von einem oder mehreren Quanten-Gattern verarbeitet werden, sie
entsprechen dem Operator $Q$ im Schr"odinger-Katzen-Computer.
Schliesslich muss man das Experiment mehrmals durchf"uhren, bis man
Wahrscheinlichkeiten mit gen"ugend grosser Genauigkeit bestimmt hat,
dass man ein Resultat der Berechnung daraus ableiten kann.

Ein solcher Computer bestimmt also mit beschr"ankter Wahrscheinlichkeit
in polynomieller Zeit eine L"osung f"ur ein Problem. Die
Klasse der Probleme, die sich mit beschr"ankter Wahrscheinlichkeit in
polynomieller Zeit auf einem Quantencomputer l"osen lassen, nennen wir
BQP.
Es ist klar, dass $\text{BPP}\subset\text{BQP}$.
\index{BQP, Komplexit\"atsklasse}
\index{Komplexit\"atsklasse!BQP}
Es besteht also die Hoffnung, dass sich in der Klasse BQP Probleme finden,
die ein klassischer Computer auch mit einem probabilistischen Algorithmus
nicht in polynomieller Zeit l"osen kann.

\subsection{Qubits}
Ein quantenmechanisches System mit zwei m"oglichen Zust"ande $|0\rangle$
und $|1\rangle$ muss sich nicht
notwendigerweise in genau einem der Zust"ande befinden.
Jede Linearkombination der beiden Zust"ande ist ebenfalls ein g"ultiger
Zustand, sofern sie als Vektor L"ange $1$ hat.
Der Zustand
\[
|\psi\rangle
=
\lambda \,|0\rangle + \mu\,|1\rangle
\]
hat L"ange $1$ wenn gilt
\[
\langle\psi|\psi\rangle
=
(
\bar\lambda
\langle 0|
+
\bar\mu
\langle 1|
)
(
\lambda \,|0\rangle + \mu\,|1\rangle
)
=
|\lambda|^2\langle 0|0\rangle + |\mu|^2\langle 1|1\rangle
=
|\lambda|^2+|\mu|^2
=
1
,
\]
da die gemischten Terme wegen $\langle 0|1\rangle=0$ wegfallen.
Es gen"ugt also, dass die Quadratsumme der Betr"age von $\lambda$ und $\mu$
den Wert $1$ ergibt.
Insbesondere gibt es selbst f"ur dieses einfache System bereits viel
mehr m"ogliche Zust"ande als bei einem klassischen Bit.
Ein solches quantenmechanisches System nennt man in Qubit.

Ein klassisches System mit zwei Bits kann in 4 m"oglichen Zust"anden sein.
Das zugeh"orige quantenmechanische System kann in einer beliebigen
"Uberlagerung der vier m"oglichen reinen Zust"ande sein:
\[
|\psi\rangle
=
\alpha_{00}|00\rangle
+
\alpha_{01}|01\rangle
+
\alpha_{10}|10\rangle
+
\alpha_{11}|11\rangle
,\qquad
|\alpha_{00}|^2
+
|\alpha_{01}|^2
+
|\alpha_{10}|^2
+
|\alpha_{11}|^2
=1.
\]
Wir nennen dies ein ``2-Qubit Register''.

Praktisch n"utzliche Quantencomputer m"ussen noch wesentlich gr"ossere
Qubits verarbeiten k"onnen.
Ein Register mit $n$ Qubits ist in einem Zustand, in dem $2^n$ Zust"ande
"uberlagert sind.

Es ist auch denkbar, dass wir beim Initialisieren eines Qubits
einzelne auf einen bestimmten Wert setzen.
Zum Beispiel k"onnen wir in einem Register mit $2$ Qubits das erste
Qubit auf $0$ setzen, wir erhalten dann einen Zustand der
Form
\[
|\psi\rangle
=
\alpha_{00}|00\rangle
+
\alpha_{01}|01\rangle
\]
Wir nennen das erste Bit ein Scratch-Bit.
\index{Scratch-bit}

\index{Tensor-Produkt}
Die Operation, aus einzelnen Qubits ein Register von Qubits zu
machen, ist so grundlegend, dass wir daf"ur eine eigene Notation
verwenden wollen.
Sind $H_1$ und $H_2$ die Hilbertr"aume, in denen die Zust"ande f"ur
die einzelnen Bits zu finden sind, dann schreiben wir $H_1\otimes H_2$
f"ur den Raum, der die kombinierten Zust"ande enth"alt.
Sind $|a\rangle\in H_1$ und $|b\rangle\in H_2$ Zust"ande einzelner
Qubits, dann schreiben wir f"ur den kombinierten Zustand auch
$|ab\rangle=|a\rangle\,|b\rangle=|a\rangle\otimes|b\rangle=|a\otimes b\rangle$.
Das Skalarprodukt in $H_1\otimes H_2$ ist gegeben durch
\[
\langle a\otimes b|c\otimes d\rangle
=
\langle a|c\rangle\,\langle b|d\rangle.
\]
Die Zust"ande $|0\otimes 0\rangle$,
$|0\otimes 1\rangle$,
$|1\otimes 0\rangle$ und
$|1\otimes 1\rangle$ sind dann immer noch orthogonal.

\begin{beispiel}
Man berechne das Skalarprodukt der beiden Zust"ande
\[
|\psi\rangle
=
\frac1{\sqrt{2}} |0\rangle\otimes|1\rangle
+
\frac1{\sqrt{2}} |1\rangle\otimes|0\rangle
\qquad
\text{und}
\qquad
|\varphi\rangle
=
\alpha |0\rangle\otimes|1\rangle
+
\beta |1\rangle\otimes|0\rangle.
\]
Wir k"onnen verwenden, dass die Basisvektoren $|a\otimes b\rangle$ orthogonal
sind, und schliessen, dass das Skalarprodukt
$\frac1{\sqrt{2}}\alpha + \frac1{\sqrt{2}}\beta$ sein muss.
Wir wollen uns aber "uberzeugen, dass die Definitionen ``funktionieren''
und rechnen daher das Skalarprodukt auch explizit aus.
Nach Definition des Skalarproduktes in $H_1\otimes H_2$ gilt
\begin{align*}
\langle\psi|\varphi\rangle
&=
\biggl(
\frac1{\sqrt{2}} \langle 0 \otimes 1|
+
\frac1{\sqrt{2}} \langle 1\otimes0|
\biggr)
\biggl(
\alpha |0\otimes 1\rangle
+
\beta |1\otimes 0\rangle.
\biggr)
\\
&=
\frac1{\sqrt{2}} \alpha\langle 0\otimes 1|0\otimes 1\rangle
+
\frac1{\sqrt{2}} \beta \langle 0\otimes 1|1\otimes 0\rangle
+
\frac1{\sqrt{2}} \alpha\langle 1\otimes 0|0\otimes 1\rangle
+
\frac1{\sqrt{2}} \beta \langle 1\otimes 0|1\otimes 0\rangle
\\
&=
\frac1{\sqrt{2}} \alpha\underbrace{\langle 0|0\rangle}_{=1} \underbrace{\langle 1|1\rangle}_{=1}
+
\frac1{\sqrt{2}} \beta \underbrace{\langle 0|1\rangle}_{=0} \underbrace{\langle 1|0\rangle}_{=0}
+
\frac1{\sqrt{2}} \alpha\underbrace{\langle 1|0\rangle}_{=0} \underbrace{\langle 0|1\rangle}_{=0}
+
\frac1{\sqrt{2}} \beta \underbrace{\langle 1|1\rangle}_{=1} \underbrace{\langle 0|0\rangle}_{=1}
\\
&=
\frac1{\sqrt{2}}\alpha + \frac1{\sqrt{2}}\beta,
\end{align*}
wir erhalten also genau das erwartete Resultat.
\end{beispiel}

\subsection{Gatter}
Klassische Computer f"uhren ihre Berechnung mit logischen Gattern durch,
die auf einem Vektor von Bits wirken.
Quantencomputer brauchen Quanten-Gatter, welche auf einem
$n$ Qubit Zustand wirken.
Solche Gatter d"urfen selbst keine Messungen machen, den die
Auswertung der Berechnung darf ja erst ganz am Ende der
Transformationen erfolgen.
Die Gatter sind also Zustands"anderungen, die umkehrbar sein m"ussen.
Eine physikalische Ralisierung eines Gatters ist nichts anderes
als eine Zeitentwicklung des urspr"unglichen Zustands durch einen
speziellen Hamilton-Operator.

\begin{definition}
\index{Gatter!Quanten-}
Ein Quanten-Gatter ist ein unit"arer Operator, der auf einem 
$n$-Qubit Zustand operiert.
\end{definition}

\begin{beispiel}
\index{Flip}
\index{Gatter!Flip}
Die {\em Flip Operation} implementiert die Negation eines einzelne Qubit:
\begin{align*}
|0\rangle&\mapsto |1\rangle & |1\rangle&\mapsto |0\rangle.
\end{align*}
Die zugeh"orige Matrix ist
\[
F=\begin{pmatrix}
0&1\\
1&0
\end{pmatrix}.
\]
$F$ ist eine unit"are Matrix, als ein Quanten-Gatter.
\end{beispiel}

\begin{beispiel}
\index{Bit-Umordnung}
\index{Gatter!Bit-Umordnung}
Die {\em Bit-Umordnung} vertauscht zwei Qubits, d.~h.~sie implementiert
die ABbildung:
\begin{align*}
|00\rangle&\mapsto |00\rangle\\
|01\rangle&\mapsto |10\rangle\\
|10\rangle&\mapsto |01\rangle\\
|11\rangle&\mapsto |11\rangle
\end{align*}
Die zugeh"orige Matrix ist
\[
X=
\begin{pmatrix}
1&0&0&0\\
0&0&1&0\\
0&1&0&0\\
0&0&0&1
\end{pmatrix}.
\]
Die Matrix $X$ ist unit"ar, weil orthogonal.
\end{beispiel}

\begin{beispiel}
\index{Phasenshift}
\index{Gatter!Phasenshift}
{\em Phasenshift} ist eine Operation, die den Zust"and $|1\rangle$ 
mit einem Phasenfaktor versieht:
\begin{align*}
|0\rangle &\mapsto |0\rangle\\
|1\rangle &\mapsto i\,|1\rangle
\end{align*}
Die zugeh"orige Matrix ist
\[
\begin{pmatrix}
1&0\\
0&i
\end{pmatrix},
\]
die Matrix ist nicht orthogonal, aber unit"ar.
\end{beispiel}

\begin{beispiel}
\index{Hadamard}
\index{Gatter!Hadamard}
Die {\em Hadamard-Operation} implementiert eine lineare Abbildung
\begin{align*}
|0\rangle &\mapsto |0\rangle + |1\rangle\\
|1\rangle &\mapsto |0\rangle - |1\rangle,
\end{align*}
wobei die Vektoren auf der rechten Seite noch normiert werden m"ussen.
Die zugeh"orige Matrix ist
\[
H=
\frac1{\sqrt{2}}
\begin{pmatrix}
1&\phantom{1}1\\
1&-1
\end{pmatrix}.
\]
Die Matrix $H$ ist orthogonal, also insbesondere auch unit"ar.
\end{beispiel}

\begin{beispiel}
\index{Gatter!UND}
\index{Gatter!ODER}
Es gibt weder ein UND- noch ein ODER-Gatter f"ur Quantencomputer.
Ein UND-Gatter m"usste die Abbildung
\begin{align*}
|00\rangle&\mapsto|00\rangle,\\
|01\rangle&\mapsto|00\rangle,\\
|10\rangle&\mapsto|00\rangle,\\
|11\rangle&\mapsto|01\rangle
\end{align*}
implementieren, welche nicht invertierbar ist, und daher auch nicht
unit"ar sein kann.
Selbst wenn man die Operationen UND und ODER in eine einzige Abbildung
kombiniert, erh"alt man
\begin{align*}
|00\rangle&\mapsto|00\rangle,\\
|01\rangle&\mapsto|10\rangle,\\
|10\rangle&\mapsto|10\rangle,\\
|11\rangle&\mapsto|11\rangle
\end{align*}
was immer noch nicht invertierbar ist, denn man kann aus den Resultatbits
nicht mehr schliessen, welches der beiden Bits gesetzt war.
Als letzte Hoffnung k"onnten wir versuchen, eines der Bits zu behalten,
aber das funktioniert auch nicht:
\begin{align*}
|00\rangle&\mapsto|00\rangle,\\
|01\rangle&\mapsto|00\rangle,\\
|10\rangle&\mapsto|10\rangle,\\
|11\rangle&\mapsto|11\rangle,
\end{align*}
immer noch nicht invertierbar.
\end{beispiel}

\begin{beispiel}
\index{CNOT-Gatter}
\index{Gatter!CNOT}
Das \textsc{CNOT}-Gatter invertiert das zweite Bit, wenn das erste Bit
gesetzt ist.
Es implementiert also die Abbildung
\begin{align*}
|00\rangle&\mapsto|00\rangle,\\
|01\rangle&\mapsto|01\rangle,\\
|10\rangle&\mapsto|11\rangle,\\
|11\rangle&\mapsto|10\rangle
\end{align*}
Die zugeh"orige Matrix ist
\[
U_{\textsc{CNOT}}=\begin{pmatrix}
1&0&0&0\\
0&1&0&0\\
0&0&0&1\\
0&0&1&0
\end{pmatrix}.
\]
Diese Matrix ist offensichtlich unit"ar.
Manchmal wird das \textsc{CNOT}-Gatter mit dem Diagramm
\begin{center}
\includegraphics{graphics/gatter-10.pdf}
\end{center}
dargestellt werden.
Man beachtet, dass dieses Diagramm im Gegensatz zu den Analysator-Diagrammen
von Kapitel~\ref{chapter:einfache-quantensysteme} von links nach rechts
gelesen werden m"ssen.
\end{beispiel}

\begin{beispiel}
\index{Tofoli-Gatter}
\index{Gatter!Tofoli}
Das {\em Tofoli-Gatter} operatiert auf zwei Qubits $a$ und $b$ und einem
Scratch-Bit $c$ und implementiert die Abbildung
\[
|abc\rangle
\mapsto
|a\rangle|b\rangle|(a\wedge b)\oplus c\rangle.
\]
Darin ist $\oplus$ die XOR-Operation. 
Im Detail beschreibt dies folgende Abbildung
\begin{align*}
|000\rangle&\mapsto|000\rangle\\
|001\rangle&\mapsto|001\rangle\\
|010\rangle&\mapsto|010\rangle\\
|011\rangle&\mapsto|011\rangle\\
|100\rangle&\mapsto|100\rangle\\
|101\rangle&\mapsto|101\rangle\\
|110\rangle&\mapsto|111\rangle\\
|111\rangle&\mapsto|110\rangle
\end{align*}
In dieser Reihenfolge der Basisvektoren geh"ort dazu die Matrix
\[
T=
\begin{pmatrix}
1&0&0&0&0&0&0&0\\
0&1&0&0&0&0&0&0\\
0&0&1&0&0&0&0&0\\
0&0&0&1&0&0&0&0\\
0&0&0&0&1&0&0&0\\
0&0&0&0&0&1&0&0\\
0&0&0&0&0&0&0&1\\
0&0&0&0&0&0&1&0\\
\end{pmatrix}.
\]
Die Matrix $T$ ist orthogonal, und damit auch unit"ar.
Das Tofoli-Gatter wird manchmal auch als Diagramm
\begin{center}
\includegraphics{graphics/gatter-11.pdf}
\end{center}
dargestellt. Das Zeichen \textsc{AND} erinnert daran, dass das Tofoli-Gatter
eigenlicht die UND-Verkn"upfung der beiden Inputs $a$ und $b$ berechnet.
\end{beispiel}

\begin{beispiel}
\begin{figure}
\centering
\includegraphics{graphics/gatter-12.pdf}
\caption{Implementation eines \textsc{CNOT}-Gatters aus einem Tofoli-Gatter
mit Hilfe eines Scratch-Bit \label{skript:cnot=tofoli}}
\end{figure}
Man kann aus einem Tofoli-Gatter mit Hilfe eines Scratch-Bit ein
\textsc{CNOT}-Gatter machen, wie in Abbildung~\ref{skript:cnot=tofoli}
dargestellt.
\end{beispiel}

\subsection{No-Cloning Theorem}
Ein klassischer Computer kann ein Ausgangssignal als Input f"ur viele
nachfolgende Gatter verwenden, das Bit wird einfach auf mehrere Leitungen
kopiert. 
Ein Quanten-Computer kann das nicht:
\begin{satz}
\label{skript:no-cloning-theorem}
Es gibt keine unit"are Matrix, die einen beliebigen Zustand kopieren
kann, d.~h.~die Abbildung
\[
|\psi\rangle\,|k\rangle\mapsto |\psi\rangle\,|\psi\rangle
\]
kann nicht unit"ar sein.
\end{satz}

\begin{proof}[Beweis]
Nehmen wir an, es g"abe eine unit"are Abbildung $U$, die ein Qubit clonen
kann:
\[
U(|\psi\rangle\,|k\rangle)=|\psi\rangle\,|\psi\rangle.
\]
Wenden wir $U$ auf einen analog konstruierten Zustand
$|\varphi\rangle,|k\rangle$ an, dann muss, weil $U$ unit"ar ist,
das Skalarprodukt der beiden beiden Zust"ande vor und nach der Anwendung
von $U$ gleich sein.
\begin{equation}
\left.
\begin{aligned}
\langle\varphi\otimes k|\psi\otimes k\rangle
&=
\langle\varphi|\psi\rangle \langle k|k\rangle
\\
\langle\varphi\otimes\varphi|\psi\otimes\psi\rangle
&=
\langle\varphi|\psi\rangle \langle \varphi|\psi\rangle
\end{aligned}
\quad
\right\}=
\end{equation}
Da $\langle k|k\rangle=1$ ist, folgt
\[
\langle \varphi|\psi\rangle^2=
\langle \varphi|\psi\rangle
\qquad\Leftrightarrow\qquad
\langle \varphi|\psi\rangle ( \langle \varphi|\psi\rangle -1) = 0
\qquad\Leftrightarrow\qquad
\langle \varphi|\psi\rangle =\begin{cases}0\\1\end{cases}
\]
Die beiden Zust"ande $|\psi\rangle$ und $|\varphi\rangle$ waren aber
beliebig, k"onnen also auch beliebige komplexe Zahlen als Skalarprodukt
haben.
Dieser Widerspruch zeigt, dass es eine solche Matrix $U$ gar nicht
geben kann.
\end{proof}

Qubits lassen sich also grunds"atzlich nicht kopieren.
Dies wird in der Quanten-Kryptographie zur Absicherung des
Schl"usselaustausches verwendet. 
Da der Quantenzustand, den ein Teilnehmer empf"angt, nicht kopiert
werden kann, ist es unm"oglich, den Schl"usselaustausch abzuh"oren,
ohne dass dies bemerkt wird.

\subsection{Quanten-Turing-Maschinen}
Eine Quanten-Berechnung stellt also zuerst einen Zustand in einem
Qubit-Register her.
Dann operiert darauf eine Quantenoperation, also ein unit"arer Operator $U$.
Zum Schluss finden Messungen an den einzelnen Qubits statt.
Das Resultat der Messung verr"at etwas "uber das Resultat der Berechnung.

Um einen Quantencomputer zu realisieren muss man also eine Methode
entwickeln, einen beliebigen Operator $U$ herzustellen.
In einem klassischen Computer geschieht das durch Verschaltung der
Gatter: jede beliebige logische Formal kann aus den elementaren
Gattern hergestellt werden.
F"ur Quantencomputer geht das nicht mehr exakt, das ist aber auch nicht
n"otig, wir sind ohnehin nur auf eine Approximation aus.

\begin{satz}
Jede unit"are $n\times n$-Matrix $U$ mit $n\ge 3$ kann beliebig genau
approximiert werden durch ein Produkt $U_l\dots U_1$ von unit"aren
Matrizen, die jeweils Hadamard-Gatter, Tofoli-Gatter oder Phasen-Shifts
sind.
\cite[Theorem 10.12]{skript:arorabarak}
\end{satz}

\subsection{Quanten-Algorithmen}

\section*{"Ubungsaufgaben}
\rhead{"Ubungsaufgaben}
\begin{uebungsaufgaben}
\item
\input uebungsaufgaben/04001.tex
\end{uebungsaufgaben}

\chapter{Klassische Mechanik\label{chapter:mechanik}}
\lhead{Klassische Mechanik}
\rhead{}
\section{Motivation}
\rhead{Motivation}
W"ahrend Galileo Galileis Beschreibung von Bewegungen im Wesentlichen
eine kinematische war, haben Isaac
Newtons Gesetze erstmals einen Zusammenhang zwischen
der Bewegung und den Kr"aften hergestellt, also von Ursache und
Wirkung. Sein erstes Gesetz besagt zum Beispiel
\[
F=ma\qquad\text{oder}\qquad
F(x)=m\frac{d^2x}{dt^2}.
\]
Mit geeigneten Anfangsbedingungen l"asst sich daraus jede Bewegung 
vorhersagen.

Diese Formulierung ist jedoch nicht allgemein genug, insbesondere f"ur die
Zwecke der Quantenmechanik. Wenn Teilchen auch Wellen sind, wie soll man
sich dann die Kraftwirkung einer Welle auf eine andere Welle vorstellen?

Die Antwort ist die Hamiltonsche Formulierung der Mechanik, die von der
Energie als der alles bestimmenden Gr"osse ausgeht.
Mit Hilfe eines Standardformalismus lassen sich daraus die Newtonschen
Bewegungsgleichungen wieder gewinnen.
Viel wichtiger ist aber, dass sich daraus auch ein Methode gewinnen
l"asst, besonders geeignete Koordinaten zur Beschreibung eines
physikalischen Systems zu finden.
Diese Hamilton-Jacobi-Theorie genannte Methode ist die Basis eines
allgemein anwendbaren Quantisierungs-Verfahrens.

\section{Lagrange-Mechanik}
\rhead{Lagrange-Mechanik}
Das Fermatsche Prinzip besagt, dass Licht immer den Weg k"urzester Zeit
nimmt. Mit diesem Prinzip konnte Fermat sowohl das Reflexionsgesetz als
auch das Brechungsgesetz erkl"aren.
Im Lichte unserer beabsichtigen Vereinheitlichung zwischen Lichtwellen
und Teilchen sollte f"ur die Teilchenwelt ein entsprechendes Gesetz
gelten. Tats"achlich hat Pierre-Louis Maupertuis ein solches Prinzip
gefunden: ein mechanisches System entwickelt sich immer so, dass die
{\em Wirkung} minimal ist. Das Prinzip wurde von Euler, Lagrange 
und schliesslich Hamilton verallgemeinert.
\subsection{Wirkung}
Ein mechanisches System, zum Beispiel in sich bewegender Massepunkt,
wird durch eine Anzahl von Koordinaten beschrieben, die wir mit
$q_i,1\le i\le n$ bezeichnen. Die Koordinaten werden im allgemeinen 
von der Zeit abh"angen, man k"onnte also auch $q_i(t)$ schreiben.
Die Entwicklung des Systems wird durch Differentialgleichungen 
f"ur diese Funktionen $q_i(t)$ und Anfangsbedingungen beschrieben.
Zwischen den Zeitpunkten $t_0$ und $t_1$ entwickelt sich das
System
von einem Zustand $q(t_0)$ in einen Zustand $q(t_1)$.

Die Bewegung eines Teilchens wird im wesentlichen durch seine 
kinetische und potentielle Energie bestimmt. Die kinetische Energie
ist ein quadratischer Ausdruck in den Geschwindigkeiten:
\[
T=\frac12m_i\dot q_i^2.
\]
die potentielle Energie ist eine Funktion, die nur von der
Position des Teilchens abh"angt:
\[
V(q_1,\dots,q_n)=V(q).
\]
Die Lagrange-Funktion eines Systems ist die Differenz:
\[
L(\dot q, q) = T(\dot q) - V(q)
\]
Selbstverst"andlich ist es auch m"oglich, dass die kinetische oder
die potentielle Energie
zus"atzlich von der Zeit abh"angt, zum Beispiel "uber eine
zeitlich ver"anderliche Masse.

Die Wirkung oder Aktion einer Zeitentwicklung zwischen den Zust"anden
$q(t_0)$ und $q(t_1)$ ist die Gr"osse
\begin{equation}
W=\int_{t_0}^{t_1} L(t, q(t), \dot q(t))\,dt.
\label{skript:wirkung}
\end{equation}
Die Wirkung hangt also von der konkreten Wahl des Pfades $q(t)$ ab,
um diese Abh"angigkeit auszudr"ucken, wird manchmal auch $W[q(t)]$
geschrieben.

Das Prinzip minimaler Wirkung besagt, ein mechanisches System unter
allen m"oglichen Pfaden, die es vom Zustand $q(t_0)$ in den Zustand
$q(t_1)$ "uberf"uhren, immer denjenigen w"ahlt, f"ur die die
Wirkung minimal wird.

\subsection{Euler-Gleichungen}
Die Minimierung der Wirkung (\ref{skript:wirkung}) ist ein sogenanntes
Variationsproblem: gesucht ist ein Weg, welcher ein bestimmtes
Integral minimiert. Es ist verwandt mit der Aufgabe, den k"urzesten
Weg zu finden, der ein Gebiet mit einem bestimmten Fl"acheninhalt
umschliesst. Johann Bernoulli hat 1696 mit seinem Brachistochronen-Problem
die mathematische Welt mit einem ersten ber"uhmten Problem dieser Art
konfronriert, mit dem sich auch einige Mathematiker befasst haben.
Systematisch mit dieser Art von Problem hat sich aber erst Leonhard
Euler auseinandergesetzt, der auch einen allgemeinen L"osungsweg angegeben hat.

Gesucht wird also eine Kurve $q(t)$, welche $W[q(t)]$ minimal macht.
Jede andere Kurve, die sich nur wenig von $q(t)$ unterscheidet, sollte
daher eine gr"ossere Wirkung haben.
\begin{figure}
\centering
\includegraphics{graphics/lagrange-1.pdf}
\caption{Kurve und Nachbarkurven f"ur die Herleitung der Eulerschen
Differentialgleichung
\label{skript:nachbarkurven}}
\end{figure}
Eine solche Nachbarkurve k"onnen wir als
\[
q(t) + \varepsilon \eta(t)
\]
ansetzen (Abbildung~\ref{skript:nachbarkurven}),
wobei $\eta(t)$ eine beliebig w"ahlbare Funktion ist,
die f"ur $t_0$ und $t_1$ verschwindet, $\eta(t_0)=\eta(t_1)=0$.
Wenn $q(t)$ L"osung des Minimalproblems ist, muss
\begin{equation}
\varepsilon\mapsto W[q(t)+\varepsilon\eta(t)]
\label{skript:variation-ansatz}
\end{equation}
ein Minimum f"ur $\varepsilon=0$ haben, oder die Ableitung
dieser Funktion nach $\varepsilon$ muss $0$ sein.

Setzten wir den Ansatz (\ref{skript:variation-ansatz}) in die Wirkung ein
und leiten wir nach $\varepsilon$ ab, erhalten wir
\begin{align*}
\frac{d}{d\varepsilon}W[q(t)+\varepsilon\eta(t)]
&=
\frac{d}{d\varepsilon}\int_{t_0}^{t_1}L(t, q(t)+\varepsilon\eta(t),
\dot q(t)+\varepsilon\dot\eta(t))\,dt
\\
&=\int_{t_0}^{t_1}
\frac{\partial L}{\partial q}(t, q(t)+\varepsilon\eta(t), \dot q(t)+\varepsilon\dot\eta(t))\eta(t)
+
\frac{\partial L}{\partial \dot q}(t, q(t)+\varepsilon\eta(t), \dot q(t)+\varepsilon\dot\eta(t))\dot \eta(t)\,dt.
\end{align*}
Da $q$ eigentlich ein Vektor ist, ist $\partial L/\partial q$ zu lesen
als ein Vektor bestehend aus allen Ableitungen von $L$ nach den verschiedenen
Koordinaten $q_i$, und analog f"ur $\partial L/\partial \dot q$.
F"ur $\varepsilon=0$ soll dieser Ausdruck verschwinden:
\begin{equation}
0
=\int_{t_0}^{t_1}
\frac{\partial L}{\partial q}(t, q(t), \dot q(t))\eta(t)
+
\frac{\partial L}{\partial \dot q}(t, q(t), \dot q(t))\dot \eta(t)\,dt.
\label{skript:erstevariationsgleichung}
\end{equation}
Zur Abk"urzung schreiben wir im Folgenden
\begin{align*}
\frac{\partial L}{\partial q}(t, q(t), \dot q(t))
&=
\frac{\partial L}{\partial q}
&
\frac{\partial L}{\partial \dot q}(t, q(t), \dot q(t))
&=
\frac{\partial L}{\partial \dot q}.
\end{align*}
Der zweite Term in (\ref{skript:erstevariationsgleichung}) kann mit 
partieller Integration umgeformt werden:
\begin{align*}
0
&=\int_{t_0}^{t_1}
\frac{\partial L}{\partial q}(t, q(t), \dot q(t))\eta(t)
+
\frac{\partial L}{\partial \dot q}(t, q(t), \dot q(t))\dot \eta(t)\,dt
\\
&=
\int_{t_0}^{t_1}\frac{\partial L}{\partial q}\eta(t)\,dt
+\left[
\frac{\partial L}{\partial \dot q} \eta(t)
\right]_{t_0}^{t_1}
-\int_{t_0}^{t_1}\frac{d}{dt}\frac{\partial L}{\partial \dot q}\eta(t)\,dt
\\
&=\int_{t_0}^{t_1}\left(\frac{\partial L}{\partial q}
-\frac{d}{dt}\frac{\partial L}{\partial \dot q}\right) \eta(t)\,dt.
\end{align*}
Diese letzte Gleichung ist nur dann f"ur jede beliebige Funktion $\eta(t)$
erf"ullbar, wenn der Klammerausdruck verschwindet. So erhalten wir die
{\em Eulerschen Gleichungen} f"ur das Variationsproblem.

\begin{satz}
Die Wirkung
\[
W[q(t)] =\int_{t_0}^{t_1} L(t, q(t), \dot q(t))\,dt
\]
wird minimiert von einer Funktion $q(t)$, welche den 
{\em Eulerschen Differentialgleichungen}
\begin{equation}
\frac{d}{dt}\frac{\partial L}{\partial \dot q}-\frac{\partial L}{\partial q}=0
\qquad\Leftrightarrow\qquad
\frac{d}{dt}\frac{\partial L}{\partial \dot q_i}-\frac{\partial L}{\partial q_i}=0\quad\forall i
\label{skript:euler-dgl}
\end{equation}
gen"ugt.
Die Eulerschen Differentialgleichungen sind gew"ohnliche
Differentialgleichungen zweiter Ordnung f"ur die Funktionen $q_i(t)$.
\end{satz}

\subsection{Bewegungsgleichungen}
\index{Maupertuis}
\index{Wirkung!Prinzip der kleinsten}
Nach Maupertuis soll das Prinzip der kleinsten Wirkung eine Bewegung
entsprechen den Newtonschen Gesetzen ergeben.
Wir wenden daher die Eulergleichungen auf das Problem eines Teilchens der
Masse $m$ in einem Potential $V(q)$ an. Die Lagrange-Funktion ist
\[
L(t, q, \dot q)=\frac12m\dot q^2-V(q).
\]
Die Ableitungen nach $q$ und $\dot q$ sind:
\begin{align*}
\frac{\partial L}{\partial \dot q}&=m\dot q=p\\
\frac{\partial L}{\partial q}&=-\frac{\partial V}{\partial q}=-\operatorname{grad}V(q).
\end{align*}
Der Ausdruck $m\dot q$ ist aus der Newtonschen Mechanik bekannt als
der Impuls eines Teilchens. Die Eulersche Gleichung wird daher zu
der Bedingung:
\[
\frac{d}{dt}\frac{\partial L}{\partial \dot q}=\frac{\partial L}{\partial q}
\qquad\Rightarrow\qquad
\frac{d}{dt}p=-\operatorname{grad}V(q).
\]
Der negative Gradient des Potentials ist die Kraft $F=-\operatorname{grad}V(q)$,
die auf das Teilchen an der Stellen $q$ wirkt.
Wir erhalten also die Newtonschen Bewegungsgleichungen.
Falls die Masse konstant ist, kann man die Ableitung nach der Zeit
noch vereinfachen, und erh"alt die wohlbekannte Form des Newtonschen
Gesetzes
\[
F= m\ddot q.
\]

Man beachte, dass diese Herleitung auch dann noch gilt, wenn die Masse von der
Zeit abh"angt (zum Beispiel bei einer Rakete).

\subsection{Fermat-Prinzip}
Das Fermat-Prinzip besagt, dass das Licht immer den Weg mit k"urzester
Laufzeit zwischen zwei Punkten w"ahlt.
Wir beschreiben den Weg, den das Licht von $x_0$ zu $x_1$ nimmt,
als eine Kurve $x(s)$, wobei $s\in[s_0,s_1]$ der Parameter entlang
der Kurve ist.
Die Lichtgeschwindigkeit h"angt vom Brechungsindex des Mediums ab.
Sei $n(x)$ der Brechungsindex an der Stelle $x$, dann ist 
$c/n(x)$ die Lichtgeschwindigkeit an der Stelle $x$.
Zwei Punkte auf dem Strahl, deren Parameter sich um $\Delta s$
unterscheiden, haben die  Entfernung $|x'(s)|\Delta s$.
Das Licht braucht daher f"ur die ganze Kurve die Zeit
\[
T=\int_{s_0}^{s_1} |x'(s)| n(x(s))\frac1c\,ds.
\]
Der Faktor $1/c$ hat keinen Einfluss auf das Minimum, wir k"onnen ihn
weglassen.
Wir k"onnen jetzt Euler-Gleichungen f"ur die Lagrange-Funktion
$L(x', x)= |x'|\,n(x)$ 
ableiten:
\begin{align*}
\frac{\partial L}{\partial x_i}
&=
|x'|\frac{\partial n}{\partial x_i}
\\
\frac{\partial L}{\partial x'_i}
&=
n(x)\frac{\partial |x'|}{\partial x_i'}
=
n(x)\frac{\partial}{\partial x'_i}\sqrt{\sum_{k=1}^3x_k'^2}
=
n(x)\frac{x_i'}{|x'|}
\end{align*}
Die Wahl der Parametrisierung der Kurve darf keinen Einfluss auf den
gew"ahlten Weg haben, also k"onnen wir eine Parametrisierung w"ahlen,
f"ur die $|x'(s)|=n(x)$ ist.
Dadurch vereinfachen sich die Ausdr"ucke f"ur die partiellen
Ableitungen zu
\begin{align*}
\frac{\partial L}{\partial x_i}
&=
n(x)\frac{\partial n}{\partial x_i}
\\
\frac{\partial L}{\partial x_i'}
&=x_i'
\end{align*}
Damit wird die Bewegungsgleichung
\begin{align*}
\frac{d}{ds}\frac{\partial L}{x_i'}-\frac{\partial L}{\partial x_i}
&=
x_i''-n\frac{\partial n}{\partial x_i}=0
\\
x_i''
&=
n\frac{\partial n}{\partial x_i}=\frac12 \frac{\partial n(x)^2}{\partial x_i}.
\end{align*}
Die Kurven sehen aus wie die Bewegung eines Teilchens in einem
``Potential'' $n(x)^2$.

Das Fermat-Prinzip beschreibt die geometrische Optik, also die Ausbreitung
von Licht mit so kurzer Wellenl"ange, dass sie gegen"uber den
Abmessungen des Problems vernachl"assigbar ist.
Dies ist genau die gleiche Art von Grenz"ubergang, die wir f"ur
die Wellen der Quantenmechanik machen m"ochten.
Das Fermat-Prinzip als Grenzwert einer Wellenausbreitung f"ur
unendlich kurze Wellenl"ange f"uhrt auf eine Bewegungsgleichung,
die aussieht wie die Bewegung eines Teilchens in einem Potential.

\section{Hamiltonsche Mechanik}
\rhead{Hamiltonsche Mechanik}
Im Fermat-Prinzip haben wir einen Zusammenhang zwischen dem Formalismus
der Mechanik und der geometrischen Optik kennengelernt, der mindestens
andeutet, dass die Mechanik nur der Grenzfall f"ur sehr kurze Wellenl"ange
einer ``Wellenmechanik'' sein k"onnte.
Die Lagrange-Mechanik ist aber nicht allgemeine genug, um diesen
Schritt zu erm"oglichen.
Die noch etwas allgemeinere Formulierung der Mechanik durch Hamilton
schafft dies.

Die Lagrange-Formulierung der Mechanik f"uhrt auf die Eulergleichungen,
die Differentialgleichungen zweiter Ordnung f"ur die Funktionen $q_i(t)$ 
sind.
Zwar l"asst sich jede Differentialgleichung zweiter Ordnung einfach
dadurch als Differentialgleichung erster Ordnung schreiben, indem
zu den Variablen $q_i$ die zus"atzlichen Variablen $v_i=\dot q_i$ und die
Gleichungen $\dot q_i=v_i$ hinzugef"ugt werden, doch ist diese Technik
nicht ganz befriedigend, weil die $v_i$ physikalisch nicht so ``gute''
Variablen sind.
Die erweiterte Formulierung der Mechanik soll also durch
Differentialgleichungen erster Ordnung mit zus"atzlichen
Variablen $p_i$ erfolgen, die eine ``spannende'' physikalische
Bedeutung haben.

\subsection{Hamilton-Funktion}
Die Hamilton-Funktion $H(p,q)$ ist die Energie eines physikalischen
Systems. Darin sind $q$ die allgemeinen Koordinaten, also
zum Beispiel die Raumkoordinaten $x$, $y$ und $z$, oder Winkel, mit
denen man die r"aumliche Ausrichtung eines Systems beschreibt.
Die Variablen $p$ sind die zugeh"origen Impulse, f"ur Ortskoordinaten
sind dies die gew"ohnlichen Impulskomponenten, f"ur die Drehwinkel ist
es der Drehimpuls.

Die Energie $H$ setzt sich zusammen aus der kinetischen Energie $T$ und der
potentiellen Energie $V$. F"ur ein kr"aftefreies Teilchen der Masse $m$
ist die Energie 
\[
H=\frac12mv^2=\frac1{2m}p^2.
\]
Befindet sich das Teilchen dagegen in einem Potential $V(q)$, dann 
ist die zugeh"orige Hamilton-Funktion
\begin{equation}
H=T+V=\frac1{2m}p^2+V(q).
\label{skript:hamilton-potential}
\end{equation}
Ein Elektron im elektrischen Feld eines Protons, welches im Nullpunkt
des $q$-Koordinatensystems sitzt, hat daher die Hamilton-Funktion
\[
H=\frac1{2m}p^2+\frac{e^2}{4\pi\varepsilon_0|q|}.
\]

Eine noch zu "uberwindende Schwierigkeit dieses Formalismus wird hier
bereits erkennbar: Kr"afte, die keine Arbeit leisten, k"onnen auch
keinen Beitrag zur Energie leisten.
Ein Beispiel ist
die Lorentzkraft eines Magnetfelds, welche die Bahn eines Elektrons 
kr"ummt, aber immer senkrecht auf der Bahn steht und daher keine Arbeit
leistet.

\subsection{Von der Lagrange-Funktion zur Hamilton-Funktion}
Die Frage, welche Impulse zu verwenden sind, wurde noch nicht beantwortet
worden.
Die Euler-Gleichung sagen, dass die Gr"osse, f"ur die wir die Zeitableitung
berechnen k"onnen, die Ableitung $\partial L/\partial\dot q$ ist.
Die Euler-Gleichung liefert also eine Bewegungsgleichung im Stile des
Newtonschen Gesetzes, wenn man setzt
\[
p=\frac{\partial L}{\partial \dot q}.
\]
Die kinetische Energie k"onnten wir auch als
\[
T=\sum_i \frac12m\dot q_i^2=\sum_i \frac12 p_i\dot q_i
\]
schreiben, also muss die gesamte Energie sein
\[
H(p,q)=T+V = 2T - (T - V)
=
\sum_k p_k\dot q_k - L(\dot q, q).
\]
Dieser Ausdruck h"angt nicht mehr von $\dot q_i$ ab, denn wir
k"onnen ausrechnen
\[
\frac{\partial H}{\partial\dot q_i}
=
\frac{\partial}{\partial \dot q_i}\sum_k p_k\dot q_k
-
\frac{\partial L}{\partial \dot q_i}
=
p_i-p_i=0.
\]
Es ist also sicher m"oglich, die Hamilton-Funktion als Funktion von $p$
und $q$ zu schreiben.

\subsection{Bewegungsgleichungen}
Die Newtonschen Bewegungsgleichungen lassen sich aus der Hamilton-Funktion
gewinnen. Um dies zu verstehen, berechnen wir die partiellen Ableitungen
von $H$ nach Koordinaten $q$ und Impulsen $p$:
\begin{align*}
\frac{\partial H}{\partial p}&=\frac{1}{m}p=v=\frac{dq}{dt} \\
\frac{\partial H}{\partial q}&=\frac{\partial V}{\partial q}
\end{align*}
Die erste Gleichung ist nichts anders als die Aussage, dass die
Geschwindigkeit die Zeitableitung der Ortskoordinaten ist, sie
stellt den Zusammenhang zwischen $p$ und $q$ her.
Die Ableitung des Potentials auf der rechten Seite der zweiten
Gleichung hat die Bedeutung einer Kraft.
Die Newtonschen Bewegungsgleichungen sagen, dass die "Anderung des
Impulses durch die Kr"afte verursacht wird, also durch die Ableitungen
des Potentials:
\[
\frac{dp}{dt}=-\frac{\partial V}{\partial q}
\]
Damit sind die Bewegungsgleichungen jetzt:
\begin{align}
\frac{dq}{dt}&= \frac{\partial H}{\partial p},\label{skript:hamilton-v}\\
\frac{dp}{dt}&=-\frac{\partial H}{\partial q}.\label{skript:hamilton-newton}
\end{align}

\begin{beispiel} Harmonischer Oszillator. Ein harmonischer Oszillator
mit Masse $m$ und Federkonstante $K$ hat die potentielle Energie $\frac12Kx^2$,
die Hamilton-Funktion ist daher
\[
H=\frac1{2m}p^2+\frac12Kx^2.
\]
Die Bewegungsgleichungen nach dem Hamilton-Formalismus sind:
\begin{align*}
\frac{dx}{dt}&=\frac{\partial H}{\partial p}=\frac{p}{m}&&\Rightarrow&\dot x&=v\\
\frac{dp}{dt}&=\frac{\partial H}{\partial x}=-Kx&&\Rightarrow&ma&=-Kx
\end{align*}
Die erste Gleichung besagt, dass die Geschwindigkeit die Ableitung
der Ortskoordinate ist.
Die zweite Gleichung ist das erste Newtonsche Gesetz, denn die
r"ucktreibende Kraft einer um die Koordinate $x$ ausgelenkten Feder
mit Federkonstanten $K$ ist $-Kx$.
Insbesondere reproduziert der Hamilton-Formalismus die bekannten
Newtonschen Bewegungsgleichungen.
\end{beispiel}

Ein besonderer Fall liegt vor, wenn die Hamilton-Funktion nicht von
der Zeit abh"angt. Dann ist
\[
\frac{\partial H}{\partial t}=0,
\]
die Energie ist erhalten. Diese Situation tritt in abgeschlossenen
Systemen immer auf, insbesondere waren alle bisherigen Beispiel
von dieser Art.

\subsection{Poisson-Klammern}
\index{Poisson-Klammer}
Die Hamiltonsche Mechanik kann mit Hilfe der Poisson-Klammer besonders
elegant formuliert werden.

\begin{definition}
Seien $F(q,p)$ und $G(q,p)$ Funktionen in den Koordinaten und Impulsen. 
Dann ist die {\em Poisson-Klammer} die Funktion
\[
(F,G)
=
\sum_{k=1}^n
\biggl(
\frac{\partial F}{\partial q_k}\frac{\partial G}{\partial p_k}
-
\frac{\partial G}{\partial q_k}\frac{\partial F}{\partial p_k}
\biggr).
\]
\end{definition}

F"ur das Rechnen mit Poisson-Klammern sind die folgenden Rechenregeln
n"utzlich.
\begin{satz}
F"ur drei Funktionen $F$, $G$ und $H$ gilt
\begin{gather*}
(F,G)=-(G,F),
\\
(F,GH)
=
(F,G)H+G(F,H),
\\
(F,(G,H))
+
(G,(H,F))
+
(H,(F,G))
=0.
\end{gather*}
Die dritte Gleichung heisst auch die Jacobi-Identit"at.
\end{satz}

\begin{proof}[Beweis]
Die erste Identit"at folgt sofort aus der Definition.
Die zweite Gleichung kann man durch Nachrechnen wie folgt erhalten:
\begin{align*}
(F,GH)
&=
\sum_{k=1}^3 \biggl(
\frac{\partial F}{\partial q_k}\frac{\partial GH}{\partial p_k}
-
\frac{\partial GH}{\partial q_k}\frac{\partial F}{\partial p_k}
\biggr)
\\
&=
\sum_{k=1}^3 \biggl(
\frac{\partial F}{\partial q_k}\frac{\partial G}{\partial p_k}H
+
G\frac{\partial F}{\partial q_k}\frac{\partial H}{\partial p_k}
-
\frac{\partial G}{\partial q_k}\frac{\partial F}{\partial p_k}H
-
G\frac{\partial H}{\partial q_k}\frac{\partial F}{\partial p_k}
\biggr)
=(F,G)H + G(F,H).
\end{align*}
Auch von der dritten Gleichung kann man sich durch Nachrechnen
"uberzeugen.
\end{proof}

Als Beispiel berechnen wir die Poisson-Klammern einer Funktion $F$
mit den Koordinaten und Impulsen
\begin{align*}
(q_j,G)
&=
\sum_{k=1}^n\biggl(
\frac{\partial q_j}{\partial q_k}\frac{\partial G}{\partial p_k}
-
\frac{\partial G}{\partial q_k}\frac{\partial q_j}{\partial p_k}
\biggr)
=
\sum_{k=1}^n\delta_{jk}\frac{\partial G}{\partial p_k}
=
\frac{\partial G}{\partial p_j}
\\
(p_j,G)
&=
\sum_{k=1}^n\biggl(
\frac{\partial p_j}{\partial q_k}\frac{\partial G}{\partial p_k}
-
\frac{\partial G}{\partial q_k}\frac{\partial p_j}{\partial p_k}
\biggr)
=
-\sum_{k=1}^n
\frac{\partial G}{\partial q_k}\delta_{jk}
=
-\frac{\partial G}{\partial q_j}.
\end{align*}
Damit k"onnen wir auch die Poisson-Klammer der Koordinaten und Impulse
ausrechnen:
\begin{align*}
(q_j,q_l)
&=
\frac{\partial q_l}{\partial p_j}
=
0
\\
(p_j,p_l)
&=
\frac{\partial p_l}{\partial q_j}
=
0
\\
(q_j,p_l)
&=
\frac{\partial p_l}{\partial p_j}
=\delta_{jl}.
\end{align*}
Die Poisson-Klammern der Koordinaten und Impulse mit der Hamilton-Funktion
erlauben, die Bewegungsgleichungen
\begin{align*}
(q_j,H)
&=
\frac{\partial H}{\partial p_j} = \frac{dq_j}{dt}
\\
(p_j,H)
&=
-\frac{\partial H}{\partial q_j} = \frac{dp_j}{dt}
\end{align*}
mit Poisson-Klammern auszudr"ucken.

\subsection{Poisson-Klammern f"ur den Drehimpuls}
Der Drehimpuls der Vektor mit den Komponenten
\[
\vec L=\vec r\times \vec p
\qquad\Rightarrow\qquad
\left\{\quad
\begin{aligned}
L_1&=x_2p_3-x_3p_2\\
L_2&=x_3p_1-x_1p_3\\
L_3&=x_1p_2-x_2p_1
\end{aligned}
\right.
\]
Die Poisson-Klammern sind
\begin{align*}
(x_1,L_1)
&=
0
&
(p_1,L_1)
&=
0
\\
(x_2,L_1)
&=
\frac{\partial L_1}{\partial p_2}
=
-x_3
&
(p_2,L_1)
&=
-\frac{\partial L_1}{\partial x_2}
=
-p_3
\\
(x_3,L_1)
&=
\frac{\partial L_1}{\partial p_3}
=
x_2
&
(p_3,L_1)
&=
-\frac{\partial L_1}{\partial x_3}
=
p_2
\end{align*}

\section{Wellen und Teilchen}
\rhead{Welle und Teilchen}
Wir zeigen jetzt, dass die Hamilton-Jacobi-Gleichung tats"achlich der
Grenzfall unendlich kleiner Wellenl"ange f"ur eine ``Wellenmechanik'' ist.
Von de Broglie wissen wir den Zusammenhang zwischen dem Impuls $p$
eines Teilchens und der zugeh"origen Wellenl"ange  $\lambda=h/p$.
Nach de Broglie entspricht einem Teilchen mit Impuls $p$ also eine Welle
der Form
\begin{equation}
\psi(x)
=
e^{2\pi i\frac{x}{\lambda}}
=
e^{\frac{i}{\hbar}xp}.
\label{skript:debrogliewelle}
\end{equation}
Nat"urlich bewegt sich diese Welle auch noch, wir schreiben die
Zeitabh"angigkeit als
\[
\psi(x,t)
=
e^{\frac{i}{\hbar}(xp-\omega t)}.
\]


Etwas allgemeiner suchen wir jetzt eine Welle, die zu einem Teilchen
passt, welches sich gem"ass einer Hamilton-Funktion $H(p,q)$ bewegt.
Wir setzen die Welle in der Form
\[
\psi(q,t) = e^{\frac{i}{\hbar}S(q,t))}
\]
Die Funktion ist die Phase der Welle, die Fl"achen gleichen Wertes von
$S$ beschreiben die Wellenfronten.
Die Konstante $\hbar$ in diesem Ansatz steuert die Wellenl"ange, wenn
wir $\hbar$ gegen Null gehen lassen, wird die Wellenl"ange unendlich kurz.

Die Welle soll sich f"ur unendlich kleine Wellenl"ange wie ein Teilchen
bewegen, welches von der Hamilton-Funktion gesteuert wird.
Den Impuls des Teilchens k"onnen wir aus der
de Broglie-Welle~(\label{skript:debrogliewelle}) durch ableiten finden,
\[
\frac{\partial}{\partial x}
e^{\frac{i}{\hbar}xp}
=
\frac{i}{\hbar}p
e^{\frac{i}{\hbar}xp}
\qquad
\Rightarrow
\qquad
p
e^{\frac{i}{\hbar}S}
=
\frac{\hbar}{i}
\frac{\partial S}{\partial q}
e^{\frac{i}{\hbar}S}
\]


\section*{"Ubungsaufgaben}
\begin{uebungsaufgaben}
\item
\input uebungsaufgaben/05001.tex
\end{uebungsaufgaben}


\chapter{Quantisierung\label{chapter:quantisierung}}
\lhead{Quantisierung}
\rhead{}

\section{Quantisierungsregeln}
% Operatoren in Ortsdarstellung
% Ebene Wellen und Fourier Transformation
\section{Schr"odingergleichung}
% Schrödingergleichung in Ortsdarstellung
\index{schrodingergleichung@Schr\"odingergleichung!in Ortsdarstellung}

\input potentialkasten.tex
\input potentialtopf.tex

\section{Wahrscheinlichkeitsstrom}
% Kontinuitätsgleichung für die Wahrscheinlichkeitsdichte <psi(x)|psi(x)>
Die Wahrscheinlichkeit ein Teilchen in einer Umgebung des
Punktes $x$ zu finden, ist $\varrho(x)=\langle \psi(x)|\psi(x)\rangle$.
Die Zeitentwicklung f"uhrt dazu, dass $\varrho$ auch von der
Zeit abh"angt.
Dies gesamte Wahrscheinlichkeit muss nat"urlichh erhalten bleiben,
das Teilchen kann ja nicht einfach verschwinden.

Wenn sich mit der Zeitentwicklung das Teilchen woanders hin bewegt,
dann ist damit ein Stromdichtevektor verbunden, der angibt, 
wieviel Wahrscheinlichkeit durch eine Fl"ache fliesst.
Das Ziel dieses Abschnittes ist, eine solche Stromdichte zu
definieren.

\subsection{Kontinuit"atsgleichung}
\index{Kontinuit\"atsgleichung}
Wir stellen uns ein Medium mit Dichte $\varrho(x,t)$ vor.
Das Medium hat im Punkt $x$ die Str"omungsgeschwindigkeit $v(x)$.
In einem Zeitinterval $\Delta t$ nimmt die Masse des Mediums
in einem Interval der L"ange $\Delta x$ um den Betrag
\[
\Delta x\frac{\partial\varrho}{\partial t}\Delta t
\]
zu.
Dies Zunahme muss dadurch erfolgen, dass durch die Endpunkte
des $\Delta x$-Intervals mehr Material zu- als abfliesst.
Der Zufluss am linken Ende des Intervals ist
$
\varrho(x) v(x),
$
der Abfluss am rechten Ende ist $\varrho(x+\Delta x)v(x+\Delta x)$.
Die Bilanz ist
\[
\Delta x\frac{\partial\varrho}{\partial t}\Delta t
=
\Delta t(
\varrho(x) v(x)
-
\varrho(x+\Delta x) v(x+\Delta x)
)
\]
Wir teilen durch $\Delta x\,\Delta y$ und lassen $\Delta x$ gegen 0 gehen:
\begin{equation}
\frac{\partial\varrho}{\partial t}
+\frac{\partial}{\partial x}(\varrho(x)v(x))
=0.
\label{skript:kontinuitaetsgleichung1d}
\end{equation}
Die Gr"osse $j(x)=\varrho(x)v(x)$ beschreibt den Materialstrom.
Die Gleichung (\ref{skript:kontinuitaetsgleichung1d}) heisst die
Kontinuit"atsgleichung.
Sie dr"uckt aus, dass im Verlauf der Str"omung kein Material verlorgen
gehen kann.

In drei Dimensionen kann man ebenfalls ein Kontinuit"atsgleichung
f"ur die Dichte $\varrho(x)$ und den Strom $\vec\j(x)=\varrho(x) \vec v(x)$
definieren, die die dreidimensionale 
Kontinuit"atsgleichung
\[
\frac{\partial\varrho}{\partial x}
+
\frac{\partial j_1}{\partial x_1}
+
\frac{\partial j_2}{\partial x_2}
+
\frac{\partial j_3}{\partial x_3}
=
\frac{\partial\varrho}{\partial t}+\operatorname{div}\vec\j
=0
\]
erf"ullt.

\subsection{Wahrscheinlichkeitsstrom}
Wir suche jetzt einen Wahrscheinlichkeitsstrom, der zusammen mit
der Wahrscheinlichkeitsdichte $|\psi(x)|^2$ eine Kontinuit"atsgleichung
erf"ullt.
\begin{align*}
\frac{\partial\varrho(x,t)}{\partial t}
&=
\frac{\partial}{\partial t} \varrho(x)
=
\frac{\partial\varrho(x)}{\partial t}
=
\frac{\partial\overline{\psi(x,t)}}{\partial t}\psi(x,t)
+
\overline{\psi}(x,t)\frac{\partial\psi(x,t)}{\partial t}
\end{align*}
Im letzten Term k"onnen wir die Zeitableitungen durch die
Schr"odingergleichung ersetzen:
\begin{align*}
\frac{\partial\varrho(x,t)}{\partial t}
&=
\frac{\partial\overline{\psi(x,t)}}{\partial t}\psi(x,t)
+
\overline{\psi}(x,t)\frac{\partial\psi(x,t)}{\partial t}
\\
&=
\overline{ \frac{i}{\hbar}H \psi(x,t) }\psi(x,t)
+
\overline{\psi}(x,t)\frac{i}{\hbar} H \psi(x,t)
\\
&=
-
\frac{\hbar^2}{2m}\overline{\frac{\partial^2\psi(x,t)}{\partial x^2}}\psi(x,t)
+
\frac{\hbar^2}{2m}\frac{\partial^2\psi(x,t)}{\partial x^2}\overline{\psi(x,t)}
+
\varphi(x,t)|\psi(x,t)|^2
\end{align*}
Die ersten zwei Terme k"onnen wir auch als Ableitung
der Funktion
\[
j(x,t)=
\frac{\hbar^2}{2m}
\biggl(
-\frac{\partial\overline{\psi}(x,t)}{\partial x}\psi(x,t)
+\overline{\psi}(x,t)\frac{\partial\psi(x,t)}{\partial x}
\biggr)
\]
erhalten, denn
\begin{align*}
\frac{\partial j(x,t)}{\partial x}
&=
\frac{\hbar^2}{2m}
\frac{\partial}{\partial x}
\biggl(
-\frac{\partial\overline{\psi}(x,t)}{\partial x}\psi(x,t)
+\overline{\psi}(x,t)\frac{\partial\psi(x,t)}{\partial x}
\biggr)
\\
&=
\frac{\hbar^2}{2m}
\frac{\partial}{\partial x}
\biggl(
-
\frac{\partial^2\overline{\psi}(x,t)}{\partial x^2}
\psi(x,t)
-
\frac{\partial\overline{\psi}(x,t)}{\partial x}
\frac{\partial\psi(x,t)}{\partial t}
+
\frac{\partial\overline{\psi}(x,t)}{\partial x}
\frac{\partial\psi(x,t)}{\partial x}
+
\overline{\psi}(x,t)
\frac{\partial^2\psi(x,t)}{\partial x^2}
\biggr)
\\
&=
\frac{\hbar^2}{2m}
\frac{\partial}{\partial x}
\biggl(
-
\frac{\partial^2\overline{\psi}(x,t)}{\partial x^2}
\psi(x,t)
+
\overline{\psi}(x,t)
\frac{\partial^2\psi(x,t)}{\partial x^2}
\biggr)
\end{align*}
Somit erf"ullt die oben definierte Funktion $j(x,t)$ die
Kontinuit"atsgleichung
\[
\frac{\partial\varrho(x,t)}{\partial t}
=
\frac{\partial j(x,t)}{\partial x} +\varphi(x,t)\varrho(x,t).
\]






\chapter{Heisenbergsche Unsch"arferelation\label{chapter:heisenberg}}
\lhead{Heisenbergsche Unsch"arferelation}
\rhead{}

In der Quantenmechanik lassen sich die Observablen nicht mehr einfach
vertauschen, wie das in der klassischen Mechanik m"oglich war.
F"ur Operatoren gilt im Allgemeinen kein Kommutativgesetz.
In diesem Abschnitt wollen wir die Konsequenzen dieser Tatsache
untersuchen.
Wir k"onnen aber bereits jetzt feststellen, dass die Gleichung
$AB=BA$ f"ur zwei quantenmechanische Observable ``fast'' stimmen
muss, in dem Sinne, dass der Unterschied der beiden Seiten sehr
klein sein muss. 

\section{Vertauschungsrelationen und Eigenvektoren}
\rhead{Vertauschungsrelationen und Eigenvektoren}
%\subsection{Definitionen}
\index{Kommutator}%
\index{Antikommutator}%
Zu zwei Operatoren $A$ und $B$ kann man Kommutator und Antikommutator
bilden:
\[
\begin{aligned}
&\text{Kommutator:}&
[A,B]&=AB-BA
\\
&\text{Antikommutator:}&
\{A,B\}&=AB+BA
\end{aligned}
\]
Wenn $[A,B]=AB-BA=0$ folgt $AB=BA$,
der Kommutator gibt also an, ob die beiden Operatoren vertauschen (kommutieren).
Wenn $\{A,B\}=AB+BA=0$ folgt $AB=-BA$,
der Kommutator gibt also an, ob die beiden Operatoren antikommutieren.

\subsection{Kommutator und Eigenvektoren}
Falls $A$ and $B$ Observable sind, die nicht vertauschen, dann k"onnen
gemeinsame Eigenvektoren nicht beliebig sein.

\begin{hilfssatz}
\label{skript:kommutatorannihliertev}
Der Kommutator $[A,B]$ annihiliert gemeinsame Eigenvektoren von $A$ und $B$.
\end{hilfssatz}

\begin{proof}[Beweis]
Nehmen wir an, $|\psi\rangle$ sei ein gemeinsamer Eigenvektor
von $A$ mit Eigenwert $\alpha$ und $B$ mit Eigenwert $\beta$.
Dann wirkt der Kommutator auf $|\psi\rangle$ wie folgt
\[
[A,B]\,|\psi\rangle
=
(AB-BA)\,|\psi\rangle 
=
A\beta\,|\psi\rangle -B\alpha\,|\psi\rangle
=
\alpha\beta\,|\psi\rangle-\beta\alpha\,|\psi\rangle
=
0.
\]
Ein gemeinsamer Eigenvektor $|\psi\rangle$ wird also von $[A,B]$
zu $0$ gemacht.
\end{proof}

In vielen F"allen ist der Kommutator ein Vielfaches der identischen
Abbildung. F"ur diese F"alle l"asst sich eine etwas sch"arfere
Folgerung ableiten.

\begin{satz}
Wenn zwei Observable einen Kommutator haben, der ein Vielfaches
der identischen Abbildung ist, dann haben die Observable keine
gemeinsamen Eigenvektoren.
\end{satz}

\begin{beispiel}
F"ur die Observablen $X$ f"ur Ort und $P$ f"ur den Impuls eines Teilchens,
k"onnen wir den Kommutator in der Ortsdarstellung explizt ausrechnen:
\begin{align*}
[X,P]\psi(x)
&=
\biggl[
x,\frac{\hbar}{i}\frac{\partial}{\partial x}
\biggr]\psi(x)
=
x\frac{\hbar}{i}\frac{\partial\psi(x)}{\partial x}
-
\frac{\hbar}{i}\frac{\partial}{\partial x}\bigl(x\psi(x)\bigr)
\\
&=
x\frac{\hbar}{i}\frac{\partial\psi(x)}{\partial x}
-
\frac{\hbar}{i}\psi(x)
-
\frac{\hbar}{i}x\frac{\partial\psi(x)}{\partial x}
=
-\frac{\hbar}{i}\psi(x).
\end{align*}
Der Kommutator ist also
\[
[X,P]=-\frac{\hbar}{i}\operatorname{id},
\]
ein Vielfaches der identischen Abbildung.
Die identische Abbildung hat nat"urlich keine Vektoren, die von ihr
annihiliert werden.
Orts- und Impuls-Operatoren k"onnen also keine gemeinsamen Eigenvektoren
haben.
\end{beispiel}

\subsection{Eigenbasen von zwei Operatoren}
Umgekehrt stellt sich die Frage, unter welchen Voraussetzungen zwei
Observable eine Basis von gemeinsamen Eigenvektoren haben k"onnen.
Der n"achste Satz liefert ein Kriterium daf"ur.
\index{Eigenbasis}%

\begin{satz}
\label{skript:kommevbasis}
Wenn zwei selbstadjungierte Operatoren vertauschen, dann gibt es eine
gemeinsame Basis von Eigenvektoren.
\end{satz}

\begin{proof}[Beweis]
Wir f"uhren den Beweis nur f"ur einen endlichdimensionalen Zustandsraum.
Seien also $A$ und $B$ zwei selbstadjungierte Operatoren mit $[A,B]=0$.
Ist $v$ ein Eigenvektor von $A$ zum Eigenwert $a$, dann gilt
$ Av=av $.
F"ur den Vektor $Bv$ gilt dann
\[
A(Bv)=BAv=Bav=aBv,
\]
$Bv$ ist also ebenfalls ein Eigenvektor von $A$ zum gleichen Eigenwert.
F"ur einen Eigenvektor $u$ von $B$ gilt analog, dass $Au$ ein Eigenvektor
von $B$ zum gleichen Eigenwert ist.

Zu jedem Eigenwert $a$ von $A$ sie $E_a$ der Raum der von den 
Eigenvektoren von $A$ zum Eigenwert $a$ aufgespannte Vektorraum.
$E_a$ heisst {\em Eigenraum} von $A$ zum Eigenwert $a$.
\index{Eigenraum}%
Nach Satz~\ref{skript:evorthogonal} sind die Eigenr"aume $E_a$ 
f"ur verschiedene $a$ orthogonal, und in jedem Eigenraum wirkt $A$
wie das $a$-fache der Einheitsmatrix.

Es gibt nur endliche viele verschiedene Eigenwert $a_1,\dots,a_k$.
Bildet man eine Basis des Zustandsraumes, indem man in jedem $E_{a_i}$
eine Basis w"ahlt, dann bekommt $A$ die Form
\[
A=\left(
\begin{array}{ccccccccccc}
a_1   &\dots &0     &      &      &      &      &      &      &      &      \\
\vdots&\ddots&\vdots&      &      &      &      &      &      &      &      \\
0     &\dots &a_1   &      &      &      &      &      &      &      &      \\
%\hline
      &      &      &a_2   &\dots &0     &      &      &      &      &      \\
      &      &      &\vdots&\ddots&\vdots&      &      &      &      &      \\
      &      &      &0     &\dots &a_2   &      &      &      &      &      \\
%\hline
      &      &      &      &      &      &\ddots&      &      &      &      \\
      &      &      &      &      &      &      &\ddots&      &      &      \\
%\hline
      &      &      &      &      &      &      &      &a_k   &\dots &0     \\
      &      &      &      &      &      &      &      &\vdots&\ddots&\vdots\\
      &      &      &      &      &      &      &      &0     &\dots &a_k   
\end{array}
\right)
\]

Weiter oben wurde gezeigt, dass der Operator $B$ die Eigenr"aume $E_a$ in sich
abbildet: $BE_a\subset E_a$. Da $B$ ausserdem selbstadjungiert ist, kann man
in jedem Eigenraum $E_{a_i}$ eine Basis
${\cal B}_i= \{b_1^{(i)},\dots,b_{l_i}^{(i)}\}$
aus Eigenvektoren f"ur $B$ finden. Die Vektoren $b_j^{(i)}$ sind nach
Konstruktion sowohl Eigenvektoren von $A$ zum Eigenwert $a_i$ als auch
Eigenvektoren von $B$. Die Vereinigung
\[
{\cal B}
=
\bigcup_{i=1}^k{\cal B}_i
=
\{
b_1^{(1)},\dots,b_{l_1}^{(1)},
b_1^{(2)},\dots,b_{l_2}^{(2)},
\dots,
b_k^{(k)},\dots,b_{l_k}^{(k)}
\}
\]
der Basen ${\cal B}_i$ ist eine Basis des Zustandsraumes aus gemeinsamen
Eigenvektoren von $A$ und $B$, in ihr haben beide Operatoren Diagonalform.
\end{proof}

\begin{satz}
Zwei Observablen haben genau dann eine gemeinsame Basis aus Eigenvektoren,
wenn sie vertauschen.
\end{satz}

\begin{proof}[Beweis]
Aus Satz~\ref{skript:kommevbasis} leiten wir ab, dass das Kriterium
hinreichend ist.
Der Kommutator $[A,B]$ annihiliert gemeinsame Eigenvektoren von $A$
und $B$.
G"abe es eine Basis aus Eigenvektoren, dann m"usste der Kommutator alle
Basisvektoren annihilieren, es folgte $[A,B]=0$. Dieser
Widerspruch zeigt, dass das Kriterium auch notwendig ist.
\end{proof}


\section{Unscharfe Kenntnis von Observablen}
\rhead{Unscharfe Kenntnis von Observablen}
Im Allgemeinen kann man nicht annehmen, dass ein Beobachtung einer
Observablen $A$ im Zustand $|\psi\rangle$ immer den gleichen Wert ergibt.
Die Messwerte werden zum Beispiel wegen der unvermeidlichen
Unzul"anglichkeiten der Messung streuen.
Dann ist der Erwartungswert $\langle \psi|\,A\,|\psi\rangle$
der Observablen $A$ im Zustand $|\psi\rangle$ die genaueste Information,
die wir erhalten k"onnen.
Zur Abk"urzung schreiben wir $\langle A\rangle=\langle\psi|\,A\,|\psi\rangle$.
Mit $A$ ist auch $A-\langle A\rangle$ eine Observable, und ihr Erwartungswert
ist nat"urlich $0$.

Damit stellt sich die Frage, ob man f"ur spezielle Zust"ande eventuell
mehr Information erhalten kann.
Da die Varianz misst, wie stark die m"oglichen Werte streuen, suchen
wir Zust"ande mit verschwindender Varianz.
Es wird sich zeigen, dass dies genau die Eigenzust"ande sind.
Genaue Kenntnis von Observablenwerten ist also genau f"ur die
Eigenzust"ande m"oglich.

\subsection{Varianz}
In der Wahrscheinlichkeitsrechnung lernt man mit der Varianz eine
Gr"osse kennen, die angibt, wie nahe bei $\langle A\rangle$ die
Werte der Observable im Mittel anzutreffen sind.
Die Varianz ist die mittlere quadratische Abweichung, also der
Erwartungswert der Gr"osse $(A-\langle A\rangle)^2$, die auch wieder
eine Observable ist. Sie kann also mit unserem Formalismus berechnet
werden:
\begin{align*}
\operatorname{var}(A)
&=
\operatorname{var}_{|\psi\rangle}(A)
=
\langle \psi|\, (A-\langle A\rangle)^2\,|\psi\rangle
=
\langle\psi|\, A^2-2\langle A\rangle A+\langle A\rangle^2\,|\psi\rangle
\\
&=
\langle\psi|\,A^2\,|\psi\rangle 
-2\langle A\rangle \langle\psi|\,A\,|\psi\rangle
+\langle A\rangle^2\langle\psi|\psi\rangle
=\langle A^2\rangle -\langle A\rangle^2.
\end{align*}
Wir schreiben den Zustand als Index zum $\operatorname{var}$-Operator
nur, wenn nicht klar ist, welcher Zustand gemeint ist.
Dies entspricht der aus der Wahrscheinlichkeitsrechung bekannten Formel
$E(X^2)-E(X)^2$.
Wir nennen die Standardabweichung, also
\[
\sqrt{\operatorname{var}(A)}
=
\sqrt{\langle (A-\langle A\rangle)^2\rangle}
=
\sqrt{\langle A^2\rangle - \langle A\rangle^2}
\]
auch die {\em Unsch"arfe} der Messung von $A$, und schreiben daf"ur abgek"urzt
auch $\Delta A$.
\index{Unscharfe@Unsch\"arfe}%

Die Varianz misst, wie scharf ein Messwert in einem bestimmten Zustand
bekannt sein kann.
Der n"achste Satz zeigt, dass Zust"ande verschwindender Varianz
automatisch Eigenzust"ande sind. 
Von einem scharf definierten Wert einer Observablen kann man also nur
bei Eigenzust"anden sprechen.

\begin{satz}
Ist $A$ eine Observable und $|\psi\rangle$ ein Zustand, dann verschwindet
die Varianz im Zustand $|\psi\rangle$ genau dann, wenn $|\psi\rangle$
ein Eigenverktor von $A$ ist.
\end{satz}

\begin{proof}[Beweis]
Wir d"urfen ohne Einschr"ankung der Allgemeinheit annehmen, dass
$\langle\psi|\psi\rangle=1$. Ausserdem k"onnen wir schreiben
\[
A\,|\psi\rangle = a_{\|}\,|\psi\rangle + a_{\perp}\,|\varphi\rangle,
\]
wobei $|\varphi\rangle$ ein auf $|\psi\rangle$ senkrecht stehender,
normierter Vektor ist. Es ist also $\langle\psi|\varphi\rangle=0$
und $\langle\varphi|\varphi\rangle=1$.

Nach diesen Voraussetzungen berechnen wir jetzt die Varianz von $A$ im
Zustand $|\psi\rangle$:
\begin{align*}
\langle A\rangle
&=
\langle\psi|\,A\,|\psi\rangle
=
\langle\psi|\,a_{\|}\,|\psi\rangle
+
\langle\psi|\,b_{\perp}\,|\varphi\rangle
=
a_{\|}\langle\psi|\psi\rangle
+
b_{\perp}\langle\psi|\varphi\rangle
=
a_{\|}
\\
\langle A^2\rangle
&=
\langle\psi|\,A^2\,|\psi\rangle
=
(\langle\psi|\,\bar a + \langle\varphi|\,\bar a_{\perp})(a\,|\psi\rangle+a_{\perp}\,|\varphi\rangle)
\\
&=
\langle\psi|    \bar a_{\|} a_{\|} |\psi\rangle
+
\langle\psi|    \bar a_{\|} a_{\perp} |\varphi\rangle
+
\langle\varphi| \bar a_{\perp} a_{\|} |\psi\rangle
+
\langle\varphi| \bar a_{\perp} a_{\perp} |\varphi\rangle
\\
&=
|a_{\|}|^2+|a_{\perp}|^2
\\
\operatorname{var}(A)
&=
\langle A^2\rangle -\langle A\rangle^2
=
|a_{\|}|^2+|a_{\perp}|^2
-
a_{\|}^2
\end{align*}
Aus der Gleichung f"ur $\langle A\rangle$ kann man ablesen, dass $a_{\|}$ 
als Erwartungswert eines selbstadjungierten Operators reell ist.
Zusammen mit der Gleichung f"ur $\operatorname{var}(A)$ folgt daraus, dass
$
\operatorname{var}(A)=|a_{\perp}|^2
$ ist.

Wenn die Varianz verschwindet, dann muss also $a_{\perp}=0$ sein, somit
gilt f"ur $|\psi\rangle$, dass
\[
A\,|\psi\rangle = a_{\|}\,|\psi\rangle,
\]
$|\psi\rangle$ ist also ein Eigenvektor von $A$ zum Eigenwert $a_{\|}$.
Ist umgekehrt $|\psi\rangle$ ein Eigenvektor, dann ist $a_{\perp}=0$,
und die Varianz verschwindet.
\end{proof}

\subsection{Gleichzeitige Kenntnis von Observablen}
Exakte Kenntnis von Observablenwerten ist also genau dann m"oglich, wenn
ein Eigenzustand vorliegt.
Exakte Kenntnis der Werte von zwei Observablen $A$ und $B$ ist genau
dann m"oglich, wenn der Zustand ein Eigenzustand von {\em beiden }
Operatoren ist.
Vollst"andige gleichzeitige Kenntnis der Werte von zwei Observablen
ist also m"oglich, wenn es eine Basis des Zustandsraumes gibt, 
die aus Eigenvektoren f"ur beide Operatoren besteht. 
Mit Satz~\ref{skript:kommevbasis} k"onnen wir jetzt ein Kriterium
daf"ur formulieren:

\begin{satz} Gleichzeitige exakte Kenntnis von Observablen ist genau
m"oglich, wenn die Observablen vertauschen.
\end{satz}

In der klassischen Mechanik haben wir keine Schwierigkeit, Ort und
Impuls gleichzeitig festzustellen. Das liegt nat"urlich daran, dass
der Kommutator den sehr kleinen Faktor $\hbar$ enth"alt.
F"ur makroskopische Zwecke sind Ort und Impuls also gleichzeitig 
feststellbar.

%Wenn es also nicht m"oglich ist, Ort und Impuls eines Teilchens
%gleichzeitig exakt zu wissen, wie genau ist es dann m"oglich,
%Ort und Impuls zu wissen?
%Um diese Frage zu beantworten m"ussen wir zun"achst ein Mass f"ur
%die Genauigkeit finden, mit der Ort oder Impuls in einem bestimmten
%Zustand bekannt sein kann.
%Wir werden es die Unsch"arfe nennen.
%Wir erwarten dann eine Ungleichung, die uns sagt, wie grosse 
%Unsch"arfe sein muss, eine Unsch"arferelation.

\subsection{Kovarianz und Unsch"arfeprodukt}
Die Varianz ist ein Spezialfall der Kovarianz, die als 
\[
\operatorname{cov}(X,Y)
=
E((X-E(X))(Y-E(Y)))
\]
definiert war, $\operatorname{var}(X)=\operatorname{cov}(X,X)$.
Analog k"onnen wir jetzt auch eine Kovarianz f"ur Observable als
\[
\langle A,B\rangle
=
\langle\psi|
\,
(A-\langle A\rangle)(B-\langle B\rangle)
\,
|\psi\rangle
\]
definieren.
Man beachte, dass die Gr"osse $\langle A,B\rangle$ nicht symmetrisch zu
sein braucht, ausser wenn die Operatoren $A$ und $B$ vertauscht werden
k"onnen.


Man kann $\langle A,B\rangle$ auch als den Wert der Transformationsfunktion 
zweier Zust"ande ansehen:
\begin{equation}
\left.
\begin{aligned}
|a\rangle &= (A-\langle A\rangle)\,|\psi\rangle\\
|b\rangle &= (B-\langle B\rangle)\,|\psi\rangle
\end{aligned}
\right\}
\quad
\Rightarrow
\quad
\langle A,B\rangle = \langle a|b\rangle.
\end{equation}
Nat"urlich gilt die Cauchy-Schwarz-Ungleichung f"ur die beiden
\index{Cauchy-Schwarz-Ungleichung}%
Zust"ande $|a\rangle$ und $|b\rangle$, also
\begin{equation}
|\langle a|b\rangle|^2
\le
\langle a|a\rangle \langle b|b\rangle
=
\operatorname{var}(A)\operatorname{var}(B)
=\Delta A^2 \cdot \Delta B^2
\label{skript:cauchy-schwarz-uncertainty}
\end{equation}
Falls also die linke Seite nicht $0$ ist, dann k"onnen wir die beiden
Gr"ossen $A$ und $B$ nicht beliebig genau kennen.
Die Ungenauigkeit, mit der $A$ und $B$ bekannt sein k"onnen, sind "uber
die Ungleichung (\ref{skript:cauchy-schwarz-uncertainty}) miteinander verkn"upft.
Wenn die Unsch"arfe verkleinert wird, mit der $A$ bekannt ist, dann 
vergr"ossert sich die Unsch"arfe, mit der $B$ bekannt ist.

Um das Unsch"arfeprodukt nach unten absch"atzen zu k"onnen, m"ussen
\index{Unscharfeprodukt@Unsch\"arfeprodukt}%
wir $\langle a|b\rangle$ ausrechnen:
\begin{align}
\langle a|b\rangle
&=
\langle\psi|\,
(A-\langle A\rangle)(B-\langle B\rangle)
\,|\psi\rangle
=
\langle\psi|\,
AB-\langle A\rangle B-A\langle B\rangle
+
\langle A\rangle\langle B\rangle
\,|\psi\rangle
\notag
\\
&=
\langle\psi|\,AB\,|\psi\rangle
-\langle B\rangle\langle\psi|\,A\,|\psi\rangle
-\langle A\rangle\langle\psi|\,B\,|\psi\rangle
+
\langle A\rangle\langle B\rangle
\notag
\\
&=
\langle\psi|\,AB\,|\psi\rangle
-
\langle A\rangle\langle B\rangle
=
\langle AB\rangle
-
\langle A\rangle\langle B\rangle
.
\label{skript:abausrechnung}
\end{align}
Damit haben wir zwar eine untere Schranke f"ur das Unsch"arfeprodukt
gefunden, allerdings k"onnen wir (\ref{skript:abausrechnung}) meistens
nicht direkt ausrechnen.

%
% Unschaerferelation
%
\section{Unsch"arferelation}
\rhead{Unsch"arferelation}
\begin{figure}
\centering
\includegraphics{graphics/heisenberg-1.pdf}
\caption{Einstein-Podolsky-Rosen Experiment
\label{skript:epr-experiment}}
\end{figure}
\index{Unscharferelation@Unsch\"arferelation}%
Werner Heisenberg hat 1926 eine Unsch"arferelation zwischen Ort und Impuls
\index{Heisenberg, Werner}%
mit Hilfe eines klassischen Gedankenexperimentes hergeleitet.
Seine Erkl"arung beruht auf dem Beobachter-Effekt: jede Messung
ver"andert das beobachtete System.
\index{Einstein-Podolsky-Rosen Experiment}%
Der Beobachter-Effekt reicht jedoch f"ur die Begr"undung nicht
\index{Beobachter-Effekt}%
aus, wie man aus dem folgenden Gedanken-Experiment von Einstein,
Podolsky und Rosen sehen kann, siehe
auch Abbildung~\ref{skript:epr-experiment}, \cite{skript:epr}.
In diesem Experiment werden im Nullpunkt aus einem ruhenden System
zwei Teilchen ausgesandt.
Da der Impuls erhalten ist, m"ussen die Teilchen exakt entgegengesetzen
Impuls haben.
Wenn man nach einer bestimmten Zeit die Position des linken Teilchens misst,
weiss man auch die Position des rechten Teilchens.
Und wenn man den Impuls des rechten Teilchens misst, dann weiss
man auch den Impuls des linken Teilchens.
Es scheint, dass man mit dieser Apparatur Heisenbergs Unsch"arferelation
umgehen kann.
Offenbar braucht es eine bessere Begr"undung f"ur die Unsch"arferelation.

Das Gedankenexperiment ist allerdings doch etwas subtiler als auf den
ersten Blick scheint. 
Die in dem Experiment tatsächlich messbaren Gr"ossen haben n"amlich
vertauschbare Operatoren, wie die Analyse in "Ubungsaufgabe~1 dieses
Kapitels zeigt.


\subsection{Robertson-Schr"odinger-Unsch"arferelation}
\index{Robertson-Schr\"odinger-Unsch\"arferelation}%
\index{Unscharferelation@Unsch\"arferelation!von Robertson-Schr\"odinger}%
Die Ungleichung (\ref{skript:cauchy-schwarz-uncertainty}) liefert eine
Unsch"arferelation, falls die linke Seite $>0$ ist. Es ist
also zu untersuchen, wie gross die linke Seite von 
(\ref{skript:cauchy-schwarz-uncertainty}) tats"achlich ist.
Setzen wir $z=\langle a|b\rangle$, dann ist $|z|^2$ die linke Seite
von (\ref{skript:cauchy-schwarz-uncertainty}). Den Betrag kann man mit Hilfe
von Real- und Imagin"arteil unter Zuhilfenahme von (\ref{skript:abausrechnung})
ausrechnen:
\[
|z|^2
=
(\operatorname{Re}z)^2+(\operatorname{Im}z)^2
=
\biggl(\frac{z+\bar z}2\biggr)^2 + \biggl(\frac{z-\bar z}{2i}\biggr)^2.
\]
(Vergleiche hierzu auch die Formeln (\ref{skript:realteil-formel}) und 
(\ref{skript:imaginaerteil-formel})).
Jetzt setzen wir $z=\langle a|b\rangle$ ein:
\begin{equation}
|\langle a|b\rangle|^2
=
\biggl(\frac{\langle a|b\rangle + \langle b|a\rangle}2\biggr)^2
+
\biggl(\frac{\langle a|b\rangle - \langle b|a\rangle}{2i}\biggr)^2.
\label{skript:unschaerfe2}
\end{equation}
Wir m"ussen also die Terme $\langle a|b\rangle + \langle b|a\rangle$
und $\langle a|b\rangle - \langle b|a\rangle$ ausrechnen:
\begin{align*}
\frac{\langle a|b\rangle + \langle b|a\rangle}2
&=
\frac{
\langle\psi|AB+BA|\psi\rangle 
}2
-\langle A\rangle\langle B\rangle
=
\frac12 \langle\,\{A,B\}\,\rangle - \langle A\rangle\langle B\rangle,
\\
\frac{\langle a|b\rangle - \langle b|a\rangle}{2i}
&=
\frac{\langle\psi|AB-BA|\psi\rangle}{2i}
=
\frac1{2i}\langle [A,B]\rangle.
\end{align*}
Eingesetzt in (\ref{skript:unschaerfe2}) finden wir
\[
|\langle a|b\rangle|^2
=
\biggl(
\frac12\langle \{A,B\}\rangle - \langle A\rangle\langle B\rangle
\biggr)^2
+
\biggl(
\frac{\langle[A,B]\rangle}{2i}
\biggr)^2.
\]
Eingesetzt in die urspr"ungliche Unsch"arferelation
(\ref{skript:cauchy-schwarz-uncertainty}) erhalten wir jetzt die Unsch"arferelation
von Robertson und Schr"odinger:
\index{Schrodinger, Erwin@Schr\"odinger, Erwin}%

\begin{satz}
\label{skript:robertson-schroedinger-unschaerfe}
Sind $A$ und $B$ selbstadjungierte Operatoren und $|\psi\rangle$ ein
Zustand, dann gilt die Unsch"arferelation
\begin{equation}
\Delta A^2\cdot\Delta B^2\ge 
\biggl(
\frac12\langle \{A,B\}\rangle - \langle A\rangle\langle B\rangle
\biggr)^2
+
\biggl(
\frac{\langle[A,B]\rangle}{2i}
\biggr)^2.
\label{skript:uncertainty}
\end{equation}
\end{satz}

F"ur Ort und Impuls haben wir den Kommutator schon ausgerechnet, es
muss also gelten
\begin{equation}
\Delta X^2\cdot \Delta P^2
\ge
\biggl(
\frac12\langle \{X,P\}\rangle - \langle X\rangle\langle P\rangle
\biggr)^2
+
\biggl(
\frac{\langle[X,P]\rangle}{2i}
\biggr)^2
\ge
\biggl(
\frac{\langle[X,P]\rangle}{2i}
\biggr)^2
\ge \frac{\hbar^2}4
\end{equation}
oder
\begin{equation}
\Delta X\cdot\Delta P\ge \frac{\hbar}2.
\end{equation}
Dies ist die Heisenbergsche Unsch"arferelation. 
\index{Unscharferelation@Unsch\"arferelation!von Heisenberg}%

\subsection{Drehimpuls und Positionswinkel}
Heisenberg-Unsch"arfe war auch schon Thema bei XKCD, siehe
Abbildung~\ref{skript:heisenberg:xkcd} und \cite{skript:xkcd}.
\begin{figure}
\centering
\includegraphics[width=0.8\hsize]{images/xkcd-location-sharing.png}
\caption{XKCD Comic zur Heisenberg Unsch"arfe. Die Benutzerin stellt sicher,
dass die Heisenbergsche Unsch"arferelation nicht verletzt wird. Man kann nicht
sowohl den Ort als auch den Impuls wissen.
\label{skript:heisenberg:xkcd}}
\end{figure}
Im Kapitel~\ref{chapter:drehimpuls} werden wir ein weiteres Paar
von Operatoren wie Ort und Impuls kennen lernen.
Dort untersuchen wir die Komponente $L_3$ des Drehimpulses in $z$-Richtung.
Die dazugeh"orige Ortskoordinaten ist in Kugelkoordinaten die geographische 
L"ange $\varphi$ des Teilchens, welche durch eine Observable mit Operator
$\Phi$ festgestellt werden kann.
In Kugelkoordinaten hat $L_3$ die Form
\[
L_3=\frac{\hbar}{i}\frac{\partial}{\partial\varphi}.
\]
Die Vertauschungsrelationen f"ur $\Phi$ und $L_3$ sind
\index{Vertauschungsrelationen!Winkel und Drehimpuls}%
\[
[\Phi,L_3]=\frac{\hbar}{i}\operatorname{id},
\]
das entspricht den Vertauschungsrelationen von $X$ und $P$.
Zu $\Phi$ und $L_3$ geh"ort nach Satz~\ref{skript:robertson-schroedinger-unschaerfe}
die Unsch"arferelation
\[
\Delta\varphi\cdot\Delta l_3 \ge \frac{\hbar}2,
\]
man kann also den Drehimpuls und die Richtung $\varphi$ nicht gleichzeitig
beliebig genau kennen.
Dies erkl"art den Mouse-Over-Text des Comics in
Abbildung~\ref{skript:heisenberg:xkcd}:
\begin{quote}
Our phones must have great angular momentum sensors because
the compasses really suck.
\end{quote}

\subsection{Unsch"arfe von Energie und Zeit}
In unserem Formalismus hat die Zeit-Koordinate eine ausgezeichnete
Bedeutung, die Schr"odingergleichung beschreibt die Entwicklung
der Zust"ande in der Ortsdarstellung.
Dies hat zur Folge, das wir eine Unsch"arferelation zwischen Energie
und Zeit nicht aus der Robertson-Schr"odinger-Unsch"arferelation
(\ref{skript:uncertainty}) herleiten k"onnen.

Dazu br"auchten wir eine Theorie, welche Orts- und Zeitkoordinaten
gleichwertig behandelt.
Eine solche Theorie ist die spezielle Relativit"atstheorie, welche
zeigt, wie Raum und Zeit untrennbar in einer vierdimensionalen
Raumzeit miteinander verbunden sind.

F"ur einen Spezialfall kann man eine Unsch"arferelation zwischen
Energie und Zeit finden.
Betrachten wir dazu ein Photon mit Energie $E=h\nu$.
Schon die klassische Elektrodynamik erkl"art das Ph"anomen des
Strahlungsdrucks, welches auch schon von Raumfahrzeugen ausgenutzt
werden ist, um durch das Sonnensystem zu `segeln'.
Strahlungsdruck bedeutet, dass Photonen Impuls $p=E/c$ transportieren
und "ubertragen.
Da alle Photonen gleich schnell sind, n"amlich $c$, ist die
Positionsunsch"arfe $\Delta x$ eines Photons im Wesentlichen die
Zeitunsch"arfe der Messung $\Delta t$.
Die Heisenbergsche Unsch"arferelation wird daher
\[
\frac{\hbar}2\le \Delta x\cdot\Delta p=c\Delta t\cdot \frac{\Delta E}{c}=
\Delta t\cdot \Delta E.
\]
Die Unsch"arfe zwischen Zeit und Energie bedeutet, dass die
Energieerhaltung um den Betrag $\Delta E$ verletzt werden darf,
wenn auch nur f"ur kurze Zeit $\Delta t$.
\index{Vertauschungsrelationen!Energie und Zeit}%

\section*{"Ubungsaufgaben}
\rhead{"Ubungsaufgaben}
\begin{uebungsaufgaben}
\item
Betrachten Sie die beiden Teilchen im Einstein-Podolsky-Rosen Experiment
mit Ortsoperatoren $X_i$ und Impulsoperatorn $P_i$ mit $i=1,2$.
F"ur diese Operatoren gelten die Vertauschungsrelationen
$[X_i,P_i]=-i\hbar\operatorname{id}$, und wir k"onnen nicht
den Ort und den Impuls eines der Teilchens gleichzeig wissen.
Mit der experimentellen Vorschrift von Einstein, Podolsky und Rosen
bestimmt man nicht $X_1$ und $X_2$, sondern
die Differenz $X=X_1-X_2$, denn man kann nicht die Position des
Ausgangssystems messen, ohne es zu ver"andern. Ebenso kann man nicht
den Impuls eines Teilchens, sondern nur den Gesamtimpuls $P=P_1-P_2$ 
bestimmen. Zeigen Sie, dass man diese beiden Gr"ossen tats"achlich
beide beliebig genau wissen kann, dass das klassische Einsten-Podolsky-Rosen
Experiment also keinen Widerspruch zur Unsch"arferelation darstellt.

\begin{loesung}
Man muss den Kommutator $[X,P]$ bestimmen:
\begin{align*}
[X,P]
&=
[X_1-X_2,P_1-P_2]
=
\underbrace{[X_1,P_1]}_{-i\hbar\operatorname{id}}
-\underbrace{[X_2,P_1]}_{=0}
-\underbrace{[X_1,P_2]}_{=0}
-\underbrace{[X_2,P_2]}_{-i\hbar\operatorname{id}}
=
-i\hbar\operatorname{id}
+i\hbar\operatorname{id}
=0
\end{align*}
Die beiden Operatoren vertauschen, also k"onnen sie tats"achlich gleichzeit
beliebig genau bestimmt sein.
\end{loesung}



\item
Kann man Impuls und Energie eines Teilchens beliebig genau wissen?
Wenn ja, formulieren Sie eine Unsch"arferelation.

\begin{loesung}
Man kann Impuls und Energie eines Teilchens genau dann bleibig genau
wissen, wenn die zugeh"origen Operatoren $P$ und $H$ vertauschen.
Also berechnen wir den Kommutator:
\begin{align*}
[P,H]
&=
\biggl[P,\frac1{2m}P^2+V\biggr]
=
[P,V].
\end{align*}
Wir k"onnen daraus schon mal ablesen, dass f"ur ein freies Teilchen,
also f"ur $V=0$, die beiden Operatoren tats"achlich vertauschbar sind,
wir k"onnen also tats"achlich Impuls und Energie gleichzeitig 
beliebig genau kennen.

Wenn $V\ne 0$ ist, m"ussen wir den Kommutator von $P$ mit $V$
bestimmen:
\begin{align*}
[P,V]
&=
\biggl[\frac{\hbar}{i}\frac{\partial}{\partial x}, V\biggr]
=
\frac{\hbar}{i}
\biggl(
\frac{\partial}{\partial x}
V
-
V
\frac{\partial}{\partial x}
\biggr)
=
\frac{\hbar}{i}
\biggl(
\frac{\partial V}{\partial x}
+
V
\frac{\partial}{\partial x}
-
V
\frac{\partial}{\partial x}
\biggr)
=
\frac{\hbar}{i}
\frac{\partial V}{\partial x}.
\end{align*}
Die zugeh"orige Unsch"arferelation ist
\[
\Delta p\cdot \Delta E
=
\frac{\hbar}{2m}\biggl\langle \frac{\partial V}{\partial x}\biggr\rangle.
\]
F"ur schwere Teilchen, also zum Beispiel f"ur makroskopische K"orper ist
die Unsch"arfe nicht wahrnehmbar.
Die Unsch"arfe ist also umso gr"osser, je gr"osser die Ableitung des
Potentials ist.
Physikalisch kann man dies damit erkl"aren, dass es bei bekanntem Impuls
keine scharfe Position eines Teilchens gibt, die Energie wird also an
verschiedenen Orten gemessen, wo verschiedene potentielle Energien 
gemessen werden.
Die Ableitung des Potentials ist aber die Kraft, die auf ein Teilchen wirkt.
Die Unsch"arfe ist also umso gr"osser, je gr"osser die Kr"afte sind, die
auf das Teilchen wirken.
\end{loesung}


\item
Kann man Energie und Position eines Teilchens beliebig genau wissen?
Wenn nicht, formulieren Sie eine entsprechende Unsch"arferelation.

\begin{loesung}
Die Energie eines Teilchens wird durch die Observable $H$, den
Hamilton-Operator gegeben.
Die Position wird hingegen durch den Operator $X$ der Position
gegeben.
Man kann beide beleibig genau wissen, d.~h.~es gibt gemeinsame
Eigenzust"ande von $H$ und $X$, wenn die beiden Operatoren vertauschen.
Wir berechnen daher den Kommutator von $X$ und $H$:
\begin{align*}
[X,H]
&=
\biggl[X,\frac1{2m}P^2\biggr]
=
\frac1{2m}[X,P^2]
=
\frac1{2m}(XPP-PPX)
=
\frac1{2m}(XPP-PXP+PXP-PPX)
\\
&=
\frac1{2m}([X,P]P+P[X,P])
=
-\frac{\hbar}{im} P
\end{align*}
Die zugeh"orige Unsch"arferelation ist
\[
\Delta E\cdot \Delta p\ge \frac{\hbar}{2}\langle P\rangle.
\]
Die Unsch"arfe h"angt also vom Impuls eines Teilchens ab, je gr"osser
der Impuls, desto gr"osser auch die Unsch"arfe.
\end{loesung}


\end{uebungsaufgaben}

\chapter{Harmonischer Oszillator\label{chapter:harmonischeroszillator}}
\lhead{Harmonischer Oszillator}
\rhead{}
Das klassische Federpendel ist eines der einfachsten mechanischen 
Systeme. Es ist nicht nur von theoretischer Bedeutung, denn sehr
viele schwingende Systeme k"onnen mit dem gleichen Modell
oder als eine Zusammensetzung solcher einfacher Oszillatoren
beschrieben werden.

In der Quantenmechanik ist die Bedeutung der quantisierten Version
dieses einfachen Systems nicht geringer. Viele Systeme k"onnen
n"aherungsweise durch einen harmonischen Oszillator beschrieben
werden, zum Beispiel die Schwingungen von einfachen Molek"ulen.

Der quantenmechanische harmonische Oszillator hat aber eine Eigenschaft,
die der klassische Oszillator nicht hat: seine Energie ist nicht
kontinuierlich, nur eine diskrete Menge von Energieniveaus ist
m"oglich. Dies ist zwar nicht "uberraschend nach den bisher studierten
Beispielen, doch werden wir hier die mathematische Technik  der
Auf- und Absteige-Operatoren kennen lernen, mit der die Energieniveaus
und auch die Zustandsvektoren leicht berechnet werden k"onnen.

In der Anwendung auf die Schwingungen einfacher Molek"ule
bedeutet das diskrete Spektrum, dass solche Molek"ule Energie nur in
fest definierten Paketen aufnehmen k"onnen.
Jedes Molek"ul hat also eine charakteristische Menge von Wellenl"angen
von typischerweise infrarotem Licht, welche es absorbieren kann.
Tats"achlich stellt die Infrarotsprektroskopie eine wichtige
Technik in der Chemie dar, mit der Molek"ule identifiziert werden k"onnen.
Wenn ein Molek"ul nur bestimmte Energiepakete aufnehmen kann, dann
wird die W"armekapazit"at ebenfalls temperaturabh"angig sein, und
mit charakteristischen Spr"ungen ansteigen, welche der Anregung
von h"oherenergetischen Molek"ulschwingungen entsprechen.

\section{Der klassische Oszillator}
\rhead{Klasischer Oszillator}
Ein Federpendel ist eine Masse $m$, welche sich unter dem Einfluss
einer Feder mit Federkonstante $K$ bewegt. Ist $x$ die Auslenkung
der Masse aus der Ruhelage, dann "ubt die Feder eine Kraft $-Kx$ 
auf die Masse aus. Die potentielle Energie bei Auslenkung $x$ ist
\[
V(x)=\int_0^xK\xi\,d\xi=\frac12Kx^2.
\]
Die Hamilton-Funktion ist daher
\[
H(p,x)=\frac1{2m}p^2+\frac12Kx^2.
\]
Zur Kontrolle leiten wir mit Hilfe der Hamiltonschen Gleichungen
die Bewegungsgleichungen ab:
\begin{align*}
\frac{\partial H}{\partial x}&=Kx
&
&\Rightarrow&
\dot p&=-\frac{\partial H}{\partial x}=-Kx
\\
\frac{\partial H}{\partial p}&=\frac{p}{m}
&
&\Rightarrow&
\dot x&=\frac{\partial H}{\partial p}=\frac{p}{m}.
\end{align*}
Wir erhalten also die klassischen  Bewegungsgleichungen eines harmonischen
Oszillators. 

Die L"osungen dieser Bewegungsgleichungen sind Schwingungen mit 
Frequenz
\[
\omega = \sqrt{\frac{K}{m}},
\]
statt der Konstanten $K$ k"onnten wir auch $K=m\omega^2$ verwenden,
die Hamilton-Funktion ist dann
\begin{equation}
H=\frac1{2m}p^2+\frac12m\omega^2x^2.
\end{equation}

\section{Quantisierung}
\rhead{Quantisierung}
Die Quantisierungsregeln verlangen wieder, dass die Impuls-Variable
durch
\[
p\rightarrow\frac{\hbar}{i}\frac{\partial}{\partial x}
\]
ersetzt wird. Der Hamilton-Operator ist also
\begin{equation}
\hat H
=
-\frac{\hbar^2}{2m}
\frac{\partial^2}{\partial x^2}
+\frac12m\omega^2x^2.
\label{skript:harmoszhamilton}
\end{equation}

Wir suchen jetzt Eigenfunktionen des Hamilton-Operators (\ref{skript:harmoszhamilton}),
also Funktionen $\psi(x)$ und Konstanten $E$ so, dass
\[
\hat H\psi=E\psi.
\]
Da in diesem Problem nur eine einzige Variable auftritt, ist dies ein
Problem "uber gew"ohnliche Differentialgleichungen:
\begin{equation}
-\frac{\hbar^2}{2m} \psi''(x)+\frac12m\omega^2x^2\psi(x)=E\psi(x).
\label{skript:harmoszgleichung}
\end{equation}
In dieser Form ist das Problem etwas umst"andlich zu l"osen.
Wir verwenden statt $x$ die Variablen 
\[
q=\sqrt{\frac{m\omega}{\hbar}}x.
\]
Nennen wir die Funktion $u(q)=\psi(x)$, dann gilt
\begin{align*}
\frac{d}{dq}u(q)&=\frac{d\psi(x)}{dx}\frac{dx}{dq}
=\psi'(x)\sqrt{\frac{\hbar}{m\omega}}
&
\psi'(x)&=\sqrt{\frac{m\omega}{\hbar}}u'(q)
\\
\frac{d^2}{dq^2}u(q)&=\psi''(x)\frac{\hbar}{m\omega}
&
\psi''(x)&=\frac{m\omega}{\hbar}u''(q).
\end{align*}
Setzen wir dies in die Differentialgleichung (\ref{skript:harmoszgleichung})
ein, erhalten wir
\begin{equation}
-\frac{\hbar\omega}{2} u''(q)
+\frac12m\omega^2x^2\psi(x)=E\psi(x).
\end{equation}
Teilen wir dies durch $-\hbar\omega/2$, erhalten wir
\begin{align*}
u''(q) -\frac{m\omega}{\hbar}x^2u(q)=-\frac{2E}{\hbar\omega}u(q),
\end{align*}
oder mit den Abk"urzungen
\[
q=\sqrt{\frac{m\omega}{\hbar}}x
\quad\text{und}\quad
\varepsilon=\frac{E}{\hbar \omega}
\]
die endg"ultige Form der Differentialgleichung
\begin{equation}
u''(q)+(2\varepsilon-q^2) u(q)=0.
\label{skript:harmq}
\end{equation}

\section{Wellenfunktionen}
\rhead{L"osungen}
In diesem Abschnitt l"osen wir die Differentialgleichung (\ref{skript:harmq}).
\subsection{Grundzustand\label{skript:hogrundzustand}}
Da das Potential f"ur grosse Werte von $q$ beliebig gross wird,
muss die Wellenfunktion $u(q)$ f"ur grosse Werte von $q$ exponentiell
schnell abfallen.
Wir versuchen daher einen Ansatz in der Form $u_0(q)=e^{-\alpha q^2}$.
Die Ableitungen von
$u_0(x)$ sind
\begin{align*}
u_0'(q)&=-2\alpha qe^{-\alpha q^2}\\
u_0''(q)&=-2\alpha e^{-\alpha q^2}+4\alpha^2q^2e^{-\alpha q^2}.
\end{align*}
Einsetzen in (\ref{skript:harmoszgleichung})  liefert
\begin{align*}
(-2\alpha e^{-\alpha q^2}+4\alpha^2q^2e^{-\alpha q^2})
+
(2\varepsilon - q^2)e^{-\alpha q^2}&=0
\\
\Leftrightarrow\qquad
\biggl(
-2\alpha +4\alpha^2q^2
+
2\varepsilon - q^2
\biggr)e^{-\alpha q^2}
&=0
\end{align*}
Die letzte Gleichung kann nur erf"ullt werden, wenn der grosse Klammerausdruck
verschwindet, wenn also gilt
\begin{align*}
-2\alpha +4\alpha^2q^2
+
2\varepsilon - q^2
&=0
\\
\Leftrightarrow\qquad
(2\varepsilon-2\alpha)
+
(4\alpha^2-1)q^2
&=0
\end{align*}
Diese Gleichung muss f"ur alle $q$ gelten, die Klammerausdr"ucke
m"ussen also beide verschwinden:
\begin{align*}
(4\alpha^2-1)q^2&=0
&
&\Rightarrow&
\alpha&=\frac12
\\
\varepsilon-\alpha&=0
&
&\Rightarrow&
\varepsilon&=\frac12.
\end{align*}
Es gibt also tats"achlich eine L"osung, sie muss von der Form
\[
u_0(q)=e^{-\frac{q^2}2}
\]
sein. Der zugeh"orige Eigenwert ist $\varepsilon_0=\frac12$.

Substituieren wir wieder die
urspr"unglichen Koordinaten, erhalten wir
\begin{equation}
\psi_0(x)=u_0\biggl(\sqrt{\frac{m\omega}{\hbar}}x^2\biggr)
=
e^{-\frac{m\omega}{2\hbar}x^2}.
\label{skript:grundzustandwellenfunktion}
\end{equation}
als Wellenfunktion f"ur den Grundzustand.


% XXX Normierung dieser L"osung
\subsection{Angeregte Zust"ande}
Wir versuchen weitere L"osungen in der Form $u_n(q)=h_n(q)e^{-\frac{q^2}2}$
zu finden, wobei $h_n(q)$ ein Polynom $n$-ten Grades ist.
Wir brauchen wieder die Ableitungen:
\begin{align*}
u'(q)
&=
\bigl(h_n'(q)- qh_n(q)\bigr)e^{-\frac{q^2}2}
\\
u''(q)
&=
\bigl(
h_n''(q)- qh_n'(q)
- h_n(q)
- q h_n'(q)
+q^2h_n(q)
\bigr)e^{-\frac{q^2}2}
\\
&=
\bigl(
h_n''(x)-2 qh_n'(q)
+(q^2-1)h_n(q)
\bigr)e^{-\frac{q^2}2}
\end{align*}
Eingesetzt in die Differentialgleichung erhalten wir
\begin{align}
\bigl(
h_n''(x)-2 qh_n'(q)
+(q^2-1)h_n(q)
\bigr)e^{-\frac{q^2}2}
+(\varepsilon-q^2)h_n(q)e^{-\frac{q^2}2}&=0
\notag
\\
\Leftrightarrow\qquad
h_n''(x)-2 qh_n'(q)
+(\varepsilon - 1)h_n(q)&=0
\label{skript:hermiteequation}
\end{align}
\begin{figure}
\centering
\includegraphics[width=\hsize]{graphics/harmonisch-1.pdf}
\caption{Wellenfunktionen des harmonischen Oszillators, dargestellt auf
den zugeh"origen Energieniveaus $\hbar \omega (n+\frac12)$ f"ur
$n=0,\dots 7$.
\label{skript:harmonisch-wellenfunktionen}}
\end{figure}%
Die Gleichung (\ref{skript:hermiteequation}) heisst auch die 
Hermitesche Differentialgleichung.
Die Funktionen $h_n(q)$ heissen die Hermiteschen Polynome.
Man kann zeigen, dass $\varepsilon=n+\frac12$ sein muss.
Wir werden die Polynome hier nicht ausrechnen, da die Technik der
Auf- und Absteigeoperatoren eine einfachere Methode hierf"ur liefern
wird.
Aus dieser Technik werden sich die Energieniveaus ebenfalls ergeben.
In Abbildung~\ref{skript:harmonisch-wellenfunktionen} sind die Wellenfunktionen
des harmonischen Oszillators f"ur die Energieniveaus bis $n=7$ dargestellt.
Man beachte, dass wie beim Potentialtopf in
Abschnitt~\ref{subsection:potentialtopf} das Teilchen sich mit positiver
Wahrscheinlichkeit in einem Bereich aufhalten kann, der mit der Energie
des Teilchens klassisch verboten ist.
In diesem Bereich f"allt die Aufenthaltswahrscheinlichkeit jedoch
"ahnlich wie beim Potentialtopf exponentiell schnell ab.

\section{Algebraische Eigenschaften}
\rhead{Algebra}
Um die Berechnung der Energieniveaus des harmonischen Oszillators
vorzubereiten, wechseln wir zun"achst von den kanoninschen Koordinaten
$x$ und $p$ zu dimensionslosen Gr"ossen $Q$ und $P$, und berechnen
deren algebraischen Eigenschaften.
\subsection{Dimensionslose Operatoren}
Dazu gehen wir zur"uck zum Hamilton-Operator
\[
\hat H=-\frac{\hbar^2}{2m}\frac{\partial^2}{\partial x^2}
+\frac12m\omega^2 x^2
\]
und schreiben ihn unter Verwendung der Operatoren
\[
P=\frac{1}{\sqrt{\hbar m\omega}}p 
\qquad
\text{und}
\qquad
Q=\sqrt{\frac{m\omega}{\hbar}}x
\]
Da $\hbar\omega$ die Dimension einer Energie hat, hat $\hbar m\omega$ die
Dimension des Quadrates eines Impulses. Weil $m\omega^2x^2$ die Dimension
einer Energie hat, ist $m\omega x^2/\hbar$ dimensionslos, also auch
$Q$.
Man kann nachrechnen, dass 
\[
\frac12(P^2+Q^2)
=
\frac{1}{2\hbar m \omega}p^2
+
\frac{m\omega}{2\hbar}x^2
=
-\frac{\hbar}{2 m \omega}\frac{\partial^2}{\partial x^2}
+
\frac{m\omega}{2\hbar}x^2
=\frac{\hat H}{\hbar\omega}
=:\hat{\cal H}.
\]
Wir haben den Hamiltonoperator also umgeschrieben mit neuen Operatoren 
$P$ und $Q$.

\subsection{Vertauschungsrelationen}
Wir sollten uns davon "uberzeugen, dass die Operatoren $P$ und $Q$ 
mit den urspr"unglichen $p$ und $x$ vergleichbar sind, und berechnen
daher die Vertauschungsrelationen:
\begin{align*}
[P,Q]
&=
\frac{1}{\sqrt{\hbar m\omega}} \sqrt{\frac{m\omega}{\hbar}}[p, x]
=
\frac1{\hbar}[p,x].
\end{align*}
Wir erinnern an die Vertauschungsrelationen f"ur die Operatoren $p$ und $x$:
\begin{align*}
\left[\frac{\hbar}{i}\frac{\partial}{\partial x}, x\right]\psi(x)
&=
\frac{\hbar}{i}\frac{\partial}{\partial x}(x\psi(x))
-x\frac{\hbar}{i}\frac{\partial}{\partial x}\psi(x)
=
\frac{\hbar}{i}\psi(x)
x\frac{\hbar}{i}\frac{\partial \psi(x)}{\partial x}
-x\frac{\hbar}{i}\frac{\partial}{\partial x}\psi(x)
=\frac{\hbar}{i}\psi(x)
\\
[p,x]&=\frac{\hbar}{i}\operatorname{id}.
\end{align*}
Damit sind jetzt auch die Vertauschungsoperatoren f"ur $P$ und $Q$ 
bekannt:
\[
[P,Q]=\frac1{i}\operatorname{id}=-i\operatorname{id}.
\]

\section{Auf- und Absteigeoperatoren}
\rhead{Auf- und Absteigeoperatoren}
Die Methode der Auf- und Absteige-Operatoren erlaubt, die Eigenwerte
und Eigenvektoren direkt aus dem Grundzustand zu ermitteln.
Wir sehen sie hier in einem einfachen Spezialfall am Werk.
Sie l"asst sich aber verallgemeinern, man kann damit zum Beispiel
auch den Drehimpuls verstehen (Kapitel~\ref{chapter:drehimpuls}).
Ja die Methode hat sogar g"anzlich von der Quantenmechanik unabh"angige
Anwendungen bei der Konstruktion von L"osungen allgemeinerer Randwertprobleme
f"ur gew"ohnliche Differentialgleichung.

\subsection{Definition}
Wir definieren die Operatoren
\[
a=\frac1{\sqrt{2}}(Q+iP)
\qquad\text{und}\qquad
a^+=\frac1{\sqrt{2}}(Q-iP).
\]
Zun"achst ist klar, dass $a$ und $a^+$ nicht selbstadjungiert sind,
vielmehr ist
\[
a^*=\frac1{\sqrt{2}}(Q^*-iP^*)=\frac1{\sqrt{2}}(Q-iP)=a^+.
\]
Damit entsprechen die Operatoren $a$ und $a^+$ nicht einer physikalisch
messbaren Gr"osse. 

Unser Ziel ist die Berechnung der Eigenwerte und Eigenvektoren
des harmonischen Oszillators mit algebraischen Mitteln. Dazu brauchen
wir die algebraischen Eigenschaften der Operatoren $a$ und $a^+$,
insbesondere deren Produkte und Vertauschungsrelationen.
\begin{align*}
aa^+
&=
\frac1{\sqrt{2}}(Q+iP)\frac1{\sqrt{2}}(Q-iP)
=
\frac12(Q^2+P^2+i[P,Q])
=
\hat{\cal H}+\frac12
\\
a^+a
&=
\frac1{\sqrt{2}}(Q-iP)\frac1{\sqrt{2}}(Q+iP)
=
\frac12(Q^2+P^2+i[Q,P])
=
\hat{\cal H}-\frac12=:N
\end{align*}
Daraus folgt auch
\[
[a,a^+]=\operatorname{id},
\]
und die weiteren Vertauschungsrelationen
\begin{align*}
[N,a^+]
&=a^+aa^+-aa^+a^+=[a,a^+]a^+=a^+
&
[N,a]
&=a^+aa-aa^+a=[a^+,a]a=-a
\\
&=[\hat{\cal H},a^+]
&
&=[\hat{\cal H}, a].
\end{align*}

\subsection{Wirkung auf Zustandsvektoren}
Nehmen wir an, $|n\rangle$ sei der $n$-te Eigenvektor von $\hat{\cal H}$,
mit Energie $\varepsilon_n$, also
\[
\hat{\cal H}|n\rangle = \varepsilon_n|n\rangle.
\]
Wenden wir die Operatoren $a$ und $a^+$ auf $|n\rangle$ an, erhalten
wir einen neuen Eigenvektor:
\begin{align*}
\hat{\cal H}a^+\,|n\rangle
&=
([\hat{\cal H}, a^+] + a^+\hat{\cal H})\,|n\rangle
=
(1 + \varepsilon_n)a^+\,|n\rangle
\\
\hat{\cal H}a\,|n\rangle
&=
([\hat{\cal H}, a] + a\hat{\cal H})\,|n\rangle
=
(\varepsilon_n - 1)a\,|n\rangle
\end{align*}
Insbesondere ist $a^+\,|n\rangle$ ein Eigenvektor mit Energie
$E_n+1$, und $a|n\rangle$ ein Eigenvektor mit Energie $\varepsilon_n-1$.
Man kann also mit dem Operator $a^+$ zu Zust"anden h"oherer Energie
aufsteigen, und mit $a$ zu Zust"anden niedrigerer Energie absteigen.

Beim Auf- und Absteigen bleibt die Normierung aber nicht notwendigerweise
erhalten, wir k"onnen also nicht davon ausgehen, dass $a^+|n\rangle$
wieder ein normierter Eigenvektor ist.
Um die Norm zu korrigieren, berechen wir die Norm des neuen Vektors:
\[
(a^+\langle n|) a^+\,|n\rangle
=
\langle n|\,aa^+\,|n\rangle
=
\langle n|\,\hat{\cal H}+{\textstyle\frac12}\,|n\rangle
=
(\varepsilon_n+{\textstyle\frac12})\langle n|n\rangle
=
\varepsilon_n+\textstyle\frac12
\]
Wenn man also durch Aufsteigen wieder einen normierten Zustandsvektor
erhalten will, muss man den erhaltenen Vektor renormieren:
\begin{equation}
|n+1\rangle = \frac1{\sqrt{\varepsilon_n+\textstyle\frac12}}|n\rangle.
\label{skript:aufsteigrenormierung}
\end{equation}

\subsection{Eigenwerte und Zustandsvektoren}
Die Energie eines harmonischen Operators ist immer positiv,
also kann man mit $a$ nicht beliebig lange absteigen. Es muss einen
Zustand $|0\rangle$ kleinster Energie geben. M"ochte man weiter
absteigen, darf kein Zustandsvektor mehr entstehen, es muss also
$a|0\rangle=0$ sein. Durch Multiplikation mit $a^+$ folgt
\[
0=a^+a|0\rangle=(\hat{\cal H}-\textstyle\frac12)\,|0\rangle
\quad
\Rightarrow
\quad
\hat{\cal H}\,|0\rangle=\frac12\,|0\rangle
\quad
\Rightarrow
\quad
\varepsilon_0=\frac12.
\]
Der Grundzustand hat also immer den Eigenwert $\frac12$, oder die
Energie $\frac12\hbar\omega$. Diesen Wert haben wir auch schon in
Abschnitt~\ref{skript:hogrundzustand} erhalten.
Die angeregten Zust"ande haben Energie
\[
E_n
=
\hbar\omega\biggl(n+\frac12\biggr),
\]
wobei $n\ge 0$ eine nat"urliche Zahl sein muss.

Beim Aufsteigen ver"andert sich jeweils die Normierung.
Will man die Energieeigenzust"ande so konstruieren, muss man nach jedem
Aufsteigeschritt die Normierung mit Hilfe von (\ref{skript:aufsteigrenormierung})
korrigieren. Dabei kann man verwenden, dass $\varepsilon_n = n+\frac12$
ist. Nach $n$-maligem Aufsteigen aus dem Grundzustand hat man die
Norm um den Faktor
\begin{align}
N_n
&=
\frac12\cdot\biggl(1+\frac12\biggr)\cdot\biggl(2+\frac12\biggr)\cdot\ldots\cdot
\biggl(n+\frac12\biggr)
=
\frac12\cdot\frac32\cdot\frac52\cdot\ldots\cdot\frac{2n+1}2
\notag
\\
&=
\frac1{2^n}
\frac{1\cdot 2\cdot 3\cdot 4\cdot 5\cdot \ldots \cdot (2n+1)\cdot 2n}%
{2\cdot 4\cdot 6\cdot \ldots \cdot 2n}
=\frac1{2^{2n}}\frac{2n!}{n!}
\label{skript:aufsteigrenormierungn}
\end{align}
ver"andert.

Der Aufsteigeoperator $a^+$ erm"oglicht jetzt auch, aus dem Grundzustand
alle h"oheren Zust"ande zu konstruieren:
\begin{equation}
|n\rangle
=
\frac{1}{\sqrt{N_n}}(a^+)^n\,|0\rangle
=
2^n \sqrt{\frac{n!}{2n!}} (a^+)^n\,|0\rangle,
\end{equation}
wobei $N_n$ der Normierungsfaktor aus (\ref{skript:aufsteigrenormierungn}) ist.

Da wir den Grundzustand bereits aus Abschnitt~\ref{skript:hogrundzustand} kennen,
k"onnen wir die Wellenfunktionen f"ur die h"oheren Zust"ande berechnen,
indem wir den Operator $a^+$ wieder in den urspr"unglichen Koordinaten
ausdr"ucken:
\begin{equation}
a^+=\frac1{\sqrt{2}}(Q+iP)
=
\frac1{\sqrt{2}}\biggl(
\sqrt{\frac{m\omega}{\hbar}}x
+i
\frac{1}{\sqrt{\hbar m\omega}}\frac{\hbar}{i}\frac{\partial}{\partial x}
\biggr)
=
\frac1{\sqrt{2}}
\biggl(
\sqrt{\frac{m\omega}{\hbar}}x
+
\sqrt{\frac{\hbar}{m\omega}}\frac{\partial}{\partial x}
\biggr)
\label{skript:aufsteigeho}
\end{equation}
Die Wellenfunktionen des Zustands $|n\rangle$ erh"alt man also, indem
man den Operator $a^+$ in der Form \ref{skript:aufsteigeho} auf die Wellenfunktion
$\psi_0$ wie in (\ref{skript:grundzustandwellenfunktion}) anwendet.

Wir fassen die Resultate "uber den harmonischen Oszillator in einem Satz
zusammen:
\begin{satz}
Ein harmonischer Oszillator mit Hamilton-Operator
\[
\frac{1}{2m}p^2+\frac{m}{2}\omega^2x^2
=
-\frac{\hbar^2}{2m}\frac{\partial^2}{\partial x^2}
+\frac{m}2\omega^2x^2
\]
hat Eigenzust"ande mit Energie
\begin{equation}
E_n
=
\hbar\omega\biggl(n+\frac12\biggr),
\label{skript:hoenergieniveaus}
\end{equation}
die Eigenzust"ande k"onnen gefunden werden durch Anwendung des
Aufsteigeoperators $a^+$
aus (\ref{skript:aufsteigeho}) auf die Wellenfunktion des Grundzustandes
$\psi(x)=e^{-\frac{m\omega}{2\hbar}x^2}$
mit geeigneter Normierung.
\end{satz}

\section{Klassischer Grenzwert}
\begin{figure}
\centering
\includegraphics{graphics/harm-1.pdf}
\caption{Wellenfunktion $|\psi_n(x)|^2$ des quantenmechanischen 
harmonischen Oszillators (rot) und Aufenthaltswahrscheinlichkeit des
klassischen harmonischen Oszillators (blau) im klassisch erlaubten
Gebiet.
\label{skript:harmklass}}
\end{figure}
F"ur grosse Energie m"ussen die Wellenfunktionen den klassischen
harmonischen Oszillator approximieren.
Ein harmonischer Oszillator mit Bahnkurve $x(t)=\cos t$ hat in einem
Interval $\Delta x$ eine Verweildaur von ungef"ahr
\[
\frac{\Delta x}{|\dot x(t)|}=\frac{\Delta x}{|\sin t|}
=\frac{\Delta x}{\sqrt{1-\cos^2t}}
\]
Als Funktion von $x$ ausgedr"uckt ist die Wahrscheinlichkeitsdichte,
den Oszillator im Punkt $x$ zu finden proportional zu
\[
\frac{1}{\sqrt{1-x^2}}
\]
In Abbildung~\ref{skript:harmklass} ist die die Aufenthaltswahrscheinlichkeit
im klassisch erlaubten Gebiet eines quantenmechanischen Oszillators f"ur
das Energieniveau $n=100$ dargestellt zusammen mit der durch die
Wellenfunktion gegebenen Wahrscheinlichkeitsdichte $|\psi_n(x)|^2$.

% XXX Anwendung auf Molek"ulschwingungen
%\section{Anwendung auf Molek"ulschwingungen}

\section*{"Ubungsaufgaben}
\rhead{"Ubungsaufgaben}
\begin{uebungsaufgaben}
\item
\input uebungsaufgaben/08001.tex
\item
\input uebungsaufgaben/08002.tex
\end{uebungsaufgaben}


\chapter{Wasserstoffatom}
\lhead{Wasserstoffatom}
\rhead{}
Der bisher entwickelte Formalismus erlaubt bereits, das Wasserstoff-Atom
zu verstehen. Die Quantenmechanik erkl"art die Stabilit"at der
Materie, und sagt das Wasserstoff-Spektrum korrekt voraus.

\chapter{St"orungstheorie}
\lhead{St"orungstheorie}
\rhead{}
Die St"orungstheorie geht davon aus, dass die Zust"ande eines
quantenmechanischen Systems bereits bekannt sind. Eine kleine
"Anderung des Systems sollte sich dann in nur kleinen "Anderungen
der Zust"ande und Ihrer Energieniveaus "aussern.
Diese Methode sollte anwendbar sein um die Ver"anderung der Energieniveaus
eines Atoms zu berechnen, wenn dieses in ein "ausseres elektrisches Feld
oder ein "ausseres Magnetfeld gebracht wird.


\section{Grundprinzip der St"orungstheorie}
\rhead{Grundprinzip}
\index{stoerungstheorie@St\"orungstheorie!Prinzip}
Wir beginnen die Untersuchung mit einer Anwendung auf die Nullstellen
eines quadratischen Polynoms. Sei also
\[
p(x) = ax^2 + bx + c
\]
ein quadratisches Polynom, nat"urlich k"onnen wir sofort die Nullstellen
mit Hilfe der L"osungsformel
\begin{equation}
x_{\pm}=\frac{-b\pm\sqrt{b^2-4ac}}{2a}
\label{mitternachtsformel}
\end{equation}
finden.  Ver"andert man die Koeffizienten von $p(x)$, werden sich auch
die beiden Nullstellen ver"andern, sie k"onnen mit der Formel
(\ref{mitternachtsformel}) bestimmt werden.

Nehmen wir also an, dass die Koeffizienten $a$, $b$ und $c$ von $p(x)$
von einem Parameter $\varepsilon$ abh"angen. Statt der Konstanten verwenden
wir also Funktionen
\begin{align*}
a(\varepsilon)&=a_0+a_1\varepsilon+a_2\varepsilon^2+\dots\\
b(\varepsilon)&=b_0+b_1\varepsilon+b_2\varepsilon^2+\dots\\
c(\varepsilon)&=c_0+c_1\varepsilon+c_2\varepsilon^2+\dots
\end{align*}
von $\varepsilon$. Mit diesen Koeffizienten wird aus dem Polynom $p(x)$
eine Familie von Polynomen
\[
p_\varepsilon(x)=a(\varepsilon)x^2 + b(\varepsilon)x+c(\varepsilon).
\]
Die L"osungsformel (\ref{mitternachtsformel}) liefert weiterhin die
Nullstellen von $p_{\varepsilon}(x)=0$.
Sei $x_0$ eine der Nullstellen von $p_0(x) = 0$.

Statt die L"osungen von $p_{\varepsilon}(x)=0$ mit Hilfe der L"osungsformel
zu finden, k"onnen wir auch versuchen, von $x_0$ ausgehend eine N"aherung
zu finden. Diese L"osung wird von $\varepsilon$ abh"angen, wir setzen
die Abh"angigkeit als Potenzreihe
\begin{equation}
x_\varepsilon = x_0 + x_1\varepsilon +x_2 \varepsilon^2+\dots
\label{stoerungsansatz}
\end{equation}
an.
Wir nennen die Potenzreihe (\ref{stoerungsansatz}) f"ur die L"osung
von $p_\varepsilon(x)=0$ die St"orungsreihe f"ur $x_\varepsilon$.
Setzen wir (\ref{stoerungsansatz}) in die Gleichung $p_{\varepsilon}(x)=0$
ein, ergibt sich die Gleichung
\begin{align*}
0&=a(\varepsilon)x_\varepsilon^2+b(\varepsilon)x_\varepsilon+c(\varepsilon)
\\
&=
(a_0+a_1\varepsilon+a_2\varepsilon^2+\dots)
(x_0+x_1\varepsilon+x_2\varepsilon^2+\dots)^2\\
&\qquad
+
(b_0+b_1\varepsilon+b_2\varepsilon^2+\dots)
(x_0+x_1\varepsilon+x_2\varepsilon^2+\dots)\\
&\qquad
+
(c_0+c_1\varepsilon+c_2\varepsilon^2+\dots)
\\
&=
(a_0+a_1\varepsilon+a_2\varepsilon^2+\dots)
(x_0^2+2x_0x_1\varepsilon +x_1^2\varepsilon^2+2x_0x_2\varepsilon^2+\dots)
\\
&\qquad
+
b_0x_0 + (b_0x_1+b_1x_0)\varepsilon+(b_0x_2+b_1x_1+b_2x_0)\varepsilon+\dots\\
&\qquad
+
c_0+c_1\varepsilon+c_2\varepsilon^2+\dots
\\
&=
a_0x_0^2 + (2a_0x_0x_1 + a_1x_0^2)\varepsilon +
(a_2x_0^2 + 2a_1x_0x_1 + a_0x_1^2 +2a_0x_0x_2)\varepsilon^2+\dots
\\
&\qquad
+
b_0x_0 + (b_0x_1+b_1x_0)\varepsilon+(b_0x_2+b_1x_1+b_2x_0)\varepsilon^2+\dots\\
&\qquad
+
c_0+c_1\varepsilon+c_2\varepsilon^2+\dots
\\
&=
a_0x_0^2 + b_0x_0+c_0
+(2a_0x_0x_1+a_1x_0^2+b_0x_1+b_1x_0+c_1)\varepsilon
\\
&\qquad
+(
a_2x_0^2 + 2a_1x_0x_1 + a_0x_1^2 +2a_0x_0x_2
+b_0x_2+b_1x_1+b_2x_0
+c_2
)\varepsilon^2+\dots
\end{align*}
Der konstante Term in der letzten Form dieser Gleichung ist $p_0(x_0)$, 
da $x_0$ eine Nullstelle ist, muss er verschwinden. Wenn die Gleichung
f"ur alle $\varepsilon$ erf"ullt sein soll, dann muss auch der
Koeffizient von $\varepsilon$ verschwinden, es muss also gelten
\begin{align*}
2a_0x_0x_1+a_1x_0^2+b_0x_1+b_1x_0+c_1&= 0
\\
(2a_0x_0+b_0)x_1&=-(a_1x_0^2+b_1x_0+c_1)\\
x_1&=-
\frac{a_1x_0^2+b_1x_0+c_1}{2a_0x_0+b_0}.
\end{align*}
Damit ist eine erste Approximation von $x_\varepsilon$ gefunden:
\[
x_\varepsilon\simeq x_0 -
\frac{a_1x_0^2+b_1x_0+c_1}{2a_0x_0+b_0}\varepsilon.
\]
Man nennt dies die St"orungsapproximation erster Ordnung (in $\varepsilon$).

Man kann die Approximation noch verbessern, indem man auch noch den
Koeffizienten von $\varepsilon^2$ in $x_\varepsilon$ berechnet.
Dazu verwendet man, dass in der Gleichung auch der Koeffizient von
$\varepsilon^2$ verschwinden muss:
\begin{align*}
a_2x_0^2 + 2a_1x_0x_1 + a_0x_1^2 +2a_0x_0x_2
+b_0x_2+b_1x_1+b_2x_0
+c_2
&=0
\\
(2a_0x_0+b_0)x_2
+a_2x_0^2 + 2a_1x_0x_1 + a_0x_1^2
+b_1x_1+b_2x_0
+c_2&=0
\end{align*}
Daraus kann man schliessen, dass
\[
x_2=-\frac{
a_2x_0^2 + 2a_1x_0x_1 + a_0x_1^2
+b_1x_1+b_2x_0
+c_2
}{2a_0x_0+b_0}.
\]
Mit den Koeffizienten $x_1$ und $x_2$ erh"alt man die St"orungsapproximation
zweiter Ordnungn.

Die St"orungsapproximation scheint also zu funktionieren, solange
der Nenner $2a_0x_0+b_0$ gross ist. Dann werden n"amlich die 
Koeffizienten $x_1$ und $x_2$ klein sein. Problematisch wird
die Approximation in dem Falle, wo $2a_0x_0+b_0$ verschwindet.
Setzt man allerdings die L"osungsformel (\ref{mitternachtsformel}) f"ur
$x_0$ ein, erh"alt man
\begin{align*}
2a_0x_0+b_0&=2a_0\frac{-b_0\pm\sqrt{b_0^2-4a_0c_0}}{2a_0}+b_0      \\
           &=\pm\sqrt{b_0^2-4a_0c_0}.
\end{align*}
Der Radikand ist die Diskriminante.
Der Nenner verschwindet also genau dann, wenn die quadratische
Gleichung eine doppelte Nullstelle hat. In diesem Fall spaltet sich
die eine doppelte L"osung in zwei verschiedene L"osungen $x_{\pm}$ auf,
es kann also keine Funktion der Form
(\ref{stoerungsansatz}) 
geben, die diese Aufspaltung wiedergeben kann.

\begin{figure}
\centering
\includegraphics[width=0.55\hsize]{graphics/pert-1.pdf}
\caption{St"orungsl"osung f"ur die Abh"angigkeit der L"osung einer
quadratischen Gleichung $a(\varepsilon)x^2+b(\varepsilon)x+c(\varepsilon)=0$
in Abh"angigkeit von $\varepsilon$,
mit $a(\varepsilon)=1+\varepsilon$, $b(\varepsilon)=\varepsilon$ und
$c(\varepsilon)=-1+\varepsilon$. Exakte L"osung {\color{red} rot}, 
St"orungsapproximation erster Ordnung {\color{blue} blau}.
\label{stoerungsloesung}}
\end{figure}

Wir fassen zusammen, was wir hier als L"osungsmethode gefunden haben:
\begin{enumerate}
\item Gegeben ist eine Gleichung $F_\varepsilon(x)=0$, gesucht
sind die L"osungen $x_\varepsilon$ der Gleichung in Abh"angigkeit von
$\varepsilon$.
\item Die L"osung $x_0$ der Gleichung f"ur den Parameterwert $\varepsilon=0$
ist bekannt.
\item Setze die L"osung in Form einer Potenzreihe an:
$x_\varepsilon = x_0+x_1\varepsilon+x_2\varepsilon^2+\dots$
\item Setze den L"osungsansatz in die Gleichung $F_\varepsilon(x)=0$ ein
und ermittle die Koeffizienten $x_1,x_2,\dots$ durch Koeffizientenvergleich.
\end{enumerate}
Schritt~4 ist eventuell nicht direkt durchf"uhrbar, wenn die Funktion
$F_\varepsilon(x)$ zu kompliziert ist. Man kann aber immer versuchen, die
Funktion als Potenzreihe in $x$ auszudr"ucken, 
\[
F_\varepsilon(x)=F_\varepsilon(x_0) + DF_\varepsilon(x_0)\cdot x
+ \frac12 D^2F_\varepsilon(x_0)\cdot(x,x)+\dots,
\]
und dann die Koeffizienten dieser Potenzreihe als Potenzreihe in $\varepsilon$.
So entsteht auf jeden Fall eine Vektorgleichung, deren Komponenten nur
Polynome in $\varepsilon$ sind, f"ur die sich Schritt~4 durchf"uhren l"asst.

\section{St"orungstheorie in der Quantenmechanik}
\rhead{St"orungstheorie in der Quantenmechanik}
Es sei $H_0$ der Hamilton-Operator eines quantenmechanischen Systems
mit den Eigenvektoren $\psi_0,\psi_1,\dots$ und Energien $E_0,E_1,\dots$.
Es gilt also 
\[
H_0\psi_i = E_i\psi_i.
\]
Der Hamilton-Operator $H_0$ wird nun durch einen zus"atzlichen Term
$H_1$ ver"andert. Wir gehen davon aus, dass $H_1$ im Vergleich zu
$H_0$ nur schwache Effekte beschreibt. Der neue Operator ist
\[
H(\varepsilon)=H_0+\varepsilon H_1.
\]
Darin ist $\varepsilon\ll 1$ ein kleiner Parameter, der ausdr"ucken
soll, dass $H_1$ den Operator nur ganz wenig "andert, insbesondere
sind die Zust"ande und Eigenwerte von $E(\varepsilon)$ nahe bei 
den $E_i$ und $\psi_i$.

Die Aufgabe der St"orungstheorie ist, die Zust"ande $\psi_i(\varepsilon)$
und Energien $E_i(\varepsilon)$ mit
\[
H(\varepsilon)\psi_i(\varepsilon)=E_i(\varepsilon)\psi_i(\varepsilon)
\]
mindestens n"aherungsweise zu bestimmen.

Die Eigenvektoren bilden eine Basis des Hilbertraumes $L^2(\mathbb R)$, 
also m"ussen sich auch die modifizierten Eigenvektoren in dieser Basis
ausdr"ucken lassen:
\[
\psi_i(\varepsilon)=\psi_i +\sum_{j=0}^\infty a_{ij}(\varepsilon)\psi_j,
\]
wobei die Koeffizienten $a_{ij}(\varepsilon)$ klein sind.
Das Problem ist gel"ost, wenn die $a_{ij}(\varepsilon)$ bestimmt sind.

\begin{beispiel}
Als Beispiel verwenden wir ein Teilchen in einem Potentialtopf, welches
wir fr"uher bereits untersucht haben. Der Hamilton-Operator wird jetzt
gest"ort um einen zus"atzlichen Potentialterm $H_1=V_1$ im Inneren
des Potentialtopfes.
W"are der zus"atzliche Term eine Konstante w"urde sich an den Eigenfunktionen
gar nichts "andern, und f"ur Eigenwerte w"urde man finden
\[
H(\varepsilon)\psi_i=(E_i + \varepsilon V_1)\psi_i,
\]
es w"are also
\begin{align*}
\psi_i(\varepsilon)&=\psi_i\\
E_i(\varepsilon)&=E_i+\varepsilon V_i
\end{align*}
Nun ist aber $V_i$ ortsabh"angig, nur im Inneren des Potentialtopfes 
unterscheidet sich $H(\varepsilon)$ von $H_0$. Da sich das Teilchen
aber praktisch nie ausserhalb des Potentialtopfes befindet, sollte
sich in erster N"aherung Eigenfunktion und Eigenwerte genau gleich
"andern.
\end{beispiel}



\section{Erste N"aherung}
\rhead{Erste N"aherung}


\section{Nicht entartete Zust"ande}
\rhead{Nicht entartete Zust"ande}

\section{Entartung}
\rhead{Entartung}

\section{St"orungsreihe}
\rhead{St"orungsreihe}

\section{"Ubungsaufgaben}
\rhead{"Ubungsaufgaben}
\begin{uebungsaufgaben}
\item
\input uebungsaufgaben/10001.tex
\end{uebungsaufgaben}

\chapter{Magnetfeld}
\lhead{Magnetfeld}
\rhead{}
\index{Lorentz-Kraft}
\index{Magnetfeld}
Elektronen interagieren auch mit Magnetfeldern. Die Lorentz-Kraft
steht aber immer senkrecht auf der Bahn eines Teilchens, insbesondere
leistet sie keine Arbeit, und kann daher auch nicht in die Energie
eingehen.
Es braucht daher eine grundlegend andere Beschreibung des Magnetfeldes
und der Hamilton-Funktion, um Magentfelder in den Formalismus
integrieren zu k"onnen.


\chapter{Drehimpuls\label{chapter:drehimpuls}}
\lhead{Drehimpuls}
\rhead{}


\chapter{Spin}
\lhead{Spin}
\rhead{}
Ein Elektron verh"alt sich nicht ausschliesslich wie ein punktf"ormiges
Teilchen.
Experimente zeigen, dass es auch mit einem Magnetfeld wechselwirken
kann, wie wenn es ein magnetisches Moment h"atte.

\chapter{Festk"orper\label{chapter:festkoerper}}
\lhead{Festk"orper}
\rhead{}
Festk"orper sind aus vielen Atomen zusammengesetzte Quantensysteme.
Im Gegensatz zu einer Fl"ussigkeit sind jedoch die Atomkerne mehr
oder weniger unverr"uckbar in einem Gitter angeordnet.
Bei elektrischen Leitern ist ein Teil der Elektronen weitgehend frei
beweglich.
Nat"urlich ist die Gitterstruktur ebenfalls die Folge der Interaktionen
zwischen den Elektronen, doch f"ur unsere Untersuchungen k"onnen wir
die Entstehung der Gitterstruktur vernachl"assigen, es einfach als
gegeben ansehen, und nur noch die Eigenschaften der frei beweglichen
Elektronen studieren.
Diese Elektronen bewegen sich in dem periodischen Potential, das von
den Atomkernen des Gitters erzeugt wird.
Jeder Atomkern erzeugt einen Potentialtopf, in dem eines oder mehrere
Elektronen gefangen sind.
Bleiben nach der Besetzung dieser lokalisierten Zust"ande noch
Elektronen "ubrig, dann haben diese mehr Energie als die Schwellen zwischen
den Potentialt"opfen hoch sind, aber nicht gen"ugend Energie, um
aus dem Festk"orper auszutreten.
Diese wie auch die Elektronen, die mit nur wenig Anregungsenergie
in einen ausgedehnten Zustand versetzt werden k"onnen, interessieren
uns in diesem Kapitel.

\section{Fermikugel}
In erster N"aherung betrachen wir den Festk"orper als einen
sehr grossen Potenialkasten, aus dem die Elektronen nicht austreten
k"onnen.
Die gebundenen Elektronen vernachl"assigen wir, sie tragen zu
Leitungsph"anomenen nichts bei.

\section{Kristalle}
\subsection{Symmetrien}
Kristalle sind Festk"orper, deren Atomkerne in einem regelm"assigen
Gitter angeordnet sind.
Im einfachsten Fall gibt es eine Menge 
$ \Gamma  \subset \mathbb R^3 $
so, dass eine Verschiebung aller Atomkerne um einen Vektor aus $\Gamma$
den Festk"orper nicht "andert.
Wir k"onnen die Verschiebung um einen Vektor $\vec v\in\Gamma$ als
Operator $T_{\vec v}$ auf dem Hilbertraum der Zustandsvektoren
betrachten:
\[
(T_{\vec v}\psi)(x)=\psi(x-\vec v).
\]
Der Hamilton-Operator "andert sich nicht, wenn man die Atomkerne um
einen Vektor $\vec v\in\Gamma$ verschiebt.
Also muss der Hamilton-Operator mit allen Translationen $T_{\vec v}$ 
vertauschen.
Es gibt also eine Basis von Eigenvektoren von $H$, welche auch
Eigenvektoren aller Operatoren $T_{\vec v}$ sind, f"ur jeden Vektor
$\vec v\in\Gamma$.

Die Menge $\Gamma$ hat noch etwas mehr Struktur als wir bisher
verwendet haben.
Wenn zwei Vektoren $\vec v_1,\vec v_2\in\Gamma$ sind, dann muss
auch deren Summe $\vec v_1+\vec v_3\in\Gamma$ sein.
Man nennt $\Gamma$ eine Gruppe.

Ausserdem ist es nicht n"otig, sich auf die Translationen $\Gamma$
zu beschr"anken.
Der Festk"orper kann durchaus noch weitere Symmetrien haben, zum
Beispiel Drehungen oder Spiegelungen.
Auch diese Symmetrien k"onnen als Operatoren beschrieben
werden, die auf den Zustandsvektoren wirken.
Die Menge aller Symmetrietransformationen nennen wir $G$.
Die Menge $G$ bildet ebenfalls eine Gruppe: jeder Verkn"upfung von
Symmetrieoperationen ist wieder eine Symmetrieoperation, und die
Inverse einer Symmetrieoperation ist ebenfalls eine Symmetrieoperation.
Die m"oglichen Symmetriegruppen von Kristallen sind vollst"andig
klassifiziert worden.

Eine Transformation darf die Norm der Zustandsvektoren nicht "andern,
die Transformationen in $G$ sind also unit"are Operatoren,
$U^*U=UU^*=\operatorname{id}$ f"ur alle $U\in G$.
Die Eigenwerte von Transformationen in $G$ sind daher alle in $U(1)$.
Die Transformationen einer Eigenfunktion von $H$ und allen Operatoren in $G$
"andert eine Wellenfunktion als h"ochstens um einen Phasenfaktor.
Zu jedem Operator $U\in G$ gibt es also einen Eigenwert $\chi(U)\in U(1)$,
und die Zusammensetzung von Operatoren muss mit der Abbildung $\chi$
vertauschen, d.~h.
\[
\begin{aligned}
\chi(UV)&=\chi(U)\chi(V)&&U,V\in G\\
\chi(U^*)&=\bar\chi(U)&&U\in G
\end{aligned}
\]
Man nennt eine solche Abbildung einen Darstellung der Gruppe $G$.
Auch die Darstellungen der Gruppen $G$ sind vollst"andig klassifiziert
worden.

\subsection{Periodisches Potential}
Als Beispiel betrachten wir einen eindimensionalen Kristall, dessen
Atomkerne an den Stellen $an$, $n\in\mathbb Z$ platziert sind.
$a$ ist der Abstand zwischen den Atomkernen.
Statt Coulomb-Potentials verwenden wir ein schmalen Potentialtopf.
Dieser Kristall hat einerseits die Translationssymmetrie $T_{an}$ um
ganzzahlige Vielfache von $a$, und andererseits eine Spiegelung $S$.
Die Translationen und die Verschiebungen vertauschen nicht, es gilt
\[
T_{na}S=ST_{-na}.
\]
Da die Spiegelung $S^2=\operatorname{id}$ erf"ullt, kann sie nur
Eigenwerte $1$ oder $-1$ haben.
Wenden wir darauf $\chi$ an, finden wir
\[
\chi(a)^n\chi(S)=\chi(S)\bar\chi(a)^n
\qquad \Rightarrow \qquad
\chi(a)^2=1
\]

Wir suchen also Zustandsvektoren, welche Eigenwerte der
Symmetrieoperatoren sind.

\section{Valenz- und Leitungsband}




\begin{appendices}
\chapter{Anhang: Komplexe Zahlen}
\lhead{Komplexe Zahlen}
\rhead{}
Der mathematische Formalismus der Quantenmechanik kann nicht ohne komplexe
Zahlen auskommen.
Leonhard Euler sah die Zahlen $\sqrt{-1}$ noch als imagin"ar an,
also als ohne Gegenst"uck in der realen Welt.
Elektroingenieure verwenden komplexe Zahlen mit grossem Erfolg in ihren
Anwendungen, sie spielen aber vor allem die Rolle eines praktischen
Werkzeugs. Die Regeln, mit denen am Schluss solcher Rechnungen sichergestellt
wird, dass die Resultate reell sind zeigt ausserdem, dass man alles auch
ohne komplexe Zahlen durchrechnen k"onnte, wenn auch wesentlich weniger
elegant.

In der Quantenmechanik geht es aber nicht mehr ohne komplexe Zahlen,
die physikalischen Gr"ossen selbst sind komplex. Es gibt zwar auch
hier wieder Regeln, die sicherstellen, dass Messresultate reell sind
(Operatoren m"ussen selbstadjungiert sein), aber sie erlauben nicht,
die ganze Quantenmechanik auf eine Art zu beschreiben, die ohne komplexe
Zahlen auskommt.

\section{Der K"orper $\mathbb C$ der komplexen Zahlen}
\rhead{Der K"orper $\mathbb C$}
In den reellen Zahlen $\mathbb R$ k"onnen alle Grundoperationen ausgef"uhrt
werden, es ist jedoch nicht m"oglich, die Quadratwurzeln aus negativen
Zahlen zu ziehen. Eine analoge Situation trifft man schon viel fr"uher.
In den nat"urlichen Zahlen $\mathbb N$ kann man zwar addieren und
multiplizieren, aber nicht subtrahieren.
F"ugt man die negativen Zahlen hinzu erh"alt man eine Menge $\mathbb Z$,
in der die Subtraktion uneingeschr"ankt m"oglich ist. Division ist aber
immer noch nur f"ur spezielle Divisoren m"oglich. F"ugt man jedoch die
Br"uche zu $\mathbb Z$ hinzu, erh"alt man die Menge der rationalen Zahlen
$\mathbb Q$, in der Division uneingeschr"ankt m"oglich ist.
Doch auch $\mathbb Q$ ist nicht vollst"andig, die Zahl $\sqrt{2}$ ist
keine rationale Zahl. Nat"urlich kann man $\sqrt{2}$ durch eine
Folge von Br"uchen $r_n\in\mathbb Q$ approximieren, doch der Grenzwert
dieser Folge $\lim_{n\to\infty}r_n=\sqrt{2}$ ist nicht in $\mathbb Q$.
F"ugt man jedoch alle Grenzwerte von konvergenten Folgen zu $\mathbb Q$
hinzu, erh"alt man die Menge $\mathbb R$ der reellen Zahlen, in der
auch beliebige Wurzeln von positiven Zahlen gezogen werden k"onnen,
oder andere Grenzwerte wie $\pi$, $e$, die Werte von $\sin x$ und $\cos x$
und weitere.

\subsection{Grundoperationen f"ur die komplexen Zahlen}
Nach analogem Muster k"onnen wir auch $\mathbb R$ erweitern, so dass auch
die Wurzeln aus negativen Zahlen bestimmt werden k"onnen. Es reicht
sogar, nur die Wurzel von $-1$ hinzuzuf"ugen, denn jede andere Wurzel
einer negativen Zahl ist $\sqrt{-a}=\sqrt{-1}\cdot\sqrt{\mathstrut a}$.
Euler hat die Bezeichnung $i=\sqrt{-1}$ f"ur die imagin"are Einheit eingef"uhrt.
Es gilt nat"urlich $i^2=-1$.

\begin{definition}
Die Menge $\mathbb C=\{a+bi\,|\,a,b\in\mathbb R\}$ heisst die Menge der
komplexen Zahlen. Die komplexe Zahl $z=a+bi$ hat
Realteil $\operatorname{Re}z=a$ und Imagin"arteil $\operatorname{Im}z=b$.
Die Rechenoperationen sind so zu verstehen, dass die Rechenregeln
der Algebra erhalten bleiben\footnote{Man nennt dies das Permanenz-Prinzip}.
\end{definition}

Die Rechenoperationen folgen aus der Definition:
\begin{align*}
(a+bi)+(c+di)&=(a+c)+(b+d)i\\
(a+bi)(c+di)&=ac-i^2bd+(ad+bc)i=ac-bd+(ad+bc)i
\end{align*}
Die Division stellt noch ein Problem dar. Hier hilft das Konzept der
konjugiert komplexen Zahl.

\begin{definition}
Die Zahl $\bar z=a-bi$ heisst die zu $z=a+bi$ konjugiert komplexe Zahl.
\end{definition}

Zun"achst kann man mit der konjugiert komplexen Zahl den Betrag einer
komplexen Zahl definieren:
\[
z\bar z=(a+bi)(a-bi)=a^2+abi-abi-i^2b^2=a^2+b^2\qquad\Rightarrow\qquad
|z|^2=z\bar z.
\]
Andererseits kann man damit auch komplexe Br"uche berechnen, indem man
mit der konjugiert komplexen Zahl des Nenners erweitert:
\begin{align*}
\frac{a+bi}{c+di}&=
\frac{a+bi}{c+di}
\cdot
\frac{c-di}{c-di}=\frac{ac-bd+(ad+bd)i}{c^2+d^2}
\end{align*}
Die komplexen Zahlen k"onnen in einer Ebene visualisert werden: 
Realteil und Imagin"arteil werden entlang orthogonaler Achsen abgetragen.
Die Punkte $(x,y)$ der $x$-$y$-Ebene entsprechen also der komplexen Zahl
$x+yi$ der komplexen Zahlenebene (Abbildung~\ref{skript:gaussebene}).
\begin{figure}
\centering
\includegraphics{graphics/komplex-1.pdf}
\caption{Komplexe Zahlenebene
\label{skript:gaussebene}}
\end{figure}

Mit Hilfe der komplexen Konjugation kann man den Real- and Imagin"arteil
einer komplexen Zahl $z=a+bi$ direkt ausdr"ucken:
\begin{align}
\operatorname{Re}z 
&=
a=\frac{(a+bi)+(a-bi)}2=\frac{z+\bar z}2
\label{skript:realteil-formel}
\\
\operatorname{Im}z
&=
b=\frac{(a+bi)-(a-bi)}{2i}=\frac{z-\bar z}{2i}.
\label{skript:imaginaerteil-formel}
\end{align}

\subsection{Polardarstellung}
Die Darstellung der komplexen Zahlen als Punkte einer Ebene suggeriert
auch eine alternative Schreibweise.
Ein Punkt $z$ der komplexen Ebene kann auch charakterisiert werden mit Hilfe von
Polarkoordinaten, also durch seine Entfernung $r=|z|$ vom Nullpunkt,
und durch Polarwinkel zwischen der reellen Achse und der Richtung
zur komplexen Zahl. Der Polarwinkel heisst auch Argument $\operatorname{arg}z$,
und es gilt
\[
\tan\operatorname{arg}z=\frac{\operatorname{Re}z}{\operatorname{Im}z}.
\]
Die Multiplikation von komplexen Zahlen bekommt in der Polardarstellung
eine besondere Interpretation:
\begin{align*}
z_1z_2
&=
(r_1\cos\varphi_1+ir_1\sin\varphi_1) (r_2\cos\varphi_2+ir_2\sin\varphi_2)
\\
&=
r_1r_2(\cos\varphi_1+i\sin\varphi_1) (\cos\varphi_2+i\sin\varphi_2)
\\
&=
r_1r_2\bigl(
\cos\varphi_1\cos\varphi_2-\sin\varphi_1\sin\varphi_2 +
(\cos\varphi_1\sin\varphi_2+\sin\varphi_1\cos\varphi_2)i\bigr)
\\
&=
r_1r_2(\cos(\varphi_1+\varphi_2)+i\sin(\varphi_1+\varphi_2))
\\
\Rightarrow \operatorname{arg}z_1z_2&=\arg z_1 + \arg z_2
\end{align*}
Die Multiplikation zweier komplexen Zahlen entspricht also der
Multiplikation der Betr"age, und der Addition der Argumente.

Wir versuchen jetzt, die Werte der Exponentialfunktion zu $e^z$ zu
bestimmen.
Die Exponentialgesetze sollten auch weiterhin gelten.
Sei also $z=a+bi$, dann ist
\[
e^z=e^{a+bi}=e^a\cdot e^{bi}.
\]
Die Exponentialfunktion reeller Zahlen ist bereits wohlbekannt, es muss
also nur noch untersucht werden, welche Bedeutung $e^{bi}$ hat.

Betrachten wir die Funktion $f(t)= e^{it}$. Die Ableitungen von $f$ sind
\begin{align}
f'(t)&=ie^{it}=if(t)\notag\\
f''(t)&=-f(t).\label{skript:exp-dgl}
\end{align}
Die Funktion $f$ muss also eine L"osung der Differentialgleichung
(\ref{skript:exp-dgl}) sein, welche die Anfangsbedingungen $f(0)=1$ und
$f'(0)=if(0)=i$ erf"ullen muss.
Doch die Differentialgleichung (\ref{skript:exp-dgl}) hat die L"osungen
\[
f(t)=a\cos t+b\sin t.
\]
Setzt man die Anfangsbedingungen ein, findet man
\begin{align*}
f(0)&=1&\Rightarrow&&a&=1\\
f'(0)&=1&\Rightarrow&&b&=i,
\end{align*}
so dass wir jetzt $e^{it}$ ausrechnen k"onnen:
\begin{satz}[Euler]
\begin{align}
e^{it}=\cos t+i\sin t.
\label{skript:euler-formula}
\end{align}
\end{satz}

Die komplexe Konjugation kehrt das Vorzeichen des Imagin"arteils, also 
von $\sin t$. Da $\sin t$ eine ungerade Funktion ist, ist dies gleichbedeuten
damit, das Vorzeichen von $t$ zu kehren: $\overline{e^{it}}=e^{-it}$.

Mit der Eulerschen Formel sind wir jetzt auch in der Lage, den Zusammenhang
zwischen einer komplexen Zahl und ihrem Betrag und Argument sehr pr"agnant
auszudr"ucken:
\[
z=|r|\cdot e^{i\operatorname{arg}z}.
\]
Die Real- und Imagin"arteile von $e^{it}$ sind $\cos t$ und $\sin t$,
wir k"onnen Sie auch mit den Formeln (\ref{skript:realteil-formel}) und
(\ref{skript:imaginaerteil-formel}) ausdr"ucken:
\begin{align*}
\cos t
&=
\operatorname{Re}e^{it}
=
\frac{e^{it}+\overline{e^{it}}}2
=
\frac{e^{it}+e^{-it}}2
\\
\sin t
&=
\operatorname{Im}e^{it}=\frac{e^{it}-e^{-it}}{2i}.
\end{align*}

\subsection{Matrixdarstellung der komplexen Zahlen\label{subsection:matrixdarstellung}}
Die Algebra der komplexen Zahlen kann man auch als eine Algebra von Matrizen
schreiben. Dazu betrachten wir die Abbildung
\[
\varphi\colon
\mathbb C\to M_2(\mathbb R):
a+bi\mapsto\begin{pmatrix}a&b\\-b&a\end{pmatrix}
\]
Die imagin"are Einheit $i$ wird von $\varphi$ auf die Matrix
\[
\varphi(i)=J=\begin{pmatrix}0&1\\-1&0\end{pmatrix}
\]
abgebildet. Man kann nachrechnen, dass $J^2=-E$, und dass die Rechenregeln
f"ur die komplexen Zahlen durch die Abbildung $\varphi$ in die Rechenregeln
f"ur Matrizen transformiert werden.
Wir illustrieren dies f"ur die Multiplikation:
\begin{align*}
&(a+bi)(c+di)&&\mapsto&
&\begin{pmatrix}a&b\\-b&a\end{pmatrix}
\begin{pmatrix}c&d\\-d&c\end{pmatrix}
\\
&(ac-bd) + i(ad-bc)&&\mapsto&
&=\begin{pmatrix}
ac-bd&ad-bc\\
-ad+bc&ac-bd
\end{pmatrix}
\end{align*}
In dieser Darstellung kann man auch $e^{Jt}$ ausrechnen, indem man $Jt$ in
die Taylorreihe von $e^x$ einsetzt
\begin{align}
e^{Jt}
&=
E + tJ + \frac{t^2}{2!} J^2 + \frac{t^3}{3!} J^3 + \frac{t^4}{4!} J^4
 + \frac{t^5}{5!} J^5 + \frac{t^6}{6!} J^6 + \frac{t^7}{7!} J^7 + \dots
\notag
\\
&=
E + tJ - \frac{t^2}{2!} E - \frac{t^3}{3!} J + \frac{t^4}{4!} E
 - \frac{t^5}{5!} J - \frac{t^6}{6!} E + \frac{t^7}{7!} J + \dots
\notag
\\
&=\biggl(1 - \frac{t^2}{2!} + \frac{t^4}{4!} - \frac{t^6}{6!} + \dots\biggr) E
+ \biggl(t - \frac{t^3}{3!} + \frac{t^5}{5!} - \frac{t^7}{7!} + \dots\biggr) J
= E \cos t + J \sin t.
\label{skript:eulermatrixdarstellung}
\end{align}
Die Eulerformel (\ref{skript:euler-formula}) l"asst sich also auch in der
Matrixdarstellung der komplexen Zahlen wiedergewinnen.

\section{Komplexe Matrizen}
\rhead{Komplexe Matrizen}
Die Lineare Algebra im Bachelor wird typischerweise nur in den
reellen Zahlen entwickelt, der einzige untersuchte Vektorraum ist
der Raum $\mathbb R^n$ der reellen $n$-dimensionalen Vektoren.
F"ur die Quantenmechanik ben"otigen wir aber auch Vektoren mit
komplexen Komponenten. Der $n$-dimensionale komplese Vektorraum
$\mathbb C^n$ ist die Menge
\[
\left\{\left.\begin{pmatrix}c_1\\\vdots\\c_n\end{pmatrix}\,\right|
c_i\in\mathbb C\right\},
\]
mit der komponentenweisen Addition und der Multiplikation mit einer
komplexen Zahl. Nichts an der elementaren linearen Algebra hat besondere
Eigenschaften der reellen Zahlen verwendet, die nicht auch die komplexen
Zahlen haben. Der Gauss-Algorithmus, die Konstruktion der Determinanten,
ja sogar der grundlegende Algorithmus zur L"osung des Eigenwertprobems
funktioniert genau gleich auch f"ur komplexe Matrizen. Nur im Bereich
des Skalarproduktes sind minimale Modifikationen notwendig, damit

\subsection{Skalarprodukt f"ur komplexe Vektorr"aume}
In der linearen Algebra im ersten Semester wird das Skalarprodukt
geometrisch mit Hilfe der Projektion eingef"uhrt.
Eine solche Konstruktion ist f"ur komplexe Vektoren nicht m"oglich,
weil es keine anschauliche komplexe Geometrie gibt.

\subsubsection{Komplexes Skalarprodukt}
Um ein komplexes Skalarprodukt zu bekommen, gehen wir daher von den
algebraischen Eingeschaften des Skalarproduktes aus:
\begin{compactenum}
\item Das Skalarprodukt $(u,v)$ von komplexen Vektoren $u$ und $v$ ist linear
in $v$.
\item $(u,u) > 0$ falls $u\ne 0$.
\item Falls $(u,v)\in\mathbb R$, dann ist $(u,v)=(v,u)$.
\end{compactenum}
Diese Eigenschaften m"ussen auch f"ur ein komplexes Skalarprodukt gelten.
Wir zeigen, dass diese Eigenschaften auch ein komplexes Skalarprodukt 
weitgehend festlegt.

Zun"achst stellen wir fest, dass wir nicht erwarten k"onnen, dass
ein Skalarprodukt linear sein kann.
Betrachten wir dazu einen eindimensionalen Vektorraum $\mathbb C^1$.
Vektoren sind hier nur komplexe Zahlen, und ein Produkt, welches
linear in beiden Faktoren ist, ist von der Form $(u,v)=uv$. Dann ist
aber $(u,u)=u^2$, aber $u^2$ kann auch negativ sein, zum Beispiel f"ur $u=i$.
Das einzige Produkt, welches immer positiv ist, ist $(u,v)=\bar uv$.
Dieses Produkt ist aber nicht linear im Faktor $u$:
\[
(\lambda u,v)=\overline{(\lambda u)}v=\bar\lambda (\bar uv)=\bar\lambda (u,v).
\]
Wir m"ussen also von einem komplexen Skalarprodukt verlangen, dass es
\index{konjugiert linear}
im ersten Faktor {\em konjugiert linear} ist:
\[
(\lambda u,v)=\bar\lambda(u,v)
\]
Eine Funktion von zwei Vektoren, welche linear im zweiten Vektor ist
und konjugiert linear im ersten heisst {\em sesquilinear}.
\index{sesquilinear}

Sei jetzt also $(\,\cdot\,|\,\cdot\,)$ eine Sesquilinearform.
Wir setzen $\lambda = 1/(u,v)$, dann gilt
$(u,\lambda v)$ reell ist. Dann gilt
\begin{align*}
1&=\lambda (u,v)=(u,\lambda v)=(\lambda v,u)=\bar\lambda(v,u)
&
&\Rightarrow&
\frac1{(\bar\lambda)}&=(v,u)
&
&\Rightarrow&
\overline{(u,v)}&=(v,u).
\end{align*}
Vertauschung der Faktoren ist also gleichbedeutend mit komplexer Konjugation
des Wertes des Skalarproduktes. Man nennt eine Funktion von zwei komplexen
Vektoren {\em hermitesch}, wenn $(u,v)=\overline{(v,u)}$ gilt.
\index{hermitesch}

Eine hermitesche Sesquilinearform heisst {\em positiv definit}, wenn
\index{positiv definit}
f"ur jeden Vektor $u\ne 0$ gilt $(u,u)>0$. Diese Eigenschaft stellt
sicher, dass $(u,u)$ sinnvoll als die ``L"ange'' eines Vektors interpretiert
werden kann.

\begin{definition}
Ein komplexes Skalarprodukt ist eine positiv definite,
hermitesche Sesquilinearform.
\end{definition}
\index{Skalarprodukt, komplexes}

Das einfachste Beispiel eines komplexen Skalarproduktes ist
\[
(u,v)=\sum_i \bar u_iv_i.
\]
Die Standardbasisvektoren sind auch in diesem Skalarprodukt
orthonormiert.

%
% Adjungierte Matrix
%
\subsubsection{Adjungierte Matrix}
\index{adjungierte Matrix}
In den reellen Vektorr"aumen konnte man zu einer linearen $A$ immer
eine lineare Abbildung $A^t$ finden mit der Eigenschaft
$(u,Av)=(A^tu,v)$. Mit Hilfe der Standardbasisvektoren konnte
man auch ausrechnen, was dies f"ur die Matrizen von $A$ bedeutet:
\[
(e_i,A^te_j),
=
(A^te_j, e_i)
=
(e_j,Ae_i)=a_{ij}
\]
d.~h.~die Matrix von $A^t$ ist die transponierte Matrix von $A$.
Eine symmetrische Matrix war eine, die sich beim Transponieren nicht
"andert, also $A^t=A$.

Dasselbe kann man jetzt auch f"ur ein komplexes Skalarprodukt
versuchen. Zu einer komplexen Matrix $A$ ist also eine neue
Matrix $A^*$ gesucht, mit der Eigenschaft, $(A^*u,v)=(u,Av)$ f"ur 
jedes Paar von Vektoren $u$ und $v$. F"ur die Standardbasisvektoren
gilt dann
\[
(e_i,A^*e_j),
=
\overline{(A^*e_j, e_i)}
=
\overline{(e_j,Ae_i)=a_{ij}},
\]
die Matrix von $A^*$ ist also nicht nur transponiert, sondern auch komplex
konjugiert. Hat $A$ die Matrixelement $a_{ij}$ dann nennt man
die Matrix $A^*$ mit den Matrixelementen $\bar a_{ji}$ die
adjungiert Matrix. Eine Matrix, die sich beim adjungieren nicht
"andert heisst selbstadjungiert.

Die Rechenregeln f"ur die adjungierte Matrix sind ganz "ahnlich wie
f"ur die Transposition:
\begin{align*}
(\lambda A)^*=\bar\lambda A^*
&
(A+B)^*&=A^*+B^*
&
(AB)^*=B^*A^*.
\end{align*}
Man beachte, dass $A\mapsto A^*$ nicht linear ist.

\subsubsection{Unit"are Matrizen}
\index{unitar@unit\"ar}
Matrizen, die in reellen Vektorr"aumen das Skalarprodukt nicht "andern,
heissen orthogonal. Sie sind charakterisiert durch die Eigenschaft
$(Ox,Oy)=(x,y)$, woraus sich mit Hilfe der Transposition ergibt:
\[
(Ox,Oy)=(O^tOx,y)=(x,y)\qquad\Rightarrow\qquad O^tO=E,
\]
woraus man weiter ablesen kann, dass bei orthogonalen Matrizen
die transponierte Matrix mit der invertierten Matrix zusammenf"allt.

F"ur komplexen Vekoren kann man wieder nach den Matrizen fragen, die das
komplexe Skalarprodukt nicht ver"andern. Eine Matrix $U$ hat diese
Eigenschaft, wenn
\[
(Ux,Uv)=(U^*Ux,y)=(x,y)\;\forall x,y
\qquad\Rightarrow\qquad
U^*U=E,
\]
eine solche Matrix heisst unit"ar. 

F"ur reelle Matrizen $A$ ist $A^t=A^*$, also sind orthogonale Matrizen
auch unit"ar.

\begin{beispiel}
Die unit"aren $1\times 1$-Matrizen sind komplexe Zahlen $z$, welche
die zus"atzliche Bedingung $\bar zz=1$ erf"ullen m"ussen.
Die Menge der unit"aren $1\times 1$-Matrizen ist also
\[
U(1)=\left\{ z\in\mathbb C\,|\, |z|=1
\right\}.
\]
In der Quantenmechanik k"onnen Zustandsvektoren in der Regel nur bis auf
einen komplesen Faktor vom Betrag $1$, also bis auf ein Element
von $U(1)$ festgelegt werden.
Man spricht oft von ``bedeutungslosen'' Phasenfaktoren.
\end{beispiel}

\subsubsection{Die spezielle unit"are Gruppe $\operatorname{SU}(2)$}
\index{spezielle unit\"are Gruppe}
\index{SU(2)}
Wir betrachten Matrizen der Form
\begin{equation}
U=
\begin{pmatrix}
a&b\\-\bar b&\bar a
\end{pmatrix}
\label{skript:su2form}
\end{equation}
mit der zus"atzlichen Bedingung $|a|^2 + |b|^2=1$. Sie erf"ullen
\begin{align*}
U^*U
&=
\begin{pmatrix}
\bar a&-b\\\bar b&a
\end{pmatrix}
\begin{pmatrix}
a&b\\-\bar b&\bar a
\end{pmatrix}
=
\begin{pmatrix}
\bar aa+b\bar b & \bar ab-b\bar a\\
\bar ba-a\bar b & \bar bb+a\bar a
\end{pmatrix}
=
E
\\
\det(U)&=\left|
\begin{matrix}
a&b\\-\bar b&\bar a
\end{matrix}
\right|
=
a\bar a+\bar bb = |a|^2 + |b|^2=1.
\end{align*}
Solche Matrizen sind also nicht nur unit"ar, sondern haben auch Determinante 1.
Multipliziert man zwei Matrizen der Form (\ref{skript:su2form}),
wird das Produkt auch wieder Determinante 1 haben, aber es ist nicht klar,
dass es sich in der Form (\ref{skript:su2form}) schreiben l"asst.
Daher rechnen wir das Produkt aus, wir erhalten
\begin{align*}
\begin{pmatrix}  a                        & b                      \\
                 -\bar b                  & \bar a                 \end{pmatrix}
\begin{pmatrix}  c                        & d                      \\
                 -\bar d                  & \bar c                 \end{pmatrix}
&=
\begin{pmatrix}  ac-b\bar d               & ab+b\bar c             \\
                 -\bar bc-\bar a\bar d    & -\bar bd +\bar a\bar c \end{pmatrix}
=
\begin{pmatrix}  ac-b\bar d               & ab+b\bar c             \\
                 -(\overline{ab+b\bar c}) & \overline{ac-b\bar d}  \end{pmatrix}
.
\end{align*}
Das Produkt zweier Matrizen der Form (\ref{skript:su2form}) ist also wieder eine
Matrix der Form (\ref{skript:su2form}).
Die Menge dieser Matrizen ist demzufolge abgeschlossen unter
Matrixmultiplikation und Bildung der Inversen.
Man nennt diese Matrizen die Gruppe der speziellen unit"aren Matrizen:
\[
\operatorname{SU}(2)=\left\{
\left.
\begin{pmatrix}
a&b\\-\bar b&\bar a
\end{pmatrix}
\,
\right|
\,
a,b\in\mathbb C\wedge
|a|^2+|b|^2=1
\right\}.
\]
Die Gruppe $\operatorname{SU}(2)$ spielt bei der Analyse des Elektronenspins
eine wichtige Rolle.

Die Gruppe $\operatorname{SU}(2)$ ist eine Teilmenge der Menge der komplexen
$2\times 2$-Matrizen:
\[
\operatorname{SU}(2)
\subset
V=
\left\{
\left.
\begin{pmatrix}a&b\\-\bar b&\bar a\end{pmatrix}\,
\right|
a,b\in\mathbb C^2
\right\}
\subset
M_2(\mathbb C)
=\left\{
\left.
\begin{pmatrix}
a&b\\c&d
\end{pmatrix}
\,
\right|
\, a,b,c,d\in\mathbb C
\right\}
\]
Wir untersuchen die Menge $V$ etwas genauer.
Zun"achst k"onnen wir $V$ als zweidimensionalen komplexen Vektorraum
betrachten wie $\mathbb C^2$.
Als reeller Vektorraum betrachtet ist $V$ ein vierdimensionaler
reeller Vektorraum. Die Abbildung
\[
\mathbb R^4\to\mathbb C^2\to V
:
\begin{pmatrix}x_1\\x_2\\x_3\\x_4\end{pmatrix}\mapsto
\begin{pmatrix}x_1+ix_2\\x_3+ix_4\end{pmatrix}\mapsto
\begin{pmatrix} x_1+ix_2 & x_3+ix_4 \\
               -x_3+ix_4 & x_1-ix_2 \end{pmatrix}
\]
ist eine Bijektion zwischen $\mathbb R^4$, $\mathbb C^2$ und $V$.
Auch geometrisch ist $\mathbb C^2=\mathbb C\times \mathbb C$ das Produkt
von zwei Ebenen, hat also eine vierdimensionale Geometrie.
Darin ist $\operatorname{SU}(2)$ die Teilmenge der vierdimensionalen Vektoren,
und zwar derjenigen f"ur die gilt
\[
1
=
|a|^2+|b|^2
= 
x_1^2 + x_2^2 + x_3^2 + x_4^2,
\]
das sind die Vektoren von $\mathbb R^4$ mit L"ange $1$.
Geometrisch ist $\operatorname{SU}(2)$ also eine dreidimensionale Kugel
eingebettet in in einen vierdimensionalen Raum.

\subsubsection{Spur und Determinante}
Spur und Determinaten k"onnen samt all ihren Rechenregeln sofort auf
komplexe Matrizen "ubertragen werden.
F"ur den Adjunktionsoperator finden wir die Rechenregeln
\begin{align*}
\det A^*&= \overline{\det A^t}=\overline{\det A},
&&\text{und}&
\operatorname{tr} A^*&=\overline{\operatorname{tr}A}.
\end{align*}
F"ur selbstadjungierte Matrizen kann man schliessen, dass sowohl
die Determinante wie auch die Spur von $A$ reell sein m"ussen.

In den komplexen Zahlen hat jedes Polynom $n$-ten Grades $n$ Nullstellen.
Daher k"onnen wir das charakteristische Polynom immer ausschreiben als
\begin{align*}
\det(A-\lambda E)
&=
(-1)^n (\lambda-\lambda_1)\dots(\lambda-\lambda_n)
\\
&=
(-1)^n(\lambda^n -(\lambda_1+\dots+\lambda_n)\lambda^{n-1}+\dots
+(-1)^n\lambda_1\dots\lambda_n
\\
&=
(-1)^n(\lambda^n - \operatorname{tr}A\lambda^{n-1}+\dots + (-1)^n\det A)
\end{align*}
Daraus k"onnen wir auch eine Formel f"ur $\det(E+tA)$ ableiten:
\begin{align}
\det(E+tA)
&=
(-t)^n\det\biggl(A-\frac1tE\biggr)
=
t^n\biggl(
\frac1{t^n}+\frac1{t^{n-1}}\operatorname{tr}A+\dots+\det A
\biggr)
\notag
\\
&=
1+t\operatorname{tr}A+\dots+t^n\operatorname{det}A
\label{skript:detandtrace}
\end{align}

%
% Eigenwertproblem fuer komplexe Matrizen
%
\subsection{Eigenwertproblem f"ur komplexe Matrizen}
\subsubsection{Definition}
Ein Vektor $v\ne 0$ heisst Eigenvektor zum Eigenwert $\lambda$ einer
Matrix $A$, wenn $Av=\lambda v$ gilt. Diese Definition ist auch f"ur
komplexe Matrizen g"ultig, ebenso funktioniert der Standardalgorithmus
f"ur die L"osung des Eigenwertproblems nach wie vor:
\begin{enumerate}
\item Finde die Nullstellen der charakteristischen Gleichung
$\det(A-\lambda E)=0$.
\item F"ur jede Nullstelle $\lambda_i$, finde Eigenvektoren
durch L"osung des Gleichungssystems $(A-\lambda_i E)v=0$.
\end{enumerate}
Der wesentliche Unterschied ist jedoch, dass in den komplexen
Zahlen ein Polynom vom Grade $n$ immer $n$ Nullstellen hat.
Die bei reellen Matrizen vorkommende Situation, dass weniger
als $n$ reelle Nullstellen existieren, und daher nicht gen"ugend
Eigenvektoren f"ur eine Eigenvektorbasis gefunden werden k"onnen,
tritt also bei komplexen Matrizen seltener auf.

\subsubsection{Selbstadjungierte Matrizen}
Der quantenmechanische Formalismus beschreibt physikalische Gr"ossen
als selbstadjungierte Matrizen. Die m"oglichen Werte einer solchen Gr"osse
sind die Eigenwerte der Matrix, und es muss sichergestellt werden,
dass keine komplexen Eigenwerte auftreten k"onnen.

\begin{satz}
\label{skript:ewreell}
Die Eigenwerte einer selbstadjungierten Matrix sind reell.
\end{satz}
\begin{proof}[Beweis]
Sei $v$ ein Eigenvektor zum Eigenwert $\lambda$ einer selbstadjungierten
Matrix $A$. Dann gilt
\begin{align*}
(v,Av)
&=
\lambda(v,v)
\\
&=(Av,v)=(\lambda v,v)=\bar\lambda(v,v)
\\
\Rightarrow \lambda&=\bar\lambda,
\end{align*}
also ist $\lambda\in\mathbb R$.
\end{proof}

Bei reellen Matrizen hat sich gezeigt, dass symmetrische Matrizen immer
diagonalisierbar sind. Dies gilt auch f"ur selbstadjungierte Matrizen
in komplexen Vektorr"aumen:

\begin{satz}
Eine selbstadjungierte Matrix ist diagonalisierbar und die Eigenvektoren zu
verschiedenen Eigenwerten sind orthogonal.
\end{satz}

\begin{proof}[Beweis]
Wir beweisen nur die Orthogonalit"at von Eigenvektoren zu verschiedenen 
Eigenwerten. Seien also $v_1,v_2$ Eigenvektoren zu zwei verschiedenen
Eigenwerten $\lambda_1,\lambda_2$. Die beiden Eigenwerte sind nach
\label{skript:ewreell} reell. Dann gilt
\begin{align*}
(v_1,Av_2)&=\lambda_2(v_1,v_2)
\\
          &=(Av_1,v_2)=\bar\lambda_1(v_1,v_2)=\lambda_1(v_1,v_2)
\\
\Rightarrow\quad
(\lambda_1-\lambda_2)(v_2,v_2)&=0
\end{align*}
Die letzte Gleichung kann wegen $\lambda_1\ne\lambda_2$ nur wahr sein,
wenn $(v_1,v_2)=0$, die Vektoren $v_1$ und $v_2$ m"ussen also orthogonal sein.
\end{proof}

%
% Lie Algebrean
%
\section{Lie-Algebren}
\rhead{Lie-Algebren}
\index{Lie-Algebra}
Die invertierbaren Matrizen $\operatorname{GL}(n)$ k"onnen weiter
unterteilt werden in invertierbare Matrizen mit zus"atzlichen
Eigenschaften wie Orthogonalit"at oder spezielle Werte der Determinanten.
Allerdings bildet die Menge der invertierbaren Matrizen keinen
Vektorraum: man kann sie nicht addieren oder mit beliebigen Zahlen
multiplizeren, weil die Invertierbarkeit oder eine der zus"atzlichen
Eigenschaften dabei verloren geht. Diese Matrizen sind also nicht
geeignet daher nicht geeignet als Observable in der Quantenmechanik.

Ist eine Matrix $A$ invertierbar, dann wird auch eine Matrix, deren
Matrixelemente nur wenig von $A$ abweichen, invertierbar sein.
Insbesondere werden die Matrizen der Form $E+tA$ invertierbar sein,
wenn nur $t$ klein genug ist. Dies ist ein "Ubergang zu einer 
infinitesimalen Transformation $A$.

Wie m"ussen die Rechenoperationen beim "Ubergang zu infinitesimalen
Transformationen "ubersetzt werden?
Wegen
\[
(E+tA)(E+tB)=E+(A+B)t + t^2AB\simeq E+(A+B)t
\]
wird beim "Ubergang zur infinitesimalen Transformation aus dem Produkt
der Matrizen die Summe von Matrizen, und aus der Inversen wird die Negation.

Betrachten wir die Bedingung Orthogonalit\"at
\[
E=
(E+\varepsilon A)^t(E+\varepsilon A)
=
E+\varepsilon (A^t+A) + \dots
\qquad
\Rightarrow
\qquad
A^t=-A.
\]
Die infinitesimale Versionen von orthogonalen Matrizen sind also
antisymmetrische Matrizen.
\index{antihermitsche Matrix}
Analog sind die infinitesimalen Versionen von unit"aren Matrizen
antihermitesch.

Die Matrizenmultiplikation ist kommutativ.
Diese Mengen tragen eine zus"atzliche algebraische Struktur. 
\index{Kommutator}
Der Kommutator $[A,B]=AB-BA$ zweier Matrizen $A$ und $B$ erh"alt
die oben genannten Eigenschaften. Ist $A$ antisymmetrisch, dann
ist auch $[A,B]$ auch antisymmetrisch:
\[
[A,B]^t
=
(AB-BA)^t
=
B^tA^t-A^tB^t
=
(-B)(-A)-(-A)(-B)
=
-(AB-BA)
=
-[A,B].
\]
\index{Jacobi-Identit\"at!f\"ur Operatoren}
Die Kommutatorklammer hat aber auch noch eine zus"atzliche Eigenschaft,
f"ur drei Matrixen $A$, $B$ und $C$ gilt n"amlich die sogenannte
Jacobi-Identit"at:
\begin{equation}
[A,[B, C]]
+
[B,[C, A]]
+
[C,[A, B]]
=
0
\label{skript:jacobi}
\end{equation}
Um dies einzusehen Rechnen wir die Kommutatoren aus:
\begin{align*}
&
[A,[B, C]]
+
[B,[C, A]]
+
[C,[A, B]]
\\
&=
A(BC-CB)-(BC-CB)A
+
B(CA-AC)-(CA-AC)B
+
C(AB-BA)-(AB-BA)C
\\
&=
ABC-ACB-BCA+CBA
+
BCA-BAC-CAB+ACB
+
CAB-CBA-ABC+BAC
\\
&=0
\end{align*}
Ein Vektorraum mit einer antisymmetrischen, bilinearen Abbildung
$[\;\cdot\;,\;\cdot\;]$,
welche die Jacobi-Identit"at (\ref{skript:jacobi}) erf"ullt, heisst eine
Lie-Algebra.
So wie der Hamilton-Operator als Erzeuger der Zeitentwicklung das
leichter zu manipulierende Objekt ist, ist die Liealgebra zu einer
Transformationsgruppe wie $\operatorname{SU}(2)$ besser geeignet
zur Beschreibung von Transformationen, und wird daher von Physikern
vorgezogen.

\begin{beispiel}
\index{SO(3)}
Zur Gruppe $\operatorname{SO}(3)$ der Drehmatrizen geh"ort die Lie-Algebra
$\operatorname{so}(3)$ der antisymmetrischen $3\times 3$-Matrizen.
Solche Matrizen haben die Form
\[
\Omega
=
\begin{pmatrix}
    0    & \omega_3&-\omega_2\\
-\omega_3&   0     & \omega_1\\
 \omega_2&-\omega_1&    0
\end{pmatrix}
\]
Der Vektorraum $\operatorname{so}(3)$ ist also dreidimensional.

Die Wirkung von $E+t\Omega$ auf einem Vektor $x$ ist
\[
(E+t\Omega)
\begin{pmatrix}x_1\\x_2\\x_3\end{pmatrix}
=
\begin{pmatrix}
    1     & t\omega_3&-t\omega_2\\
-t\omega_3&   1      & t\omega_1\\
 t\omega_2&-t\omega_1&    1
\end{pmatrix}
\begin{pmatrix}x_1\\x_2\\x_3\end{pmatrix}
=
\begin{pmatrix}
x_1-t(-\omega_3x_2+\omega_2x_3)\\
x_2-t( \omega_3x_1-\omega_1x_3)\\
x_3-t(-\omega_2x_1+\omega_1x_2)
\end{pmatrix}
=
x- t\begin{pmatrix}\omega_1\\\omega_2\\\omega_3\end{pmatrix}\times x
=
x+ tx\times \omega.
\]
Die Matrix $\Omega$ ist als die infinitesimale Version einer Drehung
um die Achse $\omega$.

Wir k"onnen die Analogie zwischen Matrizen in $\operatorname{so}(3)$ und
Vektoren in $\mathbb R^3$ noch etwas weiter treiben. Zu jedem Vektor
in $\mathbb R^3$ konstruieren wir eine Matrix in $\operatorname{so}(3)$
mit Hilfe der Abbildung
\[
\mathbb R^3\to\operatorname{so}(3)
:
\begin{pmatrix}v_1\\v_2\\v_3\end{pmatrix}
\mapsto
\begin{pmatrix}
  0 & v_3&-v_1\\
-v_3&  0 & v_2\\
 v_1&-v_2&  0
\end{pmatrix}.
\]
Der Kommutator von zwei so aus Vektoren $\vec u$ und $\vec v$
konstruierten Matrizen $U$ und $V$ ist:
\begin{align*}
[U,V]
&=
UV-VU
\\
&=
\begin{pmatrix}
  0 & u_3&-u_1\\
-u_3&  0 & u_2\\
 u_1&-u_2&  0
\end{pmatrix}
\begin{pmatrix}
  0 & v_3&-v_1\\
-v_3&  0 & v_2\\
 v_1&-v_2&  0
\end{pmatrix}
-
\begin{pmatrix}
  0 & v_3&-v_1\\
-v_3&  0 & v_2\\
 v_1&-v_2&  0
\end{pmatrix}
\begin{pmatrix}
  0 & u_3&-u_1\\
-u_3&  0 & u_2\\
 u_1&-u_2&  0
\end{pmatrix}
\\
&=
\begin{pmatrix}
u_3v_3+u_1v_1 - u_3v_3 - u_1v_1
	& u_1v_2 - u_2v_1
		& u_3v_2 - u_2v_3 
\\
u_2v_1 - u_1v_2
	& -u_3v_3-u_2v_2 + u_3v_3+u_2v_2
		& u_3v_1 - u_1v_3
\\
u_2v_3 - u_3v_2         
	& u_1v_3 - u_3v_1
		&-u_1v_1-u_2v_2 u_1v_1+u_2v_2
\end{pmatrix}
\\
&=
\begin{pmatrix}
0
	& u_1v_2 - u_2v_1
		&-(u_2v_3-u_3v_2)
\\
-( u_1v_2 - u_2v_1)
	& 0
		& u_3v_1 - u_1v_3
\\
u_2v_3 - u_3v_2         
	&-( u_3v_1 - u_1v_3)
		& 0
\end{pmatrix}
\end{align*}
Die Matrix $[U,V]$ geh"ort zum Vektor $\vec u\times\vec v$.
Damit k"onnen wir aus der Jacobi-Identit"at jetzt folgern, dass
\[
\vec u\times(\vec v\times w)
+
\vec v\times(\vec w\times u)
+
\vec w\times(\vec u\times v)
=0
\]
f"ur drei beliebige Vektoren $\vec u$, $\vec v$ und $\vec w$ ist.
Dies bedeutet, dass der dreidimensionale Vektorraum $\mathbb R^3$
mit dem Vektorprodukt zu einer Lie-Algebra wird.
In der Tat verwenden einige B"ucher statt der vertrauten Notation
$\vec u\times \vec v$ f"ur das Vektorprodukt die aus der Theorie der
Lie-Algebren entlehnte Notation $[\vec u,\vec v]$, zum Beispiel
das Lehrbuch der Theoretischen Physik \cite{skript:landaulifschitz1}
von Landau und Lifschitz.

Die Lie-Algebren sind vollst"andig klassifiziert worden, es gibt
keine nicht trivialen zweidimensionalen Lie-Algebren.
Unser dreidimensionaler ist also auch in dieser Hinsicht speziell:
es ist der kleinste Vektorraum, in dem eine nichttriviale Lie-Algebra-Struktur
m"oglich ist.
\end{beispiel}

\begin{beispiel}
Die Gruppe $\operatorname{SU}(2)$ hat als infinitesimale Erzeugende 
die antihermiteschen Matrizen, also Matrizen mit der Eigenschaft
\[
A^*=-A.
\qquad
\Rightarrow
\qquad
\begin{pmatrix}
   ia&b+ic\\
-b+ic&  id
\end{pmatrix}
\]
Darin wurde aber die Bedingung noch nicht abgebildet, dass Matrizen
in $\operatorname{SU}(2)$ die Determinante $1$ haben m"ussen, wir
haben bis jetzt nur Unitarit"at verwendet. Wir m"ussen also noch
verlangen, dass in erster N"aherung $\det(E+tA)=0$ ist.
Aus Formel (\ref{skript:detandtrace}) schliessen wir, dass die Spur
$\operatorname{tr}A$ verschwindet, oder dass $a=-d$:
\begin{equation}
\operatorname{su}(2)
=
\left\{ A\in M_2(\mathbb C)\,|\,
A^*=-A\wedge \operatorname{tr}A=0
\right\}
\end{equation}
Matrizen in $\operatorname{su}(2)$ haben also die Form
\begin{equation}
\begin{pmatrix}
   ia&b+ic\\
-b+ic& -ia
\end{pmatrix},\qquad a,b,c\in\mathbb R
\end{equation}
Die Menge $\operatorname{su}(2)$ ist also eine dreidimensionale 
Lie-Algebra.
Als Basis von $\operatorname{su}(2)$ k"onnen die Matrizen
\begin{align}
I
&=
\begin{pmatrix} 0&1 \\ -1& 0 \end{pmatrix},
&
J
&=
\begin{pmatrix} 0&i \\  i& 0 \end{pmatrix},
&
K
&=
\begin{pmatrix} i&0 \\  0&-i \end{pmatrix}
\label{skript:komplex:definitionIJK}
\end{align}
verwendet werden.
Die Matrix $I$ kennen wir schon von der Matrixdarstellung der komplexen
Zahlen in Abschnitt~\ref{subsection:matrixdarstellung}.
Die Matrizen $I$, $J$ und $K$ haben die folgenden algebraischen
Eigenschaften:
\begin{align*}
I^2
&=
-E
&
IJ
&=
\begin{pmatrix} 0&1 \\ -1& 0 \end{pmatrix}
\begin{pmatrix} 0&i \\  i& 0 \end{pmatrix}
=
\begin{pmatrix} i&0 \\  0&-i \end{pmatrix}
=
K
&
JI
&=
-K
\\
J^2
&=
-E
&
JK
&=
\begin{pmatrix} 0&i \\  i& 0 \end{pmatrix}
\begin{pmatrix} i&0 \\  0&-i \end{pmatrix}
=
\begin{pmatrix} 0&1 \\ -1& 0 \end{pmatrix}
=
I
&
KJ
&=
-I
\\
K^2
&
-E
&
KI
&=
\begin{pmatrix} i&0 \\  0&-i \end{pmatrix}
\begin{pmatrix} 0&1 \\ -1& 0 \end{pmatrix}
=
\begin{pmatrix} 0&i \\  i& 0 \end{pmatrix}
=
J
&
IK
&=
-J
\end{align*}
\index{Quaternionen}
Dies ist die Algebra der Quaternionen.
\end{beispiel}

Eigentlich h"atten wir die Bedingung, dass die Spur verschwinden muss,
schon bei den orthogonalen Matrizen fordern m"ussen.
Doch antisymmetrische Matrizen haben $0$ auf der Diagonalen, also haben
antisymmetrische Matrizen immer Spur $0$, die Bedingung ist also
automatisch erf"ullt.
Antihermitesche Matrizen haben jedoch nicht nur $0$ auf der Diagonalen,
sondern rein imagin"are Zahlen.

\section*{"Ubungsaufgaben}
\rhead{"Ubungsaufgaben}
\begin{uebungsaufgaben}
\item
\input uebungsaufgaben/15001.tex
\item
\input uebungsaufgaben/15002.tex
\item
\input uebungsaufgaben/15003.tex
\item
\input uebungsaufgaben/15004.tex
\item
\input uebungsaufgaben/15005.tex
\item
\input uebungsaufgaben/15006.tex
\end{uebungsaufgaben}


\chapter{Kugelkoordinaten\label{chapter:kugelkoordinaten}}
\lhead{Kugelkoordinaten}
\rhead{}
In den Quantisierungsregeln wird festgelegt, wie Impulskomponenten
in kartesischen Koordinaten durch Differentialgoperatoren zu
ersetzen seien.
Allerdings tr"agt diese Wahl des Koordinatensystems der Kugelsymmetrie
des Problems nicht Rechnung. 
Wenn man aber die angemessenen Kugelkoordinaten verwenden will,
dann muss man in der Lage sein, jeden beliebigen Differentialoperator
durch Ableitungen nach Kugelkoordinaten auszudr"ucken.

\section{Differentialoperatoren in Kugelkoordinaten}
\rhead{Differentialoperatoren}
Wir erinnern an die Formeln f"ur Kugelkoordinaten:
\begin{align*}
x&=
r\sin\vartheta\cos\varphi
\\
y&=
r\sin\vartheta\sin\varphi
\\
z&=
r\cos\vartheta
\end{align*}
Ein Differentialoperator ist eine Linearkombination der Operatoren
\[
\frac{\partial}{\partial x},
\quad
\frac{\partial}{\partial y},
\quad\text{und}\quad
\frac{\partial}{\partial z},
\]
oder auch h"oherer Ableitungen.  Unser Ziel ist, einen solchen Operator
durch die Ableitungen nach den Kugelkoordinaten auszudr"ucken, also
als Linearkombination von
\[
\frac{\partial}{\partial r},
\quad
\frac{\partial}{\partial \vartheta},
\quad\text{und}\quad
\frac{\partial}{\partial \varphi}.
\]
Offenbar reicht es dazu, jeden einzelnen Ableitungsoperator nach kartesischen
Koordiaten in den Ableitungsoperatoren nach Kugelkoordinaten auszudr"ucken.
Denn es gilt
\begin{align*}
\frac{\partial}{\partial x}
&=
\frac{\partial r}{\partial x} \frac{\partial}{\partial r}
+
\frac{\partial \vartheta}{\partial x} \frac{\partial}{\partial \vartheta}
+
\frac{\partial \varphi}{\partial x} \frac{\partial}{\partial \varphi}
\\
\frac{\partial}{\partial y}
&=
\frac{\partial r}{\partial y} \frac{\partial}{\partial r}
+
\frac{\partial \vartheta}{\partial y} \frac{\partial}{\partial \vartheta}
+
\frac{\partial \varphi}{\partial y} \frac{\partial}{\partial \varphi}
\\
\frac{\partial}{\partial z}
&=
\frac{\partial r}{\partial z} \frac{\partial}{\partial r}
+
\frac{\partial \vartheta}{\partial z} \frac{\partial}{\partial \vartheta}
+
\frac{\partial \varphi}{\partial z} \frac{\partial}{\partial \varphi}
\end{align*}
Um also einen Differentialoperator in Kugelkoordinaten ausdr"ucken zu
k"onnen, m"ussen wir als erstes alle Kugelkoordinaten nach kartesischen
Koordinaten ableiten k"onnen.

\section{Ableitung von Kugelkoordinaten nach kartesischen Koordinaten}
\rhead{Ableitungen nach kartesischen Koordinaten}
Wir beginnen mit den Ableitungen von $r$. Dazu rechnen wir zun"achst die
Ableitungen von $r^2$ aus:
\[
\frac{\partial r^2}{\partial x}
=
2r\frac{\partial r}{\partial x}
\qquad
\Rightarrow
\qquad
\frac{\partial r}{\partial x}
=
\frac1{2r}\frac{\partial r^2}{\partial x}.
\]
Der Ausdruck  $r^2$ ist quadratisch in den Koordinaten, und ist damit
viel einfacher auszurechnen:
\[
\frac{\partial r^2}{\partial x}=\frac{\partial}{\partial x}(x^2+y^2+z^2)=2x
\]
Damit k"onnen wir jetzt die Ableitungen von $r$ nach den kartesischen
Koordinaten zusammenstellen
\begin{equation}
\begin{aligned}
\frac{\partial r}{\partial x}
&=
\frac1{2r}2x=\frac{x}{r}=\sin\vartheta\cos\varphi
\\
\frac{\partial r}{\partial y}
&=
\frac1{2r}2y=\frac{y}{r}=\sin\vartheta\sin\varphi
\\
\frac{\partial r}{\partial z}
&=
\frac1{2r}2z=\cos\vartheta
\end{aligned}
\label{skript:ableitungenvonr}
\end{equation}
F"ur die Ableitungen von $\vartheta$ dr"ucken wir $z$ durch Kugelkoordinaten
aus, leiten nach den kartesischen Koordinaten ab und l"osen nach den
Ableitungen von $\varphi$ nach den kartesischen Koordinaten auf:
\begin{align*}
z&=r\cos\vartheta
&&\Rightarrow&
\frac{\partial z}{\partial z}
&=
\frac{\partial r}{\partial z}\cos\vartheta
-
r \sin\vartheta\frac{\partial \vartheta}{\partial z}
=1
&&\Rightarrow&
\frac{\partial\vartheta}{\partial z}
&=
-\frac1{r\sin\vartheta}
\biggl(1-\cos\vartheta\frac{\partial r}{\partial z}\biggr)
\\
&&&&
\frac{\partial z}{\partial x}
&=
\frac{\partial r}{\partial x}\cos\vartheta
	- r\sin\vartheta\frac{\partial\vartheta}{\partial x}
=0
&&\Rightarrow&
\frac{\partial\vartheta}{\partial x}
&=
\frac{\cos\vartheta}{r\sin\vartheta}\frac{\partial r}{\partial x}
\\
&&&&
\frac{\partial z}{\partial y}
&=
\frac{\partial r}{\partial y}\cos\vartheta
	- r\sin\vartheta\frac{\partial\vartheta}{\partial y}
=0
&&\Rightarrow&
\frac{\partial\vartheta}{\partial y}
&=
\frac{\cos\vartheta}{r\sin\vartheta}\frac{\partial r}{\partial y}
\end{align*}
Die Ableitungn von $r$ wurden in (\ref{skript:ableitungenvonr}) bereits berechnet,
und k"onnen hier eingesetzt werden:
\begin{equation}
\begin{aligned}
\frac{\partial\vartheta}{\partial z}
&=
-\frac{1}{r\sin\vartheta}(1-\cos^2\vartheta)
=
-\frac{1}{r\sin\vartheta}\sin^2\vartheta
=
-\frac{\sin\vartheta}{r}
\\
\frac{\partial \vartheta}{\partial x}
&=
\frac{\cos\vartheta}{r\sin\vartheta}\sin\vartheta\cos\varphi
=
\frac{\cos\vartheta\cos\varphi}{r}
\\
\frac{\partial \vartheta}{\partial y}
&=
\frac{\cos\vartheta}{r\sin\vartheta}\sin\vartheta\sin\varphi
=
\frac{\cos\vartheta\sin\varphi}{r}
\end{aligned}
\label{skript:ableitungenvonvartheta}
\end{equation}
Wir brauchen noch die Ableitungen von $\varphi$, zun"achst nach $x$:
\begin{align*}
1=
\frac{\partial x}{\partial x}
&=
\frac{\partial r}{\partial x}\sin\vartheta\cos\varphi
+
r\cos\vartheta \frac{\partial\vartheta}{\partial x}\cos\varphi
-
r\sin\vartheta\sin\varphi\frac{\partial\varphi}{\partial x}
\\
&=
\sin\vartheta \cos\varphi
\sin\vartheta\cos\varphi
+
r\cos\vartheta
\frac{\cos\vartheta \cos\varphi}{r}
\cos\varphi
-
r\sin\vartheta\sin\varphi
\frac{\partial\varphi}{\partial x}
\\
&=\cos^2\varphi-r\sin\vartheta\sin\varphi\frac{\partial\varphi}{\partial x}
\\
1-\cos^2\varphi
&=
\sin^2\varphi=
-r\sin\vartheta\sin\varphi\frac{\partial\varphi}{\partial x}
&\Rightarrow\quad
\frac{\partial\varphi}{\partial x}
&=-\frac{\sin\varphi}{r\sin\vartheta}
\\
0=
\frac{\partial y}{\partial x}
&=
\frac{\partial r}{\partial x} \sin\vartheta\sin\varphi
+
r\cos\vartheta \frac{\partial\vartheta}{\partial x}\sin\varphi
+
r\sin\vartheta\cos\varphi\frac{\partial\varphi}{\partial x}
\\
&=
\sin\vartheta\cos\varphi
\sin\vartheta\sin\varphi
+
r\cos\vartheta
\frac{\cos\vartheta\cos\varphi}{r}
\sin\varphi
+
r\sin\vartheta\cos\varphi\frac{\partial\varphi}{\partial x}
\\
&=
\cos\varphi\sin\varphi
+
r\sin\vartheta\cos\varphi
\frac{\partial\varphi}{\partial x}
&\Rightarrow\quad
\frac{\partial\varphi}{\partial x}
&=
-\frac{\sin\varphi}{r\sin\vartheta}
\end{align*}
Die Ableitung von $\varphi$ nach $y$:
\begin{align*}
0=\frac{\partial x}{\partial y}
&=
\frac{\partial r}{\partial y}
\sin\vartheta\cos\varphi
+
r\cos\vartheta
\frac{\partial\vartheta}{\partial y}
\cos\varphi
-
r\sin\vartheta\sin\varphi
\frac{\partial\varphi}{\partial y}
\\
&=
\sin\vartheta\sin\varphi
\sin\vartheta\cos\varphi
+
r\cos\vartheta
\frac{\cos\vartheta\sin\varphi}{r}
\cos\varphi
-
r\sin\vartheta\sin\varphi
\frac{\partial\varphi}{\partial y}
\\
&=
\sin^2\vartheta \sin\varphi \cos\varphi
+
\cos^2\vartheta \sin\varphi \cos\varphi
-
r\sin\vartheta\sin\varphi
\frac{\partial\varphi}{\partial y}
\\
&=
\sin\varphi \cos\varphi
-
r\sin\vartheta\sin\varphi
\frac{\partial\varphi}{\partial y}
&\Rightarrow\quad
\frac{\partial\varphi}{\partial y}
&=
\frac{\cos\varphi}{r\sin\vartheta}
\\
1=\frac{\partial y}{\partial y}
&=
\frac{\partial r}{\partial y}
\sin\vartheta \sin\varphi
+
r\cos\vartheta
\frac{\partial\vartheta}{\partial y}
\sin\varphi
+
r\sin\vartheta\cos\varphi
\frac{\partial\varphi}{\partial y}
\\
&=
\sin\vartheta \sin\varphi
\sin\vartheta \sin\varphi
+
r\cos\vartheta
\frac{\cos\vartheta\sin\varphi}{r}
\sin\varphi
+
r\sin\vartheta\cos\varphi
\frac{\partial\varphi}{\partial y}
\\
&=
\sin^2\vartheta \sin^2\varphi
+
\cos^2\vartheta
\sin^2\varphi
+
r\sin\vartheta\cos\varphi
\frac{\partial\varphi}{\partial y}
\\
1-
\sin^2\varphi
&=
r\sin\vartheta\cos\varphi
\frac{\partial\varphi}{\partial y}
&
\Rightarrow\quad
\frac{\partial\varphi}{\partial y}
&=
\frac{\cos\varphi}{r\sin\vartheta}
\end{align*}
Ableitungen von $\varphi$ nach $z$:
\begin{align*}
0=\frac{\partial x}{\partial z}
&=
\frac{\partial r}{\partial z}
\sin\vartheta\cos\varphi
+
r\cos\vartheta
\frac{\partial\vartheta}{\partial z}
\cos\varphi
-
r\sin\vartheta\sin\varphi
\frac{\partial\varphi}{\partial z}
\\
&=
\cos\vartheta
\sin\vartheta\cos\varphi
-
r\cos\vartheta
\frac{\sin\vartheta}{r}
\cos\varphi
-
r\sin\vartheta\sin\varphi
\frac{\partial\varphi}{\partial z}
\\
&=
-
r\sin\vartheta\sin\varphi
\frac{\partial\varphi}{\partial z}
&\Rightarrow\quad
\frac{\partial\varphi}{\partial z}
&=0
\\
0=\frac{\partial y}{\partial z}
&=
\frac{\partial r}{\partial z}
\sin\vartheta\cos\varphi
+
r\cos\vartheta
\frac{\partial\vartheta}{\partial z}
\cos\varphi
+
r\sin\vartheta\cos\varphi
\frac{\partial\varphi}{\partial z}
\\
&=
\cos\vartheta
\sin\vartheta\cos\varphi
-
r\cos\vartheta
\frac{\sin\vartheta}{r}
\cos\varphi
+
r\sin\vartheta\cos\varphi
\frac{\partial\varphi}{\partial z}
\\
&=
r\sin\vartheta\cos\varphi
\frac{\partial\varphi}{\partial z}
&\Rightarrow\quad
\frac{\partial\varphi}{\partial z}
&=0
\end{align*}
Wir haben alle Ableitungen der Kugelkoordinaten nach kartesischen
Koordinaten berechnet, und stellen die Resultate in der folgenden
Tabelle zusammen
\begin{center}
\begin{tabular}{|>{$}c<{$}|>{$}c<{$} >{$}c<{$} >{$}c<{$}|}
\hline
&r&\vartheta&\varphi\\
\hline
\displaystyle\frac{\partial^{\mathstrut}}{\partial x_{\mathstrut}}
	&\sin\vartheta\cos\varphi
		&\displaystyle\frac{\cos\vartheta\cos\varphi}{r}
			&\displaystyle-\frac{\sin\varphi}{r\sin\vartheta}
\\
\displaystyle\frac{\partial^{\mathstrut}}{\partial y_{\mathstrut}}
	&\sin\vartheta\sin\varphi
		&\displaystyle\frac{\cos\vartheta\sin\varphi}{r}
			&\displaystyle\frac{\cos\varphi}{r\sin\vartheta}
\\
\displaystyle\frac{\partial^{\mathstrut}}{\partial z_{\mathstrut}}
	&\cos\varphi
		&\displaystyle-\frac{\sin\vartheta}{r}
			&0
\\
\hline
\end{tabular}
\end{center}

Damit k"onnen wir jetzt die Ableitungen nach den kartesischen Koordinaten
vollst"andig durch die Ableitungen nach Kugelkoordinaten ausdr"ucken:
\begin{equation}
\begin{linsys}{4}
\displaystyle
\frac{\partial^{\mathstrut}}{\partial x_{\mathstrut}}
&=&
\displaystyle
\frac{\partial r}{\partial x}\frac{\partial}{\partial r}
+
\frac{\partial \vartheta}{\partial x}\frac{\partial}{\partial \vartheta}
+
\frac{\partial \varphi}{\partial x}\frac{\partial}{\partial \varphi}
&=&
\displaystyle
\sin\vartheta\cos\varphi
\frac{\partial}{\partial r}
&+&
\displaystyle
\frac{\cos\vartheta\cos\varphi}{r}
\frac{\partial}{\partial\vartheta}
&-&
\displaystyle
\frac{\sin\varphi}{r\sin\vartheta}
\frac{\partial}{\partial\varphi}
\\
\displaystyle
\frac{\partial^{\mathstrut}}{\partial y_{\mathstrut}}
&=&
\displaystyle
\frac{\partial r}{\partial y}\frac{\partial}{\partial r}
+
\frac{\partial \vartheta}{\partial y}\frac{\partial}{\partial \vartheta}
+
\frac{\partial \varphi}{\partial y}\frac{\partial}{\partial \varphi}
&=&
\displaystyle
\sin\vartheta\sin\varphi
\frac{\partial}{\partial r}
&+&
\displaystyle
\frac{\cos\vartheta\sin\varphi}{r}
\frac{\partial}{\partial\vartheta}
&+&
\displaystyle
\frac{\cos\varphi}{r\sin\vartheta}
\frac{\partial}{\partial\varphi}
\\
\displaystyle
\frac{\partial^{\mathstrut}}{\partial z_{\mathstrut}}
&=&
\displaystyle
\frac{\partial r}{\partial z}\frac{\partial}{\partial r}
+
\frac{\partial \vartheta}{\partial z}\frac{\partial}{\partial \vartheta}
+
\frac{\partial \varphi}{\partial z}\frac{\partial}{\partial \varphi}
&=&
\displaystyle
\cos\vartheta
\frac{\partial}{\partial r}
&-&
\displaystyle
\frac{\sin\vartheta}{r}
\frac{\partial}{\partial\vartheta}
& &
\end{linsys}
\label{skript:diffopkugel}
\end{equation}
Damit haben wir alle kartesischen Ableitungsoperatoren durch
Ableitungsoperatoren in Kugelkoordinaten ausgedr"uckt. 
Ausgehend von den Formeln (\ref{skript:diffopkugel}) kann man jetzt jeden
in kartesischen Koordinaten gegebenen Differentialgoperator
in Kugelkoordinaten ausdr"ucken, indem man jedes Vorkommen eines
Ableitungsoperators $\frac{\partial}{\partial x_i}$ durch den
entsprechenden Ausdruck aus (\ref{skript:diffopkugel}) ersetzt.

Im Kapitel~\ref{chapter:drehimpuls} wird diese Technik im
Abschnitt~\ref{section:drehimpulsortsdarstellung} verwendet.
Dort wird gezeigt, wie man die Drehimpulsoperatoren in Kugelkoordinaten
darstellen.
Dies erlaubt, den einzelnen Komponenten des Laplaceoperators, der in
Kapitel~\ref{chapter:wasserstoff} bereits untersucht wurde, eine physikalische
Bedeutung zu geben.
Man kann diese Technik aber auch verwenden, um den Laplaceoperator selbst
in Kugelkoordinaten auszudr"ucken, was im n"achsten Abschnitt als
Beispiel f"ur dieses Programm durchgef"uhrt werden soll.

\section{Laplaceoperator}
\rhead{Laplace-Operator}
Der Laplace-Operator in kartesischen Koordinaten ist
\[
\Delta
=
\frac{\partial^2}{\partial x^2}
+
\frac{\partial^2}{\partial y^2}
+
\frac{\partial^2}{\partial z^2}.
\]
In Kugelkoordinaten m"ussen wir alle Differentialoperatoren durch
ihre Versionen in Kugelkoordinaten gem"ass (\ref{skript:diffopkugel}) 
ersetzen.
Um diese Rechnung etwas "ubersichtlicher zu gestalten, werden wir erst
die Quadrate der Ableitungen nach den Koordinaten ausrechnen, und diese
am Schluss zusammenbringen.
Wir beginnen mit den zweiten Ableitungen nach $x$:
\begin{align*}
\frac{\partial^2}{\partial x^2}
&=
\biggl(
\sin\vartheta\cos\varphi
\frac{\partial}{\partial r}
+
\frac{\cos\vartheta\cos\varphi}{r}
\frac{\partial}{\partial\vartheta}
-
\frac{\sin\varphi}{r\sin\vartheta}
\frac{\partial}{\partial\varphi}
\biggr)
\biggl(
\sin\vartheta\cos\varphi
\frac{\partial}{\partial r}
+
\frac{\cos\vartheta\cos\varphi}{r}
\frac{\partial}{\partial\vartheta}
-
\frac{\sin\varphi}{r\sin\vartheta}
\frac{\partial}{\partial\varphi}
\biggr)
\\
&=
\sin^2\vartheta\cos^2\varphi\frac{\partial^2}{\partial r^2}
\\
&\qquad
+
\sin\vartheta \cos\vartheta \cos^2\varphi
\biggl(
-\frac1{r^2}
\frac{\partial}{\partial\vartheta}
+\frac1{r}
\frac{\partial^2}{\partial r\,\partial\vartheta}
\biggr)
\\
&\qquad
+
\cos\varphi\sin\varphi\biggl(
\frac1{r^2}\frac{\partial}{\partial\varphi}
-\frac1{r}\frac{\partial^2}{\partial r\,\partial\varphi}
\biggr)
\\
&\qquad
+\frac1{r}
\cos\vartheta\cos^2\varphi
\biggl(
\cos\vartheta\frac{\partial}{\partial r}
+
\sin\vartheta \frac{\partial^2}{\partial r\,\partial\vartheta}
\biggr)
\\
&\qquad
+\frac1{r^2}
\cos\vartheta\cos^2\varphi\biggl(
-\sin\vartheta
\frac{\partial}{\partial\vartheta}
+
\cos\vartheta
\frac{\partial^2}{\partial\vartheta^2}
\biggr)
\\
&\qquad
-\frac1{r^2}
\cos\vartheta\cos\varphi\sin\varphi
\biggl(
-\frac{\cos\vartheta}{\sin^2\vartheta}
\frac{\partial}{\partial\varphi}
+\frac1{\sin\vartheta}
\frac{\partial^2}{\partial\vartheta\,\partial\varphi}
\biggr)
\\
&\qquad
-\frac1r\sin\varphi\biggl(
-\sin\varphi \frac{\partial}{\partial r}
+\cos\varphi \frac{\partial^2}{\partial r\,\partial\varphi}
\biggr)
\\
&\qquad
-\frac1{r^2}
\frac{ \cos\vartheta \sin\varphi }{\sin\vartheta}\biggl(
-\sin\varphi\frac{\partial}{\partial\vartheta}
+\cos\varphi\frac{\partial^2}{\partial\vartheta\,\partial\varphi}
\biggr)
\\
&\qquad
+\frac1{r^2}\frac{\sin\varphi}{\sin^2\vartheta}\biggl(
\cos\varphi\frac{\partial}{\partial\varphi}
+\sin\varphi\frac{\partial^2}{\partial\varphi^2}
\biggr)
\\
&=
\sin^2\vartheta\cos^2\varphi \frac{\partial^2}{\partial r^2}
+
\frac1r( \cos^2\vartheta\cos^2\varphi + \sin^2\varphi)
\frac{\partial}{\partial r}
\\
&\qquad
+
\frac{2}r\sin\vartheta\cos\vartheta\cos^2\varphi
\frac{\partial^2}{\partial r\,\partial\vartheta}
-
\frac2r\cos\varphi\sin\varphi
\frac{\partial^2}{\partial r\,\partial\varphi}
\\
&\qquad
+
\frac1r\cos^2\vartheta\cos^2\varphi \frac{\partial^2}{\partial \vartheta^2}
-
\frac1{r^2}\biggl(
2\cos\vartheta\sin\vartheta\cos^2\varphi
-\frac{\sin^2\varphi\cos\vartheta}{\sin\vartheta}
\biggr)
\frac{\partial}{\partial\vartheta}
\\
&\qquad
-
\frac2{r^2}\frac{ \cos\vartheta \sin\varphi \cos\varphi }{\sin\vartheta}
\frac{\partial^2}{\partial \vartheta\,\partial\varphi}
\\
&\qquad
+
\frac1{r^2}\frac{\sin^2\varphi}{\sin^2\vartheta}
\frac{\partial^2}{\partial \varphi^2}
+
\frac1{r^2}\cos\varphi\sin\varphi\biggl(
1+\frac{\cos^2\vartheta}{\sin^2\vartheta}
+
\frac1{\sin^2\vartheta}
\biggr)
\frac{\partial}{\partial\varphi}
\end{align*}
Des weiteren die zweiten Ableitungen nach $y$:
\begin{align*}
\frac{\partial^2}{\partial y^2}
&=
\biggl(
\sin\vartheta\sin\varphi
\frac{\partial}{\partial r}
+
\frac{\cos\vartheta\sin\varphi}{r}
\frac{\partial}{\partial\vartheta}
+
\frac{\cos\varphi}{r\sin\vartheta}
\frac{\partial}{\partial\varphi}
\biggr)
\biggl(
\sin\vartheta\sin\varphi
\frac{\partial}{\partial r}
+
\frac{\cos\vartheta\sin\varphi}{r}
\frac{\partial}{\partial\vartheta}
+
\frac{\cos\varphi}{r\sin\vartheta}
\frac{\partial}{\partial\varphi}
\biggr)
\\
&=
\sin^2\vartheta \sin^2\varphi \frac{\partial^2}{\partial r^2}
\\
&\qquad
+\sin\vartheta\cos\vartheta\sin^2\varphi\biggl(
-\frac1{r^2}\frac{\partial}{\partial\vartheta}+\frac1r\frac{\partial^2}{\partial r\,\partial\vartheta}
\biggr)
\\
&\qquad
+\sin\varphi\cos\varphi\biggl(
-\frac1{r^2}\frac{\partial}{\partial\varphi}+\frac1r\frac{\partial^2}{\partial r\,\partial\varphi}
\biggr)
\\
&\qquad
+\frac1r\cos\vartheta\sin^2\varphi\biggl(
\cos\vartheta\frac{\partial}{\partial r}
+\sin\vartheta\frac{\partial^2}{\partial r\,\partial\vartheta}
\biggr)
\\
&\qquad
+\frac1{r^2} \cos\vartheta \sin^2\varphi \biggl(
-\sin\vartheta\frac{\partial}{\partial\vartheta}
+\cos\vartheta\frac{\partial^2}{\partial\vartheta^2}
\biggr)
\\
&\qquad
+\frac1{r^2}\cos\vartheta\sin\varphi\cos\varphi\biggl(
-\frac{\cos\vartheta}{\sin^2\vartheta}\frac{\partial}{\partial\varphi}
+\frac1{\sin\vartheta}\frac{\partial^2}{\partial\vartheta\,\partial\varphi}
\biggr)
\\
&\qquad
+\frac1r\cos\varphi\biggl(
\cos\varphi\frac{\partial}{\partial r}
+\sin\varphi\frac{\partial^2}{\partial\varphi\,\partial r}
\biggr)
\\
&\qquad
+\frac1{r^2}\frac{\cos\vartheta}{\sin\vartheta}\cos\varphi\biggl(
\cos\varphi\frac{\partial}{\partial\vartheta}
+\sin\varphi\frac{\partial^2}{\partial\varphi\,\partial\vartheta}
\biggr)
\\
&\qquad
+\frac1{r^2}\frac{\cos\varphi}{\sin^2\vartheta}\biggl(
-\sin\varphi\frac{\partial}{\partial\varphi}
+\cos\varphi\frac{\partial^2}{\partial\varphi^2}
\biggr)
\\
&=
\sin^2\vartheta\sin^2\varphi \frac{\partial^2}{\partial r^2}
+
\frac1r(\cos^2\vartheta\sin^2\varphi + \cos^2\varphi)\frac{\partial}{\partial r}
\\
&\qquad
+
\frac2r\sin\vartheta\cos\vartheta\sin^2\varphi
\frac{\partial^2}{\partial r\,\partial\vartheta}
+
\frac2r
\sin\varphi\cos\varphi
\frac{\partial^2}{\partial r\,\partial\varphi}
\\
&\qquad
+
\frac1{r^2}\cos^2\vartheta\sin^2\varphi
\frac{\partial^2}{\partial\vartheta^2}
+
\frac1{r^2}
\biggl(
-2\cos\vartheta\sin\vartheta\sin^2\varphi
+\frac{\cos\vartheta}{\sin\vartheta}\cos^2\varphi
\biggr)
\frac{\partial}{\partial\vartheta}
\\
&\qquad
+
\frac2{r^2}\sin\varphi\cos\varphi\frac{\cos\vartheta}{\sin\vartheta}
\frac{\partial^2}{\partial\vartheta\,\partial\varphi}
\\
&\qquad
+
\frac1{r^2}\frac{\cos^2\varphi}{\sin^2\vartheta}
\frac{\partial^2}{\partial\varphi^2}
+
\frac1{r^2}
\sin\varphi\cos\varphi
\biggl(
-1
-
\frac{\cos^2\vartheta}{\sin^2\vartheta}
-\frac1{\sin^2\vartheta}
\biggr)
\frac{\partial}{\partial\varphi}
\end{align*}

Und abschliessen die zweiten Ableitungen nach $z$:
\begin{align*}
\frac{\partial^2}{\partial z^2}
&=
\biggl(
\cos\vartheta
\frac{\partial}{\partial r}
-
\frac{\sin\vartheta}{r}
\frac{\partial}{\partial\vartheta}
\biggr)
\biggl(
\cos\vartheta
\frac{\partial}{\partial r}
-
\frac{\sin\vartheta}{r}
\frac{\partial}{\partial\vartheta}
\biggr)
\\
&=
\cos^2\vartheta\frac{\partial^2}{\partial r^2}
-
\cos\vartheta \sin\vartheta\biggl(
-\frac1{r^2}\frac{\partial}{\partial\vartheta}
+\frac{\partial^2}{\partial r\,\partial\vartheta}
\biggr)
-
\frac1r\sin\vartheta\biggl(
-\sin\vartheta\frac{\partial}{\partial r}
+\cos\vartheta\frac{\partial^2}{\partial\vartheta\,\partial r}
\biggr)
\\
&\qquad
+
\frac1{r^2}\sin\vartheta\biggl(
\cos\vartheta\frac{\partial}{\partial\vartheta}
+\sin\vartheta\frac{\partial^2}{\partial\vartheta^2}
\biggr)
\\
&=\cos^2\vartheta\frac{\partial^2}{\partial r^2}
+
\frac1r\sin^2\vartheta\frac{\partial}{\partial r}
-
\frac2{r}\sin\vartheta\cos\vartheta
\frac{\partial^2}{\partial r\,\partial\vartheta}
+
\frac1{r^2}\sin^2\vartheta \frac{\partial^2}{\partial\vartheta^2}
+
\frac2{r^2}\sin\vartheta\cos\vartheta \frac{\partial}{\partial\vartheta}
\end{align*}

Da wir jetzt alle zweiten Ableitungen ausgerechnet haben, k"onnen wir auch
den Laplace-Operator zusammensetzen:
\begin{align}
\Delta
&=
\frac{\partial^2}{\partial x^2}+
\frac{\partial^2}{\partial y^2}+
\frac{\partial^2}{\partial z^2}
\notag
\\
&=
\biggl(
\sin^2\vartheta\cos^2\varphi
+
\sin^2\vartheta\sin^2\varphi
+
\cos^2\vartheta
\biggr)
\frac{\partial^2}{\partial r^2}
\notag
\\
&\qquad
+
\frac1r\biggl(
\cos^2\vartheta\cos^2\varphi+\sin^2\varphi
+
\cos^2\vartheta\sin^2\varphi +\cos^2\varphi
+
\sin^2\vartheta
\biggr)
\frac{\partial}{\partial r}
\notag
\\
&\qquad
+
\frac2r\biggl(
\sin\vartheta\cos\vartheta\cos^2\varphi
+
\sin\vartheta\cos\vartheta\sin^2\varphi
-\sin\vartheta\cos\vartheta
\biggr)
\frac{\partial^2}{\partial r\,\partial\vartheta}
\notag
\\
&\qquad
+
\frac{2}{r}
\biggl(
-\sin\varphi\cos\varphi
+ \sin\varphi\cos\varphi
\biggr)
\frac{\partial^2}{\partial r\,\partial\varphi}
\notag
\\
&\qquad
+
\frac1{r^2}
\biggl(
\cos^2\vartheta\cos^2\varphi
+\cos^2\vartheta\sin^2\varphi
+\sin^2\vartheta
\biggr)
\frac{\partial^2}{\partial\vartheta^2}
\notag
\\
&\qquad
+
\frac1{r^2}
\sin\vartheta\cos\vartheta
\biggl(
-2 \cos^2\varphi
	+\frac{\sin^2\varphi}{\sin^2\vartheta}
-2\sin^2\varphi
+\frac{\cos^2\varphi}{\sin^2\vartheta}
+
2
\biggr)
\frac{\partial}{\partial\vartheta}
\notag
\\
&\qquad
+
\frac1{r^2}\sin\varphi\cos\varphi\frac{\cos\vartheta}{\sin\vartheta}
\biggl(
-2+2
\biggr)
\frac{\partial^2}{\partial\vartheta\,\partial\varphi}
\notag
\\
&\qquad
+
\frac1{r^2}\frac{\sin^2\varphi+\cos^2\varphi}{\sin^2\vartheta}
\frac{\partial^2}{\partial\varphi^2}
\notag
\\
&\qquad
+
\frac1{r^2}\cos\varphi\sin\varphi
\biggl(
1+\frac{\cos^2\vartheta}{\sin^2\vartheta}+\frac1{\sin^2\vartheta}
-1-\frac{\cos^2\vartheta}{\sin^2\vartheta}-\frac1{\sin^2\vartheta}
\biggr)
\frac{\partial}{\partial\varphi}
\notag
\\
&=
\frac{\partial^2}{\partial r^2}
+\frac{2}{r}\frac{\partial}{\partial r}
+\frac1{r^2}
\frac{\partial^2}{\partial\vartheta^2}
+
\frac1{r^2}
\frac{\cos\vartheta}{\sin\vartheta}
\frac{\partial}{\partial\vartheta}
+
\frac1{r^2}\frac1{\sin^2\vartheta}\frac{\partial^2}{\partial\varphi^2}
\notag
\\
&=
\frac1{r^2}\frac{\partial}{\partial r}\biggl(
r^2\frac{\partial}{\partial r}
\biggr)
+\frac1{r^2\sin\vartheta}\frac{\partial}{\partial\vartheta}\biggl(
\sin\vartheta\frac{\partial}{\partial\vartheta}
\biggr)
+\frac1{r^2}\frac1{\sin^2\vartheta}\frac{\partial^2}{\partial\varphi^2}
\label{skript:laplacekugel}
\end{align}
Dies ist die Form des Laplace-Operators in Kugelkoordinaten, die
wir in der Diskussion des Wasserstoffatoms verwenden.

\chapter{Konstanten\label{chapter:konstanten}}
\lhead{Konstanten}
\rhead{}
\begin{center}
\begin{tabular}{lcrl}
\hline
Konstante&Symbol&Wert&SI Einheit\\
\hline
Lichtgeschwindigkeit     &$c$            &$2.99792458\cdot 10^{\phantom{-}8\phantom{0}}$&m/s\\
Wirkungsquantum          &$h$            &$6.62606957\cdot 10^{-34}$   &Js\\
                         &$\hbar$        &$1.054571726\cdot 10^{-34}$  &Js\\
Elementarladung          &$e$            &$1.602176565\cdot 10^{-19}$  &C \\
Masse des Elektrons      &$m_e$          &$9.10938291\cdot 10^{-31}$   &kg\\
Masse des Protons        &$m_p$          &$1.672621777\cdot 10^{-27}$  &kg\\
Elektrische Feldkonstante&$\varepsilon_0$&$8.85418781762\cdot 10^{-12}$&As/Vm\\
Magnetische Feldkonstante&$\mu_0$        &$1.2566370614\cdot 10^{-6\phantom{0}}$  &N/A$\mathstrut^2$\\
\hline
\end{tabular}
\end{center}


\end{appendices}
\vfill
\pagebreak
\ifodd\value{page}\else\null\clearpage\fi
\lhead{Literatur}
\rhead{}
\printbibliography[heading=subbibliography]
\label{skript:literatur}
\end{refsection}

\part{Anwendungen und Weiterf"uhrende Themen}
\lhead{Anwendungen}
\chapter*{"Ubersicht}
\rhead{"Ubersicht}
\rhead{}
Im zweiten Teil kommen die Teilnehmer des Seminars selbst zu Wort.
Sie zeigen Anwendungsbeispiele f"ur die im ersten
Teil entwickelte Quantenmechanik.
Das Ziel ist nicht, die vorgestellten Anwendungen vollst"andig
berechnen zu k"onnen, sondern an Hand von vereinfachten Modellen
zu zeigen, wie der quantenmechanische Formalismus die wesentlichen
Eigenschaften der Anwendungen zu verstehen erlaubt.
Modellm"assiges Verst"andnis des Mechanismus ist das Ziel, nicht pr"azise
numerische Resultate.

Die ersten drei Beitr"age befassen sich mit Quanten-Informatik.
Das No-Cloning-Theorem~\ref{skript:no-cloning-theorem} bedeutet,
dass sich ein quantenmechanischer
Zustand nicht kopieren l"asst, die ideale Voraussetzung f"ur ein
kryptographisches Verfahren.
Dar"uber berichten {\em Benny G"achter} und {\em Tobias Stauber}.
{\em Max Obrist} und {\em Martin Stypinski} erl"autern, wie ein Quantenzustand
an einen anderen Ort gebracht, also teleportiert werden kann.
Und {\em Marc Juchli} und {\em Kirusanth Poopalasingam} zeigen,
wie Quantencomputern
das Potential haben, Probleme effizient zu l"osen, die klassische 
Computer nicht effizient l"osen k"onnen.

Die Quantenmechanik ist unerl"asslich zum Verst"andnis von
Halbleiterbauelementen.
{\em Stefan Hedinger} erl"autert mit der Tunneldiode
ein eher exotisches Bauelement, welches den quantenmechanischen
Tunneleffekt nutzt.
Auch moderne Flashspeicher brauchen den Tunneleffekt zum L"oschen 
der Speicherzellen.
Man braucht jedoch ein noch etwas tiefer gehendes Verst"andnis der
Quantenmechanik,
um auch den Prozess des Schreibens einer Flash-Zelle zu verstehen, ein
Modell daf"ur besprechen {\em Roger Billeter} und {\em Gabriel Looser}.

Eine ganze Reihe technischer Anwendungen sind ohne die Quantenmechanik
nicht denkbar.
In vielen F"allen k"onnen die quantenmechanischen Zust"ande nicht oder nur
mit grossem Aufwand quantenmechanisch exakt berechnet werden.
Meistens kann die St"orungstheorie helfen, die Zustands"uberg"ange
zu verstehen.
Illustriert wird dies in den folgenden Arbeiten.
{\em Stefan Steiner} und {\em Pascal Stump} erkl"aren die Funktion eines
Frequenznormals, einer Atomuhr. 
Die Arbeit von
{\em Michael Cerny} und {\em Stefan Schindler} illustriert, wie mit der
St"orungstheorie den Einfluss eines elektrischen Feldes auf ein
Elektron berechnet werden kann.
In vielen F"allen kann man Systeme in erster N"aherung als harmonische
Oszillatoren verstehen. {\em Joel Brunner} und {\em Christian Cavegn} berichten,
wie man mit der St"orungstheorie die ver"anderten Energieniveaus und
Wellenfunktionen berechnen kann,
wenn die Annahme der Harmonizit"at nicht mehr zutrifft.

Die Berechnung der Energieniveaus des Wasserstoffatoms hat als
Nebenprodukt die Kugelfunktionen geliefert.
Daraus l"asst sich eine Analysemethode f"ur Funktionen auf einer
Kugeloberfl"ache ableiten.
Sie hat viele technische Anwendungen, zum Beispiel die Analyse der
Abstrahlcharakteristik, die Analyse des Gravitationsfeldes des
Mondes, die Schwingungen der Sonne oder die Inhomegenit"aten des
kosmischen Mikrowellenhintergrundes.
{\em Thomas Gujer} und {\em Christoph Schmitz-Dr"ager} beschreiben
dieses Verfahren.

Die Quantenmechanik hat technische Errungenschaften erm"oglicht, die
aus dem modernen Alltag nicht mehr wegzudenken sind.
Albert Einstein hat schon 1916 die Grundlagen f"ur Laser gelegt,
doch erst in den 50er-Jahren konnte das Prinzip umgesetzt werden.
{\em Arwed Schudel} und {\em Claudio Stucki} zeigen, wie ein Laser funktioniert.
Unter den bildgebenden medizinischen Verfahren ist MRI die eindr"ucklichste
Anwendung quantenmechanischer Prinzipien.
{\em Andreas Linggi}, {\em Daniel Monti} und {\em Nicol\'as Rom\'an L"uthold}
nehmen die MRI-Grundlagen unter die Lupe.

Eine der "uberraschendsten Entdeckungen des 20.~Jahrhunderts ist das
v"ollig Wegfallen des elektrischen Widerstands bei sehr tiefen 
Temperaturen, die Supraleitung.
{\em Simon Kuster} und {\em Nicola Ochsenbein} beschreiben, wie sich
Elektronen zu Cooper-Paaren zusammenlagern k"onnen, die
den elektrischen Widerstand reduzieren k"onnen.
Supraleitung tritt nur bei tiefer Temperatur auf.
Der "Ubergang zum supraleitenden Zustand ist ebenfalls ein Quantenph"anomen.
{\em Reto Christen} und {\em Daniel Gubser} behandeln die Bose-Einstein
Kondensation, welche den Phasen"ubergang verst"andlich macht.

Gleichzeitig mit dem Seminar f"ur Bachelor-Studierende fand auch ein
Seminar auf Master-Stufe statt.
Hier wurden sowohl die mathematischen wie auch die physikalischen Grundlagen
vertiefter behandelt, einzelne Abschnitte und ganze Kapitel des ersten
Teiles wurden nur im Master-Seminar im Detail besprochen.
Zum Beispiel hat sich {\em Dorian Amiet} ausf"uhrlich
"uber den Zusammenhang zwischen Fourier-Transformation und 
Unsch"arfe-Relationen Gedanken gemacht.
Eher in die philosophischen Grundlagenfragen ist {\em Hannes Badertscher}
getaucht. Er beschreibt das Einstein-Podolsky-Rosen-Paradoxon und seine
unerwartete Aufl"osung durch die Bellsche Ungleichung und ihre experimentellen
Best"atigungen.
Mit der Problematik der Quanteneigenschaften von Feldern befasst sich
{\em Hannes Diethelm}. 
Die Quantisierung des Strahlungsfeldes liefert die Grundlagen f"ur die
Berechnung der Einsteinschen $A$- und $B$-Koeffizienten, die in 
Kapitel~\ref{chapter:laser} ben"otigt wurden.

Auf diese Weise f"uhren uns die Betr"age der Teilnehmer von den Grundlagen
der Quantenmechanik, wie sie im Skriptteil dargestellt wurden,
"uber das mindestens modellhafte Verst"andnis interessanter technischer
Anwendungen zur"uck zu dem Punkt, an dem die Quantenmechanik ihren
Anfang genommen hat, n"amlich beim Verst"andnis der Wechselwirkung
zwischen Strahlung und Materie. Sowohl Plancks Strahlungsgesetz wie
auch Einsteins Erkl"arung des Photoeffektes beantworteten eine Fragestellung
"uber das Strahlungsfeld mit quantenmechanischen Ideen.
Dieses Buch darf also durchaus den Anspruch erheben, das Verhalten kleinster
Teilchen ans Licht gebracht zu haben.


\def\chapterauthor#1{{\large #1}\bigskip\bigskip}
% Quanteninformatik
\chapter{Feldquantisierung\label{chapter:feldquantisierung}}
\lhead{Feldquantisierung}
\begin{refsection}
\chapterauthor{Hannes Diethelm}

\printbibliography[heading=subbibliography]
\end{refsection}

\section{Maxwell-Gleichungen und elektromagnetische Wellen}

Hilfreich dazu ist auch die Beschreibung von Magnetfeldern in Kapitel \ref{chapter:magnetfeld}. In diese Kapitel wird der Gradient durch $\nabla$ ersetzt \cite{fq:nabla}. Dadurch k"onnen die Gleichungen einfacher geschrieben werden. 

Die in der Elektrotechnik wohl bekannten Maxwell-Gleichungen in SI Einheiten lauten:
\begin{equation}
\begin{split}
\nabla\cdot E &= \frac{\rho}{\varepsilon_0} \\
\nabla\times B &= \mu_0( J  + \varepsilon_0\frac{\partial E}{\partial t}) \\
\nabla\cdot B &=0 \\
\nabla\times E &= -\frac{\partial B }{\partial t}\\
\end{split}
\end{equation}

Dieses Einheitensystem is willk"urlich \cite{fq:em_units}. Im Heaviside-
Lorentz System, das von nun an verwendet wird, lauten die Gleichungen:
\begin{equation}
\begin{split}
\nabla\cdot E &= \rho \\
\nabla\times B &= \frac{1}{c}( J  + \frac{\partial E}{\partial t}) \\
\nabla\cdot B &=0 \\
\nabla\times E &= -\frac{1}{c} \frac{\partial B }{\partial t}\\
\end{split}
\end{equation}

Da $\nabla \cdot B = 0 $ gilt k"onnen diese Gleichungen durch folgende Substitution umformuliert werden:
\begin{equation}
B = \nabla\times A 
\end{equation}

Dadurch gilt $\nabla \cdot B = 0 $ automatisch:
\begin{equation}
\nabla \cdot B = 0 \rightarrow \nabla \cdot ( \nabla\times A ) = 0 \text{ gilt f"ur jedes A! }
\end{equation}

Durch Einsetzen erh"alt man die Gleichung f"ur E:
\begin{equation}
\nabla\times E + \frac{1}{c} \frac{\partial B }{\partial t} = 0
\rightarrow \nabla\times E + \frac{1}{c} \frac{\partial \nabla\times A }{\partial t} = 0 \rightarrow E = -\frac{1}{c} \dfrac{\partial A}{\partial t} - \nabla \phi
\end{equation}

$\nabla \phi$ kann als Integrationskonstante angesehen werden und $\phi$ entspricht dem skalaren Potential des Feldes.

Durch weiteres Einsetzen k"onnen die vier Maxwell-Gleichungen in zwei Gleichungen umgeschrieben werden:
\begin{equation}
\begin{split}
 \nabla^2 \phi + \frac{1}{c} \dfrac{\partial \nabla A}{\partial t} &= -\rho \\
 \nabla^2 A - \frac{1}{c^2} \frac{\partial^2 A }{\partial t^2} - \nabla \left( \nabla \cdot A + \frac{1}{c} \frac{\partial \phi }{\partial t} \right) &= - \frac{1}{c} J
\end{split}
\end{equation}

dabei gelten die Korrespondenzen:
\begin{equation}
\begin{split}
B &= \nabla\times A \\
E &= -\frac{1}{c} \dfrac{\partial A}{\partial t} - \nabla \phi
\end{split}
\end{equation}

Es kann gezeigt werden, dass $\phi$ durch eine Eichtransformation (Siehe \ref{section:eichtransformation}) geeignet gew"ahlt werden kann, damit:

\begin{equation}
\nabla \cdot A + \frac{1}{c} \frac{\partial \phi }{\partial t} = 0
\end{equation}

Dadurch werden die zwei gekoppelten Gleichungen entkoppelt und es gilt:
\begin{equation}
\begin{split}
\nabla^2 \phi - \frac{1}{c^2} \dfrac{\partial^2 \nabla \phi}{\partial t^2} &= -\rho \\
\nabla^2 A - \frac{1}{c^2} \frac{\partial^2 A }{\partial t^2} &= - \frac{1}{c} J
\end{split}
\end{equation}

F"ur weiter wollen ein Feld im Vakkum betrachten. Hierf"ur gilt $J = 0$, da keine Leiter vorhanden sind.
In einem Transversalfeld im Vakkum gilt zudem $\nabla \cdot A = 0$. (ToDo: ??) Dadurch vereinfachen sich die gekoppelten Differentialgleichung zu einer Differentialgleichung in A:
\begin{equation}
\nabla^2 A - \frac{1}{c^2} \frac{\partial^2 A }{\partial t^2} = 0
\end{equation}

\section{Von der Welle zu gekoppelten Oszillatoren}
L"osungen dieser Gleichung f"ur periodische Randbedingungen und $t=0$ in einer Box mit Seitenl"ange $L = V^{1/3}$ sind durch die Fourier Reihe gegeben:

\begin{equation}
A(x,0) = \frac{1}{\sqrt{V}} \sum_K \sum_{\alpha=1,2} (c_{k,\alpha}(0) \epsilon^{(\alpha)} e^{ikx} + c^*_{k,\alpha}(0) \epsilon^{(\alpha)} e^{-ikx})
\end{equation}

oder durch setzen von $u_{k,\alpha}(x) = \epsilon^{(\alpha)} e^{ikx}$:
\begin{equation}
A(x,0) = \frac{1}{\sqrt{V}} \sum_K \sum_{\alpha=1,2} (c_{k,\alpha}(0)u_{k,\alpha}(x) + c^*_{k,\alpha}(0) u^*_{k,\alpha}(x))
\end{equation}

Wenn diese Gleichung ausgeschrieben wird, sieht man, dass $A(x,t)$ durch diese Wahl f"ur alle $c_{k,\alpha}(t)$ reell bleibt:
\begin{equation}
(a + ib)(\cos kx + i \sin kx ) + (a - ib)(\cos kx - i \sin kx ) = 2 ( a \cos kx - b \sin kx )
\end{equation}
%=a \cos kx + ib \cos kx + ia \sin kx - b \sin kx + a \cos kx - ib \cos kx - ia \sin kx - b \sin kx

$k$ ist der Ausbreitungsvektor der Welle und zeigt in die Ausbreitungsrichtung. $\epsilon^{(\alpha)}$ ist die Polarisation. Dabei wird vorausgesetzt, dass $(\epsilon^{(1)}, \epsilon^{(2)} , k/|k|)$ ein orthogonales Rechtssystem aus Einheitsvektoren bilden.

Da $\epsilon^{(\alpha)}$ und $k$ orthogonal sind gilt dabei auch automatisch:

\begin{equation}
\nabla \cdot A = \frac{1}{\sqrt{V}} \sum_K \sum_{\alpha=1,2} (i c_{k,\alpha}(0) \underbrace{\epsilon^{(\alpha)} k}_{=0} e^{ikx} - i c^*_{k,\alpha}(0) \underbrace{\epsilon^{(\alpha)} k}_{=0} e^{-ikx}) = 0
\end{equation}

Weiterhin gilt durch wegen der Orthogonalit"at auch:
\begin{equation}
\begin{split}
\dfrac{1}{A} \int c_{k,\alpha} \cdot c^*_{k',\alpha'} d^3 x &= \delta_{kk'}\delta{aa'} \\
\dfrac{1}{A} \int c_{k,\alpha} \cdot c_{k',\alpha'} d^3 x &= 0 \\
\dfrac{1}{A} \int c^*_{k,\alpha} \cdot c^*_{k',\alpha'} d^3 x &= 0
\end{split}
\end{equation}

Um $A(x,t)$ zu erhalten, wird:
\begin{equation}
c_{k,\alpha}(t) = c_{k,\alpha}(0) e^{-i \omega t}
\end{equation}

Dabei ist:
\begin{equation}
\begin{split}
\omega=|k|c \\
\lambda = \frac{2 \pi}{|k|}
\end{split}
\end{equation}

Die komplette Wellengleichung wird somit:
\begin{equation}
A(x,t) = \frac{1}{\sqrt{V}} \sum_K \sum_{\alpha=1,2} (c_{k,\alpha}(0) \epsilon^{(\alpha)} e^{i (kx - \omega t)} + c^*_{k,\alpha}(0) \epsilon^{(\alpha)} e^{-i(kx - \omega t)})
\end{equation}

Die Hamilton-Funktion einer elektromagnetischen Welle ist gegeben durch:
\begin{equation}
\begin{split}
H &= \frac{1}{2} \int (|B|^2 + |E|^2) d^3 x \\
	&= \frac{1}{2} \int (| \nabla\times A |^2 + \left| \frac{1}{c} \dfrac{\partial A}{\partial t} \right|^2) d^3 x 
\end{split}
\end{equation}

Es kann gezeigt werden, dass die L"osung dieses Integrals gegeben ist durch:
\begin{equation}
H = \sum_K \sum_{\alpha=1,2} 2 \left(\frac{\omega}{c}\right)^2 c^*_{k,\alpha}(t) c_{k,\alpha}(t)
\end{equation}

Durch folgende Definition:
\begin{equation}
Q_{k,\alpha} = \frac{1}{c}(c_{k,\alpha}(t) + c^*_{k,\alpha}(t)) \quad P_{k,\alpha} = -\frac{i\omega}{c}(c_{k,\alpha}(t) - c^*_{k,\alpha}(t)) 
\end{equation}

wird die Hamilton-Funktion zu:
\begin{equation} \label{fq:hamilton}
\begin{split}
H &= \sum_K \sum_{\alpha=1,2} 2 \left(\frac{\omega}{c}\right)^2 \left[ \frac{c(\omega Q_{k,\alpha} - i P_{k,\alpha})}{2 \omega} \right] \left[ \frac{c(\omega Q_{k,\alpha} + i P_{k,\alpha})}{2 \omega} \right] \\
&= \sum_K \sum_{\alpha=1,2} \frac{1}{2} (P_{k,\alpha}^2 + \omega^2 Q_{k,\alpha}^2)
\end{split}
\end{equation}

Hier sieh man nun, dass es m"oglich ist, eine Welle durch unabh"angige Oszillatoren dar zu stellen.

$Q_{k,\alpha}$ und $P_{k,\alpha}$ k"onnen nun als Koordinaten und Impulse der einzelnen Oszillatoren aufgefasst werden:
\begin{equation}
\dfrac{\partial H}{\partial Q_{k,\alpha}} = -\dot{P}_{k,\alpha} \quad \dfrac{\partial H}{\partial P_{k,\alpha}} = \dot{Q}_{k,\alpha}
\end{equation}

\section{Quantisierung der Welle}

Wie beim harmonischen Oszillator können $Q_{k,\alpha}$ und $P_{k,\alpha}$ nun als Opperatoren aufgefasst werden. Die Vertauschungsrelationen werden dabei zu:
\begin{equation}
\begin{split}
[Q_{k,\alpha}, P_{k',\alpha'}] &= i \hbar \delta_{kk'}\delta_{aa'} \\
[Q_{k,\alpha}, Q_{k',\alpha'}] &= 0 \\
[P_{k,\alpha}, P_{k',\alpha'}] &= 0
\end{split}
\end{equation}

Wir definieren die Operatoren:
\begin{equation}
\begin{split}
a_{k,\alpha} &= (1/\sqrt{2 \hbar \omega})(\omega Q_{k,\alpha} + iP_{k,\alpha})) \\
a^+_{k,\alpha} &= (1/\sqrt{2 \hbar \omega})(\omega Q_{k,\alpha} - iP_{k,\alpha}))\\
N_{k,\alpha} &= a^+_{k,\alpha} a_{k,\alpha}
\end{split}
\end{equation}

Ein Vergleich mit \ref{fq:hamilton} liefert:
\begin{equation}
 c_{k,\alpha} \rightarrow c \sqrt{\hbar/2 \omega} \, a_{k,\alpha} \quad c^*_{k,\alpha} \rightarrow c \sqrt{\hbar/2 \omega} \, a^+_{k,\alpha}
\end{equation}
Somit entsprechen diese Operatoren den Fourier-Koeffizienten.

Die Kommentatoren f"ur diese Operatoren sind:
\begin{equation}
\begin{split}
[a_{k,\alpha} , a^+_{k',\alpha'}] &= - \frac{i}{2 \hbar} [Q_{k,\alpha}, P_{k',\alpha'}] + \frac{i}{2 \hbar} [P_{k,\alpha}, Q_{k',\alpha'}] \\
	 &= \delta_{kk'}\delta_{aa'} \\
[a_{k,\alpha} , a_{k',\alpha'}] &= [a^+_{k,\alpha} , a^+_{k',\alpha'}] \\
	 &= 0 \\
[a_{k,\alpha} , N_{k',\alpha'}] &= [a_{k,\alpha} , a^+_{k',\alpha'}]a_{k',\alpha'} - a^+_{k',\alpha'}[a_{k',\alpha'} , a_{k,\alpha}]\\
	&= \delta_{kk'}\delta_{aa'} a_{k,\alpha} \\
[a^+_{k,\alpha} , N_{k',\alpha'}] &= -\delta_{kk'}\delta_{aa'} a^+_{k,\alpha}
\end{split}
\end{equation}


\chapter{Feldquantisierung\label{chapter:feldquantisierung}}
\lhead{Feldquantisierung}
\begin{refsection}
\chapterauthor{Hannes Diethelm}

\printbibliography[heading=subbibliography]
\end{refsection}

\section{Maxwell-Gleichungen und elektromagnetische Wellen}

Hilfreich dazu ist auch die Beschreibung von Magnetfeldern in Kapitel \ref{chapter:magnetfeld}. In diese Kapitel wird der Gradient durch $\nabla$ ersetzt \cite{fq:nabla}. Dadurch k"onnen die Gleichungen einfacher geschrieben werden. 

Die in der Elektrotechnik wohl bekannten Maxwell-Gleichungen in SI Einheiten lauten:
\begin{equation}
\begin{split}
\nabla\cdot E &= \frac{\rho}{\varepsilon_0} \\
\nabla\times B &= \mu_0( J  + \varepsilon_0\frac{\partial E}{\partial t}) \\
\nabla\cdot B &=0 \\
\nabla\times E &= -\frac{\partial B }{\partial t}\\
\end{split}
\end{equation}

Dieses Einheitensystem is willk"urlich \cite{fq:em_units}. Im Heaviside-
Lorentz System, das von nun an verwendet wird, lauten die Gleichungen:
\begin{equation}
\begin{split}
\nabla\cdot E &= \rho \\
\nabla\times B &= \frac{1}{c}( J  + \frac{\partial E}{\partial t}) \\
\nabla\cdot B &=0 \\
\nabla\times E &= -\frac{1}{c} \frac{\partial B }{\partial t}\\
\end{split}
\end{equation}

Da $\nabla \cdot B = 0 $ gilt k"onnen diese Gleichungen durch folgende Substitution umformuliert werden:
\begin{equation}
B = \nabla\times A 
\end{equation}

Dadurch gilt $\nabla \cdot B = 0 $ automatisch:
\begin{equation}
\nabla \cdot B = 0 \rightarrow \nabla \cdot ( \nabla\times A ) = 0 \text{ gilt f"ur jedes A! }
\end{equation}

Durch Einsetzen erh"alt man die Gleichung f"ur E:
\begin{equation}
\nabla\times E + \frac{1}{c} \frac{\partial B }{\partial t} = 0
\rightarrow \nabla\times E + \frac{1}{c} \frac{\partial \nabla\times A }{\partial t} = 0 \rightarrow E = -\frac{1}{c} \dfrac{\partial A}{\partial t} - \nabla \phi
\end{equation}

$\nabla \phi$ kann als Integrationskonstante angesehen werden und $\phi$ entspricht dem skalaren Potential des Feldes.

Durch weiteres Einsetzen k"onnen die vier Maxwell-Gleichungen in zwei Gleichungen umgeschrieben werden:
\begin{equation}
\begin{split}
 \nabla^2 \phi + \frac{1}{c} \dfrac{\partial \nabla A}{\partial t} &= -\rho \\
 \nabla^2 A - \frac{1}{c^2} \frac{\partial^2 A }{\partial t^2} - \nabla \left( \nabla \cdot A + \frac{1}{c} \frac{\partial \phi }{\partial t} \right) &= - \frac{1}{c} J
\end{split}
\end{equation}

dabei gelten die Korrespondenzen:
\begin{equation}
\begin{split}
B &= \nabla\times A \\
E &= -\frac{1}{c} \dfrac{\partial A}{\partial t} - \nabla \phi
\end{split}
\end{equation}

Es kann gezeigt werden, dass $\phi$ durch eine Eichtransformation (Siehe \ref{section:eichtransformation}) geeignet gew"ahlt werden kann, damit:

\begin{equation}
\nabla \cdot A + \frac{1}{c} \frac{\partial \phi }{\partial t} = 0
\end{equation}

Dadurch werden die zwei gekoppelten Gleichungen entkoppelt und es gilt:
\begin{equation}
\begin{split}
\nabla^2 \phi - \frac{1}{c^2} \dfrac{\partial^2 \nabla \phi}{\partial t^2} &= -\rho \\
\nabla^2 A - \frac{1}{c^2} \frac{\partial^2 A }{\partial t^2} &= - \frac{1}{c} J
\end{split}
\end{equation}

F"ur weiter wollen ein Feld im Vakkum betrachten. Hierf"ur gilt $J = 0$, da keine Leiter vorhanden sind.
In einem Transversalfeld im Vakkum gilt zudem $\nabla \cdot A = 0$. (ToDo: ??) Dadurch vereinfachen sich die gekoppelten Differentialgleichung zu einer Differentialgleichung in A:
\begin{equation}
\nabla^2 A - \frac{1}{c^2} \frac{\partial^2 A }{\partial t^2} = 0
\end{equation}

\section{Von der Welle zu gekoppelten Oszillatoren}
L"osungen dieser Gleichung f"ur periodische Randbedingungen und $t=0$ in einer Box mit Seitenl"ange $L = V^{1/3}$ sind durch die Fourier Reihe gegeben:

\begin{equation}
A(x,0) = \frac{1}{\sqrt{V}} \sum_K \sum_{\alpha=1,2} (c_{k,\alpha}(0) \epsilon^{(\alpha)} e^{ikx} + c^*_{k,\alpha}(0) \epsilon^{(\alpha)} e^{-ikx})
\end{equation}

oder durch setzen von $u_{k,\alpha}(x) = \epsilon^{(\alpha)} e^{ikx}$:
\begin{equation}
A(x,0) = \frac{1}{\sqrt{V}} \sum_K \sum_{\alpha=1,2} (c_{k,\alpha}(0)u_{k,\alpha}(x) + c^*_{k,\alpha}(0) u^*_{k,\alpha}(x))
\end{equation}

Wenn diese Gleichung ausgeschrieben wird, sieht man, dass $A(x,t)$ durch diese Wahl f"ur alle $c_{k,\alpha}(t)$ reell bleibt:
\begin{equation}
(a + ib)(\cos kx + i \sin kx ) + (a - ib)(\cos kx - i \sin kx ) = 2 ( a \cos kx - b \sin kx )
\end{equation}
%=a \cos kx + ib \cos kx + ia \sin kx - b \sin kx + a \cos kx - ib \cos kx - ia \sin kx - b \sin kx

$k$ ist der Ausbreitungsvektor der Welle und zeigt in die Ausbreitungsrichtung. $\epsilon^{(\alpha)}$ ist die Polarisation. Dabei wird vorausgesetzt, dass $(\epsilon^{(1)}, \epsilon^{(2)} , k/|k|)$ ein orthogonales Rechtssystem aus Einheitsvektoren bilden.

Da $\epsilon^{(\alpha)}$ und $k$ orthogonal sind gilt dabei auch automatisch:

\begin{equation}
\nabla \cdot A = \frac{1}{\sqrt{V}} \sum_K \sum_{\alpha=1,2} (i c_{k,\alpha}(0) \underbrace{\epsilon^{(\alpha)} k}_{=0} e^{ikx} - i c^*_{k,\alpha}(0) \underbrace{\epsilon^{(\alpha)} k}_{=0} e^{-ikx}) = 0
\end{equation}

Weiterhin gilt durch wegen der Orthogonalit"at auch:
\begin{equation}
\begin{split}
\dfrac{1}{A} \int c_{k,\alpha} \cdot c^*_{k',\alpha'} d^3 x &= \delta_{kk'}\delta{aa'} \\
\dfrac{1}{A} \int c_{k,\alpha} \cdot c_{k',\alpha'} d^3 x &= 0 \\
\dfrac{1}{A} \int c^*_{k,\alpha} \cdot c^*_{k',\alpha'} d^3 x &= 0
\end{split}
\end{equation}

Um $A(x,t)$ zu erhalten, wird:
\begin{equation}
c_{k,\alpha}(t) = c_{k,\alpha}(0) e^{-i \omega t}
\end{equation}

Dabei ist:
\begin{equation}
\begin{split}
\omega=|k|c \\
\lambda = \frac{2 \pi}{|k|}
\end{split}
\end{equation}

Die komplette Wellengleichung wird somit:
\begin{equation}
A(x,t) = \frac{1}{\sqrt{V}} \sum_K \sum_{\alpha=1,2} (c_{k,\alpha}(0) \epsilon^{(\alpha)} e^{i (kx - \omega t)} + c^*_{k,\alpha}(0) \epsilon^{(\alpha)} e^{-i(kx - \omega t)})
\end{equation}

Die Hamilton-Funktion einer elektromagnetischen Welle ist gegeben durch:
\begin{equation}
\begin{split}
H &= \frac{1}{2} \int (|B|^2 + |E|^2) d^3 x \\
	&= \frac{1}{2} \int (| \nabla\times A |^2 + \left| \frac{1}{c} \dfrac{\partial A}{\partial t} \right|^2) d^3 x 
\end{split}
\end{equation}

Es kann gezeigt werden, dass die L"osung dieses Integrals gegeben ist durch:
\begin{equation}
H = \sum_K \sum_{\alpha=1,2} 2 \left(\frac{\omega}{c}\right)^2 c^*_{k,\alpha}(t) c_{k,\alpha}(t)
\end{equation}

Durch folgende Definition:
\begin{equation}
Q_{k,\alpha} = \frac{1}{c}(c_{k,\alpha}(t) + c^*_{k,\alpha}(t)) \quad P_{k,\alpha} = -\frac{i\omega}{c}(c_{k,\alpha}(t) - c^*_{k,\alpha}(t)) 
\end{equation}

wird die Hamilton-Funktion zu:
\begin{equation} \label{fq:hamilton}
\begin{split}
H &= \sum_K \sum_{\alpha=1,2} 2 \left(\frac{\omega}{c}\right)^2 \left[ \frac{c(\omega Q_{k,\alpha} - i P_{k,\alpha})}{2 \omega} \right] \left[ \frac{c(\omega Q_{k,\alpha} + i P_{k,\alpha})}{2 \omega} \right] \\
&= \sum_K \sum_{\alpha=1,2} \frac{1}{2} (P_{k,\alpha}^2 + \omega^2 Q_{k,\alpha}^2)
\end{split}
\end{equation}

Hier sieh man nun, dass es m"oglich ist, eine Welle durch unabh"angige Oszillatoren dar zu stellen.

$Q_{k,\alpha}$ und $P_{k,\alpha}$ k"onnen nun als Koordinaten und Impulse der einzelnen Oszillatoren aufgefasst werden:
\begin{equation}
\dfrac{\partial H}{\partial Q_{k,\alpha}} = -\dot{P}_{k,\alpha} \quad \dfrac{\partial H}{\partial P_{k,\alpha}} = \dot{Q}_{k,\alpha}
\end{equation}

\section{Quantisierung der Welle}

Wie beim harmonischen Oszillator können $Q_{k,\alpha}$ und $P_{k,\alpha}$ nun als Opperatoren aufgefasst werden. Die Vertauschungsrelationen werden dabei zu:
\begin{equation}
\begin{split}
[Q_{k,\alpha}, P_{k',\alpha'}] &= i \hbar \delta_{kk'}\delta_{aa'} \\
[Q_{k,\alpha}, Q_{k',\alpha'}] &= 0 \\
[P_{k,\alpha}, P_{k',\alpha'}] &= 0
\end{split}
\end{equation}

Wir definieren die Operatoren:
\begin{equation}
\begin{split}
a_{k,\alpha} &= (1/\sqrt{2 \hbar \omega})(\omega Q_{k,\alpha} + iP_{k,\alpha})) \\
a^+_{k,\alpha} &= (1/\sqrt{2 \hbar \omega})(\omega Q_{k,\alpha} - iP_{k,\alpha}))\\
N_{k,\alpha} &= a^+_{k,\alpha} a_{k,\alpha}
\end{split}
\end{equation}

Ein Vergleich mit \ref{fq:hamilton} liefert:
\begin{equation}
 c_{k,\alpha} \rightarrow c \sqrt{\hbar/2 \omega} \, a_{k,\alpha} \quad c^*_{k,\alpha} \rightarrow c \sqrt{\hbar/2 \omega} \, a^+_{k,\alpha}
\end{equation}
Somit entsprechen diese Operatoren den Fourier-Koeffizienten.

Die Kommentatoren f"ur diese Operatoren sind:
\begin{equation}
\begin{split}
[a_{k,\alpha} , a^+_{k',\alpha'}] &= - \frac{i}{2 \hbar} [Q_{k,\alpha}, P_{k',\alpha'}] + \frac{i}{2 \hbar} [P_{k,\alpha}, Q_{k',\alpha'}] \\
	 &= \delta_{kk'}\delta_{aa'} \\
[a_{k,\alpha} , a_{k',\alpha'}] &= [a^+_{k,\alpha} , a^+_{k',\alpha'}] \\
	 &= 0 \\
[a_{k,\alpha} , N_{k',\alpha'}] &= [a_{k,\alpha} , a^+_{k',\alpha'}]a_{k',\alpha'} - a^+_{k',\alpha'}[a_{k',\alpha'} , a_{k,\alpha}]\\
	&= \delta_{kk'}\delta_{aa'} a_{k,\alpha} \\
[a^+_{k,\alpha} , N_{k',\alpha'}] &= -\delta_{kk'}\delta_{aa'} a^+_{k,\alpha}
\end{split}
\end{equation}


\chapter{Feldquantisierung\label{chapter:feldquantisierung}}
\lhead{Feldquantisierung}
\begin{refsection}
\chapterauthor{Hannes Diethelm}

\printbibliography[heading=subbibliography]
\end{refsection}

\section{Maxwell-Gleichungen und elektromagnetische Wellen}

Hilfreich dazu ist auch die Beschreibung von Magnetfeldern in Kapitel \ref{chapter:magnetfeld}. In diese Kapitel wird der Gradient durch $\nabla$ ersetzt \cite{fq:nabla}. Dadurch k"onnen die Gleichungen einfacher geschrieben werden. 

Die in der Elektrotechnik wohl bekannten Maxwell-Gleichungen in SI Einheiten lauten:
\begin{equation}
\begin{split}
\nabla\cdot E &= \frac{\rho}{\varepsilon_0} \\
\nabla\times B &= \mu_0( J  + \varepsilon_0\frac{\partial E}{\partial t}) \\
\nabla\cdot B &=0 \\
\nabla\times E &= -\frac{\partial B }{\partial t}\\
\end{split}
\end{equation}

Dieses Einheitensystem is willk"urlich \cite{fq:em_units}. Im Heaviside-
Lorentz System, das von nun an verwendet wird, lauten die Gleichungen:
\begin{equation}
\begin{split}
\nabla\cdot E &= \rho \\
\nabla\times B &= \frac{1}{c}( J  + \frac{\partial E}{\partial t}) \\
\nabla\cdot B &=0 \\
\nabla\times E &= -\frac{1}{c} \frac{\partial B }{\partial t}\\
\end{split}
\end{equation}

Da $\nabla \cdot B = 0 $ gilt k"onnen diese Gleichungen durch folgende Substitution umformuliert werden:
\begin{equation}
B = \nabla\times A 
\end{equation}

Dadurch gilt $\nabla \cdot B = 0 $ automatisch:
\begin{equation}
\nabla \cdot B = 0 \rightarrow \nabla \cdot ( \nabla\times A ) = 0 \text{ gilt f"ur jedes A! }
\end{equation}

Durch Einsetzen erh"alt man die Gleichung f"ur E:
\begin{equation}
\nabla\times E + \frac{1}{c} \frac{\partial B }{\partial t} = 0
\rightarrow \nabla\times E + \frac{1}{c} \frac{\partial \nabla\times A }{\partial t} = 0 \rightarrow E = -\frac{1}{c} \dfrac{\partial A}{\partial t} - \nabla \phi
\end{equation}

$\nabla \phi$ kann als Integrationskonstante angesehen werden und $\phi$ entspricht dem skalaren Potential des Feldes.

Durch weiteres Einsetzen k"onnen die vier Maxwell-Gleichungen in zwei Gleichungen umgeschrieben werden:
\begin{equation}
\begin{split}
 \nabla^2 \phi + \frac{1}{c} \dfrac{\partial \nabla A}{\partial t} &= -\rho \\
 \nabla^2 A - \frac{1}{c^2} \frac{\partial^2 A }{\partial t^2} - \nabla \left( \nabla \cdot A + \frac{1}{c} \frac{\partial \phi }{\partial t} \right) &= - \frac{1}{c} J
\end{split}
\end{equation}

dabei gelten die Korrespondenzen:
\begin{equation}
\begin{split}
B &= \nabla\times A \\
E &= -\frac{1}{c} \dfrac{\partial A}{\partial t} - \nabla \phi
\end{split}
\end{equation}

Es kann gezeigt werden, dass $\phi$ durch eine Eichtransformation (Siehe \ref{section:eichtransformation}) geeignet gew"ahlt werden kann, damit:

\begin{equation}
\nabla \cdot A + \frac{1}{c} \frac{\partial \phi }{\partial t} = 0
\end{equation}

Dadurch werden die zwei gekoppelten Gleichungen entkoppelt und es gilt:
\begin{equation}
\begin{split}
\nabla^2 \phi - \frac{1}{c^2} \dfrac{\partial^2 \nabla \phi}{\partial t^2} &= -\rho \\
\nabla^2 A - \frac{1}{c^2} \frac{\partial^2 A }{\partial t^2} &= - \frac{1}{c} J
\end{split}
\end{equation}

F"ur weiter wollen ein Feld im Vakkum betrachten. Hierf"ur gilt $J = 0$, da keine Leiter vorhanden sind.
In einem Transversalfeld im Vakkum gilt zudem $\nabla \cdot A = 0$. (ToDo: ??) Dadurch vereinfachen sich die gekoppelten Differentialgleichung zu einer Differentialgleichung in A:
\begin{equation}
\nabla^2 A - \frac{1}{c^2} \frac{\partial^2 A }{\partial t^2} = 0
\end{equation}

\section{Von der Welle zu gekoppelten Oszillatoren}
L"osungen dieser Gleichung f"ur periodische Randbedingungen und $t=0$ in einer Box mit Seitenl"ange $L = V^{1/3}$ sind durch die Fourier Reihe gegeben:

\begin{equation}
A(x,0) = \frac{1}{\sqrt{V}} \sum_K \sum_{\alpha=1,2} (c_{k,\alpha}(0) \epsilon^{(\alpha)} e^{ikx} + c^*_{k,\alpha}(0) \epsilon^{(\alpha)} e^{-ikx})
\end{equation}

oder durch setzen von $u_{k,\alpha}(x) = \epsilon^{(\alpha)} e^{ikx}$:
\begin{equation}
A(x,0) = \frac{1}{\sqrt{V}} \sum_K \sum_{\alpha=1,2} (c_{k,\alpha}(0)u_{k,\alpha}(x) + c^*_{k,\alpha}(0) u^*_{k,\alpha}(x))
\end{equation}

Wenn diese Gleichung ausgeschrieben wird, sieht man, dass $A(x,t)$ durch diese Wahl f"ur alle $c_{k,\alpha}(t)$ reell bleibt:
\begin{equation}
(a + ib)(\cos kx + i \sin kx ) + (a - ib)(\cos kx - i \sin kx ) = 2 ( a \cos kx - b \sin kx )
\end{equation}
%=a \cos kx + ib \cos kx + ia \sin kx - b \sin kx + a \cos kx - ib \cos kx - ia \sin kx - b \sin kx

$k$ ist der Ausbreitungsvektor der Welle und zeigt in die Ausbreitungsrichtung. $\epsilon^{(\alpha)}$ ist die Polarisation. Dabei wird vorausgesetzt, dass $(\epsilon^{(1)}, \epsilon^{(2)} , k/|k|)$ ein orthogonales Rechtssystem aus Einheitsvektoren bilden.

Da $\epsilon^{(\alpha)}$ und $k$ orthogonal sind gilt dabei auch automatisch:

\begin{equation}
\nabla \cdot A = \frac{1}{\sqrt{V}} \sum_K \sum_{\alpha=1,2} (i c_{k,\alpha}(0) \underbrace{\epsilon^{(\alpha)} k}_{=0} e^{ikx} - i c^*_{k,\alpha}(0) \underbrace{\epsilon^{(\alpha)} k}_{=0} e^{-ikx}) = 0
\end{equation}

Weiterhin gilt durch wegen der Orthogonalit"at auch:
\begin{equation}
\begin{split}
\dfrac{1}{A} \int c_{k,\alpha} \cdot c^*_{k',\alpha'} d^3 x &= \delta_{kk'}\delta{aa'} \\
\dfrac{1}{A} \int c_{k,\alpha} \cdot c_{k',\alpha'} d^3 x &= 0 \\
\dfrac{1}{A} \int c^*_{k,\alpha} \cdot c^*_{k',\alpha'} d^3 x &= 0
\end{split}
\end{equation}

Um $A(x,t)$ zu erhalten, wird:
\begin{equation}
c_{k,\alpha}(t) = c_{k,\alpha}(0) e^{-i \omega t}
\end{equation}

Dabei ist:
\begin{equation}
\begin{split}
\omega=|k|c \\
\lambda = \frac{2 \pi}{|k|}
\end{split}
\end{equation}

Die komplette Wellengleichung wird somit:
\begin{equation}
A(x,t) = \frac{1}{\sqrt{V}} \sum_K \sum_{\alpha=1,2} (c_{k,\alpha}(0) \epsilon^{(\alpha)} e^{i (kx - \omega t)} + c^*_{k,\alpha}(0) \epsilon^{(\alpha)} e^{-i(kx - \omega t)})
\end{equation}

Die Hamilton-Funktion einer elektromagnetischen Welle ist gegeben durch:
\begin{equation}
\begin{split}
H &= \frac{1}{2} \int (|B|^2 + |E|^2) d^3 x \\
	&= \frac{1}{2} \int (| \nabla\times A |^2 + \left| \frac{1}{c} \dfrac{\partial A}{\partial t} \right|^2) d^3 x 
\end{split}
\end{equation}

Es kann gezeigt werden, dass die L"osung dieses Integrals gegeben ist durch:
\begin{equation}
H = \sum_K \sum_{\alpha=1,2} 2 \left(\frac{\omega}{c}\right)^2 c^*_{k,\alpha}(t) c_{k,\alpha}(t)
\end{equation}

Durch folgende Definition:
\begin{equation}
Q_{k,\alpha} = \frac{1}{c}(c_{k,\alpha}(t) + c^*_{k,\alpha}(t)) \quad P_{k,\alpha} = -\frac{i\omega}{c}(c_{k,\alpha}(t) - c^*_{k,\alpha}(t)) 
\end{equation}

wird die Hamilton-Funktion zu:
\begin{equation} \label{fq:hamilton}
\begin{split}
H &= \sum_K \sum_{\alpha=1,2} 2 \left(\frac{\omega}{c}\right)^2 \left[ \frac{c(\omega Q_{k,\alpha} - i P_{k,\alpha})}{2 \omega} \right] \left[ \frac{c(\omega Q_{k,\alpha} + i P_{k,\alpha})}{2 \omega} \right] \\
&= \sum_K \sum_{\alpha=1,2} \frac{1}{2} (P_{k,\alpha}^2 + \omega^2 Q_{k,\alpha}^2)
\end{split}
\end{equation}

Hier sieh man nun, dass es m"oglich ist, eine Welle durch unabh"angige Oszillatoren dar zu stellen.

$Q_{k,\alpha}$ und $P_{k,\alpha}$ k"onnen nun als Koordinaten und Impulse der einzelnen Oszillatoren aufgefasst werden:
\begin{equation}
\dfrac{\partial H}{\partial Q_{k,\alpha}} = -\dot{P}_{k,\alpha} \quad \dfrac{\partial H}{\partial P_{k,\alpha}} = \dot{Q}_{k,\alpha}
\end{equation}

\section{Quantisierung der Welle}

Wie beim harmonischen Oszillator können $Q_{k,\alpha}$ und $P_{k,\alpha}$ nun als Opperatoren aufgefasst werden. Die Vertauschungsrelationen werden dabei zu:
\begin{equation}
\begin{split}
[Q_{k,\alpha}, P_{k',\alpha'}] &= i \hbar \delta_{kk'}\delta_{aa'} \\
[Q_{k,\alpha}, Q_{k',\alpha'}] &= 0 \\
[P_{k,\alpha}, P_{k',\alpha'}] &= 0
\end{split}
\end{equation}

Wir definieren die Operatoren:
\begin{equation}
\begin{split}
a_{k,\alpha} &= (1/\sqrt{2 \hbar \omega})(\omega Q_{k,\alpha} + iP_{k,\alpha})) \\
a^+_{k,\alpha} &= (1/\sqrt{2 \hbar \omega})(\omega Q_{k,\alpha} - iP_{k,\alpha}))\\
N_{k,\alpha} &= a^+_{k,\alpha} a_{k,\alpha}
\end{split}
\end{equation}

Ein Vergleich mit \ref{fq:hamilton} liefert:
\begin{equation}
 c_{k,\alpha} \rightarrow c \sqrt{\hbar/2 \omega} \, a_{k,\alpha} \quad c^*_{k,\alpha} \rightarrow c \sqrt{\hbar/2 \omega} \, a^+_{k,\alpha}
\end{equation}
Somit entsprechen diese Operatoren den Fourier-Koeffizienten.

Die Kommentatoren f"ur diese Operatoren sind:
\begin{equation}
\begin{split}
[a_{k,\alpha} , a^+_{k',\alpha'}] &= - \frac{i}{2 \hbar} [Q_{k,\alpha}, P_{k',\alpha'}] + \frac{i}{2 \hbar} [P_{k,\alpha}, Q_{k',\alpha'}] \\
	 &= \delta_{kk'}\delta_{aa'} \\
[a_{k,\alpha} , a_{k',\alpha'}] &= [a^+_{k,\alpha} , a^+_{k',\alpha'}] \\
	 &= 0 \\
[a_{k,\alpha} , N_{k',\alpha'}] &= [a_{k,\alpha} , a^+_{k',\alpha'}]a_{k',\alpha'} - a^+_{k',\alpha'}[a_{k',\alpha'} , a_{k,\alpha}]\\
	&= \delta_{kk'}\delta_{aa'} a_{k,\alpha} \\
[a^+_{k,\alpha} , N_{k',\alpha'}] &= -\delta_{kk'}\delta_{aa'} a^+_{k,\alpha}
\end{split}
\end{equation}


% Halbleiterbauelement
\chapter{Feldquantisierung\label{chapter:feldquantisierung}}
\lhead{Feldquantisierung}
\begin{refsection}
\chapterauthor{Hannes Diethelm}

\printbibliography[heading=subbibliography]
\end{refsection}

\section{Maxwell-Gleichungen und elektromagnetische Wellen}

Hilfreich dazu ist auch die Beschreibung von Magnetfeldern in Kapitel \ref{chapter:magnetfeld}. In diese Kapitel wird der Gradient durch $\nabla$ ersetzt \cite{fq:nabla}. Dadurch k"onnen die Gleichungen einfacher geschrieben werden. 

Die in der Elektrotechnik wohl bekannten Maxwell-Gleichungen in SI Einheiten lauten:
\begin{equation}
\begin{split}
\nabla\cdot E &= \frac{\rho}{\varepsilon_0} \\
\nabla\times B &= \mu_0( J  + \varepsilon_0\frac{\partial E}{\partial t}) \\
\nabla\cdot B &=0 \\
\nabla\times E &= -\frac{\partial B }{\partial t}\\
\end{split}
\end{equation}

Dieses Einheitensystem is willk"urlich \cite{fq:em_units}. Im Heaviside-
Lorentz System, das von nun an verwendet wird, lauten die Gleichungen:
\begin{equation}
\begin{split}
\nabla\cdot E &= \rho \\
\nabla\times B &= \frac{1}{c}( J  + \frac{\partial E}{\partial t}) \\
\nabla\cdot B &=0 \\
\nabla\times E &= -\frac{1}{c} \frac{\partial B }{\partial t}\\
\end{split}
\end{equation}

Da $\nabla \cdot B = 0 $ gilt k"onnen diese Gleichungen durch folgende Substitution umformuliert werden:
\begin{equation}
B = \nabla\times A 
\end{equation}

Dadurch gilt $\nabla \cdot B = 0 $ automatisch:
\begin{equation}
\nabla \cdot B = 0 \rightarrow \nabla \cdot ( \nabla\times A ) = 0 \text{ gilt f"ur jedes A! }
\end{equation}

Durch Einsetzen erh"alt man die Gleichung f"ur E:
\begin{equation}
\nabla\times E + \frac{1}{c} \frac{\partial B }{\partial t} = 0
\rightarrow \nabla\times E + \frac{1}{c} \frac{\partial \nabla\times A }{\partial t} = 0 \rightarrow E = -\frac{1}{c} \dfrac{\partial A}{\partial t} - \nabla \phi
\end{equation}

$\nabla \phi$ kann als Integrationskonstante angesehen werden und $\phi$ entspricht dem skalaren Potential des Feldes.

Durch weiteres Einsetzen k"onnen die vier Maxwell-Gleichungen in zwei Gleichungen umgeschrieben werden:
\begin{equation}
\begin{split}
 \nabla^2 \phi + \frac{1}{c} \dfrac{\partial \nabla A}{\partial t} &= -\rho \\
 \nabla^2 A - \frac{1}{c^2} \frac{\partial^2 A }{\partial t^2} - \nabla \left( \nabla \cdot A + \frac{1}{c} \frac{\partial \phi }{\partial t} \right) &= - \frac{1}{c} J
\end{split}
\end{equation}

dabei gelten die Korrespondenzen:
\begin{equation}
\begin{split}
B &= \nabla\times A \\
E &= -\frac{1}{c} \dfrac{\partial A}{\partial t} - \nabla \phi
\end{split}
\end{equation}

Es kann gezeigt werden, dass $\phi$ durch eine Eichtransformation (Siehe \ref{section:eichtransformation}) geeignet gew"ahlt werden kann, damit:

\begin{equation}
\nabla \cdot A + \frac{1}{c} \frac{\partial \phi }{\partial t} = 0
\end{equation}

Dadurch werden die zwei gekoppelten Gleichungen entkoppelt und es gilt:
\begin{equation}
\begin{split}
\nabla^2 \phi - \frac{1}{c^2} \dfrac{\partial^2 \nabla \phi}{\partial t^2} &= -\rho \\
\nabla^2 A - \frac{1}{c^2} \frac{\partial^2 A }{\partial t^2} &= - \frac{1}{c} J
\end{split}
\end{equation}

F"ur weiter wollen ein Feld im Vakkum betrachten. Hierf"ur gilt $J = 0$, da keine Leiter vorhanden sind.
In einem Transversalfeld im Vakkum gilt zudem $\nabla \cdot A = 0$. (ToDo: ??) Dadurch vereinfachen sich die gekoppelten Differentialgleichung zu einer Differentialgleichung in A:
\begin{equation}
\nabla^2 A - \frac{1}{c^2} \frac{\partial^2 A }{\partial t^2} = 0
\end{equation}

\section{Von der Welle zu gekoppelten Oszillatoren}
L"osungen dieser Gleichung f"ur periodische Randbedingungen und $t=0$ in einer Box mit Seitenl"ange $L = V^{1/3}$ sind durch die Fourier Reihe gegeben:

\begin{equation}
A(x,0) = \frac{1}{\sqrt{V}} \sum_K \sum_{\alpha=1,2} (c_{k,\alpha}(0) \epsilon^{(\alpha)} e^{ikx} + c^*_{k,\alpha}(0) \epsilon^{(\alpha)} e^{-ikx})
\end{equation}

oder durch setzen von $u_{k,\alpha}(x) = \epsilon^{(\alpha)} e^{ikx}$:
\begin{equation}
A(x,0) = \frac{1}{\sqrt{V}} \sum_K \sum_{\alpha=1,2} (c_{k,\alpha}(0)u_{k,\alpha}(x) + c^*_{k,\alpha}(0) u^*_{k,\alpha}(x))
\end{equation}

Wenn diese Gleichung ausgeschrieben wird, sieht man, dass $A(x,t)$ durch diese Wahl f"ur alle $c_{k,\alpha}(t)$ reell bleibt:
\begin{equation}
(a + ib)(\cos kx + i \sin kx ) + (a - ib)(\cos kx - i \sin kx ) = 2 ( a \cos kx - b \sin kx )
\end{equation}
%=a \cos kx + ib \cos kx + ia \sin kx - b \sin kx + a \cos kx - ib \cos kx - ia \sin kx - b \sin kx

$k$ ist der Ausbreitungsvektor der Welle und zeigt in die Ausbreitungsrichtung. $\epsilon^{(\alpha)}$ ist die Polarisation. Dabei wird vorausgesetzt, dass $(\epsilon^{(1)}, \epsilon^{(2)} , k/|k|)$ ein orthogonales Rechtssystem aus Einheitsvektoren bilden.

Da $\epsilon^{(\alpha)}$ und $k$ orthogonal sind gilt dabei auch automatisch:

\begin{equation}
\nabla \cdot A = \frac{1}{\sqrt{V}} \sum_K \sum_{\alpha=1,2} (i c_{k,\alpha}(0) \underbrace{\epsilon^{(\alpha)} k}_{=0} e^{ikx} - i c^*_{k,\alpha}(0) \underbrace{\epsilon^{(\alpha)} k}_{=0} e^{-ikx}) = 0
\end{equation}

Weiterhin gilt durch wegen der Orthogonalit"at auch:
\begin{equation}
\begin{split}
\dfrac{1}{A} \int c_{k,\alpha} \cdot c^*_{k',\alpha'} d^3 x &= \delta_{kk'}\delta{aa'} \\
\dfrac{1}{A} \int c_{k,\alpha} \cdot c_{k',\alpha'} d^3 x &= 0 \\
\dfrac{1}{A} \int c^*_{k,\alpha} \cdot c^*_{k',\alpha'} d^3 x &= 0
\end{split}
\end{equation}

Um $A(x,t)$ zu erhalten, wird:
\begin{equation}
c_{k,\alpha}(t) = c_{k,\alpha}(0) e^{-i \omega t}
\end{equation}

Dabei ist:
\begin{equation}
\begin{split}
\omega=|k|c \\
\lambda = \frac{2 \pi}{|k|}
\end{split}
\end{equation}

Die komplette Wellengleichung wird somit:
\begin{equation}
A(x,t) = \frac{1}{\sqrt{V}} \sum_K \sum_{\alpha=1,2} (c_{k,\alpha}(0) \epsilon^{(\alpha)} e^{i (kx - \omega t)} + c^*_{k,\alpha}(0) \epsilon^{(\alpha)} e^{-i(kx - \omega t)})
\end{equation}

Die Hamilton-Funktion einer elektromagnetischen Welle ist gegeben durch:
\begin{equation}
\begin{split}
H &= \frac{1}{2} \int (|B|^2 + |E|^2) d^3 x \\
	&= \frac{1}{2} \int (| \nabla\times A |^2 + \left| \frac{1}{c} \dfrac{\partial A}{\partial t} \right|^2) d^3 x 
\end{split}
\end{equation}

Es kann gezeigt werden, dass die L"osung dieses Integrals gegeben ist durch:
\begin{equation}
H = \sum_K \sum_{\alpha=1,2} 2 \left(\frac{\omega}{c}\right)^2 c^*_{k,\alpha}(t) c_{k,\alpha}(t)
\end{equation}

Durch folgende Definition:
\begin{equation}
Q_{k,\alpha} = \frac{1}{c}(c_{k,\alpha}(t) + c^*_{k,\alpha}(t)) \quad P_{k,\alpha} = -\frac{i\omega}{c}(c_{k,\alpha}(t) - c^*_{k,\alpha}(t)) 
\end{equation}

wird die Hamilton-Funktion zu:
\begin{equation} \label{fq:hamilton}
\begin{split}
H &= \sum_K \sum_{\alpha=1,2} 2 \left(\frac{\omega}{c}\right)^2 \left[ \frac{c(\omega Q_{k,\alpha} - i P_{k,\alpha})}{2 \omega} \right] \left[ \frac{c(\omega Q_{k,\alpha} + i P_{k,\alpha})}{2 \omega} \right] \\
&= \sum_K \sum_{\alpha=1,2} \frac{1}{2} (P_{k,\alpha}^2 + \omega^2 Q_{k,\alpha}^2)
\end{split}
\end{equation}

Hier sieh man nun, dass es m"oglich ist, eine Welle durch unabh"angige Oszillatoren dar zu stellen.

$Q_{k,\alpha}$ und $P_{k,\alpha}$ k"onnen nun als Koordinaten und Impulse der einzelnen Oszillatoren aufgefasst werden:
\begin{equation}
\dfrac{\partial H}{\partial Q_{k,\alpha}} = -\dot{P}_{k,\alpha} \quad \dfrac{\partial H}{\partial P_{k,\alpha}} = \dot{Q}_{k,\alpha}
\end{equation}

\section{Quantisierung der Welle}

Wie beim harmonischen Oszillator können $Q_{k,\alpha}$ und $P_{k,\alpha}$ nun als Opperatoren aufgefasst werden. Die Vertauschungsrelationen werden dabei zu:
\begin{equation}
\begin{split}
[Q_{k,\alpha}, P_{k',\alpha'}] &= i \hbar \delta_{kk'}\delta_{aa'} \\
[Q_{k,\alpha}, Q_{k',\alpha'}] &= 0 \\
[P_{k,\alpha}, P_{k',\alpha'}] &= 0
\end{split}
\end{equation}

Wir definieren die Operatoren:
\begin{equation}
\begin{split}
a_{k,\alpha} &= (1/\sqrt{2 \hbar \omega})(\omega Q_{k,\alpha} + iP_{k,\alpha})) \\
a^+_{k,\alpha} &= (1/\sqrt{2 \hbar \omega})(\omega Q_{k,\alpha} - iP_{k,\alpha}))\\
N_{k,\alpha} &= a^+_{k,\alpha} a_{k,\alpha}
\end{split}
\end{equation}

Ein Vergleich mit \ref{fq:hamilton} liefert:
\begin{equation}
 c_{k,\alpha} \rightarrow c \sqrt{\hbar/2 \omega} \, a_{k,\alpha} \quad c^*_{k,\alpha} \rightarrow c \sqrt{\hbar/2 \omega} \, a^+_{k,\alpha}
\end{equation}
Somit entsprechen diese Operatoren den Fourier-Koeffizienten.

Die Kommentatoren f"ur diese Operatoren sind:
\begin{equation}
\begin{split}
[a_{k,\alpha} , a^+_{k',\alpha'}] &= - \frac{i}{2 \hbar} [Q_{k,\alpha}, P_{k',\alpha'}] + \frac{i}{2 \hbar} [P_{k,\alpha}, Q_{k',\alpha'}] \\
	 &= \delta_{kk'}\delta_{aa'} \\
[a_{k,\alpha} , a_{k',\alpha'}] &= [a^+_{k,\alpha} , a^+_{k',\alpha'}] \\
	 &= 0 \\
[a_{k,\alpha} , N_{k',\alpha'}] &= [a_{k,\alpha} , a^+_{k',\alpha'}]a_{k',\alpha'} - a^+_{k',\alpha'}[a_{k',\alpha'} , a_{k,\alpha}]\\
	&= \delta_{kk'}\delta_{aa'} a_{k,\alpha} \\
[a^+_{k,\alpha} , N_{k',\alpha'}] &= -\delta_{kk'}\delta_{aa'} a^+_{k,\alpha}
\end{split}
\end{equation}


\chapter{Feldquantisierung\label{chapter:feldquantisierung}}
\lhead{Feldquantisierung}
\begin{refsection}
\chapterauthor{Hannes Diethelm}

\printbibliography[heading=subbibliography]
\end{refsection}

\section{Maxwell-Gleichungen und elektromagnetische Wellen}

Hilfreich dazu ist auch die Beschreibung von Magnetfeldern in Kapitel \ref{chapter:magnetfeld}. In diese Kapitel wird der Gradient durch $\nabla$ ersetzt \cite{fq:nabla}. Dadurch k"onnen die Gleichungen einfacher geschrieben werden. 

Die in der Elektrotechnik wohl bekannten Maxwell-Gleichungen in SI Einheiten lauten:
\begin{equation}
\begin{split}
\nabla\cdot E &= \frac{\rho}{\varepsilon_0} \\
\nabla\times B &= \mu_0( J  + \varepsilon_0\frac{\partial E}{\partial t}) \\
\nabla\cdot B &=0 \\
\nabla\times E &= -\frac{\partial B }{\partial t}\\
\end{split}
\end{equation}

Dieses Einheitensystem is willk"urlich \cite{fq:em_units}. Im Heaviside-
Lorentz System, das von nun an verwendet wird, lauten die Gleichungen:
\begin{equation}
\begin{split}
\nabla\cdot E &= \rho \\
\nabla\times B &= \frac{1}{c}( J  + \frac{\partial E}{\partial t}) \\
\nabla\cdot B &=0 \\
\nabla\times E &= -\frac{1}{c} \frac{\partial B }{\partial t}\\
\end{split}
\end{equation}

Da $\nabla \cdot B = 0 $ gilt k"onnen diese Gleichungen durch folgende Substitution umformuliert werden:
\begin{equation}
B = \nabla\times A 
\end{equation}

Dadurch gilt $\nabla \cdot B = 0 $ automatisch:
\begin{equation}
\nabla \cdot B = 0 \rightarrow \nabla \cdot ( \nabla\times A ) = 0 \text{ gilt f"ur jedes A! }
\end{equation}

Durch Einsetzen erh"alt man die Gleichung f"ur E:
\begin{equation}
\nabla\times E + \frac{1}{c} \frac{\partial B }{\partial t} = 0
\rightarrow \nabla\times E + \frac{1}{c} \frac{\partial \nabla\times A }{\partial t} = 0 \rightarrow E = -\frac{1}{c} \dfrac{\partial A}{\partial t} - \nabla \phi
\end{equation}

$\nabla \phi$ kann als Integrationskonstante angesehen werden und $\phi$ entspricht dem skalaren Potential des Feldes.

Durch weiteres Einsetzen k"onnen die vier Maxwell-Gleichungen in zwei Gleichungen umgeschrieben werden:
\begin{equation}
\begin{split}
 \nabla^2 \phi + \frac{1}{c} \dfrac{\partial \nabla A}{\partial t} &= -\rho \\
 \nabla^2 A - \frac{1}{c^2} \frac{\partial^2 A }{\partial t^2} - \nabla \left( \nabla \cdot A + \frac{1}{c} \frac{\partial \phi }{\partial t} \right) &= - \frac{1}{c} J
\end{split}
\end{equation}

dabei gelten die Korrespondenzen:
\begin{equation}
\begin{split}
B &= \nabla\times A \\
E &= -\frac{1}{c} \dfrac{\partial A}{\partial t} - \nabla \phi
\end{split}
\end{equation}

Es kann gezeigt werden, dass $\phi$ durch eine Eichtransformation (Siehe \ref{section:eichtransformation}) geeignet gew"ahlt werden kann, damit:

\begin{equation}
\nabla \cdot A + \frac{1}{c} \frac{\partial \phi }{\partial t} = 0
\end{equation}

Dadurch werden die zwei gekoppelten Gleichungen entkoppelt und es gilt:
\begin{equation}
\begin{split}
\nabla^2 \phi - \frac{1}{c^2} \dfrac{\partial^2 \nabla \phi}{\partial t^2} &= -\rho \\
\nabla^2 A - \frac{1}{c^2} \frac{\partial^2 A }{\partial t^2} &= - \frac{1}{c} J
\end{split}
\end{equation}

F"ur weiter wollen ein Feld im Vakkum betrachten. Hierf"ur gilt $J = 0$, da keine Leiter vorhanden sind.
In einem Transversalfeld im Vakkum gilt zudem $\nabla \cdot A = 0$. (ToDo: ??) Dadurch vereinfachen sich die gekoppelten Differentialgleichung zu einer Differentialgleichung in A:
\begin{equation}
\nabla^2 A - \frac{1}{c^2} \frac{\partial^2 A }{\partial t^2} = 0
\end{equation}

\section{Von der Welle zu gekoppelten Oszillatoren}
L"osungen dieser Gleichung f"ur periodische Randbedingungen und $t=0$ in einer Box mit Seitenl"ange $L = V^{1/3}$ sind durch die Fourier Reihe gegeben:

\begin{equation}
A(x,0) = \frac{1}{\sqrt{V}} \sum_K \sum_{\alpha=1,2} (c_{k,\alpha}(0) \epsilon^{(\alpha)} e^{ikx} + c^*_{k,\alpha}(0) \epsilon^{(\alpha)} e^{-ikx})
\end{equation}

oder durch setzen von $u_{k,\alpha}(x) = \epsilon^{(\alpha)} e^{ikx}$:
\begin{equation}
A(x,0) = \frac{1}{\sqrt{V}} \sum_K \sum_{\alpha=1,2} (c_{k,\alpha}(0)u_{k,\alpha}(x) + c^*_{k,\alpha}(0) u^*_{k,\alpha}(x))
\end{equation}

Wenn diese Gleichung ausgeschrieben wird, sieht man, dass $A(x,t)$ durch diese Wahl f"ur alle $c_{k,\alpha}(t)$ reell bleibt:
\begin{equation}
(a + ib)(\cos kx + i \sin kx ) + (a - ib)(\cos kx - i \sin kx ) = 2 ( a \cos kx - b \sin kx )
\end{equation}
%=a \cos kx + ib \cos kx + ia \sin kx - b \sin kx + a \cos kx - ib \cos kx - ia \sin kx - b \sin kx

$k$ ist der Ausbreitungsvektor der Welle und zeigt in die Ausbreitungsrichtung. $\epsilon^{(\alpha)}$ ist die Polarisation. Dabei wird vorausgesetzt, dass $(\epsilon^{(1)}, \epsilon^{(2)} , k/|k|)$ ein orthogonales Rechtssystem aus Einheitsvektoren bilden.

Da $\epsilon^{(\alpha)}$ und $k$ orthogonal sind gilt dabei auch automatisch:

\begin{equation}
\nabla \cdot A = \frac{1}{\sqrt{V}} \sum_K \sum_{\alpha=1,2} (i c_{k,\alpha}(0) \underbrace{\epsilon^{(\alpha)} k}_{=0} e^{ikx} - i c^*_{k,\alpha}(0) \underbrace{\epsilon^{(\alpha)} k}_{=0} e^{-ikx}) = 0
\end{equation}

Weiterhin gilt durch wegen der Orthogonalit"at auch:
\begin{equation}
\begin{split}
\dfrac{1}{A} \int c_{k,\alpha} \cdot c^*_{k',\alpha'} d^3 x &= \delta_{kk'}\delta{aa'} \\
\dfrac{1}{A} \int c_{k,\alpha} \cdot c_{k',\alpha'} d^3 x &= 0 \\
\dfrac{1}{A} \int c^*_{k,\alpha} \cdot c^*_{k',\alpha'} d^3 x &= 0
\end{split}
\end{equation}

Um $A(x,t)$ zu erhalten, wird:
\begin{equation}
c_{k,\alpha}(t) = c_{k,\alpha}(0) e^{-i \omega t}
\end{equation}

Dabei ist:
\begin{equation}
\begin{split}
\omega=|k|c \\
\lambda = \frac{2 \pi}{|k|}
\end{split}
\end{equation}

Die komplette Wellengleichung wird somit:
\begin{equation}
A(x,t) = \frac{1}{\sqrt{V}} \sum_K \sum_{\alpha=1,2} (c_{k,\alpha}(0) \epsilon^{(\alpha)} e^{i (kx - \omega t)} + c^*_{k,\alpha}(0) \epsilon^{(\alpha)} e^{-i(kx - \omega t)})
\end{equation}

Die Hamilton-Funktion einer elektromagnetischen Welle ist gegeben durch:
\begin{equation}
\begin{split}
H &= \frac{1}{2} \int (|B|^2 + |E|^2) d^3 x \\
	&= \frac{1}{2} \int (| \nabla\times A |^2 + \left| \frac{1}{c} \dfrac{\partial A}{\partial t} \right|^2) d^3 x 
\end{split}
\end{equation}

Es kann gezeigt werden, dass die L"osung dieses Integrals gegeben ist durch:
\begin{equation}
H = \sum_K \sum_{\alpha=1,2} 2 \left(\frac{\omega}{c}\right)^2 c^*_{k,\alpha}(t) c_{k,\alpha}(t)
\end{equation}

Durch folgende Definition:
\begin{equation}
Q_{k,\alpha} = \frac{1}{c}(c_{k,\alpha}(t) + c^*_{k,\alpha}(t)) \quad P_{k,\alpha} = -\frac{i\omega}{c}(c_{k,\alpha}(t) - c^*_{k,\alpha}(t)) 
\end{equation}

wird die Hamilton-Funktion zu:
\begin{equation} \label{fq:hamilton}
\begin{split}
H &= \sum_K \sum_{\alpha=1,2} 2 \left(\frac{\omega}{c}\right)^2 \left[ \frac{c(\omega Q_{k,\alpha} - i P_{k,\alpha})}{2 \omega} \right] \left[ \frac{c(\omega Q_{k,\alpha} + i P_{k,\alpha})}{2 \omega} \right] \\
&= \sum_K \sum_{\alpha=1,2} \frac{1}{2} (P_{k,\alpha}^2 + \omega^2 Q_{k,\alpha}^2)
\end{split}
\end{equation}

Hier sieh man nun, dass es m"oglich ist, eine Welle durch unabh"angige Oszillatoren dar zu stellen.

$Q_{k,\alpha}$ und $P_{k,\alpha}$ k"onnen nun als Koordinaten und Impulse der einzelnen Oszillatoren aufgefasst werden:
\begin{equation}
\dfrac{\partial H}{\partial Q_{k,\alpha}} = -\dot{P}_{k,\alpha} \quad \dfrac{\partial H}{\partial P_{k,\alpha}} = \dot{Q}_{k,\alpha}
\end{equation}

\section{Quantisierung der Welle}

Wie beim harmonischen Oszillator können $Q_{k,\alpha}$ und $P_{k,\alpha}$ nun als Opperatoren aufgefasst werden. Die Vertauschungsrelationen werden dabei zu:
\begin{equation}
\begin{split}
[Q_{k,\alpha}, P_{k',\alpha'}] &= i \hbar \delta_{kk'}\delta_{aa'} \\
[Q_{k,\alpha}, Q_{k',\alpha'}] &= 0 \\
[P_{k,\alpha}, P_{k',\alpha'}] &= 0
\end{split}
\end{equation}

Wir definieren die Operatoren:
\begin{equation}
\begin{split}
a_{k,\alpha} &= (1/\sqrt{2 \hbar \omega})(\omega Q_{k,\alpha} + iP_{k,\alpha})) \\
a^+_{k,\alpha} &= (1/\sqrt{2 \hbar \omega})(\omega Q_{k,\alpha} - iP_{k,\alpha}))\\
N_{k,\alpha} &= a^+_{k,\alpha} a_{k,\alpha}
\end{split}
\end{equation}

Ein Vergleich mit \ref{fq:hamilton} liefert:
\begin{equation}
 c_{k,\alpha} \rightarrow c \sqrt{\hbar/2 \omega} \, a_{k,\alpha} \quad c^*_{k,\alpha} \rightarrow c \sqrt{\hbar/2 \omega} \, a^+_{k,\alpha}
\end{equation}
Somit entsprechen diese Operatoren den Fourier-Koeffizienten.

Die Kommentatoren f"ur diese Operatoren sind:
\begin{equation}
\begin{split}
[a_{k,\alpha} , a^+_{k',\alpha'}] &= - \frac{i}{2 \hbar} [Q_{k,\alpha}, P_{k',\alpha'}] + \frac{i}{2 \hbar} [P_{k,\alpha}, Q_{k',\alpha'}] \\
	 &= \delta_{kk'}\delta_{aa'} \\
[a_{k,\alpha} , a_{k',\alpha'}] &= [a^+_{k,\alpha} , a^+_{k',\alpha'}] \\
	 &= 0 \\
[a_{k,\alpha} , N_{k',\alpha'}] &= [a_{k,\alpha} , a^+_{k',\alpha'}]a_{k',\alpha'} - a^+_{k',\alpha'}[a_{k',\alpha'} , a_{k,\alpha}]\\
	&= \delta_{kk'}\delta_{aa'} a_{k,\alpha} \\
[a^+_{k,\alpha} , N_{k',\alpha'}] &= -\delta_{kk'}\delta_{aa'} a^+_{k,\alpha}
\end{split}
\end{equation}


% Anwendung der Störungstheorie
\chapter{Feldquantisierung\label{chapter:feldquantisierung}}
\lhead{Feldquantisierung}
\begin{refsection}
\chapterauthor{Hannes Diethelm}

\printbibliography[heading=subbibliography]
\end{refsection}

\section{Maxwell-Gleichungen und elektromagnetische Wellen}

Hilfreich dazu ist auch die Beschreibung von Magnetfeldern in Kapitel \ref{chapter:magnetfeld}. In diese Kapitel wird der Gradient durch $\nabla$ ersetzt \cite{fq:nabla}. Dadurch k"onnen die Gleichungen einfacher geschrieben werden. 

Die in der Elektrotechnik wohl bekannten Maxwell-Gleichungen in SI Einheiten lauten:
\begin{equation}
\begin{split}
\nabla\cdot E &= \frac{\rho}{\varepsilon_0} \\
\nabla\times B &= \mu_0( J  + \varepsilon_0\frac{\partial E}{\partial t}) \\
\nabla\cdot B &=0 \\
\nabla\times E &= -\frac{\partial B }{\partial t}\\
\end{split}
\end{equation}

Dieses Einheitensystem is willk"urlich \cite{fq:em_units}. Im Heaviside-
Lorentz System, das von nun an verwendet wird, lauten die Gleichungen:
\begin{equation}
\begin{split}
\nabla\cdot E &= \rho \\
\nabla\times B &= \frac{1}{c}( J  + \frac{\partial E}{\partial t}) \\
\nabla\cdot B &=0 \\
\nabla\times E &= -\frac{1}{c} \frac{\partial B }{\partial t}\\
\end{split}
\end{equation}

Da $\nabla \cdot B = 0 $ gilt k"onnen diese Gleichungen durch folgende Substitution umformuliert werden:
\begin{equation}
B = \nabla\times A 
\end{equation}

Dadurch gilt $\nabla \cdot B = 0 $ automatisch:
\begin{equation}
\nabla \cdot B = 0 \rightarrow \nabla \cdot ( \nabla\times A ) = 0 \text{ gilt f"ur jedes A! }
\end{equation}

Durch Einsetzen erh"alt man die Gleichung f"ur E:
\begin{equation}
\nabla\times E + \frac{1}{c} \frac{\partial B }{\partial t} = 0
\rightarrow \nabla\times E + \frac{1}{c} \frac{\partial \nabla\times A }{\partial t} = 0 \rightarrow E = -\frac{1}{c} \dfrac{\partial A}{\partial t} - \nabla \phi
\end{equation}

$\nabla \phi$ kann als Integrationskonstante angesehen werden und $\phi$ entspricht dem skalaren Potential des Feldes.

Durch weiteres Einsetzen k"onnen die vier Maxwell-Gleichungen in zwei Gleichungen umgeschrieben werden:
\begin{equation}
\begin{split}
 \nabla^2 \phi + \frac{1}{c} \dfrac{\partial \nabla A}{\partial t} &= -\rho \\
 \nabla^2 A - \frac{1}{c^2} \frac{\partial^2 A }{\partial t^2} - \nabla \left( \nabla \cdot A + \frac{1}{c} \frac{\partial \phi }{\partial t} \right) &= - \frac{1}{c} J
\end{split}
\end{equation}

dabei gelten die Korrespondenzen:
\begin{equation}
\begin{split}
B &= \nabla\times A \\
E &= -\frac{1}{c} \dfrac{\partial A}{\partial t} - \nabla \phi
\end{split}
\end{equation}

Es kann gezeigt werden, dass $\phi$ durch eine Eichtransformation (Siehe \ref{section:eichtransformation}) geeignet gew"ahlt werden kann, damit:

\begin{equation}
\nabla \cdot A + \frac{1}{c} \frac{\partial \phi }{\partial t} = 0
\end{equation}

Dadurch werden die zwei gekoppelten Gleichungen entkoppelt und es gilt:
\begin{equation}
\begin{split}
\nabla^2 \phi - \frac{1}{c^2} \dfrac{\partial^2 \nabla \phi}{\partial t^2} &= -\rho \\
\nabla^2 A - \frac{1}{c^2} \frac{\partial^2 A }{\partial t^2} &= - \frac{1}{c} J
\end{split}
\end{equation}

F"ur weiter wollen ein Feld im Vakkum betrachten. Hierf"ur gilt $J = 0$, da keine Leiter vorhanden sind.
In einem Transversalfeld im Vakkum gilt zudem $\nabla \cdot A = 0$. (ToDo: ??) Dadurch vereinfachen sich die gekoppelten Differentialgleichung zu einer Differentialgleichung in A:
\begin{equation}
\nabla^2 A - \frac{1}{c^2} \frac{\partial^2 A }{\partial t^2} = 0
\end{equation}

\section{Von der Welle zu gekoppelten Oszillatoren}
L"osungen dieser Gleichung f"ur periodische Randbedingungen und $t=0$ in einer Box mit Seitenl"ange $L = V^{1/3}$ sind durch die Fourier Reihe gegeben:

\begin{equation}
A(x,0) = \frac{1}{\sqrt{V}} \sum_K \sum_{\alpha=1,2} (c_{k,\alpha}(0) \epsilon^{(\alpha)} e^{ikx} + c^*_{k,\alpha}(0) \epsilon^{(\alpha)} e^{-ikx})
\end{equation}

oder durch setzen von $u_{k,\alpha}(x) = \epsilon^{(\alpha)} e^{ikx}$:
\begin{equation}
A(x,0) = \frac{1}{\sqrt{V}} \sum_K \sum_{\alpha=1,2} (c_{k,\alpha}(0)u_{k,\alpha}(x) + c^*_{k,\alpha}(0) u^*_{k,\alpha}(x))
\end{equation}

Wenn diese Gleichung ausgeschrieben wird, sieht man, dass $A(x,t)$ durch diese Wahl f"ur alle $c_{k,\alpha}(t)$ reell bleibt:
\begin{equation}
(a + ib)(\cos kx + i \sin kx ) + (a - ib)(\cos kx - i \sin kx ) = 2 ( a \cos kx - b \sin kx )
\end{equation}
%=a \cos kx + ib \cos kx + ia \sin kx - b \sin kx + a \cos kx - ib \cos kx - ia \sin kx - b \sin kx

$k$ ist der Ausbreitungsvektor der Welle und zeigt in die Ausbreitungsrichtung. $\epsilon^{(\alpha)}$ ist die Polarisation. Dabei wird vorausgesetzt, dass $(\epsilon^{(1)}, \epsilon^{(2)} , k/|k|)$ ein orthogonales Rechtssystem aus Einheitsvektoren bilden.

Da $\epsilon^{(\alpha)}$ und $k$ orthogonal sind gilt dabei auch automatisch:

\begin{equation}
\nabla \cdot A = \frac{1}{\sqrt{V}} \sum_K \sum_{\alpha=1,2} (i c_{k,\alpha}(0) \underbrace{\epsilon^{(\alpha)} k}_{=0} e^{ikx} - i c^*_{k,\alpha}(0) \underbrace{\epsilon^{(\alpha)} k}_{=0} e^{-ikx}) = 0
\end{equation}

Weiterhin gilt durch wegen der Orthogonalit"at auch:
\begin{equation}
\begin{split}
\dfrac{1}{A} \int c_{k,\alpha} \cdot c^*_{k',\alpha'} d^3 x &= \delta_{kk'}\delta{aa'} \\
\dfrac{1}{A} \int c_{k,\alpha} \cdot c_{k',\alpha'} d^3 x &= 0 \\
\dfrac{1}{A} \int c^*_{k,\alpha} \cdot c^*_{k',\alpha'} d^3 x &= 0
\end{split}
\end{equation}

Um $A(x,t)$ zu erhalten, wird:
\begin{equation}
c_{k,\alpha}(t) = c_{k,\alpha}(0) e^{-i \omega t}
\end{equation}

Dabei ist:
\begin{equation}
\begin{split}
\omega=|k|c \\
\lambda = \frac{2 \pi}{|k|}
\end{split}
\end{equation}

Die komplette Wellengleichung wird somit:
\begin{equation}
A(x,t) = \frac{1}{\sqrt{V}} \sum_K \sum_{\alpha=1,2} (c_{k,\alpha}(0) \epsilon^{(\alpha)} e^{i (kx - \omega t)} + c^*_{k,\alpha}(0) \epsilon^{(\alpha)} e^{-i(kx - \omega t)})
\end{equation}

Die Hamilton-Funktion einer elektromagnetischen Welle ist gegeben durch:
\begin{equation}
\begin{split}
H &= \frac{1}{2} \int (|B|^2 + |E|^2) d^3 x \\
	&= \frac{1}{2} \int (| \nabla\times A |^2 + \left| \frac{1}{c} \dfrac{\partial A}{\partial t} \right|^2) d^3 x 
\end{split}
\end{equation}

Es kann gezeigt werden, dass die L"osung dieses Integrals gegeben ist durch:
\begin{equation}
H = \sum_K \sum_{\alpha=1,2} 2 \left(\frac{\omega}{c}\right)^2 c^*_{k,\alpha}(t) c_{k,\alpha}(t)
\end{equation}

Durch folgende Definition:
\begin{equation}
Q_{k,\alpha} = \frac{1}{c}(c_{k,\alpha}(t) + c^*_{k,\alpha}(t)) \quad P_{k,\alpha} = -\frac{i\omega}{c}(c_{k,\alpha}(t) - c^*_{k,\alpha}(t)) 
\end{equation}

wird die Hamilton-Funktion zu:
\begin{equation} \label{fq:hamilton}
\begin{split}
H &= \sum_K \sum_{\alpha=1,2} 2 \left(\frac{\omega}{c}\right)^2 \left[ \frac{c(\omega Q_{k,\alpha} - i P_{k,\alpha})}{2 \omega} \right] \left[ \frac{c(\omega Q_{k,\alpha} + i P_{k,\alpha})}{2 \omega} \right] \\
&= \sum_K \sum_{\alpha=1,2} \frac{1}{2} (P_{k,\alpha}^2 + \omega^2 Q_{k,\alpha}^2)
\end{split}
\end{equation}

Hier sieh man nun, dass es m"oglich ist, eine Welle durch unabh"angige Oszillatoren dar zu stellen.

$Q_{k,\alpha}$ und $P_{k,\alpha}$ k"onnen nun als Koordinaten und Impulse der einzelnen Oszillatoren aufgefasst werden:
\begin{equation}
\dfrac{\partial H}{\partial Q_{k,\alpha}} = -\dot{P}_{k,\alpha} \quad \dfrac{\partial H}{\partial P_{k,\alpha}} = \dot{Q}_{k,\alpha}
\end{equation}

\section{Quantisierung der Welle}

Wie beim harmonischen Oszillator können $Q_{k,\alpha}$ und $P_{k,\alpha}$ nun als Opperatoren aufgefasst werden. Die Vertauschungsrelationen werden dabei zu:
\begin{equation}
\begin{split}
[Q_{k,\alpha}, P_{k',\alpha'}] &= i \hbar \delta_{kk'}\delta_{aa'} \\
[Q_{k,\alpha}, Q_{k',\alpha'}] &= 0 \\
[P_{k,\alpha}, P_{k',\alpha'}] &= 0
\end{split}
\end{equation}

Wir definieren die Operatoren:
\begin{equation}
\begin{split}
a_{k,\alpha} &= (1/\sqrt{2 \hbar \omega})(\omega Q_{k,\alpha} + iP_{k,\alpha})) \\
a^+_{k,\alpha} &= (1/\sqrt{2 \hbar \omega})(\omega Q_{k,\alpha} - iP_{k,\alpha}))\\
N_{k,\alpha} &= a^+_{k,\alpha} a_{k,\alpha}
\end{split}
\end{equation}

Ein Vergleich mit \ref{fq:hamilton} liefert:
\begin{equation}
 c_{k,\alpha} \rightarrow c \sqrt{\hbar/2 \omega} \, a_{k,\alpha} \quad c^*_{k,\alpha} \rightarrow c \sqrt{\hbar/2 \omega} \, a^+_{k,\alpha}
\end{equation}
Somit entsprechen diese Operatoren den Fourier-Koeffizienten.

Die Kommentatoren f"ur diese Operatoren sind:
\begin{equation}
\begin{split}
[a_{k,\alpha} , a^+_{k',\alpha'}] &= - \frac{i}{2 \hbar} [Q_{k,\alpha}, P_{k',\alpha'}] + \frac{i}{2 \hbar} [P_{k,\alpha}, Q_{k',\alpha'}] \\
	 &= \delta_{kk'}\delta_{aa'} \\
[a_{k,\alpha} , a_{k',\alpha'}] &= [a^+_{k,\alpha} , a^+_{k',\alpha'}] \\
	 &= 0 \\
[a_{k,\alpha} , N_{k',\alpha'}] &= [a_{k,\alpha} , a^+_{k',\alpha'}]a_{k',\alpha'} - a^+_{k',\alpha'}[a_{k',\alpha'} , a_{k,\alpha}]\\
	&= \delta_{kk'}\delta_{aa'} a_{k,\alpha} \\
[a^+_{k,\alpha} , N_{k',\alpha'}] &= -\delta_{kk'}\delta_{aa'} a^+_{k,\alpha}
\end{split}
\end{equation}


\chapter{Feldquantisierung\label{chapter:feldquantisierung}}
\lhead{Feldquantisierung}
\begin{refsection}
\chapterauthor{Hannes Diethelm}

\printbibliography[heading=subbibliography]
\end{refsection}

\section{Maxwell-Gleichungen und elektromagnetische Wellen}

Hilfreich dazu ist auch die Beschreibung von Magnetfeldern in Kapitel \ref{chapter:magnetfeld}. In diese Kapitel wird der Gradient durch $\nabla$ ersetzt \cite{fq:nabla}. Dadurch k"onnen die Gleichungen einfacher geschrieben werden. 

Die in der Elektrotechnik wohl bekannten Maxwell-Gleichungen in SI Einheiten lauten:
\begin{equation}
\begin{split}
\nabla\cdot E &= \frac{\rho}{\varepsilon_0} \\
\nabla\times B &= \mu_0( J  + \varepsilon_0\frac{\partial E}{\partial t}) \\
\nabla\cdot B &=0 \\
\nabla\times E &= -\frac{\partial B }{\partial t}\\
\end{split}
\end{equation}

Dieses Einheitensystem is willk"urlich \cite{fq:em_units}. Im Heaviside-
Lorentz System, das von nun an verwendet wird, lauten die Gleichungen:
\begin{equation}
\begin{split}
\nabla\cdot E &= \rho \\
\nabla\times B &= \frac{1}{c}( J  + \frac{\partial E}{\partial t}) \\
\nabla\cdot B &=0 \\
\nabla\times E &= -\frac{1}{c} \frac{\partial B }{\partial t}\\
\end{split}
\end{equation}

Da $\nabla \cdot B = 0 $ gilt k"onnen diese Gleichungen durch folgende Substitution umformuliert werden:
\begin{equation}
B = \nabla\times A 
\end{equation}

Dadurch gilt $\nabla \cdot B = 0 $ automatisch:
\begin{equation}
\nabla \cdot B = 0 \rightarrow \nabla \cdot ( \nabla\times A ) = 0 \text{ gilt f"ur jedes A! }
\end{equation}

Durch Einsetzen erh"alt man die Gleichung f"ur E:
\begin{equation}
\nabla\times E + \frac{1}{c} \frac{\partial B }{\partial t} = 0
\rightarrow \nabla\times E + \frac{1}{c} \frac{\partial \nabla\times A }{\partial t} = 0 \rightarrow E = -\frac{1}{c} \dfrac{\partial A}{\partial t} - \nabla \phi
\end{equation}

$\nabla \phi$ kann als Integrationskonstante angesehen werden und $\phi$ entspricht dem skalaren Potential des Feldes.

Durch weiteres Einsetzen k"onnen die vier Maxwell-Gleichungen in zwei Gleichungen umgeschrieben werden:
\begin{equation}
\begin{split}
 \nabla^2 \phi + \frac{1}{c} \dfrac{\partial \nabla A}{\partial t} &= -\rho \\
 \nabla^2 A - \frac{1}{c^2} \frac{\partial^2 A }{\partial t^2} - \nabla \left( \nabla \cdot A + \frac{1}{c} \frac{\partial \phi }{\partial t} \right) &= - \frac{1}{c} J
\end{split}
\end{equation}

dabei gelten die Korrespondenzen:
\begin{equation}
\begin{split}
B &= \nabla\times A \\
E &= -\frac{1}{c} \dfrac{\partial A}{\partial t} - \nabla \phi
\end{split}
\end{equation}

Es kann gezeigt werden, dass $\phi$ durch eine Eichtransformation (Siehe \ref{section:eichtransformation}) geeignet gew"ahlt werden kann, damit:

\begin{equation}
\nabla \cdot A + \frac{1}{c} \frac{\partial \phi }{\partial t} = 0
\end{equation}

Dadurch werden die zwei gekoppelten Gleichungen entkoppelt und es gilt:
\begin{equation}
\begin{split}
\nabla^2 \phi - \frac{1}{c^2} \dfrac{\partial^2 \nabla \phi}{\partial t^2} &= -\rho \\
\nabla^2 A - \frac{1}{c^2} \frac{\partial^2 A }{\partial t^2} &= - \frac{1}{c} J
\end{split}
\end{equation}

F"ur weiter wollen ein Feld im Vakkum betrachten. Hierf"ur gilt $J = 0$, da keine Leiter vorhanden sind.
In einem Transversalfeld im Vakkum gilt zudem $\nabla \cdot A = 0$. (ToDo: ??) Dadurch vereinfachen sich die gekoppelten Differentialgleichung zu einer Differentialgleichung in A:
\begin{equation}
\nabla^2 A - \frac{1}{c^2} \frac{\partial^2 A }{\partial t^2} = 0
\end{equation}

\section{Von der Welle zu gekoppelten Oszillatoren}
L"osungen dieser Gleichung f"ur periodische Randbedingungen und $t=0$ in einer Box mit Seitenl"ange $L = V^{1/3}$ sind durch die Fourier Reihe gegeben:

\begin{equation}
A(x,0) = \frac{1}{\sqrt{V}} \sum_K \sum_{\alpha=1,2} (c_{k,\alpha}(0) \epsilon^{(\alpha)} e^{ikx} + c^*_{k,\alpha}(0) \epsilon^{(\alpha)} e^{-ikx})
\end{equation}

oder durch setzen von $u_{k,\alpha}(x) = \epsilon^{(\alpha)} e^{ikx}$:
\begin{equation}
A(x,0) = \frac{1}{\sqrt{V}} \sum_K \sum_{\alpha=1,2} (c_{k,\alpha}(0)u_{k,\alpha}(x) + c^*_{k,\alpha}(0) u^*_{k,\alpha}(x))
\end{equation}

Wenn diese Gleichung ausgeschrieben wird, sieht man, dass $A(x,t)$ durch diese Wahl f"ur alle $c_{k,\alpha}(t)$ reell bleibt:
\begin{equation}
(a + ib)(\cos kx + i \sin kx ) + (a - ib)(\cos kx - i \sin kx ) = 2 ( a \cos kx - b \sin kx )
\end{equation}
%=a \cos kx + ib \cos kx + ia \sin kx - b \sin kx + a \cos kx - ib \cos kx - ia \sin kx - b \sin kx

$k$ ist der Ausbreitungsvektor der Welle und zeigt in die Ausbreitungsrichtung. $\epsilon^{(\alpha)}$ ist die Polarisation. Dabei wird vorausgesetzt, dass $(\epsilon^{(1)}, \epsilon^{(2)} , k/|k|)$ ein orthogonales Rechtssystem aus Einheitsvektoren bilden.

Da $\epsilon^{(\alpha)}$ und $k$ orthogonal sind gilt dabei auch automatisch:

\begin{equation}
\nabla \cdot A = \frac{1}{\sqrt{V}} \sum_K \sum_{\alpha=1,2} (i c_{k,\alpha}(0) \underbrace{\epsilon^{(\alpha)} k}_{=0} e^{ikx} - i c^*_{k,\alpha}(0) \underbrace{\epsilon^{(\alpha)} k}_{=0} e^{-ikx}) = 0
\end{equation}

Weiterhin gilt durch wegen der Orthogonalit"at auch:
\begin{equation}
\begin{split}
\dfrac{1}{A} \int c_{k,\alpha} \cdot c^*_{k',\alpha'} d^3 x &= \delta_{kk'}\delta{aa'} \\
\dfrac{1}{A} \int c_{k,\alpha} \cdot c_{k',\alpha'} d^3 x &= 0 \\
\dfrac{1}{A} \int c^*_{k,\alpha} \cdot c^*_{k',\alpha'} d^3 x &= 0
\end{split}
\end{equation}

Um $A(x,t)$ zu erhalten, wird:
\begin{equation}
c_{k,\alpha}(t) = c_{k,\alpha}(0) e^{-i \omega t}
\end{equation}

Dabei ist:
\begin{equation}
\begin{split}
\omega=|k|c \\
\lambda = \frac{2 \pi}{|k|}
\end{split}
\end{equation}

Die komplette Wellengleichung wird somit:
\begin{equation}
A(x,t) = \frac{1}{\sqrt{V}} \sum_K \sum_{\alpha=1,2} (c_{k,\alpha}(0) \epsilon^{(\alpha)} e^{i (kx - \omega t)} + c^*_{k,\alpha}(0) \epsilon^{(\alpha)} e^{-i(kx - \omega t)})
\end{equation}

Die Hamilton-Funktion einer elektromagnetischen Welle ist gegeben durch:
\begin{equation}
\begin{split}
H &= \frac{1}{2} \int (|B|^2 + |E|^2) d^3 x \\
	&= \frac{1}{2} \int (| \nabla\times A |^2 + \left| \frac{1}{c} \dfrac{\partial A}{\partial t} \right|^2) d^3 x 
\end{split}
\end{equation}

Es kann gezeigt werden, dass die L"osung dieses Integrals gegeben ist durch:
\begin{equation}
H = \sum_K \sum_{\alpha=1,2} 2 \left(\frac{\omega}{c}\right)^2 c^*_{k,\alpha}(t) c_{k,\alpha}(t)
\end{equation}

Durch folgende Definition:
\begin{equation}
Q_{k,\alpha} = \frac{1}{c}(c_{k,\alpha}(t) + c^*_{k,\alpha}(t)) \quad P_{k,\alpha} = -\frac{i\omega}{c}(c_{k,\alpha}(t) - c^*_{k,\alpha}(t)) 
\end{equation}

wird die Hamilton-Funktion zu:
\begin{equation} \label{fq:hamilton}
\begin{split}
H &= \sum_K \sum_{\alpha=1,2} 2 \left(\frac{\omega}{c}\right)^2 \left[ \frac{c(\omega Q_{k,\alpha} - i P_{k,\alpha})}{2 \omega} \right] \left[ \frac{c(\omega Q_{k,\alpha} + i P_{k,\alpha})}{2 \omega} \right] \\
&= \sum_K \sum_{\alpha=1,2} \frac{1}{2} (P_{k,\alpha}^2 + \omega^2 Q_{k,\alpha}^2)
\end{split}
\end{equation}

Hier sieh man nun, dass es m"oglich ist, eine Welle durch unabh"angige Oszillatoren dar zu stellen.

$Q_{k,\alpha}$ und $P_{k,\alpha}$ k"onnen nun als Koordinaten und Impulse der einzelnen Oszillatoren aufgefasst werden:
\begin{equation}
\dfrac{\partial H}{\partial Q_{k,\alpha}} = -\dot{P}_{k,\alpha} \quad \dfrac{\partial H}{\partial P_{k,\alpha}} = \dot{Q}_{k,\alpha}
\end{equation}

\section{Quantisierung der Welle}

Wie beim harmonischen Oszillator können $Q_{k,\alpha}$ und $P_{k,\alpha}$ nun als Opperatoren aufgefasst werden. Die Vertauschungsrelationen werden dabei zu:
\begin{equation}
\begin{split}
[Q_{k,\alpha}, P_{k',\alpha'}] &= i \hbar \delta_{kk'}\delta_{aa'} \\
[Q_{k,\alpha}, Q_{k',\alpha'}] &= 0 \\
[P_{k,\alpha}, P_{k',\alpha'}] &= 0
\end{split}
\end{equation}

Wir definieren die Operatoren:
\begin{equation}
\begin{split}
a_{k,\alpha} &= (1/\sqrt{2 \hbar \omega})(\omega Q_{k,\alpha} + iP_{k,\alpha})) \\
a^+_{k,\alpha} &= (1/\sqrt{2 \hbar \omega})(\omega Q_{k,\alpha} - iP_{k,\alpha}))\\
N_{k,\alpha} &= a^+_{k,\alpha} a_{k,\alpha}
\end{split}
\end{equation}

Ein Vergleich mit \ref{fq:hamilton} liefert:
\begin{equation}
 c_{k,\alpha} \rightarrow c \sqrt{\hbar/2 \omega} \, a_{k,\alpha} \quad c^*_{k,\alpha} \rightarrow c \sqrt{\hbar/2 \omega} \, a^+_{k,\alpha}
\end{equation}
Somit entsprechen diese Operatoren den Fourier-Koeffizienten.

Die Kommentatoren f"ur diese Operatoren sind:
\begin{equation}
\begin{split}
[a_{k,\alpha} , a^+_{k',\alpha'}] &= - \frac{i}{2 \hbar} [Q_{k,\alpha}, P_{k',\alpha'}] + \frac{i}{2 \hbar} [P_{k,\alpha}, Q_{k',\alpha'}] \\
	 &= \delta_{kk'}\delta_{aa'} \\
[a_{k,\alpha} , a_{k',\alpha'}] &= [a^+_{k,\alpha} , a^+_{k',\alpha'}] \\
	 &= 0 \\
[a_{k,\alpha} , N_{k',\alpha'}] &= [a_{k,\alpha} , a^+_{k',\alpha'}]a_{k',\alpha'} - a^+_{k',\alpha'}[a_{k',\alpha'} , a_{k,\alpha}]\\
	&= \delta_{kk'}\delta_{aa'} a_{k,\alpha} \\
[a^+_{k,\alpha} , N_{k',\alpha'}] &= -\delta_{kk'}\delta_{aa'} a^+_{k,\alpha}
\end{split}
\end{equation}


\chapter{Feldquantisierung\label{chapter:feldquantisierung}}
\lhead{Feldquantisierung}
\begin{refsection}
\chapterauthor{Hannes Diethelm}

\printbibliography[heading=subbibliography]
\end{refsection}

\section{Maxwell-Gleichungen und elektromagnetische Wellen}

Hilfreich dazu ist auch die Beschreibung von Magnetfeldern in Kapitel \ref{chapter:magnetfeld}. In diese Kapitel wird der Gradient durch $\nabla$ ersetzt \cite{fq:nabla}. Dadurch k"onnen die Gleichungen einfacher geschrieben werden. 

Die in der Elektrotechnik wohl bekannten Maxwell-Gleichungen in SI Einheiten lauten:
\begin{equation}
\begin{split}
\nabla\cdot E &= \frac{\rho}{\varepsilon_0} \\
\nabla\times B &= \mu_0( J  + \varepsilon_0\frac{\partial E}{\partial t}) \\
\nabla\cdot B &=0 \\
\nabla\times E &= -\frac{\partial B }{\partial t}\\
\end{split}
\end{equation}

Dieses Einheitensystem is willk"urlich \cite{fq:em_units}. Im Heaviside-
Lorentz System, das von nun an verwendet wird, lauten die Gleichungen:
\begin{equation}
\begin{split}
\nabla\cdot E &= \rho \\
\nabla\times B &= \frac{1}{c}( J  + \frac{\partial E}{\partial t}) \\
\nabla\cdot B &=0 \\
\nabla\times E &= -\frac{1}{c} \frac{\partial B }{\partial t}\\
\end{split}
\end{equation}

Da $\nabla \cdot B = 0 $ gilt k"onnen diese Gleichungen durch folgende Substitution umformuliert werden:
\begin{equation}
B = \nabla\times A 
\end{equation}

Dadurch gilt $\nabla \cdot B = 0 $ automatisch:
\begin{equation}
\nabla \cdot B = 0 \rightarrow \nabla \cdot ( \nabla\times A ) = 0 \text{ gilt f"ur jedes A! }
\end{equation}

Durch Einsetzen erh"alt man die Gleichung f"ur E:
\begin{equation}
\nabla\times E + \frac{1}{c} \frac{\partial B }{\partial t} = 0
\rightarrow \nabla\times E + \frac{1}{c} \frac{\partial \nabla\times A }{\partial t} = 0 \rightarrow E = -\frac{1}{c} \dfrac{\partial A}{\partial t} - \nabla \phi
\end{equation}

$\nabla \phi$ kann als Integrationskonstante angesehen werden und $\phi$ entspricht dem skalaren Potential des Feldes.

Durch weiteres Einsetzen k"onnen die vier Maxwell-Gleichungen in zwei Gleichungen umgeschrieben werden:
\begin{equation}
\begin{split}
 \nabla^2 \phi + \frac{1}{c} \dfrac{\partial \nabla A}{\partial t} &= -\rho \\
 \nabla^2 A - \frac{1}{c^2} \frac{\partial^2 A }{\partial t^2} - \nabla \left( \nabla \cdot A + \frac{1}{c} \frac{\partial \phi }{\partial t} \right) &= - \frac{1}{c} J
\end{split}
\end{equation}

dabei gelten die Korrespondenzen:
\begin{equation}
\begin{split}
B &= \nabla\times A \\
E &= -\frac{1}{c} \dfrac{\partial A}{\partial t} - \nabla \phi
\end{split}
\end{equation}

Es kann gezeigt werden, dass $\phi$ durch eine Eichtransformation (Siehe \ref{section:eichtransformation}) geeignet gew"ahlt werden kann, damit:

\begin{equation}
\nabla \cdot A + \frac{1}{c} \frac{\partial \phi }{\partial t} = 0
\end{equation}

Dadurch werden die zwei gekoppelten Gleichungen entkoppelt und es gilt:
\begin{equation}
\begin{split}
\nabla^2 \phi - \frac{1}{c^2} \dfrac{\partial^2 \nabla \phi}{\partial t^2} &= -\rho \\
\nabla^2 A - \frac{1}{c^2} \frac{\partial^2 A }{\partial t^2} &= - \frac{1}{c} J
\end{split}
\end{equation}

F"ur weiter wollen ein Feld im Vakkum betrachten. Hierf"ur gilt $J = 0$, da keine Leiter vorhanden sind.
In einem Transversalfeld im Vakkum gilt zudem $\nabla \cdot A = 0$. (ToDo: ??) Dadurch vereinfachen sich die gekoppelten Differentialgleichung zu einer Differentialgleichung in A:
\begin{equation}
\nabla^2 A - \frac{1}{c^2} \frac{\partial^2 A }{\partial t^2} = 0
\end{equation}

\section{Von der Welle zu gekoppelten Oszillatoren}
L"osungen dieser Gleichung f"ur periodische Randbedingungen und $t=0$ in einer Box mit Seitenl"ange $L = V^{1/3}$ sind durch die Fourier Reihe gegeben:

\begin{equation}
A(x,0) = \frac{1}{\sqrt{V}} \sum_K \sum_{\alpha=1,2} (c_{k,\alpha}(0) \epsilon^{(\alpha)} e^{ikx} + c^*_{k,\alpha}(0) \epsilon^{(\alpha)} e^{-ikx})
\end{equation}

oder durch setzen von $u_{k,\alpha}(x) = \epsilon^{(\alpha)} e^{ikx}$:
\begin{equation}
A(x,0) = \frac{1}{\sqrt{V}} \sum_K \sum_{\alpha=1,2} (c_{k,\alpha}(0)u_{k,\alpha}(x) + c^*_{k,\alpha}(0) u^*_{k,\alpha}(x))
\end{equation}

Wenn diese Gleichung ausgeschrieben wird, sieht man, dass $A(x,t)$ durch diese Wahl f"ur alle $c_{k,\alpha}(t)$ reell bleibt:
\begin{equation}
(a + ib)(\cos kx + i \sin kx ) + (a - ib)(\cos kx - i \sin kx ) = 2 ( a \cos kx - b \sin kx )
\end{equation}
%=a \cos kx + ib \cos kx + ia \sin kx - b \sin kx + a \cos kx - ib \cos kx - ia \sin kx - b \sin kx

$k$ ist der Ausbreitungsvektor der Welle und zeigt in die Ausbreitungsrichtung. $\epsilon^{(\alpha)}$ ist die Polarisation. Dabei wird vorausgesetzt, dass $(\epsilon^{(1)}, \epsilon^{(2)} , k/|k|)$ ein orthogonales Rechtssystem aus Einheitsvektoren bilden.

Da $\epsilon^{(\alpha)}$ und $k$ orthogonal sind gilt dabei auch automatisch:

\begin{equation}
\nabla \cdot A = \frac{1}{\sqrt{V}} \sum_K \sum_{\alpha=1,2} (i c_{k,\alpha}(0) \underbrace{\epsilon^{(\alpha)} k}_{=0} e^{ikx} - i c^*_{k,\alpha}(0) \underbrace{\epsilon^{(\alpha)} k}_{=0} e^{-ikx}) = 0
\end{equation}

Weiterhin gilt durch wegen der Orthogonalit"at auch:
\begin{equation}
\begin{split}
\dfrac{1}{A} \int c_{k,\alpha} \cdot c^*_{k',\alpha'} d^3 x &= \delta_{kk'}\delta{aa'} \\
\dfrac{1}{A} \int c_{k,\alpha} \cdot c_{k',\alpha'} d^3 x &= 0 \\
\dfrac{1}{A} \int c^*_{k,\alpha} \cdot c^*_{k',\alpha'} d^3 x &= 0
\end{split}
\end{equation}

Um $A(x,t)$ zu erhalten, wird:
\begin{equation}
c_{k,\alpha}(t) = c_{k,\alpha}(0) e^{-i \omega t}
\end{equation}

Dabei ist:
\begin{equation}
\begin{split}
\omega=|k|c \\
\lambda = \frac{2 \pi}{|k|}
\end{split}
\end{equation}

Die komplette Wellengleichung wird somit:
\begin{equation}
A(x,t) = \frac{1}{\sqrt{V}} \sum_K \sum_{\alpha=1,2} (c_{k,\alpha}(0) \epsilon^{(\alpha)} e^{i (kx - \omega t)} + c^*_{k,\alpha}(0) \epsilon^{(\alpha)} e^{-i(kx - \omega t)})
\end{equation}

Die Hamilton-Funktion einer elektromagnetischen Welle ist gegeben durch:
\begin{equation}
\begin{split}
H &= \frac{1}{2} \int (|B|^2 + |E|^2) d^3 x \\
	&= \frac{1}{2} \int (| \nabla\times A |^2 + \left| \frac{1}{c} \dfrac{\partial A}{\partial t} \right|^2) d^3 x 
\end{split}
\end{equation}

Es kann gezeigt werden, dass die L"osung dieses Integrals gegeben ist durch:
\begin{equation}
H = \sum_K \sum_{\alpha=1,2} 2 \left(\frac{\omega}{c}\right)^2 c^*_{k,\alpha}(t) c_{k,\alpha}(t)
\end{equation}

Durch folgende Definition:
\begin{equation}
Q_{k,\alpha} = \frac{1}{c}(c_{k,\alpha}(t) + c^*_{k,\alpha}(t)) \quad P_{k,\alpha} = -\frac{i\omega}{c}(c_{k,\alpha}(t) - c^*_{k,\alpha}(t)) 
\end{equation}

wird die Hamilton-Funktion zu:
\begin{equation} \label{fq:hamilton}
\begin{split}
H &= \sum_K \sum_{\alpha=1,2} 2 \left(\frac{\omega}{c}\right)^2 \left[ \frac{c(\omega Q_{k,\alpha} - i P_{k,\alpha})}{2 \omega} \right] \left[ \frac{c(\omega Q_{k,\alpha} + i P_{k,\alpha})}{2 \omega} \right] \\
&= \sum_K \sum_{\alpha=1,2} \frac{1}{2} (P_{k,\alpha}^2 + \omega^2 Q_{k,\alpha}^2)
\end{split}
\end{equation}

Hier sieh man nun, dass es m"oglich ist, eine Welle durch unabh"angige Oszillatoren dar zu stellen.

$Q_{k,\alpha}$ und $P_{k,\alpha}$ k"onnen nun als Koordinaten und Impulse der einzelnen Oszillatoren aufgefasst werden:
\begin{equation}
\dfrac{\partial H}{\partial Q_{k,\alpha}} = -\dot{P}_{k,\alpha} \quad \dfrac{\partial H}{\partial P_{k,\alpha}} = \dot{Q}_{k,\alpha}
\end{equation}

\section{Quantisierung der Welle}

Wie beim harmonischen Oszillator können $Q_{k,\alpha}$ und $P_{k,\alpha}$ nun als Opperatoren aufgefasst werden. Die Vertauschungsrelationen werden dabei zu:
\begin{equation}
\begin{split}
[Q_{k,\alpha}, P_{k',\alpha'}] &= i \hbar \delta_{kk'}\delta_{aa'} \\
[Q_{k,\alpha}, Q_{k',\alpha'}] &= 0 \\
[P_{k,\alpha}, P_{k',\alpha'}] &= 0
\end{split}
\end{equation}

Wir definieren die Operatoren:
\begin{equation}
\begin{split}
a_{k,\alpha} &= (1/\sqrt{2 \hbar \omega})(\omega Q_{k,\alpha} + iP_{k,\alpha})) \\
a^+_{k,\alpha} &= (1/\sqrt{2 \hbar \omega})(\omega Q_{k,\alpha} - iP_{k,\alpha}))\\
N_{k,\alpha} &= a^+_{k,\alpha} a_{k,\alpha}
\end{split}
\end{equation}

Ein Vergleich mit \ref{fq:hamilton} liefert:
\begin{equation}
 c_{k,\alpha} \rightarrow c \sqrt{\hbar/2 \omega} \, a_{k,\alpha} \quad c^*_{k,\alpha} \rightarrow c \sqrt{\hbar/2 \omega} \, a^+_{k,\alpha}
\end{equation}
Somit entsprechen diese Operatoren den Fourier-Koeffizienten.

Die Kommentatoren f"ur diese Operatoren sind:
\begin{equation}
\begin{split}
[a_{k,\alpha} , a^+_{k',\alpha'}] &= - \frac{i}{2 \hbar} [Q_{k,\alpha}, P_{k',\alpha'}] + \frac{i}{2 \hbar} [P_{k,\alpha}, Q_{k',\alpha'}] \\
	 &= \delta_{kk'}\delta_{aa'} \\
[a_{k,\alpha} , a_{k',\alpha'}] &= [a^+_{k,\alpha} , a^+_{k',\alpha'}] \\
	 &= 0 \\
[a_{k,\alpha} , N_{k',\alpha'}] &= [a_{k,\alpha} , a^+_{k',\alpha'}]a_{k',\alpha'} - a^+_{k',\alpha'}[a_{k',\alpha'} , a_{k,\alpha}]\\
	&= \delta_{kk'}\delta_{aa'} a_{k,\alpha} \\
[a^+_{k,\alpha} , N_{k',\alpha'}] &= -\delta_{kk'}\delta_{aa'} a^+_{k,\alpha}
\end{split}
\end{equation}


% Sphärische harmonische Analyse
\chapter{Feldquantisierung\label{chapter:feldquantisierung}}
\lhead{Feldquantisierung}
\begin{refsection}
\chapterauthor{Hannes Diethelm}

\printbibliography[heading=subbibliography]
\end{refsection}

\section{Maxwell-Gleichungen und elektromagnetische Wellen}

Hilfreich dazu ist auch die Beschreibung von Magnetfeldern in Kapitel \ref{chapter:magnetfeld}. In diese Kapitel wird der Gradient durch $\nabla$ ersetzt \cite{fq:nabla}. Dadurch k"onnen die Gleichungen einfacher geschrieben werden. 

Die in der Elektrotechnik wohl bekannten Maxwell-Gleichungen in SI Einheiten lauten:
\begin{equation}
\begin{split}
\nabla\cdot E &= \frac{\rho}{\varepsilon_0} \\
\nabla\times B &= \mu_0( J  + \varepsilon_0\frac{\partial E}{\partial t}) \\
\nabla\cdot B &=0 \\
\nabla\times E &= -\frac{\partial B }{\partial t}\\
\end{split}
\end{equation}

Dieses Einheitensystem is willk"urlich \cite{fq:em_units}. Im Heaviside-
Lorentz System, das von nun an verwendet wird, lauten die Gleichungen:
\begin{equation}
\begin{split}
\nabla\cdot E &= \rho \\
\nabla\times B &= \frac{1}{c}( J  + \frac{\partial E}{\partial t}) \\
\nabla\cdot B &=0 \\
\nabla\times E &= -\frac{1}{c} \frac{\partial B }{\partial t}\\
\end{split}
\end{equation}

Da $\nabla \cdot B = 0 $ gilt k"onnen diese Gleichungen durch folgende Substitution umformuliert werden:
\begin{equation}
B = \nabla\times A 
\end{equation}

Dadurch gilt $\nabla \cdot B = 0 $ automatisch:
\begin{equation}
\nabla \cdot B = 0 \rightarrow \nabla \cdot ( \nabla\times A ) = 0 \text{ gilt f"ur jedes A! }
\end{equation}

Durch Einsetzen erh"alt man die Gleichung f"ur E:
\begin{equation}
\nabla\times E + \frac{1}{c} \frac{\partial B }{\partial t} = 0
\rightarrow \nabla\times E + \frac{1}{c} \frac{\partial \nabla\times A }{\partial t} = 0 \rightarrow E = -\frac{1}{c} \dfrac{\partial A}{\partial t} - \nabla \phi
\end{equation}

$\nabla \phi$ kann als Integrationskonstante angesehen werden und $\phi$ entspricht dem skalaren Potential des Feldes.

Durch weiteres Einsetzen k"onnen die vier Maxwell-Gleichungen in zwei Gleichungen umgeschrieben werden:
\begin{equation}
\begin{split}
 \nabla^2 \phi + \frac{1}{c} \dfrac{\partial \nabla A}{\partial t} &= -\rho \\
 \nabla^2 A - \frac{1}{c^2} \frac{\partial^2 A }{\partial t^2} - \nabla \left( \nabla \cdot A + \frac{1}{c} \frac{\partial \phi }{\partial t} \right) &= - \frac{1}{c} J
\end{split}
\end{equation}

dabei gelten die Korrespondenzen:
\begin{equation}
\begin{split}
B &= \nabla\times A \\
E &= -\frac{1}{c} \dfrac{\partial A}{\partial t} - \nabla \phi
\end{split}
\end{equation}

Es kann gezeigt werden, dass $\phi$ durch eine Eichtransformation (Siehe \ref{section:eichtransformation}) geeignet gew"ahlt werden kann, damit:

\begin{equation}
\nabla \cdot A + \frac{1}{c} \frac{\partial \phi }{\partial t} = 0
\end{equation}

Dadurch werden die zwei gekoppelten Gleichungen entkoppelt und es gilt:
\begin{equation}
\begin{split}
\nabla^2 \phi - \frac{1}{c^2} \dfrac{\partial^2 \nabla \phi}{\partial t^2} &= -\rho \\
\nabla^2 A - \frac{1}{c^2} \frac{\partial^2 A }{\partial t^2} &= - \frac{1}{c} J
\end{split}
\end{equation}

F"ur weiter wollen ein Feld im Vakkum betrachten. Hierf"ur gilt $J = 0$, da keine Leiter vorhanden sind.
In einem Transversalfeld im Vakkum gilt zudem $\nabla \cdot A = 0$. (ToDo: ??) Dadurch vereinfachen sich die gekoppelten Differentialgleichung zu einer Differentialgleichung in A:
\begin{equation}
\nabla^2 A - \frac{1}{c^2} \frac{\partial^2 A }{\partial t^2} = 0
\end{equation}

\section{Von der Welle zu gekoppelten Oszillatoren}
L"osungen dieser Gleichung f"ur periodische Randbedingungen und $t=0$ in einer Box mit Seitenl"ange $L = V^{1/3}$ sind durch die Fourier Reihe gegeben:

\begin{equation}
A(x,0) = \frac{1}{\sqrt{V}} \sum_K \sum_{\alpha=1,2} (c_{k,\alpha}(0) \epsilon^{(\alpha)} e^{ikx} + c^*_{k,\alpha}(0) \epsilon^{(\alpha)} e^{-ikx})
\end{equation}

oder durch setzen von $u_{k,\alpha}(x) = \epsilon^{(\alpha)} e^{ikx}$:
\begin{equation}
A(x,0) = \frac{1}{\sqrt{V}} \sum_K \sum_{\alpha=1,2} (c_{k,\alpha}(0)u_{k,\alpha}(x) + c^*_{k,\alpha}(0) u^*_{k,\alpha}(x))
\end{equation}

Wenn diese Gleichung ausgeschrieben wird, sieht man, dass $A(x,t)$ durch diese Wahl f"ur alle $c_{k,\alpha}(t)$ reell bleibt:
\begin{equation}
(a + ib)(\cos kx + i \sin kx ) + (a - ib)(\cos kx - i \sin kx ) = 2 ( a \cos kx - b \sin kx )
\end{equation}
%=a \cos kx + ib \cos kx + ia \sin kx - b \sin kx + a \cos kx - ib \cos kx - ia \sin kx - b \sin kx

$k$ ist der Ausbreitungsvektor der Welle und zeigt in die Ausbreitungsrichtung. $\epsilon^{(\alpha)}$ ist die Polarisation. Dabei wird vorausgesetzt, dass $(\epsilon^{(1)}, \epsilon^{(2)} , k/|k|)$ ein orthogonales Rechtssystem aus Einheitsvektoren bilden.

Da $\epsilon^{(\alpha)}$ und $k$ orthogonal sind gilt dabei auch automatisch:

\begin{equation}
\nabla \cdot A = \frac{1}{\sqrt{V}} \sum_K \sum_{\alpha=1,2} (i c_{k,\alpha}(0) \underbrace{\epsilon^{(\alpha)} k}_{=0} e^{ikx} - i c^*_{k,\alpha}(0) \underbrace{\epsilon^{(\alpha)} k}_{=0} e^{-ikx}) = 0
\end{equation}

Weiterhin gilt durch wegen der Orthogonalit"at auch:
\begin{equation}
\begin{split}
\dfrac{1}{A} \int c_{k,\alpha} \cdot c^*_{k',\alpha'} d^3 x &= \delta_{kk'}\delta{aa'} \\
\dfrac{1}{A} \int c_{k,\alpha} \cdot c_{k',\alpha'} d^3 x &= 0 \\
\dfrac{1}{A} \int c^*_{k,\alpha} \cdot c^*_{k',\alpha'} d^3 x &= 0
\end{split}
\end{equation}

Um $A(x,t)$ zu erhalten, wird:
\begin{equation}
c_{k,\alpha}(t) = c_{k,\alpha}(0) e^{-i \omega t}
\end{equation}

Dabei ist:
\begin{equation}
\begin{split}
\omega=|k|c \\
\lambda = \frac{2 \pi}{|k|}
\end{split}
\end{equation}

Die komplette Wellengleichung wird somit:
\begin{equation}
A(x,t) = \frac{1}{\sqrt{V}} \sum_K \sum_{\alpha=1,2} (c_{k,\alpha}(0) \epsilon^{(\alpha)} e^{i (kx - \omega t)} + c^*_{k,\alpha}(0) \epsilon^{(\alpha)} e^{-i(kx - \omega t)})
\end{equation}

Die Hamilton-Funktion einer elektromagnetischen Welle ist gegeben durch:
\begin{equation}
\begin{split}
H &= \frac{1}{2} \int (|B|^2 + |E|^2) d^3 x \\
	&= \frac{1}{2} \int (| \nabla\times A |^2 + \left| \frac{1}{c} \dfrac{\partial A}{\partial t} \right|^2) d^3 x 
\end{split}
\end{equation}

Es kann gezeigt werden, dass die L"osung dieses Integrals gegeben ist durch:
\begin{equation}
H = \sum_K \sum_{\alpha=1,2} 2 \left(\frac{\omega}{c}\right)^2 c^*_{k,\alpha}(t) c_{k,\alpha}(t)
\end{equation}

Durch folgende Definition:
\begin{equation}
Q_{k,\alpha} = \frac{1}{c}(c_{k,\alpha}(t) + c^*_{k,\alpha}(t)) \quad P_{k,\alpha} = -\frac{i\omega}{c}(c_{k,\alpha}(t) - c^*_{k,\alpha}(t)) 
\end{equation}

wird die Hamilton-Funktion zu:
\begin{equation} \label{fq:hamilton}
\begin{split}
H &= \sum_K \sum_{\alpha=1,2} 2 \left(\frac{\omega}{c}\right)^2 \left[ \frac{c(\omega Q_{k,\alpha} - i P_{k,\alpha})}{2 \omega} \right] \left[ \frac{c(\omega Q_{k,\alpha} + i P_{k,\alpha})}{2 \omega} \right] \\
&= \sum_K \sum_{\alpha=1,2} \frac{1}{2} (P_{k,\alpha}^2 + \omega^2 Q_{k,\alpha}^2)
\end{split}
\end{equation}

Hier sieh man nun, dass es m"oglich ist, eine Welle durch unabh"angige Oszillatoren dar zu stellen.

$Q_{k,\alpha}$ und $P_{k,\alpha}$ k"onnen nun als Koordinaten und Impulse der einzelnen Oszillatoren aufgefasst werden:
\begin{equation}
\dfrac{\partial H}{\partial Q_{k,\alpha}} = -\dot{P}_{k,\alpha} \quad \dfrac{\partial H}{\partial P_{k,\alpha}} = \dot{Q}_{k,\alpha}
\end{equation}

\section{Quantisierung der Welle}

Wie beim harmonischen Oszillator können $Q_{k,\alpha}$ und $P_{k,\alpha}$ nun als Opperatoren aufgefasst werden. Die Vertauschungsrelationen werden dabei zu:
\begin{equation}
\begin{split}
[Q_{k,\alpha}, P_{k',\alpha'}] &= i \hbar \delta_{kk'}\delta_{aa'} \\
[Q_{k,\alpha}, Q_{k',\alpha'}] &= 0 \\
[P_{k,\alpha}, P_{k',\alpha'}] &= 0
\end{split}
\end{equation}

Wir definieren die Operatoren:
\begin{equation}
\begin{split}
a_{k,\alpha} &= (1/\sqrt{2 \hbar \omega})(\omega Q_{k,\alpha} + iP_{k,\alpha})) \\
a^+_{k,\alpha} &= (1/\sqrt{2 \hbar \omega})(\omega Q_{k,\alpha} - iP_{k,\alpha}))\\
N_{k,\alpha} &= a^+_{k,\alpha} a_{k,\alpha}
\end{split}
\end{equation}

Ein Vergleich mit \ref{fq:hamilton} liefert:
\begin{equation}
 c_{k,\alpha} \rightarrow c \sqrt{\hbar/2 \omega} \, a_{k,\alpha} \quad c^*_{k,\alpha} \rightarrow c \sqrt{\hbar/2 \omega} \, a^+_{k,\alpha}
\end{equation}
Somit entsprechen diese Operatoren den Fourier-Koeffizienten.

Die Kommentatoren f"ur diese Operatoren sind:
\begin{equation}
\begin{split}
[a_{k,\alpha} , a^+_{k',\alpha'}] &= - \frac{i}{2 \hbar} [Q_{k,\alpha}, P_{k',\alpha'}] + \frac{i}{2 \hbar} [P_{k,\alpha}, Q_{k',\alpha'}] \\
	 &= \delta_{kk'}\delta_{aa'} \\
[a_{k,\alpha} , a_{k',\alpha'}] &= [a^+_{k,\alpha} , a^+_{k',\alpha'}] \\
	 &= 0 \\
[a_{k,\alpha} , N_{k',\alpha'}] &= [a_{k,\alpha} , a^+_{k',\alpha'}]a_{k',\alpha'} - a^+_{k',\alpha'}[a_{k',\alpha'} , a_{k,\alpha}]\\
	&= \delta_{kk'}\delta_{aa'} a_{k,\alpha} \\
[a^+_{k,\alpha} , N_{k',\alpha'}] &= -\delta_{kk'}\delta_{aa'} a^+_{k,\alpha}
\end{split}
\end{equation}


% Weitere Anwendungen
\chapter{Feldquantisierung\label{chapter:feldquantisierung}}
\lhead{Feldquantisierung}
\begin{refsection}
\chapterauthor{Hannes Diethelm}

\printbibliography[heading=subbibliography]
\end{refsection}

\section{Maxwell-Gleichungen und elektromagnetische Wellen}

Hilfreich dazu ist auch die Beschreibung von Magnetfeldern in Kapitel \ref{chapter:magnetfeld}. In diese Kapitel wird der Gradient durch $\nabla$ ersetzt \cite{fq:nabla}. Dadurch k"onnen die Gleichungen einfacher geschrieben werden. 

Die in der Elektrotechnik wohl bekannten Maxwell-Gleichungen in SI Einheiten lauten:
\begin{equation}
\begin{split}
\nabla\cdot E &= \frac{\rho}{\varepsilon_0} \\
\nabla\times B &= \mu_0( J  + \varepsilon_0\frac{\partial E}{\partial t}) \\
\nabla\cdot B &=0 \\
\nabla\times E &= -\frac{\partial B }{\partial t}\\
\end{split}
\end{equation}

Dieses Einheitensystem is willk"urlich \cite{fq:em_units}. Im Heaviside-
Lorentz System, das von nun an verwendet wird, lauten die Gleichungen:
\begin{equation}
\begin{split}
\nabla\cdot E &= \rho \\
\nabla\times B &= \frac{1}{c}( J  + \frac{\partial E}{\partial t}) \\
\nabla\cdot B &=0 \\
\nabla\times E &= -\frac{1}{c} \frac{\partial B }{\partial t}\\
\end{split}
\end{equation}

Da $\nabla \cdot B = 0 $ gilt k"onnen diese Gleichungen durch folgende Substitution umformuliert werden:
\begin{equation}
B = \nabla\times A 
\end{equation}

Dadurch gilt $\nabla \cdot B = 0 $ automatisch:
\begin{equation}
\nabla \cdot B = 0 \rightarrow \nabla \cdot ( \nabla\times A ) = 0 \text{ gilt f"ur jedes A! }
\end{equation}

Durch Einsetzen erh"alt man die Gleichung f"ur E:
\begin{equation}
\nabla\times E + \frac{1}{c} \frac{\partial B }{\partial t} = 0
\rightarrow \nabla\times E + \frac{1}{c} \frac{\partial \nabla\times A }{\partial t} = 0 \rightarrow E = -\frac{1}{c} \dfrac{\partial A}{\partial t} - \nabla \phi
\end{equation}

$\nabla \phi$ kann als Integrationskonstante angesehen werden und $\phi$ entspricht dem skalaren Potential des Feldes.

Durch weiteres Einsetzen k"onnen die vier Maxwell-Gleichungen in zwei Gleichungen umgeschrieben werden:
\begin{equation}
\begin{split}
 \nabla^2 \phi + \frac{1}{c} \dfrac{\partial \nabla A}{\partial t} &= -\rho \\
 \nabla^2 A - \frac{1}{c^2} \frac{\partial^2 A }{\partial t^2} - \nabla \left( \nabla \cdot A + \frac{1}{c} \frac{\partial \phi }{\partial t} \right) &= - \frac{1}{c} J
\end{split}
\end{equation}

dabei gelten die Korrespondenzen:
\begin{equation}
\begin{split}
B &= \nabla\times A \\
E &= -\frac{1}{c} \dfrac{\partial A}{\partial t} - \nabla \phi
\end{split}
\end{equation}

Es kann gezeigt werden, dass $\phi$ durch eine Eichtransformation (Siehe \ref{section:eichtransformation}) geeignet gew"ahlt werden kann, damit:

\begin{equation}
\nabla \cdot A + \frac{1}{c} \frac{\partial \phi }{\partial t} = 0
\end{equation}

Dadurch werden die zwei gekoppelten Gleichungen entkoppelt und es gilt:
\begin{equation}
\begin{split}
\nabla^2 \phi - \frac{1}{c^2} \dfrac{\partial^2 \nabla \phi}{\partial t^2} &= -\rho \\
\nabla^2 A - \frac{1}{c^2} \frac{\partial^2 A }{\partial t^2} &= - \frac{1}{c} J
\end{split}
\end{equation}

F"ur weiter wollen ein Feld im Vakkum betrachten. Hierf"ur gilt $J = 0$, da keine Leiter vorhanden sind.
In einem Transversalfeld im Vakkum gilt zudem $\nabla \cdot A = 0$. (ToDo: ??) Dadurch vereinfachen sich die gekoppelten Differentialgleichung zu einer Differentialgleichung in A:
\begin{equation}
\nabla^2 A - \frac{1}{c^2} \frac{\partial^2 A }{\partial t^2} = 0
\end{equation}

\section{Von der Welle zu gekoppelten Oszillatoren}
L"osungen dieser Gleichung f"ur periodische Randbedingungen und $t=0$ in einer Box mit Seitenl"ange $L = V^{1/3}$ sind durch die Fourier Reihe gegeben:

\begin{equation}
A(x,0) = \frac{1}{\sqrt{V}} \sum_K \sum_{\alpha=1,2} (c_{k,\alpha}(0) \epsilon^{(\alpha)} e^{ikx} + c^*_{k,\alpha}(0) \epsilon^{(\alpha)} e^{-ikx})
\end{equation}

oder durch setzen von $u_{k,\alpha}(x) = \epsilon^{(\alpha)} e^{ikx}$:
\begin{equation}
A(x,0) = \frac{1}{\sqrt{V}} \sum_K \sum_{\alpha=1,2} (c_{k,\alpha}(0)u_{k,\alpha}(x) + c^*_{k,\alpha}(0) u^*_{k,\alpha}(x))
\end{equation}

Wenn diese Gleichung ausgeschrieben wird, sieht man, dass $A(x,t)$ durch diese Wahl f"ur alle $c_{k,\alpha}(t)$ reell bleibt:
\begin{equation}
(a + ib)(\cos kx + i \sin kx ) + (a - ib)(\cos kx - i \sin kx ) = 2 ( a \cos kx - b \sin kx )
\end{equation}
%=a \cos kx + ib \cos kx + ia \sin kx - b \sin kx + a \cos kx - ib \cos kx - ia \sin kx - b \sin kx

$k$ ist der Ausbreitungsvektor der Welle und zeigt in die Ausbreitungsrichtung. $\epsilon^{(\alpha)}$ ist die Polarisation. Dabei wird vorausgesetzt, dass $(\epsilon^{(1)}, \epsilon^{(2)} , k/|k|)$ ein orthogonales Rechtssystem aus Einheitsvektoren bilden.

Da $\epsilon^{(\alpha)}$ und $k$ orthogonal sind gilt dabei auch automatisch:

\begin{equation}
\nabla \cdot A = \frac{1}{\sqrt{V}} \sum_K \sum_{\alpha=1,2} (i c_{k,\alpha}(0) \underbrace{\epsilon^{(\alpha)} k}_{=0} e^{ikx} - i c^*_{k,\alpha}(0) \underbrace{\epsilon^{(\alpha)} k}_{=0} e^{-ikx}) = 0
\end{equation}

Weiterhin gilt durch wegen der Orthogonalit"at auch:
\begin{equation}
\begin{split}
\dfrac{1}{A} \int c_{k,\alpha} \cdot c^*_{k',\alpha'} d^3 x &= \delta_{kk'}\delta{aa'} \\
\dfrac{1}{A} \int c_{k,\alpha} \cdot c_{k',\alpha'} d^3 x &= 0 \\
\dfrac{1}{A} \int c^*_{k,\alpha} \cdot c^*_{k',\alpha'} d^3 x &= 0
\end{split}
\end{equation}

Um $A(x,t)$ zu erhalten, wird:
\begin{equation}
c_{k,\alpha}(t) = c_{k,\alpha}(0) e^{-i \omega t}
\end{equation}

Dabei ist:
\begin{equation}
\begin{split}
\omega=|k|c \\
\lambda = \frac{2 \pi}{|k|}
\end{split}
\end{equation}

Die komplette Wellengleichung wird somit:
\begin{equation}
A(x,t) = \frac{1}{\sqrt{V}} \sum_K \sum_{\alpha=1,2} (c_{k,\alpha}(0) \epsilon^{(\alpha)} e^{i (kx - \omega t)} + c^*_{k,\alpha}(0) \epsilon^{(\alpha)} e^{-i(kx - \omega t)})
\end{equation}

Die Hamilton-Funktion einer elektromagnetischen Welle ist gegeben durch:
\begin{equation}
\begin{split}
H &= \frac{1}{2} \int (|B|^2 + |E|^2) d^3 x \\
	&= \frac{1}{2} \int (| \nabla\times A |^2 + \left| \frac{1}{c} \dfrac{\partial A}{\partial t} \right|^2) d^3 x 
\end{split}
\end{equation}

Es kann gezeigt werden, dass die L"osung dieses Integrals gegeben ist durch:
\begin{equation}
H = \sum_K \sum_{\alpha=1,2} 2 \left(\frac{\omega}{c}\right)^2 c^*_{k,\alpha}(t) c_{k,\alpha}(t)
\end{equation}

Durch folgende Definition:
\begin{equation}
Q_{k,\alpha} = \frac{1}{c}(c_{k,\alpha}(t) + c^*_{k,\alpha}(t)) \quad P_{k,\alpha} = -\frac{i\omega}{c}(c_{k,\alpha}(t) - c^*_{k,\alpha}(t)) 
\end{equation}

wird die Hamilton-Funktion zu:
\begin{equation} \label{fq:hamilton}
\begin{split}
H &= \sum_K \sum_{\alpha=1,2} 2 \left(\frac{\omega}{c}\right)^2 \left[ \frac{c(\omega Q_{k,\alpha} - i P_{k,\alpha})}{2 \omega} \right] \left[ \frac{c(\omega Q_{k,\alpha} + i P_{k,\alpha})}{2 \omega} \right] \\
&= \sum_K \sum_{\alpha=1,2} \frac{1}{2} (P_{k,\alpha}^2 + \omega^2 Q_{k,\alpha}^2)
\end{split}
\end{equation}

Hier sieh man nun, dass es m"oglich ist, eine Welle durch unabh"angige Oszillatoren dar zu stellen.

$Q_{k,\alpha}$ und $P_{k,\alpha}$ k"onnen nun als Koordinaten und Impulse der einzelnen Oszillatoren aufgefasst werden:
\begin{equation}
\dfrac{\partial H}{\partial Q_{k,\alpha}} = -\dot{P}_{k,\alpha} \quad \dfrac{\partial H}{\partial P_{k,\alpha}} = \dot{Q}_{k,\alpha}
\end{equation}

\section{Quantisierung der Welle}

Wie beim harmonischen Oszillator können $Q_{k,\alpha}$ und $P_{k,\alpha}$ nun als Opperatoren aufgefasst werden. Die Vertauschungsrelationen werden dabei zu:
\begin{equation}
\begin{split}
[Q_{k,\alpha}, P_{k',\alpha'}] &= i \hbar \delta_{kk'}\delta_{aa'} \\
[Q_{k,\alpha}, Q_{k',\alpha'}] &= 0 \\
[P_{k,\alpha}, P_{k',\alpha'}] &= 0
\end{split}
\end{equation}

Wir definieren die Operatoren:
\begin{equation}
\begin{split}
a_{k,\alpha} &= (1/\sqrt{2 \hbar \omega})(\omega Q_{k,\alpha} + iP_{k,\alpha})) \\
a^+_{k,\alpha} &= (1/\sqrt{2 \hbar \omega})(\omega Q_{k,\alpha} - iP_{k,\alpha}))\\
N_{k,\alpha} &= a^+_{k,\alpha} a_{k,\alpha}
\end{split}
\end{equation}

Ein Vergleich mit \ref{fq:hamilton} liefert:
\begin{equation}
 c_{k,\alpha} \rightarrow c \sqrt{\hbar/2 \omega} \, a_{k,\alpha} \quad c^*_{k,\alpha} \rightarrow c \sqrt{\hbar/2 \omega} \, a^+_{k,\alpha}
\end{equation}
Somit entsprechen diese Operatoren den Fourier-Koeffizienten.

Die Kommentatoren f"ur diese Operatoren sind:
\begin{equation}
\begin{split}
[a_{k,\alpha} , a^+_{k',\alpha'}] &= - \frac{i}{2 \hbar} [Q_{k,\alpha}, P_{k',\alpha'}] + \frac{i}{2 \hbar} [P_{k,\alpha}, Q_{k',\alpha'}] \\
	 &= \delta_{kk'}\delta_{aa'} \\
[a_{k,\alpha} , a_{k',\alpha'}] &= [a^+_{k,\alpha} , a^+_{k',\alpha'}] \\
	 &= 0 \\
[a_{k,\alpha} , N_{k',\alpha'}] &= [a_{k,\alpha} , a^+_{k',\alpha'}]a_{k',\alpha'} - a^+_{k',\alpha'}[a_{k',\alpha'} , a_{k,\alpha}]\\
	&= \delta_{kk'}\delta_{aa'} a_{k,\alpha} \\
[a^+_{k,\alpha} , N_{k',\alpha'}] &= -\delta_{kk'}\delta_{aa'} a^+_{k,\alpha}
\end{split}
\end{equation}


\chapter{Feldquantisierung\label{chapter:feldquantisierung}}
\lhead{Feldquantisierung}
\begin{refsection}
\chapterauthor{Hannes Diethelm}

\printbibliography[heading=subbibliography]
\end{refsection}

\section{Maxwell-Gleichungen und elektromagnetische Wellen}

Hilfreich dazu ist auch die Beschreibung von Magnetfeldern in Kapitel \ref{chapter:magnetfeld}. In diese Kapitel wird der Gradient durch $\nabla$ ersetzt \cite{fq:nabla}. Dadurch k"onnen die Gleichungen einfacher geschrieben werden. 

Die in der Elektrotechnik wohl bekannten Maxwell-Gleichungen in SI Einheiten lauten:
\begin{equation}
\begin{split}
\nabla\cdot E &= \frac{\rho}{\varepsilon_0} \\
\nabla\times B &= \mu_0( J  + \varepsilon_0\frac{\partial E}{\partial t}) \\
\nabla\cdot B &=0 \\
\nabla\times E &= -\frac{\partial B }{\partial t}\\
\end{split}
\end{equation}

Dieses Einheitensystem is willk"urlich \cite{fq:em_units}. Im Heaviside-
Lorentz System, das von nun an verwendet wird, lauten die Gleichungen:
\begin{equation}
\begin{split}
\nabla\cdot E &= \rho \\
\nabla\times B &= \frac{1}{c}( J  + \frac{\partial E}{\partial t}) \\
\nabla\cdot B &=0 \\
\nabla\times E &= -\frac{1}{c} \frac{\partial B }{\partial t}\\
\end{split}
\end{equation}

Da $\nabla \cdot B = 0 $ gilt k"onnen diese Gleichungen durch folgende Substitution umformuliert werden:
\begin{equation}
B = \nabla\times A 
\end{equation}

Dadurch gilt $\nabla \cdot B = 0 $ automatisch:
\begin{equation}
\nabla \cdot B = 0 \rightarrow \nabla \cdot ( \nabla\times A ) = 0 \text{ gilt f"ur jedes A! }
\end{equation}

Durch Einsetzen erh"alt man die Gleichung f"ur E:
\begin{equation}
\nabla\times E + \frac{1}{c} \frac{\partial B }{\partial t} = 0
\rightarrow \nabla\times E + \frac{1}{c} \frac{\partial \nabla\times A }{\partial t} = 0 \rightarrow E = -\frac{1}{c} \dfrac{\partial A}{\partial t} - \nabla \phi
\end{equation}

$\nabla \phi$ kann als Integrationskonstante angesehen werden und $\phi$ entspricht dem skalaren Potential des Feldes.

Durch weiteres Einsetzen k"onnen die vier Maxwell-Gleichungen in zwei Gleichungen umgeschrieben werden:
\begin{equation}
\begin{split}
 \nabla^2 \phi + \frac{1}{c} \dfrac{\partial \nabla A}{\partial t} &= -\rho \\
 \nabla^2 A - \frac{1}{c^2} \frac{\partial^2 A }{\partial t^2} - \nabla \left( \nabla \cdot A + \frac{1}{c} \frac{\partial \phi }{\partial t} \right) &= - \frac{1}{c} J
\end{split}
\end{equation}

dabei gelten die Korrespondenzen:
\begin{equation}
\begin{split}
B &= \nabla\times A \\
E &= -\frac{1}{c} \dfrac{\partial A}{\partial t} - \nabla \phi
\end{split}
\end{equation}

Es kann gezeigt werden, dass $\phi$ durch eine Eichtransformation (Siehe \ref{section:eichtransformation}) geeignet gew"ahlt werden kann, damit:

\begin{equation}
\nabla \cdot A + \frac{1}{c} \frac{\partial \phi }{\partial t} = 0
\end{equation}

Dadurch werden die zwei gekoppelten Gleichungen entkoppelt und es gilt:
\begin{equation}
\begin{split}
\nabla^2 \phi - \frac{1}{c^2} \dfrac{\partial^2 \nabla \phi}{\partial t^2} &= -\rho \\
\nabla^2 A - \frac{1}{c^2} \frac{\partial^2 A }{\partial t^2} &= - \frac{1}{c} J
\end{split}
\end{equation}

F"ur weiter wollen ein Feld im Vakkum betrachten. Hierf"ur gilt $J = 0$, da keine Leiter vorhanden sind.
In einem Transversalfeld im Vakkum gilt zudem $\nabla \cdot A = 0$. (ToDo: ??) Dadurch vereinfachen sich die gekoppelten Differentialgleichung zu einer Differentialgleichung in A:
\begin{equation}
\nabla^2 A - \frac{1}{c^2} \frac{\partial^2 A }{\partial t^2} = 0
\end{equation}

\section{Von der Welle zu gekoppelten Oszillatoren}
L"osungen dieser Gleichung f"ur periodische Randbedingungen und $t=0$ in einer Box mit Seitenl"ange $L = V^{1/3}$ sind durch die Fourier Reihe gegeben:

\begin{equation}
A(x,0) = \frac{1}{\sqrt{V}} \sum_K \sum_{\alpha=1,2} (c_{k,\alpha}(0) \epsilon^{(\alpha)} e^{ikx} + c^*_{k,\alpha}(0) \epsilon^{(\alpha)} e^{-ikx})
\end{equation}

oder durch setzen von $u_{k,\alpha}(x) = \epsilon^{(\alpha)} e^{ikx}$:
\begin{equation}
A(x,0) = \frac{1}{\sqrt{V}} \sum_K \sum_{\alpha=1,2} (c_{k,\alpha}(0)u_{k,\alpha}(x) + c^*_{k,\alpha}(0) u^*_{k,\alpha}(x))
\end{equation}

Wenn diese Gleichung ausgeschrieben wird, sieht man, dass $A(x,t)$ durch diese Wahl f"ur alle $c_{k,\alpha}(t)$ reell bleibt:
\begin{equation}
(a + ib)(\cos kx + i \sin kx ) + (a - ib)(\cos kx - i \sin kx ) = 2 ( a \cos kx - b \sin kx )
\end{equation}
%=a \cos kx + ib \cos kx + ia \sin kx - b \sin kx + a \cos kx - ib \cos kx - ia \sin kx - b \sin kx

$k$ ist der Ausbreitungsvektor der Welle und zeigt in die Ausbreitungsrichtung. $\epsilon^{(\alpha)}$ ist die Polarisation. Dabei wird vorausgesetzt, dass $(\epsilon^{(1)}, \epsilon^{(2)} , k/|k|)$ ein orthogonales Rechtssystem aus Einheitsvektoren bilden.

Da $\epsilon^{(\alpha)}$ und $k$ orthogonal sind gilt dabei auch automatisch:

\begin{equation}
\nabla \cdot A = \frac{1}{\sqrt{V}} \sum_K \sum_{\alpha=1,2} (i c_{k,\alpha}(0) \underbrace{\epsilon^{(\alpha)} k}_{=0} e^{ikx} - i c^*_{k,\alpha}(0) \underbrace{\epsilon^{(\alpha)} k}_{=0} e^{-ikx}) = 0
\end{equation}

Weiterhin gilt durch wegen der Orthogonalit"at auch:
\begin{equation}
\begin{split}
\dfrac{1}{A} \int c_{k,\alpha} \cdot c^*_{k',\alpha'} d^3 x &= \delta_{kk'}\delta{aa'} \\
\dfrac{1}{A} \int c_{k,\alpha} \cdot c_{k',\alpha'} d^3 x &= 0 \\
\dfrac{1}{A} \int c^*_{k,\alpha} \cdot c^*_{k',\alpha'} d^3 x &= 0
\end{split}
\end{equation}

Um $A(x,t)$ zu erhalten, wird:
\begin{equation}
c_{k,\alpha}(t) = c_{k,\alpha}(0) e^{-i \omega t}
\end{equation}

Dabei ist:
\begin{equation}
\begin{split}
\omega=|k|c \\
\lambda = \frac{2 \pi}{|k|}
\end{split}
\end{equation}

Die komplette Wellengleichung wird somit:
\begin{equation}
A(x,t) = \frac{1}{\sqrt{V}} \sum_K \sum_{\alpha=1,2} (c_{k,\alpha}(0) \epsilon^{(\alpha)} e^{i (kx - \omega t)} + c^*_{k,\alpha}(0) \epsilon^{(\alpha)} e^{-i(kx - \omega t)})
\end{equation}

Die Hamilton-Funktion einer elektromagnetischen Welle ist gegeben durch:
\begin{equation}
\begin{split}
H &= \frac{1}{2} \int (|B|^2 + |E|^2) d^3 x \\
	&= \frac{1}{2} \int (| \nabla\times A |^2 + \left| \frac{1}{c} \dfrac{\partial A}{\partial t} \right|^2) d^3 x 
\end{split}
\end{equation}

Es kann gezeigt werden, dass die L"osung dieses Integrals gegeben ist durch:
\begin{equation}
H = \sum_K \sum_{\alpha=1,2} 2 \left(\frac{\omega}{c}\right)^2 c^*_{k,\alpha}(t) c_{k,\alpha}(t)
\end{equation}

Durch folgende Definition:
\begin{equation}
Q_{k,\alpha} = \frac{1}{c}(c_{k,\alpha}(t) + c^*_{k,\alpha}(t)) \quad P_{k,\alpha} = -\frac{i\omega}{c}(c_{k,\alpha}(t) - c^*_{k,\alpha}(t)) 
\end{equation}

wird die Hamilton-Funktion zu:
\begin{equation} \label{fq:hamilton}
\begin{split}
H &= \sum_K \sum_{\alpha=1,2} 2 \left(\frac{\omega}{c}\right)^2 \left[ \frac{c(\omega Q_{k,\alpha} - i P_{k,\alpha})}{2 \omega} \right] \left[ \frac{c(\omega Q_{k,\alpha} + i P_{k,\alpha})}{2 \omega} \right] \\
&= \sum_K \sum_{\alpha=1,2} \frac{1}{2} (P_{k,\alpha}^2 + \omega^2 Q_{k,\alpha}^2)
\end{split}
\end{equation}

Hier sieh man nun, dass es m"oglich ist, eine Welle durch unabh"angige Oszillatoren dar zu stellen.

$Q_{k,\alpha}$ und $P_{k,\alpha}$ k"onnen nun als Koordinaten und Impulse der einzelnen Oszillatoren aufgefasst werden:
\begin{equation}
\dfrac{\partial H}{\partial Q_{k,\alpha}} = -\dot{P}_{k,\alpha} \quad \dfrac{\partial H}{\partial P_{k,\alpha}} = \dot{Q}_{k,\alpha}
\end{equation}

\section{Quantisierung der Welle}

Wie beim harmonischen Oszillator können $Q_{k,\alpha}$ und $P_{k,\alpha}$ nun als Opperatoren aufgefasst werden. Die Vertauschungsrelationen werden dabei zu:
\begin{equation}
\begin{split}
[Q_{k,\alpha}, P_{k',\alpha'}] &= i \hbar \delta_{kk'}\delta_{aa'} \\
[Q_{k,\alpha}, Q_{k',\alpha'}] &= 0 \\
[P_{k,\alpha}, P_{k',\alpha'}] &= 0
\end{split}
\end{equation}

Wir definieren die Operatoren:
\begin{equation}
\begin{split}
a_{k,\alpha} &= (1/\sqrt{2 \hbar \omega})(\omega Q_{k,\alpha} + iP_{k,\alpha})) \\
a^+_{k,\alpha} &= (1/\sqrt{2 \hbar \omega})(\omega Q_{k,\alpha} - iP_{k,\alpha}))\\
N_{k,\alpha} &= a^+_{k,\alpha} a_{k,\alpha}
\end{split}
\end{equation}

Ein Vergleich mit \ref{fq:hamilton} liefert:
\begin{equation}
 c_{k,\alpha} \rightarrow c \sqrt{\hbar/2 \omega} \, a_{k,\alpha} \quad c^*_{k,\alpha} \rightarrow c \sqrt{\hbar/2 \omega} \, a^+_{k,\alpha}
\end{equation}
Somit entsprechen diese Operatoren den Fourier-Koeffizienten.

Die Kommentatoren f"ur diese Operatoren sind:
\begin{equation}
\begin{split}
[a_{k,\alpha} , a^+_{k',\alpha'}] &= - \frac{i}{2 \hbar} [Q_{k,\alpha}, P_{k',\alpha'}] + \frac{i}{2 \hbar} [P_{k,\alpha}, Q_{k',\alpha'}] \\
	 &= \delta_{kk'}\delta_{aa'} \\
[a_{k,\alpha} , a_{k',\alpha'}] &= [a^+_{k,\alpha} , a^+_{k',\alpha'}] \\
	 &= 0 \\
[a_{k,\alpha} , N_{k',\alpha'}] &= [a_{k,\alpha} , a^+_{k',\alpha'}]a_{k',\alpha'} - a^+_{k',\alpha'}[a_{k',\alpha'} , a_{k,\alpha}]\\
	&= \delta_{kk'}\delta_{aa'} a_{k,\alpha} \\
[a^+_{k,\alpha} , N_{k',\alpha'}] &= -\delta_{kk'}\delta_{aa'} a^+_{k,\alpha}
\end{split}
\end{equation}


% Supraleitung
\chapter{Feldquantisierung\label{chapter:feldquantisierung}}
\lhead{Feldquantisierung}
\begin{refsection}
\chapterauthor{Hannes Diethelm}

\printbibliography[heading=subbibliography]
\end{refsection}

\section{Maxwell-Gleichungen und elektromagnetische Wellen}

Hilfreich dazu ist auch die Beschreibung von Magnetfeldern in Kapitel \ref{chapter:magnetfeld}. In diese Kapitel wird der Gradient durch $\nabla$ ersetzt \cite{fq:nabla}. Dadurch k"onnen die Gleichungen einfacher geschrieben werden. 

Die in der Elektrotechnik wohl bekannten Maxwell-Gleichungen in SI Einheiten lauten:
\begin{equation}
\begin{split}
\nabla\cdot E &= \frac{\rho}{\varepsilon_0} \\
\nabla\times B &= \mu_0( J  + \varepsilon_0\frac{\partial E}{\partial t}) \\
\nabla\cdot B &=0 \\
\nabla\times E &= -\frac{\partial B }{\partial t}\\
\end{split}
\end{equation}

Dieses Einheitensystem is willk"urlich \cite{fq:em_units}. Im Heaviside-
Lorentz System, das von nun an verwendet wird, lauten die Gleichungen:
\begin{equation}
\begin{split}
\nabla\cdot E &= \rho \\
\nabla\times B &= \frac{1}{c}( J  + \frac{\partial E}{\partial t}) \\
\nabla\cdot B &=0 \\
\nabla\times E &= -\frac{1}{c} \frac{\partial B }{\partial t}\\
\end{split}
\end{equation}

Da $\nabla \cdot B = 0 $ gilt k"onnen diese Gleichungen durch folgende Substitution umformuliert werden:
\begin{equation}
B = \nabla\times A 
\end{equation}

Dadurch gilt $\nabla \cdot B = 0 $ automatisch:
\begin{equation}
\nabla \cdot B = 0 \rightarrow \nabla \cdot ( \nabla\times A ) = 0 \text{ gilt f"ur jedes A! }
\end{equation}

Durch Einsetzen erh"alt man die Gleichung f"ur E:
\begin{equation}
\nabla\times E + \frac{1}{c} \frac{\partial B }{\partial t} = 0
\rightarrow \nabla\times E + \frac{1}{c} \frac{\partial \nabla\times A }{\partial t} = 0 \rightarrow E = -\frac{1}{c} \dfrac{\partial A}{\partial t} - \nabla \phi
\end{equation}

$\nabla \phi$ kann als Integrationskonstante angesehen werden und $\phi$ entspricht dem skalaren Potential des Feldes.

Durch weiteres Einsetzen k"onnen die vier Maxwell-Gleichungen in zwei Gleichungen umgeschrieben werden:
\begin{equation}
\begin{split}
 \nabla^2 \phi + \frac{1}{c} \dfrac{\partial \nabla A}{\partial t} &= -\rho \\
 \nabla^2 A - \frac{1}{c^2} \frac{\partial^2 A }{\partial t^2} - \nabla \left( \nabla \cdot A + \frac{1}{c} \frac{\partial \phi }{\partial t} \right) &= - \frac{1}{c} J
\end{split}
\end{equation}

dabei gelten die Korrespondenzen:
\begin{equation}
\begin{split}
B &= \nabla\times A \\
E &= -\frac{1}{c} \dfrac{\partial A}{\partial t} - \nabla \phi
\end{split}
\end{equation}

Es kann gezeigt werden, dass $\phi$ durch eine Eichtransformation (Siehe \ref{section:eichtransformation}) geeignet gew"ahlt werden kann, damit:

\begin{equation}
\nabla \cdot A + \frac{1}{c} \frac{\partial \phi }{\partial t} = 0
\end{equation}

Dadurch werden die zwei gekoppelten Gleichungen entkoppelt und es gilt:
\begin{equation}
\begin{split}
\nabla^2 \phi - \frac{1}{c^2} \dfrac{\partial^2 \nabla \phi}{\partial t^2} &= -\rho \\
\nabla^2 A - \frac{1}{c^2} \frac{\partial^2 A }{\partial t^2} &= - \frac{1}{c} J
\end{split}
\end{equation}

F"ur weiter wollen ein Feld im Vakkum betrachten. Hierf"ur gilt $J = 0$, da keine Leiter vorhanden sind.
In einem Transversalfeld im Vakkum gilt zudem $\nabla \cdot A = 0$. (ToDo: ??) Dadurch vereinfachen sich die gekoppelten Differentialgleichung zu einer Differentialgleichung in A:
\begin{equation}
\nabla^2 A - \frac{1}{c^2} \frac{\partial^2 A }{\partial t^2} = 0
\end{equation}

\section{Von der Welle zu gekoppelten Oszillatoren}
L"osungen dieser Gleichung f"ur periodische Randbedingungen und $t=0$ in einer Box mit Seitenl"ange $L = V^{1/3}$ sind durch die Fourier Reihe gegeben:

\begin{equation}
A(x,0) = \frac{1}{\sqrt{V}} \sum_K \sum_{\alpha=1,2} (c_{k,\alpha}(0) \epsilon^{(\alpha)} e^{ikx} + c^*_{k,\alpha}(0) \epsilon^{(\alpha)} e^{-ikx})
\end{equation}

oder durch setzen von $u_{k,\alpha}(x) = \epsilon^{(\alpha)} e^{ikx}$:
\begin{equation}
A(x,0) = \frac{1}{\sqrt{V}} \sum_K \sum_{\alpha=1,2} (c_{k,\alpha}(0)u_{k,\alpha}(x) + c^*_{k,\alpha}(0) u^*_{k,\alpha}(x))
\end{equation}

Wenn diese Gleichung ausgeschrieben wird, sieht man, dass $A(x,t)$ durch diese Wahl f"ur alle $c_{k,\alpha}(t)$ reell bleibt:
\begin{equation}
(a + ib)(\cos kx + i \sin kx ) + (a - ib)(\cos kx - i \sin kx ) = 2 ( a \cos kx - b \sin kx )
\end{equation}
%=a \cos kx + ib \cos kx + ia \sin kx - b \sin kx + a \cos kx - ib \cos kx - ia \sin kx - b \sin kx

$k$ ist der Ausbreitungsvektor der Welle und zeigt in die Ausbreitungsrichtung. $\epsilon^{(\alpha)}$ ist die Polarisation. Dabei wird vorausgesetzt, dass $(\epsilon^{(1)}, \epsilon^{(2)} , k/|k|)$ ein orthogonales Rechtssystem aus Einheitsvektoren bilden.

Da $\epsilon^{(\alpha)}$ und $k$ orthogonal sind gilt dabei auch automatisch:

\begin{equation}
\nabla \cdot A = \frac{1}{\sqrt{V}} \sum_K \sum_{\alpha=1,2} (i c_{k,\alpha}(0) \underbrace{\epsilon^{(\alpha)} k}_{=0} e^{ikx} - i c^*_{k,\alpha}(0) \underbrace{\epsilon^{(\alpha)} k}_{=0} e^{-ikx}) = 0
\end{equation}

Weiterhin gilt durch wegen der Orthogonalit"at auch:
\begin{equation}
\begin{split}
\dfrac{1}{A} \int c_{k,\alpha} \cdot c^*_{k',\alpha'} d^3 x &= \delta_{kk'}\delta{aa'} \\
\dfrac{1}{A} \int c_{k,\alpha} \cdot c_{k',\alpha'} d^3 x &= 0 \\
\dfrac{1}{A} \int c^*_{k,\alpha} \cdot c^*_{k',\alpha'} d^3 x &= 0
\end{split}
\end{equation}

Um $A(x,t)$ zu erhalten, wird:
\begin{equation}
c_{k,\alpha}(t) = c_{k,\alpha}(0) e^{-i \omega t}
\end{equation}

Dabei ist:
\begin{equation}
\begin{split}
\omega=|k|c \\
\lambda = \frac{2 \pi}{|k|}
\end{split}
\end{equation}

Die komplette Wellengleichung wird somit:
\begin{equation}
A(x,t) = \frac{1}{\sqrt{V}} \sum_K \sum_{\alpha=1,2} (c_{k,\alpha}(0) \epsilon^{(\alpha)} e^{i (kx - \omega t)} + c^*_{k,\alpha}(0) \epsilon^{(\alpha)} e^{-i(kx - \omega t)})
\end{equation}

Die Hamilton-Funktion einer elektromagnetischen Welle ist gegeben durch:
\begin{equation}
\begin{split}
H &= \frac{1}{2} \int (|B|^2 + |E|^2) d^3 x \\
	&= \frac{1}{2} \int (| \nabla\times A |^2 + \left| \frac{1}{c} \dfrac{\partial A}{\partial t} \right|^2) d^3 x 
\end{split}
\end{equation}

Es kann gezeigt werden, dass die L"osung dieses Integrals gegeben ist durch:
\begin{equation}
H = \sum_K \sum_{\alpha=1,2} 2 \left(\frac{\omega}{c}\right)^2 c^*_{k,\alpha}(t) c_{k,\alpha}(t)
\end{equation}

Durch folgende Definition:
\begin{equation}
Q_{k,\alpha} = \frac{1}{c}(c_{k,\alpha}(t) + c^*_{k,\alpha}(t)) \quad P_{k,\alpha} = -\frac{i\omega}{c}(c_{k,\alpha}(t) - c^*_{k,\alpha}(t)) 
\end{equation}

wird die Hamilton-Funktion zu:
\begin{equation} \label{fq:hamilton}
\begin{split}
H &= \sum_K \sum_{\alpha=1,2} 2 \left(\frac{\omega}{c}\right)^2 \left[ \frac{c(\omega Q_{k,\alpha} - i P_{k,\alpha})}{2 \omega} \right] \left[ \frac{c(\omega Q_{k,\alpha} + i P_{k,\alpha})}{2 \omega} \right] \\
&= \sum_K \sum_{\alpha=1,2} \frac{1}{2} (P_{k,\alpha}^2 + \omega^2 Q_{k,\alpha}^2)
\end{split}
\end{equation}

Hier sieh man nun, dass es m"oglich ist, eine Welle durch unabh"angige Oszillatoren dar zu stellen.

$Q_{k,\alpha}$ und $P_{k,\alpha}$ k"onnen nun als Koordinaten und Impulse der einzelnen Oszillatoren aufgefasst werden:
\begin{equation}
\dfrac{\partial H}{\partial Q_{k,\alpha}} = -\dot{P}_{k,\alpha} \quad \dfrac{\partial H}{\partial P_{k,\alpha}} = \dot{Q}_{k,\alpha}
\end{equation}

\section{Quantisierung der Welle}

Wie beim harmonischen Oszillator können $Q_{k,\alpha}$ und $P_{k,\alpha}$ nun als Opperatoren aufgefasst werden. Die Vertauschungsrelationen werden dabei zu:
\begin{equation}
\begin{split}
[Q_{k,\alpha}, P_{k',\alpha'}] &= i \hbar \delta_{kk'}\delta_{aa'} \\
[Q_{k,\alpha}, Q_{k',\alpha'}] &= 0 \\
[P_{k,\alpha}, P_{k',\alpha'}] &= 0
\end{split}
\end{equation}

Wir definieren die Operatoren:
\begin{equation}
\begin{split}
a_{k,\alpha} &= (1/\sqrt{2 \hbar \omega})(\omega Q_{k,\alpha} + iP_{k,\alpha})) \\
a^+_{k,\alpha} &= (1/\sqrt{2 \hbar \omega})(\omega Q_{k,\alpha} - iP_{k,\alpha}))\\
N_{k,\alpha} &= a^+_{k,\alpha} a_{k,\alpha}
\end{split}
\end{equation}

Ein Vergleich mit \ref{fq:hamilton} liefert:
\begin{equation}
 c_{k,\alpha} \rightarrow c \sqrt{\hbar/2 \omega} \, a_{k,\alpha} \quad c^*_{k,\alpha} \rightarrow c \sqrt{\hbar/2 \omega} \, a^+_{k,\alpha}
\end{equation}
Somit entsprechen diese Operatoren den Fourier-Koeffizienten.

Die Kommentatoren f"ur diese Operatoren sind:
\begin{equation}
\begin{split}
[a_{k,\alpha} , a^+_{k',\alpha'}] &= - \frac{i}{2 \hbar} [Q_{k,\alpha}, P_{k',\alpha'}] + \frac{i}{2 \hbar} [P_{k,\alpha}, Q_{k',\alpha'}] \\
	 &= \delta_{kk'}\delta_{aa'} \\
[a_{k,\alpha} , a_{k',\alpha'}] &= [a^+_{k,\alpha} , a^+_{k',\alpha'}] \\
	 &= 0 \\
[a_{k,\alpha} , N_{k',\alpha'}] &= [a_{k,\alpha} , a^+_{k',\alpha'}]a_{k',\alpha'} - a^+_{k',\alpha'}[a_{k',\alpha'} , a_{k,\alpha}]\\
	&= \delta_{kk'}\delta_{aa'} a_{k,\alpha} \\
[a^+_{k,\alpha} , N_{k',\alpha'}] &= -\delta_{kk'}\delta_{aa'} a^+_{k,\alpha}
\end{split}
\end{equation}


\chapter{Feldquantisierung\label{chapter:feldquantisierung}}
\lhead{Feldquantisierung}
\begin{refsection}
\chapterauthor{Hannes Diethelm}

\printbibliography[heading=subbibliography]
\end{refsection}

\section{Maxwell-Gleichungen und elektromagnetische Wellen}

Hilfreich dazu ist auch die Beschreibung von Magnetfeldern in Kapitel \ref{chapter:magnetfeld}. In diese Kapitel wird der Gradient durch $\nabla$ ersetzt \cite{fq:nabla}. Dadurch k"onnen die Gleichungen einfacher geschrieben werden. 

Die in der Elektrotechnik wohl bekannten Maxwell-Gleichungen in SI Einheiten lauten:
\begin{equation}
\begin{split}
\nabla\cdot E &= \frac{\rho}{\varepsilon_0} \\
\nabla\times B &= \mu_0( J  + \varepsilon_0\frac{\partial E}{\partial t}) \\
\nabla\cdot B &=0 \\
\nabla\times E &= -\frac{\partial B }{\partial t}\\
\end{split}
\end{equation}

Dieses Einheitensystem is willk"urlich \cite{fq:em_units}. Im Heaviside-
Lorentz System, das von nun an verwendet wird, lauten die Gleichungen:
\begin{equation}
\begin{split}
\nabla\cdot E &= \rho \\
\nabla\times B &= \frac{1}{c}( J  + \frac{\partial E}{\partial t}) \\
\nabla\cdot B &=0 \\
\nabla\times E &= -\frac{1}{c} \frac{\partial B }{\partial t}\\
\end{split}
\end{equation}

Da $\nabla \cdot B = 0 $ gilt k"onnen diese Gleichungen durch folgende Substitution umformuliert werden:
\begin{equation}
B = \nabla\times A 
\end{equation}

Dadurch gilt $\nabla \cdot B = 0 $ automatisch:
\begin{equation}
\nabla \cdot B = 0 \rightarrow \nabla \cdot ( \nabla\times A ) = 0 \text{ gilt f"ur jedes A! }
\end{equation}

Durch Einsetzen erh"alt man die Gleichung f"ur E:
\begin{equation}
\nabla\times E + \frac{1}{c} \frac{\partial B }{\partial t} = 0
\rightarrow \nabla\times E + \frac{1}{c} \frac{\partial \nabla\times A }{\partial t} = 0 \rightarrow E = -\frac{1}{c} \dfrac{\partial A}{\partial t} - \nabla \phi
\end{equation}

$\nabla \phi$ kann als Integrationskonstante angesehen werden und $\phi$ entspricht dem skalaren Potential des Feldes.

Durch weiteres Einsetzen k"onnen die vier Maxwell-Gleichungen in zwei Gleichungen umgeschrieben werden:
\begin{equation}
\begin{split}
 \nabla^2 \phi + \frac{1}{c} \dfrac{\partial \nabla A}{\partial t} &= -\rho \\
 \nabla^2 A - \frac{1}{c^2} \frac{\partial^2 A }{\partial t^2} - \nabla \left( \nabla \cdot A + \frac{1}{c} \frac{\partial \phi }{\partial t} \right) &= - \frac{1}{c} J
\end{split}
\end{equation}

dabei gelten die Korrespondenzen:
\begin{equation}
\begin{split}
B &= \nabla\times A \\
E &= -\frac{1}{c} \dfrac{\partial A}{\partial t} - \nabla \phi
\end{split}
\end{equation}

Es kann gezeigt werden, dass $\phi$ durch eine Eichtransformation (Siehe \ref{section:eichtransformation}) geeignet gew"ahlt werden kann, damit:

\begin{equation}
\nabla \cdot A + \frac{1}{c} \frac{\partial \phi }{\partial t} = 0
\end{equation}

Dadurch werden die zwei gekoppelten Gleichungen entkoppelt und es gilt:
\begin{equation}
\begin{split}
\nabla^2 \phi - \frac{1}{c^2} \dfrac{\partial^2 \nabla \phi}{\partial t^2} &= -\rho \\
\nabla^2 A - \frac{1}{c^2} \frac{\partial^2 A }{\partial t^2} &= - \frac{1}{c} J
\end{split}
\end{equation}

F"ur weiter wollen ein Feld im Vakkum betrachten. Hierf"ur gilt $J = 0$, da keine Leiter vorhanden sind.
In einem Transversalfeld im Vakkum gilt zudem $\nabla \cdot A = 0$. (ToDo: ??) Dadurch vereinfachen sich die gekoppelten Differentialgleichung zu einer Differentialgleichung in A:
\begin{equation}
\nabla^2 A - \frac{1}{c^2} \frac{\partial^2 A }{\partial t^2} = 0
\end{equation}

\section{Von der Welle zu gekoppelten Oszillatoren}
L"osungen dieser Gleichung f"ur periodische Randbedingungen und $t=0$ in einer Box mit Seitenl"ange $L = V^{1/3}$ sind durch die Fourier Reihe gegeben:

\begin{equation}
A(x,0) = \frac{1}{\sqrt{V}} \sum_K \sum_{\alpha=1,2} (c_{k,\alpha}(0) \epsilon^{(\alpha)} e^{ikx} + c^*_{k,\alpha}(0) \epsilon^{(\alpha)} e^{-ikx})
\end{equation}

oder durch setzen von $u_{k,\alpha}(x) = \epsilon^{(\alpha)} e^{ikx}$:
\begin{equation}
A(x,0) = \frac{1}{\sqrt{V}} \sum_K \sum_{\alpha=1,2} (c_{k,\alpha}(0)u_{k,\alpha}(x) + c^*_{k,\alpha}(0) u^*_{k,\alpha}(x))
\end{equation}

Wenn diese Gleichung ausgeschrieben wird, sieht man, dass $A(x,t)$ durch diese Wahl f"ur alle $c_{k,\alpha}(t)$ reell bleibt:
\begin{equation}
(a + ib)(\cos kx + i \sin kx ) + (a - ib)(\cos kx - i \sin kx ) = 2 ( a \cos kx - b \sin kx )
\end{equation}
%=a \cos kx + ib \cos kx + ia \sin kx - b \sin kx + a \cos kx - ib \cos kx - ia \sin kx - b \sin kx

$k$ ist der Ausbreitungsvektor der Welle und zeigt in die Ausbreitungsrichtung. $\epsilon^{(\alpha)}$ ist die Polarisation. Dabei wird vorausgesetzt, dass $(\epsilon^{(1)}, \epsilon^{(2)} , k/|k|)$ ein orthogonales Rechtssystem aus Einheitsvektoren bilden.

Da $\epsilon^{(\alpha)}$ und $k$ orthogonal sind gilt dabei auch automatisch:

\begin{equation}
\nabla \cdot A = \frac{1}{\sqrt{V}} \sum_K \sum_{\alpha=1,2} (i c_{k,\alpha}(0) \underbrace{\epsilon^{(\alpha)} k}_{=0} e^{ikx} - i c^*_{k,\alpha}(0) \underbrace{\epsilon^{(\alpha)} k}_{=0} e^{-ikx}) = 0
\end{equation}

Weiterhin gilt durch wegen der Orthogonalit"at auch:
\begin{equation}
\begin{split}
\dfrac{1}{A} \int c_{k,\alpha} \cdot c^*_{k',\alpha'} d^3 x &= \delta_{kk'}\delta{aa'} \\
\dfrac{1}{A} \int c_{k,\alpha} \cdot c_{k',\alpha'} d^3 x &= 0 \\
\dfrac{1}{A} \int c^*_{k,\alpha} \cdot c^*_{k',\alpha'} d^3 x &= 0
\end{split}
\end{equation}

Um $A(x,t)$ zu erhalten, wird:
\begin{equation}
c_{k,\alpha}(t) = c_{k,\alpha}(0) e^{-i \omega t}
\end{equation}

Dabei ist:
\begin{equation}
\begin{split}
\omega=|k|c \\
\lambda = \frac{2 \pi}{|k|}
\end{split}
\end{equation}

Die komplette Wellengleichung wird somit:
\begin{equation}
A(x,t) = \frac{1}{\sqrt{V}} \sum_K \sum_{\alpha=1,2} (c_{k,\alpha}(0) \epsilon^{(\alpha)} e^{i (kx - \omega t)} + c^*_{k,\alpha}(0) \epsilon^{(\alpha)} e^{-i(kx - \omega t)})
\end{equation}

Die Hamilton-Funktion einer elektromagnetischen Welle ist gegeben durch:
\begin{equation}
\begin{split}
H &= \frac{1}{2} \int (|B|^2 + |E|^2) d^3 x \\
	&= \frac{1}{2} \int (| \nabla\times A |^2 + \left| \frac{1}{c} \dfrac{\partial A}{\partial t} \right|^2) d^3 x 
\end{split}
\end{equation}

Es kann gezeigt werden, dass die L"osung dieses Integrals gegeben ist durch:
\begin{equation}
H = \sum_K \sum_{\alpha=1,2} 2 \left(\frac{\omega}{c}\right)^2 c^*_{k,\alpha}(t) c_{k,\alpha}(t)
\end{equation}

Durch folgende Definition:
\begin{equation}
Q_{k,\alpha} = \frac{1}{c}(c_{k,\alpha}(t) + c^*_{k,\alpha}(t)) \quad P_{k,\alpha} = -\frac{i\omega}{c}(c_{k,\alpha}(t) - c^*_{k,\alpha}(t)) 
\end{equation}

wird die Hamilton-Funktion zu:
\begin{equation} \label{fq:hamilton}
\begin{split}
H &= \sum_K \sum_{\alpha=1,2} 2 \left(\frac{\omega}{c}\right)^2 \left[ \frac{c(\omega Q_{k,\alpha} - i P_{k,\alpha})}{2 \omega} \right] \left[ \frac{c(\omega Q_{k,\alpha} + i P_{k,\alpha})}{2 \omega} \right] \\
&= \sum_K \sum_{\alpha=1,2} \frac{1}{2} (P_{k,\alpha}^2 + \omega^2 Q_{k,\alpha}^2)
\end{split}
\end{equation}

Hier sieh man nun, dass es m"oglich ist, eine Welle durch unabh"angige Oszillatoren dar zu stellen.

$Q_{k,\alpha}$ und $P_{k,\alpha}$ k"onnen nun als Koordinaten und Impulse der einzelnen Oszillatoren aufgefasst werden:
\begin{equation}
\dfrac{\partial H}{\partial Q_{k,\alpha}} = -\dot{P}_{k,\alpha} \quad \dfrac{\partial H}{\partial P_{k,\alpha}} = \dot{Q}_{k,\alpha}
\end{equation}

\section{Quantisierung der Welle}

Wie beim harmonischen Oszillator können $Q_{k,\alpha}$ und $P_{k,\alpha}$ nun als Opperatoren aufgefasst werden. Die Vertauschungsrelationen werden dabei zu:
\begin{equation}
\begin{split}
[Q_{k,\alpha}, P_{k',\alpha'}] &= i \hbar \delta_{kk'}\delta_{aa'} \\
[Q_{k,\alpha}, Q_{k',\alpha'}] &= 0 \\
[P_{k,\alpha}, P_{k',\alpha'}] &= 0
\end{split}
\end{equation}

Wir definieren die Operatoren:
\begin{equation}
\begin{split}
a_{k,\alpha} &= (1/\sqrt{2 \hbar \omega})(\omega Q_{k,\alpha} + iP_{k,\alpha})) \\
a^+_{k,\alpha} &= (1/\sqrt{2 \hbar \omega})(\omega Q_{k,\alpha} - iP_{k,\alpha}))\\
N_{k,\alpha} &= a^+_{k,\alpha} a_{k,\alpha}
\end{split}
\end{equation}

Ein Vergleich mit \ref{fq:hamilton} liefert:
\begin{equation}
 c_{k,\alpha} \rightarrow c \sqrt{\hbar/2 \omega} \, a_{k,\alpha} \quad c^*_{k,\alpha} \rightarrow c \sqrt{\hbar/2 \omega} \, a^+_{k,\alpha}
\end{equation}
Somit entsprechen diese Operatoren den Fourier-Koeffizienten.

Die Kommentatoren f"ur diese Operatoren sind:
\begin{equation}
\begin{split}
[a_{k,\alpha} , a^+_{k',\alpha'}] &= - \frac{i}{2 \hbar} [Q_{k,\alpha}, P_{k',\alpha'}] + \frac{i}{2 \hbar} [P_{k,\alpha}, Q_{k',\alpha'}] \\
	 &= \delta_{kk'}\delta_{aa'} \\
[a_{k,\alpha} , a_{k',\alpha'}] &= [a^+_{k,\alpha} , a^+_{k',\alpha'}] \\
	 &= 0 \\
[a_{k,\alpha} , N_{k',\alpha'}] &= [a_{k,\alpha} , a^+_{k',\alpha'}]a_{k',\alpha'} - a^+_{k',\alpha'}[a_{k',\alpha'} , a_{k,\alpha}]\\
	&= \delta_{kk'}\delta_{aa'} a_{k,\alpha} \\
[a^+_{k,\alpha} , N_{k',\alpha'}] &= -\delta_{kk'}\delta_{aa'} a^+_{k,\alpha}
\end{split}
\end{equation}


\chapter{Feldquantisierung\label{chapter:feldquantisierung}}
\lhead{Feldquantisierung}
\begin{refsection}
\chapterauthor{Hannes Diethelm}

\printbibliography[heading=subbibliography]
\end{refsection}

\section{Maxwell-Gleichungen und elektromagnetische Wellen}

Hilfreich dazu ist auch die Beschreibung von Magnetfeldern in Kapitel \ref{chapter:magnetfeld}. In diese Kapitel wird der Gradient durch $\nabla$ ersetzt \cite{fq:nabla}. Dadurch k"onnen die Gleichungen einfacher geschrieben werden. 

Die in der Elektrotechnik wohl bekannten Maxwell-Gleichungen in SI Einheiten lauten:
\begin{equation}
\begin{split}
\nabla\cdot E &= \frac{\rho}{\varepsilon_0} \\
\nabla\times B &= \mu_0( J  + \varepsilon_0\frac{\partial E}{\partial t}) \\
\nabla\cdot B &=0 \\
\nabla\times E &= -\frac{\partial B }{\partial t}\\
\end{split}
\end{equation}

Dieses Einheitensystem is willk"urlich \cite{fq:em_units}. Im Heaviside-
Lorentz System, das von nun an verwendet wird, lauten die Gleichungen:
\begin{equation}
\begin{split}
\nabla\cdot E &= \rho \\
\nabla\times B &= \frac{1}{c}( J  + \frac{\partial E}{\partial t}) \\
\nabla\cdot B &=0 \\
\nabla\times E &= -\frac{1}{c} \frac{\partial B }{\partial t}\\
\end{split}
\end{equation}

Da $\nabla \cdot B = 0 $ gilt k"onnen diese Gleichungen durch folgende Substitution umformuliert werden:
\begin{equation}
B = \nabla\times A 
\end{equation}

Dadurch gilt $\nabla \cdot B = 0 $ automatisch:
\begin{equation}
\nabla \cdot B = 0 \rightarrow \nabla \cdot ( \nabla\times A ) = 0 \text{ gilt f"ur jedes A! }
\end{equation}

Durch Einsetzen erh"alt man die Gleichung f"ur E:
\begin{equation}
\nabla\times E + \frac{1}{c} \frac{\partial B }{\partial t} = 0
\rightarrow \nabla\times E + \frac{1}{c} \frac{\partial \nabla\times A }{\partial t} = 0 \rightarrow E = -\frac{1}{c} \dfrac{\partial A}{\partial t} - \nabla \phi
\end{equation}

$\nabla \phi$ kann als Integrationskonstante angesehen werden und $\phi$ entspricht dem skalaren Potential des Feldes.

Durch weiteres Einsetzen k"onnen die vier Maxwell-Gleichungen in zwei Gleichungen umgeschrieben werden:
\begin{equation}
\begin{split}
 \nabla^2 \phi + \frac{1}{c} \dfrac{\partial \nabla A}{\partial t} &= -\rho \\
 \nabla^2 A - \frac{1}{c^2} \frac{\partial^2 A }{\partial t^2} - \nabla \left( \nabla \cdot A + \frac{1}{c} \frac{\partial \phi }{\partial t} \right) &= - \frac{1}{c} J
\end{split}
\end{equation}

dabei gelten die Korrespondenzen:
\begin{equation}
\begin{split}
B &= \nabla\times A \\
E &= -\frac{1}{c} \dfrac{\partial A}{\partial t} - \nabla \phi
\end{split}
\end{equation}

Es kann gezeigt werden, dass $\phi$ durch eine Eichtransformation (Siehe \ref{section:eichtransformation}) geeignet gew"ahlt werden kann, damit:

\begin{equation}
\nabla \cdot A + \frac{1}{c} \frac{\partial \phi }{\partial t} = 0
\end{equation}

Dadurch werden die zwei gekoppelten Gleichungen entkoppelt und es gilt:
\begin{equation}
\begin{split}
\nabla^2 \phi - \frac{1}{c^2} \dfrac{\partial^2 \nabla \phi}{\partial t^2} &= -\rho \\
\nabla^2 A - \frac{1}{c^2} \frac{\partial^2 A }{\partial t^2} &= - \frac{1}{c} J
\end{split}
\end{equation}

F"ur weiter wollen ein Feld im Vakkum betrachten. Hierf"ur gilt $J = 0$, da keine Leiter vorhanden sind.
In einem Transversalfeld im Vakkum gilt zudem $\nabla \cdot A = 0$. (ToDo: ??) Dadurch vereinfachen sich die gekoppelten Differentialgleichung zu einer Differentialgleichung in A:
\begin{equation}
\nabla^2 A - \frac{1}{c^2} \frac{\partial^2 A }{\partial t^2} = 0
\end{equation}

\section{Von der Welle zu gekoppelten Oszillatoren}
L"osungen dieser Gleichung f"ur periodische Randbedingungen und $t=0$ in einer Box mit Seitenl"ange $L = V^{1/3}$ sind durch die Fourier Reihe gegeben:

\begin{equation}
A(x,0) = \frac{1}{\sqrt{V}} \sum_K \sum_{\alpha=1,2} (c_{k,\alpha}(0) \epsilon^{(\alpha)} e^{ikx} + c^*_{k,\alpha}(0) \epsilon^{(\alpha)} e^{-ikx})
\end{equation}

oder durch setzen von $u_{k,\alpha}(x) = \epsilon^{(\alpha)} e^{ikx}$:
\begin{equation}
A(x,0) = \frac{1}{\sqrt{V}} \sum_K \sum_{\alpha=1,2} (c_{k,\alpha}(0)u_{k,\alpha}(x) + c^*_{k,\alpha}(0) u^*_{k,\alpha}(x))
\end{equation}

Wenn diese Gleichung ausgeschrieben wird, sieht man, dass $A(x,t)$ durch diese Wahl f"ur alle $c_{k,\alpha}(t)$ reell bleibt:
\begin{equation}
(a + ib)(\cos kx + i \sin kx ) + (a - ib)(\cos kx - i \sin kx ) = 2 ( a \cos kx - b \sin kx )
\end{equation}
%=a \cos kx + ib \cos kx + ia \sin kx - b \sin kx + a \cos kx - ib \cos kx - ia \sin kx - b \sin kx

$k$ ist der Ausbreitungsvektor der Welle und zeigt in die Ausbreitungsrichtung. $\epsilon^{(\alpha)}$ ist die Polarisation. Dabei wird vorausgesetzt, dass $(\epsilon^{(1)}, \epsilon^{(2)} , k/|k|)$ ein orthogonales Rechtssystem aus Einheitsvektoren bilden.

Da $\epsilon^{(\alpha)}$ und $k$ orthogonal sind gilt dabei auch automatisch:

\begin{equation}
\nabla \cdot A = \frac{1}{\sqrt{V}} \sum_K \sum_{\alpha=1,2} (i c_{k,\alpha}(0) \underbrace{\epsilon^{(\alpha)} k}_{=0} e^{ikx} - i c^*_{k,\alpha}(0) \underbrace{\epsilon^{(\alpha)} k}_{=0} e^{-ikx}) = 0
\end{equation}

Weiterhin gilt durch wegen der Orthogonalit"at auch:
\begin{equation}
\begin{split}
\dfrac{1}{A} \int c_{k,\alpha} \cdot c^*_{k',\alpha'} d^3 x &= \delta_{kk'}\delta{aa'} \\
\dfrac{1}{A} \int c_{k,\alpha} \cdot c_{k',\alpha'} d^3 x &= 0 \\
\dfrac{1}{A} \int c^*_{k,\alpha} \cdot c^*_{k',\alpha'} d^3 x &= 0
\end{split}
\end{equation}

Um $A(x,t)$ zu erhalten, wird:
\begin{equation}
c_{k,\alpha}(t) = c_{k,\alpha}(0) e^{-i \omega t}
\end{equation}

Dabei ist:
\begin{equation}
\begin{split}
\omega=|k|c \\
\lambda = \frac{2 \pi}{|k|}
\end{split}
\end{equation}

Die komplette Wellengleichung wird somit:
\begin{equation}
A(x,t) = \frac{1}{\sqrt{V}} \sum_K \sum_{\alpha=1,2} (c_{k,\alpha}(0) \epsilon^{(\alpha)} e^{i (kx - \omega t)} + c^*_{k,\alpha}(0) \epsilon^{(\alpha)} e^{-i(kx - \omega t)})
\end{equation}

Die Hamilton-Funktion einer elektromagnetischen Welle ist gegeben durch:
\begin{equation}
\begin{split}
H &= \frac{1}{2} \int (|B|^2 + |E|^2) d^3 x \\
	&= \frac{1}{2} \int (| \nabla\times A |^2 + \left| \frac{1}{c} \dfrac{\partial A}{\partial t} \right|^2) d^3 x 
\end{split}
\end{equation}

Es kann gezeigt werden, dass die L"osung dieses Integrals gegeben ist durch:
\begin{equation}
H = \sum_K \sum_{\alpha=1,2} 2 \left(\frac{\omega}{c}\right)^2 c^*_{k,\alpha}(t) c_{k,\alpha}(t)
\end{equation}

Durch folgende Definition:
\begin{equation}
Q_{k,\alpha} = \frac{1}{c}(c_{k,\alpha}(t) + c^*_{k,\alpha}(t)) \quad P_{k,\alpha} = -\frac{i\omega}{c}(c_{k,\alpha}(t) - c^*_{k,\alpha}(t)) 
\end{equation}

wird die Hamilton-Funktion zu:
\begin{equation} \label{fq:hamilton}
\begin{split}
H &= \sum_K \sum_{\alpha=1,2} 2 \left(\frac{\omega}{c}\right)^2 \left[ \frac{c(\omega Q_{k,\alpha} - i P_{k,\alpha})}{2 \omega} \right] \left[ \frac{c(\omega Q_{k,\alpha} + i P_{k,\alpha})}{2 \omega} \right] \\
&= \sum_K \sum_{\alpha=1,2} \frac{1}{2} (P_{k,\alpha}^2 + \omega^2 Q_{k,\alpha}^2)
\end{split}
\end{equation}

Hier sieh man nun, dass es m"oglich ist, eine Welle durch unabh"angige Oszillatoren dar zu stellen.

$Q_{k,\alpha}$ und $P_{k,\alpha}$ k"onnen nun als Koordinaten und Impulse der einzelnen Oszillatoren aufgefasst werden:
\begin{equation}
\dfrac{\partial H}{\partial Q_{k,\alpha}} = -\dot{P}_{k,\alpha} \quad \dfrac{\partial H}{\partial P_{k,\alpha}} = \dot{Q}_{k,\alpha}
\end{equation}

\section{Quantisierung der Welle}

Wie beim harmonischen Oszillator können $Q_{k,\alpha}$ und $P_{k,\alpha}$ nun als Opperatoren aufgefasst werden. Die Vertauschungsrelationen werden dabei zu:
\begin{equation}
\begin{split}
[Q_{k,\alpha}, P_{k',\alpha'}] &= i \hbar \delta_{kk'}\delta_{aa'} \\
[Q_{k,\alpha}, Q_{k',\alpha'}] &= 0 \\
[P_{k,\alpha}, P_{k',\alpha'}] &= 0
\end{split}
\end{equation}

Wir definieren die Operatoren:
\begin{equation}
\begin{split}
a_{k,\alpha} &= (1/\sqrt{2 \hbar \omega})(\omega Q_{k,\alpha} + iP_{k,\alpha})) \\
a^+_{k,\alpha} &= (1/\sqrt{2 \hbar \omega})(\omega Q_{k,\alpha} - iP_{k,\alpha}))\\
N_{k,\alpha} &= a^+_{k,\alpha} a_{k,\alpha}
\end{split}
\end{equation}

Ein Vergleich mit \ref{fq:hamilton} liefert:
\begin{equation}
 c_{k,\alpha} \rightarrow c \sqrt{\hbar/2 \omega} \, a_{k,\alpha} \quad c^*_{k,\alpha} \rightarrow c \sqrt{\hbar/2 \omega} \, a^+_{k,\alpha}
\end{equation}
Somit entsprechen diese Operatoren den Fourier-Koeffizienten.

Die Kommentatoren f"ur diese Operatoren sind:
\begin{equation}
\begin{split}
[a_{k,\alpha} , a^+_{k',\alpha'}] &= - \frac{i}{2 \hbar} [Q_{k,\alpha}, P_{k',\alpha'}] + \frac{i}{2 \hbar} [P_{k,\alpha}, Q_{k',\alpha'}] \\
	 &= \delta_{kk'}\delta_{aa'} \\
[a_{k,\alpha} , a_{k',\alpha'}] &= [a^+_{k,\alpha} , a^+_{k',\alpha'}] \\
	 &= 0 \\
[a_{k,\alpha} , N_{k',\alpha'}] &= [a_{k,\alpha} , a^+_{k',\alpha'}]a_{k',\alpha'} - a^+_{k',\alpha'}[a_{k',\alpha'} , a_{k,\alpha}]\\
	&= \delta_{kk'}\delta_{aa'} a_{k,\alpha} \\
[a^+_{k,\alpha} , N_{k',\alpha'}] &= -\delta_{kk'}\delta_{aa'} a^+_{k,\alpha}
\end{split}
\end{equation}


%\chapter{Feldquantisierung\label{chapter:feldquantisierung}}
\lhead{Feldquantisierung}
\begin{refsection}
\chapterauthor{Hannes Diethelm}

\printbibliography[heading=subbibliography]
\end{refsection}

\section{Maxwell-Gleichungen und elektromagnetische Wellen}

Hilfreich dazu ist auch die Beschreibung von Magnetfeldern in Kapitel \ref{chapter:magnetfeld}. In diese Kapitel wird der Gradient durch $\nabla$ ersetzt \cite{fq:nabla}. Dadurch k"onnen die Gleichungen einfacher geschrieben werden. 

Die in der Elektrotechnik wohl bekannten Maxwell-Gleichungen in SI Einheiten lauten:
\begin{equation}
\begin{split}
\nabla\cdot E &= \frac{\rho}{\varepsilon_0} \\
\nabla\times B &= \mu_0( J  + \varepsilon_0\frac{\partial E}{\partial t}) \\
\nabla\cdot B &=0 \\
\nabla\times E &= -\frac{\partial B }{\partial t}\\
\end{split}
\end{equation}

Dieses Einheitensystem is willk"urlich \cite{fq:em_units}. Im Heaviside-
Lorentz System, das von nun an verwendet wird, lauten die Gleichungen:
\begin{equation}
\begin{split}
\nabla\cdot E &= \rho \\
\nabla\times B &= \frac{1}{c}( J  + \frac{\partial E}{\partial t}) \\
\nabla\cdot B &=0 \\
\nabla\times E &= -\frac{1}{c} \frac{\partial B }{\partial t}\\
\end{split}
\end{equation}

Da $\nabla \cdot B = 0 $ gilt k"onnen diese Gleichungen durch folgende Substitution umformuliert werden:
\begin{equation}
B = \nabla\times A 
\end{equation}

Dadurch gilt $\nabla \cdot B = 0 $ automatisch:
\begin{equation}
\nabla \cdot B = 0 \rightarrow \nabla \cdot ( \nabla\times A ) = 0 \text{ gilt f"ur jedes A! }
\end{equation}

Durch Einsetzen erh"alt man die Gleichung f"ur E:
\begin{equation}
\nabla\times E + \frac{1}{c} \frac{\partial B }{\partial t} = 0
\rightarrow \nabla\times E + \frac{1}{c} \frac{\partial \nabla\times A }{\partial t} = 0 \rightarrow E = -\frac{1}{c} \dfrac{\partial A}{\partial t} - \nabla \phi
\end{equation}

$\nabla \phi$ kann als Integrationskonstante angesehen werden und $\phi$ entspricht dem skalaren Potential des Feldes.

Durch weiteres Einsetzen k"onnen die vier Maxwell-Gleichungen in zwei Gleichungen umgeschrieben werden:
\begin{equation}
\begin{split}
 \nabla^2 \phi + \frac{1}{c} \dfrac{\partial \nabla A}{\partial t} &= -\rho \\
 \nabla^2 A - \frac{1}{c^2} \frac{\partial^2 A }{\partial t^2} - \nabla \left( \nabla \cdot A + \frac{1}{c} \frac{\partial \phi }{\partial t} \right) &= - \frac{1}{c} J
\end{split}
\end{equation}

dabei gelten die Korrespondenzen:
\begin{equation}
\begin{split}
B &= \nabla\times A \\
E &= -\frac{1}{c} \dfrac{\partial A}{\partial t} - \nabla \phi
\end{split}
\end{equation}

Es kann gezeigt werden, dass $\phi$ durch eine Eichtransformation (Siehe \ref{section:eichtransformation}) geeignet gew"ahlt werden kann, damit:

\begin{equation}
\nabla \cdot A + \frac{1}{c} \frac{\partial \phi }{\partial t} = 0
\end{equation}

Dadurch werden die zwei gekoppelten Gleichungen entkoppelt und es gilt:
\begin{equation}
\begin{split}
\nabla^2 \phi - \frac{1}{c^2} \dfrac{\partial^2 \nabla \phi}{\partial t^2} &= -\rho \\
\nabla^2 A - \frac{1}{c^2} \frac{\partial^2 A }{\partial t^2} &= - \frac{1}{c} J
\end{split}
\end{equation}

F"ur weiter wollen ein Feld im Vakkum betrachten. Hierf"ur gilt $J = 0$, da keine Leiter vorhanden sind.
In einem Transversalfeld im Vakkum gilt zudem $\nabla \cdot A = 0$. (ToDo: ??) Dadurch vereinfachen sich die gekoppelten Differentialgleichung zu einer Differentialgleichung in A:
\begin{equation}
\nabla^2 A - \frac{1}{c^2} \frac{\partial^2 A }{\partial t^2} = 0
\end{equation}

\section{Von der Welle zu gekoppelten Oszillatoren}
L"osungen dieser Gleichung f"ur periodische Randbedingungen und $t=0$ in einer Box mit Seitenl"ange $L = V^{1/3}$ sind durch die Fourier Reihe gegeben:

\begin{equation}
A(x,0) = \frac{1}{\sqrt{V}} \sum_K \sum_{\alpha=1,2} (c_{k,\alpha}(0) \epsilon^{(\alpha)} e^{ikx} + c^*_{k,\alpha}(0) \epsilon^{(\alpha)} e^{-ikx})
\end{equation}

oder durch setzen von $u_{k,\alpha}(x) = \epsilon^{(\alpha)} e^{ikx}$:
\begin{equation}
A(x,0) = \frac{1}{\sqrt{V}} \sum_K \sum_{\alpha=1,2} (c_{k,\alpha}(0)u_{k,\alpha}(x) + c^*_{k,\alpha}(0) u^*_{k,\alpha}(x))
\end{equation}

Wenn diese Gleichung ausgeschrieben wird, sieht man, dass $A(x,t)$ durch diese Wahl f"ur alle $c_{k,\alpha}(t)$ reell bleibt:
\begin{equation}
(a + ib)(\cos kx + i \sin kx ) + (a - ib)(\cos kx - i \sin kx ) = 2 ( a \cos kx - b \sin kx )
\end{equation}
%=a \cos kx + ib \cos kx + ia \sin kx - b \sin kx + a \cos kx - ib \cos kx - ia \sin kx - b \sin kx

$k$ ist der Ausbreitungsvektor der Welle und zeigt in die Ausbreitungsrichtung. $\epsilon^{(\alpha)}$ ist die Polarisation. Dabei wird vorausgesetzt, dass $(\epsilon^{(1)}, \epsilon^{(2)} , k/|k|)$ ein orthogonales Rechtssystem aus Einheitsvektoren bilden.

Da $\epsilon^{(\alpha)}$ und $k$ orthogonal sind gilt dabei auch automatisch:

\begin{equation}
\nabla \cdot A = \frac{1}{\sqrt{V}} \sum_K \sum_{\alpha=1,2} (i c_{k,\alpha}(0) \underbrace{\epsilon^{(\alpha)} k}_{=0} e^{ikx} - i c^*_{k,\alpha}(0) \underbrace{\epsilon^{(\alpha)} k}_{=0} e^{-ikx}) = 0
\end{equation}

Weiterhin gilt durch wegen der Orthogonalit"at auch:
\begin{equation}
\begin{split}
\dfrac{1}{A} \int c_{k,\alpha} \cdot c^*_{k',\alpha'} d^3 x &= \delta_{kk'}\delta{aa'} \\
\dfrac{1}{A} \int c_{k,\alpha} \cdot c_{k',\alpha'} d^3 x &= 0 \\
\dfrac{1}{A} \int c^*_{k,\alpha} \cdot c^*_{k',\alpha'} d^3 x &= 0
\end{split}
\end{equation}

Um $A(x,t)$ zu erhalten, wird:
\begin{equation}
c_{k,\alpha}(t) = c_{k,\alpha}(0) e^{-i \omega t}
\end{equation}

Dabei ist:
\begin{equation}
\begin{split}
\omega=|k|c \\
\lambda = \frac{2 \pi}{|k|}
\end{split}
\end{equation}

Die komplette Wellengleichung wird somit:
\begin{equation}
A(x,t) = \frac{1}{\sqrt{V}} \sum_K \sum_{\alpha=1,2} (c_{k,\alpha}(0) \epsilon^{(\alpha)} e^{i (kx - \omega t)} + c^*_{k,\alpha}(0) \epsilon^{(\alpha)} e^{-i(kx - \omega t)})
\end{equation}

Die Hamilton-Funktion einer elektromagnetischen Welle ist gegeben durch:
\begin{equation}
\begin{split}
H &= \frac{1}{2} \int (|B|^2 + |E|^2) d^3 x \\
	&= \frac{1}{2} \int (| \nabla\times A |^2 + \left| \frac{1}{c} \dfrac{\partial A}{\partial t} \right|^2) d^3 x 
\end{split}
\end{equation}

Es kann gezeigt werden, dass die L"osung dieses Integrals gegeben ist durch:
\begin{equation}
H = \sum_K \sum_{\alpha=1,2} 2 \left(\frac{\omega}{c}\right)^2 c^*_{k,\alpha}(t) c_{k,\alpha}(t)
\end{equation}

Durch folgende Definition:
\begin{equation}
Q_{k,\alpha} = \frac{1}{c}(c_{k,\alpha}(t) + c^*_{k,\alpha}(t)) \quad P_{k,\alpha} = -\frac{i\omega}{c}(c_{k,\alpha}(t) - c^*_{k,\alpha}(t)) 
\end{equation}

wird die Hamilton-Funktion zu:
\begin{equation} \label{fq:hamilton}
\begin{split}
H &= \sum_K \sum_{\alpha=1,2} 2 \left(\frac{\omega}{c}\right)^2 \left[ \frac{c(\omega Q_{k,\alpha} - i P_{k,\alpha})}{2 \omega} \right] \left[ \frac{c(\omega Q_{k,\alpha} + i P_{k,\alpha})}{2 \omega} \right] \\
&= \sum_K \sum_{\alpha=1,2} \frac{1}{2} (P_{k,\alpha}^2 + \omega^2 Q_{k,\alpha}^2)
\end{split}
\end{equation}

Hier sieh man nun, dass es m"oglich ist, eine Welle durch unabh"angige Oszillatoren dar zu stellen.

$Q_{k,\alpha}$ und $P_{k,\alpha}$ k"onnen nun als Koordinaten und Impulse der einzelnen Oszillatoren aufgefasst werden:
\begin{equation}
\dfrac{\partial H}{\partial Q_{k,\alpha}} = -\dot{P}_{k,\alpha} \quad \dfrac{\partial H}{\partial P_{k,\alpha}} = \dot{Q}_{k,\alpha}
\end{equation}

\section{Quantisierung der Welle}

Wie beim harmonischen Oszillator können $Q_{k,\alpha}$ und $P_{k,\alpha}$ nun als Opperatoren aufgefasst werden. Die Vertauschungsrelationen werden dabei zu:
\begin{equation}
\begin{split}
[Q_{k,\alpha}, P_{k',\alpha'}] &= i \hbar \delta_{kk'}\delta_{aa'} \\
[Q_{k,\alpha}, Q_{k',\alpha'}] &= 0 \\
[P_{k,\alpha}, P_{k',\alpha'}] &= 0
\end{split}
\end{equation}

Wir definieren die Operatoren:
\begin{equation}
\begin{split}
a_{k,\alpha} &= (1/\sqrt{2 \hbar \omega})(\omega Q_{k,\alpha} + iP_{k,\alpha})) \\
a^+_{k,\alpha} &= (1/\sqrt{2 \hbar \omega})(\omega Q_{k,\alpha} - iP_{k,\alpha}))\\
N_{k,\alpha} &= a^+_{k,\alpha} a_{k,\alpha}
\end{split}
\end{equation}

Ein Vergleich mit \ref{fq:hamilton} liefert:
\begin{equation}
 c_{k,\alpha} \rightarrow c \sqrt{\hbar/2 \omega} \, a_{k,\alpha} \quad c^*_{k,\alpha} \rightarrow c \sqrt{\hbar/2 \omega} \, a^+_{k,\alpha}
\end{equation}
Somit entsprechen diese Operatoren den Fourier-Koeffizienten.

Die Kommentatoren f"ur diese Operatoren sind:
\begin{equation}
\begin{split}
[a_{k,\alpha} , a^+_{k',\alpha'}] &= - \frac{i}{2 \hbar} [Q_{k,\alpha}, P_{k',\alpha'}] + \frac{i}{2 \hbar} [P_{k,\alpha}, Q_{k',\alpha'}] \\
	 &= \delta_{kk'}\delta_{aa'} \\
[a_{k,\alpha} , a_{k',\alpha'}] &= [a^+_{k,\alpha} , a^+_{k',\alpha'}] \\
	 &= 0 \\
[a_{k,\alpha} , N_{k',\alpha'}] &= [a_{k,\alpha} , a^+_{k',\alpha'}]a_{k',\alpha'} - a^+_{k',\alpha'}[a_{k',\alpha'} , a_{k,\alpha}]\\
	&= \delta_{kk'}\delta_{aa'} a_{k,\alpha} \\
[a^+_{k,\alpha} , N_{k',\alpha'}] &= -\delta_{kk'}\delta_{aa'} a^+_{k,\alpha}
\end{split}
\end{equation}


\chapter{Feldquantisierung\label{chapter:feldquantisierung}}
\lhead{Feldquantisierung}
\begin{refsection}
\chapterauthor{Hannes Diethelm}

\printbibliography[heading=subbibliography]
\end{refsection}

\section{Maxwell-Gleichungen und elektromagnetische Wellen}

Hilfreich dazu ist auch die Beschreibung von Magnetfeldern in Kapitel \ref{chapter:magnetfeld}. In diese Kapitel wird der Gradient durch $\nabla$ ersetzt \cite{fq:nabla}. Dadurch k"onnen die Gleichungen einfacher geschrieben werden. 

Die in der Elektrotechnik wohl bekannten Maxwell-Gleichungen in SI Einheiten lauten:
\begin{equation}
\begin{split}
\nabla\cdot E &= \frac{\rho}{\varepsilon_0} \\
\nabla\times B &= \mu_0( J  + \varepsilon_0\frac{\partial E}{\partial t}) \\
\nabla\cdot B &=0 \\
\nabla\times E &= -\frac{\partial B }{\partial t}\\
\end{split}
\end{equation}

Dieses Einheitensystem is willk"urlich \cite{fq:em_units}. Im Heaviside-
Lorentz System, das von nun an verwendet wird, lauten die Gleichungen:
\begin{equation}
\begin{split}
\nabla\cdot E &= \rho \\
\nabla\times B &= \frac{1}{c}( J  + \frac{\partial E}{\partial t}) \\
\nabla\cdot B &=0 \\
\nabla\times E &= -\frac{1}{c} \frac{\partial B }{\partial t}\\
\end{split}
\end{equation}

Da $\nabla \cdot B = 0 $ gilt k"onnen diese Gleichungen durch folgende Substitution umformuliert werden:
\begin{equation}
B = \nabla\times A 
\end{equation}

Dadurch gilt $\nabla \cdot B = 0 $ automatisch:
\begin{equation}
\nabla \cdot B = 0 \rightarrow \nabla \cdot ( \nabla\times A ) = 0 \text{ gilt f"ur jedes A! }
\end{equation}

Durch Einsetzen erh"alt man die Gleichung f"ur E:
\begin{equation}
\nabla\times E + \frac{1}{c} \frac{\partial B }{\partial t} = 0
\rightarrow \nabla\times E + \frac{1}{c} \frac{\partial \nabla\times A }{\partial t} = 0 \rightarrow E = -\frac{1}{c} \dfrac{\partial A}{\partial t} - \nabla \phi
\end{equation}

$\nabla \phi$ kann als Integrationskonstante angesehen werden und $\phi$ entspricht dem skalaren Potential des Feldes.

Durch weiteres Einsetzen k"onnen die vier Maxwell-Gleichungen in zwei Gleichungen umgeschrieben werden:
\begin{equation}
\begin{split}
 \nabla^2 \phi + \frac{1}{c} \dfrac{\partial \nabla A}{\partial t} &= -\rho \\
 \nabla^2 A - \frac{1}{c^2} \frac{\partial^2 A }{\partial t^2} - \nabla \left( \nabla \cdot A + \frac{1}{c} \frac{\partial \phi }{\partial t} \right) &= - \frac{1}{c} J
\end{split}
\end{equation}

dabei gelten die Korrespondenzen:
\begin{equation}
\begin{split}
B &= \nabla\times A \\
E &= -\frac{1}{c} \dfrac{\partial A}{\partial t} - \nabla \phi
\end{split}
\end{equation}

Es kann gezeigt werden, dass $\phi$ durch eine Eichtransformation (Siehe \ref{section:eichtransformation}) geeignet gew"ahlt werden kann, damit:

\begin{equation}
\nabla \cdot A + \frac{1}{c} \frac{\partial \phi }{\partial t} = 0
\end{equation}

Dadurch werden die zwei gekoppelten Gleichungen entkoppelt und es gilt:
\begin{equation}
\begin{split}
\nabla^2 \phi - \frac{1}{c^2} \dfrac{\partial^2 \nabla \phi}{\partial t^2} &= -\rho \\
\nabla^2 A - \frac{1}{c^2} \frac{\partial^2 A }{\partial t^2} &= - \frac{1}{c} J
\end{split}
\end{equation}

F"ur weiter wollen ein Feld im Vakkum betrachten. Hierf"ur gilt $J = 0$, da keine Leiter vorhanden sind.
In einem Transversalfeld im Vakkum gilt zudem $\nabla \cdot A = 0$. (ToDo: ??) Dadurch vereinfachen sich die gekoppelten Differentialgleichung zu einer Differentialgleichung in A:
\begin{equation}
\nabla^2 A - \frac{1}{c^2} \frac{\partial^2 A }{\partial t^2} = 0
\end{equation}

\section{Von der Welle zu gekoppelten Oszillatoren}
L"osungen dieser Gleichung f"ur periodische Randbedingungen und $t=0$ in einer Box mit Seitenl"ange $L = V^{1/3}$ sind durch die Fourier Reihe gegeben:

\begin{equation}
A(x,0) = \frac{1}{\sqrt{V}} \sum_K \sum_{\alpha=1,2} (c_{k,\alpha}(0) \epsilon^{(\alpha)} e^{ikx} + c^*_{k,\alpha}(0) \epsilon^{(\alpha)} e^{-ikx})
\end{equation}

oder durch setzen von $u_{k,\alpha}(x) = \epsilon^{(\alpha)} e^{ikx}$:
\begin{equation}
A(x,0) = \frac{1}{\sqrt{V}} \sum_K \sum_{\alpha=1,2} (c_{k,\alpha}(0)u_{k,\alpha}(x) + c^*_{k,\alpha}(0) u^*_{k,\alpha}(x))
\end{equation}

Wenn diese Gleichung ausgeschrieben wird, sieht man, dass $A(x,t)$ durch diese Wahl f"ur alle $c_{k,\alpha}(t)$ reell bleibt:
\begin{equation}
(a + ib)(\cos kx + i \sin kx ) + (a - ib)(\cos kx - i \sin kx ) = 2 ( a \cos kx - b \sin kx )
\end{equation}
%=a \cos kx + ib \cos kx + ia \sin kx - b \sin kx + a \cos kx - ib \cos kx - ia \sin kx - b \sin kx

$k$ ist der Ausbreitungsvektor der Welle und zeigt in die Ausbreitungsrichtung. $\epsilon^{(\alpha)}$ ist die Polarisation. Dabei wird vorausgesetzt, dass $(\epsilon^{(1)}, \epsilon^{(2)} , k/|k|)$ ein orthogonales Rechtssystem aus Einheitsvektoren bilden.

Da $\epsilon^{(\alpha)}$ und $k$ orthogonal sind gilt dabei auch automatisch:

\begin{equation}
\nabla \cdot A = \frac{1}{\sqrt{V}} \sum_K \sum_{\alpha=1,2} (i c_{k,\alpha}(0) \underbrace{\epsilon^{(\alpha)} k}_{=0} e^{ikx} - i c^*_{k,\alpha}(0) \underbrace{\epsilon^{(\alpha)} k}_{=0} e^{-ikx}) = 0
\end{equation}

Weiterhin gilt durch wegen der Orthogonalit"at auch:
\begin{equation}
\begin{split}
\dfrac{1}{A} \int c_{k,\alpha} \cdot c^*_{k',\alpha'} d^3 x &= \delta_{kk'}\delta{aa'} \\
\dfrac{1}{A} \int c_{k,\alpha} \cdot c_{k',\alpha'} d^3 x &= 0 \\
\dfrac{1}{A} \int c^*_{k,\alpha} \cdot c^*_{k',\alpha'} d^3 x &= 0
\end{split}
\end{equation}

Um $A(x,t)$ zu erhalten, wird:
\begin{equation}
c_{k,\alpha}(t) = c_{k,\alpha}(0) e^{-i \omega t}
\end{equation}

Dabei ist:
\begin{equation}
\begin{split}
\omega=|k|c \\
\lambda = \frac{2 \pi}{|k|}
\end{split}
\end{equation}

Die komplette Wellengleichung wird somit:
\begin{equation}
A(x,t) = \frac{1}{\sqrt{V}} \sum_K \sum_{\alpha=1,2} (c_{k,\alpha}(0) \epsilon^{(\alpha)} e^{i (kx - \omega t)} + c^*_{k,\alpha}(0) \epsilon^{(\alpha)} e^{-i(kx - \omega t)})
\end{equation}

Die Hamilton-Funktion einer elektromagnetischen Welle ist gegeben durch:
\begin{equation}
\begin{split}
H &= \frac{1}{2} \int (|B|^2 + |E|^2) d^3 x \\
	&= \frac{1}{2} \int (| \nabla\times A |^2 + \left| \frac{1}{c} \dfrac{\partial A}{\partial t} \right|^2) d^3 x 
\end{split}
\end{equation}

Es kann gezeigt werden, dass die L"osung dieses Integrals gegeben ist durch:
\begin{equation}
H = \sum_K \sum_{\alpha=1,2} 2 \left(\frac{\omega}{c}\right)^2 c^*_{k,\alpha}(t) c_{k,\alpha}(t)
\end{equation}

Durch folgende Definition:
\begin{equation}
Q_{k,\alpha} = \frac{1}{c}(c_{k,\alpha}(t) + c^*_{k,\alpha}(t)) \quad P_{k,\alpha} = -\frac{i\omega}{c}(c_{k,\alpha}(t) - c^*_{k,\alpha}(t)) 
\end{equation}

wird die Hamilton-Funktion zu:
\begin{equation} \label{fq:hamilton}
\begin{split}
H &= \sum_K \sum_{\alpha=1,2} 2 \left(\frac{\omega}{c}\right)^2 \left[ \frac{c(\omega Q_{k,\alpha} - i P_{k,\alpha})}{2 \omega} \right] \left[ \frac{c(\omega Q_{k,\alpha} + i P_{k,\alpha})}{2 \omega} \right] \\
&= \sum_K \sum_{\alpha=1,2} \frac{1}{2} (P_{k,\alpha}^2 + \omega^2 Q_{k,\alpha}^2)
\end{split}
\end{equation}

Hier sieh man nun, dass es m"oglich ist, eine Welle durch unabh"angige Oszillatoren dar zu stellen.

$Q_{k,\alpha}$ und $P_{k,\alpha}$ k"onnen nun als Koordinaten und Impulse der einzelnen Oszillatoren aufgefasst werden:
\begin{equation}
\dfrac{\partial H}{\partial Q_{k,\alpha}} = -\dot{P}_{k,\alpha} \quad \dfrac{\partial H}{\partial P_{k,\alpha}} = \dot{Q}_{k,\alpha}
\end{equation}

\section{Quantisierung der Welle}

Wie beim harmonischen Oszillator können $Q_{k,\alpha}$ und $P_{k,\alpha}$ nun als Opperatoren aufgefasst werden. Die Vertauschungsrelationen werden dabei zu:
\begin{equation}
\begin{split}
[Q_{k,\alpha}, P_{k',\alpha'}] &= i \hbar \delta_{kk'}\delta_{aa'} \\
[Q_{k,\alpha}, Q_{k',\alpha'}] &= 0 \\
[P_{k,\alpha}, P_{k',\alpha'}] &= 0
\end{split}
\end{equation}

Wir definieren die Operatoren:
\begin{equation}
\begin{split}
a_{k,\alpha} &= (1/\sqrt{2 \hbar \omega})(\omega Q_{k,\alpha} + iP_{k,\alpha})) \\
a^+_{k,\alpha} &= (1/\sqrt{2 \hbar \omega})(\omega Q_{k,\alpha} - iP_{k,\alpha}))\\
N_{k,\alpha} &= a^+_{k,\alpha} a_{k,\alpha}
\end{split}
\end{equation}

Ein Vergleich mit \ref{fq:hamilton} liefert:
\begin{equation}
 c_{k,\alpha} \rightarrow c \sqrt{\hbar/2 \omega} \, a_{k,\alpha} \quad c^*_{k,\alpha} \rightarrow c \sqrt{\hbar/2 \omega} \, a^+_{k,\alpha}
\end{equation}
Somit entsprechen diese Operatoren den Fourier-Koeffizienten.

Die Kommentatoren f"ur diese Operatoren sind:
\begin{equation}
\begin{split}
[a_{k,\alpha} , a^+_{k',\alpha'}] &= - \frac{i}{2 \hbar} [Q_{k,\alpha}, P_{k',\alpha'}] + \frac{i}{2 \hbar} [P_{k,\alpha}, Q_{k',\alpha'}] \\
	 &= \delta_{kk'}\delta_{aa'} \\
[a_{k,\alpha} , a_{k',\alpha'}] &= [a^+_{k,\alpha} , a^+_{k',\alpha'}] \\
	 &= 0 \\
[a_{k,\alpha} , N_{k',\alpha'}] &= [a_{k,\alpha} , a^+_{k',\alpha'}]a_{k',\alpha'} - a^+_{k',\alpha'}[a_{k',\alpha'} , a_{k,\alpha}]\\
	&= \delta_{kk'}\delta_{aa'} a_{k,\alpha} \\
[a^+_{k,\alpha} , N_{k',\alpha'}] &= -\delta_{kk'}\delta_{aa'} a^+_{k,\alpha}
\end{split}
\end{equation}


\chapter{Feldquantisierung\label{chapter:feldquantisierung}}
\lhead{Feldquantisierung}
\begin{refsection}
\chapterauthor{Hannes Diethelm}

\printbibliography[heading=subbibliography]
\end{refsection}

\section{Maxwell-Gleichungen und elektromagnetische Wellen}

Hilfreich dazu ist auch die Beschreibung von Magnetfeldern in Kapitel \ref{chapter:magnetfeld}. In diese Kapitel wird der Gradient durch $\nabla$ ersetzt \cite{fq:nabla}. Dadurch k"onnen die Gleichungen einfacher geschrieben werden. 

Die in der Elektrotechnik wohl bekannten Maxwell-Gleichungen in SI Einheiten lauten:
\begin{equation}
\begin{split}
\nabla\cdot E &= \frac{\rho}{\varepsilon_0} \\
\nabla\times B &= \mu_0( J  + \varepsilon_0\frac{\partial E}{\partial t}) \\
\nabla\cdot B &=0 \\
\nabla\times E &= -\frac{\partial B }{\partial t}\\
\end{split}
\end{equation}

Dieses Einheitensystem is willk"urlich \cite{fq:em_units}. Im Heaviside-
Lorentz System, das von nun an verwendet wird, lauten die Gleichungen:
\begin{equation}
\begin{split}
\nabla\cdot E &= \rho \\
\nabla\times B &= \frac{1}{c}( J  + \frac{\partial E}{\partial t}) \\
\nabla\cdot B &=0 \\
\nabla\times E &= -\frac{1}{c} \frac{\partial B }{\partial t}\\
\end{split}
\end{equation}

Da $\nabla \cdot B = 0 $ gilt k"onnen diese Gleichungen durch folgende Substitution umformuliert werden:
\begin{equation}
B = \nabla\times A 
\end{equation}

Dadurch gilt $\nabla \cdot B = 0 $ automatisch:
\begin{equation}
\nabla \cdot B = 0 \rightarrow \nabla \cdot ( \nabla\times A ) = 0 \text{ gilt f"ur jedes A! }
\end{equation}

Durch Einsetzen erh"alt man die Gleichung f"ur E:
\begin{equation}
\nabla\times E + \frac{1}{c} \frac{\partial B }{\partial t} = 0
\rightarrow \nabla\times E + \frac{1}{c} \frac{\partial \nabla\times A }{\partial t} = 0 \rightarrow E = -\frac{1}{c} \dfrac{\partial A}{\partial t} - \nabla \phi
\end{equation}

$\nabla \phi$ kann als Integrationskonstante angesehen werden und $\phi$ entspricht dem skalaren Potential des Feldes.

Durch weiteres Einsetzen k"onnen die vier Maxwell-Gleichungen in zwei Gleichungen umgeschrieben werden:
\begin{equation}
\begin{split}
 \nabla^2 \phi + \frac{1}{c} \dfrac{\partial \nabla A}{\partial t} &= -\rho \\
 \nabla^2 A - \frac{1}{c^2} \frac{\partial^2 A }{\partial t^2} - \nabla \left( \nabla \cdot A + \frac{1}{c} \frac{\partial \phi }{\partial t} \right) &= - \frac{1}{c} J
\end{split}
\end{equation}

dabei gelten die Korrespondenzen:
\begin{equation}
\begin{split}
B &= \nabla\times A \\
E &= -\frac{1}{c} \dfrac{\partial A}{\partial t} - \nabla \phi
\end{split}
\end{equation}

Es kann gezeigt werden, dass $\phi$ durch eine Eichtransformation (Siehe \ref{section:eichtransformation}) geeignet gew"ahlt werden kann, damit:

\begin{equation}
\nabla \cdot A + \frac{1}{c} \frac{\partial \phi }{\partial t} = 0
\end{equation}

Dadurch werden die zwei gekoppelten Gleichungen entkoppelt und es gilt:
\begin{equation}
\begin{split}
\nabla^2 \phi - \frac{1}{c^2} \dfrac{\partial^2 \nabla \phi}{\partial t^2} &= -\rho \\
\nabla^2 A - \frac{1}{c^2} \frac{\partial^2 A }{\partial t^2} &= - \frac{1}{c} J
\end{split}
\end{equation}

F"ur weiter wollen ein Feld im Vakkum betrachten. Hierf"ur gilt $J = 0$, da keine Leiter vorhanden sind.
In einem Transversalfeld im Vakkum gilt zudem $\nabla \cdot A = 0$. (ToDo: ??) Dadurch vereinfachen sich die gekoppelten Differentialgleichung zu einer Differentialgleichung in A:
\begin{equation}
\nabla^2 A - \frac{1}{c^2} \frac{\partial^2 A }{\partial t^2} = 0
\end{equation}

\section{Von der Welle zu gekoppelten Oszillatoren}
L"osungen dieser Gleichung f"ur periodische Randbedingungen und $t=0$ in einer Box mit Seitenl"ange $L = V^{1/3}$ sind durch die Fourier Reihe gegeben:

\begin{equation}
A(x,0) = \frac{1}{\sqrt{V}} \sum_K \sum_{\alpha=1,2} (c_{k,\alpha}(0) \epsilon^{(\alpha)} e^{ikx} + c^*_{k,\alpha}(0) \epsilon^{(\alpha)} e^{-ikx})
\end{equation}

oder durch setzen von $u_{k,\alpha}(x) = \epsilon^{(\alpha)} e^{ikx}$:
\begin{equation}
A(x,0) = \frac{1}{\sqrt{V}} \sum_K \sum_{\alpha=1,2} (c_{k,\alpha}(0)u_{k,\alpha}(x) + c^*_{k,\alpha}(0) u^*_{k,\alpha}(x))
\end{equation}

Wenn diese Gleichung ausgeschrieben wird, sieht man, dass $A(x,t)$ durch diese Wahl f"ur alle $c_{k,\alpha}(t)$ reell bleibt:
\begin{equation}
(a + ib)(\cos kx + i \sin kx ) + (a - ib)(\cos kx - i \sin kx ) = 2 ( a \cos kx - b \sin kx )
\end{equation}
%=a \cos kx + ib \cos kx + ia \sin kx - b \sin kx + a \cos kx - ib \cos kx - ia \sin kx - b \sin kx

$k$ ist der Ausbreitungsvektor der Welle und zeigt in die Ausbreitungsrichtung. $\epsilon^{(\alpha)}$ ist die Polarisation. Dabei wird vorausgesetzt, dass $(\epsilon^{(1)}, \epsilon^{(2)} , k/|k|)$ ein orthogonales Rechtssystem aus Einheitsvektoren bilden.

Da $\epsilon^{(\alpha)}$ und $k$ orthogonal sind gilt dabei auch automatisch:

\begin{equation}
\nabla \cdot A = \frac{1}{\sqrt{V}} \sum_K \sum_{\alpha=1,2} (i c_{k,\alpha}(0) \underbrace{\epsilon^{(\alpha)} k}_{=0} e^{ikx} - i c^*_{k,\alpha}(0) \underbrace{\epsilon^{(\alpha)} k}_{=0} e^{-ikx}) = 0
\end{equation}

Weiterhin gilt durch wegen der Orthogonalit"at auch:
\begin{equation}
\begin{split}
\dfrac{1}{A} \int c_{k,\alpha} \cdot c^*_{k',\alpha'} d^3 x &= \delta_{kk'}\delta{aa'} \\
\dfrac{1}{A} \int c_{k,\alpha} \cdot c_{k',\alpha'} d^3 x &= 0 \\
\dfrac{1}{A} \int c^*_{k,\alpha} \cdot c^*_{k',\alpha'} d^3 x &= 0
\end{split}
\end{equation}

Um $A(x,t)$ zu erhalten, wird:
\begin{equation}
c_{k,\alpha}(t) = c_{k,\alpha}(0) e^{-i \omega t}
\end{equation}

Dabei ist:
\begin{equation}
\begin{split}
\omega=|k|c \\
\lambda = \frac{2 \pi}{|k|}
\end{split}
\end{equation}

Die komplette Wellengleichung wird somit:
\begin{equation}
A(x,t) = \frac{1}{\sqrt{V}} \sum_K \sum_{\alpha=1,2} (c_{k,\alpha}(0) \epsilon^{(\alpha)} e^{i (kx - \omega t)} + c^*_{k,\alpha}(0) \epsilon^{(\alpha)} e^{-i(kx - \omega t)})
\end{equation}

Die Hamilton-Funktion einer elektromagnetischen Welle ist gegeben durch:
\begin{equation}
\begin{split}
H &= \frac{1}{2} \int (|B|^2 + |E|^2) d^3 x \\
	&= \frac{1}{2} \int (| \nabla\times A |^2 + \left| \frac{1}{c} \dfrac{\partial A}{\partial t} \right|^2) d^3 x 
\end{split}
\end{equation}

Es kann gezeigt werden, dass die L"osung dieses Integrals gegeben ist durch:
\begin{equation}
H = \sum_K \sum_{\alpha=1,2} 2 \left(\frac{\omega}{c}\right)^2 c^*_{k,\alpha}(t) c_{k,\alpha}(t)
\end{equation}

Durch folgende Definition:
\begin{equation}
Q_{k,\alpha} = \frac{1}{c}(c_{k,\alpha}(t) + c^*_{k,\alpha}(t)) \quad P_{k,\alpha} = -\frac{i\omega}{c}(c_{k,\alpha}(t) - c^*_{k,\alpha}(t)) 
\end{equation}

wird die Hamilton-Funktion zu:
\begin{equation} \label{fq:hamilton}
\begin{split}
H &= \sum_K \sum_{\alpha=1,2} 2 \left(\frac{\omega}{c}\right)^2 \left[ \frac{c(\omega Q_{k,\alpha} - i P_{k,\alpha})}{2 \omega} \right] \left[ \frac{c(\omega Q_{k,\alpha} + i P_{k,\alpha})}{2 \omega} \right] \\
&= \sum_K \sum_{\alpha=1,2} \frac{1}{2} (P_{k,\alpha}^2 + \omega^2 Q_{k,\alpha}^2)
\end{split}
\end{equation}

Hier sieh man nun, dass es m"oglich ist, eine Welle durch unabh"angige Oszillatoren dar zu stellen.

$Q_{k,\alpha}$ und $P_{k,\alpha}$ k"onnen nun als Koordinaten und Impulse der einzelnen Oszillatoren aufgefasst werden:
\begin{equation}
\dfrac{\partial H}{\partial Q_{k,\alpha}} = -\dot{P}_{k,\alpha} \quad \dfrac{\partial H}{\partial P_{k,\alpha}} = \dot{Q}_{k,\alpha}
\end{equation}

\section{Quantisierung der Welle}

Wie beim harmonischen Oszillator können $Q_{k,\alpha}$ und $P_{k,\alpha}$ nun als Opperatoren aufgefasst werden. Die Vertauschungsrelationen werden dabei zu:
\begin{equation}
\begin{split}
[Q_{k,\alpha}, P_{k',\alpha'}] &= i \hbar \delta_{kk'}\delta_{aa'} \\
[Q_{k,\alpha}, Q_{k',\alpha'}] &= 0 \\
[P_{k,\alpha}, P_{k',\alpha'}] &= 0
\end{split}
\end{equation}

Wir definieren die Operatoren:
\begin{equation}
\begin{split}
a_{k,\alpha} &= (1/\sqrt{2 \hbar \omega})(\omega Q_{k,\alpha} + iP_{k,\alpha})) \\
a^+_{k,\alpha} &= (1/\sqrt{2 \hbar \omega})(\omega Q_{k,\alpha} - iP_{k,\alpha}))\\
N_{k,\alpha} &= a^+_{k,\alpha} a_{k,\alpha}
\end{split}
\end{equation}

Ein Vergleich mit \ref{fq:hamilton} liefert:
\begin{equation}
 c_{k,\alpha} \rightarrow c \sqrt{\hbar/2 \omega} \, a_{k,\alpha} \quad c^*_{k,\alpha} \rightarrow c \sqrt{\hbar/2 \omega} \, a^+_{k,\alpha}
\end{equation}
Somit entsprechen diese Operatoren den Fourier-Koeffizienten.

Die Kommentatoren f"ur diese Operatoren sind:
\begin{equation}
\begin{split}
[a_{k,\alpha} , a^+_{k',\alpha'}] &= - \frac{i}{2 \hbar} [Q_{k,\alpha}, P_{k',\alpha'}] + \frac{i}{2 \hbar} [P_{k,\alpha}, Q_{k',\alpha'}] \\
	 &= \delta_{kk'}\delta_{aa'} \\
[a_{k,\alpha} , a_{k',\alpha'}] &= [a^+_{k,\alpha} , a^+_{k',\alpha'}] \\
	 &= 0 \\
[a_{k,\alpha} , N_{k',\alpha'}] &= [a_{k,\alpha} , a^+_{k',\alpha'}]a_{k',\alpha'} - a^+_{k',\alpha'}[a_{k',\alpha'} , a_{k,\alpha}]\\
	&= \delta_{kk'}\delta_{aa'} a_{k,\alpha} \\
[a^+_{k,\alpha} , N_{k',\alpha'}] &= -\delta_{kk'}\delta_{aa'} a^+_{k,\alpha}
\end{split}
\end{equation}


%\chapter{Feldquantisierung\label{chapter:feldquantisierung}}
\lhead{Feldquantisierung}
\begin{refsection}
\chapterauthor{Hannes Diethelm}

\printbibliography[heading=subbibliography]
\end{refsection}

\section{Maxwell-Gleichungen und elektromagnetische Wellen}

Hilfreich dazu ist auch die Beschreibung von Magnetfeldern in Kapitel \ref{chapter:magnetfeld}. In diese Kapitel wird der Gradient durch $\nabla$ ersetzt \cite{fq:nabla}. Dadurch k"onnen die Gleichungen einfacher geschrieben werden. 

Die in der Elektrotechnik wohl bekannten Maxwell-Gleichungen in SI Einheiten lauten:
\begin{equation}
\begin{split}
\nabla\cdot E &= \frac{\rho}{\varepsilon_0} \\
\nabla\times B &= \mu_0( J  + \varepsilon_0\frac{\partial E}{\partial t}) \\
\nabla\cdot B &=0 \\
\nabla\times E &= -\frac{\partial B }{\partial t}\\
\end{split}
\end{equation}

Dieses Einheitensystem is willk"urlich \cite{fq:em_units}. Im Heaviside-
Lorentz System, das von nun an verwendet wird, lauten die Gleichungen:
\begin{equation}
\begin{split}
\nabla\cdot E &= \rho \\
\nabla\times B &= \frac{1}{c}( J  + \frac{\partial E}{\partial t}) \\
\nabla\cdot B &=0 \\
\nabla\times E &= -\frac{1}{c} \frac{\partial B }{\partial t}\\
\end{split}
\end{equation}

Da $\nabla \cdot B = 0 $ gilt k"onnen diese Gleichungen durch folgende Substitution umformuliert werden:
\begin{equation}
B = \nabla\times A 
\end{equation}

Dadurch gilt $\nabla \cdot B = 0 $ automatisch:
\begin{equation}
\nabla \cdot B = 0 \rightarrow \nabla \cdot ( \nabla\times A ) = 0 \text{ gilt f"ur jedes A! }
\end{equation}

Durch Einsetzen erh"alt man die Gleichung f"ur E:
\begin{equation}
\nabla\times E + \frac{1}{c} \frac{\partial B }{\partial t} = 0
\rightarrow \nabla\times E + \frac{1}{c} \frac{\partial \nabla\times A }{\partial t} = 0 \rightarrow E = -\frac{1}{c} \dfrac{\partial A}{\partial t} - \nabla \phi
\end{equation}

$\nabla \phi$ kann als Integrationskonstante angesehen werden und $\phi$ entspricht dem skalaren Potential des Feldes.

Durch weiteres Einsetzen k"onnen die vier Maxwell-Gleichungen in zwei Gleichungen umgeschrieben werden:
\begin{equation}
\begin{split}
 \nabla^2 \phi + \frac{1}{c} \dfrac{\partial \nabla A}{\partial t} &= -\rho \\
 \nabla^2 A - \frac{1}{c^2} \frac{\partial^2 A }{\partial t^2} - \nabla \left( \nabla \cdot A + \frac{1}{c} \frac{\partial \phi }{\partial t} \right) &= - \frac{1}{c} J
\end{split}
\end{equation}

dabei gelten die Korrespondenzen:
\begin{equation}
\begin{split}
B &= \nabla\times A \\
E &= -\frac{1}{c} \dfrac{\partial A}{\partial t} - \nabla \phi
\end{split}
\end{equation}

Es kann gezeigt werden, dass $\phi$ durch eine Eichtransformation (Siehe \ref{section:eichtransformation}) geeignet gew"ahlt werden kann, damit:

\begin{equation}
\nabla \cdot A + \frac{1}{c} \frac{\partial \phi }{\partial t} = 0
\end{equation}

Dadurch werden die zwei gekoppelten Gleichungen entkoppelt und es gilt:
\begin{equation}
\begin{split}
\nabla^2 \phi - \frac{1}{c^2} \dfrac{\partial^2 \nabla \phi}{\partial t^2} &= -\rho \\
\nabla^2 A - \frac{1}{c^2} \frac{\partial^2 A }{\partial t^2} &= - \frac{1}{c} J
\end{split}
\end{equation}

F"ur weiter wollen ein Feld im Vakkum betrachten. Hierf"ur gilt $J = 0$, da keine Leiter vorhanden sind.
In einem Transversalfeld im Vakkum gilt zudem $\nabla \cdot A = 0$. (ToDo: ??) Dadurch vereinfachen sich die gekoppelten Differentialgleichung zu einer Differentialgleichung in A:
\begin{equation}
\nabla^2 A - \frac{1}{c^2} \frac{\partial^2 A }{\partial t^2} = 0
\end{equation}

\section{Von der Welle zu gekoppelten Oszillatoren}
L"osungen dieser Gleichung f"ur periodische Randbedingungen und $t=0$ in einer Box mit Seitenl"ange $L = V^{1/3}$ sind durch die Fourier Reihe gegeben:

\begin{equation}
A(x,0) = \frac{1}{\sqrt{V}} \sum_K \sum_{\alpha=1,2} (c_{k,\alpha}(0) \epsilon^{(\alpha)} e^{ikx} + c^*_{k,\alpha}(0) \epsilon^{(\alpha)} e^{-ikx})
\end{equation}

oder durch setzen von $u_{k,\alpha}(x) = \epsilon^{(\alpha)} e^{ikx}$:
\begin{equation}
A(x,0) = \frac{1}{\sqrt{V}} \sum_K \sum_{\alpha=1,2} (c_{k,\alpha}(0)u_{k,\alpha}(x) + c^*_{k,\alpha}(0) u^*_{k,\alpha}(x))
\end{equation}

Wenn diese Gleichung ausgeschrieben wird, sieht man, dass $A(x,t)$ durch diese Wahl f"ur alle $c_{k,\alpha}(t)$ reell bleibt:
\begin{equation}
(a + ib)(\cos kx + i \sin kx ) + (a - ib)(\cos kx - i \sin kx ) = 2 ( a \cos kx - b \sin kx )
\end{equation}
%=a \cos kx + ib \cos kx + ia \sin kx - b \sin kx + a \cos kx - ib \cos kx - ia \sin kx - b \sin kx

$k$ ist der Ausbreitungsvektor der Welle und zeigt in die Ausbreitungsrichtung. $\epsilon^{(\alpha)}$ ist die Polarisation. Dabei wird vorausgesetzt, dass $(\epsilon^{(1)}, \epsilon^{(2)} , k/|k|)$ ein orthogonales Rechtssystem aus Einheitsvektoren bilden.

Da $\epsilon^{(\alpha)}$ und $k$ orthogonal sind gilt dabei auch automatisch:

\begin{equation}
\nabla \cdot A = \frac{1}{\sqrt{V}} \sum_K \sum_{\alpha=1,2} (i c_{k,\alpha}(0) \underbrace{\epsilon^{(\alpha)} k}_{=0} e^{ikx} - i c^*_{k,\alpha}(0) \underbrace{\epsilon^{(\alpha)} k}_{=0} e^{-ikx}) = 0
\end{equation}

Weiterhin gilt durch wegen der Orthogonalit"at auch:
\begin{equation}
\begin{split}
\dfrac{1}{A} \int c_{k,\alpha} \cdot c^*_{k',\alpha'} d^3 x &= \delta_{kk'}\delta{aa'} \\
\dfrac{1}{A} \int c_{k,\alpha} \cdot c_{k',\alpha'} d^3 x &= 0 \\
\dfrac{1}{A} \int c^*_{k,\alpha} \cdot c^*_{k',\alpha'} d^3 x &= 0
\end{split}
\end{equation}

Um $A(x,t)$ zu erhalten, wird:
\begin{equation}
c_{k,\alpha}(t) = c_{k,\alpha}(0) e^{-i \omega t}
\end{equation}

Dabei ist:
\begin{equation}
\begin{split}
\omega=|k|c \\
\lambda = \frac{2 \pi}{|k|}
\end{split}
\end{equation}

Die komplette Wellengleichung wird somit:
\begin{equation}
A(x,t) = \frac{1}{\sqrt{V}} \sum_K \sum_{\alpha=1,2} (c_{k,\alpha}(0) \epsilon^{(\alpha)} e^{i (kx - \omega t)} + c^*_{k,\alpha}(0) \epsilon^{(\alpha)} e^{-i(kx - \omega t)})
\end{equation}

Die Hamilton-Funktion einer elektromagnetischen Welle ist gegeben durch:
\begin{equation}
\begin{split}
H &= \frac{1}{2} \int (|B|^2 + |E|^2) d^3 x \\
	&= \frac{1}{2} \int (| \nabla\times A |^2 + \left| \frac{1}{c} \dfrac{\partial A}{\partial t} \right|^2) d^3 x 
\end{split}
\end{equation}

Es kann gezeigt werden, dass die L"osung dieses Integrals gegeben ist durch:
\begin{equation}
H = \sum_K \sum_{\alpha=1,2} 2 \left(\frac{\omega}{c}\right)^2 c^*_{k,\alpha}(t) c_{k,\alpha}(t)
\end{equation}

Durch folgende Definition:
\begin{equation}
Q_{k,\alpha} = \frac{1}{c}(c_{k,\alpha}(t) + c^*_{k,\alpha}(t)) \quad P_{k,\alpha} = -\frac{i\omega}{c}(c_{k,\alpha}(t) - c^*_{k,\alpha}(t)) 
\end{equation}

wird die Hamilton-Funktion zu:
\begin{equation} \label{fq:hamilton}
\begin{split}
H &= \sum_K \sum_{\alpha=1,2} 2 \left(\frac{\omega}{c}\right)^2 \left[ \frac{c(\omega Q_{k,\alpha} - i P_{k,\alpha})}{2 \omega} \right] \left[ \frac{c(\omega Q_{k,\alpha} + i P_{k,\alpha})}{2 \omega} \right] \\
&= \sum_K \sum_{\alpha=1,2} \frac{1}{2} (P_{k,\alpha}^2 + \omega^2 Q_{k,\alpha}^2)
\end{split}
\end{equation}

Hier sieh man nun, dass es m"oglich ist, eine Welle durch unabh"angige Oszillatoren dar zu stellen.

$Q_{k,\alpha}$ und $P_{k,\alpha}$ k"onnen nun als Koordinaten und Impulse der einzelnen Oszillatoren aufgefasst werden:
\begin{equation}
\dfrac{\partial H}{\partial Q_{k,\alpha}} = -\dot{P}_{k,\alpha} \quad \dfrac{\partial H}{\partial P_{k,\alpha}} = \dot{Q}_{k,\alpha}
\end{equation}

\section{Quantisierung der Welle}

Wie beim harmonischen Oszillator können $Q_{k,\alpha}$ und $P_{k,\alpha}$ nun als Opperatoren aufgefasst werden. Die Vertauschungsrelationen werden dabei zu:
\begin{equation}
\begin{split}
[Q_{k,\alpha}, P_{k',\alpha'}] &= i \hbar \delta_{kk'}\delta_{aa'} \\
[Q_{k,\alpha}, Q_{k',\alpha'}] &= 0 \\
[P_{k,\alpha}, P_{k',\alpha'}] &= 0
\end{split}
\end{equation}

Wir definieren die Operatoren:
\begin{equation}
\begin{split}
a_{k,\alpha} &= (1/\sqrt{2 \hbar \omega})(\omega Q_{k,\alpha} + iP_{k,\alpha})) \\
a^+_{k,\alpha} &= (1/\sqrt{2 \hbar \omega})(\omega Q_{k,\alpha} - iP_{k,\alpha}))\\
N_{k,\alpha} &= a^+_{k,\alpha} a_{k,\alpha}
\end{split}
\end{equation}

Ein Vergleich mit \ref{fq:hamilton} liefert:
\begin{equation}
 c_{k,\alpha} \rightarrow c \sqrt{\hbar/2 \omega} \, a_{k,\alpha} \quad c^*_{k,\alpha} \rightarrow c \sqrt{\hbar/2 \omega} \, a^+_{k,\alpha}
\end{equation}
Somit entsprechen diese Operatoren den Fourier-Koeffizienten.

Die Kommentatoren f"ur diese Operatoren sind:
\begin{equation}
\begin{split}
[a_{k,\alpha} , a^+_{k',\alpha'}] &= - \frac{i}{2 \hbar} [Q_{k,\alpha}, P_{k',\alpha'}] + \frac{i}{2 \hbar} [P_{k,\alpha}, Q_{k',\alpha'}] \\
	 &= \delta_{kk'}\delta_{aa'} \\
[a_{k,\alpha} , a_{k',\alpha'}] &= [a^+_{k,\alpha} , a^+_{k',\alpha'}] \\
	 &= 0 \\
[a_{k,\alpha} , N_{k',\alpha'}] &= [a_{k,\alpha} , a^+_{k',\alpha'}]a_{k',\alpha'} - a^+_{k',\alpha'}[a_{k',\alpha'} , a_{k,\alpha}]\\
	&= \delta_{kk'}\delta_{aa'} a_{k,\alpha} \\
[a^+_{k,\alpha} , N_{k',\alpha'}] &= -\delta_{kk'}\delta_{aa'} a^+_{k,\alpha}
\end{split}
\end{equation}


%\chapter{Feldquantisierung\label{chapter:feldquantisierung}}
\lhead{Feldquantisierung}
\begin{refsection}
\chapterauthor{Hannes Diethelm}

\printbibliography[heading=subbibliography]
\end{refsection}

\section{Maxwell-Gleichungen und elektromagnetische Wellen}

Hilfreich dazu ist auch die Beschreibung von Magnetfeldern in Kapitel \ref{chapter:magnetfeld}. In diese Kapitel wird der Gradient durch $\nabla$ ersetzt \cite{fq:nabla}. Dadurch k"onnen die Gleichungen einfacher geschrieben werden. 

Die in der Elektrotechnik wohl bekannten Maxwell-Gleichungen in SI Einheiten lauten:
\begin{equation}
\begin{split}
\nabla\cdot E &= \frac{\rho}{\varepsilon_0} \\
\nabla\times B &= \mu_0( J  + \varepsilon_0\frac{\partial E}{\partial t}) \\
\nabla\cdot B &=0 \\
\nabla\times E &= -\frac{\partial B }{\partial t}\\
\end{split}
\end{equation}

Dieses Einheitensystem is willk"urlich \cite{fq:em_units}. Im Heaviside-
Lorentz System, das von nun an verwendet wird, lauten die Gleichungen:
\begin{equation}
\begin{split}
\nabla\cdot E &= \rho \\
\nabla\times B &= \frac{1}{c}( J  + \frac{\partial E}{\partial t}) \\
\nabla\cdot B &=0 \\
\nabla\times E &= -\frac{1}{c} \frac{\partial B }{\partial t}\\
\end{split}
\end{equation}

Da $\nabla \cdot B = 0 $ gilt k"onnen diese Gleichungen durch folgende Substitution umformuliert werden:
\begin{equation}
B = \nabla\times A 
\end{equation}

Dadurch gilt $\nabla \cdot B = 0 $ automatisch:
\begin{equation}
\nabla \cdot B = 0 \rightarrow \nabla \cdot ( \nabla\times A ) = 0 \text{ gilt f"ur jedes A! }
\end{equation}

Durch Einsetzen erh"alt man die Gleichung f"ur E:
\begin{equation}
\nabla\times E + \frac{1}{c} \frac{\partial B }{\partial t} = 0
\rightarrow \nabla\times E + \frac{1}{c} \frac{\partial \nabla\times A }{\partial t} = 0 \rightarrow E = -\frac{1}{c} \dfrac{\partial A}{\partial t} - \nabla \phi
\end{equation}

$\nabla \phi$ kann als Integrationskonstante angesehen werden und $\phi$ entspricht dem skalaren Potential des Feldes.

Durch weiteres Einsetzen k"onnen die vier Maxwell-Gleichungen in zwei Gleichungen umgeschrieben werden:
\begin{equation}
\begin{split}
 \nabla^2 \phi + \frac{1}{c} \dfrac{\partial \nabla A}{\partial t} &= -\rho \\
 \nabla^2 A - \frac{1}{c^2} \frac{\partial^2 A }{\partial t^2} - \nabla \left( \nabla \cdot A + \frac{1}{c} \frac{\partial \phi }{\partial t} \right) &= - \frac{1}{c} J
\end{split}
\end{equation}

dabei gelten die Korrespondenzen:
\begin{equation}
\begin{split}
B &= \nabla\times A \\
E &= -\frac{1}{c} \dfrac{\partial A}{\partial t} - \nabla \phi
\end{split}
\end{equation}

Es kann gezeigt werden, dass $\phi$ durch eine Eichtransformation (Siehe \ref{section:eichtransformation}) geeignet gew"ahlt werden kann, damit:

\begin{equation}
\nabla \cdot A + \frac{1}{c} \frac{\partial \phi }{\partial t} = 0
\end{equation}

Dadurch werden die zwei gekoppelten Gleichungen entkoppelt und es gilt:
\begin{equation}
\begin{split}
\nabla^2 \phi - \frac{1}{c^2} \dfrac{\partial^2 \nabla \phi}{\partial t^2} &= -\rho \\
\nabla^2 A - \frac{1}{c^2} \frac{\partial^2 A }{\partial t^2} &= - \frac{1}{c} J
\end{split}
\end{equation}

F"ur weiter wollen ein Feld im Vakkum betrachten. Hierf"ur gilt $J = 0$, da keine Leiter vorhanden sind.
In einem Transversalfeld im Vakkum gilt zudem $\nabla \cdot A = 0$. (ToDo: ??) Dadurch vereinfachen sich die gekoppelten Differentialgleichung zu einer Differentialgleichung in A:
\begin{equation}
\nabla^2 A - \frac{1}{c^2} \frac{\partial^2 A }{\partial t^2} = 0
\end{equation}

\section{Von der Welle zu gekoppelten Oszillatoren}
L"osungen dieser Gleichung f"ur periodische Randbedingungen und $t=0$ in einer Box mit Seitenl"ange $L = V^{1/3}$ sind durch die Fourier Reihe gegeben:

\begin{equation}
A(x,0) = \frac{1}{\sqrt{V}} \sum_K \sum_{\alpha=1,2} (c_{k,\alpha}(0) \epsilon^{(\alpha)} e^{ikx} + c^*_{k,\alpha}(0) \epsilon^{(\alpha)} e^{-ikx})
\end{equation}

oder durch setzen von $u_{k,\alpha}(x) = \epsilon^{(\alpha)} e^{ikx}$:
\begin{equation}
A(x,0) = \frac{1}{\sqrt{V}} \sum_K \sum_{\alpha=1,2} (c_{k,\alpha}(0)u_{k,\alpha}(x) + c^*_{k,\alpha}(0) u^*_{k,\alpha}(x))
\end{equation}

Wenn diese Gleichung ausgeschrieben wird, sieht man, dass $A(x,t)$ durch diese Wahl f"ur alle $c_{k,\alpha}(t)$ reell bleibt:
\begin{equation}
(a + ib)(\cos kx + i \sin kx ) + (a - ib)(\cos kx - i \sin kx ) = 2 ( a \cos kx - b \sin kx )
\end{equation}
%=a \cos kx + ib \cos kx + ia \sin kx - b \sin kx + a \cos kx - ib \cos kx - ia \sin kx - b \sin kx

$k$ ist der Ausbreitungsvektor der Welle und zeigt in die Ausbreitungsrichtung. $\epsilon^{(\alpha)}$ ist die Polarisation. Dabei wird vorausgesetzt, dass $(\epsilon^{(1)}, \epsilon^{(2)} , k/|k|)$ ein orthogonales Rechtssystem aus Einheitsvektoren bilden.

Da $\epsilon^{(\alpha)}$ und $k$ orthogonal sind gilt dabei auch automatisch:

\begin{equation}
\nabla \cdot A = \frac{1}{\sqrt{V}} \sum_K \sum_{\alpha=1,2} (i c_{k,\alpha}(0) \underbrace{\epsilon^{(\alpha)} k}_{=0} e^{ikx} - i c^*_{k,\alpha}(0) \underbrace{\epsilon^{(\alpha)} k}_{=0} e^{-ikx}) = 0
\end{equation}

Weiterhin gilt durch wegen der Orthogonalit"at auch:
\begin{equation}
\begin{split}
\dfrac{1}{A} \int c_{k,\alpha} \cdot c^*_{k',\alpha'} d^3 x &= \delta_{kk'}\delta{aa'} \\
\dfrac{1}{A} \int c_{k,\alpha} \cdot c_{k',\alpha'} d^3 x &= 0 \\
\dfrac{1}{A} \int c^*_{k,\alpha} \cdot c^*_{k',\alpha'} d^3 x &= 0
\end{split}
\end{equation}

Um $A(x,t)$ zu erhalten, wird:
\begin{equation}
c_{k,\alpha}(t) = c_{k,\alpha}(0) e^{-i \omega t}
\end{equation}

Dabei ist:
\begin{equation}
\begin{split}
\omega=|k|c \\
\lambda = \frac{2 \pi}{|k|}
\end{split}
\end{equation}

Die komplette Wellengleichung wird somit:
\begin{equation}
A(x,t) = \frac{1}{\sqrt{V}} \sum_K \sum_{\alpha=1,2} (c_{k,\alpha}(0) \epsilon^{(\alpha)} e^{i (kx - \omega t)} + c^*_{k,\alpha}(0) \epsilon^{(\alpha)} e^{-i(kx - \omega t)})
\end{equation}

Die Hamilton-Funktion einer elektromagnetischen Welle ist gegeben durch:
\begin{equation}
\begin{split}
H &= \frac{1}{2} \int (|B|^2 + |E|^2) d^3 x \\
	&= \frac{1}{2} \int (| \nabla\times A |^2 + \left| \frac{1}{c} \dfrac{\partial A}{\partial t} \right|^2) d^3 x 
\end{split}
\end{equation}

Es kann gezeigt werden, dass die L"osung dieses Integrals gegeben ist durch:
\begin{equation}
H = \sum_K \sum_{\alpha=1,2} 2 \left(\frac{\omega}{c}\right)^2 c^*_{k,\alpha}(t) c_{k,\alpha}(t)
\end{equation}

Durch folgende Definition:
\begin{equation}
Q_{k,\alpha} = \frac{1}{c}(c_{k,\alpha}(t) + c^*_{k,\alpha}(t)) \quad P_{k,\alpha} = -\frac{i\omega}{c}(c_{k,\alpha}(t) - c^*_{k,\alpha}(t)) 
\end{equation}

wird die Hamilton-Funktion zu:
\begin{equation} \label{fq:hamilton}
\begin{split}
H &= \sum_K \sum_{\alpha=1,2} 2 \left(\frac{\omega}{c}\right)^2 \left[ \frac{c(\omega Q_{k,\alpha} - i P_{k,\alpha})}{2 \omega} \right] \left[ \frac{c(\omega Q_{k,\alpha} + i P_{k,\alpha})}{2 \omega} \right] \\
&= \sum_K \sum_{\alpha=1,2} \frac{1}{2} (P_{k,\alpha}^2 + \omega^2 Q_{k,\alpha}^2)
\end{split}
\end{equation}

Hier sieh man nun, dass es m"oglich ist, eine Welle durch unabh"angige Oszillatoren dar zu stellen.

$Q_{k,\alpha}$ und $P_{k,\alpha}$ k"onnen nun als Koordinaten und Impulse der einzelnen Oszillatoren aufgefasst werden:
\begin{equation}
\dfrac{\partial H}{\partial Q_{k,\alpha}} = -\dot{P}_{k,\alpha} \quad \dfrac{\partial H}{\partial P_{k,\alpha}} = \dot{Q}_{k,\alpha}
\end{equation}

\section{Quantisierung der Welle}

Wie beim harmonischen Oszillator können $Q_{k,\alpha}$ und $P_{k,\alpha}$ nun als Opperatoren aufgefasst werden. Die Vertauschungsrelationen werden dabei zu:
\begin{equation}
\begin{split}
[Q_{k,\alpha}, P_{k',\alpha'}] &= i \hbar \delta_{kk'}\delta_{aa'} \\
[Q_{k,\alpha}, Q_{k',\alpha'}] &= 0 \\
[P_{k,\alpha}, P_{k',\alpha'}] &= 0
\end{split}
\end{equation}

Wir definieren die Operatoren:
\begin{equation}
\begin{split}
a_{k,\alpha} &= (1/\sqrt{2 \hbar \omega})(\omega Q_{k,\alpha} + iP_{k,\alpha})) \\
a^+_{k,\alpha} &= (1/\sqrt{2 \hbar \omega})(\omega Q_{k,\alpha} - iP_{k,\alpha}))\\
N_{k,\alpha} &= a^+_{k,\alpha} a_{k,\alpha}
\end{split}
\end{equation}

Ein Vergleich mit \ref{fq:hamilton} liefert:
\begin{equation}
 c_{k,\alpha} \rightarrow c \sqrt{\hbar/2 \omega} \, a_{k,\alpha} \quad c^*_{k,\alpha} \rightarrow c \sqrt{\hbar/2 \omega} \, a^+_{k,\alpha}
\end{equation}
Somit entsprechen diese Operatoren den Fourier-Koeffizienten.

Die Kommentatoren f"ur diese Operatoren sind:
\begin{equation}
\begin{split}
[a_{k,\alpha} , a^+_{k',\alpha'}] &= - \frac{i}{2 \hbar} [Q_{k,\alpha}, P_{k',\alpha'}] + \frac{i}{2 \hbar} [P_{k,\alpha}, Q_{k',\alpha'}] \\
	 &= \delta_{kk'}\delta_{aa'} \\
[a_{k,\alpha} , a_{k',\alpha'}] &= [a^+_{k,\alpha} , a^+_{k',\alpha'}] \\
	 &= 0 \\
[a_{k,\alpha} , N_{k',\alpha'}] &= [a_{k,\alpha} , a^+_{k',\alpha'}]a_{k',\alpha'} - a^+_{k',\alpha'}[a_{k',\alpha'} , a_{k,\alpha}]\\
	&= \delta_{kk'}\delta_{aa'} a_{k,\alpha} \\
[a^+_{k,\alpha} , N_{k',\alpha'}] &= -\delta_{kk'}\delta_{aa'} a^+_{k,\alpha}
\end{split}
\end{equation}


%\chapter{Feldquantisierung\label{chapter:feldquantisierung}}
\lhead{Feldquantisierung}
\begin{refsection}
\chapterauthor{Hannes Diethelm}

\printbibliography[heading=subbibliography]
\end{refsection}

\section{Maxwell-Gleichungen und elektromagnetische Wellen}

Hilfreich dazu ist auch die Beschreibung von Magnetfeldern in Kapitel \ref{chapter:magnetfeld}. In diese Kapitel wird der Gradient durch $\nabla$ ersetzt \cite{fq:nabla}. Dadurch k"onnen die Gleichungen einfacher geschrieben werden. 

Die in der Elektrotechnik wohl bekannten Maxwell-Gleichungen in SI Einheiten lauten:
\begin{equation}
\begin{split}
\nabla\cdot E &= \frac{\rho}{\varepsilon_0} \\
\nabla\times B &= \mu_0( J  + \varepsilon_0\frac{\partial E}{\partial t}) \\
\nabla\cdot B &=0 \\
\nabla\times E &= -\frac{\partial B }{\partial t}\\
\end{split}
\end{equation}

Dieses Einheitensystem is willk"urlich \cite{fq:em_units}. Im Heaviside-
Lorentz System, das von nun an verwendet wird, lauten die Gleichungen:
\begin{equation}
\begin{split}
\nabla\cdot E &= \rho \\
\nabla\times B &= \frac{1}{c}( J  + \frac{\partial E}{\partial t}) \\
\nabla\cdot B &=0 \\
\nabla\times E &= -\frac{1}{c} \frac{\partial B }{\partial t}\\
\end{split}
\end{equation}

Da $\nabla \cdot B = 0 $ gilt k"onnen diese Gleichungen durch folgende Substitution umformuliert werden:
\begin{equation}
B = \nabla\times A 
\end{equation}

Dadurch gilt $\nabla \cdot B = 0 $ automatisch:
\begin{equation}
\nabla \cdot B = 0 \rightarrow \nabla \cdot ( \nabla\times A ) = 0 \text{ gilt f"ur jedes A! }
\end{equation}

Durch Einsetzen erh"alt man die Gleichung f"ur E:
\begin{equation}
\nabla\times E + \frac{1}{c} \frac{\partial B }{\partial t} = 0
\rightarrow \nabla\times E + \frac{1}{c} \frac{\partial \nabla\times A }{\partial t} = 0 \rightarrow E = -\frac{1}{c} \dfrac{\partial A}{\partial t} - \nabla \phi
\end{equation}

$\nabla \phi$ kann als Integrationskonstante angesehen werden und $\phi$ entspricht dem skalaren Potential des Feldes.

Durch weiteres Einsetzen k"onnen die vier Maxwell-Gleichungen in zwei Gleichungen umgeschrieben werden:
\begin{equation}
\begin{split}
 \nabla^2 \phi + \frac{1}{c} \dfrac{\partial \nabla A}{\partial t} &= -\rho \\
 \nabla^2 A - \frac{1}{c^2} \frac{\partial^2 A }{\partial t^2} - \nabla \left( \nabla \cdot A + \frac{1}{c} \frac{\partial \phi }{\partial t} \right) &= - \frac{1}{c} J
\end{split}
\end{equation}

dabei gelten die Korrespondenzen:
\begin{equation}
\begin{split}
B &= \nabla\times A \\
E &= -\frac{1}{c} \dfrac{\partial A}{\partial t} - \nabla \phi
\end{split}
\end{equation}

Es kann gezeigt werden, dass $\phi$ durch eine Eichtransformation (Siehe \ref{section:eichtransformation}) geeignet gew"ahlt werden kann, damit:

\begin{equation}
\nabla \cdot A + \frac{1}{c} \frac{\partial \phi }{\partial t} = 0
\end{equation}

Dadurch werden die zwei gekoppelten Gleichungen entkoppelt und es gilt:
\begin{equation}
\begin{split}
\nabla^2 \phi - \frac{1}{c^2} \dfrac{\partial^2 \nabla \phi}{\partial t^2} &= -\rho \\
\nabla^2 A - \frac{1}{c^2} \frac{\partial^2 A }{\partial t^2} &= - \frac{1}{c} J
\end{split}
\end{equation}

F"ur weiter wollen ein Feld im Vakkum betrachten. Hierf"ur gilt $J = 0$, da keine Leiter vorhanden sind.
In einem Transversalfeld im Vakkum gilt zudem $\nabla \cdot A = 0$. (ToDo: ??) Dadurch vereinfachen sich die gekoppelten Differentialgleichung zu einer Differentialgleichung in A:
\begin{equation}
\nabla^2 A - \frac{1}{c^2} \frac{\partial^2 A }{\partial t^2} = 0
\end{equation}

\section{Von der Welle zu gekoppelten Oszillatoren}
L"osungen dieser Gleichung f"ur periodische Randbedingungen und $t=0$ in einer Box mit Seitenl"ange $L = V^{1/3}$ sind durch die Fourier Reihe gegeben:

\begin{equation}
A(x,0) = \frac{1}{\sqrt{V}} \sum_K \sum_{\alpha=1,2} (c_{k,\alpha}(0) \epsilon^{(\alpha)} e^{ikx} + c^*_{k,\alpha}(0) \epsilon^{(\alpha)} e^{-ikx})
\end{equation}

oder durch setzen von $u_{k,\alpha}(x) = \epsilon^{(\alpha)} e^{ikx}$:
\begin{equation}
A(x,0) = \frac{1}{\sqrt{V}} \sum_K \sum_{\alpha=1,2} (c_{k,\alpha}(0)u_{k,\alpha}(x) + c^*_{k,\alpha}(0) u^*_{k,\alpha}(x))
\end{equation}

Wenn diese Gleichung ausgeschrieben wird, sieht man, dass $A(x,t)$ durch diese Wahl f"ur alle $c_{k,\alpha}(t)$ reell bleibt:
\begin{equation}
(a + ib)(\cos kx + i \sin kx ) + (a - ib)(\cos kx - i \sin kx ) = 2 ( a \cos kx - b \sin kx )
\end{equation}
%=a \cos kx + ib \cos kx + ia \sin kx - b \sin kx + a \cos kx - ib \cos kx - ia \sin kx - b \sin kx

$k$ ist der Ausbreitungsvektor der Welle und zeigt in die Ausbreitungsrichtung. $\epsilon^{(\alpha)}$ ist die Polarisation. Dabei wird vorausgesetzt, dass $(\epsilon^{(1)}, \epsilon^{(2)} , k/|k|)$ ein orthogonales Rechtssystem aus Einheitsvektoren bilden.

Da $\epsilon^{(\alpha)}$ und $k$ orthogonal sind gilt dabei auch automatisch:

\begin{equation}
\nabla \cdot A = \frac{1}{\sqrt{V}} \sum_K \sum_{\alpha=1,2} (i c_{k,\alpha}(0) \underbrace{\epsilon^{(\alpha)} k}_{=0} e^{ikx} - i c^*_{k,\alpha}(0) \underbrace{\epsilon^{(\alpha)} k}_{=0} e^{-ikx}) = 0
\end{equation}

Weiterhin gilt durch wegen der Orthogonalit"at auch:
\begin{equation}
\begin{split}
\dfrac{1}{A} \int c_{k,\alpha} \cdot c^*_{k',\alpha'} d^3 x &= \delta_{kk'}\delta{aa'} \\
\dfrac{1}{A} \int c_{k,\alpha} \cdot c_{k',\alpha'} d^3 x &= 0 \\
\dfrac{1}{A} \int c^*_{k,\alpha} \cdot c^*_{k',\alpha'} d^3 x &= 0
\end{split}
\end{equation}

Um $A(x,t)$ zu erhalten, wird:
\begin{equation}
c_{k,\alpha}(t) = c_{k,\alpha}(0) e^{-i \omega t}
\end{equation}

Dabei ist:
\begin{equation}
\begin{split}
\omega=|k|c \\
\lambda = \frac{2 \pi}{|k|}
\end{split}
\end{equation}

Die komplette Wellengleichung wird somit:
\begin{equation}
A(x,t) = \frac{1}{\sqrt{V}} \sum_K \sum_{\alpha=1,2} (c_{k,\alpha}(0) \epsilon^{(\alpha)} e^{i (kx - \omega t)} + c^*_{k,\alpha}(0) \epsilon^{(\alpha)} e^{-i(kx - \omega t)})
\end{equation}

Die Hamilton-Funktion einer elektromagnetischen Welle ist gegeben durch:
\begin{equation}
\begin{split}
H &= \frac{1}{2} \int (|B|^2 + |E|^2) d^3 x \\
	&= \frac{1}{2} \int (| \nabla\times A |^2 + \left| \frac{1}{c} \dfrac{\partial A}{\partial t} \right|^2) d^3 x 
\end{split}
\end{equation}

Es kann gezeigt werden, dass die L"osung dieses Integrals gegeben ist durch:
\begin{equation}
H = \sum_K \sum_{\alpha=1,2} 2 \left(\frac{\omega}{c}\right)^2 c^*_{k,\alpha}(t) c_{k,\alpha}(t)
\end{equation}

Durch folgende Definition:
\begin{equation}
Q_{k,\alpha} = \frac{1}{c}(c_{k,\alpha}(t) + c^*_{k,\alpha}(t)) \quad P_{k,\alpha} = -\frac{i\omega}{c}(c_{k,\alpha}(t) - c^*_{k,\alpha}(t)) 
\end{equation}

wird die Hamilton-Funktion zu:
\begin{equation} \label{fq:hamilton}
\begin{split}
H &= \sum_K \sum_{\alpha=1,2} 2 \left(\frac{\omega}{c}\right)^2 \left[ \frac{c(\omega Q_{k,\alpha} - i P_{k,\alpha})}{2 \omega} \right] \left[ \frac{c(\omega Q_{k,\alpha} + i P_{k,\alpha})}{2 \omega} \right] \\
&= \sum_K \sum_{\alpha=1,2} \frac{1}{2} (P_{k,\alpha}^2 + \omega^2 Q_{k,\alpha}^2)
\end{split}
\end{equation}

Hier sieh man nun, dass es m"oglich ist, eine Welle durch unabh"angige Oszillatoren dar zu stellen.

$Q_{k,\alpha}$ und $P_{k,\alpha}$ k"onnen nun als Koordinaten und Impulse der einzelnen Oszillatoren aufgefasst werden:
\begin{equation}
\dfrac{\partial H}{\partial Q_{k,\alpha}} = -\dot{P}_{k,\alpha} \quad \dfrac{\partial H}{\partial P_{k,\alpha}} = \dot{Q}_{k,\alpha}
\end{equation}

\section{Quantisierung der Welle}

Wie beim harmonischen Oszillator können $Q_{k,\alpha}$ und $P_{k,\alpha}$ nun als Opperatoren aufgefasst werden. Die Vertauschungsrelationen werden dabei zu:
\begin{equation}
\begin{split}
[Q_{k,\alpha}, P_{k',\alpha'}] &= i \hbar \delta_{kk'}\delta_{aa'} \\
[Q_{k,\alpha}, Q_{k',\alpha'}] &= 0 \\
[P_{k,\alpha}, P_{k',\alpha'}] &= 0
\end{split}
\end{equation}

Wir definieren die Operatoren:
\begin{equation}
\begin{split}
a_{k,\alpha} &= (1/\sqrt{2 \hbar \omega})(\omega Q_{k,\alpha} + iP_{k,\alpha})) \\
a^+_{k,\alpha} &= (1/\sqrt{2 \hbar \omega})(\omega Q_{k,\alpha} - iP_{k,\alpha}))\\
N_{k,\alpha} &= a^+_{k,\alpha} a_{k,\alpha}
\end{split}
\end{equation}

Ein Vergleich mit \ref{fq:hamilton} liefert:
\begin{equation}
 c_{k,\alpha} \rightarrow c \sqrt{\hbar/2 \omega} \, a_{k,\alpha} \quad c^*_{k,\alpha} \rightarrow c \sqrt{\hbar/2 \omega} \, a^+_{k,\alpha}
\end{equation}
Somit entsprechen diese Operatoren den Fourier-Koeffizienten.

Die Kommentatoren f"ur diese Operatoren sind:
\begin{equation}
\begin{split}
[a_{k,\alpha} , a^+_{k',\alpha'}] &= - \frac{i}{2 \hbar} [Q_{k,\alpha}, P_{k',\alpha'}] + \frac{i}{2 \hbar} [P_{k,\alpha}, Q_{k',\alpha'}] \\
	 &= \delta_{kk'}\delta_{aa'} \\
[a_{k,\alpha} , a_{k',\alpha'}] &= [a^+_{k,\alpha} , a^+_{k',\alpha'}] \\
	 &= 0 \\
[a_{k,\alpha} , N_{k',\alpha'}] &= [a_{k,\alpha} , a^+_{k',\alpha'}]a_{k',\alpha'} - a^+_{k',\alpha'}[a_{k',\alpha'} , a_{k,\alpha}]\\
	&= \delta_{kk'}\delta_{aa'} a_{k,\alpha} \\
[a^+_{k,\alpha} , N_{k',\alpha'}] &= -\delta_{kk'}\delta_{aa'} a^+_{k,\alpha}
\end{split}
\end{equation}


%\chapter{Feldquantisierung\label{chapter:feldquantisierung}}
\lhead{Feldquantisierung}
\begin{refsection}
\chapterauthor{Hannes Diethelm}

\printbibliography[heading=subbibliography]
\end{refsection}

\section{Maxwell-Gleichungen und elektromagnetische Wellen}

Hilfreich dazu ist auch die Beschreibung von Magnetfeldern in Kapitel \ref{chapter:magnetfeld}. In diese Kapitel wird der Gradient durch $\nabla$ ersetzt \cite{fq:nabla}. Dadurch k"onnen die Gleichungen einfacher geschrieben werden. 

Die in der Elektrotechnik wohl bekannten Maxwell-Gleichungen in SI Einheiten lauten:
\begin{equation}
\begin{split}
\nabla\cdot E &= \frac{\rho}{\varepsilon_0} \\
\nabla\times B &= \mu_0( J  + \varepsilon_0\frac{\partial E}{\partial t}) \\
\nabla\cdot B &=0 \\
\nabla\times E &= -\frac{\partial B }{\partial t}\\
\end{split}
\end{equation}

Dieses Einheitensystem is willk"urlich \cite{fq:em_units}. Im Heaviside-
Lorentz System, das von nun an verwendet wird, lauten die Gleichungen:
\begin{equation}
\begin{split}
\nabla\cdot E &= \rho \\
\nabla\times B &= \frac{1}{c}( J  + \frac{\partial E}{\partial t}) \\
\nabla\cdot B &=0 \\
\nabla\times E &= -\frac{1}{c} \frac{\partial B }{\partial t}\\
\end{split}
\end{equation}

Da $\nabla \cdot B = 0 $ gilt k"onnen diese Gleichungen durch folgende Substitution umformuliert werden:
\begin{equation}
B = \nabla\times A 
\end{equation}

Dadurch gilt $\nabla \cdot B = 0 $ automatisch:
\begin{equation}
\nabla \cdot B = 0 \rightarrow \nabla \cdot ( \nabla\times A ) = 0 \text{ gilt f"ur jedes A! }
\end{equation}

Durch Einsetzen erh"alt man die Gleichung f"ur E:
\begin{equation}
\nabla\times E + \frac{1}{c} \frac{\partial B }{\partial t} = 0
\rightarrow \nabla\times E + \frac{1}{c} \frac{\partial \nabla\times A }{\partial t} = 0 \rightarrow E = -\frac{1}{c} \dfrac{\partial A}{\partial t} - \nabla \phi
\end{equation}

$\nabla \phi$ kann als Integrationskonstante angesehen werden und $\phi$ entspricht dem skalaren Potential des Feldes.

Durch weiteres Einsetzen k"onnen die vier Maxwell-Gleichungen in zwei Gleichungen umgeschrieben werden:
\begin{equation}
\begin{split}
 \nabla^2 \phi + \frac{1}{c} \dfrac{\partial \nabla A}{\partial t} &= -\rho \\
 \nabla^2 A - \frac{1}{c^2} \frac{\partial^2 A }{\partial t^2} - \nabla \left( \nabla \cdot A + \frac{1}{c} \frac{\partial \phi }{\partial t} \right) &= - \frac{1}{c} J
\end{split}
\end{equation}

dabei gelten die Korrespondenzen:
\begin{equation}
\begin{split}
B &= \nabla\times A \\
E &= -\frac{1}{c} \dfrac{\partial A}{\partial t} - \nabla \phi
\end{split}
\end{equation}

Es kann gezeigt werden, dass $\phi$ durch eine Eichtransformation (Siehe \ref{section:eichtransformation}) geeignet gew"ahlt werden kann, damit:

\begin{equation}
\nabla \cdot A + \frac{1}{c} \frac{\partial \phi }{\partial t} = 0
\end{equation}

Dadurch werden die zwei gekoppelten Gleichungen entkoppelt und es gilt:
\begin{equation}
\begin{split}
\nabla^2 \phi - \frac{1}{c^2} \dfrac{\partial^2 \nabla \phi}{\partial t^2} &= -\rho \\
\nabla^2 A - \frac{1}{c^2} \frac{\partial^2 A }{\partial t^2} &= - \frac{1}{c} J
\end{split}
\end{equation}

F"ur weiter wollen ein Feld im Vakkum betrachten. Hierf"ur gilt $J = 0$, da keine Leiter vorhanden sind.
In einem Transversalfeld im Vakkum gilt zudem $\nabla \cdot A = 0$. (ToDo: ??) Dadurch vereinfachen sich die gekoppelten Differentialgleichung zu einer Differentialgleichung in A:
\begin{equation}
\nabla^2 A - \frac{1}{c^2} \frac{\partial^2 A }{\partial t^2} = 0
\end{equation}

\section{Von der Welle zu gekoppelten Oszillatoren}
L"osungen dieser Gleichung f"ur periodische Randbedingungen und $t=0$ in einer Box mit Seitenl"ange $L = V^{1/3}$ sind durch die Fourier Reihe gegeben:

\begin{equation}
A(x,0) = \frac{1}{\sqrt{V}} \sum_K \sum_{\alpha=1,2} (c_{k,\alpha}(0) \epsilon^{(\alpha)} e^{ikx} + c^*_{k,\alpha}(0) \epsilon^{(\alpha)} e^{-ikx})
\end{equation}

oder durch setzen von $u_{k,\alpha}(x) = \epsilon^{(\alpha)} e^{ikx}$:
\begin{equation}
A(x,0) = \frac{1}{\sqrt{V}} \sum_K \sum_{\alpha=1,2} (c_{k,\alpha}(0)u_{k,\alpha}(x) + c^*_{k,\alpha}(0) u^*_{k,\alpha}(x))
\end{equation}

Wenn diese Gleichung ausgeschrieben wird, sieht man, dass $A(x,t)$ durch diese Wahl f"ur alle $c_{k,\alpha}(t)$ reell bleibt:
\begin{equation}
(a + ib)(\cos kx + i \sin kx ) + (a - ib)(\cos kx - i \sin kx ) = 2 ( a \cos kx - b \sin kx )
\end{equation}
%=a \cos kx + ib \cos kx + ia \sin kx - b \sin kx + a \cos kx - ib \cos kx - ia \sin kx - b \sin kx

$k$ ist der Ausbreitungsvektor der Welle und zeigt in die Ausbreitungsrichtung. $\epsilon^{(\alpha)}$ ist die Polarisation. Dabei wird vorausgesetzt, dass $(\epsilon^{(1)}, \epsilon^{(2)} , k/|k|)$ ein orthogonales Rechtssystem aus Einheitsvektoren bilden.

Da $\epsilon^{(\alpha)}$ und $k$ orthogonal sind gilt dabei auch automatisch:

\begin{equation}
\nabla \cdot A = \frac{1}{\sqrt{V}} \sum_K \sum_{\alpha=1,2} (i c_{k,\alpha}(0) \underbrace{\epsilon^{(\alpha)} k}_{=0} e^{ikx} - i c^*_{k,\alpha}(0) \underbrace{\epsilon^{(\alpha)} k}_{=0} e^{-ikx}) = 0
\end{equation}

Weiterhin gilt durch wegen der Orthogonalit"at auch:
\begin{equation}
\begin{split}
\dfrac{1}{A} \int c_{k,\alpha} \cdot c^*_{k',\alpha'} d^3 x &= \delta_{kk'}\delta{aa'} \\
\dfrac{1}{A} \int c_{k,\alpha} \cdot c_{k',\alpha'} d^3 x &= 0 \\
\dfrac{1}{A} \int c^*_{k,\alpha} \cdot c^*_{k',\alpha'} d^3 x &= 0
\end{split}
\end{equation}

Um $A(x,t)$ zu erhalten, wird:
\begin{equation}
c_{k,\alpha}(t) = c_{k,\alpha}(0) e^{-i \omega t}
\end{equation}

Dabei ist:
\begin{equation}
\begin{split}
\omega=|k|c \\
\lambda = \frac{2 \pi}{|k|}
\end{split}
\end{equation}

Die komplette Wellengleichung wird somit:
\begin{equation}
A(x,t) = \frac{1}{\sqrt{V}} \sum_K \sum_{\alpha=1,2} (c_{k,\alpha}(0) \epsilon^{(\alpha)} e^{i (kx - \omega t)} + c^*_{k,\alpha}(0) \epsilon^{(\alpha)} e^{-i(kx - \omega t)})
\end{equation}

Die Hamilton-Funktion einer elektromagnetischen Welle ist gegeben durch:
\begin{equation}
\begin{split}
H &= \frac{1}{2} \int (|B|^2 + |E|^2) d^3 x \\
	&= \frac{1}{2} \int (| \nabla\times A |^2 + \left| \frac{1}{c} \dfrac{\partial A}{\partial t} \right|^2) d^3 x 
\end{split}
\end{equation}

Es kann gezeigt werden, dass die L"osung dieses Integrals gegeben ist durch:
\begin{equation}
H = \sum_K \sum_{\alpha=1,2} 2 \left(\frac{\omega}{c}\right)^2 c^*_{k,\alpha}(t) c_{k,\alpha}(t)
\end{equation}

Durch folgende Definition:
\begin{equation}
Q_{k,\alpha} = \frac{1}{c}(c_{k,\alpha}(t) + c^*_{k,\alpha}(t)) \quad P_{k,\alpha} = -\frac{i\omega}{c}(c_{k,\alpha}(t) - c^*_{k,\alpha}(t)) 
\end{equation}

wird die Hamilton-Funktion zu:
\begin{equation} \label{fq:hamilton}
\begin{split}
H &= \sum_K \sum_{\alpha=1,2} 2 \left(\frac{\omega}{c}\right)^2 \left[ \frac{c(\omega Q_{k,\alpha} - i P_{k,\alpha})}{2 \omega} \right] \left[ \frac{c(\omega Q_{k,\alpha} + i P_{k,\alpha})}{2 \omega} \right] \\
&= \sum_K \sum_{\alpha=1,2} \frac{1}{2} (P_{k,\alpha}^2 + \omega^2 Q_{k,\alpha}^2)
\end{split}
\end{equation}

Hier sieh man nun, dass es m"oglich ist, eine Welle durch unabh"angige Oszillatoren dar zu stellen.

$Q_{k,\alpha}$ und $P_{k,\alpha}$ k"onnen nun als Koordinaten und Impulse der einzelnen Oszillatoren aufgefasst werden:
\begin{equation}
\dfrac{\partial H}{\partial Q_{k,\alpha}} = -\dot{P}_{k,\alpha} \quad \dfrac{\partial H}{\partial P_{k,\alpha}} = \dot{Q}_{k,\alpha}
\end{equation}

\section{Quantisierung der Welle}

Wie beim harmonischen Oszillator können $Q_{k,\alpha}$ und $P_{k,\alpha}$ nun als Opperatoren aufgefasst werden. Die Vertauschungsrelationen werden dabei zu:
\begin{equation}
\begin{split}
[Q_{k,\alpha}, P_{k',\alpha'}] &= i \hbar \delta_{kk'}\delta_{aa'} \\
[Q_{k,\alpha}, Q_{k',\alpha'}] &= 0 \\
[P_{k,\alpha}, P_{k',\alpha'}] &= 0
\end{split}
\end{equation}

Wir definieren die Operatoren:
\begin{equation}
\begin{split}
a_{k,\alpha} &= (1/\sqrt{2 \hbar \omega})(\omega Q_{k,\alpha} + iP_{k,\alpha})) \\
a^+_{k,\alpha} &= (1/\sqrt{2 \hbar \omega})(\omega Q_{k,\alpha} - iP_{k,\alpha}))\\
N_{k,\alpha} &= a^+_{k,\alpha} a_{k,\alpha}
\end{split}
\end{equation}

Ein Vergleich mit \ref{fq:hamilton} liefert:
\begin{equation}
 c_{k,\alpha} \rightarrow c \sqrt{\hbar/2 \omega} \, a_{k,\alpha} \quad c^*_{k,\alpha} \rightarrow c \sqrt{\hbar/2 \omega} \, a^+_{k,\alpha}
\end{equation}
Somit entsprechen diese Operatoren den Fourier-Koeffizienten.

Die Kommentatoren f"ur diese Operatoren sind:
\begin{equation}
\begin{split}
[a_{k,\alpha} , a^+_{k',\alpha'}] &= - \frac{i}{2 \hbar} [Q_{k,\alpha}, P_{k',\alpha'}] + \frac{i}{2 \hbar} [P_{k,\alpha}, Q_{k',\alpha'}] \\
	 &= \delta_{kk'}\delta_{aa'} \\
[a_{k,\alpha} , a_{k',\alpha'}] &= [a^+_{k,\alpha} , a^+_{k',\alpha'}] \\
	 &= 0 \\
[a_{k,\alpha} , N_{k',\alpha'}] &= [a_{k,\alpha} , a^+_{k',\alpha'}]a_{k',\alpha'} - a^+_{k',\alpha'}[a_{k',\alpha'} , a_{k,\alpha}]\\
	&= \delta_{kk'}\delta_{aa'} a_{k,\alpha} \\
[a^+_{k,\alpha} , N_{k',\alpha'}] &= -\delta_{kk'}\delta_{aa'} a^+_{k,\alpha}
\end{split}
\end{equation}


%\chapter{Feldquantisierung\label{chapter:feldquantisierung}}
\lhead{Feldquantisierung}
\begin{refsection}
\chapterauthor{Hannes Diethelm}

\printbibliography[heading=subbibliography]
\end{refsection}

\section{Maxwell-Gleichungen und elektromagnetische Wellen}

Hilfreich dazu ist auch die Beschreibung von Magnetfeldern in Kapitel \ref{chapter:magnetfeld}. In diese Kapitel wird der Gradient durch $\nabla$ ersetzt \cite{fq:nabla}. Dadurch k"onnen die Gleichungen einfacher geschrieben werden. 

Die in der Elektrotechnik wohl bekannten Maxwell-Gleichungen in SI Einheiten lauten:
\begin{equation}
\begin{split}
\nabla\cdot E &= \frac{\rho}{\varepsilon_0} \\
\nabla\times B &= \mu_0( J  + \varepsilon_0\frac{\partial E}{\partial t}) \\
\nabla\cdot B &=0 \\
\nabla\times E &= -\frac{\partial B }{\partial t}\\
\end{split}
\end{equation}

Dieses Einheitensystem is willk"urlich \cite{fq:em_units}. Im Heaviside-
Lorentz System, das von nun an verwendet wird, lauten die Gleichungen:
\begin{equation}
\begin{split}
\nabla\cdot E &= \rho \\
\nabla\times B &= \frac{1}{c}( J  + \frac{\partial E}{\partial t}) \\
\nabla\cdot B &=0 \\
\nabla\times E &= -\frac{1}{c} \frac{\partial B }{\partial t}\\
\end{split}
\end{equation}

Da $\nabla \cdot B = 0 $ gilt k"onnen diese Gleichungen durch folgende Substitution umformuliert werden:
\begin{equation}
B = \nabla\times A 
\end{equation}

Dadurch gilt $\nabla \cdot B = 0 $ automatisch:
\begin{equation}
\nabla \cdot B = 0 \rightarrow \nabla \cdot ( \nabla\times A ) = 0 \text{ gilt f"ur jedes A! }
\end{equation}

Durch Einsetzen erh"alt man die Gleichung f"ur E:
\begin{equation}
\nabla\times E + \frac{1}{c} \frac{\partial B }{\partial t} = 0
\rightarrow \nabla\times E + \frac{1}{c} \frac{\partial \nabla\times A }{\partial t} = 0 \rightarrow E = -\frac{1}{c} \dfrac{\partial A}{\partial t} - \nabla \phi
\end{equation}

$\nabla \phi$ kann als Integrationskonstante angesehen werden und $\phi$ entspricht dem skalaren Potential des Feldes.

Durch weiteres Einsetzen k"onnen die vier Maxwell-Gleichungen in zwei Gleichungen umgeschrieben werden:
\begin{equation}
\begin{split}
 \nabla^2 \phi + \frac{1}{c} \dfrac{\partial \nabla A}{\partial t} &= -\rho \\
 \nabla^2 A - \frac{1}{c^2} \frac{\partial^2 A }{\partial t^2} - \nabla \left( \nabla \cdot A + \frac{1}{c} \frac{\partial \phi }{\partial t} \right) &= - \frac{1}{c} J
\end{split}
\end{equation}

dabei gelten die Korrespondenzen:
\begin{equation}
\begin{split}
B &= \nabla\times A \\
E &= -\frac{1}{c} \dfrac{\partial A}{\partial t} - \nabla \phi
\end{split}
\end{equation}

Es kann gezeigt werden, dass $\phi$ durch eine Eichtransformation (Siehe \ref{section:eichtransformation}) geeignet gew"ahlt werden kann, damit:

\begin{equation}
\nabla \cdot A + \frac{1}{c} \frac{\partial \phi }{\partial t} = 0
\end{equation}

Dadurch werden die zwei gekoppelten Gleichungen entkoppelt und es gilt:
\begin{equation}
\begin{split}
\nabla^2 \phi - \frac{1}{c^2} \dfrac{\partial^2 \nabla \phi}{\partial t^2} &= -\rho \\
\nabla^2 A - \frac{1}{c^2} \frac{\partial^2 A }{\partial t^2} &= - \frac{1}{c} J
\end{split}
\end{equation}

F"ur weiter wollen ein Feld im Vakkum betrachten. Hierf"ur gilt $J = 0$, da keine Leiter vorhanden sind.
In einem Transversalfeld im Vakkum gilt zudem $\nabla \cdot A = 0$. (ToDo: ??) Dadurch vereinfachen sich die gekoppelten Differentialgleichung zu einer Differentialgleichung in A:
\begin{equation}
\nabla^2 A - \frac{1}{c^2} \frac{\partial^2 A }{\partial t^2} = 0
\end{equation}

\section{Von der Welle zu gekoppelten Oszillatoren}
L"osungen dieser Gleichung f"ur periodische Randbedingungen und $t=0$ in einer Box mit Seitenl"ange $L = V^{1/3}$ sind durch die Fourier Reihe gegeben:

\begin{equation}
A(x,0) = \frac{1}{\sqrt{V}} \sum_K \sum_{\alpha=1,2} (c_{k,\alpha}(0) \epsilon^{(\alpha)} e^{ikx} + c^*_{k,\alpha}(0) \epsilon^{(\alpha)} e^{-ikx})
\end{equation}

oder durch setzen von $u_{k,\alpha}(x) = \epsilon^{(\alpha)} e^{ikx}$:
\begin{equation}
A(x,0) = \frac{1}{\sqrt{V}} \sum_K \sum_{\alpha=1,2} (c_{k,\alpha}(0)u_{k,\alpha}(x) + c^*_{k,\alpha}(0) u^*_{k,\alpha}(x))
\end{equation}

Wenn diese Gleichung ausgeschrieben wird, sieht man, dass $A(x,t)$ durch diese Wahl f"ur alle $c_{k,\alpha}(t)$ reell bleibt:
\begin{equation}
(a + ib)(\cos kx + i \sin kx ) + (a - ib)(\cos kx - i \sin kx ) = 2 ( a \cos kx - b \sin kx )
\end{equation}
%=a \cos kx + ib \cos kx + ia \sin kx - b \sin kx + a \cos kx - ib \cos kx - ia \sin kx - b \sin kx

$k$ ist der Ausbreitungsvektor der Welle und zeigt in die Ausbreitungsrichtung. $\epsilon^{(\alpha)}$ ist die Polarisation. Dabei wird vorausgesetzt, dass $(\epsilon^{(1)}, \epsilon^{(2)} , k/|k|)$ ein orthogonales Rechtssystem aus Einheitsvektoren bilden.

Da $\epsilon^{(\alpha)}$ und $k$ orthogonal sind gilt dabei auch automatisch:

\begin{equation}
\nabla \cdot A = \frac{1}{\sqrt{V}} \sum_K \sum_{\alpha=1,2} (i c_{k,\alpha}(0) \underbrace{\epsilon^{(\alpha)} k}_{=0} e^{ikx} - i c^*_{k,\alpha}(0) \underbrace{\epsilon^{(\alpha)} k}_{=0} e^{-ikx}) = 0
\end{equation}

Weiterhin gilt durch wegen der Orthogonalit"at auch:
\begin{equation}
\begin{split}
\dfrac{1}{A} \int c_{k,\alpha} \cdot c^*_{k',\alpha'} d^3 x &= \delta_{kk'}\delta{aa'} \\
\dfrac{1}{A} \int c_{k,\alpha} \cdot c_{k',\alpha'} d^3 x &= 0 \\
\dfrac{1}{A} \int c^*_{k,\alpha} \cdot c^*_{k',\alpha'} d^3 x &= 0
\end{split}
\end{equation}

Um $A(x,t)$ zu erhalten, wird:
\begin{equation}
c_{k,\alpha}(t) = c_{k,\alpha}(0) e^{-i \omega t}
\end{equation}

Dabei ist:
\begin{equation}
\begin{split}
\omega=|k|c \\
\lambda = \frac{2 \pi}{|k|}
\end{split}
\end{equation}

Die komplette Wellengleichung wird somit:
\begin{equation}
A(x,t) = \frac{1}{\sqrt{V}} \sum_K \sum_{\alpha=1,2} (c_{k,\alpha}(0) \epsilon^{(\alpha)} e^{i (kx - \omega t)} + c^*_{k,\alpha}(0) \epsilon^{(\alpha)} e^{-i(kx - \omega t)})
\end{equation}

Die Hamilton-Funktion einer elektromagnetischen Welle ist gegeben durch:
\begin{equation}
\begin{split}
H &= \frac{1}{2} \int (|B|^2 + |E|^2) d^3 x \\
	&= \frac{1}{2} \int (| \nabla\times A |^2 + \left| \frac{1}{c} \dfrac{\partial A}{\partial t} \right|^2) d^3 x 
\end{split}
\end{equation}

Es kann gezeigt werden, dass die L"osung dieses Integrals gegeben ist durch:
\begin{equation}
H = \sum_K \sum_{\alpha=1,2} 2 \left(\frac{\omega}{c}\right)^2 c^*_{k,\alpha}(t) c_{k,\alpha}(t)
\end{equation}

Durch folgende Definition:
\begin{equation}
Q_{k,\alpha} = \frac{1}{c}(c_{k,\alpha}(t) + c^*_{k,\alpha}(t)) \quad P_{k,\alpha} = -\frac{i\omega}{c}(c_{k,\alpha}(t) - c^*_{k,\alpha}(t)) 
\end{equation}

wird die Hamilton-Funktion zu:
\begin{equation} \label{fq:hamilton}
\begin{split}
H &= \sum_K \sum_{\alpha=1,2} 2 \left(\frac{\omega}{c}\right)^2 \left[ \frac{c(\omega Q_{k,\alpha} - i P_{k,\alpha})}{2 \omega} \right] \left[ \frac{c(\omega Q_{k,\alpha} + i P_{k,\alpha})}{2 \omega} \right] \\
&= \sum_K \sum_{\alpha=1,2} \frac{1}{2} (P_{k,\alpha}^2 + \omega^2 Q_{k,\alpha}^2)
\end{split}
\end{equation}

Hier sieh man nun, dass es m"oglich ist, eine Welle durch unabh"angige Oszillatoren dar zu stellen.

$Q_{k,\alpha}$ und $P_{k,\alpha}$ k"onnen nun als Koordinaten und Impulse der einzelnen Oszillatoren aufgefasst werden:
\begin{equation}
\dfrac{\partial H}{\partial Q_{k,\alpha}} = -\dot{P}_{k,\alpha} \quad \dfrac{\partial H}{\partial P_{k,\alpha}} = \dot{Q}_{k,\alpha}
\end{equation}

\section{Quantisierung der Welle}

Wie beim harmonischen Oszillator können $Q_{k,\alpha}$ und $P_{k,\alpha}$ nun als Opperatoren aufgefasst werden. Die Vertauschungsrelationen werden dabei zu:
\begin{equation}
\begin{split}
[Q_{k,\alpha}, P_{k',\alpha'}] &= i \hbar \delta_{kk'}\delta_{aa'} \\
[Q_{k,\alpha}, Q_{k',\alpha'}] &= 0 \\
[P_{k,\alpha}, P_{k',\alpha'}] &= 0
\end{split}
\end{equation}

Wir definieren die Operatoren:
\begin{equation}
\begin{split}
a_{k,\alpha} &= (1/\sqrt{2 \hbar \omega})(\omega Q_{k,\alpha} + iP_{k,\alpha})) \\
a^+_{k,\alpha} &= (1/\sqrt{2 \hbar \omega})(\omega Q_{k,\alpha} - iP_{k,\alpha}))\\
N_{k,\alpha} &= a^+_{k,\alpha} a_{k,\alpha}
\end{split}
\end{equation}

Ein Vergleich mit \ref{fq:hamilton} liefert:
\begin{equation}
 c_{k,\alpha} \rightarrow c \sqrt{\hbar/2 \omega} \, a_{k,\alpha} \quad c^*_{k,\alpha} \rightarrow c \sqrt{\hbar/2 \omega} \, a^+_{k,\alpha}
\end{equation}
Somit entsprechen diese Operatoren den Fourier-Koeffizienten.

Die Kommentatoren f"ur diese Operatoren sind:
\begin{equation}
\begin{split}
[a_{k,\alpha} , a^+_{k',\alpha'}] &= - \frac{i}{2 \hbar} [Q_{k,\alpha}, P_{k',\alpha'}] + \frac{i}{2 \hbar} [P_{k,\alpha}, Q_{k',\alpha'}] \\
	 &= \delta_{kk'}\delta_{aa'} \\
[a_{k,\alpha} , a_{k',\alpha'}] &= [a^+_{k,\alpha} , a^+_{k',\alpha'}] \\
	 &= 0 \\
[a_{k,\alpha} , N_{k',\alpha'}] &= [a_{k,\alpha} , a^+_{k',\alpha'}]a_{k',\alpha'} - a^+_{k',\alpha'}[a_{k',\alpha'} , a_{k,\alpha}]\\
	&= \delta_{kk'}\delta_{aa'} a_{k,\alpha} \\
[a^+_{k,\alpha} , N_{k',\alpha'}] &= -\delta_{kk'}\delta_{aa'} a^+_{k,\alpha}
\end{split}
\end{equation}


\vfill
\pagebreak
\ifodd\value{page}\else\null\clearpage\fi
\lhead{Index}
\rhead{}
%
% skript.tex -- Skript ueber Quantenmechanik
%
% (c) 2012 Prof. Dr. Andreas Mueller, HSR
% $Id: ws-skript.tex,v 1.34 2008/11/02 22:46:16 afm Exp $
%
%\documentclass[a4paper,12pt]{book}
%\documentclass[a4paper]{book}
\documentclass{book}
\usepackage{etex}
\usepackage{geometry}
\geometry{papersize={170mm,240mm},total={140mm,200mm},top=21mm,bindingoffset=10mm}
\usepackage[ngerman]{babel}
\usepackage{times}
\usepackage{amsmath}
\usepackage{amssymb}
\usepackage{amsfonts}
\usepackage{amsthm}
\usepackage{graphicx}
\usepackage{fancyhdr}
\usepackage{textcomp}
\usepackage[all]{xy}
\usepackage{txfonts}
\usepackage{alltt}
\usepackage{verbatim}
\usepackage{paralist}
\usepackage{makeidx}
\usepackage{array}
\usepackage{hyperref}
\usepackage{tikz}
\usepackage{placeins}
\usepackage{subfigure}
\usepackage{csquotes}
\usepackage{float}
\usepackage{enumitem}
\usepackage{wasysym}
%\usetikzlibrary{arrows,decorations.pathmorphing,positioning,fit,petri}
\usetikzlibrary{calc,intersections,through,backgrounds,graphs,positioning,shapes,arrows}
\usetikzlibrary{patterns,decorations.pathreplacing}
%\usetikzlibrary{shapes,snakes,trees}
\usetikzlibrary{decorations.pathreplacing}
%\usetikzlibrary{patterns}
\usepackage{siunitx}
\usepackage{tabularx}
\usetikzlibrary{arrows}
\usepackage{listings}
\lstdefinestyle{Matlab}{
  numbers=left,
  belowcaptionskip=1\baselineskip,
  breaklines=true,
  frame=L,
  xleftmargin=\parindent,
  language=Matlab,
  showstringspaces=false,
  basicstyle=\footnotesize\ttfamily,
  keywordstyle=\bfseries\color{green!40!black},
  commentstyle=\itshape\color{purple!40!black},
  identifierstyle=\color{blue},
  stringstyle=\color{orange},
  numberstyle=\ttfamily\tiny
}
\usepackage{caption}
\usepackage{standalone}
\usepackage[backend=bibtex]{biblatex}
\addbibresource{references.bib}
\addbibresource{tunneldiode/main.bib}
\addbibresource{atomuhr/main.bib}
\addbibresource{rtm/main.bib}
\addbibresource{efeld/main.bib}
\addbibresource{kugel/main.bib}
\addbibresource{orbitale/main.bib}
\addbibresource{flash/main.bib}
\addbibresource{franckhertz/main.bib}
\addbibresource{anharmonisch/main.bib}
\addbibresource{laser/main.bib}
\addbibresource{mri/main.bib}
\addbibresource{crypto/main.bib}
\addbibresource{teleport/main.bib}
\addbibresource{heisenberg/main.bib}
\addbibresource{stark/main.bib}
\addbibresource{bell/main.bib}
\AtEndDocument{\clearpage\ifodd\value{page}\else\null\clearpage\fi}
\makeindex
\begin{document}
\pagestyle{fancy}
\frontmatter
\newcommand\HRule{\noindent\rule{\linewidth}{1.5pt}}
\begin{titlepage}
\vspace*{\stretch{1}}
\HRule
\vspace*{5pt}
\begin{flushright}
{
\LARGE
Mathematisches Seminar\\
\vspace*{20pt}
\Huge
Quantenmechanik%
}
\vspace*{5pt}
\end{flushright}
\HRule
\begin{flushright}
\vspace{60pt}
\Large
Leitung: Andreas M"uller\\
\vspace{40pt}
\Large
Dorian Amiet,
Hannes Badertscher,
Roger Billeter,
Joel Brunner,
Dominik B"usser,
Christian Cavegn,
Michael Cerny,
Reto Christen,
Hannes Diethelm,
Benny G"achter,
Daniel Gubser,
Thomas Gujer,
Stefan Hedinger,
Marc Juchli,
Simon Kuster,
Gabriel Looser,
Andreas Linggi,
Raphael Mattle,
Daniel Monti,
Max Obrist,\\
Nicola Ochsenbein,
Kirusanth Poopalasingam,\\
Nicol\'as Rom\'an L"uthold,
Stefan Schindler,\\
Christoph Schmitz-Dr"ager,
Arwed Schudel,
Michael Schwenter,
Tobias Stauber,
Stefan Steiner,
Claudio Stucki,
Pascal Stump,\\
Martin Stypinski
\end{flushright}
\vspace*{\stretch{2}}
\begin{center}
Hochschule f"ur Technik, Rapperswil, 2015
\end{center}
\end{titlepage}
\hypersetup{
    colorlinks=true,
    linktoc=all,
    linkcolor=blue
}
\newcounter{beispiel}
\newenvironment{beispiele}{
\bgroup\smallskip\parindent0pt\bf Beispiele\egroup

\begin{list}{\arabic{beispiel}.}
  {\usecounter{beispiel}
  \setlength{\labelsep}{5mm}
  \setlength{\rightmargin}{0pt}
}}{\end{list}}
\newcounter{uebungsaufgabe}
% environment fuer uebungsaufgaben
\newenvironment{uebungsaufgaben}{
\begin{list}{\arabic{uebungsaufgabe}.}
  {\usecounter{uebungsaufgabe}
  \setlength{\labelwidth}{2cm}
  \setlength{\leftmargin}{0pt}
  \setlength{\labelsep}{5mm}
  \setlength{\rightmargin}{0pt}
  \setlength{\itemindent}{0pt}
}}{\end{list}\vfill\pagebreak}
\newenvironment{teilaufgaben}{
\begin{enumerate}
\renewcommand{\labelenumi}{\alph{enumi})}
}{\end{enumerate}}
% Loesung
\def\swallow#1{
%nothing
}
\newenvironment{loesung}{%
\begin{proof}[L"osung]%
\renewcommand{\qedsymbol}{$\bigcirc$}
}{\end{proof}}
\newenvironment{diskussion}{}{}
\def\keineloesungen{%
\renewenvironment{loesung}{\swallow\begingroup}{\endgroup}%
\renewenvironment{diskussion}{\swallow\begingroup}{\endgroup}%
}
\newenvironment{beispiel}{%
\begin{proof}[Beispiel]%
\renewcommand{\qedsymbol}{$\bigcirc$}
}{\end{proof}}

\input linsys.tex
\allowdisplaybreaks

\lhead{Inhaltsverzeichnis}
\rhead{}
\tableofcontents
\newtheorem{satz}{Satz}[chapter]
\newtheorem{hilfssatz}{Hilfssatz}[chapter]
\newtheorem{definition}{Definition}[chapter]
\newtheorem{annahme}{Annahme}[chapter]
\mainmatter
\input vorwort.tex
\part{Grundlagen}
\begin{refsection}
\input einleitung.tex
\input einfach.tex
\input hilbert.tex
\input quantencomputer.tex
\input hamilton.tex
\input quantisierung.tex
\input heisenberg.tex
\input harmonisch.tex
\input h.tex
\input stoerungstheorie.tex
\input magnetfeld.tex
\input drehimpuls.tex
\input spin.tex
\input festkoerper.tex
\input komplex.tex
\input kugelkoordinaten.tex
\vfill
\pagebreak
\lhead{}
\rhead{}
\printbibliography[heading=subbibliography]
\end{refsection}

\part{Anwendungen und Weiterf"uhrende Themen}
\lhead{Anwendungen}
\input uebersicht.tex
\def\chapterauthor#1{{\large #1}\bigskip\bigskip}
%\input sample/sample.tex
\input tunneldiode/main.tex
\input atomuhr/main.tex
\input rtm/main.tex
\input efeld/main.tex
\input kugel/main.tex
\input orbitale/main.tex
\input flash/main.tex
\input franckhertz/main.tex
\input anharmonisch/main.tex
\input laser/main.tex
\input mri/main.tex
\input crypto/main.tex
\input teleport/main.tex
\input heisenberg/main.tex
\input stark/main.tex
\input bell/main.tex
\vfill
\pagebreak
\ifodd\value{page}\else\null\clearpage\fi
\lhead{Index}
\rhead{}
\input skript.ind

\end{document}


\end{document}
