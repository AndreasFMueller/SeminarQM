\chapter{Spin}
\lhead{Spin}
\rhead{}
Ein Elektron verh"alt sich nicht wie ein punktf"ormiges Teilchen.
Die L"osung der Schr"odingergleichung, und vor allem die "Ubereinstimmung
der L"osung mit Messungen zeigt, dass das Elektron sich beliebig nahe 
am Atomkern aufhalten kann, es gibt also keinen messbaren Radius.
Insbesondere ist es also auch nicht m"oglich, dass das Elektron
rotieren k"onnte, denn dazu br"auchte es eine positive Ausdehnung.

In dem in Kapitel~\ref{chapter:quantisierung} entwickelte Formalismus
enth"alt keinen Platz f"ur eine Wechselwirkung der Elektronen
mit einem Magnetfeld.
Experimente zeigen jedoch, dass ein Elektron auch mit einem Magnetfeld
wechselwirken kann, als ob es ein magnetisches Moment h"atte. 

\section{Spin}
Die Experimente zeigen, dass Elektronen sich verhalten, wie wenn sie
trotz allem rotieren k"onnten.
Es ist zwar nicht m"oglich, mit Hilfe der Drehimpulsoperatoren
aus Kapitel~\ref{chapter:drehimpuls} dem Elektron einen Eigendrehimpuls
zuzuteilen.
Trotzdem kann man feststellen, dass im Elektron zus"atzlicher
Drehimpuls stecken muss, denn gew"ohnlicher Drehimpuls kann verschwinden,
da der Drehimpuls erhalten ist, muss er in das Elektron "ubergegangen
sein.

Wir schliessen daraus, dass es einen Operator $\vec S$ gibt, der sich
genau so verh"alt wie ein Drehimpulsoperator, allerdings wirkt er nicht
auf die Ortskoordinaten.
Der Operator $\vec S$ heisst der Spin eines Elektrons.
Experimentell weiss man auch, dass Elektronen nur zwei verschiedene
Spinzust"ande haben k"onnen, was bedeutet, dass sich $\vec S$ wie ein
Drehimpulsoperator auf Zustandsvektoren mit $j=\frac12$ verh"alt.
Der Spin-Operator muss auf einen zweidimensionalen Hilbertraum
wirken, die Komponenten des Spin-Operators m"ussen also hermitesche
$2\times 2$-Matrizen sein, die die Vertauschungsrelationen eines
Drehimpulsoperators erf"ullen.

Die Matrizen $I$, $J$ und $K$ in (\ref{komplex:definitionIJK}) haben
algebraische Eigenschaften, die nahe an dem sind, was wir suchen,
allerdings sind sie nicht hermitesch.
Multiplizieren wir sie mit $i$, erhalten wir die hermiteschen Matrizen
\begin{align}
\sigma_1
&=
\begin{pmatrix}
0&1\\1&0
\end{pmatrix}
&
\sigma_2
&=
\begin{pmatrix}
0&-i\\i&0
\end{pmatrix}
&
\sigma_3
&=
\begin{pmatrix}
1&0\\0&-1
\end{pmatrix}
\label{spin:paulimatrizen}
\end{align}
die auch {\em Pauli-Matrizen} genannt werden.
Diese Matrizen haben die folgenden Multiplikationseigenschaften:
\begin{align*}
%\sigma_1^2
%=
%\sigma_2^2
%=
%\sigma_3^2
%&=E,
%&
\sigma_1\sigma_2
&=
\begin{pmatrix} i&0\\0&-i \end{pmatrix} = i\sigma_3,
&
\sigma_2\sigma_3
&=
\begin{pmatrix} 0&i\\i&0 \end{pmatrix} = i\sigma_1,
&
\sigma_3\sigma_1
&=
\begin{pmatrix} 0&1\\-1&0 \end{pmatrix} = i\sigma_2,
\\
%&&
\sigma_2\sigma_1
&=
\begin{pmatrix} -i&0\\0&i \end{pmatrix} = -i\sigma_3,
&
\sigma_3\sigma_2
&=
\begin{pmatrix} 0&-i\\-i&0 \end{pmatrix} = -i\sigma_1,
&
\sigma_1\sigma_3
&=
\begin{pmatrix} 0&-1\\1&0 \end{pmatrix} = -i\sigma_2.
\end{align*}
Die Vertauschungsrelationen sind
\begin{align*}
[\sigma_1,\sigma_2]
&=
2i\sigma_3,
&
[\sigma_2,\sigma_3]
&=
2i\sigma_1,
&
[\sigma_3,\sigma_1]
&=
2i\sigma_2.
\end{align*}
Dies sind bis auf einen Faktor $\hbar/2$ die Vertauschungsrelationen,
die wir von einem Drehimpulsoperator erwarten,
daher k"onnen wir den Vektoroperator
\begin{equation}
\vec S=\frac{\hbar}2\vec\sigma
=
\frac{\hbar}2\begin{pmatrix}\sigma_1\\\sigma_2\\\sigma_3\end{pmatrix}
\label{spin:vektoroperator}
\end{equation}
als Spin-Operator verwenden.
Der Betrag des Spin-Operators ist
\begin{align*}
\vec S^2
=
\frac{\hbar}{4}(\sigma_1^2+\sigma_2^2+\sigma_3^2)=\hbar^2\frac12\frac32E.
\end{align*}
Die Standardbasisvektoren sind Eigenvektoren sowohl von $\vec S^3$ wie
auch von $S_3$.
Wir bezeichnen sie daher auch $|\uparrow\rangle=e_1$ und
$|\downarrow\rangle=e_2$, es gilt:
\begin{align*}
\vec S^2|\uparrow\rangle&=\frac{3\hbar}4|\uparrow\rangle
&
\vec S^2|\downarrow\rangle&=\frac{3\hbar}4|\downarrow\rangle
\\
S_3|\uparrow\rangle&=\frac{\hbar}{2}|\uparrow\rangle
&
S_3|\downarrow\rangle&=-\frac{\hbar}{2}|\downarrow\rangle
\end{align*}

Die Auf- und Absteigeoperatoren f"ur den Spin sind
\begin{align}
S_+
&=
S_1+iS_2
=
\frac{\hbar}2
(\sigma_1+i\sigma_2)
=
\frac{\hbar}2
\begin{pmatrix} 0&1+1\\1-1&0 \end{pmatrix}
=\hbar\begin{pmatrix}0&1\\0&0\end{pmatrix},
\\
S_-
&=
S_1-iS_2
=
\hbar\begin{pmatrix} 0&0\\1&0\end{pmatrix},
\\
S_+S_-+S_-S_+
&=
\hbar^2E,
\end{align}
genau wie f"ur den Operator $\vec L$.
Auf den Standardbasisvektoren wirken diese Operatoren wie
\begin{align*}
S_+|\uparrow\rangle&=0,
&
S_-|\uparrow\rangle&=|\downarrow\rangle,
\\
S_+|\downarrow\rangle&=|\uparrow\rangle,
&
S_-|\downarrow\rangle&=0.
\end{align*}
$S_+$ erh"oht also den Spin in $z$-Richtung, $S_-$ erniedrigt ihn.

\section{Spin im Magnetfeld}

\section{Spin und Statistik}
Die Theorie sagt aus, dass ein Teilchen nur einen totalen Spin haben kann,
der ein Vielfaches von $\frac12$ ist. Tats"achlich gibt es Teilchen
ganz ohne Spin, die $\pi$-Mesonen geh"oren dazu, aber auch das Higgs-Boson.
Elektronen, Protonen und Neutronen haben Spin $\frac12$. 
Atomkerne setzen sich aus vielen Protonen und Neutronen zusammen, ihr
Spin kann daher viel h"oher sein. Im Grundzustand haben jedoch auch
sie einen kleinen Spin.

\subsection{Austauschoperator}
Ein einzelnes Teilchen hat in der Ortsdarstellung als Zustand
eine Wellenfunktion, die von den Ortskoordinaten und der Zeit abh"angt,
sie hat zwei Komponenten, eine f"ur jeden m"oglichen Spinzustand.
Wir k"onnten schreiben
\begin{equation}
|\psi\rangle
=
\begin{pmatrix}
\psi_\uparrow(x,y,z,t)\\
\psi_\downarrow(x,y,z,t)
\end{pmatrix}
\label{spin:vector}
\end{equation}
Statt die beiden m"oglichen Spinzust"ande durch einen Index zu unterscheiden,
k"onnten wir sie auch als Argumente der Wellenfunktion schreiben, also
als $\psi(x,y,z,t,s)$, wobei $s$ die Spinvariable ist. Der Zusammenhang
mit den Komponenten in (\ref{spin:vector}) ist
\[
\psi_\uparrow(x,y,z,t)
=
\psi(x,y,z,t,\uparrow)
\qquad\text{und}\qquad
\psi_\downarrow(x,y,z,t)
=
\psi(x,y,z,t,\downarrow).
\]
Wie sieht die Wellenfunktion eines Systems mit zwei Teilchen aus?
In die Wellenfunktion dieses Systems m"ussen die Koordinaten
$(x_1,y_1,z_1)$ und $(x_2,y_2,z_2)$ beider Teilchen eingehen, 
sowie die beiden Spins. Wir k"onnen die beiden Spins wieder als
Koordinaten einer Wellenfunktion
\[
|\psi_2\rangle
=
\psi_2(x_1,y_1,z_1,s_1,x_2,y_2,z_2,s_2,t)
\]
betrachten.
Elementarteilchen sind nicht unterscheidbar. Das bedeutet, dass sich
physikalisch nichts "andern darf, wenn wir die Koordinaten der beiden
Teilchen in $\psi_2$ vertauschen. Alle daraus berechneten physikalischen
Gr"ossen m"ussen gleich bleiben.

Wir k"onnen die Operation, die die Koordinaten vertauscht, also
Operator $A$ betrachten:
\[
|\psi_2\rangle
=
\psi_2(x_1,y_1,z_1,s_1,x_2,y_2,z_2,s_2,t)
\mapsto
A|\psi_2\rangle
=
\psi_2(x_2,y_2,z_2,s_2,x_1,y_1,z_1,s_1,t)
\]
$A$ ist ein selbstadjungierter Operator.
Die Energie darf sich nicht "andern, wenn die beiden Teilchen vertauscht
werden, also muss der Hamilton-Operator $H$ mit $A$ vertauschen:
$[H,A]=0$. Wir k"onnen die Eigenzust"ande von $H$ also immer so
w"ahlen, dass sie auch Eigenzust"ande von $A$ sind.
Wenn $|\psi\rangle$ ein Eigenzustand von $H$ ist, dann k"onnen wir
daraus immer neue Zustandsvektoren
\begin{align*}
|\psi\rangle + A|\psi\rangle
\end{align*}

Wendet man den Operator zweimal an, bleibt der Zustand unver"andert.
Der Operator $A$ erf"ullt also die Gleichung $A^2=\operatorname{id}$,
insbesondere kann $A$ nur Eigenwerte $\pm 1$ haben.
Ein Eigenzustand von $A$ erf"ullt also immer eine der beiden 
Gleichungen
\begin{align*}
A|\psi_2\rangle &= |\psi_2\rangle
\tag{symmetrisch}
\\
A|\psi_2\rangle &=-|\psi_2\rangle
\tag{antisymmetrisch}
\end{align*}

\subsection{Pauli-Prinzip}

\subsection{Periodensystem}



