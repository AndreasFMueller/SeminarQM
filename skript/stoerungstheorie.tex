\chapter{St"orungstheorie}
\lhead{St"orungstheorie}
\rhead{}
Die St"orungstheorie geht davon aus, dass die Zust"ande eines
quantenmechanischen Systems bereits bekannt sind. Eine kleine
"Anderung des Systems sollte sich dann in nur kleinen "Anderungen
der Zust"ande und Ihrer Energieniveaus "aussern.

\section{Problemstellung}
Es sei $H_0$ der Hamilton-Operator eines quantenmechanischen Systems
mit den Eigenvektoren $\psi_0,\psi_1,\dots$ und Energien $E_0,E_1,\dots$.
Es gilt also 
\[
H_0\psi_i = E_i\psi_i.
\]
Der Hamilton-Operator $H_0$ wird nun durch einen zus"atzlichen Term
$H_1$ ver"andert. Wir gehen davon aus, dass $H_1$ im Vergleich zu
$H_0$ nur schwache Effekte beschreibt. Der neue Operator ist
\[
H(\varepsilon)=H_0+\varepsilon H_1.
\]
Darin ist $\varepsilon\ll 1$ ein kleiner Parameter, der ausdr"ucken
soll, dass $H_1$ den Operator nur ganz wenig "andert, insbesondere
sind die Zust"ande und Eigenwerte von $E(\varepsilon)$ nahe bei 
den $E_i$ und $\psi_i$.

Die Aufgabe der St"orungstheorie ist, die Zust"ande $\psi_i(\varepsilon)$
und Energien $E_i(\varepsilon)$ mit
\[
H(\varepsilon)\psi_i(\varepsilon)=E_i(\varepsilon)\psi_i(\varepsilon)
\]
mindestens n"aherungsweise zu bestimmen.

Die Eigenvektoren bilden eine Basis des Hilbertraumes $L^2(\mathbb R)$, 
also m"ussen sich auch die modifizierten Eigenvektoren in dieser Basis
ausdr"ucken lassen:
\[
\psi_i(\varepsilon)=\psi_i +\sum_{j=0}^\infty a_{ij}(\varepsilon)\psi_j,
\]
wobei die Koeffizienten $a_{ij}(\varepsilon)$ klein sind.
Das Problem ist gel"ost, wenn die $a_{ij}(\varepsilon)$ bestimmt sind.

\begin{beispiel}
Als Beispiel verwenden wir ein Teilchen in einem Potentialtopf, welches
wir fr"uher bereits untersucht haben. Der Hamilton-Operator wird jetzt
gest"ort um einen zus"atzlichen Potentialterm $H_1=V_1$ im Inneren
des Potentialtopfes.
W"are der zus"atzliche Term eine Konstante w"urde sich an den Eigenfunktionen
gar nichts "andern, und f"ur Eigenwerte w"urde man finden
\[
H(\varepsilon)\psi_i=(E_i + \varepsilon V_1)\psi_i,
\]
es w"are also
\begin{align*}
\psi_i(\varepsilon)&=\psi_i\\
E_i(\varepsilon)&=E_i+\varepsilon V_i
\end{align*}
Nun ist aber $V_i$ ortsabh"angig, nur im Inneren des Potentialtopfes 
unterscheidet sich $H(\varepsilon)$ von $H_0$. Da sich das Teilchen
aber praktisch nie ausserhalb des Potentialtopfes befindet, sollte
sich in erster N"aherung Eigenfunktion und Eigenwerte genau gleich
"andern.
\end{beispiel}



\section{Erste N"aherung}


\section{Nicht entartete Zust"ande}

\section{Entartung}

\section{St"orungsreihe}
