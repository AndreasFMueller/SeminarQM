\chapter{St"orungstheorie}
\lhead{St"orungstheorie}
\rhead{}
Die St"orungstheorie geht davon aus, dass die Zust"ande eines
quantenmechanischen Systems bereits bekannt sind. Eine kleine
"Anderung des Systems sollte sich dann in nur kleinen "Anderungen
der Zust"ande und Ihrer Energieniveaus "aussern.
Diese Methode sollte anwendbar sein um die Ver"anderung der Energieniveaus
eines Atoms zu berechnen, wenn dieses in ein "ausseres elektrisches Feld
oder ein "ausseres Magnetfeld gebracht wird.


\section{Grundprinzip der St"orungstheorie}
\index{stoerungstheorie@St\"orungstheorie!Prinzip}
Wir beginnen die Untersuchung mit einer Anwendung auf die Nullstellen
eines quadratischen Polynoms. Sei also
\[
p(x) = ax^2 + bx + c
\]
ein quadratisches Polynom, nat"urlich k"onnen wir sofort die Nullstellen
mit Hilfe der L"osungsformel
\begin{equation}
x_{\pm}=\frac{-b\pm\sqrt{b^2-4ac}}{2a}
\label{mitternachtsformel}
\end{equation}
finden.  Ver"andert man die Koeffizienten von $p(x)$, werden sich auch
die beiden Nullstellen ver"andern, sie k"onnen mit der Formel
(\ref{mitternachtsformel}) bestimmt werden.

Nehmen wir also an, dass die Koeffizienten $a$, $b$ und $c$ von $p(x)$
von einem Parameter $\varepsilon$ abh"angen. Statt der Konstanten verwenden
wir also Funktionen
\begin{align*}
a(\varepsilon)&=a_0+a_1\varepsilon+a_2\varepsilon^2+\dots\\
b(\varepsilon)&=b_0+b_1\varepsilon+b_2\varepsilon^2+\dots\\
c(\varepsilon)&=c_0+c_1\varepsilon+c_2\varepsilon^2+\dots
\end{align*}
von $\varepsilon$. Mit diesen Koeffizienten wird aus dem Polynom $p(x)$
eine Familie von Polynomen
\[
p_\varepsilon(x)=a(\varepsilon)x^2 + b(\varepsilon)x+c(\varepsilon).
\]
Die L"osungsformel (\ref{mitternachtsformel}) liefert weiterhin die
Nullstellen von $p_{\varepsilon}(x)=0$.
Sei $x_0$ eine der Nullstellen von $p_0(x) = 0$.

Statt die L"osungen von $p_{\varepsilon}(x)=0$ mit Hilfe der L"osungsformel
zu finden, k"onnen wir auch versuchen, von $x_0$ ausgehend eine N"aherung
zu finden. Diese L"osung wird von $\varepsilon$ abh"angen, wir setzen
die Abh"angigkeit als Potenzreihe
\begin{equation}
x_\varepsilon = x_0 + x_1\varepsilon +x_2 \varepsilon^2+\dots
\label{stoerungsansatz}
\end{equation}
an.
Wir nennen die Potenzreihe (\ref{stoerungsansatz}) f"ur die L"osung
von $p_\varepsilon(x)=0$ die St"orungsreihe f"ur $x_\varepsilon$.
Setzen wir (\ref{stoerungsansatz}) in die Gleichung $p_{\varepsilon}(x)=0$
ein, ergibt sich die Gleichung
\begin{align*}
0&=a(\varepsilon)x_\varepsilon^2+b(\varepsilon)x_\varepsilon+c(\varepsilon)
\\
&=
(a_0+a_1\varepsilon+a_2\varepsilon^2+\dots)
(x_0+x_1\varepsilon+x_2\varepsilon^2+\dots)^2\\
&\qquad
+
(b_0+b_1\varepsilon+b_2\varepsilon^2+\dots)
(x_0+x_1\varepsilon+x_2\varepsilon^2+\dots)\\
&\qquad
+
(c_0+c_1\varepsilon+c_2\varepsilon^2+\dots)
\\
&=
(a_0+a_1\varepsilon+a_2\varepsilon^2+\dots)
(x_0^2+2x_0x_1\varepsilon +x_1^2\varepsilon^2+2x_0x_2\varepsilon^2+\dots)
\\
&\qquad
+
b_0x_0 + (b_0x_1+b_1x_0)\varepsilon+(b_0x_2+b_1x_1+b_2x_0)\varepsilon+\dots\\
&\qquad
+
c_0+c_1\varepsilon+c_2\varepsilon^2+\dots
\\
&=
a_0x_0^2 + (2a_0x_0x_1 + a_1x_0^2)\varepsilon +
(a_2x_0^2 + 2a_1x_0x_1 + a_0x_1^2 +2a_0x_0x_2)\varepsilon^2+\dots
\\
&\qquad
+
b_0x_0 + (b_0x_1+b_1x_0)\varepsilon+(b_0x_2+b_1x_1+b_2x_0)\varepsilon^2+\dots\\
&\qquad
+
c_0+c_1\varepsilon+c_2\varepsilon^2+\dots
\\
&=
a_0x_0^2 + b_0x_0+c_0
+(2a_0x_0x_1+a_1x_0^2+b_0x_1+b_1x_0+c_1)\varepsilon
\\
&\qquad
+(
a_2x_0^2 + 2a_1x_0x_1 + a_0x_1^2 +2a_0x_0x_2
+b_0x_2+b_1x_1+b_2x_0
+c_2
)\varepsilon^2+\dots
\end{align*}
Der konstante Term in der letzten Form dieser Gleichung ist $p_0(x_0)$, 
da $x_0$ eine Nullstelle ist, muss er verschwinden. Wenn die Gleichung
f"ur alle $\varepsilon$ erf"ullt sein soll, dann muss auch der
Koeffizient von $\varepsilon$ verschwinden, es muss also gelten
\begin{align*}
2a_0x_0x_1+a_1x_0^2+b_0x_1+b_1x_0+c_1&= 0
\\
(2a_0x_0+b_0)x_1&=-(a_1x_0^2+b_1x_0+c_1)\\
x_1&=-
\frac{a_1x_0^2+b_1x_0+c_1}{2a_0x_0+b_0}.
\end{align*}
Damit ist eine erste Approximation von $x_\varepsilon$ gefunden:
\[
x_\varepsilon\simeq x_0 -
\frac{a_1x_0^2+b_1x_0+c_1}{2a_0x_0+b_0}\varepsilon.
\]
Man nennt dies die St"orungsapproximation erster Ordnung (in $\varepsilon$).

Man kann die Approximation noch verbessern, indem man auch noch den
Koeffizienten von $\varepsilon^2$ in $x_\varepsilon$ berechnet.
Dazu verwendet man, dass in der Gleichung auch der Koeffizient von
$\varepsilon^2$ verschwinden muss:
\begin{align*}
a_2x_0^2 + 2a_1x_0x_1 + a_0x_1^2 +2a_0x_0x_2
+b_0x_2+b_1x_1+b_2x_0
+c_2
&=0
\\
(2a_0x_0+b_0)x_2
+a_2x_0^2 + 2a_1x_0x_1 + a_0x_1^2
+b_1x_1+b_2x_0
+c_2&=0
\end{align*}
Daraus kann man schliessen, dass
\[
x_2=-\frac{
a_2x_0^2 + 2a_1x_0x_1 + a_0x_1^2
+b_1x_1+b_2x_0
+c_2
}{2a_0x_0+b_0}.
\]
Mit den Koeffizienten $x_1$ und $x_2$ erh"alt man die St"orungsapproximation
zweiter Ordnungn.

Die St"orungsapproximation scheint also zu funktionieren, solange
der Nenner $2a_0x_0+b_0$ gross ist. Dann werden n"amlich die 
Koeffizienten $x_1$ und $x_2$ klein sein. Problematisch wird
die Approximation in dem Falle, wo $2a_0x_0+b_0$ verschwindet.
Setzt man allerdings die L"osungsformel (\ref{mitternachtsformel}) f"ur
$x_0$ ein, erh"alt man
\begin{align*}
2a_0x_0+b_0&=2a_0\frac{-b_0\pm\sqrt{b_0^2-4a_0c_0}}{2a_0}+b_0      \\
           &=\pm\sqrt{b_0^2-4a_0c_0}.
\end{align*}
Der Radikand ist die Diskriminante.
Der Nenner verschwindet also genau dann, wenn die quadratische
Gleichung eine doppelte Nullstelle hat. In diesem Fall spaltet sich
die eine doppelte L"osung in zwei verschiedene L"osungen $x_{\pm}$ auf,
es kann also keine Funktion der Form
(\ref{stoerungsansatz}) 
geben, die diese Aufspaltung wiedergeben kann.

\begin{figure}
\centering
\includegraphics[width=0.55\hsize]{graphics/pert-1.pdf}
\caption{St"orungsl"osung f"ur die Abh"angigkeit der L"osung einer
quadratischen Gleichung $a(\varepsilon)x^2+b(\varepsilon)x+c(\varepsilon)=0$
in Abh"angigkeit von $\varepsilon$,
mit $a(\varepsilon)=1+\varepsilon$, $b(\varepsilon)=\varepsilon$ und
$c(\varepsilon)=-1+\varepsilon$. Exakte L"osung {\color{red} rot}, 
St"orungsapproximation erster Ordnung {\color{blue} blau}.
\label{stoerungsloesung}}
\end{figure}

Wir fassen zusammen, was wir hier als L"osungsmethode gefunden haben:
\begin{enumerate}
\item Gegeben ist eine Gleichung $F_\varepsilon(x)=0$, gesucht
sind die L"osungen $x_\varepsilon$ der Gleichung in Abh"angigkeit von
$\varepsilon$.
\item Die L"osung $x_0$ der Gleichung f"ur den Parameterwert $\varepsilon=0$
ist bekannt.
\item Setze die L"osung in Form einer Potenzreihe an:
$x_\varepsilon = x_0+x_1\varepsilon+x_2\varepsilon^2+\dots$
\item Setze den L"osungsansatz in die Gleichung $F_\varepsilon(x)=0$ ein
und ermittle die Koeffizienten $x_1,x_2,\dots$ durch Koeffizientenvergleich.
\end{enumerate}
Schritt~4 ist eventuell nicht direkt durchf"uhrbar, wenn die Funktion
$F_\varepsilon(x)$ zu kompliziert ist. Man kann aber immer versuchen, die
Funktion als Potenzreihe in $x$ auszudr"ucken, 
\[
F_\varepsilon(x)=F_\varepsilon(x_0) + DF_\varepsilon(x_0)\cdot x
+ \frac12 D^2F_\varepsilon(x_0)\cdot(x,x)+\dots,
\]
und dann die Koeffizienten dieser Potenzreihe als Potenzreihe in $\varepsilon$.
So entsteht auf jeden Fall eine Vektorgleichung, deren Komponenten nur
Polynome in $\varepsilon$ sind, f"ur die sich Schritt~4 durchf"uhren l"asst.

\section{St"orungstheorie in der Quantenmechanik}
Es sei $H_0$ der Hamilton-Operator eines quantenmechanischen Systems
mit den Eigenvektoren $\psi_0,\psi_1,\dots$ und Energien $E_0,E_1,\dots$.
Es gilt also 
\[
H_0\psi_i = E_i\psi_i.
\]
Der Hamilton-Operator $H_0$ wird nun durch einen zus"atzlichen Term
$H_1$ ver"andert. Wir gehen davon aus, dass $H_1$ im Vergleich zu
$H_0$ nur schwache Effekte beschreibt. Der neue Operator ist
\[
H(\varepsilon)=H_0+\varepsilon H_1.
\]
Darin ist $\varepsilon\ll 1$ ein kleiner Parameter, der ausdr"ucken
soll, dass $H_1$ den Operator nur ganz wenig "andert, insbesondere
sind die Zust"ande und Eigenwerte von $E(\varepsilon)$ nahe bei 
den $E_i$ und $\psi_i$.

Die Aufgabe der St"orungstheorie ist, die Zust"ande $\psi_i(\varepsilon)$
und Energien $E_i(\varepsilon)$ mit
\[
H(\varepsilon)\psi_i(\varepsilon)=E_i(\varepsilon)\psi_i(\varepsilon)
\]
mindestens n"aherungsweise zu bestimmen.

Die Eigenvektoren bilden eine Basis des Hilbertraumes $L^2(\mathbb R)$, 
also m"ussen sich auch die modifizierten Eigenvektoren in dieser Basis
ausdr"ucken lassen:
\[
\psi_i(\varepsilon)=\psi_i +\sum_{j=0}^\infty a_{ij}(\varepsilon)\psi_j,
\]
wobei die Koeffizienten $a_{ij}(\varepsilon)$ klein sind.
Das Problem ist gel"ost, wenn die $a_{ij}(\varepsilon)$ bestimmt sind.

\begin{beispiel}
Als Beispiel verwenden wir ein Teilchen in einem Potentialtopf, welches
wir fr"uher bereits untersucht haben. Der Hamilton-Operator wird jetzt
gest"ort um einen zus"atzlichen Potentialterm $H_1=V_1$ im Inneren
des Potentialtopfes.
W"are der zus"atzliche Term eine Konstante w"urde sich an den Eigenfunktionen
gar nichts "andern, und f"ur Eigenwerte w"urde man finden
\[
H(\varepsilon)\psi_i=(E_i + \varepsilon V_1)\psi_i,
\]
es w"are also
\begin{align*}
\psi_i(\varepsilon)&=\psi_i\\
E_i(\varepsilon)&=E_i+\varepsilon V_i
\end{align*}
Nun ist aber $V_i$ ortsabh"angig, nur im Inneren des Potentialtopfes 
unterscheidet sich $H(\varepsilon)$ von $H_0$. Da sich das Teilchen
aber praktisch nie ausserhalb des Potentialtopfes befindet, sollte
sich in erster N"aherung Eigenfunktion und Eigenwerte genau gleich
"andern.
\end{beispiel}



\section{Erste N"aherung}


\section{Nicht entartete Zust"ande}

\section{Entartung}

\section{St"orungsreihe}
