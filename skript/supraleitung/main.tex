\chapter{Cooper-Paare und Supraleitung\label{chapter:supraleitung}}
\lhead{Cooper-Paare und Supraleitung}
\begin{refsection}
\chapterauthor{Simon Kuster und Nicola Ochsenbein}

\newcommand{\marrow}[5]{%
    \fmfcmd{style_def marrow#1
    expr p = drawarrow subpath (1/4, 3/4) of p shifted 6 #2 withpen pencircle scaled 0.4;
    label.#3(btex #4 etex, point 0.5 of p shifted 6 #2);
    enddef;}
    \fmf{marrow#1,tension=0}{#5}}
\begin{center}
\begin{fmffile}{supraleitung/phonon}
\begin{fmfgraph*}(100,60)
\fmfleftn{i}{2}
\fmfrightn{o}{2}
\fmflabel{$\vec k'-\vec q$}{i1}
\fmflabel{$\vec k'$}{i2}
\fmflabel{$\vec k+\vec q$}{o1}
\fmflabel{$\vec k$}{o2}
\fmf{fermion}{i2,v1,i1}
\fmf{fermion}{o2,v2,o1}
\fmf{zigzag,label=$\vec q$}{v1,v2}
\fmfdot{v1,v2}
\marrow{c}{up}{top}{}{v1,v2}
\end{fmfgraph*}
\qquad
\qquad
\qquad
\begin{fmfgraph*}(100,60)
\fmfleftn{i}{2}
\fmfrightn{o}{2}
\fmflabel{$\vec k'+\vec q$}{i1}
\fmflabel{$\vec k'$}{i2}
\fmflabel{$\vec k-\vec q$}{o1}
\fmflabel{$\vec k$}{o2}
\fmf{fermion}{i2,v1,i1}
\fmf{fermion}{o2,v2,o1}
\fmf{zigzag,label=$-\vec q$}{v1,v2}
\fmfdot{v1,v2}
\marrow{c}{up}{top}{}{v2,v1}
\end{fmfgraph*}
\end{fmffile}
\end{center}


Entstehung von Cooper-Paaren und Supraleitung nach Feynman
\cite{supraleitung:feynman}.


\newpage
\section{Einleitung}
\rhead{Einleitung}
TODO

\section{Elektron-Elektron Wechselwirkung}
\rhead{Elektron-Elektron Wechselwirkung}
TODO


\section{Hinzuf"ugen von Elektrone zu einer Fermikugel}
\rhead{Hinzuf"ugen von Elektrone zu einer Fermikugel}
Nachdem im vorherigen Kapitel die Wechselwirkung zwischen Elektronen erkl"art wurde, befassen wir uns in diesem Kapitel mit dem Hinzuf"ugen von Elektronen zu einer Fermikugel. Die Fermikugel wurde in den Grundlagen Kapitel 14 behandelt.
\\
\\
In welcher Form k"onnen zwei Elektronen am energetisch g"unstigsten zu einer Fermikugel hinzugef"ugt werden? Mit dieser Frage wollen wir uns nun besch"aftigen.
\\
\\
Die Wechselwirkung zwischen den Elektronen besteht einerseits aus der altbekannten Coulombsche Wechselwirkung und der neu eingef"uhrten Elektronen-Elektronen Wechselwirkung. Welcher dieser Kr"afte "uberwiegt, entscheidet die St"arke der Elektron-Phononen-Kopplung.
\\
\\
Um diese neue Wechselwirkung zu betrachten gehen wir von einem idealisierten Falls aus. Wir betrachten eine Fermikugel in der k-Ebene, in welcher alle inneren Zust"ande besetzt sind, alle "ausseren Zust"ande seien freu. Zu diesem System werden nun zwei Elektronen $(k_1,E(k_1))$ und $(k_2,E(k_2))$ hinzugef"ugt. Wir beschr"anken uns auf die Wechselwirkungen in welchen keine Energie an das Atomgitter abgegeben wird. Die zul"assige Wechselwirkung zwischen den Elektronen beschr"ankt sich damit auf die Energie eines Phonons.

\[
|E(k+q)-E(k)|\le\hbar\omega_q
\]

Wellenfunktion
Die Wellenfunktion des Elektronenpaares wir durch die Anwendung zweier Erzeugnisoperatoren auf den Grundzustand $|G\rangle$ der gef"ullten Fermikugel aufgebaut, indem wir "uber alle m"oglichen $k_1$ und $k_2$ $(k_i \le k_f)$ und "uber die Elektronenspins summieren.

\[
\Psi_{12}=\sum \limits_{k_1k_2\sigma_1\sigma_2} a_{\sigma_1\sigma_2}(k_1,k_2)c^+_{k_1\sigma_1}c^+_{k_2\sigma_2}|G\rangle
\]

Um einen Zustand mit definiertem Gesamtimpuls zu erhalten, f"uhren wir die Summation unter der Nebenbedingung $K=k_1+k_2=const$ aus.
Das n"achste Ziel wird nun sein diese Wechselwirkungsenergie zu berechnen. Sie ist umso gr"osser, je mehr Summationsglieder zu obigen Wellenfunktion beitragen. W"ahlt man $K=0$ erzielt man die gr"osste Wechselwirkungsenergie. Um diesen "Uberlegungsschritt nachzuvollziehen betrachten wir die Abbildung ...TODO.
 
  
Eine Wechselwirkung zwischen den Elektronen soll nur dann bestehen, wenn sich ihre Zust"ande ausserhalb der Fermikugel mit den Energie $(E_F \le E(k_i) \le E_F+\hbar\omega_p)$ befinden. Da $K=k_1+k_2$ gilt ergeben sich bei $K>0$ im k-Raum folgende grau eingef"arbte Bereich. Die Fl"ache kann maximiert werden indem $K=0$ gesetzt wird. F"ur die folgenden Berechnungen werden wir uns auf diesen Fall beschr"anken.
Zus"atzlich nehmen wir an, dass die Elektronenspins antiparallel sind. Hierzu kann die "Uberlegung mit dem entstehenden magnetischen Feld herbeigezogen werden. W"aren der Elektronenspin parallel w"urden sich die entstehenden magnetischen Felder abstossen. Wobei hingegen bei antiparallelem Elektronenspin die Wechselwirkungsenergie durch die anziehenden magnetischen Felder verst"arkt wird.
Somit ergibt sich die neue Wellenfunktion

\[
\Psi_{12}=\sum \limits_{k} a(k)c^+_{k}c^+_{-k}|G\rangle
\]

$k$ ordnen wir zugleich „Spin aufw"arts“ und $–k$ f"ur „Spin abw"arts“ zu.
Der g"unstigste Prozess ist also wenn die Elektronen entgegengesetzte Wellenzahlvektoren $(k_1 = -k_2)$ und entgegengesetzten Spin $(\sigma_1 = -\sigma_2)$ aufweisen.
\\
%\section{Elektronenverhalten in einem Festk"orper}
%\rhead{Elektronenverhalten in einem Festk"orper}
\\
\printbibliography[heading=subbibliography]
\end{refsection}

