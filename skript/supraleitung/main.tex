\chapter{Cooper-Paare und Supraleitung\label{chapter:supraleitung}}
\lhead{Cooper-Paare und Supraleitung}
\begin{refsection}
\chapterauthor{Simon Kuster und Nicola Ochsenbein}

\newcommand{\marrow}[5]{%
    \fmfcmd{style_def marrow#1
    expr p = drawarrow subpath (1/4, 3/4) of p shifted 6 #2 withpen pencircle scaled 0.4;
    label.#3(btex #4 etex, point 0.5 of p shifted 6 #2);
    enddef;}
    \fmf{marrow#1,tension=0}{#5}}
\begin{center}
\begin{fmffile}{supraleitung/phonon}
\begin{fmfgraph*}(100,60)
\fmfleftn{i}{2}
\fmfrightn{o}{2}
\fmflabel{$\vec k'-\vec q$}{i1}
\fmflabel{$\vec k'$}{i2}
\fmflabel{$\vec k+\vec q$}{o1}
\fmflabel{$\vec k$}{o2}
\fmf{fermion}{i2,v1,i1}
\fmf{fermion}{o2,v2,o1}
\fmf{zigzag,label=$\vec q$}{v1,v2}
\fmfdot{v1,v2}
\marrow{c}{up}{top}{}{v1,v2}
\end{fmfgraph*}
\qquad
\qquad
\qquad
\begin{fmfgraph*}(100,60)
\fmfleftn{i}{2}
\fmfrightn{o}{2}
\fmflabel{$\vec k'+\vec q$}{i1}
\fmflabel{$\vec k'$}{i2}
\fmflabel{$\vec k-\vec q$}{o1}
\fmflabel{$\vec k$}{o2}
\fmf{fermion}{i2,v1,i1}
\fmf{fermion}{o2,v2,o1}
\fmf{zigzag,label=$-\vec q$}{v1,v2}
\fmfdot{v1,v2}
\marrow{c}{up}{top}{}{v2,v1}
\end{fmfgraph*}
\end{fmffile}
\end{center}

Entstehung von Cooper-Paaren und Supraleitung nach Feynman
%\caption{Entstehung von Cooper-Paaren und Supraleitung nach Feynman
%\label{supraleitung:FeynmanDiagram1}}
\cite{supraleitung:feynman}.


\newpage
%----------------------------Einleitung------------------------
\section{Einleitung}
\rhead{Einleitung}
TODO

%----------------------------Elektron-Elektron Wechselwirkung------------------------
\section{Elektron-Elektron Wechselwirkung}
\rhead{Elektron-Elektron Wechselwirkung}
Es gibt neben der coulombschen Wechselwirkung zwischen den Elektronen noch mehr Wechselwirkungen. Zum einen gibt es eine Elektronen-Gitter Wechselwirkung. Diese ist f"ur die Betrachung der Cooper-Paare aber nicht relevant. Sie wird deshalb nachfolgend nicht ber"ucksichtigt. Zum anderen gibt es eine Elektron-Elektron Wechselwirkung "uber die sogenannten Phononen. Wie diese Wechselwirkung im Allgemeinen funktioniert und was diese Phononen sind wird nachfolgend erl"autert.
\\
In Abbildung \ref{supraleitung:Gitter1} sieht man einen Ausschnitt der Gitterstruktur eines K"orpers. Fliegt nun ein Elektron durch dieses Gitter, so wird das Gitter durch die Felder des Elektrons deformiert. Das Elektron ver"andert seine Flugbahn. Dargestellt ist dies in Abbildung \ref{supraleitung:Gitter2}. Das deformierte Gitter bewegt sich nun weiter als Schwingung. In Abbildung \ref{supraleitung:Gitter3} ist diese Schwingung dargestellt und mit $q$ bezeichnet. Wenn nun ein weiteres Elektron ins Gitter fliegt, so wird dieses von der Schwingung im Gitter beeinflusst, ersichtlich in Abbilung \ref{supraleitung:Gitter4}. Die Schwingung im Gitter wird wieder vernichtet und das Elektron ver"andert seine Flugbahn.
\\
Die Schwingng $q$ welche vom Elektron $k'$ zum Elektron $k$ transportiert wurde entspricht einem harmonischen Oszillator. Das Bedeuted, dass die m"oglichen Frequenzen, und die damit die enthaltene Energie, quantisiert sind. Wenn diese Energie quantisiert ist, kann man die Energie auch als Teilchen betrachten. Eine Darstellung der Wechselwirkung mit der Gitterschwinung als Teilchen vereinfacht die Darstellung stark. Diese neu eingef"uhrten Teilchen werden Phononen genannt.
Diese Betrachtung der Gitterschwingung als Phonon ist nach Feynman in Abbildung \ref{supraleitung:FeynmanDiagram1} ersichtlich.%TODO hier sollte auf das Feynmandiagramm verwiesen werden. (wie?)
%Kann das verhalten der Bilder so beeinflusst werden, dass sie wenigstens im gleichen unterkapitel angezeigt werden? (momentan ist die Darstellung der Bilder über so viele Seiten nicht geeignet.
\begin{figure}[h]	
\centering
\includegraphics[width=0.4\textwidth]{supraleitung/gitter-1.pdf} %Bild Ungestoertes Gitter
\caption{Ungest"ortes Gitter
\label{supraleitung:Gitter1}}
\end{figure}
\begin{figure}[h]
\centering
\includegraphics[width=0.4\textwidth]{supraleitung/gitter-2.pdf} %Bild Gitter mit Elektron1
\caption{Elektron fliegt ins Gitter und st"ort dieses
\label{supraleitung:Gitter2}}
\end{figure}
\begin{figure}[h]
\centering
\includegraphics[width=0.4\textwidth]{supraleitung/gitter-3.pdf} %Bild Gitter mit Stoerung
\caption{Gitter mit Schwingung q
\label{supraleitung:Gitter3}}
\end{figure}
\begin{figure}[h]
\centering
\includegraphics[width=0.4\textwidth]{supraleitung/gitter-4.pdf} %Bild Gitter mit Elektron2
\caption{Elektron fliegt ins Gitter und wird beeinflusst
\label{supraleitung:Gitter4}}
\end{figure}
Diese Wechselwirkung zwischen den Elektronen kann auch mathematisch beschrieben werden. F"ur die mathematische Betrachtung beginnen wir da wo wir einen Anhaltspunkt haben, also beim Phonon.
\\
Wir haben gesehen, dass das Phonon quantisiert ist. F"ur dieses gibt es also Auf- und Absteigeoperatoren. Wir nennen das Phonon $a$ mit der Energie $q$. Wir haben also den Aufsteigoperator $a^+_q$ und den Absteigeoperator $a_q$. Wenn die an das Phonon abgegebe Energie quantisiert ist, so muss das Elektron ebenfalls Auf- und Absteigeoperatoren haben. Wir nennen die Elektronen nun $c$ mit dem Wellenzahl $k$. Daraus folgt, dass der Aufsteigoperator f"ur das Elektron $c^+_k$ und der Absteigeoperator $c_k$ sind.
Wir f"uhren nun noch einen Anfangszustand $|a\rangle$, einen Endzustand $\langle e|$ und einen Zwischenzustand $|z_1\rangle$ ein. Der Vorgang kann nun so beschrieben werden (ohne Energiebetrachtung):
\[
\langle e|c^+_{k+q} c_k a_q |z_1\rangle\langle z_1| c^+_{k'-q} c_{k'} a^+_q |a\rangle
\]
Zu beachten ist nun, dass die Leserichtung von rechts nach links ist. Es wird also vom Anfangszustand aus ein Phonon $a$ mit der Energie $q$ erzeugt, ein Elektron $c$ mit der Wellenzahl $k'$ vernichtet und ein Elektron $c$ mit der Wellenzahl $k'+q$ erzeugt, um zum Zwischenzustand $z_1$ zu gelangen.
Von diesem Zwischenzustand aus geht es weiter, indem man das Phonon $a_q$ wieder vernichtet, das Elektron $c_k$ vernichtet und ein Elektron $c^+_{k+q}$ erzeugt.
%TODO nun muss die Formel 81.2 eingefuehrt werden -> Nenner noch nachvollziehbar beschreiben.
%danach richtung 81.4 weiterziehen
\\
Weitere Formeln:
\[
\frac{1}{2}
\sum \limits_{kk'q} |M_q|^2
\left\{
\frac
{\langle e|c^+_{k+q} c_k a_q |z_1\rangle\langle z_1| c^+_{k'-q} c_{k'} a^+_q |a\rangle }
{E(k')-E(k'-q)-\hbar\omega_q}
+
\frac
{\langle e|c^+_{k'-q} c_{k'} a_{-q}|z_1\rangle\langle z_1| c^+_{k+q} c_k a^+_{-q} |a\rangle }
{E(k')-E(k'-q)-\hbar\omega_q}
\right\}
\]
\[
\frac{1}{2}
\sum \limits_{kk'q} 
\langle e|V_{kk'q}c^+_{k+q}c^+_{k'-q}c_{k'}c_k|a \rangle
\]
\[
V_{kk'q} = - |M_q|^2 \left\{\frac{1}{\hbar\omega_q+(E(k+q)-E(k))}
+
\frac{1}{\hbar\omega_q-(E(k+q)-E(k))}
\right\}
= \frac
{2|M_q|^2\hbar\omega_q}
{(E(k+q)-E(k))^2-(\hbar\omega_q)^2)}
\]

//
TODO: Formeln noch einbauen und erklären

%----------------------------Hinzufuegen von Elektronen zu der Fermikugel------------------------
\section{Hinzuf"ugen von Elektronen zu der Fermikugel}
\rhead{Hinzuf"ugen von Elektronen zu der Fermikugel}
Nachdem im vorherigen Kapitel die Wechselwirkung zwischen Elektronen erkl"art wurde, befassen wir uns in diesem Kapitel mit dem Hinzuf"ugen von Elektronen zu der Fermikugel. Die Fermikugel wurde in den Grundlagen Kapitel 14 behandelt.
\\
\\
In welcher Form k"onnen zwei Elektronen am energetisch g"unstigsten zu der Fermikugel hinzugef"ugt werden? Mit dieser Frage wollen wir uns nun besch"aftigen.
\\
\\
Die Wechselwirkung zwischen den Elektronen besteht einerseits aus der altbekannten Coulombsche Wechselwirkung und der neu eingef"uhrten Elektronen-Elektronen Wechselwirkung. Welcher dieser Kr"afte "uberwiegt, entscheidet die St"arke der Elektron-Phononen-Kopplung.
\\
\\
Um diese neue Wechselwirkung zu betrachten gehen wir von einem idealisierten Falls aus. Wir betrachten die Fermikugel in der k-Ebene, in welcher alle inneren Zust"ande besetzt sind, alle "ausseren Zust"ande seien frei. Zu diesem System werden nun zwei Elektronen $(k_1,E(k_1))$ und $(k_2,E(k_2))$ hinzugef"ugt. Wir beschr"anken uns auf die Wechselwirkungen in welchen keine Energie an das Atomgitter abgegeben wird. Die zul"assige Wechselwirkung zwischen den Elektronen beschr"ankt sich damit auf die Energie eines Phonons.

\[
|E(k+q)-E(k)|\le\hbar\omega_q
\]

Wellenfunktion
Die Wellenfunktion des Elektronenpaares wir durch die Anwendung zweier Erzeugnisoperatoren auf den Grundzustand $|G\rangle$ der gef"ullten Fermikugel aufgebaut, indem wir "uber alle m"oglichen $k_1$ und $k_2$ $(k_i \le k_f)$ und "uber die Elektronenspins summieren.

\[
\Psi_{12}=\sum \limits_{k_1k_2\sigma_1\sigma_2} a_{\sigma_1\sigma_2}(k_1,k_2)c^+_{k_1\sigma_1}c^+_{k_2\sigma_2}|G\rangle
\]

Um einen Zustand mit definiertem Gesamtimpuls zu erhalten, f"uhren wir die Summation unter der Nebenbedingung $K=k_1+k_2=const$ aus.
Das n"achste Ziel wird nun sein diese Wechselwirkungsenergie zu berechnen. Sie ist umso gr"osser, je mehr Summationsglieder zu obigen Wellenfunktion beitragen. W"ahlt man $K=0$ erzielt man die gr"osste Wechselwirkungsenergie. Um diesen "Uberlegungsschritt nachzuvollziehen betrachten wir die Abbildung ...TODO.
 
  
Eine Wechselwirkung zwischen den Elektronen soll nur dann bestehen, wenn sich ihre Zust"ande ausserhalb der Fermikugel mit den Energie $(E_F \le E(k_i) \le E_F+\hbar\omega_p)$ befinden. Da $K=k_1+k_2$ gilt ergeben sich bei $K>0$ im k-Raum folgende grau eingef"arbte Bereich. Die Fl"ache kann maximiert werden indem $K=0$ gesetzt wird. F"ur die folgenden Berechnungen werden wir uns auf diesen Fall beschr"anken.
Zus"atzlich nehmen wir an, dass die Elektronenspins antiparallel sind. Hierzu kann die "Uberlegung mit dem entstehenden magnetischen Feld herbeigezogen werden. W"aren der Elektronenspin parallel w"urden sich die entstehenden magnetischen Felder abstossen. Wobei hingegen bei antiparallelem Elektronenspin die Wechselwirkungsenergie durch die anziehenden magnetischen Felder verst"arkt wird.
Somit ergibt sich die neue Wellenfunktion

\[
\Psi_{12}=\sum \limits_{k} a(k)c^+_{k}c^+_{-k}|G\rangle
\]

$k$ ordnen wir zugleich „Spin aufw"arts“ und $–k$ f"ur „Spin abw"arts“ zu.
Der g"unstigste Prozess ist also wenn die Elektronen entgegengesetzte Wellenzahlvektoren $(k_1 = -k_2)$ und entgegengesetzten Spin $(\sigma_1 = -\sigma_2)$ aufweisen.
\\
%\section{Elektronenverhalten in einem Festk"orper}
%\rhead{Elektronenverhalten in einem Festk"orper}
\\
\printbibliography[heading=subbibliography]
\end{refsection}

