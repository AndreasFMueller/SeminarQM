\newcommand{\RM}[1]{\MakeUppercase{\romannumeral #1{}}}

\chapter{Tunneldiode\label{chapter:tunneldiode}}
\lhead{Tunneldiode}
\begin{refsection}
\chapterauthor{Stefan Hedinger}

\newpage
\section{Einleitung}
\rhead{Einleitung}

Die Tunneldiode ist ein nicht sehr h"aufig verwendetes, aktives und dynamisches Halbleiterelement. Als Halbleitermaterial wird f"ur ihren  Aufbau Germanium verwendet. Um die Tunneldiode und ihren Aufbau besser erkl"aren zu k"onnen, ziehen wir zum Vergleich eine normale Germaniumdiode heran. 
Der Aufbau einer Diode besteht im wesentlichen aus einer positiv und einer negativ dotierten Seite, dem sogenannten p-n-"Ubergang. Dazwischen befindet sich die Sperrzone, welche man auch als Potentialbarriere betrachten kann.

\begin{figure}	%Bild Bändermodell Germaniumdiode Vf = 0
\centering
\includegraphics[width=0.5\textwidth]{tunneldiode/tunnel-9.pdf}
\caption{B"andermodell der Germaniumdiode mit $V_F = 0$
\label{skript:BaendermodellG0}}
\end{figure}

Ohne Vorw"artsspannung sind die Energieniveaus \index{Germaniumdiode}wie in Abbildung~\ref{skript:BaendermodellG0} verteilt. Das Ferminivau $E_F$ ist in p und n auf derselben H"ohe. Das Leitungsband $E_C$ ist "uber dem Ferminiveau  und das Valenzband $E_V$ unter dem Ferminiveau. Solange die Energie der Elektronen kleiner als das Potential der Barriere ist, leitet die Diode nicht. 

\begin{figure}	%Bild Bändermodell Germaniumdiode Vf > 0
\centering
\includegraphics[width=0.5\textwidth]{tunneldiode/tunnel-11.pdf}
\caption{B"andermodell der Germaniumdiode mit $V_F > 0$
\label{skript:BaendermodellG}}
\end{figure}

Legt man eine positive Spannung in Vorw"artsrichtung an, hebt man damit die Energie der Elektronen an. Wie in \index{Germaniumdiode} Abbildung~\ref{skript:BaendermodellG} ersichtlich, wird auch das Ferminiveau auf der n-Seite gr"osser als auf der p-Seite. Bei etwa 200mV Vorw"artsspannung beginnt die Germaniumdiode zu leiten. Zum Vergleich wurden zwei verschiedene Typen Germaniumdioden vermessen. \index{Germaniumdiode}In Abbildung~\ref{skript:Germaniumdioden} ist der Vorw"artsstrom durch die Diode in Abh"angigkeit zur Vorw"artsspannung aufgetragen.

\begin{figure}	%Bild Kennlinie der Germaniumdioden
\centering
\includegraphics[width=0.7\textwidth]{tunneldiode/tunnel-6.pdf}
\caption{Kennlinien von zwei verschiedenen Germaniumdiodentypen
\label{skript:Germaniumdioden}}
\end{figure}

Die Tunneldiode unterscheidet sich im Vergleich zur Germaniumdiode in zwei Punkten. Zum einen ist die Dotierung des p-n-"Ubergangs gr"osser, zum anderen die Sperrschicht viel d"unner. Diese beiden Unterschiede erm"oglichen den Tunneleffekt.


\section{Der Tunneleffekt}
\rhead{Tunneleffekt}
Das B"andermodell der Tunneldiode ist fast identisch mit dem der Germaniumdiode. Auch hier ist das Ferminiveau ohne Vorw"artsspannung auf derselben H"ohe. Auf der p-Seite ist sowohl das Valenzband als auch das Leitungsband "uber dem Ferminiveau. Auf der n-Seite ist es genau umgekehrt, da sind beide B"ander unter dem Ferminiveau. \index{Tunneldiode}(Abbildung~\ref{skript:Baendermodell0})
Bei der Tunneldiode tritt der Tunneleffekt bereits dann ein, wenn eine positive Vorw"artsspannung angelegt wird. Die Diode beginnt zu leiten, obwohl die Elektronen nicht "uber die Potentialbarriere r"uber k"onnen. Sie tunneln durch die Barriere hindurch. Der Tunnelstrom erreicht das Maximum, wenn die "Uberlappung am gr"ossten ist. \index{Tunneldiode}(Abbildung~\ref{skript:Baendermodellmax}) Danach nimmt der Tunnelstrom wieder ab, da die "Uberlappung kleiner wird. \index{Tunneldiode}(Abbildung~\ref{skript:Baendermodellmax})

\begin{figure}	%Bild Bändermodell Tunneldiode Vf = 0
\centering
\includegraphics[width=0.5\textwidth]{tunneldiode/tunnel-10.pdf}
\caption{B"andermodell der Tunneldiode mit $V_F = 0$
\label{skript:Baendermodell0}}
\end{figure}

\begin{figure}	%Bild Bändermodell Tunneldiode IT = max
\centering
\includegraphics[width=0.5\textwidth]{tunneldiode/tunnel-12.pdf}
\caption{B"andermodell der Tunneldiode mit maximalem Tunnelstrom
\label{skript:Baendermodellmax}}
\end{figure}

\begin{figure}	%Bild Bändermodell Tunneldiode IT abnehmend
\centering
\includegraphics[width=0.5\textwidth]{tunneldiode/tunnel-13.pdf}
\caption{B"andermodell der Tunneldiode mit $V_F > 0$, kein Tunnelstrom
\label{skript:Baendermodellmin}}
\end{figure}

Die spezielle Kennlinie der Tunneldiode ist in der \index{Tunneldiode}Abbildung~\ref{skript:Tunneldiode} ersichtlich.

\begin{figure}	%Bild Kennlinie der Tunneldiode
\centering
\includegraphics[width=0.7\textwidth]{tunneldiode/tunnel-7.pdf}
\caption{Die spezielle Kennlinie der Tunneldiode
\label{skript:Tunneldiode}}
\end{figure}

\section{Berechnung des Tunneleffekts}
\rhead{Tunneleffekt}
Als erstes betrachten wir das Potential 
\[
V(x)=\begin{cases}
V_0& \qquad \text{wenn } x \in [-a,a]\\
0&   \qquad \text{sonst}
\end{cases}
\]
in den verschiedenen Bereichen.

Ein Teilchen kommt von links und m"ochte nach rechts. Die Energie vom einfallenden Teilchen ist $E$ und es gilt
\[
0 < E < V_0.
\]
In der Mitte ist aber eine Barriere, welche ein h"oheres Potential als das Teilchen selber aufweist. In der normalen Mechanik ist klar, dass dieses Teilchen nicht von links nach rechts kommt sondern an der Barriere bei -a reflektiert wird. In der Quantenmechanik k"onnen Teilchen jedoch durch die Barriere durchtunneln. \index{Tunneldiode}(Abbildung~\ref{skript:Kastenpotential})

\begin{figure}	%Bild Kastenpotential
\centering
\includegraphics[width=0.7\textwidth]{tunneldiode/tunnel-4.pdf}
\caption{Kastenpotential
\label{skript:Kastenpotential}}
\end{figure}

Die nachfolgenden Berechnungen sollen helfen, den Tunneleffekt zu verstehen.  Als erstes betrachten wir die statin"are Schr"odingergleichung f"ur die Wellenfunktion $\Phi(x)$. Dabei ist $m$ die Masse und $E$ die Energie des Teilchens.
\[
E\Phi(x) = -\frac{\hbar^2}{2m}\frac{d^2}{dx^2}\Phi(x) + V(x)\Phi(x)
\]
Im Bereich \RM{1} und \RM{3} verwenden wir f"ur die Wellenfunktion den allgemeine Ansatz
\[
\Phi(x) = Ae^{ikx}+Be^{-ikx}.
\]
F"ur $k$ gilt dabei
\[
k = \sqrt{\frac{2mE}{\hbar^2}}.
\]
Der Koeffizient $A$ steht in beiden Bereichen f"ur die einlaufende Welle, $B$ f"ur die reflektierte Welle.

F"ur die Wellenfunktion im Bereich \RM{2} lautet die Gleichung
\[
\Phi_\RM{2}(x) = \alpha e^{\kappa x} + \beta e^{-\kappa x}
\]
da wir in der Potentialbarriere einen exponentiellen Abfall oder Anstieg erwarten. $\kappa$ ist
\[
\kappa = \sqrt{\frac{2m}{\hbar^2}(V_0 - E)}.
\]
Insgesamt haben wir nun sechs Unbekannte. Im Bereich $\RM{1}$ die Koeffizienten $A_1$ und $B_1$, im Bereich $\RM{2}$ $\alpha$ und $\beta$ und im Bereich $\RM{3}$ $A_2$ und $B_2$. 
Damit die Wellenfunktion ohne Spr"unge zusammengesetzt werden kann muss
\[
\Phi_\RM{1}(-a) = \Phi_\RM{2}(-a)
\]
\[
\Phi_\RM{2}(a) = \Phi_\RM{3}(a)
\]
\[
\Phi_\RM{1}'(-a) = \Phi_\RM{2}'(-a)
\]
\[
\Phi_\RM{2}'(a) = \Phi_\RM{3}'(a)
\]
gelten. Somit haben wir ein Gleichungssystem mit vier Gleichungen f"ur sechs Unbekannte. Wir brauchen also die Ableitungen der allgemeinen Wellenfunktionen .
\[
\Phi_\RM{1}'(x) = Aike^{ikx} - Bike^{-ikx}
\]
\[
\Phi_\RM{2}'(x) = \alpha \kappa e^{\kappa x} - \beta \kappa e^{-\kappa x}
\]
Wir rechnen ab hier mit Matrizen weiter. Dies hat den Vorteil, dass sich gewisse Terme vereinfachen lassen. Auch mit der Matizenrechnung ist es nicht m"oglich alle sechs Unbekannten zu bestimmen. Es ist aber m"oglich das Verh"altnis zwischen den einfallenden und transmittierten Teilchen zu berechnen. Die Matrix 
\[
\underbrace{
\begin{pmatrix}
e^{-ika}
&
e^{ika}
\\
ike^{-ika}
&
-ike^{ika}
\end{pmatrix}
}_{T_1(-a)}
\begin{pmatrix}
A_1
\\
B_1
\end{pmatrix}
 = 
\underbrace{
\begin{pmatrix}
e^{-\kappa a}
&
e^{\kappa a}
\\
\kappa e^{-\kappa a}
&
-\kappa e^{\kappa a}
\end{pmatrix}
}_{T_2(-a)}
\begin{pmatrix}
\alpha
\\
\beta
\end{pmatrix}
\]
gilt an der Stelle $-a$.
An der Stelle $a$ gilt
\[
\underbrace{
\begin{pmatrix}
e^{\kappa a}
&
e^{-\kappa a}
\\
\kappa e^{\kappa a}
&
-\kappa e^{-\kappa a}
\end{pmatrix}
}_{T_2(a)}
\begin{pmatrix}
\alpha
\\
\beta
\end{pmatrix}
 = 
\underbrace{
\begin{pmatrix}
e^{ika}
&
e^{-ika}
\\
ike^{ika}
&
-ike^{-ika}
\end{pmatrix}
}_{T_1(a)}
\begin{pmatrix}
A_2
\\
B_2
\end{pmatrix}
\].
In der ersten Zeile der Matrizen stehen die Funktionswerte und in der zweiten Zeile die Ableitungen an den entsprechenden Stellen.

Mit den Matrizen rechnen wir nun wie folgt weiter:
\[
\underbrace{
T_1(-a)^{-1}
\underbrace{
T_2(-a)T_2(a)^{-1}
}
T_1(a)
}
\]

Zuerst multiplizieren wir die inneren Matrizen miteinander. Anschliessen rechnen wir von links nach rechts. Die Berechnung von diesem Matizenprodukt hat f"ur uns Maxima erledigt. Die resultierende Matrix lautet:
\[
\begin{pmatrix}
\frac{i\,e^{2\,i\,a\,k}\,\left(\kappa-k\right)\,\left(\kappa+k
 \right)\,\sinh \left(2\,a\,\kappa\right)+2\,k\,e^{2\,i\,a\,k}\,
 \kappa\,\cosh \left(2\,a\,\kappa\right)}{2\,k\,\kappa}
&
\frac{i\,
 \left(\kappa^2+k^2\right)\,\sinh \left(2\,a\,\kappa\right)}{2\,
 k\,\kappa}
\\
-\frac{i\,\left(\kappa^2+k^2\right)\,\sinh \left(2\,a\,
 \kappa\right)}{2\,k\,\kappa}
&
\frac{2\,k\,e^ {- 2\,i\,a\,k }\,
 \kappa\,\cosh \left(2\,a\,\kappa\right)-i\,e^ {- 2\,i\,a\,k }\,
 \left(\kappa-k\right)\,\left(\kappa+k\right)\,\sinh \left(2\,a\,
 \kappa\right)}{2\,k\,\kappa}
\end{pmatrix}
\]

Die Matrix wird jetzt mit dem Vektor im Bereich \RM{3} multipliziert. Wir normieren die einfallende Welle in diesem Bereich auf 1. Weiter nehmen wir an, dass die reflektierte Welle 0 ist.
\[
\begin{pmatrix}
A_2
\\
B_2
\end{pmatrix}
=
\begin{pmatrix}
1
\\
0
\end{pmatrix}
\]

In der von Maxima berechneten Matrix ersetzen wir alle $2a$ durch $l$. Als Resultat erhalten wir den Vektor
\[
\begin{pmatrix}
\displaystyle
e^{ilk}\biggl(\cosh(l\kappa)
+\frac{i}{2}\biggl(\frac{\kappa}{k}-\frac{k}{\kappa}\biggr)\sinh(l\kappa)
\biggr)
\\
\displaystyle
- \frac{i}{2}
\biggl(\frac{\kappa}{k}+\frac{k}{\kappa}\biggr)
\sinh(l\kappa)
\end{pmatrix}.
\]
In diesem Vektor steht nun, wie viel wahrscheinlicher ein einfallendes
Teilchen im Vergleich zu einem durchgelassenen Teilchen ist.

\section{Vermessen der Tunneldiode}
\rhead{Vermessen Tunneldiode}

%TODO: 

%\section{"Ubergangsmatrix}
Die Matrix
\[
\begin{pmatrix}
\frac{i\,e^{2\,i\,a\,k}\,\left(\kappa-k\right)\,\left(\kappa+k
 \right)\,\sinh \left(2\,a\,\kappa\right)+2\,k\,e^{2\,i\,a\,k}\,
 \kappa\,\cosh \left(2\,a\,\kappa\right)}{2\,k\,\kappa}
&
\frac{i\,
 \left(\kappa^2+k^2\right)\,\sinh \left(2\,a\,\kappa\right)}{2\,
 k\,\kappa}
\\
-\frac{i\,\left(\kappa^2+k^2\right)\,\sinh \left(2\,a\,
 \kappa\right)}{2\,k\,\kappa}
&
\frac{2\,k\,e^ {- 2\,i\,a\,k }\,
 \kappa\,\cosh \left(2\,a\,\kappa\right)-i\,e^ {- 2\,i\,a\,k }\,
 \left(\kappa-k\right)\,\left(\kappa+k\right)\,\sinh \left(2\,a\,
 \kappa\right)}{2\,k\,\kappa}
\end{pmatrix}
\]
berechnet die Amplitude der einfallenden und reflektierten Wellen
aus den Amplituden der Wellen rechts von der Barriere.
Davon brauchen wir nur die erste Spalte.

In allen Termen kann man $2a$ durch die Dicke $l$ der Barriere ersetzen.
Die erste Spalte gibt wieder, wieviel mal wahrscheinlicher ein einfallendes
teilchen im Vergleich zu einem durchgelassenen Teilchen ist:
\[
\begin{pmatrix}
\displaystyle
e^{ilk}\biggl(\cosh(l\kappa)
+\frac{i}{2}\biggl(\frac{\kappa}{k}-\frac{k}{\kappa}\biggr)\sinh(l\kappa)
\biggr)
\\
\displaystyle
- \frac{i}{2}
\biggl(\frac{\kappa}{k}+\frac{k}{\kappa}\biggr)
\sinh(l\kappa)
\end{pmatrix}
\]

\section{Amplitude der Wellen links von der Barriere}
Die Amplitude der einfallenden Welle ist
\begin{align*}
A_{\text{in}}
&=
\frac{i e^{i l k} (\kappa-k) (\kappa+k) 
 \sinh (l \kappa)+2 k e^{i l k} \kappa \cosh 
 (l \kappa)}{2 k \kappa}
\\
&=
e^{i l k}
\biggl(
\frac{
i
(\kappa^2-k^2)
}{2 k \kappa}
\sinh (l \kappa)
+
\cosh (l \kappa)
\biggr)
\\
|A_{\text{in}}|^2
&=
\cosh^2(l\kappa)
+
\frac14\biggl(
\frac{\kappa}{k}-\frac{k}{\kappa}
\biggr)^2
\sinh^2(l\kappa)
\\
A_{\text{reflect}}
&=
-\frac{i}{2}
\biggl(\frac{\kappa}{k}+\frac{k}{\kappa}\biggr)
\sinh (2 a \kappa)
\\
|A_{\text{reflect}}|^2
&=
\frac14
\biggl(\frac{\kappa}{k}+\frac{k}{\kappa}\biggr)^2
\sinh^2 (2 a \kappa)
\end{align*}

\section{Wellenfunktion in der Barriere}
Betragsquadrat der Wellenfunktion in der Barriere
\begin{align*}
|\psi_{\text{Barrier}}(x)|^2
&=
\frac{
(\kappa^2+k^2)\cosh (2\kappa x-2a\kappa)+(\kappa-k)(\kappa+k)
}{
2\kappa ^2
}
\\
&=
\frac12
\biggl(1+\frac{k^2}{\kappa^2}\biggr)
\cosh (2\kappa(x-a))
+
\frac12
\biggl(1-\frac{k^2}{\kappa^2}\biggr)
\\
&=
\cosh (2\kappa(x-a))+1
+
\frac{k^2}{\kappa^2}
(\underbrace{\cosh (2\kappa(x-a))-1}_{\ge 0})
>0
\end{align*}
f"ur $x\le a$.

Intensit"atsverh"altnis zwischen einfallender und durchgelassener
Welle
\[
\frac{1}{
\displaystyle
\cosh^2 l\kappa 
+
4\biggl(\displaystyle\frac{\kappa}{k}-\frac{k}{\kappa}\biggr)^2\sinh^2l\kappa
}
\]

Intensit"atsverh"altnis zwischen einfallender und reflektierter
Welle:
\[
\frac{
4\coth^2l\kappa + \biggl(\displaystyle\frac{\kappa}{k}-\frac{k}{\kappa}\biggr)^2
}{
\biggl(\displaystyle\frac{\kappa}{k}+\frac{k}{\kappa}\biggr)^2
}
\]

F"ur die Amplitude der Wellenfunktion links von der Barriere
ist es n"utzlich, die beiden Gr"ossen
\begin{align*}
K_p&=\frac{\kappa^2+k^2}{2\kappa k}
\\
K_m&=\frac{\kappa^2-k^2}{2\kappa k}
\end{align*}
als Abk"urzung zu verwenden. Damit kann man die von \texttt{maxima} berechnete
Wahrscheinlichkeit wie folgt umformen:
\begin{align*}
|\psi_{\text{left}}(x)|^2
&=
\frac{
-
2 \sin (2 \delta) k \kappa (\kappa^2+k^2) \sinh (4 a \kappa)
+
(\kappa^2+k^2)^2 \cosh (4 a \kappa)
}{4 k ^2 \kappa^2}
\\
&\qquad
+
\frac{
-
\cos (2 \delta) (\kappa-k) (\kappa+k) (\kappa^2+k^2) \cosh (4 a \kappa)
-
(\kappa-k)^2 (\kappa+k)^2
}{4 k ^2 \kappa^2}
\\
&\qquad
+
\frac{
\cos (2 \delta) (\kappa-k) (\kappa+k) (\kappa^2+k^2)
}{4 k ^2 \kappa^2}
\\
&=
-
\frac{
2 \sin (2 \delta) k \kappa (\kappa^2+k^2) \sinh (4 a \kappa)
}{4 k ^2 \kappa^2}
+
\frac{
(\kappa^2+k^2)^2 \cosh (4 a \kappa)
}{4 k ^2 \kappa^2}
\\
&\qquad
-
\frac{
\cos (2 \delta) (\kappa-k) (\kappa+k) (\kappa^2+k^2) \cosh (4 a \kappa)
}{4 k ^2 \kappa^2}
-
\frac{
(\kappa-k)^2 (\kappa+k)^2
}{4 k ^2 \kappa^2}
\\
&\qquad
+
\frac{
\cos (2 \delta) (\kappa-k) (\kappa+k) (\kappa^2+k^2)
}{4 k ^2 \kappa^2}
\\
&=
-
\frac{
(\kappa^2+k^2)
}{2 k \kappa}
\sin (2 \delta)
\sinh (4 a \kappa)
+
\frac{
(\kappa^2+k^2)^2
}{4 k ^2 \kappa^2}
\cosh (4 a \kappa)
\\
&\qquad
-
\frac{
(\kappa^2-k^2) (\kappa^2+k^2)
}{4 k ^2 \kappa^2}
\cos (2 \delta)
\cosh (4 a \kappa)
-
\frac{
(\kappa-k)^2 (\kappa+k)^2
}{4 k ^2 \kappa^2}
\\
&\qquad
+
\frac{
(\kappa^2-k^2) (\kappa^2+k^2)
}{4 k ^2 \kappa^2}
\cos (2 \delta)
\\
&=
-
K_p
\sin (2 \delta)
\sinh (4 a \kappa)
+
K_p^2
\cosh (4 a \kappa)
-
K_pK_m
\cos (2 \delta)
\cosh (4 a \kappa)
-
K_m^2
+
K_pK_m
\cos (2 \delta)
\\
&=
-
K_p \sin (2 \delta) \sinh (4 a \kappa)
+
K_p^2 \cosh (4 a \kappa)
-
K_pK_m \cos (2 \delta) (\cosh (4 a \kappa)-1)
-
K_m^2
\end{align*}


\section{Fall $E>V_0$}

\[
\begin{pmatrix}
\frac{2\,k\,e^{2\,i\,a\,k}\,\kappa\,\cos \left(2\,a\,\kappa
 \right)-i\,e^{2\,i\,a\,k}\,\left(i\,\kappa^2+\kappa^2-i\,k^2+k^2
 \right)\,\sin \left(2\,a\,\kappa\right)}{2\,k\,\kappa}
&
-\frac{i\,
 \left(\kappa+k\right)\,\left(i\,\kappa+\kappa+i\,k-k\right)\,\sin 
 \left(2\,a\,\kappa\right)}{2\,k\,\kappa}
\\
\frac{i\,\left(\kappa-
 k\right)\,\left(i\,\kappa+\kappa-i\,k+k\right)\,\sin \left(2\,a\,
 \kappa\right)}{2\,k\,\kappa}
&
\frac{i\,e^ {- 2\,i\,a\,k }\,\left(i
 \,\kappa^2+\kappa^2-i\,k^2+k^2\right)\,\sin \left(2\,a\,\kappa
 \right)+2\,k\,e^ {- 2\,i\,a\,k }\,\kappa\,\cos \left(2\,a\,\kappa
 \right)}{2\,k\,\kappa}
\end{pmatrix}
\]


%
% Tunneldioden-Oszillator
%
% (c) 2015 Prof Dr Andreas Mueller, Hochschule Rapperswil
%
\section{Tunneldioden-Oszillator}
\rhead{Tunneldioden-Oszillator}
\begin{figure}
\centering
\includestandalone{tunneldiode/schematic}
\caption{Oszillator-Schaltung mit Tunneldiode
\label{tunnel:tunneldioden-oszillator}}
\end{figure}
Eine Tunneldiode kann dazu verwendet werden, einen Schwingkreis zu
entd"ampfen und so einen einfachen Oszillator zu konstruieren.
Eine m"ogliche Schaltung zeigt Abbildung~\ref{tunnel:tunneldioden-oszillator}.

Um die Schaltung zu analysieren, brauchen wir ein Modell f"ur die
Charakteristik der Tunneldiode. Wir nehmen dazu an, dass $U_0$ kleiner
ist also die Spannung, bei der die normale Diodenleitung einsetzt.
Dann k"onnen wir zwischen dem Strom $I_T$ durch die Tunneldiode
und der Vorw"artsspannung $U_T$ einen linearen Zusammenhang
\begin{equation}
I_T=b-aU_T
\label{tunnel:charakteristik}
\end{equation}
annehmen. $b$ entspricht in etwa dem Peak-Strom $I_{\text{peak}}$, der
Tunnelstrom bricht bei der Spannung $b/a$ zusammen.

Jetzt k"onnen wir die Differentialgleichung f"ur die Spannung $U(t)$
aufstellen. Wir bezeichnen den Spannungsabfall "uber dem Bauteil $X$
mit $U_X$ und den Strom durch das Bauteil mit $I_X$. Die Knotengleichung
bei $U(t)$ besagt:
\[
I_L+I_C=I_T
\]
Daraus und aus $U_T=U_0-U(t)$ kann $I_T$ mit Hilfe der angenommenen
Charakteristik~(\ref{tunnel:charakteristik}) der Tunneldiode eliminert werden.
Es ist
\begin{equation}
I_L+I_C=b-aU_T=b-a(U_0-U(t)).
\label{tunnel:knotengleichung}
\end{equation}
F"ur die Str"ome im Schwingkreis gilt
\begin{align*}
C\dot U_C&=I_C \qquad\text{und}\\
U_L&=L\dot I_L
\end{align*}
Nat"urlich ist $U_L=U_C=U(t)$.
Da wir in der zweiten Gleichung nur die Ableitung von $I_L$ nach
der Zeit haben, leiten wir
auch die erste und die Gleichung~(\ref{tunnel:knotengleichung}) nach
der Zeit ab, und setzen die abgeleiteten Str"ome
in~(\ref{tunnel:knotengleichung}) ein.
So erhalten wir die Differentialgleichung
\begin{equation*}
\frac1{L}U(t) +C\ddot U(t)=a\dot U(t)
\end{equation*}
oder
\begin{equation}
\ddot U(t)
-\frac{a}{C}\dot U(t)
+\frac1{LC}U(t)
=0.
\label{tunnel:dgl1}
\end{equation}
F"ur $a=0$ wird daraus eine gew"ohnliche Schwingungsdifferentialgleichung
mit Kreisfrequenz $\omega_0^2=1/LC$.

\begin{figure}
\centering
\includegraphics{tunneldiode/tunnel-8.pdf}
\caption{Designparameter $L$ und $C$ des Tunneldioden-Oszillators nach
Abbildung~\ref{tunnel:tunneldioden-oszillator}. 
Die Hyperbeln verbinden Punkte gleicher Eigenfrequenz des Schwingkreises.
Dass hellgraue Gebiet wird ausgeschlossen durch die Bedingung, dass die
Kapazit"at gross genug sein muss.
Das dunkelgraue Gebiet schliesst Werte der Induktivit"at aus, wenn die
Kreisfrequenz $\omega$ erreicht werden soll.
\label{tunnel:designparameter}}
\end{figure}

Das charakteristische Polynom der Differentialgleichung~(\ref{tunnel:dgl1})
ist
\[
\lambda^2-\frac{a}{C}\lambda +\omega_0^2\lambda=0
\]
mit den Nullstellen
\[
\lambda_\pm = \frac{a}{2C}\pm\sqrt{\frac{a^2}{4C^2}-\omega_0^2}.
\]
Damit tats"achlich Schwingungsl"osungen entstehen, muss der Radikand
negativ sein, es muss also gelten
\[
\frac{a^2}{4C^2}<\frac1{LC}
\qquad\Rightarrow\qquad
\frac{C}{L} > \frac{a^2}{4}.
\]
Die Kapazit"at $C$ muss also ausreichend gross sein.
Abbildung~\ref{tunnel:designparameter} zeigt den Raum der Design-Parameter
$C$ und $L$.

Eine grosse Kapazit"at $C$ bewirkt aber auch, dass der Koeffizient
$-a/C$ des $\dot U(t)$-Terms kleiner wird. Mit zunehmender Kapazit"at
wird als die Entd"ampfung geringer, und je nach parasit"arer D"ampfung
kann es schwierig werden, dass die Schaltung "uberhaupt anschwingt.

Die Schwingungsfrequenz wird f"ur $a\ne 0$ modifiziert zu
\begin{equation}
\omega
=
\sqrt{
\frac{1}{LC}
-
\frac{a^2}{4C^2}
}.
\label{tunnel:oszillatorfrequenz}
\end{equation}
Bei vorgegebener Schwingungsfrequenz und Induktivit"at ist dies eine
quadratische Gleichung f"ur $C$, wir schreiben sie in der gebr"auchlicheren
Form
\[
\omega^2 C^2 -\frac1{L}C+\frac{a^2}{4}=0.
\]
Diese Gleichung hat nur dann reelle L"osungen f"ur $C$, wenn die Diskriminante
positiv ist, also
\begin{equation}
\begin{aligned}
\frac1{L^2}-4\omega^2\frac{a^2}4&>0
&
&\Rightarrow&
\frac1{L^2}&>\omega^2a^2
&
&\Rightarrow&
L<&\frac1{a\omega}
\end{aligned}
\end{equation}
Wenn also die Kreisfrequenz $\omega$ angestrebt werden soll, dann darf die
Induktivit"at nicht zu gross sein.
Dies wird in Abbildung~\ref{tunnel:designparameter} durch das dunkelgraue
Gebiet dargestellt.

\begin{beispiel}
Die in diesem Kapitel genauer untersuchte Tunneldiode 1N3721 hat einen
Spitzenstrom von $I_{\text{peak}}=25\,\text{mA}$, der Tunnelstrom
bricht bei $450\,\text{mV}$ zusammen.
Wir k"onnen also von $a=25/450=0.0555\text{A/V}$ ausgehen.
Mit einer Induktivit"at von $L=0.1\mu\text{H}$ l"asst sich dann bestenfalls
die Kreisfrequenz
$\omega=1/(aL)=1/(10^{-7}\cdot 0.055)=180\cdot 10^6\,\text{s}^{-1}$
erreichen,
also die Frequenz $28.7\,\text{MHz}$.
Die Kapazit"at muss mindestens $C>La^2/4=77\text{pF}$ sein.

Mit einem $100\,\text{nF}$-Kondensator gilt f"ur die Eigenkreisfrequenz
$\omega_0$ des Schwingkreises $\omega_0^2=1/LC=10^{14}$, die Eigenfrequenz
ist also etwa $1.5915\,\text{MHz}$.
F"ur die Frequenz des Oszillators muss allerdings die
Formel~(\ref{tunnel:oszillatorfrequenz}) verwendet werden.
Da aber in unserem Fall $a^2/4\simeq 0.00077$ ziemlich klein ist, ist
die Frequenz fast gleich wie die Eigenfrequenz des Schwingkreises.
Anwendung der Formel~(\ref{tunnel:oszillatorfrequenz})
ergibt $1.5909\,\text{MHz}$, also tats"achlich nur ein sehr kleiner
Unterschied zur Eigenfrequenz.
\end{beispiel}



\printbibliography[heading=subbibliography]
\end{refsection}

