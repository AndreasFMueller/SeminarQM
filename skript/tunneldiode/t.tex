\section{"Ubergangsmatrix}
Die Matrix
\[
\begin{pmatrix}
\frac{i\,e^{2\,i\,a\,k}\,\left(\kappa-k\right)\,\left(\kappa+k
 \right)\,\sinh \left(2\,a\,\kappa\right)+2\,k\,e^{2\,i\,a\,k}\,
 \kappa\,\cosh \left(2\,a\,\kappa\right)}{2\,k\,\kappa}
&
\frac{i\,
 \left(\kappa^2+k^2\right)\,\sinh \left(2\,a\,\kappa\right)}{2\,
 k\,\kappa}
\\
-\frac{i\,\left(\kappa^2+k^2\right)\,\sinh \left(2\,a\,
 \kappa\right)}{2\,k\,\kappa}
&
\frac{2\,k\,e^ {- 2\,i\,a\,k }\,
 \kappa\,\cosh \left(2\,a\,\kappa\right)-i\,e^ {- 2\,i\,a\,k }\,
 \left(\kappa-k\right)\,\left(\kappa+k\right)\,\sinh \left(2\,a\,
 \kappa\right)}{2\,k\,\kappa}
\end{pmatrix}
\]
berechnet die Amplitude der einfallenden und reflektierten Wellen
aus den Amplituden der Wellen rechts von der Barriere.
Davon brauchen wir nur die erste Spalte.

In allen Termen kann man $2a$ durch die Dicke $l$ der Barriere ersetzen.
Die erste Spalte gibt wieder, wieviel mal wahrscheinlicher ein einfallendes
teilchen im Vergleich zu einem durchgelassenen Teilchen ist:
\[
\begin{pmatrix}
\displaystyle
e^{ilk}\biggl(\cosh(l\kappa)
+\frac{i}{2}\biggl(\frac{\kappa}{k}-\frac{k}{\kappa}\biggr)\sinh(l\kappa)
\biggr)
\\
\displaystyle
- \frac{i}{2}
\biggl(\frac{\kappa}{k}+\frac{k}{\kappa}\biggr)
\sinh(l\kappa)
\end{pmatrix}
\]

\section{Amplitude der Wellen links von der Barriere}
Die Amplitude der einfallenden Welle ist
\begin{align*}
A_{\text{in}}
&=
\frac{i e^{i l k} (\kappa-k) (\kappa+k) 
 \sinh (l \kappa)+2 k e^{i l k} \kappa \cosh 
 (l \kappa)}{2 k \kappa}
\\
&=
e^{i l k}
\biggl(
\frac{
i
(\kappa^2-k^2)
}{2 k \kappa}
\sinh (l \kappa)
+
\cosh (l \kappa)
\biggr)
\\
|A_{\text{in}}|^2
&=
\cosh^2(l\kappa)
+
\frac14\biggl(
\frac{\kappa}{k}-\frac{k}{\kappa}
\biggr)^2
\sinh^2(l\kappa)
\\
A_{\text{reflect}}
&=
-\frac{i}{2}
\biggl(\frac{\kappa}{k}+\frac{k}{\kappa}\biggr)
\sinh (2 a \kappa)
\\
|A_{\text{reflect}}|^2
&=
\frac14
\biggl(\frac{\kappa}{k}+\frac{k}{\kappa}\biggr)^2
\sinh^2 (2 a \kappa)
\end{align*}

\section{Wellenfunktion in der Barriere}
Betragsquadrat der Wellenfunktion in der Barriere
\begin{align*}
|\psi_{\text{Barrier}}(x)|^2
&=
\frac{
(\kappa^2+k^2)\cosh (2\kappa x-2a\kappa)+(\kappa-k)(\kappa+k)
}{
2\kappa ^2
}
\\
&=
\frac12
\biggl(1+\frac{k^2}{\kappa^2}\biggr)
\cosh (2\kappa(x-a))
+
\frac12
\biggl(1-\frac{k^2}{\kappa^2}\biggr)
\\
&=
\cosh (2\kappa(x-a))+1
+
\frac{k^2}{\kappa^2}
(\underbrace{\cosh (2\kappa(x-a))-1}_{\ge 0})
>0
\end{align*}
f"ur $x\le a$.

Intensit"atsverh"altnis zwischen einfallender und durchgelassener
Welle
\[
\frac{1}{
\displaystyle
\cosh^2 l\kappa 
+
4\biggl(\displaystyle\frac{\kappa}{k}-\frac{k}{\kappa}\biggr)^2\sinh^2l\kappa
}
\]

Intensit"atsverh"altnis zwischen einfallender und reflektierter
Welle:
\[
\frac{
4\coth^2l\kappa + \biggl(\displaystyle\frac{\kappa}{k}-\frac{k}{\kappa}\biggr)^2
}{
\biggl(\displaystyle\frac{\kappa}{k}+\frac{k}{\kappa}\biggr)^2
}
\]

F"ur die Amplitude der Wellenfunktion links von der Barriere
ist es n"utzlich, die beiden Gr"ossen
\begin{align*}
K_p&=\frac{\kappa^2+k^2}{2\kappa k}
\\
K_m&=\frac{\kappa^2-k^2}{2\kappa k}
\end{align*}
als Abk"urzung zu verwenden. Damit kann man die von \texttt{maxima} berechnete
Wahrscheinlichkeit wie folgt umformen:
\begin{align*}
|\psi_{\text{left}}(x)|^2
&=
\frac{
-
2 \sin (2 \delta) k \kappa (\kappa^2+k^2) \sinh (4 a \kappa)
+
(\kappa^2+k^2)^2 \cosh (4 a \kappa)
}{4 k ^2 \kappa^2}
\\
&\qquad
+
\frac{
-
\cos (2 \delta) (\kappa-k) (\kappa+k) (\kappa^2+k^2) \cosh (4 a \kappa)
-
(\kappa-k)^2 (\kappa+k)^2
}{4 k ^2 \kappa^2}
\\
&\qquad
+
\frac{
\cos (2 \delta) (\kappa-k) (\kappa+k) (\kappa^2+k^2)
}{4 k ^2 \kappa^2}
\\
&=
-
\frac{
2 \sin (2 \delta) k \kappa (\kappa^2+k^2) \sinh (4 a \kappa)
}{4 k ^2 \kappa^2}
+
\frac{
(\kappa^2+k^2)^2 \cosh (4 a \kappa)
}{4 k ^2 \kappa^2}
\\
&\qquad
-
\frac{
\cos (2 \delta) (\kappa-k) (\kappa+k) (\kappa^2+k^2) \cosh (4 a \kappa)
}{4 k ^2 \kappa^2}
-
\frac{
(\kappa-k)^2 (\kappa+k)^2
}{4 k ^2 \kappa^2}
\\
&\qquad
+
\frac{
\cos (2 \delta) (\kappa-k) (\kappa+k) (\kappa^2+k^2)
}{4 k ^2 \kappa^2}
\\
&=
-
\frac{
(\kappa^2+k^2)
}{2 k \kappa}
\sin (2 \delta)
\sinh (4 a \kappa)
+
\frac{
(\kappa^2+k^2)^2
}{4 k ^2 \kappa^2}
\cosh (4 a \kappa)
\\
&\qquad
-
\frac{
(\kappa^2-k^2) (\kappa^2+k^2)
}{4 k ^2 \kappa^2}
\cos (2 \delta)
\cosh (4 a \kappa)
-
\frac{
(\kappa-k)^2 (\kappa+k)^2
}{4 k ^2 \kappa^2}
\\
&\qquad
+
\frac{
(\kappa^2-k^2) (\kappa^2+k^2)
}{4 k ^2 \kappa^2}
\cos (2 \delta)
\\
&=
-
K_p
\sin (2 \delta)
\sinh (4 a \kappa)
+
K_p^2
\cosh (4 a \kappa)
-
K_pK_m
\cos (2 \delta)
\cosh (4 a \kappa)
-
K_m^2
+
K_pK_m
\cos (2 \delta)
\\
&=
-
K_p \sin (2 k (x+a)) \sinh (4 a \kappa)
+
K_p^2 \cosh (4 a \kappa)
-
K_pK_m \cos (2 k(x+a)) (\cosh (4 a \kappa)-1)
-
K_m^2
\end{align*}


\section{Fall $E>V_0$}

\[
\begin{pmatrix}
\frac{2\,k\,e^{2\,i\,a\,k}\,\kappa\,\cos \left(2\,a\,\kappa
 \right)-i\,e^{2\,i\,a\,k}\,\left(i\,\kappa^2+\kappa^2-i\,k^2+k^2
 \right)\,\sin \left(2\,a\,\kappa\right)}{2\,k\,\kappa}
&
-\frac{i\,
 \left(\kappa+k\right)\,\left(i\,\kappa+\kappa+i\,k-k\right)\,\sin 
 \left(2\,a\,\kappa\right)}{2\,k\,\kappa}
\\
\frac{i\,\left(\kappa-
 k\right)\,\left(i\,\kappa+\kappa-i\,k+k\right)\,\sin \left(2\,a\,
 \kappa\right)}{2\,k\,\kappa}
&
\frac{i\,e^ {- 2\,i\,a\,k }\,\left(i
 \,\kappa^2+\kappa^2-i\,k^2+k^2\right)\,\sin \left(2\,a\,\kappa
 \right)+2\,k\,e^ {- 2\,i\,a\,k }\,\kappa\,\cos \left(2\,a\,\kappa
 \right)}{2\,k\,\kappa}
\end{pmatrix}
\]

\begin{figure}
\centering
\includegraphics{tunneldiode/tunnel-1.pdf}
\end{figure}
\begin{figure}
\centering
\includegraphics{tunneldiode/tunnel-2.pdf}
\end{figure}
