%
% Tunneldioden-Oszillator
%
% (c) 2015 Prof Dr Andreas Mueller, Hochschule Rapperswil
%
\section{Tunneldioden-Oszillator}
\rhead{Tunneldioden-Oszillator}
\begin{figure}
\centering
\includestandalone{tunneldiode/schematic}
\caption{Oszillator-Schaltung mit Tunneldiode
\label{tunnel:tunneldioden-oszillator}}
\end{figure}
Eine Tunneldiode kann dazu verwendet werden, einen Schwingkreis zu
entd"ampfen und so einen einfachen Oszillator zu konstruieren.
Eine m"ogliche Schaltung zeigt Abbildung~\ref{tunnel:tunneldioden-oszillator}.

Um die Schaltung zu analysieren, brauchen wir ein Modell f"ur die
Charakteristik der Tunneldiode. Wir nehmen dazu an, dass $U_0$ kleiner
ist also die Spannung, bei der die normale Diodenleitung einsetzt.
Dann k"onnen wir zwischen dem Strom $I_T$ durch die Tunneldiode
und der Vorw"artsspannung $U_T$ einen linearen Zusammenhang
\begin{equation}
I_T=b-aU_T
\label{tunnel:charakteristik}
\end{equation}
annehmen. $b$ entspricht in etwa dem Peak-Strom $I_{\text{peak}}$, der
Tunnelstrom bricht bei der Spannung $b/a$ zusammen.

Jetzt k"onnen wir die Differentialgleichung f"ur die Spannung $U(t)$
aufstellen. Wir bezeichnen den Spannungsabfall "uber dem Bauteil $X$
mit $U_X$ und den Strom durch das Bauteil mit $I_X$. Die Knotengleichung
bei $U(t)$ besagt:
\[
I_L+I_C=I_T
\]
Daraus und aus $U_T=U_0-U(t)$ kann $I_T$ mit Hilfe der angenommenen
Charakteristik~(\ref{tunnel:charakteristik}) der Tunneldiode eliminert werden.
Es ist
\begin{equation}
I_L+I_C=b-aU_T=b-a(U_0-U(t)).
\label{tunnel:knotengleichung}
\end{equation}
F"ur die Str"ome im Schwingkreis gilt
\begin{align*}
C\dot U_C&=I_C \qquad\text{und}\\
U_L&=L\dot I_L
\end{align*}
Nat"urlich ist $U_L=U_C=U(t)$.
Da wir in der zweiten Gleichung nur die Ableitung von $I_L$ nach
der Zeit haben, leiten wir
auch die erste und die Gleichung~(\ref{tunnel:knotengleichung}) nach
der Zeit ab, und setzen die abgeleiteten Str"ome
in~(\ref{tunnel:knotengleichung}) ein.
So erhalten wir die Differentialgleichung
\begin{equation*}
\frac1{L}U(t) +C\ddot U(t)=a\dot U(t)
\end{equation*}
oder
\begin{equation}
\ddot U(t)
-\frac{a}{C}\dot U(t)
+\frac1{LC}U(t)
=0.
\label{tunnel:dgl1}
\end{equation}
F"ur $a=0$ wird daraus eine gew"ohnliche Schwingungsdifferentialgleichung
mit Kreisfrequenz $\omega_0^2=1/LC$.

\begin{figure}
\centering
\includegraphics{tunneldiode/tunnel-8.pdf}
\caption{Designparameter $L$ und $C$ des Tunneldioden-Oszillators nach
Abbildung~\ref{tunnel:tunneldioden-oszillator}. 
Die Hyperbeln verbinden Punkte gleicher Eigenfrequenz des Schwingkreises.
Dass hellgraue Gebiet wird ausgeschlossen durch die Bedingung, dass die
Kapazit"at gross genug sein muss.
Das dunkelgraue Gebiet schliesst Werte der Induktivit"at aus, wenn die
Kreisfrequenz $\omega$ erreicht werden soll.
\label{tunnel:designparameter}}
\end{figure}

Das charakteristische Polynom der Differentialgleichung~(\ref{tunnel:dgl1})
ist
\[
\lambda^2-\frac{a}{C}\lambda +\omega_0^2\lambda=0
\]
mit den Nullstellen
\[
\lambda_\pm = \frac{a}{2C}\pm\sqrt{\frac{a^2}{4C^2}-\omega_0^2}.
\]
Damit tats"achlich Schwingungsl"osungen entstehen, muss der Radikand
negativ sein, es muss also gelten
\[
\frac{a^2}{4C^2}<\frac1{LC}
\qquad\Rightarrow\qquad
\frac{C}{L} > \frac{a^2}{4}.
\]
Die Kapazit"at $C$ muss also ausreichend gross sein.
Abbildung~\ref{tunnel:designparameter} zeigt den Raum der Design-Parameter
$C$ und $L$.

Eine grosse Kapazit"at $C$ bewirkt aber auch, dass der Koeffizient
$-a/C$ des $\dot U(t)$-Terms kleiner wird. Mit zunehmender Kapazit"at
wird als die Entd"ampfung geringer, und je nach parasit"arer D"ampfung
kann es schwierig werden, dass die Schaltung "uberhaupt anschwingt.

Die Schwingungsfrequenz wird f"ur $a\ne 0$ modifiziert zu
\begin{equation}
\omega
=
\sqrt{
\frac{1}{LC}
-
\frac{a^2}{4C^2}
}.
\label{tunnel:oszillatorfrequenz}
\end{equation}
Bei vorgegebener Schwingungsfrequenz und Induktivit"at ist dies eine
quadratische Gleichung f"ur $C$, wir schreiben sie in der gebr"auchlicheren
Form
\[
\omega^2 C^2 -\frac1{L}C+\frac{a^2}{4}=0.
\]
Diese Gleichung hat nur dann reelle L"osungen f"ur $C$, wenn die Diskriminante
positiv ist, also
\begin{equation}
\begin{aligned}
\frac1{L^2}-4\omega^2\frac{a^2}4&>0
&
&\Rightarrow&
\frac1{L^2}&>\omega^2a^2
&
&\Rightarrow&
L<&\frac1{a\omega}
\end{aligned}
\end{equation}
Wenn also die Kreisfrequenz $\omega$ angestrebt werden soll, dann darf die
Induktivit"at nicht zu gross sein.
Dies wird in Abbildung~\ref{tunnel:designparameter} durch das dunkelgraue
Gebiet dargestellt.

\begin{beispiel}
Die in diesem Kapitel genauer untersuchte Tunneldiode 1N3721 hat einen
Spitzenstrom von $I_{\text{peak}}=25\,\text{mA}$, der Tunnelstrom
bricht bei $450\,\text{mV}$ zusammen.
Wir k"onnen also von $a=25/450=0.0555\text{A/V}$ ausgehen.
Mit einer Induktivit"at von $L=0.1\mu\text{H}$ l"asst sich dann bestenfalls
die Kreisfrequenz
$\omega=1/(aL)=1/(10^{-7}\cdot 0.055)=180\cdot 10^6\,\text{s}^{-1}$
erreichen,
also die Frequenz $28.7\,\text{MHz}$.
Die Kapazit"at muss mindestens $C>La^2/4=77\text{pF}$ sein.

Mit einem $100\,\text{nF}$-Kondensator gilt f"ur die Eigenkreisfrequenz
$\omega_0$ des Schwingkreises $\omega_0^2=1/LC=10^{14}$, die Eigenfrequenz
ist also etwa $1.5915\,\text{MHz}$.
F"ur die Frequenz des Oszillators muss allerdings die
Formel~(\ref{tunnel:oszillatorfrequenz}) verwendet werden.
Da aber in unserem Fall $a^2/4\simeq 0.00077$ ziemlich klein ist, ist
die Frequenz fast gleich wie die Eigenfrequenz des Schwingkreises.
Anwendung der Formel~(\ref{tunnel:oszillatorfrequenz})
ergibt $1.5909\,\text{MHz}$, also tats"achlich nur ein sehr kleiner
Unterschied zur Eigenfrequenz.
\end{beispiel}

