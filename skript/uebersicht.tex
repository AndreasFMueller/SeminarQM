\chapter*{"Ubersicht}
\rhead{"Ubersicht}
\rhead{}
Im zweiten Teil kommen die Teilnehmer des Seminars selbst zu Wort.
Sie zeigen Anwendungsbeispiele f"ur die im ersten
Teil entwickelte Quantenmechanik.
Das Ziel ist nicht, die vorgestellten Anwendungen vollst"andig
berechnen zu k"onnen, sondern an Hand von vereinfachten Modellen
zu zeigen, wie der quantenmechanische Formalismus die wesentlichen
Eigenschaften der Anwendungen zu verstehen erlaubt.
Modellm"assiges Verst"andnis des Mechanismus ist das Ziel, nicht pr"azise
numerische Resultate.

Die ersten drei Beitr"age befassen sich mit Quanten-Informatik.
Das No-Cloning-Theorem~\ref{skript:no-cloning-theorem} bedeutet,
dass sich ein quantenmechanischer
Zustand nicht kopieren l"asst, die ideale Voraussetzung f"ur ein
kryptographisches Verfahren.
Dar"uber berichten {\em Benny G"achter} und {\em Tobias Stauber}.
{\em Max Obrist} und {\em Martin Stypinski} erl"autern, wie ein Quantenzustand
an einen anderen Ort gebracht, also teleportiert werden kann.
Und {\em Marc Juchli} und {\em Kirusanth Poopalasingam} zeigen,
wie Quantencomputern
das Potential haben, Probleme effizient zu l"osen, die klassische 
Computer nicht effizient l"osen k"onnen.

Die Quantenmechanik ist unerl"asslich zum Verst"andnis von
Halbleiterbauelementen.
{\em Stefan Hedinger} erl"autert mit der Tunneldiode
ein eher exotisches Bauelement, welches den quantenmechanischen
Tunneleffekt nutzt.
Auch moderne Flashspeicher brauchen den Tunneleffekt zum L"oschen 
der Speicherzellen.
Man braucht jedoch ein noch etwas tiefer gehendes Verst"andnis der
Quantenmechanik,
um auch den Prozess des Schreibens einer Flash-Zelle zu verstehen, ein
Modell daf"ur besprechen {\em Roger Billeter} und {\em Gabriel Looser}.

Eine ganze Reihe technischer Anwendungen sind ohne die Quantenmechanik
nicht denkbar.
In vielen F"allen k"onnen die quantenmechanischen Zust"ande nicht oder nur
mit grossem Aufwand quantenmechanisch exakt berechnet werden.
Meistens kann die St"orungstheorie helfen, die Zustands"uberg"ange
zu verstehen.
Illustriert wird dies in den folgenden Arbeiten.
{\em Stefan Steiner} und {\em Pascal Stump} erkl"aren die Funktion eines
Frequenznormals, einer Atomuhr. 
Die Arbeit von
{\em Michael Cerny} und {\em Stefan Schindler} illustriert, wie mit der
St"orungstheorie den Einfluss eines elektrischen Feldes auf ein
Elektron berechnet werden kann.
In vielen F"allen kann man Systeme in erster N"aherung als harmonische
Oszillatoren verstehen. {\em Joel Brunner} und {\em Christian Cavegn} berichten,
wie man mit der St"orungstheorie die ver"anderten Energieniveaus und
Wellenfunktionen berechnen kann,
wenn die Annahme der Harmonizit"at nicht mehr zutrifft.

Die Berechnung der Energieniveaus des Wasserstoffatoms hat als
Nebenprodukt die Kugelfunktionen geliefert.
Daraus l"asst sich eine Analysemethode f"ur Funktionen auf einer
Kugeloberfl"ache ableiten.
Sie hat viele technische Anwendungen, zum Beispiel die Analyse der
Abstrahlcharakteristik, die Analyse des Gravitationsfeldes des
Mondes, die Schwingungen der Sonne oder die Inhomegenit"aten des
kosmischen Mikrowellenhintergrundes.
{\em Thomas Gujer} und {\em Christoph Schmitz-Dr"ager} beschreiben
dieses Verfahren.

Die Quantenmechanik hat technische Errungenschaften erm"oglicht, die
aus dem modernen Alltag nicht mehr wegzudenken sind.
Albert Einstein hat schon 1916 die Grundlagen f"ur Laser gelegt,
doch erst in den 50er-Jahren konnte das Prinzip umgesetzt werden.
{\em Arwed Schudel} und {\em Claudio Stucki} zeigen, wie ein Laser funktioniert.
Unter den bildgebenden medizinischen Verfahren ist MRI die eindr"ucklichste
Anwendung quantenmechanischer Prinzipien.
{\em Andreas Linggi}, {\em Daniel Monti} und {\em Nicol\'as Rom\'an L"uthold}
nehmen die MRI-Grundlagen unter die Lupe.

Eine der "uberraschendsten Entdeckungen des 20.~Jahrhunderts ist das
v"ollig Wegfallen des elektrischen Widerstands bei sehr tiefen 
Temperaturen, die Supraleitung.
{\em Simon Kuster} und {\em Nicola Ochsenbein} beschreiben, wie sich
Elektronen zu Cooper-Paaren zusammenlagern k"onnen, die
den elektrischen Widerstand reduzieren k"onnen.
Supraleitung tritt nur bei tiefer Temperatur auf.
Der "Ubergang zum supraleitenden Zustand ist ebenfalls ein Quantenph"anomen.
{\em Reto Christen} und {\em Daniel Gubser} behandeln die Bose-Einstein
Kondensation, welche den Phasen"ubergang verst"andlich macht.

Gleichzeitig mit dem Seminar f"ur Bachelor-Studierende fand auch ein
Seminar auf Master-Stufe statt.
Hier wurden sowohl die mathematischen wie auch die physikalischen Grundlagen
vertiefter behandelt, einzelne Abschnitte und ganze Kapitel des ersten
Teiles wurden nur im Master-Seminar im Detail besprochen.
Zum Beispiel hat sich {\em Dorian Amiet} ausf"uhrlich
"uber den Zusammenhang zwischen Fourier-Transformation und 
Unsch"arfe-Relationen Gedanken gemacht.
Eher in die philosophischen Grundlagenfragen ist {\em Hannes Badertscher}
getaucht. Er beschreibt das Einstein-Podolsky-Rosen-Paradoxon und seine
unerwartete Aufl"osung durch die Bellsche Ungleichung und ihre experimentellen
Best"atigungen.
Mit der Problematik der Quanteneigenschaften von Feldern befasst sich
{\em Hannes Diethelm}. 
Die Quantisierung des Strahlungsfeldes liefert die Grundlagen f"ur die
Berechnung der Einsteinschen $A$- und $B$-Koeffizienten, die in 
Kapitel~\ref{chapter:laser} ben"otigt wurden.

Auf diese Weise f"uhren uns die Betr"age der Teilnehmer von den Grundlagen
der Quantenmechanik, wie sie im Skriptteil dargestellt wurden,
"uber das mindestens modellhafte Verst"andnis interessanter technischer
Anwendungen zur"uck zu dem Punkt, an dem die Quantenmechanik ihren
Anfang genommen hat, n"amlich beim Verst"andnis der Wechselwirkung
zwischen Strahlung und Materie. Sowohl Plancks Strahlungsgesetz wie
auch Einsteins Erkl"arung des Photoeffektes beantworteten eine Fragestellung
"uber das Strahlungsfeld mit quantenmechanischen Ideen.
Dieses Buch darf also durchaus den Anspruch erheben, das Verhalten kleinster
Teilchen ans Licht gebracht zu haben.

