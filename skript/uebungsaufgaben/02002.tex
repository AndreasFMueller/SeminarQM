Beschreiben Sie die folgende Variante des Doppelspaltexperiments mit Hilfe 
von Analysatorkreisen.
Wie beim Originalexperiment werden zwei Teilchen auf die zwei Spalten
geschossen, die mit gleicher Wahrscheinlichkeit durch die Spalten fliegen.
Hinter den Spalten betrachten wir einen Sensor in der Mitte des Schirmes,
der konstruktive Interferenz feststellt.
Nun wird der eine Spalt ersetzt durch eine Kavit"at, in der sich die Phase
der Wellenfunktion eines Teilchens schneller "andert, als wenn es durch
den anderen Spalt fliegt\footnote{Elektronen, die eine eine Zone tieferen
Potentials durchqueren, erfahren genau so eine Phasenverschiebung in
ihrere Wellenfunktion.}
Die totale Phasenverschiebung ist einstellbar, sie ist $e^{i\delta}$.
Was beobachtet man im Sensor, wenn man $\delta$ ver"andert.

\begin{loesung}
Aus der Quelle kommen Teilchen im Zustand
\[
\frac1{\sqrt{2}}(\,|1\rangle+\,|2\rangle)
=
\frac1{\sqrt{2}}
\begin{pmatrix}
1\\1
\end{pmatrix}
\]
die am Spalt modifiziert werden mit der Matrix
\[
\begin{pmatrix}
1&0\\
0&e^{i\delta}
\end{pmatrix},
\]
bevor sie im Sensor $\langle S|$ "uberlagert werden.
Zusammengesetzt:
\[
\langle S|\,
\begin{pmatrix}
1&0\\
0&e^{i\delta}
\end{pmatrix}
\frac1{\sqrt{2}}
\begin{pmatrix}
1\\1
\end{pmatrix}
=
\frac1{\sqrt{2}}(
\langle S|1\rangle + e^{i\delta}\langle S|2\rangle
)
\]
Die Zahlen
$\langle S|l\rangle$
beschreiben die Wahrscheinlichkeit, dass ein Teilchen, welches durch
Spalt $l$ fliegt, im Detektor festgestellt wird.
Auf Grund der Annahmen der Aufgabe k"onnen wir davon ausgehen, dass
sie gleich, wir nennen sie $a$. So erhalten wir
\[
\frac{a}{\sqrt{2}}(1+e^{i\delta})
\]
Der Betrag des Klammerausdruckes ist
\begin{align*}
|1+e^{i\delta}|^2
&=2(1-\cos(\pi-\delta))
=4\sin^2\frac{\pi-\delta}2
\\
|1+e^{i\delta}|
&=
2\sin\frac{\pi-\delta}2
\end{align*}
Insbesondere wird bei einer Phasenverschiebung von $\delta=\pi$ 
destruktive Interferenz eintreten, und man wird im Sensor nichts
mehr detektieren.
\end{loesung}

