Berechnen Sie die Entwicklung der Eigenzust"ande des Zweizustandssystems
mit zeitabh"angigem Hamilton-Operator
\[
H(t)
=
\begin{pmatrix}
E_1+\varepsilon\cos\omega t & 0                          \\
0                           & E_2-\varepsilon\cos\omega t
\end{pmatrix}
\]
f"ur $\varepsilon\ll E_i$.

\begin{loesung}
Die Schr"odingergleichung f"ur den zeitabh"angigen Zustandsvektor
\[
|\psi(t)\rangle
=
\begin{pmatrix}
v_1(t) \\
v_2(t)
\end{pmatrix}
\]
ist das Differentialgleichungssystem
\begin{align*}
i\hbar\dot v_1&=(E_1+\varepsilon\cos\omega t)v_1,\\
i\hbar\dot v_2&=(E_2-\varepsilon\cos\omega t)v_2.
\end{align*}
Durch Separieren
\[
\int\frac{dv_1}{v_1}
=
\frac{1}{i\hbar}\int E_1 + \varepsilon\cos\omega t\,dt
=
\frac1{i\hbar}\biggl(E_1t+\frac{\varepsilon}{\omega}\sin\omega t\biggr)
\]
kann man die L"osungen
\begin{align*}
v_1(t)
&=
v_1(0)e^{\frac1{i\hbar}(E_1t+\frac{\varepsilon}{\omega}\sin\omega t)}
,\qquad\text{und}
\\
v_2(t)
&=
v_2(0)e^{\frac1{i\hbar}(E_2t-\frac{\varepsilon}{\omega}\sin\omega t)}
\end{align*}
erraten, was man auch durch Einsetzen in die Schr"odingergleichung
nachpr"ufen kann.
\end{loesung}

