Betrachten Sie den Raum der Funktionen
\[
L^2(\mathbb R)=\left\{
f:\mathbb R\to\mathbb C\,\left|
\text{$f$ stetig und }
\int_{-\infty}^\infty |f(x)|^2\,dx<\infty
\right.
\right\}.
\]
Wir staten diesen Raum mit dem Skalarprodukt
\[
(f,g)=\int_{-\infty}^\infty \overline{f(x)}g(x)\,dx.
\]
Zeigen Sie, dass $L^2(\mathbb R)$ ein Pr"ahilbertraum ist.

\begin{loesung}
Wir m"ussen zeigen dass das Skalarprodukt sesquilinear, hermitesch und positiv
definit ist.
Dazu rechnen wir nach:
\begin{align*}
(\lambda f,g)
&=
\int_{-\infty}^\infty
\overline{\lambda f(x)}g(x)
\,dx
=
\int_{-\infty}^\infty
\bar \lambda \bar f(x)
g(x)
\,dx
=
\bar \lambda
\int_{-\infty}^\infty
\bar f(x) g(x)
\,dx
=\bar\lambda(f,g).
\\
(f,\lambda g)
&=
\int_{-\infty}^\infty
\overline{f(x)}\lambda g(x)
\,dx
=
\lambda
\int_{-\infty}^\infty
\overline{f(x)}g(x)
\,dx
=
\lambda(f,g).
\end{align*}
Das Skalarprodukt ist also tats"achlich sesquilinear.

F"ur die Hermitezit"at berechnen wir
\begin{align*}
(f,g)
&=
\int_{-\infty}^{\infty} \overline{f(x)} g(x)\,dx
=
\overline{
\int_{-\infty}^{\infty} f(x) \overline{g(x)}\,dx
}
=
\overline{
\int_{-\infty}^{\infty}
\overline{g(x)}
f(x)
\,dx
}
=
\overline{(g,f)},
\end{align*}
also ist das skalarprodukt hermitesch.

Sei jetzt $f$ eine beliebige stetige Funktion, die nicht die Nullfunktion
ist. Zu einem gen"ugend kleinen $\varepsilon>0$ gibt es daher ein kleines
Interval $I\subset\mathbb R$ mit $|f(x)|>\varepsilon$ f"ur alle $x\in I$.
Dann folgt
\[
(f,f)
=
\int_{-\infty}^\infty
|f(x)|^2
\,dx
\ge
\int_I|f(x)|^2\,dx
=
\int_I\varepsilon\,dx
=|I|\varepsilon >0.
\]
Damit ist bewiesen, dass das Skalarprodukt positiv definit ist.
\end{loesung}

