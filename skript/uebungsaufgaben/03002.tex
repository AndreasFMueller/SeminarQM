Zeigen Sie, dass die Funktionen $e_k$ mit
$e_k(x)=e^{ikx}$ eine Hilbertbasis von $L^2([-\pi,\pi])$
sind.

\begin{loesung}
Wir m"ussen zeigen, dass die $e_k$ orthonormiert sind
\begin{align*}
(e_k,e_k)
&=
\frac1{2\pi}\int_{-\pi}^{\pi}\bar e_k(x)e_k(x)\,dx
=
\frac1{2\pi}\int_{-\pi}^{\pi}
e^{-ikx}e^{ikx}
\,dx
=
\frac1{2\pi}\int_{-\pi}^{\pi}\,dx=1,
\\
(e_k,e_l)
&=
\frac1{2\pi}\int_{-\pi}^{\pi} e^{-ikx}e^{ilx}\,dx
=
\frac1{2\pi}\int_{-\pi}^{\pi} e^{i(l-k)x}\,dx
=
\frac1{2\pi}\left[
\frac1{i(l-k)}e^{i(l-k)x}
\right]_{-\pi}^{\pi}
=0\qquad \text{f"ur $k\ne l$}
\end{align*}
Die Fouriertheorie sagt, dass jede $L^2$-Funktion auf dem Interval
$[-\pi,\pi]$ mit Ihrer Fourierreihe approximiert werden kann, also
sind bilden die Funktionen $e_k$ eine Hilbertbasis.
\end{loesung}

