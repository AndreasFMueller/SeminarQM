Betrachte die Menge
\[
c_0=\left\{
(a_k)_{k\in\mathbb N}
\,|\,
\text{es gibt eine Zahl $n$ so, dass $a_k=0$ f"ur $k>n$}
\right\}
\subset l^2.
\]
$c_0$ besteht aus Folgen, die nur f"ur endlich viele Folgenglieder
von $0$ verschieden sind.
\begin{teilaufgaben}
\item Zeigen Sie, dass $c_0$ ein Pr"ahilbertraum ist mit dem gleichen
Skalarprodukt wie $l^2$.
\item Zeigen Sie, dass die Folgen $e^{(k)}$, die nur ein von $0$
verschiedenes Folgenglied 
$e^{(k)}_l=\delta_{kl}$
haben, eine Hilbertbasis ist.
\end{teilaufgaben}
Dies zeigt, dass $l^2$ die Vervollst"andigung von $c_0$ ist.

\begin{loesung}
\begin{teilaufgaben}
\item Das Skalarprodukt in $c_0$  hat sicher die n"otigen Eigenschaften,
es ja dasselbe ist wie in $l^2$. Es bleibt also nur noch zu "uberlegen,
dass die Summe von zwei Vektoren aus $c_0$ wieder in $c_0$ ist.
Wenn die Folge $a_k$ ab Index $n_a$ Null ist, und die Folge $b_k$ ab
Index $n_b$, dann ist die Folge $a_k+b_k$ ab Index $\operatorname{max}(a,b)$
Null, also ist $(a_k+b_k)_{k\in\mathbb N}$ eine Folge in $c_0$.
\item Die Vektoren $e_k$ sind einerseits in $c_0$, und andererseits
sind sie schon in $l^2$ eine Hilbertbasis, also erst recht eine Hilbertbasis
in $c_0$.
\end{teilaufgaben}
\end{loesung}

