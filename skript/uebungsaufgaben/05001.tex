Betrachten Sie die Lagrange-Funktion eines eindimensionalen Teilchens
\[
\frac12m\dot q^2 + A(t,q)\dot q.
\]
\begin{teilaufgaben}
\item
Leiten Sie die zugeh"origen Bewegungsgleichungen ab.
\item
Stellen Sie die Hamilton-Funktion auf.
\item
Leiten Sie die Hamiltonschen Bewegungsgleichungen ab.
\end{teilaufgaben}


\begin{loesung}
\begin{teilaufgaben}
\item
Die Bewegungsgleichungen k"onnen aus den Euler-Gleichungen abgeleitet
werden:
\begin{align}
\frac{\partial L}{\partial q}&=\dot q \frac{\partial A}{\partial q}
\notag
\\
\frac{\partial L}{\partial \dot q}&=m\dot q+ A(t,q)
\label{05001:impuls}
\\
\frac{d}{dt}\frac{\partial L}{\partial\dot q}
&=
m\ddot q+\frac{\partial A}{\partial t}+\frac{\partial A}{\partial q}\dot q
\label{05001:derivative}
\\
\frac{d}{dt}\frac{\partial L}{\partial\dot q}-\frac{\partial L}{\partial q}
&=
m\ddot q+\frac{\partial A}{\partial t}
=0
\notag
\end{align}
Aus der letzten Gleichung lesen wir die Bewegungsgleichung in Newtonscher
Form ab:
\begin{align*}
m\ddot q&=-\frac{\partial A}{\partial t}.
\end{align*}
\item
Aus (\ref{05001:impuls}) lesen wir ab, dass wir als verallgemeinerte
Impuls-Koordinate f"ur die Hamilton-Gleichungen die Gr"osse
$P=m\dot q+A=p+A$ w"ahlen m"ussen.
Die Hamilton-Funktion muss jetzt die Energie in $q$ und dieser neuen
Koordinaten $P$ ausdr"ucken, man muss also jedes vorkommen von $p$
im Ausdruck f"ur die Energie durch $P-A$ ersetzen.
Die Hamilton-Funktion wird damit zu
\[
H(P,q)=\frac1{2m}(P-A(t,q))^2.
\]
\item
Die Hamiltonschen Bewegungsgleichungen sind
\begin{align*}
\frac{dq}{dt}
&=
\frac{\partial H}{\partial P}
=
\frac1m(P-A(t,q))
\\
\frac{dP}{dt}
&=
-\frac{\partial H}{\partial q}
=
+\frac1m(P-A(t,q))\frac{\partial A}{\partial q}
=
\dot q\frac{\partial A}{\partial q}
\end{align*}
In der zweiten Gleichung verschwindet das Vorzeichen wegen des zus"atzlichen
Vorzeichens von der inneren Ableitung.
Setzen wir in der zweiten Gleichung wieder $P=m\dot q+A$ ein, erhalten wir
f"ur die linke Seite
\begin{align*}
\frac{dP}{dt}
&=
\frac{d}{dt}(m\dot q+A(t,q))
=
m\ddot q+\frac{\partial A}{\partial t} + \frac{\partial A}{\partial q}\dot q
\end{align*}
F"ur die rechte Seite erhalten wir
\[
-\frac1m(P-A(t,q))\frac{\partial A}{\partial q}
=
-\dot q\frac{\partial A}{\partial q}.
\]
Zusammen erhalten wir die Bewegungsgleichungen
\begin{align*}
m\ddot q + \frac{\partial A}{\partial t}+\frac{\partial A}{\partial q}\dot q
&=
\frac{\partial A}{\partial q}\dot q
\\
m\ddot q
&=
-\frac{\partial A}{\partial t}.
\end{align*}
\end{teilaufgaben}
\end{loesung}

\begin{diskussion}
Wenn $A$ nicht von der Zeit abh"angt, ist die Energie erhalten.
Also leistet der Term $A$ keine Arbeit.
Die Wirkung von $A$ hat also "ahnliche Auswirkungen wie ein Magnetfeld
auf eine geladenes Teilchen,
denn die Lorentz steht immer senkrecht auf der Bewegungsrichtung und
leistet daher ebenfalls keine Arbeit.
Tats"achlich w"urden in einer dreidimensionalen Version dieses Problems
im Ausdruck (\ref{05001:derivative}) zus"atzliche Terme auftreten,
die in der Bewegungsgleichung die Lorentzkraft ergeben w"urden.
Dies wird im Abschnit~\ref{section:hamilton-funktion-im-magnetfeld}
voref"uhrt.
\end{diskussion}
 
