Rechnen Sie die Jacobi-Identit"at f"ur die Poisson-Klammern nach:
\begin{equation}
(F,(G,H))+ (G,(H,F))+ (H,(F,G))=0.
\label{skript:jacobipoisson}
\end{equation}

\begin{loesung}
Wir setzen die Definition f"ur die erste Poisson-Klammer in
(\ref{skript:jacobipoisson}) ein:
\begin{align*}
(F,(G,H))
&=
\sum_{k=1}^n\biggl(
\frac{\partial F}{\partial q_k} \frac{\partial (G,H)}{\partial p_k}
-
\frac{\partial (G,H)}{\partial q_k} \frac{\partial F}{\partial p_k}
\biggr)
\\
&=
\sum_{k,l=1}^n\biggl(
\frac{\partial F}{\partial q_k}
\frac{\partial}{\partial p_k}\biggl(
\frac{\partial G}{\partial q_l}\frac{\partial H}{\partial p_l}
-
\frac{\partial H}{\partial q_l}\frac{\partial G}{\partial p_l}
\biggr)
-
\frac{\partial}{\partial q_k}\biggl(
\frac{\partial G}{\partial q_l}\frac{\partial H}{\partial p_l}
-
\frac{\partial H}{\partial q_l}\frac{\partial G}{\partial p_l}
\biggr)
\frac{\partial F}{\partial p_k}
\biggr)
\\
&=
\sum_{k,l=1}^n\biggl(
\frac{\partial F}{\partial q_k}
\biggl(
\frac{\partial^2 G}{\partial q_l\partial p_k}\frac{\partial H}{\partial p_l}
+
\frac{\partial G}{\partial q_l}\frac{\partial^2 H}{\partial p_l\partial p_k}
-
\frac{\partial^2 H}{\partial q_l\partial p_k}\frac{\partial G}{\partial p_l}
-
\frac{\partial H}{\partial q_l}\frac{\partial^2 G}{\partial p_l\partial p_k}
\biggr)
\\
&\qquad\qquad
-
\biggl(
\frac{\partial^2 G}{\partial q_l\partial q_k}\frac{\partial H}{\partial p_l}
+
\frac{\partial G}{\partial q_l}\frac{\partial^2 H}{\partial p_l\partial q_k}
-
\frac{\partial^2 H}{\partial q_l\partial q_k}\frac{\partial G}{\partial p_l}
-
\frac{\partial H}{\partial q_l}\frac{\partial^2 G}{\partial p_l\partial q_k}
\biggr)
\frac{\partial F}{\partial p_k}
\biggr)
\end{align*}
Die volle Jacobi-Identit"at kann aus diesem Ausdruck durch zyklische
Vertauschung von $F$, $G$ und $H$ und Addition gewonnen werden.
Dabei entsteht eine grosse Zahl von Termen.
Um die "Ubersicht zu behalten, sammeln wir nur die zweiten Ableitungen
von $G$,
und unterscheiden ausserdem nach den verschiedenen Variablen:
\begin{align*}
\frac{\partial^2 G}{\partial q_k\partial q_l}:&
&&
\sum_{k,l=1}^n
\biggl(
-\frac{\partial F}{\partial p_k} \frac{\partial H}{\partial p_l}
+\frac{\partial F}{\partial p_l} \frac{\partial H}{\partial p_k}
\biggr)
\frac{\partial^2 G}{\partial q_k\partial q_l}
\\
\frac{\partial^2 G}{\partial p_k\partial q_l}:&
&&
\sum_{k,l=1}^n
\biggl(
\frac{\partial F}{\partial q_k}\frac{\partial H}{\partial p_l}
+
\frac{\partial F}{\partial p_l}\frac{\partial H}{\partial q_k}
-
\frac{\partial F}{\partial q_k}\frac{\partial H}{\partial p_l}
-
\frac{\partial F}{\partial p_l}\frac{\partial H}{\partial q_k}
\biggr)
\frac{\partial ^2G}{\partial p_k\partial q_l}
\\
\frac{\partial^2 G}{\partial p_k\partial p_l}:&
&&
\sum_{k,l=1}^n
\biggl(
-\frac{\partial F}{\partial q_k}\frac{\partial H}{\partial q_l}
+
\frac{\partial F}{\partial q_l}\frac{\partial H}{\partial q_k}
\biggr)
\frac{\partial^2 G}{\partial p_k\partial p_l}
\end{align*}
Jeder dieser Terme verschwindet, also verschwindet auch der gesamte
Ausdruck, was die Jacobi-Identit"at beweist.
\end{loesung}

