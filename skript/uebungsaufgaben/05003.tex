In dieser Aufgabe finden Sie die Bewegungsgleichungen eines Pendels.
Ein Pendel ist eine Masse $m$, die an einem Faden der L"ange $l$ 
aufgeh"angt ist, 

\begin{teilaufgaben}
\item Finden Sie die Lagrange-Funktion $L(\varphi)$.
\item Was muss als zu $\varphi$ geh"origer Impuls $p_\varphi$ verwendet werden?
\item Finden Sie die Hamilton-Funktion $H(\varphi, p_\varphi)$.
\item Stellen Sie die Bewegungsgleichungen auf.
\end{teilaufgaben}

\begin{loesung}
\begin{teilaufgaben}
\item Die Lagrange-Funktion ist die Differenz von kinetischer und
potentieller Energie:
\[
T
=
\frac12mv^2-mgh
=
\frac12m(l\dot\varphi)^2-mlg(1-\cos\varphi).
\]
\item
Als konjugierter Impuls muss die Ableitung von $L$ nach $\dot\varphi$
verwendet werden:
\[
p_\varphi
=
\frac{\partial L}{\partial\dot\varphi}
=
ml^2\dot\varphi.
\]
Man beachte, dass $ml^2$ das Tr"agheitsmoment der Masse $m$ um den
Aufh"angepunkt des Pendels ist, $p_\varphi$ ist also der Drehimpuls.
Wir dr"ucken $\dot\varphi$ noch durch den Drehimpuls aus:
\[
\dot\varphi
=
\frac{1}{ml^2}p_{\varphi}.
\]
\item
Die Hamilton-Funktion kann mit Hilfe der
Formel~(\ref{skript:von-lagrange-zu-hamilton})
bestimmt werden:
\begin{align*}
H(\varphi,p_\varphi)
&=
p_\varphi\dot\varphi-L(\varphi,\dot\varphi)
=
p_\varphi\dot\varphi - \frac12 m(l\dot\varphi)^2+mlg(1-\cos\varphi)
\\
&=
p_\varphi\frac{1}{ml^2}p_\varphi-\frac12m\frac{1}{m^2l^2}p_\varphi^2
+mlg(1-\cos\varphi)
\\
&=
\frac1{2ml^2}p_\varphi^2 + mlg(1-\cos\varphi).
\end{align*}
\item
Die Bewegungsgleichungen in Hamiltonscher Form sind
\begin{align*}
\frac{d\varphi}{dt}
&=
\frac{\partial H}{\partial p_\varphi}
=
\frac{1}{ml^2}p_\varphi,
\\
\frac{dp_\varphi}{dt}
&=
-\frac{\partial H}{\partial\varphi}
=
-mlg\sin\varphi.
\end{align*}
Die erste Gleichung ist nat"urlich identisch mit dem in b) gefundenen
Zusammenhang zwischen $p_\varphi$ und $\dot\varphi$.
Setzen wir diesen Zusammenhang in die zweite Gleichung ein, erhalten
wir
\begin{align*}
\frac{d}{dt}p_\varphi
&=
\frac{d}{dt}(ml^2\dot\varphi)
=
ml^2\ddot\varphi
=
-mlg\sin\varphi
\\
\Leftrightarrow
\qquad
\ddot\varphi
&=
-\frac{g}{l}\sin\varphi.
\end{align*}
Dies ist die bekannte Bewegungsgleichung f"ur ein Pendel. 
F"ur kleine Amplituden kann man die rechte Seite approximieren durch
$\sin\varphi\simeq\varphi$, und erh"alt eine Schwingungsdifferentialgleichung
mit Kreisfrequenz $\sqrt{g/l}$.
\end{teilaufgaben}
\end{loesung}

