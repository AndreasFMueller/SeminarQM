Quantisieren Sie das mechanische System mit der klassischen Hamilton-Funktion
\[
H(x,p)=\frac{1}{2m}p^2+ax^4.
\]
Berechnen Sie auch den Kommutator von $x$ mit dem Hamilton-Operator in
der Ortsdarstellung.

\begin{loesung}
Nach den Quantisierungsregeln in
Satz~\ref{skript:quantisierungsregeln-ortsdarstellung}
muss die Variable $p$ durch den Differentialoperator
\[
\hat p=\frac{\hbar}{i}\frac{\partial}{\partial x}
\]
ersetzt werden.
Man erh"alt
\[
\hat H
=
-\frac{\hbar^2}{2m}\frac{\partial^2}{\partial x^2}+ax^4
\]
als Hamilton-Operator.
Den Kommutator $[x,\hat H]$ in der Ortsdarstellung ermitteln wir durch 
seine Wirkung auf eine Wellenfunktion $\psi(x)$:
\begin{align*}
[x,\hat H]\psi(x)
&=
x\biggl(
-\frac{\hbar^2}{2m}\frac{\partial^2}{\partial x^2}+ax^4
\biggr)\psi(x)
-
\biggl(
-\frac{\hbar^2}{2m}\frac{\partial^2}{\partial x^2}+ax^4
\biggr)x\psi(x)
\\
&=
-\frac{\hbar^2}{2m}\biggl(
x\psi''(x)
-
x\psi''(x)
-
2\psi'(x)
\biggr)
=
\frac{\hbar^2}{m}\psi'(x)
=
-\frac{\hbar}{i}\frac1{m}\frac{\hbar}{i}\frac{\partial}{\partial x}\psi(x)
=
-\frac{\hbar}{i}\frac1{m}\hat p\psi(x).
\end{align*}
Zu Beginn der zweiten Zeile brauchen wir die Beziehung
\[
\frac{\partial^2}{\partial x^2}(x\psi(x))
=
\frac{\partial}{\partial x}\biggl(
\psi(x)+x\psi'(x)
\biggr)
=
\psi'(x)+\psi'(x)+x\psi''(x)
=
x\psi''(x)+2\psi'(x).
\]
Die Quantisierungsregeln aus Satz~\ref{skript:quantisierung-poisson}
besagen, dass Kommutatoren mit Poisson-Klammern in Beziehung stehen:
\begin{align*}
-\frac{i}{\hbar}
[x,\hat H]
&=
\frac1{m}\hat p.
\end{align*}
Der Operator auf der rechten Seite ist der Operator f"ur die
Geschwindigkeit,
und tats"achlich muss der Kommutator mit $\hat H$ ja die Zeitwentwicklung
der Observablen $x$ ausdr"ucken.
\end{loesung}

