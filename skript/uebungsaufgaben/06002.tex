Man berechne die Energieniveaus eines Teilchens der Masse $m$ in einem
Potential $V(x)=v|x|$.

\begin{loesung}
Die Quantisierungsregeln f"uhren auf die Schr"odingergleichung
\[
-\frac{\hbar^2}{2m}\psi''(x)+v|x|\psi(x)=E\psi(x).
\]
Wir k"onnen die L"osung $\psi(x)$ f"ur die ganze relle Achse gewinnen,
indem wir eine L"osung f"ur positive $x$ gerade oder ungerade
auf die negative reelle Achse fortsetzen. 
Ersteres funktioniert nur, wenn $\psi'(0)=0$ ist, zweites nur
wenn $\psi(0)=0$.

Wenn wir aber nur L"osungen f"ur $x>0$ suchen, dann k"onnen wir
auf das Betragszeichen verzichten.
Wir suchen also L"osungen der Differentialgleichung
\begin{equation}
-\frac{\hbar^2}{2m}\psi''(x)+vx \psi(x)=E\psi(x),
\label{06002:dgl}
\end{equation}
die f"ur $x\to\infty$ beschr"ankt bleiben und die oder deren Ableitung
bei $x=0$ verschwindet.

Die Differentialgleichung kann durch Verschiebung des Arguments $x$ 
und durch Skalierung in die Form
\[
y''-xy=0
\]
gebracht, werden, die Airysche Differentialgleichungen. 
Ihre L"osungen sind Linearkombinationen der Airy-Funktionen
$\operatorname{Ai}(x)$ und $\operatorname{Bi}(x)$ (siehe zum Beispiel
\cite{skript:airy})..
Da $\operatorname{Bi}(x)$ f"ur $x\to\infty$ unbeschr"ankt
w"achst, darf $\operatorname{Bi}(x)$ in der L"osung nicht
vorkommen.
Daraus folgt, dass die L"osung bis auf die Normierung von der
Form
\[
\psi(x)=\operatorname{Ai}(ax-b)
\]
sein muss.
Dies setzen wir jetzt in die Differentialgleichung~(\ref{06002:dgl}) ein,
und erhalten die Gleichung
\begin{align*}
-\frac{\hbar^2}{2m}a^2\operatorname{Ai}''(ax-b)
+
vx\operatorname{Ai}(ax-b)
&=
E\operatorname{Ai}(ax-b)
\\
-\frac{\hbar^2}{2m}a^2(ax-b)\operatorname{Ai}(ax-b)
+
vx\operatorname{Ai}(ax-b)
&=
E\operatorname{Ai}(ax-b)
\\
-\frac{\hbar^2}{2m}a^2(ax-b)
+
vx
&=
E
\end{align*}
Beide Seiten dieser Gleichung sind Polynome, durch Koeffizientenvergleich
finden wir daher die Bedingungen f"ur $a$ und $b$
\begin{align*}
-\frac{\hbar^2}{2m}a^3+v&=0
&
-\frac{\hbar^2}{2m}a^2b&=E
\\
\Rightarrow\qquad
a &= \biggl(\frac{\hbar^2}{2mv}\biggr)^{-\frac13}
&
\Rightarrow\qquad
b &= -\frac{E}{v}\biggl(\frac{\hbar^2}{2mv}\biggr)^{1-\frac23}
= -\frac{E}{v}\biggl(\frac{\hbar^2}{2mv}\biggr)^{-\frac13}
\end{align*}
Damit die L"osung f"ur $x>0$ sich zu einer L"osung f"ur alle $x$
erweitern l"asst, muss $\psi(0)=0$ oder $\psi'(0)=0$ sein,
es muss also $b$ eine Nullstelle von $\operatorname{Ai}(x)$ oder
von $\operatorname{Ai}'(x)$ sein.
Sind $x_n$ die Nullstellen von $\operatorname{Ai}(x)$ und
$\operatorname{Ai}'(x)$ mit $x_{n+1}<x_n$, dann sind die zugeh"origen
Energieniveaus:
\[
E_n=v\root 3\of{\frac{\hbar^2}{2mv}} x_n.
\]
Auch die Wellenfunktionen kann man angeben.
F"ur ungerade $n$ ist $x_n$ eine Nullstelle von $\operatorname{Ai}'(x)$,
die Wellenfunktion ist ungerade:
\begin{equation}
\psi(x)
=
\begin{cases}
\displaystyle
\operatorname{Ai}\biggl( \root3\of{\frac{2mv}{\hbar^2}} (x-x_n)\biggr)
	&\qquad x\ge 0\\
\displaystyle
-\operatorname{Ai}\biggl( \root3\of{\frac{2mv}{\hbar^2}}(x-x_n)\biggr)
	&\qquad x < 0
\end{cases}
\end{equation}
F"ur gerade $n$ ist $x_n$ eine Nullstelle von $\operatorname{Ai}(x)$
und die Wellenfunktion ist gerade
\begin{equation}
\psi(x)
=
\operatorname{Ai}\biggl( \root3\of{\frac{2mv}{\hbar^2}}(|x|-x_n) \biggr).
\end{equation}
\end{loesung}

