Betrachten Sie die beiden Teilchen im Einstein-Podolsky-Rosen Experiment
mit Ortsoperatoren $X_i$ und Impulsoperatorn $P_i$ mit $i=1,2$.
F"ur diese Operatoren gelten die Vertauschungsrelationen
$[X_i,P_i]=-i\hbar\operatorname{id}$, und wir k"onnen nicht
den Ort und den Impuls eines der Teilchens gleichzeig wissen.
Mit der experimentellen Vorschrift von Einstein, Podolsky und Rosen
bestimmt man nicht $X_1$ und $X_2$, sondern
die Differenz $X=X_1-X_2$, denn man kann nicht die Position des
Ausgangssystems messen, ohne es zu ver"andern. Ebenso kann man nicht
den Impuls eines Teilchens, sondern nur den Gesamtimpuls $P=P_1-P_2$ 
bestimmen. Zeigen Sie, dass man diese beiden Gr"ossen tats"achlich
beide beliebig genau wissen kann, dass das klassische Einsten-Podolsky-Rosen
Experiment also keinen Widerspruch zur Unsch"arferelation darstellt.

\begin{loesung}
Man muss den Kommutator $[X,P]$ bestimmen:
\begin{align*}
[X,P]
&=
[X_1-X_2,P_1-P_2]
=
\underbrace{[X_1,P_1]}_{-i\hbar\operatorname{id}}
-\underbrace{[X_2,P_1]}_{=0}
-\underbrace{[X_1,P_2]}_{=0}
-\underbrace{[X_2,P_2]}_{-i\hbar\operatorname{id}}
=
-i\hbar\operatorname{id}
+i\hbar\operatorname{id}
=0
\end{align*}
Die beiden Operatoren vertauschen, also k"onnen sie tats"achlich gleichzeit
beliebig genau bestimmt sein.
\end{loesung}


