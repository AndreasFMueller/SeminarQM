Kann man Impuls und Energie eines Teilchens beliebig genau wissen?
Wenn ja, formulieren Sie eine Unsch"arferelation.

\begin{loesung}
Man kann Impuls und Energie eines Teilchens genau dann bleibig genau
wissen, wenn die zugeh"origen Operatoren $P$ und $H$ vertauschen.
Also berechnen wir den Kommutator:
\begin{align*}
[P,H]
&=
\biggl[P,\frac1{2m}P^2+V\biggr]
=
[P,V].
\end{align*}
Wir k"onnen daraus schon mal ablesen, dass f"ur ein freies Teilchen,
also f"ur $V=0$, die beiden Operatoren tats"achlich vertauschbar sind,
wir k"onnen also tats"achlich Impuls und Energie gleichzeitig 
beliebig genau kennen.

Wenn $V\ne 0$ ist, m"ussen wir den Kommutator von $P$ mit $V$
bestimmen:
\begin{align*}
[P,V]
&=
\biggl[\frac{\hbar}{i}\frac{\partial}{\partial x}, V\biggr]
=
\frac{\hbar}{i}
\biggl(
\frac{\partial}{\partial x}
V
-
V
\frac{\partial}{\partial x}
\biggr)
=
\frac{\hbar}{i}
\biggl(
\frac{\partial V}{\partial x}
+
V
\frac{\partial}{\partial x}
-
V
\frac{\partial}{\partial x}
\biggr)
=
\frac{\hbar}{i}
\frac{\partial V}{\partial x}.
\end{align*}
Die zugeh"orige Unsch"arferelation ist
\[
\Delta p\cdot \Delta E
=
\frac{\hbar}{2m}\biggl\langle \frac{\partial V}{\partial x}\biggr\rangle.
\]
F"ur schwere Teilchen, also zum Beispiel f"ur makroskopische K"orper ist
die Unsch"arfe nicht wahrnehmbar.
Die Unsch"arfe ist also umso gr"osser, je gr"osser die Ableitung des
Potentials ist.
Physikalisch kann man dies damit erkl"aren, dass es bei bekanntem Impuls
keine scharfe Position eines Teilchens gibt, die Energie wird also an
verschiedenen Orten gemessen, wo verschiedene potentielle Energien 
gemessen werden.
Die Ableitung des Potentials ist aber die Kraft, die auf ein Teilchen wirkt.
Die Unsch"arfe ist also umso gr"osser, je gr"osser die Kr"afte sind, die
auf das Teilchen wirken.
\end{loesung}

