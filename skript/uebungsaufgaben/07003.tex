Kann man Energie und Position eines Teilchens beliebig genau wissen?
Wenn nicht, formulieren Sie eine entsprechende Unsch"arferelation.

\begin{loesung}
Die Energie eines Teilchens wird durch die Observable $H$, den
Hamilton-Operator gegeben.
Die Position wird hingegen durch den Operator $X$ der Position
gegeben.
Man kann beide beleibig genau wissen, d.~h.~es gibt gemeinsame
Eigenzust"ande von $H$ und $X$, wenn die beiden Operatoren vertauschen.
Wir berechnen daher den Kommutator von $X$ und $H$:
\begin{align*}
[X,H]
&=
\biggl[X,\frac1{2m}P^2\biggr]
=
\frac1{2m}[X,P^2]
=
\frac1{2m}(XPP-PPX)
=
\frac1{2m}(XPP-PXP+PXP-PPX)
\\
&=
\frac1{2m}([X,P]P+P[X,P])
=
-\frac{\hbar}{im} P
\end{align*}
Die zugeh"orige Unsch"arferelation ist
\[
\Delta E\cdot \Delta p\ge \frac{\hbar}{2}\langle P\rangle.
\]
Die Unsch"arfe h"angt also vom Impuls eines Teilchens ab, je gr"osser
der Impuls, desto gr"osser auch die Unsch"arfe.
\end{loesung}

