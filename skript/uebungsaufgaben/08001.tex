Berechnen Sie die Energieniveaus eines Teilchens der Masse $m$ in einem
eindimensionalen Potential
\[
V(x)=c+bx+ax^2.
\]

\begin{loesung}
Der zugeh"orige Hamilton-Operator ist
\[
H=\frac1{2m}p^2 + c+bx+ax^2.
\]
Dieser Operator sieht dem Operator eines harmonischen Oszillators sehr
"ahnlich, es st"ort jedoch der Term $bx$.
Diesen k"onnen wir aber durch quadratisches Erg"anzen entfernen:
\[
ax^2+bx+c
=
a\biggl(x+\frac{b}{2a}\biggr)^2-\frac{b^2-4ac}{4a}
\]
Schreiben wir jetzt
\[
x_0=-b/2a,\qquad
\varepsilon_0=-\frac{b^2-4ac}{4a}
\qquad\text{und}\qquad
\omega=\sqrt{\frac{2a}{m}},
\]
dann wird der Hamilton-Operator zu
\[
H=\frac1{2m}p^2 +
\frac{m}{2}\omega^2(x-x_0)^2
+\varepsilon_0
\]
Die Schr"odingergleichung f"ur diesen Operator ist
\begin{align*}
\biggl(
\frac1{2m}p^2 +
\frac{m}{2}\omega^2(x-x_0)^2
+\varepsilon_0
\biggr)\psi&=E\psi
\\
\biggl(
\frac1{2m}p^2 +
\frac{m}{2}\omega^2(x-x_0)^2
\biggr)\psi&=(E-\varepsilon_0)\psi
\end{align*}
Die linke Seite ist ein harmonischer Oszillator, dessen Energieniveaus
sind
\[
E-\varepsilon=\hbar\omega\biggl(n+\frac12\biggr)
\qquad\Rightarrow\qquad
E=\varepsilon_0+\hbar\omega\biggl(n+\frac12\biggr)
=\frac{b^2-4ac}{4a}+\hbar\sqrt{\frac{2a}{m}}\biggl(n+\frac12\biggr)
\]
f"ur beliebige nat"urliche Zahlen $n$.
\end{loesung}

