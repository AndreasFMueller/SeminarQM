Betrachten Sie die Hamilton-Funktion f"ur ein zweidimensionales Potential
der Form
\[
H(p, q)=\frac1{2m}(p_x^2+p_y^2)+ax^2+2bxy+cy^2.
\]
Berechnen Sie die Energieniveaus des quantisierten Systems.

\begin{loesung}
W"are der Term $2bxy$ nicht, k"onnte man das System einfach als zwei
unabh"angige harmonische Oszillatoren betrachten, deren Energieniveaus
wir bereits berechnet haben.
Das Ziel muss also sein, mit Hilfe einer Rotation diesen 
Wechselwirkungsterm zu entfernen.

Das Potential kann mit Hilfe einer Matrix geschrieben werden:
\[
ax^2+2bxy+cy^2
=
\begin{pmatrix}x&y\end{pmatrix}
\begin{pmatrix}a&b\\b&c\end{pmatrix}
\begin{pmatrix}x\\y\end{pmatrix}
\]
Durch eine Drehung kann die Matrix in Diagonalform gebracht werden,
auf der Diagonalen stehen dann die Eigenwerte der Matrix.
Da wir uns nur f"ur die Energieniveaus interessieren, und nicht
auch f"ur die Wellenfunktionen, reicht es, die Eigenwerte zu
berechnen.
Diese kann man mit dem 
\begin{align*}
0
&=
\left|\begin{matrix}
a-\lambda&b\\
b&c-\lambda
\end{matrix}\right|
=
(a-\lambda)(c-\lambda)-b^2
=
\lambda^2-(a+c)\lambda+ac-b^2
\\
\lambda_{\pm}
&=
\frac{a+c}2\pm\sqrt{\frac{(a+c)^2}4-ac+b^2}
=
\frac{a+c}2\pm\sqrt{\frac{(a-c)^2}4+b^2}
\end{align*}
Man kann daraus ablesen, dass die Eigenwerte immer reell sind.
F"ur $b=0$ sind die Eigenwerte $a$ und $c$, und es ist nichts
zu unternehmen.

Wir nennen die gedrehten Koordinaten $\xi$ und $\eta$.
In diesen Koordinaten wird die gedrehte Hamilton-Funktion sein
\[
H(p_\xi,p_\eta,\xi,\eta)
=
\frac1{2m}(p_\xi^2+p_\eta^2)+\lambda_+\xi^2+\lambda_-\eta^2
\]
Das quantisierte System hat den Hamilton-Operator
\[
H
=
-\frac{\hbar^2}{2m}\Delta + \lambda_+\xi^2+\lambda_-\eta^2
\]
In diesem Hamilton-Operator gibt es keine Produkte von $\xi$ und $\eta$ mehr.
Wir k"onnen daher versuchen, eine Wellenfunktion als Produkt 
\[
\psi(\xi,\eta)=X(\xi)Y(\eta)
\]
zu finden. Wir setzen diesen Ansatz in die Schr"odingergleichung ein
\begin{align*}
\frac{\hbar^2}{2m}(X''(\xi)Y(\eta)+X(\xi)Y''(\eta))
+\lambda_+\xi^2X(\xi)Y(\eta)+\lambda_-\eta^2X(\xi)Y(\eta)=EX(\xi)Y(\eta)
\end{align*}
und separieren die Variablen:
\begin{align*}
\frac{\hbar^2}{2m}\frac{X''(\xi)}{X(\xi)}
+
\frac{\hbar^2}{2m}\frac{Y''(\eta)}{Y(\eta)}
+
\lambda_+\xi^2+\lambda_-\eta^2&=E
\\
\Rightarrow\qquad
\frac{\hbar^2}{2m}\frac{X''(\xi)}{X(\xi)}
+\lambda_+\xi^2
&=
-\frac{\hbar^2}{2m}\frac{Y''(\eta)}{Y(\eta)}
-\lambda_-\eta^2
+E
\end{align*}
Die linke Seite h"angt nur von $\xi$ ab, die rechte nur von $\eta$,
sie mussen daher beide konstant sein.
Wir nennen die gemeinsame Konstante $\mu$, und erhalten zwei
separierte Systeme
\begin{align*}
\biggl(
\frac{\hbar^2}{2m}\frac{\partial^2}{\partial\xi^2}+\lambda_+\xi^2
\biggr)X&=\mu X
\\
\biggl(
\frac{\hbar^2}{2m}\frac{\partial^2}{\partial\eta^2}+\lambda_-\eta^2
\biggr)Y
&=E-\mu
\end{align*}
Setzt man darin
\[
\omega_\pm=\sqrt{\frac{2\lambda_\pm}{m}}
\]
erh"alt man 
\begin{align*}
\biggl(
\frac{\hbar^2}{2m}\frac{\partial^2}{\partial\xi^2}+\frac{m}2\omega_+^2\xi^2
\biggr)X&=\mu X
\\
\biggl(
\frac{\hbar^2}{2m}\frac{\partial^2}{\partial\eta^2}+\frac{m}2\omega_-^2\eta^2
\biggr)Y
&=E-\mu
\end{align*}
Dies sind die Schr"odingergleichungen f"ur zwei harmonische Operatoren
mit den Kreisfrequenzen $\omega_\pm$.
Die Energieniveaus dieser Oszillatoren sind
\begin{align*}
\mu
&=
\hbar\omega_+\biggl(n_++\frac12\biggr),
&
E-\mu&=
\hbar\omega_-\biggl(n_-+\frac12\biggr).
\end{align*}
Daraus k"onnen wir jetzt die Energieniveaus f"ur das Gesamtsystem ableiten:
\[
E=
\hbar\omega_+\biggl(n_++\frac12\biggr)
+
\hbar\omega_-\biggl(n_-+\frac12\biggr),
\]
wobei $n_\pm$ beliebige nat"urliche Zahlen sind.
Die Energie des Grundzustandes ist
\[
\frac{\hbar}2(\omega_++\omega_-).
\]
\end{loesung}


