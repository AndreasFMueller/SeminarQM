Gegeben ist die Matrix
\[
H(\varepsilon)
=
\begin{pmatrix}\varepsilon&1\\1&-\varepsilon\end{pmatrix}
=
\begin{pmatrix}0&1\\1&0\end{pmatrix}
+
\varepsilon
\begin{pmatrix}1&0\\0&-1\end{pmatrix}
.
\]
Finden Sie eine St"orungsl"osung erster Ordnung f"ur Eigenwerte und
Eigenvektoren von $H(\varepsilon)$ in Abh"angigkeit vom St"orungsparameter
$\varepsilon$.

\begin{loesung}
Zun"achst brauchen wir die exakte L"osung f"ur den ungest"orten Fall,
also f"ur $\varepsilon=0$. Das charakteristische Polynom ist
\[
\left|
\begin{matrix}-\lambda&1\\1&-\lambda\end{matrix}
\right|
=\lambda^2-1=(\lambda+1)(\lambda-1)=0,
\]
die Eigenwerte sind also $\lambda_\pm=\pm 1$. Die zugeh"origen
Eigenvektoren k"onnen mit dem Gaussalgorithmus berechnet werden:
\[
\begin{aligned}
\begin{tabular}{|>{$}c<{$}>{$}c<{$}|}
\hline
-1&1\\
1&-1\\
\hline
\end{tabular}
&\rightarrow
\begin{tabular}{|>{$}c<{$}>{$}c<{$}|}
\hline
1&-1\\
0&0\\
\hline
\end{tabular}
&&\Rightarrow&
v_+&=\frac1{\sqrt{2}}\begin{pmatrix}1\\1\end{pmatrix}
\\
\begin{tabular}{|>{$}c<{$}>{$}c<{$}|}
\hline
1&1\\
1&1\\
\hline
\end{tabular}
&\rightarrow
\begin{tabular}{|>{$}c<{$}>{$}c<{$}|}
\hline
1&1\\
0&0\\
\hline
\end{tabular}
&&\Rightarrow&
v_+&=\frac1{\sqrt{2}}\begin{pmatrix}1\\-1\end{pmatrix}
\end{aligned}
\]
Auf diese ungest"orte L"osung k"onnen wir jetzt die St"orungstheorie
anwenden.

Nach der allgemeinen Theorie k"onnen wir die Korrekturterme f"ur die
Eigenwerte und Eigenvektoren mit den Formeln
(\ref{skript:stoerungsloesung1ordnung}) berechnen.
\begin{align*}
E_+^{(1)}&=v_+^tH_1v_+
	=\frac12\begin{pmatrix}1&1\end{pmatrix}
		\begin{pmatrix}1&0\\0&-1\end{pmatrix}
		\begin{pmatrix}1\\1\end{pmatrix}
	=\frac12\begin{pmatrix}1&1\end{pmatrix}
		\begin{pmatrix}1\\-1\end{pmatrix}
	=0
\\
E_-^{(1)}&=v_-^tH_1v_-=0
	=\frac12\begin{pmatrix}1&-1\end{pmatrix}
		\begin{pmatrix}1&0\\0&-1\end{pmatrix}
		\begin{pmatrix}1\\-1\end{pmatrix}
	=\frac12\begin{pmatrix}1&-1\end{pmatrix}
		\begin{pmatrix}1\\1\end{pmatrix}
	=0
\end{align*}
In erster Ordnung werden die Eigenwerte also nicht ver"andert.
Die gemischten Koeffizienten sind
\begin{align*}
\frac{v_+^tH_1v_-}{E_+-E_-}
&=
\frac{1}{2}
\frac12
	\begin{pmatrix}1&-1\end{pmatrix}
	\begin{pmatrix}1&0\\0&-1\end{pmatrix}
	\begin{pmatrix}1\\1\end{pmatrix}
=
-\frac14
	\begin{pmatrix}1&-1\end{pmatrix}
	\begin{pmatrix}1\\-1\end{pmatrix}
=
-\frac12
\\
\frac{v_-^tH_1v_+}{E_--E_+}
&=
\frac{1}{-2}
\frac12
	\begin{pmatrix}1&1\end{pmatrix}
	\begin{pmatrix}1&0\\0&-1\end{pmatrix}
	\begin{pmatrix}1\\-1\end{pmatrix}
=
\frac14
	\begin{pmatrix}1&1\end{pmatrix}
	\begin{pmatrix}-1\\-1\end{pmatrix}
=
-\frac12
\end{align*}
Damit sind die neuen Eigenvektoren:
\begin{align*}
v_+(\varepsilon)
&=
(1+i\varepsilon\gamma)
\frac1{\sqrt{2}}
\begin{pmatrix}1\\-1\end{pmatrix}
-\varepsilon\frac1{2\sqrt{2}}\begin{pmatrix}1\\1\end{pmatrix}
\\
v_-(\varepsilon)
&=
(1+i\varepsilon\gamma)
\frac1{\sqrt{2}}
\begin{pmatrix}1\\1\end{pmatrix}
-\varepsilon\frac1{2\sqrt{2}}\begin{pmatrix}1\\-1\end{pmatrix}
\end{align*}
Darin ist $\gamma$ so zu w"ahlen, dass die Vektoren L"ange $1$ erhalten.
\end{loesung}

