Ein System mit zwei Zust"anden $|1\rangle$ und $|2\rangle$
hat den Hamilton-Operator
\[
H_0=\begin{pmatrix}
E_1&  0\\
  0&E_2
\end{pmatrix},
\]
mit $E_1=1$ und $E_2=2$.
Zus"atzlich wird jetzt eine Wechselwirkung eingeschaltet, die den
Hamilton-Operator
\[
\varepsilon H_1=
\varepsilon \begin{pmatrix}
     -1&\frac12\\
\frac12&      1\\
\end{pmatrix}
\]
hat.
\begin{teilaufgaben}
\item 
Berechnen Sie in erster N"aherung die Energieniveaus $E_k(\varepsilon)$
und die Zust"ande $|k(\varepsilon)\rangle$ des gest"orten Systems.
\item
Setzen Sie die gefundenen N"aherungen in die Schr"odingergleichung ein
und pr"ufen Sie damit nach, dass die N"aherung in erster Ordnung
korrekt ist.
\item 
Stellen sie durch geeignete Wahl von $\gamma$ in $|1(\varepsilon)\rangle$ 
sicher, dass der Zustandsvektor in erster N"aherung normiert ist.
\end{teilaufgaben}

\begin{loesung}
\begin{teilaufgaben}
\item
Nach Voraussetzung sind die Zust"ande nicht entartet, da ihre Energien
alle verschieden sind.
Wir k"onnen also die Formeln der St"orungstheorie f"ur ein nicht
entartetes System verwenden.

Die Formeln verwenden die
Matrixelemente $\langle k|\,H_1\,|l\rangle$,
die wir zun"achst berechnen wollen.
Die Zustandsvektoren $|k\rangle$ sind die Standardbasisvektoren
des zweidimensionalen Zustandsraumes, in dem die Operatoren $H_0$
und $H_1$ wirken.
Daher sind die Matrixelemente $\langle k|\,H_1\,|l\rangle$ gerade
die Elemente der Matrix $H_1$.

Diese liefern f"ur die
Energien
\[
E_k^{(1)}=\langle k|\, H_1\,|k\rangle,
\]
also 
\begin{align*}
E_1(\varepsilon)&=E_1-\varepsilon=1-\varepsilon,\\
E_2(\varepsilon)&=E_2+\varepsilon=2+\varepsilon.
\end{align*}
F"ur die Zustandsvektoren m"ussen die Koeffizienten 
$\langle\psi_l^{(0)}|\psi_k^{(1)}\rangle$ bestimmt werden:
\[
\langle\psi_l^{(0)}|\psi_k^{(1)}\rangle
=
\frac{\langle\psi_l^{(0)}|\,H_1\,|\psi_k^{(0)}\rangle}{E_k^{(0)}-E_l^{(0)}}
=
\frac{\langle l|\,H_1\,|k\rangle}{E_k-E_l}
\]
Als Matrix angeordnet sind diese Koeffizienten
\[
\langle\psi_k^{(1)}|\psi_l^{(0)}\rangle
=
\begin{pmatrix}
*
	&\displaystyle\frac{\frac12}{E_1-E_2}\\
\displaystyle\frac{\frac12}{E_2-E_1}
	&*
\end{pmatrix}
=
\begin{pmatrix}
*
	&\displaystyle-\frac12\\
\displaystyle\frac12
	&*
\end{pmatrix}.
\]
Daraus kann man jetzt die gest"orten Zustandsvektoren konstruieren:
\begin{align*}
|\psi_k^{(1)}\rangle
&=
i\gamma|\psi_k^{(0)}\rangle
+\sum_{k\ne l}\frac{\langle\psi_l^{(0)}|\,H_1\,|\psi_k^{(0)}\rangle}{E_k^{(0)}-E_l^{(0)}}
\\
\Rightarrow\qquad
|1(\varepsilon)\rangle
&=
(1+i\varepsilon\gamma)\,|1\rangle
	+ \frac{\varepsilon}{2(E_1-E_2)}\,|2\rangle 
=
(1+i\varepsilon\gamma)\,|1\rangle
	- \frac{\varepsilon}{2}\,|2\rangle 
=
\begin{pmatrix}
1+i\varepsilon\gamma \\ \displaystyle-\frac{\varepsilon}2
\end{pmatrix},
\\
|2(\varepsilon)\rangle
&=
(1+i\varepsilon\gamma)\,|2\rangle
	+ \frac{\varepsilon}{2(E_2-E_1)}\,|1\rangle
=
(1+i\varepsilon\gamma)\,|2\rangle
	+ \frac{\varepsilon}{2}\,|1\rangle
=
\begin{pmatrix}
\displaystyle\frac{\varepsilon}2
\\
1+i\varepsilon\gamma
\end{pmatrix}.
\end{align*}
\item
Diese L"osung setzen wir jetzt in die Schr"odingergleichung ein.
F"ur den ersten Zustand m"ussen wir "uberpr"ufen, ob die Gleichung
$H(\varepsilon)\, |1(\varepsilon)\rangle
=E_1(\varepsilon)\,|1(\varepsilon)\rangle$
bis auf Terme mindestens zweiter Ordnung in $\varepsilon$ erf"ullt ist.
Setzen alles ein und erhalten
\begin{align*}
H(\varepsilon)\, |1(\varepsilon)\rangle -E_1(\varepsilon)\,|1(\varepsilon)\rangle
&=
\begin{pmatrix}
1-\varepsilon-(1-\varepsilon)&\frac{\varepsilon}2\\
\frac{\varepsilon}2&2+\varepsilon-(1-\varepsilon)
\end{pmatrix}
\begin{pmatrix}
1+i\gamma\varepsilon\\-\frac{\varepsilon}2
\end{pmatrix}
\\
&=
\begin{pmatrix}
0&\frac{\varepsilon}2\\
\frac{\varepsilon}2&1+2\varepsilon
\end{pmatrix}
\begin{pmatrix}
1+i\gamma\varepsilon\\-\frac{\varepsilon}2
\end{pmatrix}
=
\begin{pmatrix}
-\frac{\varepsilon^2}{4}
\\
-i\gamma\frac{\varepsilon^2}2-\varepsilon^2
\end{pmatrix}
=
\begin{pmatrix}
-\frac14
\\
-i\frac{\gamma}2-1
\end{pmatrix}
\varepsilon^2,
\\
H(\varepsilon)\, |2(\varepsilon)\rangle -E_2(\varepsilon)\,|2(\varepsilon)\rangle
&=
\begin{pmatrix}
1-\varepsilon - (2+\varepsilon) & \frac{\varepsilon}2\\
\frac{\varepsilon}2&2+\varepsilon -(2+\varepsilon)
\end{pmatrix}
\begin{pmatrix}
\frac{\varepsilon}2\\1+i\gamma\varepsilon
\end{pmatrix}
\\
&=
\begin{pmatrix}
-1-2\varepsilon & \frac{\varepsilon}2\\
\frac{\varepsilon}2&0
\end{pmatrix}
\begin{pmatrix}
\frac{\varepsilon}2\\1+i\gamma\varepsilon
\end{pmatrix}
=
\begin{pmatrix}
-\varepsilon^2+i\gamma\frac{\varepsilon^2}{2}
\\
\frac{\varepsilon^2}{4}
\end{pmatrix}
=\begin{pmatrix}
-1+i\frac{\gamma}2\\\frac14
\end{pmatrix}\varepsilon^2.
\end{align*}
In beiden F"allen ist der Fehler der St"orungsn"aherung ein Vielfaches
von $\varepsilon^2$ und damit insbesondere von zweiter Ordnung in $\varepsilon$.
\item Die Norm des Zustandsvektors ist
\[
\langle 1(\varepsilon)|1(\varepsilon)\rangle
=
\frac{\varepsilon^2}4+1+\gamma^2 \varepsilon^2
=
1+\varepsilon^2\biggl(\frac14+\gamma^2\biggr),
\]
Offenbar ist der Vektor in erster Ordnung immer normiert, unabh"angig von
der Wahl von $\gamma$.
\end{teilaufgaben}
\end{loesung}

