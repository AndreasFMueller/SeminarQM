Gegeben sind zwei Paare von Operatoren $a_i$ und $a_i^+$ mit $i=1,2$
mit den Vertauschungsrelationen
\begin{align*}
[a_1,a_1^+]&=1&[a_1,a_2^+]&=0&[a_1,a_2]&=0\\
[a_2,a_2^+]&=1&[a_2,a_1^+]&=0&[a_1^+,a_2^+]&=0
\end{align*}
Ausserdem setzen wir 
\[
N_i=a_i^+a_i
\qquad
\Rightarrow
\qquad
a_ia_i^+=a_i^+a_i+1=N_i+1
\]
und
\begin{align*}
J_+
&=
\hbar a_1^+a_2,
&
J_1
&=
\frac12(J_++J_-)
=
\frac{\hbar}2(a_1^+a_2+a_1a_2^+),
\\
J_-
&=
\hbar a_2^+a_1,
&
J_2
&=
\frac1{2i}(J_+-J_-)
=
\frac{\hbar}{2i}(a_1^+a_2-a_1a_2^+),
\\
&&
J_3
&=
\frac{\hbar}2(a_1^+a_1-a_2^+a_2)
=
\frac{\hbar}2(N_1-N_2).
\end{align*}
\begin{teilaufgaben}
\item Berechnen Sie $[N_i,a_i^+]$ und $[N_i,a_i]$ und leiten Sie daraus
die Regeln
\begin{align*}
N_ia_i^+&= a_i^+ N_i + a_i^+,
&
N_ia_i  &= a_i   N_i - a_i,
\\
a_i^+N_i&= N_ia_i^+ - a_i^+,
&
a_i N_i  &= N_i a_i   + a_i
\end{align*}
ab.
\item Berechnen Sie $[J_+,J_-]$.
\item Zeigen Sie, dass gilt $[J_1,J_2]=i\hbar J_3$.
\item Berechnen Sie $[J_3,J_+]$ und $[J_3,J_-]$.
\item Zeigen Sie, dass gilt $[J_2,J_3]=i\hbar J_1$ und $[J_3,J_1]=i\hbar J_2$.
\item Zeigen Sie, dass $[\vec J^2, J_3]=0$ und $[\vec J^2,J_\pm]=0$.
\end{teilaufgaben}

\begin{loesung}
\begin{teilaufgaben}
\item Die Rechnung folgt der entsprechenden Rechnung f"ur die Auf- und
Absteigeoperatoren beim harmonischen Oszillator.
\begin{align*}
[N_i,a_i^+]
&=
a_i^+a_ia_i^+-a_i^+a_i^+a_i
=
a_i^+[a_i,a_i^+]=a_i^+,
\\
[N_i,a_i]
&=
a_i^+a_ia_i-a_ia_i^+a_i
=
[a_i^+,a_i]a_i
=-a_i.
\end{align*}
Die verlangten Rechenregeln bekommt man durch Aufl"osen des Kommutators.
\item
Wir setzen die Definition der Operatoren $J_+$ und $J_-$ ein und
erhalten
\begin{align*}
[J_+,J_-]
&=
\hbar^2(a_1^+a_2a_2^+a_1 - a_2^+a_1a_1^+a_2)
=
\hbar^2(a_1^+a_1 a_2a_2^+ - a_1a_1^+ a_2^+a_2)
\\
&=
\hbar^2\bigg(N_1(N_2+1)- (N_1+1)N_2\bigg)
=2\hbar\frac{\hbar}2(N_1N_2+N_1-N_1N_2-N_2)=2\hbar J_3.
\end{align*}
\item
Da sich $J_1$ und $J_2$ durch $J_+$ und $J_-$ ausdr"ucken lassen,
k"onnen wir den Kommutator ebenfalls in $J_+$ und $J_-$
ausdr"ucken:
\begin{align*}
[J_1,J_2]
&=
\frac{1}{4i}[J_++J_-,J_+-J_-]
=
\frac{1}{4i}\bigl(
[J_+,J_+] -[J_+,J_-] +[J_-,J_+] -[J_-,-J_-]
\bigr)
\\
&=
-\frac1{i}[J_+,J_-]
=
-\frac1{i}\hbar J_3
=i\hbar J_3.
\end{align*}
\item
Wir setzen wiederum die Definition ein und erhalten
\begin{align*}
[J_3,J_+]
&=
\frac{\hbar^2}2\bigl(
(N_1-N_2) a_1^+a_2 - a_1^+a_2 (N_1-N_2)
\bigr)
\\
&=
\frac{\hbar^2}2\bigl(
N_1 a_1^+a_2 - N_2 a_1^+ a_2 - a_1^+a_2 N_1 + a_1^+a_2 N_2
\bigr)
\\
&=
\frac{\hbar^2}2\bigl(
N_1a_1^+ a_2
-
a_1^+ N_2 a_2
-
(N_1a_1^+-a_1^+)a_2
+
a_1^+(N_2a_2 +a_2)
\bigl)
\\
&=\frac{\hbar^2}{2}
\bigl( a_1^+a_2 +a_1^+a_2
\bigr)
=\hbar^2 a_1^+a_2=\hbar J_+,
\\
[J_3,J_-]
&=
\frac{\hbar^2}2\bigl(
(N_1-N_2) a_2^+a_1 - a_2^+a_1 (N_1-N_2)
\bigr)
\\
&=
\frac{\hbar^2}2\bigl(
N_1 a_2^+a_1 - N_2 a_2^+ a_1 - a_2^+a_1 N_1 + a_2^+a_1 N_2
\bigr)
\\
&=
\frac{\hbar^2}2\bigl( 
a_2^+N_1 a_1 - N_2 a_2^+ a_1 - a_2^+(N_1a_1+a_1) + (N_2a_2^+-a_2^+)a_1
\bigr)
\\
&=\frac{\hbar^2}{2}
\bigl(
-a_2^+a_1 - a_2^+a_1
\bigr)
=-\hbar^2 a_2^+a_1=-\hbar J_-.
\end{align*}
\item 
Da man $J_1$ und $J_2$ durch die $J_\pm$ ausdr"ucken kann, kann man
die verlangten Kommutatoren aus den
Kommutatoren $[J_3,J_\pm]$ ermitteln, die in der letzten Teilaufgabe
ausgerechnet worden sind.
Man erh"alt nach einsetzen der Definition f"ur $J_2$ und $J_1$:
\begin{align*}
[J_2,J_3]
&=
\frac1{2i}[J_+-J_-,J_3]
=
i\frac{1}{2}\bigl([J_3,J_+] - [J_3,J_-]\bigr)
=
\frac1{2i}\bigl(\hbar J_+-(-\hbar J_-)\bigr)
=i\hbar J_1,
\\
[J_3,J_1]
&=
\frac{1}{2}[J_3,J_++J_-]
=
\frac{1}{2}\bigl(
[J_3,J_+]
+
[J_3,J_-]
\bigr)
=\frac12\bigl(
\hbar J_+
-
\hbar J_-
\bigr)
=i\hbar L_2.
\end{align*}
\item Die Teilaufgaben c) und e) zeigen, dass die Operatoren $J_i$
die gleichen algebraischen Regeln erf"ullen wie die Drehimpulsoperatoren.
Die Regeln f"ur die Drehimpulsoperatoren haben gen"ugt, um die verlangten
Vertauschungsrelationen f"ur den Drehimpuls zu beweisen. Da die $J_i$
die gleichen algebraischen Regeln erf"ullt, gelten die Vertauschungsrelationen
auch f"ur die Komponenten von $\vec J$.
\end{teilaufgaben}
\end{loesung}

\begin{diskussion}
\begin{figure}
\centering
\includegraphics{graphics/drehimpuls-2.pdf}
\caption{Die Auf- und Absteigeoperatoren $a_i^+$ und $a_i$ und die
damit aus dem Grundzustand erzeugbaren Zust"ande.
\label{aufgabe12001:operatoren}}
\end{figure}
\begin{figure}
\centering
\includegraphics{graphics/drehimpuls-1.pdf}
\caption{Mit $l=\frac12(N_1+N_2)$ und $m=\frac12(N_1-N_2)$ entsteht
die bekannte Darstellung der m"oglichen Drehimpulszust"ande.
\label{aufgabe12001:drehimpuls}}
\end{figure}
Man kann die Operatoren $J_i$ als eine alternative Darstellung der Operatoren
$L_i$ betrachten (Abbildungen~\ref{aufgabe12001:operatoren} und \ref{aufgabe12001:drehimpuls}).
W"ahrend sich mit $L_\pm$ nur die Drehimpulszust"ande f"ur festes
$l$ zum Beispiel aus dem Zustand mit $m=0$ konstruieren lassen,
kann man mit Hilfe der Operatoren $a_i$ und $a_i^+$ alle Drehimpulszust"ande
aus einem Zustand mit $l=0$ und $m=0$ konstruieren.
\end{diskussion}

