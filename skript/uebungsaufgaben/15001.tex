Berechnen Sie
\begin{teilaufgaben}
\item
$(18 + 48i)(20 + 14i)$
\item
$(1+2i)/(9+i)$
\item
$(0.6 + 0.8i)^{47}$
\end{teilaufgaben}

\begin{loesung}
\begin{teilaufgaben}
\item 
$
(18 + 48i)(20 + 14i)
=
18\cdot 20 + (48\cdot 20+18\cdot 14)i-48\cdot 14
=
-312+1212i$
\item Wir erweitern mit $\overline{9+i}=9-i$
\[
\frac{1+2i}{9+i}\frac{9-i}{9-i}
=
\frac{9+2+18i-i}{9+1}
=
1.1+1.7i.
\]
\item
Wir schreiben $0.6+0.8i$ in Polardarstellung.
Dabei stellen wir fest, dass der Betrag $|0.6+0.8i]=1$ ist. Das Argument ist
\[
0.6+0.8i=
e^{i\varphi},\qquad\text{mit $\varphi=\arctan\frac43=53.13010^\circ$}.
\]
Die 47te Potenz ist dann
\[
(0.6+0.8i)^{47}=e^{i47\varphi}
=\cos 47\varphi+i\sin\varphi
=0.92129 - i0.38889.
\]
\end{teilaufgaben}
\end{loesung}


