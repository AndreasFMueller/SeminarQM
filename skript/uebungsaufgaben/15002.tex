Finden Sie Eigenwerte und Eigenvektoren der Matrizen
\[
A=\begin{pmatrix}
0&i\\
-i&0
\end{pmatrix}
\qquad\text{und}\qquad
B=\begin{pmatrix}
0&0&1\\
1&0&0\\
0&1&0
\end{pmatrix}.
\]

\begin{loesung}
Wir m"ussen das charakteristische Polynom und seine Nullstellen berechnen
\begin{align*}
0
&=
\left|\begin{matrix}
-\lambda&i\\
-i&-\lambda
\end{matrix}\right|
=\lambda^2-1
=(\lambda + 1)(\lambda - 1)
&
\lambda_\pm&=\pm 1
\\
0
&=
\left|\begin{matrix}
-\lambda&    0   &   1    \\
    1   &-\lambda&   0    \\
    0   &    1   &-\lambda
\end{matrix}\right|
=
-\lambda^3+1
=-(\lambda - 1)(\lambda^2+\lambda+1)
&
\lambda_1&=1\\
&&\lambda_{2,3}&=-\frac12\pm\sqrt{\frac14-1}=\frac{-1\pm i\sqrt{3}}2.
\end{align*}
Da $A$ eine hermitesche Matrix ist, sind die Eigenwerte reell, und
wir k"onnen die Eigenvektoren sofort bestimmen:
\begin{align*}
\begin{tabular}{|>{$}c<{$}>{$}c<{$}|}
\hline
-1&i\\
-i&-1\\
\hline
\end{tabular}
&\rightarrow
\begin{tabular}{|>{$}c<{$}>{$}c<{$}|}
\hline
 1&-i\\
 0& 0\\
\hline
\end{tabular}
&v_+&=\frac{1}{\sqrt{2}}\begin{pmatrix}i\\1\end{pmatrix}
\\
\begin{tabular}{|>{$}c<{$}>{$}c<{$}|}
\hline
 1&i\\
-i& 1\\
\hline
\end{tabular}
&\rightarrow
\begin{tabular}{|>{$}c<{$}>{$}c<{$}|}
\hline
 1& i\\
 0& 0\\
\hline
\end{tabular}
&v_-&=\frac{1}{\sqrt{2}}\begin{pmatrix}-i\\1\end{pmatrix}
\end{align*}
F"ur die Eigenvektoren von $B$ ist es n"utzlich,
sich der Regeln von Vieta zu erinnern, insbesondere
der Tatsache, dass $\lambda_2\lambda_3=1$, oder $1/\lambda_2=\lambda_3$.
Ausserdem gilt $\lambda_3=1$, oder $\lambda_3^2=1/\lambda_3=\lambda_2$.
Wegen $\lambda^3=1$ folgt auch $|\lambda_2|=|\lambda_3|=1$.
Damit k"onnen wir ausrechnen
\begin{align*}
\begin{tabular}{|>{$}c<{$}>{$}c<{$}>{$}c<{$}|}
\hline
-1& 0& 1\\
 1&-1& 0\\
 0& 1&-1\\
\hline
\end{tabular}
&
\rightarrow
\begin{tabular}{|>{$}c<{$}>{$}c<{$}>{$}c<{$}|}
\hline
 1& 0&-1\\
 0&-1& 1\\
 0& 1&-1\\
\hline
\end{tabular}
\rightarrow
\begin{tabular}{|>{$}c<{$}>{$}c<{$}>{$}c<{$}|}
\hline
 1& 0&-1\\
 0& 1&-1\\
 0& 0& 0\\
\hline
\end{tabular}
&
v_1&=\frac1{\sqrt{3}}\begin{pmatrix}1\\1\\1\end{pmatrix}
\\
\begin{tabular}{|>{$}c<{$}>{$}c<{$}>{$}c<{$}|}
\hline
-\lambda_2&         0&         1\\
         1&-\lambda_2&         0\\
         0&         1&-\lambda_2\\
\hline
\end{tabular}
&
\rightarrow
\begin{tabular}{|>{$}c<{$}>{$}c<{$}>{$}c<{$}|}
\hline
        1&         0&-\lambda_3\\
        0&-\lambda_2& \lambda_3\\
        0&         1&-\lambda_2\\
\hline
\end{tabular}
\rightarrow
\begin{tabular}{|>{$}c<{$}>{$}c<{$}>{$}c<{$}|}
\hline
        1&        0&-\lambda_3  \\
        0&        1&-\lambda_3^2\\
        0&        1&-\lambda_2  \\
\hline
\end{tabular}
\\
&\rightarrow
\begin{tabular}{|>{$}c<{$}>{$}c<{$}>{$}c<{$}|}
\hline
        1&        0&-\lambda_3\\
        0&        1&-\lambda_2\\
        0&        0&         0\\
\hline
\end{tabular}
&
v_2&=\frac1{\sqrt{3}}\begin{pmatrix}\lambda_3\\\lambda_2\\1\end{pmatrix}
\\
\begin{tabular}{|>{$}c<{$}>{$}c<{$}>{$}c<{$}|}
\hline
-\lambda_3&         0&         1\\
         1&-\lambda_3&         0\\
         0&         1&-\lambda_3\\
\hline
\end{tabular}
&
\rightarrow
\begin{tabular}{|>{$}c<{$}>{$}c<{$}>{$}c<{$}|}
\hline
        1&         0&-\lambda_2\\
        0&-\lambda_3& \lambda_2\\
        0&         1&-\lambda_3\\
\hline
\end{tabular}
\rightarrow
\begin{tabular}{|>{$}c<{$}>{$}c<{$}>{$}c<{$}|}
\hline
        1&        0&-\lambda_2  \\
        0&        1&-\lambda_2^2\\
        0&        1&-\lambda_3  \\
\hline
\end{tabular}
\\
&\rightarrow
\begin{tabular}{|>{$}c<{$}>{$}c<{$}>{$}c<{$}|}
\hline
        1&        0&-\lambda_2\\
        0&        1&-\lambda_3\\
        0&        0&         0\\
\hline
\end{tabular}
&
v_3&=\frac1{\sqrt{3}}\begin{pmatrix}\lambda_2\\\lambda_3\\1\end{pmatrix}
\\
\end{align*}
Ausgeschrieben sind die Eigenvektoren von $B$
\begin{align*}
v_1&=\frac{1}{\sqrt{3}}\begin{pmatrix}1\\1\\1\end{pmatrix}
&
v_2&=
\frac1{\sqrt{3}}\begin{pmatrix}
\frac{-1-i\sqrt{3}}2\\
\frac{-1+i\sqrt{3}}2\\
1
\end{pmatrix}
&
v_3&=
\frac1{\sqrt{3}}\begin{pmatrix}
\frac{-1+i\sqrt{3}}2\\
\frac{-1-i\sqrt{3}}2\\
1
\end{pmatrix}.
\end{align*}
\end{loesung}

