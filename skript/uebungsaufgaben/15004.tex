F"ur jede ganze Zahl $k\in\mathbb Z$ sei
\[
e_k(x)=e^{ikx}.
\]
Berechnen Sie  $(e_k,e_l)$ f"ur das Skalarprodukt von Vektoren definiert
durch
\[
(f,g)=\frac1{2\pi}\int_{-\pi}^{\pi}\bar f(x)g(x)\,dx.
\]

\begin{loesung}
\begin{align*}
(e_k,e_l)
&=
\frac1{2\pi}\int_{-\pi}^{\pi} \bar e_k(x)e_l(x)\,dx
=
\frac1{2\pi}\int_{-\pi}^{\pi} e^{-ikx}e^{ilx} \,dx
=
\frac1{2\pi}\int_{-\pi}^{\pi} e^{i(l-k)x} \,dx
\end{align*}
F"ur $k=l$ wird der Integrand konstant:
\begin{align*}
(e_k,e_l)
&=
\frac1{2\pi}\int_{-\pi}^{\pi} \,dx=\frac1{2\pi}2\pi=1.
\end{align*}
F"ur $k\ne l$ kann man eine Stammfunktion angeben
\begin{align*}
(e_k,e_l)
&=
\frac1{2\pi}\biggl[
\frac1{i(l-k)}e^{i(l-k)x}
\biggr]_{-\pi}^\pi
=\frac1{2\pi i(l-k)}(e^{i(l-k)\pi}-e^{-i(l-k)\pi})=0.
\end{align*}
Die Funktionen $e_k$ sind also orthonormiert im Hilbertraum $L^2([-\pi,\pi])$.
\end{loesung}

