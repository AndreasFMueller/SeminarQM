Berechnen Sie die Poissonklammer $(x,p)$.
\begin{itemize}
\item[\hbox to 1cm{K)\hfill}] $(x,p)=-1$
\item[\hbox to 1cm{I)\hfill}] $(x,p)=1$
\item[\hbox to 1cm{J)\hfill}] $(x,p)=0$
\end{itemize}

\begin{loesung}
Die Poissonklammer ist nach Definition:
\[
(x,p)
=
\underbrace{\frac{\partial x}{\partial x}}_{=1}\underbrace{\frac{\partial p}{\partial p}}_{=1}
-
\underbrace{\frac{\partial x}{\partial p}}_{=0}\underbrace{\frac{\partial p}{\partial x}}_{=0}
=1.
\]
Antwort I) ist richtig.
\end{loesung}

