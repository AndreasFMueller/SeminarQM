Berechnen Sie den Kommutator $[\hat x,\hat p]$?
\begin{itemize}
\item[\hbox to 1cm{$\clubsuit$)\hfill}]
$[\hat x,\hat p]=\frac{\hbar}{i}\operatorname{id}$
\item[\hbox to 1cm{$\heartsuit$)\hfill}]
$[\hat x,\hat p]=-\frac{\hbar}{i}\operatorname{id}$
\item[\hbox to 1cm{$\diamondsuit$)\hfill}]
$[\hat x,\hat p]=\frac{i}{\hbar}\hat p$
\item[\hbox to 1cm{$\spadesuit$)\hfill}]
$[\hat x,\hat p]=\hat x$
\end{itemize}

\begin{hinweis}
Sie k"onnen zur besseren "Ubersicht eine Wellenfunktion rechts neben
die Operatoren schreiben, um besser zu verstehen, wie sie wirken, und
diese am Schluss wieder weglassen.
\end{hinweis}

\begin{loesung}
Der Operator $\hat p$ operiert als Ableitung in der Ortsdarstellung:
\begin{align*}
[\hat x,\hat p]\Psi(x)
&=
(\hat x\hat p-\hat p\hat x)\Psi(x)
=
x\frac{\hbar}{i}\frac{\partial}{\partial x}\Psi(x)
-
\frac{\hbar}{i}\frac{\partial}{\partial x}x\Psi(x)
\\
&=
x\frac{\hbar}{i}\frac{\partial\Psi(x)}{\partial x}\Psi(x)
-
\frac{\hbar}{i}\Psi(x)
-
\frac{\hbar}{i}x\frac{\partial \Psi(x)}{\partial x}\Psi(x)
=
-\frac{\hbar}{i}\Psi(x)
\end{align*}
Ohne die Wellenfunktion geschrieben:
\[
[\hat x,\hat p]=-\frac{\hbar}{i}\operatorname{id},
\]
wobei $\operatorname{id}$ die identische Abbildung ist.
Antwort $\heartsuit$) ist richtig.

Man kann auch argumentieren, dass die Poissonklammer von $x$ und $p$ $1$ ist,
und daher nach den Quantisierungsregeln der Kommutator 
$-\frac{\hbar}{i}\operatorname{id}$ sein muss.
\end{loesung}


