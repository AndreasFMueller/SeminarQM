Betrachten Sie den Hamilton-Operator
\[
\hat H=-\frac{\hbar^2}{2m_e}\frac{\partial^2}{\partial x^2}-V_0.
\]
Welche der folgenden Wellenfunktionen ist eine L"osung der zugeh"origen
zeitabh"angigen Schr"odingergleichung?
\begin{itemize}
\item[\hbox to1cm{L)\hfill}]
$
\Psi(x,t)
=
(V_0+x+Et)^{i\hbar}
$
\item[\hbox to1cm{M)\hfill}]
$
\Psi(x,t)
=
\exp\left(\frac{i}{\hbar}(x\sqrt{2m(E+V_0)}-Et)\right)
$
\item[\hbox to1cm{N)\hfill}]
$
\Psi(x,t)
=
\sin(\hbar (Et-xV_0))
$
\end{itemize}

\begin{loesung}
Die Schr"odingergleichung ist 
\[
-\frac{\hbar^2}{2m}\frac{\partial^2}{\partial x^2}\Psi(x,t)-V_0\Psi(x,t)=i\hbar\frac{\partial}{\partial t}\Psi(x,t).
\]
Wir berechnen die einzelnen Teile f"ur den einzig wahrscheinlichen
Kandidaten M):
\begin{align*}
-\frac{\hbar^2}{2m}
\frac{\partial^2}{\partial x^2}
\Psi(x,t)
&=
-\frac{\hbar^2}{2m}
\biggl(\frac{i}{\hbar}\sqrt{2m(E+V_0)}\biggr)^2
\exp\biggl(\frac{i}{\hbar}(x\sqrt{2m(E+V_0)}-Et)\biggr)
\\
&=
-\frac{\hbar^2}{2m}
\biggl(-\frac{1}{\hbar^2}2m(E+V_0)\biggr)
\exp\biggl(\frac{i}{\hbar}(x\sqrt{2m(E+V_0)}-Et)\biggr)
\\
&=
(E+V_0)
\exp\biggl(\frac{i}{\hbar}(x\sqrt{2m(E+V_0)}-Et)\biggr)
\\
&=(E+V_0)\Psi(x,t)
\\
-\frac{\hbar^2}{2m}
\frac{\partial^2}{\partial x^2}
\Psi(x,t)
-V_0\Psi(x,t)
&=
E\Psi(x,t)
\\
i\hbar
\frac{\partial}{\partial t} \Psi(x,t)
&=
i\hbar
\biggl(-\frac{i}{\hbar}E\biggr)
\exp\biggl(\frac{i}{\hbar}(x\sqrt{2m(E+V_0)}-Et)\biggr)
\\
&=
E\Psi(x,t)
\end{align*}
Die Schr"odingergleichung ist erf"ullt, M) ist die richtige L"osung.
\end{loesung}
