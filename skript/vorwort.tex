\chapter*{Vorwort}
\lhead{Vorwort}
\rhead{}
Dieses Buch entstand im Rahmen des Mathematischen Seminars
im Fr"uhjahrssemester 2015 an der Hochschule f"ur Technik Rapperswil.
Die Teilnehmer, Studierende der Abteilungen f"ur Elektrotechnik und
Informatik der
HSR, erarbeiteten nach einer Einf"uhrung in das Themengebiet jeweils
einzelne Aspekte des Gebietes in Form einer Seminararbeit, "uber
deren Resultate sie auch in einem Vortrag informierten. 

Im Fr"uhjahr 2015 war das Thema des Seminars ``Quantenmechanik''.
Die Einf"uhrung bestand aus einigen Vorlesungsstunden, deren
Inhalt im ersten Teil dieses Skripts zusammengefasst ist.
Es ging darum, die mathematischen Grundlagen der Quantenmechanik zu
legen und auf die Berechnung einfacher quantenmechanischer 
Systeme anzuwenden. Das Ziel war, ein auch quantitatives Verst"andnis
daf"ur zu entwickeln, wie die Welt im kleinen funktioniert, und 
dieses Wissen auch zum Beispiel f"ur ein besseres Verst"andnis der
Mikroelektronik zu nutzen. Eine vor allem f"ur Informatiker interessante
Vertiefungsrichtung war zu verstehen, was mit einem Quantencomputer
gemeint ist.

Im zweiten Teil dieses Skripts kommen dann die Teilnehmer
selbst zu Wort. Ihre Arbeiten wurden jeweils als einzelne
Kapitel mit meist nur typographischen "Anderungen "ubernommen.
Diese weiterf"uhrenden Kapitel sind sehr verschiedenartig.
Eine "Ubersicht und Einf"uhrung befindet sich in der Einleitung
zum zweiten Teil auf Seite~\pageref{skript:uebersicht}.

In einigen Arbeiten wurde auch Code zur Demonstration der 
besprochenen Methoden und Resultate geschrieben, soweit
m"oglich und sinnvoll wurde dieser Code im Github-Repository
dieses Kurses \url{https://github.com/AndreasFMueller/SeminarQM.git}
abgelegt, in anderen F"allen verweisen die Artikel selbst auf
das zugeh"orige Code-Repository.

Im genannten Repository findet sich auch der Source-Code dieses
Skriptes, es wird hier unter einer Creative Commons Lizenz
zur Verf"ugung gestellt.

