\documentclass{beamer}
\usetheme{Copenhagen}
\usecolortheme{default}

\usepackage{amsmath,amssymb}
\usepackage[ngerman]{babel}
\usepackage{siunitx}

\author{Hannes Badertscher}
\title{Bellsche Ungleichung}
\date{\today}

\AtBeginSection[]
{
    \begin{frame}<beamer>
        \frametitle{\contentsname}
        \tableofcontents[currentsection]
    \end{frame}
}

\begin{document}

\maketitle

\begin{frame}
    \tableofcontents
\end{frame}

\section{Das EPR Gedankenexperiment}
\begin{frame}{Das EPR Gedankenexperiment}
    \centering
    \includegraphics{../../skript/bell/images/experiment_setup.pdf}
\end{frame}
\begin{frame}{Forderungen}
    \begin{itemize}
        \item Lokalit\"at
        \item Realismus
        \vspace{1cm}
        \item Korrektheit
        \item Vollst\"andigkeit
    \end{itemize}
\end{frame}
\begin{frame}{Herleitung}
    \begin{itemize}
        \item Pauli-Matrizen:
            \begin{align*}
                S_x &= \frac{\hbar}{2} \begin{pmatrix}
                0 & 1 \\ 1 & 0
                \end{pmatrix}
                &
                S_y &= \frac{\hbar}{2} \begin{pmatrix}
                0 & -i \\ i & 0
                \end{pmatrix}
                &
                S_z &= \frac{\hbar}{2} \begin{pmatrix}
                1 & 0 \\ 0 & -1
                \end{pmatrix}\label{equ:bell:paulimatrizen}
            \end{align*}
        \item Eigenwerte:
            \begin{align*}
                |{+}x\rangle &= \frac{1}{\sqrt{2}}\begin{pmatrix} 1\\1 \end{pmatrix} &
                |{-}x\rangle &= \frac{1}{\sqrt{2}}\begin{pmatrix} 1\\-1 \end{pmatrix} \\
                |{+}z\rangle &= \begin{pmatrix} 1\\0 \end{pmatrix} &
                |{-}z\rangle &= \begin{pmatrix} 0\\1 \end{pmatrix} &
            \end{align*}
    \end{itemize}
\end{frame}
\begin{frame}{Die L\"osung: Verborgene Variablen}
    \centering
    \includegraphics[scale=1.5]{../../skript/bell/images/hidden_vars.pdf}
\end{frame}

\section{Die Bellsche Ungleichung}
\begin{frame}{Grundlagen}
    \begin{itemize}
        \item Verallgemeinerung auf beliebige Messrichtungen $\vec{a}$, $\vec{b}$
        \item Messresultate: $A(\vec{a})$, $B(\vec{b})$ mit
                \[
                    A(\vec{a}) = \begin{cases}
                        +1 & \text{Spin Up} \\
                        -1 & \text{Spin Down} \\
                    \end{cases}
                \]
        \item Erwartungswert beider Spin-Messungen:
                \begin{align*}
                    E_{QM}(\vec{a},\vec{b})
                    &= \left\langle \phi \left| 
                    \left( \vec{\hat{\sigma}}_A \cdot \vec{a} \right)
                            \otimes \left( \vec{\hat{\sigma}}_B \cdot \vec{b} \right)
                    \right| \phi \right\rangle
                    = \dots \\
                    &= - \vec{a} \cdot \vec{b} = - \cos(\theta_{ab})
                \end{align*}
    \end{itemize}
\end{frame}
\begin{frame}{Grundlagen}
    \begin{itemize}
        \item Verborgene Variable $\lambda \in \Lambda$ mit W'keitsdichte $\rho(\lambda)$
        \item Erwartungswert:
            \[
                E(\vec{a},\vec{b}) = \int_{\lambda\in\Lambda} 
                    \rho(\lambda) A(\vec{a},\lambda) B(\vec{b},\lambda) d\lambda
            \]
        \item<2-> Anti-Korrelation der Messungen:
            \[
                A(\vec{a},\lambda) = -B(\vec{a},\lambda)
            \]
        \only<3->{
            \[
                E(\vec{a},\vec{b}) = -\int_{\lambda\in\Lambda} 
                    \rho(\lambda) A(\vec{a},\lambda) A(\vec{b},\lambda) d\lambda
            \]
        }
    \end{itemize}
\end{frame}
\begin{frame}{Die Bellsche Ungleichung}
    \begin{itemize}
        \item Bellsche Ungleichung:
            \[
                \left| E(\vec{a},\vec{b}) - E(\vec{a},\vec{c}) \right| 
                \leq
                1 + E(\vec{b},\vec{c})
            \]
        \item Beispiel:
            \begin{align*}
                    \measuredangle(\vec{a},\vec{b}) = \SI{45}{\degree}, &&
                    \measuredangle(\vec{b},\vec{c}) = \SI{45}{\degree} &&
                    \text{und} &&
                    \measuredangle(\vec{a},\vec{c}) = \SI{90}{\degree}.
            \end{align*}
        \item Eingesetzt in $E_{QM} = -\cos(\theta_{ab})$:
            \[
                \frac{1}{\sqrt{2}} \nleq 1 - \frac{1}{\sqrt{2}}
            \]
    \end{itemize}
\end{frame}
\begin{frame}{Bell-Clauser-Horne-Shimony-Holt Ungleichung}
    \begin{itemize}[<+->]
        \item Keine Perfekte Antikorrelation: 
                $E(\vec{a},\vec{a}) = -1+\delta$
        \item Mittelwert \"uber mehrere Messungen: 
                $\left|\bar{A}(\vec{a},\lambda)\right| \leq 1$
        \item 4 Messrichtungen:
                $\vec{a},\vec{a}\,',\vec{b},\vec{b}'$
        \item Differenz:
                $\left| E(\vec{a},\vec{b}) - E(\vec{a},\vec{b}') \right|$
        \item BCHSH-Ungleichung
                \[
                    -2 \leq 
                    E(\vec{a},\vec{b}) - E(\vec{a},\vec{b'}) + E(\vec{a'},\vec{b}) + E(\vec{a'},\vec{b'})
                    \leq 2
                \]
    \end{itemize}
\end{frame}
\begin{frame}{Bell-Clauser-Horne-Shimony-Holt Ungleichung}
    \begin{itemize}
        \item Bell-Clauser-Horne-Shimony-Holt Ungleichung:
            \[
                -2 \leq 
                E(\vec{a},\vec{b}) - E(\vec{a},\vec{b'}) + E(\vec{a'},\vec{b}) + E(\vec{a'},\vec{b'})
                \leq 2
            \]
        \item Beispiel:
            \begin{align*}
                \measuredangle(\vec{a},\vec{b}) = \SI{45}{\degree}, && 
                \measuredangle(\vec{b},\vec{a}') = \SI{45}{\degree} && 
                \text{und} &&
                \measuredangle(\vec{a}',\vec{b}') = \SI{45}{\degree}.
            \end{align*}
        \item Eingesetzt in $E_{QM} = -\cos(\theta_{ab})$:
            \[
                2\sqrt{2} \nleq 2
            \]
    \end{itemize}
\end{frame}

\section{Experimente}
\begin{frame}{Herausforderungen}
    \begin{itemize}
        \item Effizienz der Detektoren
        \item Sicherstellen der Lokalit\"at
        \item R\"aumliche Korrelation
    \end{itemize}
\end{frame}
\begin{frame}{1. Generation -- Freedman Clauser (1972)}
    \begin{itemize}
        \item Generieren der Photonen: \\
            {
                \centering
                \includegraphics{../../skript/bell/images/freedman.pdf}
            }
        \item Resultat: Verletzung der Bellschen Ungleichung!
    \end{itemize}

\end{frame}
\begin{frame}{2. Generation -- Alain Aspect (1980 -- 1985)}
    \begin{itemize}
        \item Dissertation Alain Aspect (Universit\'e Paris-Sud, Orsay)
        \item R\"aumliche Trennung der Detektoren: \SI{13}{\meter}
        \item Resultat: Verletzung der Bellschen Ungleichung!
    \end{itemize}
\end{frame}
\begin{frame}{3. Generation (ab 1990)}
    \begin{itemize}[<+->]
        \item Weihs (Innsbruck, 1998):
            \begin{itemize}[<.->]
                \item Generieren der Photonen: Parametrische Fluoreszenz
                \item R\"aumliche Trennung von \SI{400}{\meter} $\to$ Lokalit\"at sichergestellt
                \item Ungen\"ugende Detektoreffizienz
            \end{itemize}
        \item Rowe (Boulder, 2001):
            \begin{itemize}[<.->]
                \item Beryllium-Ionen statt Photonen $\to$ hohe Detektoreffizienz
                \item Abstand von \SI{3}{\micro\meter}
            \end{itemize}
            \item Resultate: Verletzung der Bellschen Ungleichung!
    \end{itemize}
\end{frame}

\section{Fazit}
\begin{frame}{Fazit}
    \begin{itemize}
        \item Alle bisherigen Experimente unterst\"utzen Bellsche Ungleichung.
        \item $\Rightarrow$ Intuition Einsteins war falsch!
        \item<2-> Auswege:
            \begin{itemize}
                \item Die Welt ist nicht-lokal.
                \item Es existiert keine objektive Realit\"at.
            \end{itemize}
    \end{itemize}
\end{frame}

\end{document}
