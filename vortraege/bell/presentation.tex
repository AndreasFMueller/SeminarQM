\documentclass{beamer}
\usetheme{Copenhagen}
\usecolortheme{default}

\usepackage{amsmath,amssymb}
\usepackage[ngerman]{babel}

\author{Hannes Badertscher}
\title{Bellsche Ungleichung}
\date{\today}

\AtBeginSection[]
{
    \begin{frame}<beamer>
        \frametitle{\contentsname}
        \tableofcontents[currentsection]
    \end{frame}
}

\begin{document}

\maketitle

\begin{frame}
    \tableofcontents
\end{frame}

\section{Das EPR Gedankenexperiment}
\begin{frame}{Das EPR Gedankenexperiment}
    \centering
    \includegraphics{../../skript/bell/images/experiment_setup.pdf}
\end{frame}
\begin{frame}{Forderungen}
    \begin{itemize}
        \item Lokalit\"at
        \item Realismus
        \item Korrektheit
        \item Vollst\"andigkeit
    \end{itemize}
\end{frame}
\begin{frame}{Herleitung}
    \begin{itemize}
        \item Pauli-Matrizen:
            \begin{align*}
                S_x &= \frac{\hbar}{2} \begin{pmatrix}
                0 & 1 \\ 1 & 0
                \end{pmatrix}
                &
                S_y &= \frac{\hbar}{2} \begin{pmatrix}
                0 & -i \\ i & 0
                \end{pmatrix}
                &
                S_z &= \frac{\hbar}{2} \begin{pmatrix}
                1 & 0 \\ 0 & -1
                \end{pmatrix}\label{equ:bell:paulimatrizen}
            \end{align*}
        \item Eigenwerte:
            \begin{align*}
                |{+}x\rangle &= \frac{1}{\sqrt{2}}\begin{pmatrix} 1\\1 \end{pmatrix} &
                |{-}x\rangle &= \frac{1}{\sqrt{2}}\begin{pmatrix} 1\\-1 \end{pmatrix} &
                |{+}z\rangle &= \begin{pmatrix} 1\\0 \end{pmatrix} &
                |{-}z\rangle &= \begin{pmatrix} 0\\1 \end{pmatrix} &
            \end{align*}
        % Tafel: 
        %  - Erklärung Tensor-Produkt (Def 29.5)
        %  - Herleitung Zustand (29.3)
        \item<2-> Zustand des Systems:
            \[
                |\psi\rangle = \frac{1}{\sqrt{2}} \Big( 
                    |{+}z\rangle \otimes |{-}z\rangle - |{-}z\rangle \otimes |{+}z\rangle
                 \Big)
            \]
        % Tafel: Überleitung zu (29.4)
        \item<3-> \"Aquivalenter Zustand:
            \[
                |\psi\rangle = -\frac{1}{\sqrt{2}} \Big( 
                              |{+}x\rangle \otimes |{-}x\rangle - |{-}x\rangle \otimes |{+}x\rangle
                           \Big)
            \]
    \end{itemize}


\end{frame}
\begin{frame}{Widerspruch}
    \begin{itemize}
        \item Zustand nach Messung $A$
            \begin{align*}
                |\psi_{1}\rangle &= |{+}z\rangle \otimes |{-}z\rangle
                & \text{bzw.} && 
                |\psi_{2}\rangle &= |{-}z\rangle \otimes |{+}z\rangle
            \end{align*}
        \item Zustand nach Messung $B$
            \begin{align*}
                |\psi_{a}\rangle &= |{+}x\rangle \otimes |{-}x\rangle
                & \text{bzw.} && 
                |\psi_{b}\rangle &= |{-}x\rangle \otimes |{+}x\rangle
            \end{align*}
        \item<2-> Durch
            \begin{equation*}
                [S_x, S_z] =  -i \hbar S_y \neq 0
            \end{equation*}
            unm\"oglich!
    \end{itemize}
\end{frame}
\begin{frame}{Die L\"osung: Verborgene Variablen}
    \centering
    \includegraphics[scale=1.5]{../../skript/bell/images/hidden_vars.pdf}
\end{frame}

\section{Die Bellsche Ungleichung}
\begin{frame}{Grundlagen}
    \begin{itemize}
        \item Verborgene Variable $\lambda \in \Lambda$ mit Verteilung $\rho(\lambda)$
        \item Beliebige Messrichtungen $\vec{a}$, $\vec{b}$
        \item Messresultate: $A(\vec{a},\lambda)$, $B(\vec{b},\lambda)$ mit
            \[
                A(\vec{a},\lambda) = \pm 1 
                \qquad \text{und}\qquad 
                B(\vec{b},\lambda) = \pm 1
            \]
        \item Perfekte Antikorrelation der Messungen:
            \[
                A(\vec{a},\lambda) = -B(\vec{a},\lambda)
            \]
    \end{itemize}
\end{frame}
\begin{frame}{Herleitung}
    \begin{itemize}
        %Tafel: Herleitung (29.20) - E(a,b)
        \item Erwartungswert $E$ der Messung von $\vec{a}$ und $\vec{b}$:
            \[
                E(\vec{a},\vec{b}) = -\int_{\lambda\in\Lambda} 
                    \rho(\lambda) A(\vec{a},\lambda) A(\vec{b},\lambda) d\lambda
            \]
        %Tafel: Komplette Herleitung der Bellschen Ungleichung
        \item<2-> Bellsche Ungleichung:
            \[
                \left| E(\vec{a},\vec{b}) - E(\vec{a},\vec{c}) \right| 
                \leq
                1 + E(\vec{b},\vec{c})
            \]
        %Tafel: 4 Richtungen, Differenz für CHSH Ungleichung (29.27)
        \item<3-> CHSH Ungleichung:
            \[
                -2 \leq 
                E(\vec{a},\vec{b}) - E(\vec{a},\vec{b'}) + E(\vec{a'},\vec{b}) + E(\vec{a'},\vec{b'})
                \leq 2
            \]
    \end{itemize}
\end{frame}

\section{Experimente}
\begin{frame}{Herausforderungen}
    \begin{itemize}[<+->]
        \item Effizienz der Detektoren
        \item Sicherstellen der Lokalit\"at
        \item R\"aumliche Korrelation
    \end{itemize}
\end{frame}
\begin{frame}{1. Generation -- Freedman Clauser (1972)}
    \begin{itemize}
        \item Generieren der Photonen: \\
            {
                \centering
                \includegraphics{../../skript/bell/images/freedman.pdf}
            }
        \item Resultat
    \end{itemize}

\end{frame}
\begin{frame}{2. Generation -- Alain Aspect (1980 -- 1985)}
\end{frame}
\begin{frame}{3. Generation (ab 1990)}
\end{frame}

\begin{frame}{Bibliography}
    
\end{frame}

\end{document}
