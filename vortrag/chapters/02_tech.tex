% 02_analyse

\subsection{Regelkreis}
\begin{frame}
  \frametitle{Technischer Aufbau}

  \begin{figure}
    \centering
    \begin{tikzpicture}
      [node distance=1.3cm,
     conn/.style={->,shorten >=0.1cm,>=stealth',very thick}]
    \node[draw, circle, inner sep=1.5mm] (lamp) {\ce{^{87}Ru}};
    \node[draw, circle, inner sep=1.5mm] (absorber) [right=of lamp]
    {\(
      \begin{matrix}
        & \downarrow & \uparrow\\
        \uparrow & \downarrow &\\
        & \uparrow & \downarrow\\
        \downarrow & & \uparrow
      \end{matrix}
    \)};
    \node (diode) [right=of absorber] {};
    \draw(diode) ++ (0.0,-1.0) to[empty photodiode] ++ (0.0,2.0);
    \node[draw, rectangle, inner sep=5mm, align=center] (vcxo) [right=of diode]
    {\SI{20}{\mega\hertz}\\VCXO};
    \node[draw, rectangle, inner sep=5mm, yshift=-1.5cm] (hfgen)
    [below=of diode] {Multip.};

    % connections
    \draw[conn] (lamp.east) -- (absorber.west);
    \draw[conn, shorten >=0.4cm] (absorber.east) -- (diode.west);
    \draw[conn] (diode.east) ++ (0.3cm,0.0) -- (vcxo.west);
    \draw[conn] (vcxo.south) |- (hfgen.east);
    \draw[conn] (hfgen.west) -| (absorber.south);
  \end{tikzpicture}
 \end{figure}

\end{frame}

\begin{frame}
  
\begin{circuitikz}\draw
  (0,0) node[anchor=east](start){rought} ++ (0,1)
         to[R=\(R_{\mathrm{rough}}\),o-] ++ (3,0)
;\end{circuitikz}
\end{frame}





%%% Local Variables: 
%%% mode: latex
%%% TeX-master: "../main"
%%% End: 
