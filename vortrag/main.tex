%%%%%%%%%%%%%%%%%%%%%%%%%%%%%%%%%%%%%%%%%%%%%%%%%%%%%%%%%%%%%%%%%%%%%%
%         FILE:  main.tex
%
%        USAGE:  ---
%
%  DESCRIPTION:  Presentation for MathSem Seminar QM
%
%      OPTIONS:  see different input's below
% REQUIREMENTS:  full LaTeX installations
%         BUGS:  ---
%        NOTES:  ---
%       AUTHOR:  Pascal Stump, pascal-stump@bluewin.ch
%      VERSION:  ---
%      CREATED:  26.Apr.2014
%     REVISION:  ---
%%%%%%%%%%%%%%%%%%%%%%%%%%%%%%%%%%%%%%%%%%%%%%%%%%%%%%%%%%%%%%%%%%%%%%

%%% LaTeX PREAMBLE
\documentclass[compressd]
{beamer}
\usecolortheme[named=orange]{structure}
\mode<presentation>
{
	\usetheme{Frankfurt}
}

\setbeamertemplate{itemize items}[default]
\setbeamertemplate{enumerate items}[default]

%%%%%%%%%%%%%%%%%%%%%%%%%%%%%%%%%%%%%%%%%%%%%%%%%%%%%%%%%%%%%%%%%%%%%%
%         FILE:  normalHeader.tex
%
%        USAGE:  ---
%
%  DESCRIPTION:  normal header for LaTeX files
%
%      OPTIONS:  ---
% REQUIREMENTS:  full LaTeX installations
%                main document
%                biber for bibliography
%         BUGS:  
%        NOTES:  ---
%       AUTHOR:  Pascal Stump, pascal-stump@bluewin.ch
%      VERSION:  see git repo
%      CREATED:  18.Apr.2014 - 17:50
%     REVISION:  see git repo
%%%%%%%%%%%%%%%%%%%%%%%%%%%%%%%%%%%%%%%%%%%%%%%%%%%%%%%%%%%%%%%%%%%%%%

%%% language definitions
\usepackage[T1]{fontenc}         % Umlaute as one character
\usepackage[utf8]{inputenc}      % character encoding

\usepackage[main=ngerman,        % main language in document
            english,             % other languages
            ngerman]
           {babel}               % correct language support
                                 %   change language within document :
                                 %     \selectlanguage{}

\usepackage[babel,               % use babel in background
            german=swiss]        % define german's style, also:
                                 %   quotes; guillemets; swiss
           {csquotes}            % quotes standardisation for whole
                                 %   document, usage: \enquote{}
                                 %   csquotes does not now
                                 %   nswissgerman, therefor definition:
\DeclareQuoteAlias{german}{nswissgerman}
                                 %   definition below only works with
                                 %     babel=german
\defineshorthand{"`}{\openautoquote}
\defineshorthand{"'}{\closeautoquote}

%%% miscellaneous
\usepackage{hyperref}            % url, clickable pdf

\usepackage[style=ieee,
            backend=biber,
            language=auto]
            {biblatex}           % citing of references
                                 %   usage:
                                 %     \bibliography{pathTo}
                                 %     \cite[prenote][postnote]{key}
                                 %     \printbibliography
\DeclareLanguageMapping{nswissgerman}{ngerman}

\usepackage{lipsum}              % lorem ipsum text include
                                 %   \lipsum
                                 %   \lipsum[4-6]
                                 
%%% tabularx
\usepackage{tabularx}

%%% caption
\usepackage[font=small]{caption}
% ähndern der Schriftgrösse bei caption


%%% preview
%\usepackage[showlables,sections,floats,textmath,displaymath]{preview}

%%% picture & graphics
\usepackage{tikz}                % TikZ graphics
\usetikzlibrary{arrows,        % include into main only what needed
%                 decorations,
%                 pathmorphing,
%                 backgrounds,
%                 positioning,
%                 fit,
%                 petri,
                calc,
%                 intersections,
%                 through
                shapes}

%%% units
\usepackage{siunitx}             % SI unit  \si{unit} or \SI{value}{unit}
\sisetup{per-mode=symbol-or-fraction,
         detect-shape,
         range-phrase = { \translate{bis} } }
                                 %   fraction; font shape auto detect

%%% chemistry
\usepackage[version=3]{mhchem}   % easy typesetting of chemical
                                 %   formula, usage:
                                 %     \ce{1/2H20}
                                 %     \ce{^{227}_{90}Th+}

\usepackage{chemfig}             % drawing molecules (with TikZ)
                                 %   usage:
                                 %     \chemfig{}
% http://mirror.switch.ch/ftp/mirror/tex/macros/generic/chemfig/chemfig_doc_en.pdf

%%% electronic                   % drawing electrical circuits (with
\usepackage[europeanvoltages,
            europeancurrents,
            europeanresistors,   % rectangular shape
            americaninductors,   % "4-bumbs" shape
            europeanports,       % rectangular logic ports
            siunitx,             % #1<#2>
            emptydiodes,
            noarrowmos,
            smartlabels]         % lables are rotated in a smart way
           {circuitikz}          %   TikZ), usage:
                                 %   \begin{circuitikz}
% http://mirror.switch.ch/ftp/mirror/tex/graphics/pgf/contrib/circuitikz/circuitikzmanual.pdf



%%% other packages
\usepackage{standalone}

%%% BibTeX document
%\bibliography{testTex/testBibtex}

%%% Title
\title[MathSem QM]{Mathematisches Seminar Quantenmechanik}
\subtitle{Atomuhr}
\author{Stefan Steiner \and Pascal Stump}
\institute{HSR Hochschule f�r Technik Rapperswil}
\date{04.\,Mai 2014}


%%%%%%%%%%%%%%%%%%%%%%%%%%%%%%%%%%%%%%%%%%%%%%%%%%%%%%%%%%%%%%%%%%%%%%
%%% BEGIN DOCUMENT
\begin{document}
	
	%%% input tex files
	\begin{frame}
		\titlepage
	\end{frame}
	
	%%%
	% 00_Themen; Begr�ssung
	\section{Begr�ssung}
	% 00_themen

%\subsection*[Themen]{Themen}
\begin{frame}
\tableofcontents
\end{frame}





%%% Local Variables: 
%%% mode: latex
%%% TeX-master: "../main"
%%% End: 

	
	% 01_Einleitung
	\section{Einleitung}
	% 01_einleitung

\subsection{Vergleich Atomuhren}

\begin{frame}
  \frametitle{Vergleich Verschiedener Atomuhren}
    
\end{frame}





%%% Local Variables: 
%%% mode: latex
%%% TeX-master: "../main"
%%% End: 

	
	% 02_Tech; Technischer Aufbau
	\section[Tech]{Technische Betrachtung}
	% 02_analyse

\subsection{Regelkreis}
\begin{frame}
  \frametitle{Technischer Aufbau}

  \begin{figure}
    \centering
    \begin{tikzpicture}
      [node distance=1.3cm,
     conn/.style={->,shorten >=0.1cm,>=stealth',very thick}]
    \node[draw, circle, inner sep=1.5mm] (lamp) {\ce{^{87}Ru}};
    \node[draw, circle, inner sep=1.5mm] (absorber) [right=of lamp]
    {\(
      \begin{matrix}
        & \downarrow & \uparrow\\
        \uparrow & \downarrow &\\
        & \uparrow & \downarrow\\
        \downarrow & & \uparrow
      \end{matrix}
    \)};
    \node (diode) [right=of absorber] {};
    \draw(diode) ++ (0.0,-1.0) to[empty photodiode] ++ (0.0,2.0);
    \node[draw, rectangle, inner sep=5mm, align=center] (vcxo) [right=of diode]
    {\SI{20}{\mega\hertz}\\VCXO};
    \node[draw, rectangle, inner sep=5mm, yshift=-1.5cm] (hfgen)
    [below=of diode] {Multip.};

    % connections
    \draw[conn] (lamp.east) -- (absorber.west);
    \draw[conn, shorten >=0.4cm] (absorber.east) -- (diode.west);
    \draw[conn] (diode.east) ++ (0.3cm,0.0) -- (vcxo.west);
    \draw[conn] (vcxo.south) |- (hfgen.east);
    \draw[conn] (hfgen.west) -| (absorber.south);
  \end{tikzpicture}
 \end{figure}

\end{frame}

\begin{frame}
  
\begin{circuitikz}\draw
  (0,0) node[anchor=east](start){rought} ++ (0,1)
         to[R=\(R_{\mathrm{rough}}\),o-] ++ (3,0)
;\end{circuitikz}
\end{frame}





%%% Local Variables: 
%%% mode: latex
%%% TeX-master: "../main"
%%% End: 
 
	
	% 03_Math; Quantenmechanische Beschreibung
	\section[Math]{Quantenmechanische Betrachtung}
	% 03_hw

\subsection{Quantenmechanische Betrachtung}

\begin{frame}
  \frametitle{Repetition Wasserstoffatom}
	\begin{columns}
		 \column{.5\textwidth}
			 \begin{itemize} 
			\item[]	$h\nu = \Delta E$ 
			\item[]   $h:$ Planksches  Wirkungsquantum
		 	\item[]   $\nu:$ Frequenz
		 	\item[]   $E: $ Energie
		 	\end{itemize}
		 	\vspace{.5cm}		 	 
		 	�bergang $\infty \rightarrow 1: \lambda_1 = 91nm$
		 	
		 	$\nu_1 = \dfrac{c}{\lambda_1} = 3.3 PHz \quad(10^{15})$
		 	\vspace{.5cm}
		 	
		 	�bergang $6 \rightarrow 5: \lambda_2 = 7457nm$
		 	
		 	$\nu_2 = \dfrac{c}{\lambda_2} = 40,2 THz$
		 			 	
		\column{.5\textwidth}
		 	\includegraphics[width = 5cm]{./pictures/wasserstoffBohr}
		 	
		 	Spektrum Balmer-Serie
		 	
		 	\includegraphics[width = 5cm]{./pictures/wasserstoffSpektrum}
	
	\end{columns}

\end{frame}

\begin{frame}
  \frametitle{Feinstruktur�bergang}
	\begin{columns}
		\column{.5\textwidth}
		\begin{itemize}
		\item[] Betrachtung Spektrum
		\item[] Spin-Bahn Kopplung
		\end{itemize}
		
		\column{.5\textwidth}
		\includegraphics[width = 4cm]{./pictures/feinstrukturelektron}
		
	
		
	\end{columns}
	
\end{frame}

\begin{frame}
	\frametitle{Hyperfeinstruktur�bergang}
	Wechselwirkung zwischen Spin und elektron
	Anwendung St�rungstheorie -> kleine Effekte summieren sich
	Atom haben niedriges niveau -> anregung -> einige werden h�heres Niveau bekommen
	
\end{frame}

\begin{frame}
	\frametitle{Alkalimetalle}
	Abgeschlossene Schalen haben verschwindenden Drehimplus -> nur �usserstes elektron reagiert
	1 Valenzelektron
\end{frame}


%%% Local Variables: 
%%% mode: latex
%%% TeX-master: "../main"
%%% End: 

	
	% 04_Schluss/Fragen
	\section[?]{ Schluss/Fragen}
	% 07_schluss

\subsection{Schluss}

\begin{frame}
  \frametitle{Schluss}

\end{frame}

\subsection{Fragen}
\begin{frame}
  \frametitle{Fragen}
\end{frame}



%%% Local Variables: 
%%% mode: latex
%%% TeX-master: "../main"
%%% End: 

	
	%%% END DOCUMENT
\end{document}